% -----------------------------------------------------------------------------

% CS 246
\usepackage{float}
\usepackage{listings}

\definecolor{light-gray}{gray}{0.95}
\newcommand{\code}[1]{\texttt{#1}}

% -----------------------------------------------------------------------------

% CO 250
\usepackage{tkz-berge}

% -----------------------------------------------------------------------------

% Core Packages
\usepackage{xfrac}
\usepackage[margin=1in]{geometry}
\usepackage[unicode]{hyperref}
\usepackage[shortlabels]{enumitem}
\usepackage[parfill]{parskip}
\usepackage[theorems,breakable]{tcolorbox}
\usepackage{graphicx}
\usepackage[ruled,linesnumbered,vlined,dotocloa]{algorithm2e}
\usepackage[delims=\lbrack\rbrack]{spalign}
\usepackage{mathtools}
\usepackage{cleveref}
\usepackage{pdfpages}
\usepackage{minted}
\usepackage{tikz}
\usetikzlibrary{patterns}
\usetikzlibrary{positioning}
\usepackage{pgfplots}
\pgfplotsset{compat=1.17}


% -----------------------------------------------------------------------------

% Better Tables
\usepackage{multicol}
\usepackage{booktabs}
\usepackage{adjustbox}
\usepackage{tabularx}
\newcolumntype{Y}{>{\centering\arraybackslash}X}
\newcolumntype{Z}{>{\centering\arraybackslash\columncolor{light-gray}}X}
\newcolumntype{B}{>{\centering\arraybackslash\bfseries}X}

% -----------------------------------------------------------------------------

% Intervals
\usepackage{interval}
\intervalconfig{
    soft open fences,
    separator symbol={,}
}

% -----------------------------------------------------------------------------

\graphicspath{ {./figures/} }

\DeclareMathOperator{\rank}{rank}
\DeclareMathOperator{\slack}{slack}
\DeclareMathOperator{\row}{row}
\DeclareMathOperator{\cone}{cone}
\DeclareMathOperator{\nullspace}{Null}
\DeclareMathOperator{\ch}{char}
\DeclareMathOperator{\ord}{ord}
\DeclareMathOperator{\lcm}{lcm}

\usepackage{etoolbox}

% Functions
\providecommand\given{} % just to make sure it exists
\DeclarePairedDelimiterXPP{\E}[1]{\mathbb{E}}[]{}{
    \renewcommand\given{\nonscript\:\delimsize\vert\nonscript\:\mathopen{}}
    #1}
\DeclarePairedDelimiterXPP{\Var}[1]{\mathbb{V}}(){}{
    \renewcommand\given{\nonscript\:\delimsize\vert\nonscript\:\mathopen{}}
    #1}
\DeclarePairedDelimiterXPP\Prob[1]{\mathbb{P}}(){}{
    \renewcommand\given{\nonscript\:\delimsize\vert\nonscript\:\mathopen{}}
    \ifblank{#1}{\:\cdot\:}
    #1}
\DeclarePairedDelimiterXPP\Ind[1]{\mathbb{I}}\{\}{}{
\renewcommand\given{\nonscript\:\delimsize\vert\nonscript\:\mathopen{}}
\ifblank{#1}{\:\cdot\:}
#1}
\newcommand{\indep}{\perp\!\!\!\perp}
\DeclarePairedDelimiterXPP{\Corr}[1]{\text{Corr}}(){}{#1}
\DeclarePairedDelimiterXPP{\Cov}[1]{\text{Cov}}(){}{#1}
\DeclarePairedDelimiterXPP{\Sd}[1]{\text{Sd}}(){}{#1}
\DeclarePairedDelimiterXPP{\Se}[1]{\text{Se}}(){}{#1}
\let\SS=\relax
\DeclarePairedDelimiterXPP{\SS}[1]{\text{SS}}(){}{\text{#1}}
\DeclarePairedDelimiterXPP{\MS}[1]{\text{MS}}(){}{\text{#1}}
\DeclarePairedDelimiterXPP{\Span}[1]{\text{Span}}(){}{#1}
\DeclarePairedDelimiterXPP{\Spanc}[1]{\overline{\text{Span}}}(){}{#1}
\DeclareMathOperator{\VIF}{VIF}
\DeclarePairedDelimiterXPP{\expon}[1]{\text{exp}}\{\}{}{#1}

% Distributions
\DeclarePairedDelimiterXPP{\N}[1]{\mathcal{N}}(){}{#1}
\DeclarePairedDelimiterXPP{\uniform}[1]{\text{Uniform}}(){}{#1}
\DeclarePairedDelimiterXPP{\hyp}[1]{\text{Hypergeometric}}(){}{#1}
\DeclarePairedDelimiterXPP{\bern}[1]{\text{Bernoulli}}(){}{#1}
\DeclarePairedDelimiterXPP{\bin}[1]{\text{Binomial}}(){}{#1}
\DeclarePairedDelimiterXPP{\nb}[1]{\text{Negative Binomial}}(){}{#1}
\DeclarePairedDelimiterXPP{\geo}[1]{\text{Geometric}}(){}{#1}
\DeclarePairedDelimiterXPP{\poi}[1]{\text{Poisson}}(){}{#1}
\DeclarePairedDelimiterXPP{\mult}[1]{\text{Multinomial}}(){}{#1}
\DeclarePairedDelimiterXPP{\gam}[1]{\text{Gamma}}(){}{#1}
\DeclarePairedDelimiterXPP{\weib}[1]{\text{Weibull}}(){}{#1}
\DeclarePairedDelimiterXPP{\Mvn}[1]{\text{MVN}}(){}{#1}
\DeclarePairedDelimiterXPP{\Bvn}[1]{\text{BVN}}(){}{#1}
\DeclarePairedDelimiterXPP{\exponential}[1]{\text{Exponential}}(){}{#1}

\DeclarePairedDelimiterXPP{\tr}[1]{\text{tr}}(){}{#1}

\DeclarePairedDelimiterX\innerp[2]{\langle}{\rangle}{
    \ifblank{#1}{\:\cdot\:,}#1,
    \ifblank{#2}{\:\cdot\:}#2
}

\DeclarePairedDelimiterXPP{\MA}[1]{\text{MA}}(){}{#1}
\DeclarePairedDelimiterXPP{\AR}[1]{\text{AR}}(){}{#1}
\DeclarePairedDelimiterXPP{\ARMA}[1]{\text{ARMA}}(){}{#1}
\DeclarePairedDelimiterXPP{\AIC}[1]{\text{AIC}}(){}{#1}
\DeclarePairedDelimiterXPP{\BIC}[1]{\text{BIC}}(){}{#1}
\DeclarePairedDelimiterXPP{\ARIMA}[1]{\text{ARIMA}}(){}{#1}
\DeclarePairedDelimiterXPP{\GARCH}[1]{\text{GARCH}}(){}{#1}
\DeclarePairedDelimiterXPP{\NNAR}[1]{\text{NNAR}}(){}{#1}
\DeclarePairedDelimiterXPP{\NNSAR}[2]{\text{NNSAR}}(){_{#2}}{#1}
\DeclarePairedDelimiterXPP{\ARCH}[1]{\text{ARCH}}(){}{#1}

% -----------------------------------------------------------------------------

% Table of Contents
\author{Cameron Roopnarine}
\date{Last updated: \today}
\hypersetup{colorlinks, linkcolor=[rgb]{0,0.5,1}}

% -----------------------------------------------------------------------------

% Heading Dates
\newcommand{\makeheading}[1]
{
    \begin{figure}[H]
        \centering
        \rule{\columnwidth}{1pt}\\
        {\large \scshape{#1}}\\[-0.6\baselineskip]
        \rule{\columnwidth}{1pt}
        \vspace*{-20pt}
    \end{figure}
}

% -----------------------------------------------------------------------------

% Definitions
\definecolor{myyellow}{RGB}{255,255,168}
% Theorems
\definecolor{mypurple}{RGB}{216,216,255}
% Algorithms
\definecolor{mygray}{RGB}{232,232,232}
% Examples
\definecolor{mygreen}{RGB}{216,255,216}
% Exercises
\definecolor{myred}{RGB}{255,216,216}
% Remarks
\definecolor{mycyan}{RGB}{204,229,229}

\tcbset{
    common/.style={
            fonttitle=\bfseries,
            coltitle=black,
            boxrule=0pt
        },
    theorem/.style={
            common,
            colback=mypurple,
            colframe=mypurple!95!black,
            fontupper=\itshape{}
        },
}


\newtcbtheorem[number within=section, crefname={definition}{definitions}]
{Definition}{DEFINITION}{
    common,
    colback=myyellow,
    colframe=myyellow!95!black
}{def}

\newtcbtheorem[use counter from=Definition, crefname={example}{examples}]
{Example}{EXAMPLE}{
    common,
    colback=mygreen,
    colframe=mygreen!95!black,
    breakable
}{ex}

\newtcbtheorem[use counter from=Definition, crefname={exercise}{exercises}]
{Exercise}{EXERCISE}{
    common,
    colback=myred,
    colframe=myred!95!black,
    breakable
}{exercise}

\newtcbtheorem[use counter from=Definition, crefname={remark}{remarks}]
{Remark}{REMARK}{
    common,
    colback=mycyan,
    colframe=mycyan!95!black,
}{remark}

\newtcbtheorem[use counter from=Definition, crefname={theorem}{theorems}]
{Theorem}{THEOREM}{
    theorem,
}{thm}

\newtcbtheorem[use counter from=Definition, crefname={proposition}{propositions}]
{Proposition}{PROPOSITION}{
    theorem,
}{prop}

\newtcbtheorem[use counter from=Definition, crefname={corollary}{corollaries}]
{Corollary}{COROLLARY}{
    theorem,
}{cor}

\newtcbtheorem[use counter from=Definition, crefname={lemma}{lemmas}]
{Lemma}{LEMMA}{
    theorem,
}{lem}

\newtcbtheorem[no counter]
{Proof}{Proof of}{
    common,
    colframe=black!10,
    breakable,
    separator sign={}
}{pf}

\DeclarePairedDelimiter\norm{\lVert}{\rVert}
\DeclarePairedDelimiter\abs{\lvert}{\rvert}
\DeclarePairedDelimiter\set{\{}{\}}
\DeclareMathOperator*{\argmax}{arg\,max}
\DeclareMathOperator*{\argmin}{arg\,min}
\DeclareMathOperator*{\arginf}{arg\,inf}
\DeclareMathOperator*{\argsup}{arg\,sup}

% just to make sure it exists
\providecommand\onto{}
% can be useful to refer to this outside \Set
\newcommand\SetSymbol[1][]{%
    \nonscript\:#1\vert{}
    \allowbreak\nonscript\:
    \mathopen{}}
\DeclarePairedDelimiterXPP\Proj[1]{\text{Proj}}(){}{%
    \renewcommand\onto{\SetSymbol[\delimsize]}
    #1
}

\AtBeginDocument{%
    \let\mathbb\relax
    \DeclareMathAlphabet{\mathbb}{U}{msb}{m}{n}%
}
