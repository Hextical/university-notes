\documentclass[oneside]{book}
\usepackage[dvipsnames]{xcolor}
\usepackage[margin=1in]{geometry}

\usepackage{amsmath, amssymb, amsthm}
\usepackage[nice]{nicefrac}
% Fonts
\usepackage{lmodern}
\usepackage{XCharter}
\usepackage[T1]{fontenc}
\usepackage[utf8]{inputenc}
\usepackage{array}
\usepackage{tkz-berge}
\usepackage{hyperref}
\usepackage{enumerate}
\usepackage[ruled,linesnumbered]{algorithm2e}
\usepackage[parfill]{parskip}
\usepackage[most]{tcolorbox}
\usepackage{graphicx}
\usepackage{bm}
\usepackage{contour}
\usepackage[normalem]{ulem}
\usepackage{esvect}
\usepackage{stackengine}
\usepackage{listings}
\usepackage[explicit]{titlesec}
\usepackage{xhfill}
\usepackage{varwidth}
\graphicspath{ {./figures/} }

% input multiple
\usepackage{multido}
\newcommand{\forLoop}[4][1]{\multido{\i=#2+#1}{#3}{#4}}
\definecolor{light-gray}{gray}{0.95}
\newcommand{\code}[1]{\colorbox{light-gray}{\texttt{#1}}}

\author{Cameron Roopnarine}
\date{Last updated: \today}

\DeclareMathOperator{\rank}{rank}
\DeclareMathOperator{\du}{DU}
\DeclareMathOperator{\bin}{Bin}
\DeclareMathOperator{\hyp}{Hyp}
\DeclareMathOperator{\nb}{NB}
\DeclareMathOperator{\geo}{Geo}
\DeclareMathOperator{\poi}{Poi}
\DeclareMathOperator{\mult}{Mult}
\DeclareMathOperator{\slack}{slack}
\DeclareMathOperator{\row}{row}
\DeclareMathOperator{\cone}{cone}
\DeclareMathOperator{\nullspace}{Null}
\DeclareMathOperator{\ch}{char}
\DeclareMathOperator{\ord}{ord}
\hypersetup{colorlinks, linkcolor=[rgb]{0,0.5,1}}

% Definitions
\definecolor{myyellow}{RGB}{255,255,168}
% Examples
\definecolor{mygreen}{RGB}{216,255,216}
% Theorems
\definecolor{mypurple}{RGB}{216,216,255}
% Algorithms
\definecolor{mygray}{RGB}{232,232,232}

\newtcolorbox{defbox}[1][]{
    colback=myyellow,
    boxrule=0pt,
    sharp corners,
    #1
}

\newtcolorbox{thmbox}[1][]{
    colback=mypurple,
    boxrule=0pt,
    sharp corners,
    #1
}

\newtcolorbox{exbox}[1][]{
    breakable,
    colback=mygreen,
    boxrule=0pt,
    sharp corners,
    #1
}

\newtcolorbox{algbox}[1][]{
    colback=mygray,
    boxrule=0pt,
    sharp corners,
    #1
}

\newcolumntype{L}[1]{>{\raggedright\let\newline\\\arraybackslash\hspace{0pt}}m{#1}}
\newcolumntype{C}[1]{>{\centering\let\newline\\\arraybackslash\hspace{0pt}}m{#1}}
\newcolumntype{R}[1]{>{\raggedleft\let\newline\\\arraybackslash\hspace{0pt}}m{#1}}

% credits to: https://alexwlchan.net/2017/10/latex-underlines/
\renewcommand{\ULdepth}{1.8pt}
\contourlength{0.8pt}

\newcommand{\myuline}[1]{%
  \uline{\phantom{#1}}%
  \llap{\contour{white}{#1}}%
}

\theoremstyle{plain}
\newtheorem{theorem}{THEOREM}[section] % reset theorem numbering for each chapter
\theoremstyle{plain}
\newtheorem{corollary}[theorem]{COROLLARY} % reset theorem numbering for each chapter
\theoremstyle{plain}
\newtheorem{lemma}[theorem]{LEMMA} % reset theorem numbering for each chapter
\theoremstyle{definition}
\newtheorem{definition}[theorem]{DEFINITION} % definition numbers are dependent on theorem numbers
\theoremstyle{definition}
\newtheorem{example}[theorem]{EXAMPLE} % definition numbers are dependent on theorem numbers

\def\dotfill#1{\cleaders\hbox to #1{.}\hfill}
\makeatletter
\def\myrulefill{\leavevmode\leaders\hrule height .7ex width 1ex depth -0.6ex\hfill\kern\z@}
\makeatother

\newcommand{\makeheading}[1]%
{%
\begin{center}%
\rule{\columnwidth}{1pt}\\%
{\large \scshape{#1}}\\[-0.6\baselineskip]%
\rule{\columnwidth}{1pt}%
\end{center}
}


\pagestyle{headings}