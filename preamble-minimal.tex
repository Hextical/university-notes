
% -----------------------------------------------------------------------------

% Class
\documentclass{article}
% Must load first
\usepackage{xcolor}

% -----------------------------------------------------------------------------

% Fonts
\usepackage{fontspec}
\setmainfont{XCharter}
\usepackage{anyfontsize}
\usepackage{microtype}
\usepackage[math-style=ISO,bold-style=ISO]{unicode-math}

% -----------------------------------------------------------------------------

% CS 246
\usepackage{float}
\usepackage{soulutf8}
\usepackage{listings}

\definecolor{light-gray}{gray}{0.95}
\newcommand{\code}[1]{\sethlcolor{light-gray}\hl{\texttt{#1}}}

% -----------------------------------------------------------------------------

% CO 250
\usepackage{tkz-berge}

% -----------------------------------------------------------------------------

% Core Packages
\usepackage{amsmath, amsthm}
\swapnumbers{}
\usepackage{bm}
\usepackage{xfrac}
\usepackage[margin=1in]{geometry}
\usepackage{array}
\usepackage[unicode]{hyperref}
\usepackage{enumitem}
\usepackage[parfill]{parskip}
\usepackage[theorems,breakable]{tcolorbox}
\usepackage{graphicx}
\usepackage[ruled,linesnumbered,vlined,dotocloa]{algorithm2e}
\usepackage[delims=\lbrack\rbrack]{spalign}
\usepackage{dsfont}
\usepackage{mathtools}
\usepackage{cleveref}
\usetikzlibrary{patterns}

% -----------------------------------------------------------------------------

% Better Tables
\usepackage{multicol}
\usepackage{booktabs}
\usepackage{adjustbox}
\usepackage{tabularx}
\newcolumntype{Y}{>{\centering\arraybackslash}X}

% -----------------------------------------------------------------------------

% Intervals
\usepackage{interval}
\intervalconfig{
    soft open fences,
    separator symbol={,}
}

% -----------------------------------------------------------------------------

\graphicspath{ {./figures/} }

\DeclareMathOperator{\rank}{rank}
\DeclareMathOperator{\du}{\mathcal{U}}
\DeclareMathOperator{\uniform}{\mathcal{U}}
\DeclareMathOperator{\bin}{Binomial}
\DeclareMathOperator{\hyp}{Hyp}
\DeclareMathOperator{\nb}{NB}
\DeclareMathOperator{\geo}{Geometric}
\DeclareMathOperator{\poi}{Poisson}
\DeclareMathOperator{\exponential}{Exponential}
\DeclareMathOperator{\mult}{Multinomial}
\DeclareMathOperator{\slack}{slack}
\DeclareMathOperator{\row}{row}
\DeclareMathOperator{\cone}{cone}
\DeclareMathOperator{\nullspace}{Null}
\DeclareMathOperator{\ch}{char}
\DeclareMathOperator{\ord}{ord}
\DeclareMathOperator{\lcm}{lcm}

% Statistics
\DeclarePairedDelimiterXPP{\E}[1]{\symsfup{E}}[]{}{#1}
\DeclarePairedDelimiterXPP{\Var}[1]{\symsfup{Var}}(){}{#1}
\DeclarePairedDelimiterXPP{\Corr}[1]{\symsfup{Corr}}(){}{#1}
\DeclarePairedDelimiterXPP{\Cov}[1]{\symsfup{Cov}}(){}{#1}
\DeclarePairedDelimiterXPP{\Sd}[1]{\symsfup{Sd}}(){}{#1}
\DeclarePairedDelimiterXPP{\Se}[1]{\symsfup{Se}}(){}{#1}
\DeclarePairedDelimiterXPP{\Ss}[1]{\symsfup{Ss}}(){}{#1}

% -----------------------------------------------------------------------------

% Table of Contents
\author{Cameron Roopnarine}
\date{Last updated: \today}
\hypersetup{colorlinks, linkcolor=[rgb]{0,0.5,1}}

% -----------------------------------------------------------------------------

% Definitions
\definecolor{myyellow}{RGB}{255,255,168}
\newtcolorbox{defbox}[1][]{
    colback=myyellow,
    boxrule=0pt,
    sharp corners,
    #1
}

% Theorems
\definecolor{mypurple}{RGB}{216,216,255}
\newtcolorbox{thmbox}[1][]{
    colback=mypurple,
    boxrule=0pt,
    sharp corners,
    #1
}

% Algorithms
\definecolor{mygray}{RGB}{232,232,232}
\newtcolorbox{algbox}[1][]{
    colback=mygray,
    boxrule=0pt,
    sharp corners,
    #1
}

% Examples
\definecolor{mygreen}{RGB}{216,255,216}
\newtcolorbox{exbox}[1][]{
    breakable,
    colback=mygreen,
    boxrule=0pt,
    sharp corners,
    #1
}

\definecolor{myred}{RGB}{255,216,216}

% -----------------------------------------------------------------------------

\newtheorem{theorem}{Theorem}[section] % reset theorem numbering for each chapter

\theoremstyle{definition}
\newtheorem{definition}[theorem]{Definition} % definition numbers are dependent on theorem numbers
\newtheorem{example}[theorem]{Example} % same for example numbers
\newtheorem{remark}[theorem]{Note}

% -----------------------------------------------------------------------------

% Heading Dates
\newcommand{\makeheading}[1]
{
    \begin{figure}[H]
        \centering
        \rule{\columnwidth}{1pt}\\
        {\large \scshape{#1}}\\[-0.6\baselineskip]
        \rule{\columnwidth}{1pt}
        \vspace*{-20pt}
    \end{figure}
}

% -----------------------------------------------------------------------------

\tcbset{
    common/.style={
            fonttitle=\bfseries,
            coltitle=black,
            boxrule=0pt
        },
    theorem/.style={
            common,
            colback=mypurple,
            colframe=mypurple!95!black,
            fontupper=\itshape{}
        },
}

\definecolor{mycyan}{RGB}{204,229,229}
\newtcbtheorem[number within=section, crefname={definition}{definitions}]
{Definition}{DEFINITION}{
    common,
    colback=myyellow,
    colframe=myyellow!95!black
}{def}

\newtcbtheorem[use counter from=Definition, crefname={example}{examples}]
{Example}{EXAMPLE}{
    common,
    colback=mygreen,
    colframe=mygreen!95!black,
    breakable
}{ex}

\newtcbtheorem[use counter from=Definition, crefname={exercise}{exercises}]
{Exercise}{EXERCISE}{
    common,
    colback=myred,
    colframe=myred!95!black,
    breakable
}{ex}

\newtcbtheorem[use counter from=Definition, crefname={remark}{remarks}]
{Remark}{REMARK}{
    common,
    colback=mycyan,
    colframe=mycyan!95!black,
}{remark}

\newtcbtheorem[use counter from=Definition, crefname={theorem}{theorems}]
{Theorem}{THEOREM}{
    theorem,
}{thm}

\newtcbtheorem[use counter from=Definition, crefname={proposition}{propositions}]
{Proposition}{PROPOSITION}{
    theorem,
}{prop}

\newtcbtheorem[use counter from=Definition, crefname={corollary}{corollaries}]
{Corollary}{COROLLARY}{
    theorem,
}{cor}

\newtcbtheorem[use counter from=Definition, crefname={lemma}{lemmas}]
{Lemma}{LEMMA}{
    theorem,
}{lem}

\newtcbtheorem[no counter]
{Proof}{Proof of}{
    common,
    colframe=black!10,
    breakable
}{pf}

\DeclarePairedDelimiter\norm{\lVert}{\rVert}
\DeclarePairedDelimiter\abs{\lvert}{\rvert}
\DeclarePairedDelimiter\set{\{}{\}}
