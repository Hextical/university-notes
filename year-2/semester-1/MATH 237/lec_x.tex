\section{Q}
\begin{defbox}
\subsection{Definition (Directional Derivative)}
The \emph{directional derivative} of $ f(x,y) $ at a point $ (a,b) $ 
in the direction of a unit vector $ \symbfit{u}=(u_1,u_2) $ 
defined by
\[ D_{\symbfit{u}}f(a,b)=
\frac{d}{ds} f(a+su_1,b+su_2)\Bigr\rvert_{s=0} \]
provided the derivative exists.
\end{defbox}

\begin{thmbox}
\subsection{Theorem}
If $ f(x,y) $ is differentiable at $ (a,b) $ and 
$ \symbfit{u}=(u_1,u_2) $ is a unit vector, then
\[ D_{\symbfit{u}}f(a,b)=\nabla f(a,b) \cdot \symbfit{u} \]
\end{thmbox}

\begin{remark}
    Be careful to check the condition of Theorem before
    applying it. If $ f $ is not differentiable at
    $ (a,b) $, then we must apply the definition of the
    directional derivative.
\end{remark}
\begin{remark}
    If we choose $ \symbfit{u}=\symbfit{i}=(1,0) $ or
    $ \symbfit{u}=\symbfit{j}=(0,1) $, then the directional
    derivative is equal to the partial derivatives $ f_x $ or $ f_y $
    respectively.
\end{remark}

\begin{thmbox}
\subsection{Theorem}
If $ f(x,y) $ is differentiable at $ (a,b) $, and
$ \nabla f(a,b)\neq (0,0) $, then the largest value of
$ D_{\symbfit{u}}f(a,b) $ is $ ||\nabla f(a,b)|| $, and
occurs when $ \symbfit{u} $ is in the direction of
$ \nabla f(a,b) $.
\end{thmbox}

\begin{thmbox}
\subsection{Theorem}
If $ f(x,y)\in C^1 $ in a neighborhood of $ (a,b) $ and
$ \nabla f(a,b)\neq (0,0) $, then $ \nabla f(a,b) $ is
orthogonal to the level curve $ f(x,y)=k $ through
$ (a,b) $.
\end{thmbox}

\begin{thmbox}
\subsection{Theorem}
If $ f(x,y,z)\in C^1 $ in a neighborhood of $ (a,b,c) $ and
$ \nabla f(a,b,c)\neq (0,0,0) $, then $ \nabla f(a,b,c) $ is
orthogonal to the level curve $ f(x,y,z)=k $ through
$ (a,b,c) $.
\end{thmbox}

\begin{defbox}
\subsection{Definition (2nd degree Taylor polynomial)}
The \emph{second degree Taylor polynomial} $ P_{2,(a,b)} $ 
of $ f(x,y) $ at $ (a,b) $ is given by
\[
\begin{aligned}
P_{2,(a,b)}(x,y)=
f(a,b)+f_x(a,b)(x-a)+f_y(a,b)(y-b)+\\
\frac{1}{2} [f_{xx}(a,b)(x-a)^2+2f_{xy}(a,b)(x-a)(y-b)+
f_{yy}(a,b)(y-b)^2]
\end{aligned}
\]
\end{defbox}

\begin{thmbox}
\subsection{Theorem}
If $ f^{\prime\prime}(x) $ exists on $ [a,x] $, then there exists
a number $ c $ between $ a $ and $ x $ such that
\[ f(x)=f(a)+f^\prime (a)(x-a)+R_{1,a}(x) \]
where
\[ R_{1,a}(x)=\frac{1}{2} f^{\prime\prime}(c)(x-a)^2 \]
\end{thmbox}

\begin{thmbox}
\subsection{Theorem (Taylor's Theorem}
If $ f(x,y)\in C^2 $ in some neighborhood $ N(a,b) $ of
$ (a,b) $, then for all $ (x,y)\in N(a,b) $ there
exists a point $ (c,d) $ on the line segment joining
$ (a,b) $ and $ (x,y) $ such that
\[ f(x,y)=f(a,b)+f_x(a,b)(x-a)+f_y(a,b)(y-b)+R_{1,(a,b)}(x,y) \]
where
\[ R_{1,(a,b)}(x,y)=\frac{1}{2}
[f_{xx}(c,d)(x-a)^2+2f_{xy}(c,d)(x-a)(y-b)+f_{yy}(c,d)(y-b)^2] \]
\end{thmbox}

\begin{remark}
    Like the one variable case, Taylor's Theorem for $ f(x,y) $ is an
    existence theorem. That is, it only tells us that the point $ (c,d) $
    exists, but not how to find it.
\end{remark}

\begin{remark}
    The most important thing about the error term $ R_{1,(a,b)(x,y)} $ is not
    its explicit form, but rather its dependence on the magnitude of the 
    displacement $ ||(x,y)−(a,b)|| $. We state the result as a Corollary.
\end{remark}

\begin{thmbox}
\subsection{Corollary}
If $ f(x,y)\in C^2 $ in some closed neighborhood
$ N(a,b) $ of $ (a,b) $, then there exists a positive constant
$ M $ such that
\[ R_{1,(a,b)}(x,y)\le M||(x,y)-(a,b)||^2 \]
for all $ (x,y)\in N(a,b) $.
\end{thmbox}

\begin{thmbox}
\subsection{Taylor's Theorem of order k}
If $ f(x,y)\in C^{k+1} $ at each point on the line segment
joining $ (a,b) $ and $ (x,y) $, then there exists
a point $ (c,d) $ on the line segment between $ (a,b) $ and
$ (x,y) $ such that
\[ f(x,y)=P_{k,(a,b)}(x,y)+R_{k,(a,b)}(x,y) \]
where
\[ R_{k,(a,b)}(x,y)=\frac{1}{(k+1)!}
    |(x-a)D_1+(y-b)D_2|^{k+1}f(c,d)\]
\end{thmbox}

\begin{thmbox}
\subsection{Corollary}
If $ f(x,y)\in C^k $ in some neighborhood of
$ (a,b) $ then
\[ \lim\limits_{{(x,y)} \to {(a,b)}} 
\frac{|f(x,y)-P_{k,(a,b)}(x,y)|}{||(x,y)-(a,b)||^k}=0  \]
\end{thmbox}

\begin{thmbox}
\subsection{Corollary}
If $ f(x,y)\in C^{k+1} $ in some closed neighborhood
$ N(a,b) $ of $ (a,b) $, then there exists a constant
$ M>0 $ such that
\[ |f(x,y)-P_{k,(a,b)}(x,y)|\le M||(x,y)-(a,b)||^{k+1} \]
for all $ (x,y)\in N(a,b) $.
\end{thmbox}