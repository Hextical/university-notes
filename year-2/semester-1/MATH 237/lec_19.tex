\section{2019-10-23}
\begin{defbox}
    \subsection{Definition (Local Maximum and Minimum)}
    A point $ (a,b) $ is a \emph{local maximum point} for $ f(x,y) $ if
    \[ f(x,y)\le f(a,b) \]
    for all $ (x,y) $ in some neighborhood of $ (a,b) $.

    A point $ (a,b) $ is a \emph{local minimum point} for $ f(x,y) $ if
    \[ f(x,y)\ge f(a,b) \]
    for all $ (x,y) $ in some neighborhood of $ (a,b) $.
\end{defbox}

\begin{thmbox}
    \subsection{Theorem}
    Let $ f(x,y) $ have continuous partials. If $ (a,b) $ is a local
    maximum or minimum point of $ f $, then
    \[ \nabla f(a,b)=0 \]
    or at least one of $ f_x $, $ f_y $ does not exist at $ (a,b) $.
\end{thmbox}

\begin{proof}
    Let $ (a,b) $ be a local maximum or minimum point of $ f $. Fix
    $ x=a $, consider $ f(a,y)=z $ (cross section), it has a local
    maximum/minimum point at 
    $ y=b \implies \frac{\partial f}{\partial y}(a,b)=0 $ (or DNE) when $ y=b $.

    Similarly, $ \frac{\partial f}{\partial x}(a,b)=0 $ (or DNE).
\end{proof}

\begin{defbox}
    \subsection{Definition (Critical Point)}
    A point $ (a,b) $ in the domain of $ f(x,y) $ is called a 
    \emph{critical point} of $ f $ if $ \frac{\partial f}{\partial x}(a,b)=0 $
    or $ \frac{\partial f}{\partial x}(a,b) $ does not exist, and
    $ \frac{\partial f}{\partial y}(a,b)=0 $
    or $ \frac{\partial f}{\partial y}(a,b) $ does not exist.
\end{defbox}


\subsection{Examples}
Consider $ f(x,y)=\sqrt{x^2+y^2} $ which is a cone (upper half).
$ (0,0) $ is a local minimum point.
\[ f(x,y)=\sqrt{x^2+y^2}>0=f(0,0) \]
However, $ f_x(0,0) $ and $ f_y(0,0) $ does not exist.

Consider $ g(x,y)=x^2-y^2 $ which is a hyperbolic paraboloid (saddle surface)
\begin{align*}
    g_x=2x\\
    g_y=2y
\end{align*}
So, $(0,0)$ is the only critical point of $ g $, but
\begin{align*}
    g(x,0)>g(0,0)\\
    h(0,y)<h(0,0)
\end{align*}
for all $ x,y\in\mathbb{R} $, so $ (0,0) $ is neither a local
maximum or minimum point. We classify it as a \emph{saddle point}.

To summarize, all critical points are either local maxima, minima
or saddle points.

\subsection{Example (Finding Critical Points)}
Find all critical points of $ f(x,y)=xy(1-x^2-y^2) $.

\begin{align}
    f_x&=y(1-x^2-y^2)+xy(-2x)\\
    &=y[(1-x^2-y^2)+x(-2x)]\\
    &=y(1-x^2-y^2-2x^2)\\
    &=y(1-y^2-3x^2)\\
\end{align}

Similarly, we get
\begin{align}
    f_y=x(1-3y^2-x^2)=0
\end{align}

Note that (1) and (6) are both non-linear systems.
(6) yields roots 
$ y=0 $ and $ y=\pm \sqrt{1-3x^2} $ for roots, split into two cases.

\textbf{Case 1} $ y=0 $

Substituting into (6) we get $ x(1-x^2)=0 $, giving $ x={-1},0,1 $. Thus,
the corresponding critical points are: $ ({-1},0),(0,0),(1,0) $.

\textbf{Case 2} $ y=\pm \sqrt{1-3x^2} $

Substituting into (6) we get $ x(8x^2-2) $, giving 
$ x=0,\frac{1}{2},-\frac{1}{2} $. To find the corresponding $ y $ values,
plug the $ x $ values into $ y=\pm \sqrt{1-3x^2} $. Thus, the corresponding
critical points are: 
$ (0,0),
\underbrace{(\frac{1}{2},\frac{1}{2}),(-\frac{1}{2},\frac{1}{2})}_
{\text{+ sqrt}},
\underbrace{(\frac{1}{2},-\frac{1}{2}),(-\frac{1}{2},-\frac{1}{2})}_
{\text{- sqrt}},$. 

We need an analogy to the 2nd derivative test for $ y=f(x) $.

$ f\prime\prime>0 \rightarrow$ local minimum, 
$ f\prime\prime<0 \rightarrow$ local maximum

Consider the Taylor Series for $ f(x,y) $ about $ (a,b) $ such that 
$ \nabla f(a,b)=(0,0) $

\begin{align} 
    f(x,y)-f(a,b)\approx
    \frac{1}{2} \left[f_{xx}(a,b)(x-a)^2+2f_{xy}(a,b)(y-b)+f_{yy}(a,b)(y-b)^2\right]+
    \underbrace{\cdots}_{\text{H.O.T}}
\end{align}

If $ x $ is close to $ a $ and $ y $ is close to $ b $, then the
higher order terms can be neglected. So,
\begin{align}
    f(x,y)-f(a,b)\approx 
\frac{1}{2} \left[f_{xx}(a,b)(x-a)^2+2f_{xy}(a,b)(y-b)+f_{yy}(a,b)(y-b)^2\right]
\end{align}

\begin{thmbox}
    \subsection{Theorem (Second Partials Derivatives Test)}
    Suppose $ f(x,y)\in C^2 $ in some neighborhood of $ (a,b) $ 
    and that
    \[ \nabla f(a,b)=0 \]

    \begin{enumerate}[(1)]
        \item If $ f(x,y)-f(a,b)>0 $ (positive definite) for all 
        $ (x,y) $ near $ (a,b) $, $ (x,y)\neq(0,0)\neq(a,b) $ 
        then $ (a,b) $ is a local minimum point of $ f $.
        \item If $ f(x,y)-f(a,b)<0 $ (negative definite) for all 
        $ (x,y) $ near $ (a,b) $, $ (x,y)\neq(0,0)\neq(a,b) $ 
        then $ (a,b) $ is a local maximum point of $ f $.
        \item If  $ f(x,y)-f(a,b)< 0 $ for some $ (x,y) $ near $ (a,b) $ and
        $ f(x,y)-f(a,b)> 0 $ for some other $ (x,y) $ near $ (a,b) $, then
        $ (a,b) $ is a saddle point (indefinite) of $ f $.
    \end{enumerate}
\end{thmbox}