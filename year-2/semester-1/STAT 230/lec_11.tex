\section{Lecture 11}
\begin{defbox}
    \subsection{Definition (Cumulative Distribution Function)}
    The cumulative distribution function of $X$ is the function usually
    denoted by $F(x)$
    \[ F(x)=P(X\le x) \]
    for all $ x\in\mathbb{R} $.
\end{defbox}

\textbf{Example}

\begin{tabular}{| *{5}{>{\centering\arraybackslash}p{1.5cm} |}}
    \hline
    $x$ & $0$ & $1$ & $2$ & $3$\\
    \hline
    $f(x)$ & $\nicefrac{125}{216}$ & $\nicefrac{75}{216}$ & $\nicefrac{15}{216}$ & $\nicefrac{1}{216}$ \\
    \hline
    $F(x)$ & $\nicefrac{125}{216}$ & $\nicefrac{200}{216}$ & $\nicefrac{215}{216}$ & $1$ \\
    \hline
\end{tabular}

\begin{thmbox}
    \subsection{Theorem (Properties of Cumulative Distribution Functions)}
    All cumulative distribution functions must have the following properties:

    1. $ F(x) $ is a non-decreasing function of $ x $ for all $ x\in\mathbb{R} $

    2. $ 0\le F(x)\le 1 $ for all $ x\in\mathbb{R} $

    3. $ \lim\limits_{{x} \to {-\infty}} F(x)=0 $

    4.$ \lim\limits_{{x} \to {\infty}} F(x)=1 $
\end{thmbox}

\begin{thmbox}
    \subsection{Theorem}
    If $ X $ takes on integer values for $ x $ such that $ x\in A $ and $ (x-1)\in A $,
    \[ f(x)=F(x)-F(x-1) \]
\end{thmbox}
\begin{proof}
    $ F(x)-F(x-1)=P(X\le x)-P(X\le x-1)=P(X=x)=f(x) $
\end{proof}

\textbf{5.2 Discrete Uniform Distribution}
\begin{defbox}
    \subsection{Definition (Discrete Uniform Distribution)}
    Suppose $ X $ takes values $ a,a+1,\ldots ,b $ with all values
    being equally likely. Then $ X $ has a Discrete Uniform distribution
    on the set $ \{a,a+1,\ldots ,b\} $ and we write
    \[ X \thicksim \du[a,b] \]
\end{defbox}

\textbf{Find the probability function, $ f(x)$}

we note there are $ (b-a+1) $ values $ X $ can take so the probability
at each of these values must be $ \frac{1}{b-a+1} $ so that
$ \sum\limits_{x=a}^{b} f(x)=1 $. Hence
\[ f(x)=
\begin{cases}
    \frac{1}{b-a+1},\,x=a,\ldots,b\\
    0,\,\text{otherwise}
\end{cases} \]

\textbf{5.3 Hypergeometric Distribution}
\begin{defbox}
    \subsection{Definition (Hypergeometric Distribution)}
    Suppose we have a collection of $ N $ objects which can be
    classified into two different types, a success (S) and a failure (F).
    Suppose there are $ r $ success and $ N-r $ failures. Pick $ n $
    objects at random without replacement. Let $ X $ be the number of successes
    obtained. Then $ X $ has a \emph{Hypergeometric distribution} and we write
    \[ X \thicksim \hyp(N,r,n) \]
\end{defbox}

\textbf{Find the probability function, $ f(x)$}

There are $ \binom{N}{n} $ points in the sample space $ S $ if we don't
consider the order of selection. There are $ \binom{r}{x} $ ways to choose
the successes from the $ r $ available AND $ \binom{N-r}{n-x} $ ways to choose
the remaining $ (n-x) $ objects from the $ (N-r) $ failures. Hence
\[ f(x)=\frac{\binom{r}{x}\binom{N-r}{n-x}}{\binom{N}{n}} \]
for $ x=0,\ldots ,\min(r,n) $.

\textbf{Example}

Suppose we have $ 10 $ cards, with $ 7 $ that are money cards and $ 3 $
non-money cards. Let $ X= $ \# of money cards in your hand. Then
\[ X \thicksim Hyp(10,7,5) \]
\[ f(x)=\frac{\binom{7}{x}\binom{3}{5-x}}{\binom{10}{5}} \]
for $ x=2,3,4,5 $ (any less and you run out of non-money cards).

\textbf{5.4 Binomial Distribution}
\begin{defbox}
    \subsection{Definition (Bernoulli Trials)}
    (1) Suppose an experiment has two distinct outcomes, call them a success (S)
    and a failure (F).
    
    (2) Let their probabilities $ p $ for S and $ (1-p) $ for F.
    
    (3) Repeat the experiment $ n $ independent times.
    
    Then, the $ n $ individual experiments in the process are called
    \emph{Bernoulli trials}.
\end{defbox}
\begin{defbox}
    \subsection{Definition (Binomial Distribution)}
    Suppose an experiment follows the Bernoulli trials. Then $ X $ has a
    \emph{Binomial distribution} and we write
    \[ X \thicksim \bin(n,p) \]
\end{defbox}

\textbf{Find the probability function, $ f(x)$}

There are $ \binom{n}{x} $ different arrangements of $ x $ S's and $ (n-x) $ F's
over $ n $ trials. Since the trials are independent, the probability of a success
is $ p $ multiplied $ x $ times, and a failure is $ (1-p) $ multiplied $ (n-x) $ times.
Thus we have
\[ f(x)=\binom{n}{x}p^x(1-p)^{n-x} \]
for $ x=0,\ldots ,n $.
