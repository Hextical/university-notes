\section{Lecture 13}
\subsection{Summary}

Today we stared with our SWAG on counting cards in Blackjack (more info on Learn), which is a great application of the Hypergeometric distribution (and why, if you were to play with an infinite number of decks, it would be a Binomial.)

Then we defined the Negative Binomial distribution. It is also based on Bernoulli trials, but instead of doing n trials and counting the S's (as with Binomial), we do trials until we obtain k Successes and count how many Failures occurred along the way. You can usually recognize a Negative Binomial situation if we are waiting until something occurs. The range is all non-negative integers, and we found its pf is $f(x)=\binom{x+k-1}{k-1}p^x(1-p)^x$. It's sometimes tricky to tell the difference between Binomial and Negative Binomial, but the key is that the very last trial must be a Success with NB, or else we would have stopped doing trials sooner.

We looked at an example of deriving the distribution of the total number of trials using the Negative Binomial for the number of Fails.

Unfortunately there is no nice closed-form expression for the cdf $F(x)$ for either the Hypergeometric, Binomial, or Negative Binomial rvs. If you wanted it,you would just have to add up the $f(x)$ values.

Good luck on the midterm tomorrow!

\textbf{Recall}

Binomial approximation to Hypergeometric distribution:
If $ X\thicksim\hyp (N,r,n) $, we can approximate it with 
$ \bin (n,\frac{r}{N}) $ if $ n $ is a small proportion of $ N $.

\subsection{Negative Binomial Distribution (5.5)}
Setup: Bernoulli Trials
\begin{itemize}
    \item independent
    \item each trial is a success or fail (S or F)
    \item $ P(\text{success})=p=\text{constant} $ 
\end{itemize}
Suppose we want to get $ k $ S's. We do trials until we get
$ k $ S's and let $ X= $ \# of F's. We get
\[ X\thicksim \nb(k,p) \]
in a total of $ k+X $ trials.


\begin{center}
    \begin{tabular}{|c|c|}
        \hline
        Binomial & Negative Binomial \\
        \hline
        know \# of trials & unknown \# of trials\\
        unknown \# of S's & known \# of S's\\
        $ \binom{n}{x}p^x(1-p)^{n-x} $ & 
        $ \binom{x+k-1}{k-1}p^k(1-p)^x $\\
        \hline
    \end{tabular}
\end{center}

\subsection{Example}
How many tails until we get the $10$th head on a fair coin.
$ X\thicksim\nb(10,\frac{1}{2}) $

\subsection{Example}
If courses were independent with probability $ p $ of passing
and you need $ 40 $ courses, then the number of failed courses
would be $ \nb(40,p) $.

\subsection{Range and Probability Function of the Negative Binomial Distribution}
range $ x\in\{0,1,\dots\} $ (countably infinite)
\begin{align*}
    f(x)=P(X=x)
    &=p(x\text{ F's before $ k $th S})\\
    &=\binom{x+k-1}{x}p^k(1-p)^x\\
    &=\binom{x+k-1}{k-1}p^k(1-p)^x
\end{align*}
In a picture:
\[\overbrace{\underbrace{\text{\_ \_ \_ $\ldots$ \_}\mid}_
{(k-1) \text{ S's, }x \text{ F's }}
\underbrace{\text{ S}}_{k\text{th S}}}^{x+(k-1)\text{ Trials}}\]

\subsection{Example}
Suppose a startup is looking for $ 5 $ investors. They ask
investors repeatedly where each independently has a $ 20\% $ chance
of saying yes. Let $ X= $ total \# of investors that they ask and
note that $ X $ does not follow a negative binomial distribution.
Find $ f(x) $ and $ f(10) $.

Let $ Y= $ \# who say no before $ 5 $ say yes.
$ Y\thicksim\nb(5,0.2) $, and $ X=Y+5 $. So,
\begin{align*}
    f(x)&=P(X=x)\\
    &=P(Y+5=x)\\
    &=P(Y=x-5)\\
    &=\binom{(x-5)+5-1}{5-1}(0.2)^5(0.8)^{x-5}\\
    &=\binom{x-1}{4}(0.2)^5(0.8)^{x-5} \qquad \text{for } x=5,\ldots
\end{align*}
\[ f(10)=\binom{9}{4}(0.2)^5(0.8)^5 \]
note that it's $ \binom{9}{4} $ and not $ \binom{10}{5} $ because
the 10th investor must have said yes.