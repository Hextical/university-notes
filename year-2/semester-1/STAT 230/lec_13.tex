\section{Lecture 13}
\textbf{Recall}

Binomial approximation to Hypergeometric distribution:
If $ X\thicksim\hyp (N,r,n) $, we can approximate it with 
$ \bin (n,\frac{r}{N}) $ if $ n $ is a small proportion of $ N $.

\subsection{Negative Binomial Distribution}
Setup: Bernoulli Trials
\begin{itemize}
    \item independent
    \item each trial is a success or fail (S or F)
    \item $ P(\text{success})=p=\text{constant} $ 
\end{itemize}
Suppose we want to get $ k $ S's. We do trials until we get
$ k $ S's and let $ X= $ \# of F's. We get
\[ X\thicksim \nb(k,p) \]
in a total of $ k+X $ trials.

\begin{tabular}{|c|c|}
    Binomial & Negative Binomial \\
    \hline
    know \# of trials & unknown \# of trials\\
    unknown \# of S's & known \# of S's\\
    $ \binom{n}{x}p^x(1-p)^{n-x} $ & 
    $ \binom{x+k-1}{k-1}p^k(1-p)^x $ 
\end{tabular}

\subsection{Example}
How many tails until we get the $10$th head on a fair coin.
$ X\thicksim\nb(10,\frac{1}{2}) $

\subsection{Example}
If courses were independent with probability $ p $ of passing
and you need $ 40 $ courses, then the number of failed courses
would be $ \nb(40,p) $.

\subsection{Range and Probability Function of the Negative Binomial Distribution}
range $ x\in\{0,1,\dots\} $ (countably infinite)
\begin{align*}
    f(x)=P(X=x)
    &=p(x\text{ F's before $ k $th S})\\
    &=\binom{x+k-1}{x}p^k(1-p)^x\\
    &=\binom{x+k-1}{k-1}p^k(1-p)^x
\end{align*}
In a picture:
\[\underbrace{\text{\_ \_ \_ $\ldots$ \_}\mid}_
{k-1 \text{ S's, }x \text{ F's}}
\underbrace{\text{ S}}_{k\text{th S}}\]
with $ x+k-1 $ trials.

\subsection{Example}
Suppose a startup is looking for $ 5 $ investors. They ask
investors repeatedly where each independently has a $ 20\% $ chance
of saying yes. Let $ X= $ total \# of investors that they ask and
note that $ X $ does not follow a negative binomial distribution.
Find $ f(x) $ and $ f(10) $.

Let $ Y= $ \# who say no before $ 5 $ say yes.
$ Y\thicksim\nb(5,0.2) $, and $ X=Y+5 $. So,
\begin{align*}
    f(x)&=P(X=x)\\
    &=P(Y+5=x)\\
    &=P(Y=x-5)\\
    &=\binom{(x-5)+5-1}{5-1}(0.2)^5(0.8)^{x-5}\\
    &=\binom{x-1}{4}(0.2)^5(0.8)^{x-5} \qquad \text{for } x=5,\ldots
\end{align*}
\[ f(10)=\binom{9}{4}(0.2)^5(0.8)^5 \]
note that it's $ \binom{9}{4} $ and not $ \binom{10}{5} $ because
the 10th investor must have said yes.