\section{Lecture 4}
Always ask:

1. Can you get the same object twice?

- Yes $ \rightarrow $ $ n^k $

- No $ \rightarrow $ Step 2.

2. Does the order matter?

- Yes $ \rightarrow $ $ n^{(k)} $

- No $ \rightarrow $ today's lesson $ \binom{n}{k} $

\textbf{Examples}

4 numbers between $ 0 $ and $ 255 $
\begin{itemize}
    \item total possible: $256^4$
    \item all odd numbers $ 128^4 $ so $ P(\text{all odd})=\nicefrac{1}{16} $
    \item at least one odd number; work with the opposite: all even: $ 128^4 $,
    so $ P(\text{at least one odd})=1-\nicefrac{128^4}{256^4} $
\end{itemize}

\textbf{Example}

$ 5 $ people $ A,\,B,\,C,\,D $ $ 4 $ co-op jobs $ 1,\,2,\,3,\,4 $

Find the probability that $ A $ gets a job.

\textbf{Solution.}

1. order matters $ \rightarrow $ (permutation of some sort)

2. 1 + without replacement $ \rightarrow $ $ n^{(k)} $

so total ways is $ 5^{(4)}=120 $

$ A,\_,\_,\_ $ or $ \_,A,\_,\_ $ or $\_,\_,A,\_$ or $\_,\_,\_,A$ 

$ 4^{(3)}=96 $

So probability they do is$ \frac{96}{120} =0.8 $

Alternatively, \# of ways for $ A $ to not get a job is $ 4^{(4)} $ or $ 4! $
(they are the same quantity).
So probability they do is $ 1-\nicefrac{4!}{120}=0.8 $.

Intuitively this makes sense because each of the $ 5 $ is equally likely
$ (\nicefrac{1}{5}) $ to not get a job.

Find probability that $ B $ and $ C $ get adjacent jobs.

$ BC,\_,\_ $ or $ \_,BC,\_ $ or $ \_,\_,BC $

$ CB,\underbrace{\_,\_}_{3^{(2)}} $ or $ \_,CB,\_ $ or $ \_,\_,CB $

So total ways is $ 6\times 3^{(2)}=36 $, probability=$ \nicefrac{36}{120} $.
Alternatively, treat $ BC $ as one unit with $ 2 $ ways it can look
(BC or CB).

\textbf{3.3 Counting Subsets or Combinations}

\begin{defbox}
    \subsection{Definition (Combination)}
    A \emph{combination} is an unordered selection of $ k $
    objects chosen from $ n $ objects.

    If we select the objects above without replacement, we write
    \[ ^nC_k=\frac{n!}{k!(n-k)!}=\binom{n}{k} \]
\end{defbox}

How many ways? If we did care, $ n^{(k)} $. Then deliberately forgot the order.

e.g. select $ 3 $ digits $ 0-9 $
\[ \{8,2,1\} \]
if we care about the order, each set is counted $ 3!=6 $ times.

So, there are $ \nicefrac{10^{(3)}}{3!}=120 $ possible sets of $ 3 $ digits.

This quantity is called
\[ \binom{n}{k}=\frac{n^{(k)}}{k!}=\frac{n!}{k!(n-k)!}=\, ^nC_k \]
"$ n $ choose $ k $", "binomial coefficient",
"$ \binom{n}{k} $ is the $ k^{th} $ element of the $ n^{th} $ row of Pascal's
$ \Delta $"

\textbf{Example}

Lotto $ 6/49 $, choose $ 6 $ winning \# from $ 49 $. The order of the numbers
does not matter.

\[ \binom{49}{6}\text{ ways }\approx 13.9 \text{ million} \]
