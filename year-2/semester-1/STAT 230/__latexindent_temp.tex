\makeheading{Lecture 15}
\textbf{Example}

Naomi invites $ 12 $ people to her party. If each independently comes with
probability $ p $.
Let $ X= $ \# of guests.

\[ X \sim \bin(12,p) \]

\textbf{Example}

$ 20 $ toys in a machine. Each time you grab one with a claw.
Let $ X= $ \# of tries to get one toy you want.

\emph{None.}

\textbf{Example}

Trying to catch a Pokémon, each time has a probability $ p $ of succeeding.
Let $ X= $ \# of failed attempts.

\[ X \sim \geo(p) \]

\textbf{Example}

You have $ 5 $ classes randomly scheduled in a row.

Let $ X= $ \# of classes before your favourite.

The range of $ X $ is: $ 0,1,2,3,4 $, and the probability is
$ \sfrac{1}{5} $ for each of the range. We have the following:

\[ X \sim \du[1,4] \]

\section{Poisson Distribution from Binomial}

Suppose we have a $ X \sim \bin(n,p) $ where $ n $ is very large
and $ p $ is very small. Then, as $ n\rightarrow \infty $ and $ p\rightarrow 0 $
such that $ np $ remains constant, the probability function of X
approaches a limit.

Let $ np=\mu $, so $ p=\frac{\mu}{n} $. Then
\begin{align*}
    \lim\limits_{{n} \to {\infty}} f\left(x\right)
     & =\lim\limits_{{n} \to {\infty}} \binom{n}{x}p^x\left(1-p\right)^{n-x}                            \\
     & =\lim\limits_{{n} \to {\infty}} \frac{n\left(n-1\right)\cdots\left(n-x+1\right)}{x!}
    \frac{\mu^x}{n^x} \left(1-\frac{\mu}{n}\right)^n\left(1-\frac{\mu}{n}\right)^{-x}                   \\
     & =\frac{\mu^x}{x!} \lim\limits_{{n} \to {\infty}} \frac{n}{n}
    \frac{n-1}{n} \cdots \frac{n-x+1}{n}\left(1-\frac{\mu}{n}\right)^n\left(1-\frac{\mu}{n}\right)^{-x} \\
     & =\frac{\mu^x}{x!} \lim\limits_{{n} \to {\infty}} \left(1-\frac{\mu}{n}\right)^n                  \\
     & =\frac{e^{-\mu}\mu^x}{x!}                                                                        \\
\end{align*}

We write: $ X \sim \poi\left(\mu\right) $, range: $ 0,1,\ldots $

We can use the Poisson random variable as an approximation to the Binomial
when $ n $ is large, and $ p $ is small. The only thing we need to do
is $ \mu=np $.

\textbf{Example}

Tim Hortons roll up the rim says $ 1 $ in $ 6 $ cups win a prize. Suppose
you have $ 80 $ cups. Find the probability that you get $ 10 $ or fewer
winners.

Let $ X= $ \# of winning cups. $ X \sim \bin(80,\sfrac{1}{6}) $
We want
\begin{align*}
    F(10) & =P(X\le 10)                                                                   \\
          & =\sum\limits_{x=0}^{10} f(x)                                                  \\
          & =\binom{80}{0}\left(\frac{1}{6}\right)^0\left(\frac{5}{6}\right)^{80}+\cdots+
    \binom{80}{10}\left(\frac{1}{6}\right)^{10}\left(\frac{5}{6}\right)^{70}              \\
          & =0.2002 \text{ (tedious) }
\end{align*}
Try a Poisson approximation.
$ Y \sim \poi(\mu=np=\frac{80}{6}\approx 13.33) $. Then,

$ P(Y\le 10)=e^{-13.33}\left[1+\frac{13.33}{1!}+\cdots+\frac{13.33^{10}}{10!}\right]=0.224 $

Not a good approximation since $ p $ was too large.

\section{Poisson Distribution from Poisson Process}

\begin{defbox}
    \subsection{Definition (Poisson Process)}
    \emph{Poisson Process}:
    Suppose events occur randomly in time or space according to three conditions:
    \begin{enumerate}[label=(\arabic*)]
        \item Independence: the number of events in one period cannot affect another non-overlapping period
        \item Individuality: events occur one at a time (cannot have two at the exact same time)
        \item Homogeneity or Uniformity: events occur at a constant rate
    \end{enumerate}
\end{defbox}
