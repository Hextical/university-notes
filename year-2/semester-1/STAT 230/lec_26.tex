\section{Lecture 26}
\subsection{Example}
If $ Z \thicksim N(0,1) $, find $ d $ such that $ P(|Z|<d)=0.9 $.
\begin{align*}
    P(-d<Z<d)&=1-P(-d\ge Z\ge d)\\
    &=1-P(Z\le -d)\\
    &=1-P(Z>d)\\
    &=1-[1-P(Z\le d)]\\
    &=P(Z\le d)
\end{align*}
\[ P(Z\le d)=0.9\]
So $ d=1.286 $.

\subsection{Basic Terminology and Techniques (9.1)}
We have models for a single RV (both discrete or cts) but we often
care about two or more RV's at the same time (and their relationship)
Examples:
\begin{itemize}
    \item two stock returns
    \item heights and weights
    \item number of cards of a rank vs number of a suit
    \item treatment vs recovery time
    \item all machine learning classification and regression
\end{itemize}
In this course, we focus on all discrete random variables

\begin{defbox}
    The joint probability function of two random variables $ X $ and $ Y $ is
    \[ f(x,y)=P(X=x,Y=y) \]
    for all $ (x,y) $ in the joint range.
\end{defbox}

\subsection{Example}
Suppose we flip a coin 3 times. Let $ X= $ \# heads.
Let
\[ Y=\begin{cases}
    1,\,\text{ if first flip is a H}\\
    0,\, \text{ otherwise}
\end{cases} \]
Find $ f(x,y) $.

\begin{tabular}{| *{5}{>{\centering\arraybackslash}p{2cm} |}}
    \hline
    $y\backslash x$ & 0 & 1 & 2 & 3\\
    \hline
    $0$ & $ \nicefrac{1}{8}$ & $ \nicefrac{2}{8} $ & $ \nicefrac{1}{8} $ & $\nicefrac{0}{8} $\\
    \hline
    $1$ & $ \nicefrac{0}{8}$ & $ \nicefrac{1}{8} $ & $ \nicefrac{2}{8} $ & $\nicefrac{1}{8} $\\
    \hline
\end{tabular}
$ f(x,y) $ can be represented in a table or as a function of $ x $ and $ y $
(not usually a histogram).

In general, the joint pf of $ X_1,\ldots,X_n $ is
\[ f(x_1,\ldots,x_n) =P(X_1=x_1,\ldots,X_n=x_n) \]

Properties:
\begin{itemize}
    \item $ \sum\limits_{x} \sum\limits_{y} f(x,y)=1 $
    \item $ f(x,y)\ge 0 $ for all $ (x,y) $
\end{itemize}

Now suppose we are only interested in one of the random variables. e.g. suppose
we are only want to find out about $ X $.
\[ P(X=x)=f(0,0)+f(0,1)=\frac{1}{8} +0=\frac{1}{8} \]
In general, we define the marginal pf of $ X $ as
\[ f_X(x)=\sum\limits_{y}f(x,y)=P(X=x) \]
and the marginal pf of $ Y $ is
\[ f_Y(y)=\sum\limits_{x}f(x,y)=P(Y=y) \]

\begin{defbox}
    Two discrete random variables $ X $ and $ Y $ are independent iff
    \[ P(X=x,Y=y)=P(X=x)P(Y=y) \]
    \[ f(x,y)=f_X(x)f_Y(y) \]
    for all $ (x,y) $
\end{defbox}
From example: Are $ X $ and $ Y $ independent? No.
$ f(0,0)=\frac{1}{8} \neq f_X(0)f_Y(0)=\frac{1}{8}\cdot \frac{1}{2} $
Shortcut: any $ 0 $ in your table $ \rightarrow $ dependent.
