\section{Lecture 3}

\textbf{\myuline{Chapter 3: Probability and Counting Techniques}}

\myuline{3.1 Addition and Multiplication Rules}

We need a systematic way to count outcomes without listing them.

\begin{defbox}
    \subsection{Counting Rules}
    There are two basic counting rules:

    1. The \emph{Addition} Rule:
    Suppose we can do job $1$ in $p$ ways and job $2$ in $q$ ways.
    Then we can do either job $1$ OR job $2$ (but not both), in $p + q$ ways.
    
    2. The \emph{Multiplication} Rule:
    Suppose we can do job $1$ in $p$ ways and job $2$ in $q$ ways.
    Then we can do both job $1$ AND job $2$ in $p \times q$ ways.
\end{defbox}

\myuline{3.2 Counting Arrangements or Permutations}

\myuline{Sampling with replacement}: it is possible to obtain
the same result more than once. e.g. die rolls, coin flip, slot machine, password.

\myuline{Sampling without replacement}: once a result occurs, it cannot happen again.
e.g. drawing cards, balls from an urn, eating candy of different colour.

\begin{defbox}
    \subsection{Definition (Permutation)}
    A \emph{permutation} is an ordered selection of $ k $
    objects chosen from $ n $ objects.

    If we select the objects above without replacement, we write
    \[ ^nP_k =n(n-1)\cdots(n-k+1)=n^{(k)} \]
    If we select the objects above with replacement, we write
    \[ n^k \]
\end{defbox}
$ n^{(0)}=1 $

$ n^{(n)}=n! $

$ k>n\rightarrow 0 $ not possible

\myuline{Example}

IP addresses: an ordered sequence of four numbers between $ 0 $ and $ 255 $.
e.g. $ 192.168.1.1 $, $ 129.97.95.107 $, etc. Determine the
total possible outcomes with and without replacement.

\emph{Solution.}

Since order matters, we are immediately looking at a permutation.

With replacement: $ 256^4 $

Without replacement: $ 256^{(4)} $