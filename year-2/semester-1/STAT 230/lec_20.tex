\section{Lecture 20*}
\subsection{Summary}
We started with one last Chapter 7 example (using the mean and variance of the Poisson, and the properties of linear transformations of random variables) and our SWAG.

Then we summarized everything we know about discrete random variables, which turned out to be quite a lot! I talked about which aspects will remain the same and which will change when we move into the continuous case.

Surprisingly, a lot stays the same. The main differences come from the fact that we have an uncountable number of values the variable can take on, so $P(X=x) = 0$ for all $x$, and all our sums become integrals. We found that we need the derivative of the cumulative distribution function (which we call the probability density function or pdf $f(x)$) to tell us information about the random variables local behaviour, and so the relationship between $F(x)$ and $f(x)$ is now a derivative-integral relationship rather than a sum-difference relationship like in the discrete case. Most of the properties of $F(x)$ and $f(x)$ remain the same, but not all. Many of the discrete distributions have a similar distribution in the continuous world: in this course we'll only talk about 3: Uniform, Exponential, and Normal. Everything about expected value and variance remains the same, including how linear transformations work, we just have to evaluate them with integrals instead of sums.

Next time a guest lecturer will formally define all these quantities and properties, and look at some examples of continuous random variables.

\subsection{Example}
Suppose the amount of data you use on your phone (in units of $ 100 $MB) has
a Poisson distribution with mean $ 7 $ per month. You pay $ 15 $ per
month plus $ 3 $ per $ 100 $MB. Find the standard deviation of random
month's phone bill.

Let $ X= $ \# of units of data used. $ X \thicksim \poi(7) $. Let
$ Y=15+3X\rightarrow E[Y]=15+3(7)=36 $. 

$ SD(Y)=3SD(X)=3\sqrt{7}=7.94. $