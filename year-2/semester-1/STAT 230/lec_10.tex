\section{Lecture 10}
MLIW \# 3: Naïve Bayes' Classifier

In ML classification, we use evidence to decide what category something
belongs to. That is,
\[ P(\text{category}|\text{evidence})=\frac{P(\text{category})P(\text{evidence})}
{\sum\limits_{\text{cat }i} P(\text{cat }i)P(\text{evidence}|\text{cat }i)} \]
The NBC assumes that if there are multiple types of evidence, their probabilities
are independent \myuline{conditional on the category}.

For example, Spam detection. Let $ A_1= $ fail rdns checked (sender spoofed).
Let $ A_2= $ sends to over $ 100 $ people. Let $ A_3= $ link text doesn't
match actual URL. Then what is
\[P(\text{Spam}|A_1A_2A_3) =\frac{P(A_1A_2A_3|\text{Spam})}{P(A_1A_2A_3|\text{Spam})P(\text{Spam})+
P(A_1A_2A_3|\overline{\text{Spam}})P(\overline{\text{Spam}})}  \]

\textbf{\myuline{Chapter 5: Random Variables}}

\myuline{5.1 Random Variables and Probability Functions}
\begin{defbox}
    \subsection{Definition (Random Variable)}
    A \emph{random variable} is a function that assigns a real number to each point in
    a sample space $S$.
\end{defbox}
We typically use $ X,\,Y,\,Z $ as random variables and $ x,\,y,\,z $ as the
possible values the random variable can take on. There are two types
of random variables based on the range.
\begin{defbox}
    \subsection{Definition (Discrete Random Variables)}
    \emph{Discrete random variables} take integer values or, more generally, values in a
    countable set.
\end{defbox}
\begin{defbox}
    \subsection{Definition (Continuous Random Variables)}
    \emph{Continuous random variables} take values in some interval of real numbers
    like $(0,1)$ or $(1,\infty)$ or $ (-\infty,\infty) $.
\end{defbox}
We can define multiple random variables on the same sample space $ S $. For example,
roll a fair $ 6 $-sided die $ 3 $ times.
\[ S=\{(x,y,z)|1\le x, y, z\le 6\} \]

\begin{itemize}
    \item Let $ X= $ sum on the three die. $range(X) =\{3,\ldots ,18\} $.
    \item Let $ Y= $ product on the three die. $range(Y) =\{1,\ldots ,216\} $.
    \item Let $ Z= $ number on the first die. $range(Z) =\{1,\ldots ,6\} $
    \item Let $ \bar{X}= $ average. $range(\bar{X}) =\{1,\nicefrac{4}{3},\ldots ,6\} $
    \item Let $ W= $ \# of dice that are $ 1 $. $range(W)= \{0,1,2,3\} $
\end{itemize}

Examples of continuous random variables include:
\begin{itemize}
    \item $ T= $ time until an event
    \item $ P= $ positive in space of a particle
    \item $ H= $ height of a random person.
\end{itemize}

For Chapter 5, we will focus on discrete random variables.
\begin{defbox}
    \subsection{Definition (Probability Function, Probability Distribution)}
    Let $X$ be a discrete random variable with $range(X)$ = $A$. 
    The \emph{probability function} of $X$ is the function
    \[ f(x)=P(X=x) \]
    for all $ x\in A $.

    The set of pairs $ \{(x,f(x)):x\in A\} $ is called the \emph{probability
    distribution} of $ X $.
\end{defbox}

\begin{thmbox}
    \subsection{Theorem (Properties of Probability Functions)}
    All probability functions must have the two properties:
    
    1. $0\le f(x)\le 1 $ for all $ x\in A $

    2. $ \sum\limits_{\text{all }x\in A} f(x)=1 $
\end{thmbox}

\myuline{Example}

Let $ X= $ \# of dice that are $ 1 $.

\begin{tabular}{| *{5}{>{\centering\arraybackslash}p{1.5cm} |}}
    \hline
    $x$ & $0$ & $1$ & $2$ & $3$\\
    \hline
    $f(x)$ & $\nicefrac{5^3}{6^3}$ & $\nicefrac{3\times 5^2}{6^2}$ & $\nicefrac{3\times 5}{6^3}$ & $\nicefrac{1^3}{6^3}$ \\
    \hline
\end{tabular}