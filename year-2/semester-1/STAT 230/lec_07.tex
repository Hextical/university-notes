\section{Lecture 7}
Roll two fair $ 12 $-sided die. What is the probability at least one of them
is greater than $ 7 $.
\begin{align*}
    1-P(\text{neither})&=1-P(\overline{A\cap B})\\
    &=1-P(\bar{A}\cup \bar{B})\\
    &=1-\frac{7}{12}\frac{7}{12}
\end{align*}
In this example, we relied on the multiplication rule to find a probability
on both events, but this requires the events to not influence each other.

\myuline{4.3 Intersections of Events and Independence}

\begin{defbox}
    \subsection{Definition (Independent, Dependent)}
    $ A $ and $ B $ are \emph{independent} if and only if
    \[ P(AB) =P(A)P(B)\]
    If the events are not independent, we call the events \emph{dependent}.
\end{defbox}
We can use this in two ways.

1. If we know both events are independent, we can calculate $ P(AB) $.

2. Calculate/estimate all three probabilities and check whether independent.
e.g. treatment vs recover, smoking vs cancer, income vs politics

Note for $ A,\,B,\,C $ to be independent, we need
\begin{align*}
    &P(AB)=P(A)P(B)\\
    &P(BC)=P(B)P(C)\\
    &P(AC=P(A)P(C)\\
    &P(ABC)=P(A)P(B)P(C)
\end{align*}
\myuline{Example}

Roll $ 2 $ fair $ 6 $-sided dice. Let $ A=$ the first die is $ 3 $. Let
$ B= $ the total is $ 7 $. Are $ A $ and $ B $ independent?

\begin{align*}
    &P(A)=\frac{1}{6}\\
    &P(B)=\frac{6}{36}\\
    &P(AB)=\frac{1}{36}\\
    &P(AB)=\frac{1}{36}=\frac{1}{6} \frac{1}{36}=P(A)P(B)
\end{align*}

Now, let $ C =$ the total is $ 8 $. Are $ A $ and $ C $ independent?

\begin{align*}
    &P(C)=\frac{5}{36}\\
    &P(AC)=\frac{1}{36}\\
    &P(AC)=\frac{1}{36}\neq\frac{1}{6} \frac{5}{36}=P(A)P(C)
\end{align*}
Why? With $ 7 $ there is always a possible second roll, but with $ 8 $ it's
not always possible (e.g. if the first die was a $ 1 $).

\myuline{Independence vs. Mutual Exclusive}

\begin{tabular}{| *{4}{>{\centering\arraybackslash}p{3cm} |}}
    \hline
    & ME & ID & Both \\ \hline
    math & $AB=\emptyset,\,P(AB)=0$ & $ P(AB)=P(A)P(B) $ & $ P(A)P(B)=0 $ \\ \hline
    logic & both can't happen & one doesn't affect the other & at least one is impossible \\ \hline
\end{tabular}

If events are dependent, we might want to quantify the effect of one on the other.

\myuline{4.4 Conditional Probability}

\begin{defbox}
    \subsection{Definition (Conditional Probability)}
    The \emph{conditional probability} of $ A $, given $ B $ is
    \[ P(A|B)=\frac{P(AB)}{P(B)} \]
    provided $ P(B)>0 $.
\end{defbox}
Why? The classical definition of probability:
\[ \frac{\text{\# ways $A$ can occur}} {\text{\# ways $ B $ can occur}} \]
Since we need to restrict $ S $ to just be $ B $,
\[ P(A|C)=\frac{P(AC)}{P(C)}=\frac{\frac{1}{36}}{\frac{5}{36}}=\frac{1}{5}>P(A) \]
or $ C:\{(2,6),(3,5),(4,4),(5,3),(6,2)\} $
\[ P(C|A)=\frac{P(CA)}{P(A)}=\frac{\frac{1}{36}}{\frac{1}{6}}=\frac{1}{6}>P(C) \]
or $ A:\{(3,1),(3,2),(3,3),(3,4),(3,5),(3,6)\} $

Dependence is a two way relationship. Both influence the other in the same
direction.