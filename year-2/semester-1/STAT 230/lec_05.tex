\section{Lecture 5}
\begin{thmbox}
    \subsection{Theorem (Properties of n choose k)}
    Let $ n,\,k\in \mathbb{Z} $ be non-negative.
    
    1. $ n^{(k)}=\frac{n!}{(n-k)!}=n(n-1)^{(k-1)} $ for $ k\ge 1 $

    2. $ \binom{n}{k}=\frac{n^{(k)}}{k!}=\frac{n!}{k!(n-k)!} $

    3. $\binom{n}{k}=\binom{n}{n-k}$ for all $ k=1,\ldots,n $

    4. If we define $ 0!=1 $, then $\binom{n}{0}=\binom{n}{n}=1$

    5. Pascal's Identity: $\binom{n}{k}=\binom{n-1}{k-1}+\binom{n-1}{k}$

    6. Binomial Theorem: $ (1+x)^n=\binom{n}{0}+\cdots +\binom{n}{n}x^n $
\end{thmbox}

\myuline{Example}

Lotto $ 6/49 $ $ \binom{49}{6} $ possible sets of winning \#'s. If your
ticket contains all $ 6 $ winning \#'s you win.
\[ P(\text{win})=\frac{\binom{6}{6}\binom{43}{0}}{\binom{49}{6}} \]
\[ P(\text{match }5)=\frac{\binom{6}{5}\binom{43}{1}}{\binom{49}{6}} \]

\myuline{Example}

Suppose you select $ 5 $ cards from $ 52 $ ($ 13 $ cards of each $ 4 $ suits).
Find the probability of $ 3 $ of one rank, $ 2 $ of another rank.

Total \# of hands: $ \binom{52}{5} $.
\# with $ 3 $ of one, $2$ of another:
\[ \underbrace{\text{rank of triple}}_{13}\times \binom{4}{3}\times
\underbrace{\text{rank of pair}}_{12}\times \binom{4}{2} \]

\myuline{3.4 Number of Arrangements When Symbols Are Repeated}

Suppose we have $ 5 $ objects, $ 2 $ of which are alike:
\[ \text{D I A N A} \]
If we arrange the objects in order, how many results can we get?

If we could tell them apart: $ 5! $ ways. But every possible arrangement
has a matching one with A's flipped.

So, $ \nicefrac{5!}{2!} =60 $ ways (removing double counting)

In general, if we have $ n $ objects:
\[
\left.
\begin{array}{ccc}
    n_1&\text{ of } & \text{type $1$}\\
    &\vdots\\
    n_k&\text{ of } & \text{type $k$}\\
\end{array}
\right\}n_1+\cdots +n_k=n
\]
How many ways can the objects be arranged, where objects
of the same type are identical?

$STATISTICS$: $ n=10 $
\[
\begin{array}{cc}
    S:3& n_1\\
    T:3& n_2\\
    A:1& n_3\\
    I:2& n_4\\
    C:1& n_5
\end{array}
\]
ways to place:
\[
\begin{array}{cc}
    S& \binom{10}{3}\\
    T& \binom{7}{3}\\
    A& \binom{4}{1}\\
    I& \binom{3}{2}\\
    C& \binom{1}{1}
\end{array}
\]
So in total there are
\begin{align*}
    \binom{10}{3}\binom{7}{3}\binom{4}{1}\binom{3}{2}\binom{1}{1}&=
    \frac{10!}{3!7!}\frac{7!}{3!4!} \frac{4!}{1!3!} \frac{3!}{2!1!}\frac{1!}{1!0!}\\
    &=\frac{10!}{3!3!2!}  
\end{align*}
In general,
\[ \frac{n!}{n_1!\cdots n_k!} \]