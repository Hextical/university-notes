\section{Lecture 17}
\subsection{Expectation of a Random Variable (7.2)}
Imagine we know the theoretical probability of each
\# of kids in a family.

\begin{tabular}{| *{6}{>{\centering\arraybackslash}p{1cm} |}}
    \hline
    $x$ & 1 & 2 & 3 & 4 & 5\\
    \hline
    $f(x)$ & 0.43 & 0.4 & 0.12 & 0.04 & 0.01\\
    \hline
\end{tabular}

Now we replace the observed proportion in the sample mean
with $ f(x) $
\[ \sum\limits_{\text{all } x} xf(x)=(1)(0.43)+(2)(0.4)+
(3)(0.12)+(4)(0.04)+(5)(0.01)=1.8 \]
which is the theoretical mean.

Why do we have sample mean > theoretical mean?
\begin{itemize}
    \item urban vs rural population
    \item income level
    \item sampled max family size but theoretical includes growing families
    \item selection bias: more likely to choose one with more sister/brothers from population
\end{itemize}

\begin{defbox}
    \subsubsection{Definition (Expected Value)}
    Let $X$ be a discrete random variable and probability function $f(x)$. The
    \emph{expected value} (also called the mean or the expectation) of $X$ is
    given by
    \[ \mu=E[x]=\sum\limits_{\text{all } x}xf(x) \]
\end{defbox}
\begin{remark}
    $ \mu $ will be within the range but not necessarily
    equal to a possible value of $ x $.

    We might be interested in the expected value of
    some function of $ X $, $ g(X) $.
\end{remark}

\subsection{Example}
Tax credit of $ \$ 1000 $ plus $ \$250 $ per kid. Find the
average cost.

\begin{tabular}{| *{6}{>{\centering\arraybackslash}p{1cm} |}}
    \hline
    $x$ & 1 & 2 & 3 & 4 & 5\\
    \hline
    $g(x)$ & 1250 & 1500 & 1750 & 2000 & 2250\\
    \hline
\end{tabular}

Average cost = weighted average of $ g(x) $ values
$ =(1250)(0.43)+\cdots+(2250)(0.01)=1450 $


\begin{thmbox}
    \subsubsection{Theorem}
    Let $X$ be a discrete random variable and probability function $f(x)$. The
    expected value of a some function $ g(X) $ of $ X $ is given by
    \[ E[g(X)]=\sum\limits_{\text{all } x} g(x)f(x) \]
\end{thmbox}

Note that $ g(x)=1000+250x$ from last example.

\[ E[g(X)]=1000+250E[x]=1450 \]

