\chapter{Chapter 2: Mathematical Probability Models}
\makeheading{Lecture 2}
\section{Sample Spaces and Probability}

We need a mathematical model to define probability.

\begin{defbox}
    \subsection{Definition (Experiment, Trial, Outcome)}
    We define an \emph{experiment} as a process that can be repeated with
    multiple possible results. We define a \emph{trial} as a single
    repetition of an experiment. We define an \emph{outcome} as
    the results on one trial of an experiment.
\end{defbox}

\begin{defbox}
    \subsection{Definition (Set)}
    A set is a collection of well defined and distinct objects.
\end{defbox}
\begin{remark}
    A set is an unordered list with no repetition.
\end{remark}

\begin{defbox}
    \subsection{Definition (Sample Space)}
    A \emph{sample space} $ S $ is a set of distinct outcomes for an experiment
    or process, with the property that in a single trial, one and only one of
    these outcomes occur.
\end{defbox}

Sample spaces can be \emph{discrete} or \emph{non-discrete}.
\begin{defbox}
    \subsection{Definition (Discrete)}
    Let $ S $ be a sample space. We say $ S $ is \emph{discrete} if it consists
    of a finite or countably infinite set of simple events.
\end{defbox}

\textbf{Example}

Roll a fair 6-sided die repeatedly. Determine some possible sample spaces
for this experiment.

\textbf{Solution.}

Sample space:
\begin{align*}
     & S_1=\{1,2,3,4,5,6\} \text{ $\star$ easiest to work with}                        \\
     & S_2=\{\text{odd, even}\}                                                        \\
     & S_3=\{\text{prime, non-prime}\}                                                 \\
     & S_4=\{6, \text{not } 6\}\text{ $\star$ outcomes don't have to be equally likely
    in a sample space}                                                                 \\
     & S_5=\{\text{a number}\}
\end{align*}
$\star$ need not have equally likely outcomes

\begin{defbox}
    \subsection{Definition (Event, Simple Event, Compound Event)}
    Let $ S $ be a discrete sample space. An \emph{event} in a discrete
    sample space is a subset $ A\subseteq S $. If the event is indivisible so it
    contains only one point, e.g. $A_1$ = $ \{a_1\} $ we call it a \emph{simple event}.
    An event $A$ made up of two or more simple events
    such as $ A=\{a_1,a_2\} $ is called a \emph{compound event}.
\end{defbox}
\begin{remark}
    The notation $ A\subseteq S $ means $ a\in A \implies a\in B $.
\end{remark}

\textbf{Example}

Let $ A= $ a 5 is rolled. Let $ B= $ an odd \# is rolled. Determine which of
the events are simple events and compound events.

\textbf{Solution.}

$ A=\{5\}\subseteq S $, $ B=\{1,3,5\}\subseteq S $. Thus,
$ A $ is a simple event and $ B $ is a compound event.

When the trial is conducted, the outcome determines which events
occur.

If outcome is in the set, it occurs

5 rolled $ \rightarrow $ $ A $ and $ B $ both occur

3 rolled $ \rightarrow $ $ A $ does not occur, $ B $ occurs

2 rolled $ \rightarrow $ neither events occur

\begin{defbox}
    \subsection{Definition (Probability, Probability Distribution)}
    Let $ S=\{a_1,\ldots\} $ be a discrete sample space $ S $. Assign
    numbers (\emph{probabilities}) $ P(a_i) $ for $ i=1,\ldots $ to the $ a_i $'s
    such that the following two conditions hold:
    \begin{enumerate}[label=(\arabic*)]
        \item $ 0\le P(a_i)\le 1 $
        \item $\sum\limits_{\text{all } i}P(a_i)=1 $
    \end{enumerate}
    The set of probabilities $ \{P(a_i),\,i=1,\ldots \} $ is called
    a \emph{probability distribution} on $ S $.
\end{defbox}

\begin{defbox}
    \subsection{Definition (Probability of an Event)}
    The \emph{probability $ P(A) $ of an event} $ A $ is the sum of the
    probabilities for all the simple events that make up $ A $ or
    \[ P(A)=\sum\limits_{a\in A}P(a) \]
\end{defbox}

If a sample space $ S $ has equally likely outcomes then,
\[ P(\text{simple event})=\frac{1}{|S|} \]
\[ P(A)=\sum\limits_{a\in A} P(a_i)=\frac{|A|}{|S|} \]
