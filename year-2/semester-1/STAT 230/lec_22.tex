\section{Lecture 22}
\subsection{Example}
$ F(x)=x^2 $ for $ 0<x<1 $.

\textsc{Find the mean, median, and mode}

Mean:
\[ E[X]=\int\limits_{0}^{1} x2x d{x} =\int\limits_{0}^{1} 2x^2 d{x}=
\left[\frac{2x^3}{3}\right]_0^1=\frac{2}{3} \]

Median:
$ x_{0.5} $ satisfies $ F(x_{0.5})=0.5 \implies (x_{0.5})^2=0.5\implies x_{0.5}=
\sqrt{0.5}=0.707$

Mode: 1 ($ x $ value that maximizes $ f(x) $)

\subsection{Continuous Uniform Distribution (8.2)}
A \emph{continuous} random variable takes real values between $ a $ and
$ b $ with $ a<b $ such that any interval of fixed size is equally likely.

\textsc{Notation}

$ X \thicksim U(a,b) $
\begin{remark}
    Can include or exclude endpoints, doesn't matter.
\end{remark}

\textsc{Find $f(x)$}

$ f(x)=c $, (since it can't depend on $ x $). We need
\[ \int\limits_{-\infty}^{\infty} f(x)\,d{x} =1 \]
\[ \int\limits_{a}^{b} c\,d{x} =1 \]
$ \left[cx\right]_a^b=1\implies c(b-a)=1\implies c=\frac{1}{b-a} $

So,
\[ f(x)=\begin{cases}
    \frac{1}{b-a},\, a\le x\le b\\
    0,\, \text{otherwise}
\end{cases} \]

\textsc{Find $ F(x) $}
\[ F(x)=\int\limits_{-\infty}^{x} f(u) d{u}
=\int\limits_{a}^{x} \frac{1}{b-a} d{u}
=\left[\frac{u}{b-a}\right]_a^x=\frac{x-a}{b-a} \]
\[ F(x)=\begin{cases}
    \frac{x-a}{b-a},\, a<x<b\\
    0,\,x<a\\
    1,\,x>b
\end{cases} \]
\textsc{Find the mean, median and mode}

Mean:
\[ E[X]=\int\limits_{a}^{b} x \frac{1}{b-a} d{x}
=\left[\left(\frac{x^2}{2}\right)\left(\frac{1}{b-a}\right)\right]_a^b
=\frac{b^2-a^2}{2(b-a)}=\frac{b+a}{2} \]

Median: is also $ \frac{a+b}{2} $

Mode: no unique mode

Similarly,
\[ Var(x)=\frac{(b-a)^2}{12} \]

\textsc{Special case}

$ U \thicksim U(0,1) $ (i.e. $ a=0,\,b=0 $)
\[ f(u)=
\begin{cases}
    1,\,0<u<1\\
    0,\,\text{otherwise}
\end{cases} \]
\[ F(u)=
\begin{cases}
    u,\,0<u<1\\
    0,\,u<0\\
    1,\,u>1
\end{cases} \]
$ U(0,1) $ random variables are easy to generate.

\subsection{Change of Variables}
Suppose you know the distribution of $ X $ and you want the distribution of
$ Y=g(X) $.
\begin{enumerate}[1.]
    \item Write the cumulative distribution function of $ Y $ in terms of the cumulative distribution function of $ X $
    \item Sub in what we know about $ X $, then differentiate to get the pdf
    \item Determine the range of $ Y $
\end{enumerate}
\subsection{Example (Change of Variable)}
Let $ X \thicksim U(0,4) $, $ F_X(x)=\frac{x}{4} $, $ f_X(x)=\frac{1}{4} $ 
$ x\in(0,4) $

Let $ Y=\frac{1}{X} $

1.
\begin{align*}
    F_Y(y)&=P\left(Y\le y\right)\\
    &=P\left(\frac{1}{X} \le y\right)\\
    &=P\left(X>\frac{1}{y}\right)\\
    &=1-F_X\left(\frac{1}{y}\right)
\end{align*}

2.
\begin{align*}
    F_Y(y)&=1-F_X\left(\frac{1}{y}\right)\\
    &=1-\frac{\frac{1}{y}}{4}\\
    &=1-\frac{1}{4y}
\end{align*}
$ f_Y(y)=\frac{d}{dx}F_Y(y)=\frac{1}{4y^2} $

OR differentiate $ F_Y(y) $ before
substituting in the information about $ X $. You need the chain rule!
\[ \frac{d}{dy}\left[1-F_X\left(\frac{1}{y}\right)\right]=
-f_X\left(\frac{1}{y}\right)\left(-\frac{1}{y^2}\right)=\frac{1}{4y^2} \]

3. $ y\in(\frac{1}{4},\infty) $
