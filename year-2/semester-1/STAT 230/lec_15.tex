\section{Lecture 15}
\subsection{Summary}

Today we started by looking at some examples of identifying which distribution to use. 

Then we talked about our last discrete distribution: the Poisson distribution. It arises as a limiting case of the Binomial when n gets large and p gets small,such that the product $np = \mu$. We showed that as n approaches infinity, the probability function $f(x)$ approaches $ \frac{e^{-\mu}\mu^x}{x!} $ which is the Poisson pf. In practice, we can use the Poisson to approximate the Binomial when n is reasonably large and p is reasonably close to 0. Like all approximations, the accuracy is better when the conditions are more closely satisfied.

In addition to being a limiting case of the Binomial, it also comes from a Poisson process, which is events occurring throughout time/space with 3 conditions: independence, individuality, and uniformity/homogeneity. Many different processes can be modelled with a Poisson process, and next time we'll see some examples. See if you can think of any before next class!

\subsection{Example}
Naomi invites $ 12 $ people to her party. If each independently comes with
probability $ p $.
Let $ X= $ \# of guests.

\emph{Binomial:} $ X \thicksim \bin(12,p) $

\subsection{Example}
$ 20 $ toys in a machine. Each time you grab one with a claw. 
Let $ X= $ \# of tries to get one toy you want.

\emph{None.}

\subsection{Example}
Trying to catch a pokemon, each time has a probability $ p $ of succeeding.
Let $ X= $ \# of failed attempts.

\emph{Geometric:} $ X \thicksim \geo(p) $

\subsection{Example}
You have $ 5 $ classes randomly scheduled in a row.
Let $ X= $ \# of classes before your favourite.

range: $ 0,1,2,3,4 $, and the probability is $ \nicefrac{1}{5} $ for each
of the range.

\emph{Discrete Uniform:} $ X \thicksim \du[1,4] $

\subsection{Poisson Distribution from Binomial (5.7)}
Suppose we have a $ X \thicksim \bin(n,p) $ where $ n $ is very large
and $ p $ is very small. Then, as $ n\rightarrow \infty $ and $ p\rightarrow 0 $
such that $ np $ remains constant, the probability function of X
approaches a limit.

Let $ np=\mu $, so $ p=\frac{\mu}{n} $. Then
\begin{align*}
    \lim\limits_{{n} \to {\infty}} f(x)
    &=\lim\limits_{{n} \to {\infty}} \binom{n}{x}p^x(1-p)^{n-x}\\
    &=\lim\limits_{{n} \to {\infty}} \frac{n(n-1)\cdots(n-x+1)}{x!}
    \frac{\mu^x}{n^x} (1-\frac{\mu}{n})^n(1-\frac{\mu}{n})^{-x}\\
    &=\frac{\mu^x}{x!} \lim\limits_{{n} \to {\infty}} \frac{n}{n}
    \frac{n-1}{n} \cdots \frac{n-x+1}{n}(1-\frac{\mu}{n})^n(1-\frac{\mu}{n})^{-x}\\
    &=\frac{\mu^x}{x!} \lim\limits_{{n} \to {\infty}} (1-\frac{\mu}{n})^n\\
    &=\frac{e^{-\mu}\mu^x}{x!}\\
\end{align*}

We write: $ X \thicksim \poi(\mu) $, range: $ 0,1,\cdots $

We can use the Poisson random variable as an approximation to the Binomial
when $ n $ is large, and $ p $ is small. The only thing we need to do
is $ \mu=np $.

\subsection{Example}
Tim Hortons roll up the rim says $ 1 $ in $ 6 $ cups win a prize. Suppose
you have $ 80 $ cups. Find the probability that you get $ 10 $ or fewer
winners.

Let $ X= $ \# of winning cups. $ X \thicksim \bin(80,\nicefrac{1}{6}) $
We want
\begin{align*}
    F(10)&=P(X\le 10)\\
    &=\sum\limits_{x=0}^{10} f(x)\\
    &=\binom{80}{0}\left(\frac{1}{6}\right)^0\left(\frac{5}{6}\right)^{80}+\cdots+
    \binom{80}{10}\left(\frac{1}{6}\right)^{10}\left(\frac{5}{6}\right)^{70}\\
    &=0.2002 \text{ (tedious) }
\end{align*}
Try a Poisson approximation.
$ Y \thicksim \poi(\mu=np=\frac{80}{6}\approx 13.33) $. Then, 

$ P(Y\le 10)=e^{-13.33}\left[1+\frac{13.33}{1!}+\cdots+\frac{13.33^10}{10!}\right]=0.224 $

Not a good approximation since $ p $ was too large.

\subsection{Poisson Distribution from Poisson Process (5.8)}
The Poisson Process: Suppose events occur randomly in time or space
according to three conditions:
\begin{enumerate}[(1)]
    \item Independence: the number of events in one period cannot affect another non-overlapping period
    \item Individuality: events occur one at a time (cannot have two at the exact same time)
    \item Homogeneity or Uniformity: events occur at a constant rate
\end{enumerate}