\section{Lecture 21}
\subsection{Summary}
Today you talked more about the properties of $F(x)$ and $f(x)$ in the continuous case. For continuous random variables, the cdf $F(x)$ has the same definition and properties as in the discrete case. In addition, since any specific point $x$ has $P(X=x) = 0$, 
$P(a<X<b) = P(a\le X\le b) = P(a<X\le b) = P(a\le X<b)$
 and $F(x)$ is a continuous function of $x$. 

You also defined the probability density function $f(x)$ = $F^\prime(x)$. It represents the relative likelihood of X being "near" the value x in the following way: $f(x)*\epsilon\approx P(x\le X\le x+\epsilon)$ for a small $\epsilon$. The pdf has similar properties to the pf of a discrete random variable, but not quite the same as it is not a probability. It must be non-negative, and integrates to 1 over the range of $X$. We can also get $F(x)$ from $F(x)$ using an integral from the bottom of the range to $x$.

Finally, you talked about how to calculate expectation and variance in the continuous case, and how to find percentiles of a continuous distribution.

A \emph{continuous random variable} $ X $ maps points in a continuous
sample space to real numbers such that the range is uncountably infinite.

\textsc{Examples of continuous random variables}

Let $ X $ be the number the point spots at.
\begin{enumerate}[(1)]
    \item temperature of a day
    \item length of time until a bus arrives
    \item height of a random person
    \item average height of 10 people
\end{enumerate}

$F(x)=P(X\le x)$

$F(a)=P(X\le a) $

\subsection{Example}
For $ x<0 $, no chance of the point stopping at a number $ <0 $.

For $ x>4 $, $ F(x)=1 $ since the point is certain to stop at a number
below $ 4 $.

$ P(0<x\le 1)=\frac{1}{4}=F(1) $

\[ F(x)=\begin{cases}
    0,\, x<0\\
    \frac{x}{4},\,0\le x\le 4 \\
    1,\,x>4
\end{cases} \]

\textsc{Properties of $ F(x) $}

(1) For all $ x $, $ P(X=x)=0 $. So,
\begin{align*}
    P(a<x\le b)&=P(a\le x\le b)\\
    &=P(a<x<b)\\
    &=P(a\le x<b)
\end{align*}
\begin{remark}
    End points don't matter.
\end{remark}

(2) \begin{align*}
    \lim\limits_{{\epsilon} \to {0}} F(x)-F(x-\epsilon)&=
    \lim\limits_{{\epsilon} \to {0}} P(x-\epsilon<X\le x)\\
    &=P(X=x)\\
    &=0
\end{align*}
Thus $ \lim\limits_{{\epsilon} \to {0}} F(x-\epsilon)=F(x) $, so
$ F(x) $ is continuous.

(3) $ F(x) $ is non-decreasing.

(4) $ \lim\limits_{{x} \to {+\infty}} F(x)=1 $,
$ \lim\limits_{{x} \to {-\infty}} F(x)=0 $

(5) $ 0\le F(x)\le 1 $

\begin{defbox}
    \subsubsection{Definition (Probability Density Function)}
    The \emph{probability density function} (p.d.f) $ f(x) $ for a continuous
    random variable $ X $ is the derivative
    \[ f(x)=\frac{d}{dx}F(x) \]
    where $ F(x) $ is the cumulative distribution function for $ X $.
\end{defbox}

\begin{remark}
    $ f(x) $ is not a probability. It can be $ >1 $ relative
    likelihood that $ X $ takes a value near $ X $.
\end{remark}

\textsc{Properties of $ f(x) $}

(1)
\begin{align*}
    P(a\le X\le b)&=F(b)-F(a)\\
    &=\int\limits_{a}^{b} f(x) d{x}
\end{align*}

(2)
\begin{align*}
    \int\limits_{-\infty}^{+\infty} f(x) d{x}&=F(+\infty)-F(-\infty)\\
    &=1-0\\
    &=1
\end{align*}

(3) $ f(x)\ge 0 $ (since $ F(x) $ is non-decreasing, it's derivative is non-negative)

(4) \[ F(x)=\int\limits_{-\infty}^{x} f(u) d{u} \]

\subsection{Example}
Suppose a continuous random variable $ X $ is on the range $ [0,1] $ has the 
cumulative distribution function $ F(x)=x^2 $.

\textsc{What is the probability density function?}

$ f(x)=\frac{d}{dx} F(x)=2x $.

\textsc{What is $ P(X=0.25) $?}

$ P(X=0.25)=0 $

\textsc{What is $ P(X\le 0.25) $?}

(1) $ P(X\le 0.25)=F(0.25)=(0.25)^2=0.0625 $

(2)
\[  P(X\le 0.25)=
\int\limits_{0}^{0.25} f(x) d{x} =\int\limits_{0}^{0.25} 2x d{x} =0.625 \]

\begin{defbox}
    Expectation:
    \[ E[X]=\int\limits_{-\infty}^{+\infty} xf(x) d{x} =
    \int\limits_{x\in\text{range}}^{} xf(x) d{x}  \]
\end{defbox}

\begin{defbox}
    Variance:
    \[ Var(X)=E[X^2]-E[X]^2=\int\limits_{-\infty}^{+\infty} x^2f(x) d{x}-
    \int\limits_{-\infty}^{+\infty} xf(x) d{x} \]
\end{defbox}

\begin{defbox}
    \subsubsection{Definition (Percentiles)}
    The $ p^\text{th} $ \emph{percentile} of a distribution $ x_p $ such that
    $ F(x_p)=p $.
\end{defbox}