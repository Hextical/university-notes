\makeheading{Lecture 31}
\section{Linear Combinations of Random Variables}
Suppose two variables $ X $ and $ Y $ have \textbf{non-zero} covariance.
What can we say?

(a) $ X $ and $ Y $ are independent.\ (b) $ X $ and $ Y $ are not independent.\
(c) we cannot tell if they are independent.

Same question, but for \textbf{zero} covariance.

\textbf{Today}
\begin{itemize}
    \item Linear Combinations of random variables
          (9.5 \& 9.6), which connects nicely to CLT
    \item a couple of examples
\end{itemize}

\textbf{Friday}
\begin{itemize}
    \item Indicator Variables
\end{itemize}

\textbf{Rules of Linear Combinations}

$ P=\alpha X+(1-\alpha)Y\rightarrow\text{two stocks} $

$ S=0.05A+0.3M+0.15Q+0.5F $

\textbf{Means}

1. $ E(aX+bY)=aE(X)+bE(Y) $

2. $ E\left(\sum\limits_{i=1}^{n} a_i X_i\right)=\sum\limits_{i=1}^{n}a_i E(X_i)=
    a_1E(X_1)+a_2E(X_2)+\cdots +a_n E(X_n) $

3. $ E\left(\sum\limits_{i=1}^{n} \frac{X_i}{n}\right)=\sum\limits_{i=1}^{n}\frac{\mu}{n}=\frac{n}{n} \mu=\mu $

$ X_i $'s all have mean $ \mu $

$ \bar{X} $ = $ \frac{1}{n} \sum\limits_{i=1}^{n} X_i\implies E(\bar{X})=\mu $

\textbf{Variances}

1. $ Var(aX+bY)=a^2Var(X)+b^2Var(Y)+2abCov(X,Y) $

2. $ Var(\bar{X})=Var\left(\sum\limits_{i=1}^{n} \frac{X_i}{n}\right)=\sum\limits_{i=1}^{n}\left( \frac{1}{n}  \right)^2
    Var(X_i)=\frac{n\sigma^2}{n^2}=\frac{\sigma^2}{n} $

where $ X_i $'s are independent. Thus, if we have independent RV's
$ X_1,\ldots,X_n $ all with $ \mu,\sigma^2 $, then $ E(\bar{X})=\mu $,
$ Var(\bar{X})=\frac{\sigma^2}{n} $

$\boxed{\sfrac{\sigma}{\sqrt{n}}\rightarrow\text{ std error of the mean}}$

\textbf{Note}:
\begin{align*}
    Cov(X,X)
     & =E[XX]-E[X]E[X]  \\
     & =E[X^2]-(E[X])^2 \\
     & =Var(X)
\end{align*}
$ \implies Corr(X,X)=1 $

\textbf{Covariances}

\[ Cov(aX+bY,cZ+dW)=acCov(X,Z)+adCov(X,W)+bcCov(Y,Z)+bdCov(Y,W) \]

\section{Linear Combinations of Normal Random Variables}
\textbf{Claim:}
If $ X_i \thicksim N(\mu_i,\sigma_i^2) $ for $ i=1,2,\ldots ,n $
are random variables, then
\[ \sum\limits_{i=1}^{n} a_i X_i \thicksim
    N\left(\sum\limits_{i=1}^{n}a_i\mu_i,\sum\limits_{i=1}^{n} a_i^2\sigma_i^2\right) \]
\[ \boxed{X_i \thicksim N(\mu,\sigma^2)\implies\bar{X} \thicksim N(\mu,\sigma^2/n)} \]

\textbf{Example}

Weight of a cat $ C \thicksim N(4.1,1.6^2) $, weight of a dog
$ D \thicksim (9.4,3.6^2) $. Find the probability that a cat weighs more than
a dog.

\textbf{Solution.}

$ P(C>D)\implies P(C-D>0)\rightarrow C-D \thicksim N(4.1-9.4,1.6^2+(-3.6)^2)$
\begin{align*}
     & =P\left(\frac{C-D-(-5.3)}{\sqrt{15.52}}>\frac{0-(-5.3)}{\sqrt{15.52}}\right) \\
     & =P(Z>1.35)                                                                   \\
     & =1-0.91149                                                                   \\
     & =0.08851
\end{align*}

\textbf{Example}

Heights of cats are $ N(24,1.5^2) $. Find probability a cat has height
within 1cm of average.

\textbf{Solution.}

\begin{align*}
    P(23<X<25) & =
    P\left( \frac{23-24}{1.5}<\frac{X-24}{1.5}<\frac{25-24}{1.5} \right) \\
               & =P(-0.67<Z<0.67)                                        \\
               & =2(0.74857)-1                                           \\
               & =0.49714
\end{align*}

\textbf{Example}

Find the probability the \textbf{average height} of 5 cats is within 1cm of average.

\textbf{Solution.}

$ \bar{X}=\sum\limits_{i=1}^{5} X_i \thicksim N(24,\sfrac{1.5^2}{5}) $

\begin{align*}
    P\left(\left|\bar{X}-24\right|<1\right) & =
    P(23<\bar{X}<25)                                                                          \\
                                            & =P\left( \frac{23-24}{\sfrac{1.5}{\sqrt{5}}}<Z<
    \frac{25-25}{\sfrac{1.5}{\sqrt{5}}} \right)                                               \\
                                            & =P(-1.49<Z<1.49)                                \\
                                            & =0.86378
\end{align*}
