\section{Lecture 16}
\subsection{Summary}
Today we finished up Chapter 5 with more about the Poisson process and the Poisson distribution, and started Chapter 7.

If we define $X$ to be the number of events observed in a time period of length $t$ in a Poisson process with rate $\lambda$, then I claimed that $X$ has a Poisson distribution with parameter $\mu = \lambda t$. The proof in the course notes uses all three conditions for the Poisson process, as well as some calculus, and I encourage you to check it out if you enjoy neat proofs.

Many different processes can be modelled with a Poisson process, and all we have to do is identify the values of $\lambda$ and $t$ (making sure they are measured in the same time units) and define $\mu$ to be the product. We looked at some ways to tell when to use the Poisson distribution.

Then we looked at a detailed example of combining other models with the Poisson distribution. These types of problems are often found on tests/exams, and there are lots of other examples in the course notes to practice with. See if you can get the answer to part (d) of the question in class, and I'll post the solution as a follow-up below.

We won't be spending any class time on R (Chapter 6), but I'll post some resources on Learn if you want to find out more about this neat statistical programming language.

Finally we started Chapter 7. We looked at several ways of summarizing data: a frequency table, a frequency histogram, as well as three single numbers: the sample mean (average), the median (middle value), and the mode (most common value). These quantities are not necessarily equal to each other, but by chance they were all 2 when we used some information taken from the class on the number of kids in their family!

I've been running out of time for our SWAGs lately but I'll post 5 and 6 on Learn (about Liar's Dice and Pokemon, respectively.)

Consider a Poisson Process with rate $ \lambda $, (i.e. $ \lambda $ events
occur on average per unit time). Observe the process for $ t $ units of time.
Let $ X= $ \# of events that occur. Then, $ X \thicksim \poi(\mu) $, where
$ \mu=\lambda t $. That is,
\[ f(x)=\frac{e^{-\mu}\mu^x}{x!} \]

\subsection{Example}
Request coming in from a web server at a rate of $ 100 $ requests per minute.
$ \lambda = 100, t=\frac{1}{60} $
The \# of requests per second would be
\[ \poi\left(\mu=\frac{100}{60}=\frac{5}{3}\right) \]

\subsection{Combining Other Models with the Poisson Process (5.9)}
Problems may involve many different random variables!

\subsection{Example (Continued)}
We say that a second is quiet if it has no requests.
\begin{enumerate}[(a)]
    \item Find probability that a second is quiet
    \item In a minute ($60$ non-overlapping seconds), find the probability of $10$ quiet seconds
    \item Find the probability of having to wait $ 30 $ non-overlapping seconds to get $ 2 $ quiet seconds
    \item Given (c), find the probability of $ 1 $ quiet second in the first $ 15 $ seconds
\end{enumerate}
(a) Let $ X= $ \# requests in a second. $ X \thicksim \poi(\nicefrac{5}{3}) $.

We want $ P(X=0)=\frac{e^{-\frac{5}{3}}\left(\frac{5}{3}\right)^0}{0!}=0.189 $


(b) Let $ Y= $ \# quiet seconds out of 60. $ Y \thicksim \bin(60,0.189) $.

We want $ P(Y=10)=\binom{60}{10}(0.189)^{10}(0.811)^{50}=0.124 $

(c) Let $ Z= $ \# non-quiet seconds before getting $ 2 $ quiet seconds. 
$ Z \thicksim \nb(2,0.189) $.

We want $ P(Z=28)=\binom{29}{1}(0.189)^2(0.811)^{28}=0.003 $

(d) $ D_x=$ \# of quiet seconds out of 15. $ D_x \thicksim \bin(15,0.189) $.
\[ P(D_x=1)=\binom{15}{1}(0.189)^1(0.811)^{14} \]
We get,

$ P(\text{1 quiet second in the first 15 seconds}\mid\text{wait 30 to get 2 quiet})= $
\begin{align}
    &=\frac{P(\text{1 quiet second in the first 15 seconds}\text{ AND wait 30 to get 2 quiet})}{P(\text{wait 30 to get 2 quiet})}\\
    &=\frac{P(\text{1 quiet second in the first 15 seconds} \text{ AND wait an additional 15 to get 1 additional quiet})}{P(C)} \\
    &=\frac{P(\text{1 quiet second in the first 15 seconds})P(\text{wait an additional 15 to get 1 additional quiet}) }{P(C)}\\
    &=\frac{\binom{15}{1}(0.189)^1(0.811)^{14}\times (0.811)^{14}(0.189)}
    {\binom{29}{28}(0.189)^2(0.811)^{28}}\\
    &=\frac{\binom{15}{1}}{\binom{29}{28}}\\
    &=\frac{15}{29} 
\end{align}
In (3) we used the independence of non-overlapping time intervals and constant
probability of events.

\subsection{Summarizing Data on Random Variables (7.1)}
Let $ X= $ \# of kids in a family.


\begin{center}
    \begin{tabular}{| *{2}{>{\centering\arraybackslash}p{3cm} |}}
        \hline
        Value & Frequency \\ \hline
        1 & 3\\
        2 & 10\\
        3 & 1\\
        4 & 1 \\ \hline
    \end{tabular}
\end{center}

\begin{defbox}
    \subsubsection{Definition (Median)}
    The \emph{median} of a sample is a value such that half the results are 
    below it and half above it, when the results are arranged in numerical 
    order.
\end{defbox}

\begin{defbox}
    \subsubsection{Definition (Mode)}
    The \emph{mode} of the sample is the value which occurs most often. There
    is no guarantee there will be only a single mode.
\end{defbox}
Mean: average $ \rightarrow
\frac{1\times 3+2\times 10+3\times 1+4\times 1}{15} $

Median: $ 2 $

Mode: $ 2 $