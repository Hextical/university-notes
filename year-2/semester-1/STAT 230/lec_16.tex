\section{Lecture 16}
\myuline{Example}

Request coming in from a web server at a rate of $ 100 $ requests per minute.
$ \lambda = 100, t=\frac{1}{60} $
The \# of requests per second would be
\[ \poi\left(\mu=\frac{100}{60}=\frac{5}{3}\right) \]

\myuline{5.9 Combining Other Models with the Poisson Process}

Problems may involve many different random variables!

\myuline{Example (Continued)}

We say that a second is quiet if it has no requests.
\begin{enumerate}[(a)]
    \item Find probability that a second is quiet
    \item In a minute ($60$ non-overlapping seconds), find the probability of $10$ quiet seconds
    \item Find the probability of having to wait $ 30 $ non-overlapping seconds to get $ 2 $ quiet seconds
    \item Given (c), find the probability of $ 1 $ quiet second in the first $ 15 $ seconds
\end{enumerate}
(a) Let $ X= $ \# requests in a second. $ X \thicksim \poi(\nicefrac{5}{3}) $.

We want $ P(X=0)=\frac{e^{-\frac{5}{3}}\left(\frac{5}{3}\right)^0}{0!}=0.189 $


(b) Let $ Y= $ \# quiet seconds out of 60. $ Y \thicksim \bin(60,0.189) $.

We want $ P(Y=10)=\binom{60}{10}(0.189)^{10}(0.811)^{50}=0.124 $

(c) Let $ Z= $ \# non-quiet seconds before getting $ 2 $ quiet seconds. 
$ Z \thicksim \nb(2,0.189) $.

We want $ P(Z=28)=\binom{29}{1}(0.189)^2(0.811)^{28}=0.003 $

(d) $ D_x=$ \# of quiet seconds out of 15. $ D_x \thicksim \bin(15,0.189) $.
\[ P(D_x=1)=\binom{15}{1}(0.189)^1(0.811)^{14} \]
We get,

$ P(\text{1 quiet second in the first 15 seconds}\mid\text{wait 30 to get 2 quiet})= $
\begin{align}
    &=\frac{P(\text{1 quiet second in the first 15 seconds}\text{ AND wait 30 to get 2 quiet})}{P(\text{wait 30 to get 2 quiet})}\\
    &=\frac{P(\text{1 quiet second in the first 15 seconds} \text{ AND wait an additional 15 to get 1 additional quiet})}{P(C)} \\
    &=\frac{P(\text{1 quiet second in the first 15 seconds})P(\text{wait an additional 15 to get 1 additional quiet}) }{P(C)}\\
    &=\frac{\binom{15}{1}(0.189)^1(0.811)^{14}\times (0.811)^{14}(0.189)}
    {\binom{29}{28}(0.189)^2(0.811)^{28}}\\
    &=\frac{\binom{15}{1}}{\binom{29}{28}}\\
    &=\frac{15}{29} 
\end{align}
In (3) we used the independence of non-overlapping time intervals and constant
probability of events.

\begin{center}
    \scalebox{0.68}{
        \begin{tabular}{ L{2.5cm} | L{3cm} | L{3cm} | L{3cm} | L{3.5cm} | L{3cm} | L{3cm} }
                                                                    & Discrete Uniform & Hypergeometric & Binomial & Negative Binomial & Geometric & Poisson \\
            \hline
            function \newline range \newline parameters             &
            $ \du[a,b] $\newline $a,a+1,\ldots,b$                   &
            $ \hyp(N,r,n) $\newline bad                             &
            $ \bin(n,p) $\newline $ 0,1,\ldots,n $                  &
            $ \nb(k,p) $\newline $ 0,1,\ldots $                     &
            $ \geo(p) $\newline $ 0,1,\ldots $                      &
            $ \poi(\mu) $\newline $ 0,1,\ldots $\newline
            $ \mu=np $, $\mu=\lambda t $                                                                                                                     \\
            \hline
            pf, $ f(x) $                                            &
            $ \frac{1}{b-a+1} $                                     &
            $ \frac{\binom{r}{x}{\binom{N-r}{n-x}}}{\binom{N}{n}} $ &
            $ \binom{n}{x}p^x(1-p)^{n-x} $                          &
            $ \binom{x+k-1}{k-1}p^k(1-p)^x $                        &
            $ p(1-p)^x $                                            &
            $ \frac{e^{-\mu}\mu^x}{x!} $                                                                                                                     \\
            \hline
            cumulative distribution function,     $ F(x) $          &
            $ \frac{x-a+1}{b-a+1} $                                 &                  &                &          &
            $ 1-(1-p)^{x+1} $                                       &
            $ e^{\mu}[1+\frac{\mu^1}{1!}+\cdots\frac{\mu^x}{x!}] $                                                                                           \\
            \hline
            how to tell                                             &
            ``equally likely''
            \newline know min. \& max.                              &

            know total \# objects
            \newline know \# S's
            \newline know \# trials
            \newline without replacement
            \newline selecting a subset
                                                                    &

            Bernoulli trials
            \newline know \# trials
            \newline count \# S's
                                                                    &

            Bernoulli trials
            \newline ``until''
            \newline ``it take... to get''
            \newline ``before''
            \newline know how many S's we want
                                                                    &

            ``until we get''
            \newline ``before the first''
                                                                    &

            Bin. with large amount of trials, small probability
            \newline rate specified (\#events/time)
            \newline no pre-specified max.
            \newline events occurring at any time (randomly)
            \newline Poisson process \& know time \& count events
            \newline doesn't make sense to ask how often
            an event did \textbf{not} occur
        \end{tabular}}
\end{center}

\textbf{\myuline{Chapter 7: Expected Value and Variance}}

\myuline{Summarizing Data on Random Variables (7.1)}

Let $ X= $ \# of kids in a family.

\begin{center}
    \begin{tabular}{| *{2}{>{\centering\arraybackslash}p{3cm} |}}
        \hline
        Value & Frequency \\ \hline
        1 & 3\\
        2 & 10\\
        3 & 1\\
        4 & 1 \\ \hline
    \end{tabular}
\end{center}

\begin{defbox}
    \subsection{Definition (Median)}
    The \emph{median} of a sample is a value such that half the results are 
    below it and half above it, when the results are arranged in numerical 
    order.
\end{defbox}

\begin{defbox}
    \subsection{Definition (Mode)}
    The \emph{mode} of the sample is the value which occurs most often. There
    is no guarantee there will be only a single mode.
\end{defbox}
Mean: average $ \rightarrow
\frac{1\times 3+2\times 10+3\times 1+4\times 1}{15}=2 $

Median: $ 2 $

Mode: $ 2 $