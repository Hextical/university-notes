\section{Lecture 27}

\subsection{Thought Question}
For a full-time UW Math Faculty student, let $ X=$ number of courses taking
and $ Y=1 $ if in co-op, or $ 0 $ if in regular. The joint pf is given by
(this is real data)

\begin{tabular}{| *{6}{>{\centering\arraybackslash}p{2cm} |}}
    \hline
    $y\backslash x$ & 3    & 4    & 5    & 6    & $ f_Y(y) $ \\
    \hline
    $0$             & 0.09 & 0.17 & 0.22 & 0.01              \\
    \hline
    $1$             & 0.05 & 0.10  & 0.32 & 0.04 & 0.51       \\
    \hline
    $ f_X(x) $      &      &      & 0.54 &      & 1          \\
    \hline
\end{tabular}

Are $ X $ and $ Y $ independent?

a) Yes, b) No, c) Not enough information

Correct answer is b):
No. $ f(5,1)=0.32\neq f_X(5)f_Y(1)=(0.54)(0.51)=0.2754 $

\subsection{Example}
Imagine you have a card game with a total of 12 cards. Classified in three
different categories: 5 cards (money), 4 cards (action), 3 cards (useless).
Draw a hand of them, in this case 3 without replacement, and let
$ X= $ \# of useless, $ Y= $ \# action.

\textsc{Find the joint pf and range}

\begin{tabular}{| *{6}{>{\centering\arraybackslash}p{2cm} |}}
    \hline
    $y\backslash x$ & 0                    & 1                     & 2                    & 3                   & $ f_Y(y) $            \\
    \hline
    $0$             & $\nicefrac{10}{220}$ & $\nicefrac{30}{220}$  & $\nicefrac{15}{220}$ & $\nicefrac{1}{220}$ & $\nicefrac{56}{220}$  \\
    \hline
    $1$             & $\nicefrac{40}{220}$ & $\nicefrac{60}{220}$  & $\nicefrac{12}{220}$ & 0                   & $\nicefrac{112}{220}$ \\
    \hline
    $2$             & $\nicefrac{30}{220}$ & $\nicefrac{18}{220}$  & 0                    & 0                   & $\nicefrac{48}{220}$  \\
    \hline
    $3$             & $\nicefrac{4}{220}$  & 0                     & 0                    & 0                   & $\nicefrac{4}{220}$   \\
    \hline
    $ f_X(x) $      & $\nicefrac{84}{220}$ & $\nicefrac{108}{220}$ & $\nicefrac{27}{220}$ & $\nicefrac{1}{220}$ & 1                     \\
    \hline
\end{tabular}

Range: $ x\in \{0,1,2,3\} $, $ y\in \{0,1,2,3\} $ such that $ x+y\le 3 $

$ f(0,0) $ (no useless, no action)=$ P $(all money)
\[ \frac{\binom{5}{3}}{\binom{12}{3}}=\frac{10}{220} \]
$ f(1,1) $ (1 useless, 1 action)=$ P $(one of each type)
\[ \frac{\binom{3}{1}\binom{4}{1}\binom{5}{1}}{\binom{12}{3}} =\frac{60}{220}  \]
\[ f(x,y)=\frac{\binom{3}{x}\binom{4}{y}\binom{5}{3-x-y}}{\binom{12}{3}}  \]

Find marginal pfs (sum), $ X \thicksim \hyp(12,3,3) $. $ Y \thicksim \hyp(12,4,3) $. Check that the marginal pfs match.

Are they independent? No (don't have a cartesian product)

Recall: conditional probability:
\[ P(A\mid B)=\frac{P(AB)}{P(B)} \]

\begin{defbox}
    \subsection{Definition (Conditional Probability Function)}
    The \emph{conditional probability function} of $ X $ given $ Y=y $ is
    \[ f(x\mid y)=P(X=x\mid Y=y)=\frac{P(X=x,Y=y)}{P(Y=y)}=\frac{f(x,y)}{f_Y(y)} \]
    provided $ f_Y(y)>0 $.

    Similarly, the \emph{conditional probability function} of $ Y $ given $ X=x $ is
    \[ f(y\mid x)=\frac{f(x,y)}{f_X(x)} \]
    provided $ f_X(x)>0 $.
\end{defbox}

\subsection{Example}
What is the probability that someone taking 4 courses is a co-op student?

In other words, $ P(Y=1\mid X=4)=\frac{0.1}{0.27}=0.37$

For $ 6 $ courses, $ \frac{0.04}{0.05}=0.80 $.

\subsection{Example}
If you have $ 1 $ action card, find the pf of the number of useless cards.

i.e. the pf of $ X\mid Y=1 $
\begin{tabular}{| *{4}{>{\centering\arraybackslash}p{2cm} |}}
    \hline
    $x$          & 0                     & 1                      & 2                      \\
    \hline
    $f(x\mid 1)$ & $ \nicefrac{40}{112}$ & $ \nicefrac{60}{112} $ & $ \nicefrac{12}{112} $ \\
    \hline
\end{tabular}

\subsection{Functions of 2 or more random variables}
Suppose $ U $ is some function of $ X $ and $ Y $, e.g. $ U=X-Y $. To find the
pf of $ U $.

\begin{tabular}{| *{5}{>{\centering\arraybackslash}p{2cm} |}}
    \hline
    $y\backslash x$ & 3 & 4 & 5 & 6 \\
    \hline
    $0$             & 3 & 4 & 5 & 6 \\
    \hline
    $1$             & 2 & 3 & 4 & 5 \\
    \hline
\end{tabular}

1. determine the possible values of $ U $ for each pair $ (x,y) $

so the range is $ u\in \{2,3,4,5,6\} $

2. $ f(u) $ is the sum of $ f(x,y) $ for all combos that map to $ u $.
\[ f(u)=\sum\limits_{(x,y)\text{ s.t. }x-y=u} f(x,y)\]

\begin{tabular}{| *{6}{>{\centering\arraybackslash}p{1cm} |}}
    \hline
    $u$    & 2    & 3    & 4    & 5    & 6    \\
    \hline
    $f(u)$ & 0.05 & 0.19 & 0.49 & 0.24 & 0.01 \\
    \hline
\end{tabular}

$ u=2\rightarrow (x=3,y=1)=0.05$

$ u=3\rightarrow (x=4,y=1)+(x=3,y=0)=0.1+0.09=0.19$

Using the earlier table:
\begin{tabular}{| *{5}{>{\centering\arraybackslash}p{2cm} |}}
    \hline
    $y\backslash x$ & 3    & 4    & 5    & 6    \\
    \hline
    $0$             & 0.09 & 0.17 & 0.22 & 0.07 \\
    \hline
    $1$             & 0.05 & 0.1  & 0.32 & 0.04 \\
    \hline
\end{tabular}