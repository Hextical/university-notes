\section{2019-09-26}
\subsection{Summary of outcomes}
(LP)
\[\max \{\mathbf{c}^\top \mathbf{x} \mid A\mathbf{x}=\mathbf{b}\text{, }
\mathbf{x}\ge\mathbf{0}\}\]
\begin{itemize}
    \item if $\exists \mathbf{y}\in\mathbb{R}^m$ such that
    \begin{enumerate}
        \item $\mathbf{y}^\top A\ge\mathbf{0}^\top $
        \item $\mathbf{y}^\top \mathbf{b}<0$
    \end{enumerate}
    then (LP) is infeasible.
    \item if $\exists(\bar{\mathbf{x}}, \mathbf{d})\in\mathbb{R}^n$ such that $A\mathbf{x}=b$, $\mathbf{x}\ge \mathbf{0}$, 
    \begin{enumerate}
        \item $A\mathbf{d}=\mathbf{0}$
        \item $\mathbf{d}\ge \mathbf{0}$
        \item $\mathbf{c}^\top \mathbf{d}>0$
    \end{enumerate}
    then (LP) is unbounded.
    \item if $\exists(\mathbf{\bar{x}},\mathbf{y})$ such that
    \begin{enumerate}
        \item $A\mathbf{\bar{x}}=\mathbf{b}$, $\mathbf{x}\ge 0$
        \item $A^\top \mathbf{y}\ge\mathbf{c}$
        \item $\mathbf{c}^\top  \mathbf{\bar{x}}=\mathbf{y}^\top \mathbf{b}$
    \end{enumerate}
    then $\mathbf{\bar{x}}$ is an optimal solution of (LP).
\end{itemize}

\subsection{Standard Equality Form (SEF)}
Every LP can be converted to SEF. A pair of LP problems LP1 and LP2 are equivalent if they both have the
same status (infeasible, unbounded, optimal) and certificate of such a status for one problem can easily
be converted into a certificate of the same type for the other LP.
Given an arbitrary LP problem, 
\begin{itemize}
    \item if the objective function is a minimization problem, then 
    $\min \mathbf{c}^\top \mathbf{x}\rightarrow -(\max -\mathbf{c}^\top \mathbf{x})$
        \begin{remark}
        We often omit one negative sign
        from a TA on Piazza: "It's more just a convention of not putting - 
        before max when doing this and it's understood that the 
        optimal value of one is the negative of the optimal value of the other"
        \end{remark} 
    \item if there are constraints $\mathbf{\alpha}\mathbf{x}\le \mathbf{\alpha}$, introduce a new
    non-negative \emph{slack variable} $x_{n+1}$, $x_{n+1}\ge 0$.
    \item if some $x_j$ has no constraint on it, such variables are called \emph{free variables} and
    we represent that free variable as a difference of two non-negative variables,
    $x_j=x_j^+-x_j^-$, $x_j^+\ge 0$, $x_j^-\ge 0$.
    \item if some $x_j<0$ flip all signs correlating to $x_j$
\end{itemize}

\subsection{Example (Converting an LP to SEF)}
(P)
\[\max 100x_1+200x_2\]
subject to
\[
\begin{array}{ccccc}
    11x_1 & + & 12x_2 & \le & 2000\\
    21x_1 & + & 22x_2 & \ge & 1000
\end{array}
\]
\[x_1 \ge 0\]
Converting into SEF we get (P$^\prime$):
\[\max 100x_1+200(x_2^+-x_2^-)\]
subject to
\[
\begin{array}{ccccccccc}
    11x_1 & + & 12(x_2^+-x_2^-) & + & x_3 & & & = & 2000\\
    21x_1 & + & 22(x_2^+-x_2^-) & - & & x_4 & & = & 1000
\end{array}
\]
\[x_1,x_2^+,x_2^-,x_3,x_4\ge 0\]
(P) and (P$^\prime$) are equivalent.

Let
$(\bar{x_1}, \bar{x_2^+}, \bar{x_2^-}, \bar{x_3}, \bar{x_4})^\top$
be a feasible solution of (P$^\prime$).


If
\begin{align*}
    \hat{x_1}:=\bar{x_1}\\
    \hat{x_2}:=\bar{x_2^+}-\bar{x_2^-}
\end{align*}
Then 
$(\hat{x_1},\hat{x_2})^\top$
is a feasible solution of (P).

Let
$(\bar{x_1}, \bar{x_2})^\top$
be a feasible solution of (P).

If
\begin{align*}
    \bar{x_3}:=2000-11\bar{x_1}-12\bar{x_2}\\
    \bar{x_4}:=21\bar{x_1}+22\bar{x_2}-1000
\end{align*}
and 
if $\bar{x_2}\ge 0$
\[\bar{x_2^+}:=\bar{x_2}\]
\[\bar{x_2^-}:=0\]
or $\bar{x_2}< 0$
\[\bar{x_2^+}:=0\]
\[\bar{x_2^-}:=-\bar{x_2}\]
then $(\bar{x_1},\bar{x_2^+},\bar{x_2^-},\bar{x_3},\bar{x_4})^\top$
is a feasible solution of (P$^\prime$).
\begin{remark}
    Example 7.2 was a question from a past midterm and not covered in class. The class example was
    not helpful.
\end{remark}

\subsection{Definition (Basis)}
Let $A\in M_{m\times n}(\mathbb{R})$, $B\subseteq\{1,\dots,n\}$ such that $|B|=m$. If
\[A_B:=
\begin{bmatrix}
    a_i \mid i\in B
\end{bmatrix}\in M_{m\times m}(\mathbb{R})
\]
where $A_B$ is non-singular (i.e. IMT holds), then $B$ is a basis of $A$.


If $B$ is a basis of $A$, then 
$\begin{bmatrix}
    a_i \mid i\in B
\end{bmatrix}$
is a basis for $R^m$.

\subsection{Definition (Basic solution)}
Let $\mathbf{b}\in\mathbb{R}^m$ be given. $\mathbf{\bar{x}}$ is a \emph{basic solution} 
corresponding to $B$ if
\begin{enumerate}
    \item $B$ is a basis
    \item $x_i=0 \qquad \forall i\notin B$
\end{enumerate}

\subsection{Definition (Basic feasible solution)}
In addition to a \emph{basic solution}, we need  $\mathbf{\bar{x}}\ge \mathbf{0}$.

\subsection{Example (Bases of A)}
\[A:=
\begin{bmatrix}
    1 & 0 & 2 & -1 & 1\\
    0 & 1 & 2 & -1 & 2
\end{bmatrix}
\text{, }
\mathbf{b}:=
\begin{bmatrix}
    2\\
    5
\end{bmatrix}
\]
Bases of A: $\{1,2\}$, $\{2,3\}$, $\{1,4\}$.\\
Not a bases of A: $\emptyset$, $\{1\}$, $\{1,2,3\}$, $\{3,4\}$.\\
To find the basic solution determined by $B:=\{1,4\}$, solve
\[
\begin{bmatrix}
    1 & -1\\
    0 & -1
\end{bmatrix}
\begin{bmatrix}
    x_1\\
    x_4
\end{bmatrix}
=
\begin{bmatrix}
    2\\
    5
\end{bmatrix}
\]
and we get $\mathbf{\bar{x}}=(-3,0,0,-5,0)^\top $.