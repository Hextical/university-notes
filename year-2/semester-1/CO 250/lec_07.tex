\section{2019-09-26}
\subsection{Summary of outcomes}
(P)
\[\max \{\bm{c}^\top \bm{x} : A\bm{x}=\bm{b},\,
\bm{x}\ge\bm{0}\}\]
\begin{itemize}
    \item If there exists a vector $ \bm{y} $ such that
    \begin{enumerate}[(1)]
        \item $\bm{y}^\top A\ge\bm{0}^\top $
        \item $\bm{y}^\top \bm{b}<0$
    \end{enumerate}
    then (P) is infeasible. We call $ \bm{y} $ a certificate of infeasibility.

    \item If there exists a feasible solution $\bar{\bm{x}}$ and a vector $ \bm{d} $ such that:
    \begin{enumerate}[(1)]
        \item $A\bm{d}=\bm{0}$
        \item $\bm{d}\ge \bm{0}$
        \item $\bm{c}^\top \bm{d}>0$
    \end{enumerate}
    then (P) is unbounded. We call a pair of vectors $ \bar{\bm{x}}, \bm{d} $
    a certificate of unboundedness.

    \item If there exists a feasible solution $ \bm{\bar{x}} $ and a vector $ \bm{y} $ such that:
    \begin{enumerate}[(1)]
        \item $A^\top \bm{y}\ge\bm{c}$
        \item $\bm{c}^\top \bm{\bar{x}}=\bm{y}^\top \bm{b}$
    \end{enumerate}
    then $\bm{\bar{x}}$ is an optimal solution of (P). We call $ \bm{y} $
    a certificate of optimality.
\end{itemize}

\begin{defbox}
    \subsection{Definition (Standard Equality Form)}
    An LP is said to be in \emph{Standard Equality Form} (SEF) if it has the Form
    \[ \max \{\bm{c}^T \bm{x}+\bar{z} : A \bm{x}=\bm{b},\,\bm{x}\ge \bm{0}\}\]
    where $ \bar{z} $ is a constant.
    In other words, it satisfies all of the conditions:
    \begin{enumerate}[(1)]
        \item It is a maximization problem
        \item All constraints are equations (other than non-negativity
        constraints)
        \item Every variable has a non-negativity constraint
    \end{enumerate}
\end{defbox}
Every LP can be converted to SEF. A pair of LP problems LP1 and LP2 are equivalent if they both have the
same status (infeasible, unbounded, or optimal) and certificate of such a status for one problem can easily
be converted into a certificate of the same type for the other LP.

Given an arbitrary LP problem, 
\begin{itemize}
    \item if the objective function is a minimization problem, then 
    $\min \bm{c}^\top \bm{x}\rightarrow -(\max -\bm{c}^\top \bm{x})$
        \begin{remark}
        We often omit one negative sign
        from a TA on Piazza: "It's more just a convention of not putting $ - $
        before max when doing this and it's understood that the 
        optimal value of one is the negative of the optimal value of the other"
        \end{remark} 
    \item if there are constraints $\alpha\bm{x}\le \alpha$, introduce a new
    non-negative \emph{slack variable} $x_{n+1}$, $x_{n+1}\ge 0$.
    \item if some $x_j$ has no constraint on it, such variables are called \emph{free variables} and
    we represent that free variable as a difference of two non-negative variables,
    $x_j=x_j^+-x_j^-$, $x_j^+\ge 0$, $x_j^-\ge 0$.
    \item if some $x_j<0$ flip all signs correlating to $x_j$
\end{itemize}

\subsection{Example (Converting an LP to SEF)}
(P)
\[\max 100x_1+200x_2\]
subject to
\[
\begin{array}{ccccc}
    x_1 & + & 2x_2 & \le & 20\\
    3x_1 & + & 4x_2 & \ge & 10
\end{array}
\]
\[x_1 \ge 0\]
Converting into SEF we get (P$^\prime$):
\[\max 100x_1+200(x_2^+-x_2^-)\]
subject to
\[
\begin{array}{ccccccccc}
    x_1 & + & 2(x_2^+-x_2^-) & + & x_3 & & & = & 20\\
    3x_1 & + & 4(x_2^+-x_2^-) & - & & x_4 & & = & 10
\end{array}
\]
\[x_1,x_2^+,x_2^-,x_3,x_4\ge 0\]
(P) and (P$^\prime$) are equivalent.

Let
$(\bar{x_1}, \bar{x_2^+}, \bar{x_2}^-, \bar{x_3}, \bar{x_4})^\top$
be a feasible solution of (P$^\prime$).


If
\begin{align*}
    &\hat{x_1}:=\bar{x_1}\\
    &\hat{x_2}:=\bar{x_2}^+-\bar{x_2}^-
\end{align*}
Then 
$(\hat{x_1},\hat{x_2})^\top$
is a feasible solution of (P).

Let
$(\bar{x_1}, \bar{x_2})^\top$
be a feasible solution of (P).

If
\begin{align*}
    \bar{x_3}:=20-\bar{x_1}-2\bar{x_2}\\
    \bar{x_4}:=3\bar{x_1}+4\bar{x_2}-10
\end{align*}
and 
if $\bar{x_2}\ge 0$
\[\bar{x_2}^+:=\bar{x_2}\]
\[\bar{x_2}^-:=0\]
or $\bar{x_2}< 0$
\[\bar{x_2}^+:=0\]
\[\bar{x_2}^-:=-\bar{x_2}\]
then $(\bar{x_1},\bar{x_2}^+,\bar{x_2}^-,\bar{x_3},\bar{x_4})^\top$
is a feasible solution of (P$^\prime$).
\begin{remark}
    This example was very similar to a past midterm question and was not covered in class.
    The class example was useless.
\end{remark}

\begin{defbox}
    \subsection{Definition (Basis)}
    Let $A\in \mathbb{R}^{m\times n}$, $B\subseteq\{1,\dots,n\}$ such that $|B|=m$. If
    \[A_B:= \left[\begin{array}{c|c} a_i & i\in B \end{array}\right]
        \in \mathbb{R}^{m\times m}
    \]
    where $A_B$ is non-singular (i.e. IMT holds), then $B$ is a basis of $A$.
    If $B$ is a basis of $A$, then $A_B$ is a basis for $\mathbb{R}^m$. We
    denote $N$ as the set that does not have the elements of $B$.
\end{defbox}

\begin{defbox}
    \subsection{Definition (Basic Solution)}
    A vector $\bm{\bar{x}}$ is a \emph{basic solution} of $ A \bm{x}=\bm{b} $
    for a basis $ B $ if the following conditions hold:
    \begin{enumerate}[(1)]
        \item $A \bm{\bar{x}}=\bm{b}$
        \item $\bm{\bar{x}_N}=0$
    \end{enumerate}
\end{defbox}

\begin{defbox}
    \subsection{Definition (Basic Feasible Solution)}
    If $\bm{\bar{x}}$ is a \emph{basic solution} and $\bm{\bar{x}}\ge \bm{0}$,
    then $\bm{\bar{x}}$ is a \emph{basic feasible solution} .
\end{defbox}

\subsection{Example (Bases of A)}
\[A:=
\begin{bmatrix}
    1 & 0 & 2 & -1 & 1\\
    0 & 1 & 2 & -1 & 2
\end{bmatrix}
\text{, }
\bm{b}:=
\begin{bmatrix}
    2\\
    5
\end{bmatrix}
\]
Bases of A: $\{1,2\}$, $\{2,3\}$, $\{1,4\}$.\\
Not a bases of A: $\emptyset$, $\{1\}$, $\{1,2,3\}$, $\{3,4\}$.\\
To find the basic solution determined by $B:=\{1,4\}$, solve
\[
\begin{bmatrix}
    1 & -1\\
    0 & -1
\end{bmatrix}
\begin{bmatrix}
    x_1\\
    x_4
\end{bmatrix}
=
\begin{bmatrix}
    2\\
    5
\end{bmatrix}
\]
and we get $\bm{\bar{x}}=(-3,0,0,-5,0)^\top $.