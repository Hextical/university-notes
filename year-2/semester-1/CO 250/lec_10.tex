\section{2019-10-08}
\subsection{Convergence of Simplex Algorithm}
In each iteration, we choose $k\in N$ such that $c_k>0$. Then, we compute 
$t=\min\{\nicefrac{b_i}{A_{ik}}\mid A_{ik}>0\}$. Then, throughout the
rest of the Simplex iterations, we mever see the same basis again. 
There are at most $\binom{n}{m}$ bases of A. Therefore, if $t>0$ in each iteration,
the Simplex Algorithm will terminate in at most $\binom{n}{m}$ iterations. The only
way the algorithm will not terminate is when $t=0$ for all iterations
(after some \# of iterations). If our choises for $k$ and $l$ are deterministic and
consistent in this case if we repeat a basis we call it a \emph{cycle}.

\subsection{Theorem}
The Simplex Algorithm starting from a basic feasible solution and using the
smallest subscript rule terminates in at most $\binom{n}{m}$ iterations.

\subsection{Implementation of the Simplex Algorithm in "Big Data"}
In a given iteration of the Simplex Algorithm, what information do we need to
execute the algorithm?

We have the original data ($A$, $\mathbf{b}$, $\mathbf{c}$) and we have the
current $B$, $\mathbf{\bar{x}}$, $\mathbf{\bar{v}}$.

Pick any $k\in N$ such that $\bar{c_k}\ge 0$. $\bar{c_k}=c_k-\mathbf{\bar{y}}^T A\mathbf{x}$ 
(where $\mathbf{\bar{y}}^T=c_B A_B^{-1}$).

Then to compute $t$, we need
$t=\min\{\nicefrac{b_i}{A_{ik}}\mid A_{ik}>0\}$.

So, we need
$\bar{A_k}$: $\bar{A_k}=A_B^{-1}A_k$ and note that $\mathbf{\bar{x_N}}=\mathbf{0}$,
$\mathbf{\bar{x_B}}=\mathbf{\bar{b}}$


We solve linear systems $A_B^T\mathbf{y}=\mathbf{c_B}$ and $A_B\mathbf{d_B}=A_k$.

In implementations, we typically express $A_B$ or $A_B^{-1}$ as a product of elementary
matrices.

In practice, good implementations of the Simplex Algorithm terminates after $2m$ to
$\nicefrac{n}{2}$ iterations. Each iteration is very fast.

It is an open problem whether there exists a variant of Simplex Algorithm which is
guaranteed to terminate in at most $pn^q$ iterations for LP problems in SEF with $n$
variables, where $p$, $q$ are constants.
