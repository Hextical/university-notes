\makeheading{2019-11-21}
\begin{defbox}
    \begin{definition}
        Given a set $ S\subseteq \mathbb{R}^n $, the \emph{convex hull} of $ S $
        is the smallest convex set which contains $ S $.
        Equivalently,
        \[ \text{conv}(S)=\bigcap C \]
        where $ C\subseteq S $ is convex.
    \end{definition}
\end{defbox}

Let $ S_1\supseteq S_2 $, $ c\in\mathbb{R}^n $ and consider

\[ (P_2)\; \max \{\bm{c}^\top\bm{x}:\bm{x}\in S_2\} \]
\[ (P_1)\; \max \{\bm{c}^\top\bm{x}:\bm{x}\in S_1\} \]
If we have an optimal solution $ \bm{\bar{x}} $ of $ (P_1) $ and $ \bm{\bar{x}}\in S_2 $,
then $ \bm{\bar{x}} $ is an optimal solution of $ (P_2) $. Regardless of whether
$ \bm{\bar{x}}\in S_2 $, $ \bm{c}^\top \bm{\bar{x}} $ is an upper bound on the optimal
value of $ (P_2) $.

\[ (IP)\; \max \bm{c}^\top \bm{x} \]
subject to
\[
    \begin{array}{ccc}
        A\bm{x} & =         & \bm{b}       \\
        \bm{x}  & \geqslant & \bm{0}       \\
        \bm{x}  & \in       & \mathbb{Z}^n
    \end{array} \]

\[ (LP)\; \max \bm{c}^\top \bm{x} \]
subject to
\[
    \begin{array}{ccc}
        A\bm{x} & =         & \bm{b} \\
        \bm{x}  & \geqslant & \bm{0}
    \end{array} \]
Find an optimal solution $ \bm{\bar{x}} $ of $ (LP) $. If $ \bm{\bar{x}}\in\mathbb{Z}^n $,
then $ \bm{\bar{x}} $ is optimal in $ (IP) $. Otherwise,
find a cut ($ \bm{\bar{x}}\in\mathbb{R}^n\setminus \mathbb{Z}^n $)
\[ \bm{a}^\top \bm{x} \geqslant \alpha \]
such that
\begin{enumerate}[(i)]
    \item $ \bm{a}^\top\bm{x}\leqslant \alpha $
    \item $ \bm{a}^\top \bm{\bar{x}}>\alpha $
\end{enumerate}
for all $ \bm{x} $ feasible in $ (IP) $. The inequality ``cuts''
the current optimal solution $ \bm{\bar{x}} $ of $ (LP) $.

\begin{exbox}
    \begin{example}[Cutting Plane Algorithm]
        \[ (IP) \max x_2 \]
        subject to
        \[
            \begin{array}{ccc}
                3x_1+2x_2             & \leqslant & 6            \\
                -3x_1+2x_2            & \leqslant & 0            \\
                \bm{x}=(x_1,x_2)^\top & \geqslant & \bm{0}       \\
                \bm{x}                & \in       & \mathbb{Z}^2
            \end{array}
        \]
        Introduce slack variables $ x_3, x_4\in\mathbb{Z}_{\geqslant 0} $.
        Then solve the LP relaxation:
        \[
            \begin{array}{cccccccc}
                z   & =   &   & -\frac{1}{4} x_3 & - & \frac{1}{4} x_4 & + & \frac{3}{2} \\
                x_1 &     & + & \frac{1}{6} x_3  & - & x_4             & = & 1           \\
                    & x+2 & + & \frac{1}{4} x_3  & + & \frac{1}{4} x_3 & = & \frac{3}{2}
            \end{array}
        \]
        For every feasible solution of $ (IP) $,
        \begin{align*}
             & x_3+\frac{1}{4} x_3+ \frac{1}{4} x_4 = \frac{3}{2}                                                                          \\
             & \implies x_2 + \left\lfloor \frac{1}{4} \right\rfloor x_3 + \left\lfloor \frac{1}{4} \right\rfloor x_4\leqslant \frac{3}{2}
        \end{align*}
        Since there are no integers in $ (1,\nicefrac{3}{2}) $, every feasible solution of the $ (IP) $
        \[ x_2\leqslant \left\lfloor \frac{3}{2} \right\rfloor=1 \]
        we call this a \emph{cut} since:
        \begin{enumerate}[(i)]
            \item we proved above
            \item $ \bm{\bar{x}}_2=\nicefrac{3}{2}>1  $
        \end{enumerate}
        Add the constraints $ x_2+x_5=1 $, with $ x_5\geqslant 0 $ to the LP relaxation and solve.
        \[
            \begin{array}{cccccccccc}
                z   & = &     &     &   &                 & - & x_5             & + & 1           \\
                x_1 &   &     &     & - & \frac{1}{3} x_4 & + & \frac{2}{5} x_5 & = & \frac{2}{3} \\
                    &   & x_2 &     &   &                 & + & x_5             & = & 1           \\
                    &   &     & x_3 & + & x_4             & - & 4x_5            & = & 2
            \end{array}
        \]
        Add the constraints $ x_1+x_4+x_6=0 $ with $ x_6\geqslant 0 $ to the LP relaxation and solve.
        \[
            \begin{array}{cccccccccccc}
                z   & = &     &     &  &  & - & x_5  &   &                   & + & 1 \\
                x_1 &   &     &     &  &  & + & x_5  & - & \frac{1}{2} x_6 = & = & 1 \\
                    &   & x_2 &     &  &  & + & x_5  &   &                   & = & 1 \\
                    &   &     & x_3 &  &  & - & 5x_5 & + & \frac{3}{2} x_6   & = & 1
            \end{array}
        \]
        $ \bm{x}^*=(1,1)^\top $ with objective value $ z=1 $ is optimal in $ (LP) $.
    \end{example}
\end{exbox}

Another idea for solving IPs is \textbf{Branch-and-Bound} (related to \emph{Divide-and-Conquer}).
Seperate the problem at hand into exaustive and mutually exclusive subproblems (\emph{Branching}).
For each subproblem, solve its relaxation and get an upper bound on the optimal objective
value of the subproblem. If the upper bound is less than the objective value of the current
best integer solution, fathom this branch.