\makeheading{2019-10-22}
Given any LP problem, we know how to convert it into an equivalent LP
problem in SEF:

(P)
\[\max z:=\bm{c}^\top \bm{x}\]
subject to
\begin{align*}
    A \bm{x}=\bm{b} \\
    \bm{x}\geqslant  0
\end{align*}
where $ A\in\mathbb{R}^{m\times n}$ has $\rank(A)=m $.

Given an LP in SEF, with a given basic feasible solution, we know
how to solve it.

\section{Finding Feasible Solutions}
Given an LP in SEF with $ \rank(A)=m $, how do we find a feasible
solution or prove that none exists.

We will construct an \emph{auxiliary LP problem}.

We can always make sure $ \bm{b}\geqslant  \bm{0} $. (If any $ b_i<0 $, multiply both
sides of that equation by $ (-1) $) Introduce artificial variables
$ x_{n+1},\ldots,x_{n+m} $

\begin{defbox}
    \begin{definition}
        Given (P): $\max \left\{ \bm{c}^\top \bm{x},\,A \bm{x}=\bm{b},\bm{x}\geqslant  \bm{0} \right\}$,
        we define the \emph{auxiliary linear program} of (P) as:

        $ (P_{aux}) $
        \[ \min w:=x_{n+1}+\cdots+x_{n+m} \]
        subject to
        \[
            \left[\begin{array}{c|c}
                    A & I
                \end{array}\right]
            \underbrace{\begin{bmatrix}
                    x_1     \\
                    \vdots  \\
                    x_{n}   \\
                    x_{n+1} \\
                    \vdots  \\
                    x_{n+m}
                \end{bmatrix}}_{\bm{x}}
            =\bm{b}\]
        \[ \bm{x} \geqslant  \bm{0}\]
        where $ \bm{b}\geqslant  \bm{0} $, and $ I $ is the $ m\times m $ identity matrix.

        We call the variables $ x_{n+1},\ldots,x_{n+m} $ \emph{auxiliary variables}.
    \end{definition}
\end{defbox}

For every feasible solution of $ (P_{aux}) $, $ w\geqslant  0 $.

Therefore, $ (P_{aux}) $ is not unbounded.

If the optimal value of $ (P_{aux}) $ is zero, let
$ (\hat{x}_1,\ldots,\hat{x}_{n+m})^\top$
be the basic feasible solution of $ (P_{aux}) $. Then,
$ (\hat{x}_1,\ldots,\hat{x}_{n})^\top$
is a basic feasible solution of (P).

It is basic since $ \{A_j : \hat{x_j}>0\} $ is linearly independent where
$ J $ is the column indices $ j $ of $ A $ for which $ \hat{x}_j\neq 0 $.

If $ |\{j:\hat{x_j}>0\}|=m $, this index set is a basis of $ A $ which
determines $ (\hat{x}_1,\ldots,\hat{x}_{n})^\top$.

If $ |\{j:\hat{x_j}>0\}|\leqslant m-1 $, we can extend this index set
to be a basis of $ A $, since $ \rank(A)=m $.

If the optimal value of $ (P_{aux}) $ is positive, then (P) is
infeasible. We state this as a theorem.

\begin{thmbox}
    \begin{theorem}
        Let $ \bar{\bm{x}}=(\bar{x}_1,\ldots ,\bar{x}_{n+m})^\top $ be an optimal solution
        to $ (P_{aux}) $.
        \begin{enumerate}[(1)]
            \item if $ w=0 $, then $ (\bar{x}_1,\ldots,\bar{x}_n)^\top $ is a solution to (P).
            \item if $ w>0 $, then (P) is infeasible.
        \end{enumerate}
    \end{theorem}
\end{thmbox}

\begin{proof}
    Let $ \bar{\bm{x}}=(\bar{x}_1,\ldots ,\bar{x}_{n+m})^\top $ be an optimal solution
    to $ (P_{aux}) $.

    (1) Assume $ w=0 $, then $ \bar{x}_{n+1}=\cdots=\bar{x}_{n+m}=0 $. Thus
    $ (\bar{x}_1,\ldots \bar{x}_n)^\top $ is a feasible solution of (P).

    (2) Assume $ w>0 $. Suppose for a contradiction that there exists a feasible
    solution $(\bar{x}_1,\ldots \bar{x}_n)^\top$ to (P). Then,
    $ (\bar{x}_1,\ldots \bar{x}_n,\underbrace{0,\ldots ,0}_{m\text{ terms}}) $ is a feasible solution to $ (P_{aux}) $
    with optimal objective value $0$ which is a contradiction to the fact that
    $ \bm{\bar{x}} $ is optimal.
\end{proof}

\begin{algbox}
    \begin{algorithm}[H]
        \caption{Two Phase Method}
        \SetKwInOut{Input}{Input}
        \SetKwInOut{Output}{Output}
        \Input{$A,\bm{b}, \bm{c}$ data for LP in SEF such that
            full row rank and $ \bm{b}\geqslant  \bm{0} $.}
        Construct $ (P_{aux}) $ put into SEF, $ B:=\{n+1,n+2,\ldots,n+m\} $\\
        Put $ (P_{aux}) $ into the canonical form determined by $ B $.\\
        Solve $ (P_{aux}) $ starting with basis $ B $ by Simplex Method.\\
        If the optimal value of $ (P_{aux}) $ is zero, then we have a basic
        feasible solution of (P). Solve (P) using Simplex Method. This is
        known as Phase II.\\
        If the optimal objective value of $ (P_{aux}) $ is not zero, then
        (P) is infeasible (a certificate of infeasibility is given by
        the last $ \bm{\bar{y}} $ computed).\\
    \end{algorithm}
\end{algbox}

As seen above, the original LP can either have an optimal
solution or be infeasible when performing the Two Phase Method.

\subsection{The Two Phase Simplex Algorithm---An Optimal Example}
\begin{exbox}
    \begin{example}[Two Phase---Optimal]
        (P)
        \[ \max z:= \begin{bmatrix} 1 & 2 & -1 \end{bmatrix} \bm{x} \]
        subject to
        \[
            \begin{bmatrix}
                1  & -2 & -3 \\
                -1 & 1  & 1
            \end{bmatrix}
            \bm{x} =
            \begin{bmatrix}
                -3 \\
                1
            \end{bmatrix}
        \]
        \[ \bm{x}\geqslant  \bm{0} \]

        Since $ b_1<0 $
        we write
        \[ \begin{bmatrix}
                -1 & 2 & 3 \\
                -1 & 1 & 1
            \end{bmatrix}
            x=
            \begin{bmatrix}
                3 \\
                1
            \end{bmatrix}
        \]
        we do this because we will not have a feasible solution if $ \bm{b}<\bm{0} $.

        Introduce artificial variables: $ x_4, x_5 $

        \underline{Phase I}

        $ (P_{aux}) $
        \[\max -w:=\begin{bmatrix} 0 & 0 & 0 & -1 & -1 \end{bmatrix} \bm{x} \]
        subject to
        \[
            \begin{bmatrix}
                -1 & 2 & 3 & 1 & 0 \\
                -1 & 1 & 1 & 0 & 1
            \end{bmatrix}
            \bm{x}=
            \begin{bmatrix}
                3 \\
                1
            \end{bmatrix}
        \]
        \[ \bm{x}:=(x_1,x_2,x_3,x_4,x_5)^\top \geqslant  \bm{0} \]

        Turn $ (P_{aux}) $ into canonical form for $ B:=\{4,5\} $
        \[\max -w:=\begin{bmatrix} -2 & 3 & 4 & 0 & 0 \end{bmatrix} \bm{x} \]
        subject to
        \[
            \begin{bmatrix}
                -1 & 2 & 3 & 1 & 0 \\
                -1 & 1 & 1 & 0 & 1
            \end{bmatrix}
            \bm{x}=
            \begin{bmatrix}
                3 \\
                1
            \end{bmatrix}
        \]
        \[ \bm{x}\geqslant  \bm{0} \]
        We solve the LP via Simplex Algorithm and obtain the following
        LP corresponding to the optimal basis of $ B=\{1,2\} $
        \[\max -w:=\begin{bmatrix} 0 & 0 & 0 & -1 & -1 \end{bmatrix} \bm{x} \]
        subject to
        \[
            \begin{bmatrix}
                1 & 0 & 1 & 1 & -2 \\
                0 & 1 & 2 & 1 & -1
            \end{bmatrix}
            \bm{x}=
            \begin{bmatrix}
                1 \\
                2
            \end{bmatrix}
        \]
        \[ \bm{x}\geqslant  \bm{0} \]
        End of Phase I.

        \underline{Phase II}

        We rewrite the original LP in canonical form corresponding
        to basis $ B=\{1,2\} $ to obtain
        \[ \max z:= \begin{bmatrix}0 & 0 & -6 \end{bmatrix}\bm{x}+5 \]
        subject to
        \[ \begin{bmatrix}
                1 & 0 & 1 \\
                0 & 1 & 2
            \end{bmatrix}\bm{x}=
            \begin{bmatrix}
                1 \\
                2
            \end{bmatrix} \]
        \[ \bm{x}:=(x_1,x_2,x_3)^\top\geqslant  \bm{0} \]

        We obtain the optimal basic feasible solution of $ (P_{aux}) $ via the Simplex
        Algorithm
        \[(\hat{x_1}, \hat{x_2}, \hat{x_3}, \hat{x_4}, \hat{x_5}):=
            (1,2,0,0,0)^\top\]
        Thus, the corresponding basic feasible solution of (P) is
        \[(\hat{x_1}, \hat{x_2}, \hat{x_3})=(1,2,0)^\top\]
        with an optimal objective value of $ z:= 5 $.
    \end{example}
\end{exbox}

\subsection{The Two Phase Simplex Algorithm---An Infeasible Example}
\begin{exbox}
    \begin{example}[Two Phase---Infeasible]
        (P)
        \[ \max z:=\begin{bmatrix} 3 & 2 & 4 \end{bmatrix} \bm{x} \]
        subject to
        \[
            \underbrace{
                \begin{bmatrix}
                    5  & 1 & 1 \\
                    -1 & 1 & 2 \\
                \end{bmatrix}}_{A}
            \bm{x}=
            \begin{bmatrix}
                1 \\
                5
            \end{bmatrix}
        \]
        \[\bm{x}\geqslant  \bm{0}\]
        $ (P_{aux}) $
        \[ \max -w:=\underbrace{\left[\begin{array}{ccccc}
                        0 & 0 & 0 & -1 & -1
                    \end{array}\right]}_{\bm{\tilde{c_B}}} \bm{x} \]
        subject to
        \[
            \underbrace{\left[
                    \begin{array}{ccccc}
                        5  & 1 & 1 & 1 & 0 \\
                        -1 & 1 & 2 & 0 & 1
                    \end{array}\right]}_{\tilde{A}}
            \bm{x}=
            \begin{bmatrix}
                1 \\
                5
            \end{bmatrix}
        \]
        \[ \bm{x}:=(x_1,x_2,x_3,x_4,x_5)^\top \geqslant  \bm{0} \]
        Turn $ (P_{aux}) $ into canonical form for $ B:=\{4,5\} $ (by adding
        the constraints up to the original objective function).
        \[ \max -w:=\left[\begin{array}{ccccc}
                    4 & 2 & 3 & 0 & 0
                \end{array}\right] \bm{x} - 4 \]
        subject to
        \[
            \left[
                \begin{array}{ccccc}
                    5  & 1 & 1 & 1 & 0 \\
                    -1 & 1 & 2 & 0 & 1
                \end{array}\right]
            \bm{x}=
            \begin{bmatrix}
                1 \\
                5
            \end{bmatrix}
        \]
        \[ \bm{x}\geqslant  \bm{0} \]
        Starting with the basis $ B=\{4,5\} $, solve $ (P_{aux}) $
        with the Simplex Algorithm to get:
        \[ \max -w=\begin{bmatrix}-11 & -1 & 0 & -3 & 0\end{bmatrix} \bm{x} - 3 \]
        subject to
        \[
            \begin{bmatrix}
                5   & 1  & 1 & 1  & 0 \\
                -11 & -1 & 0 & -2 & 1
            \end{bmatrix}
            \bm{x}=
            \begin{bmatrix}
                1 \\
                3
            \end{bmatrix}
        \]
        \[ \bm{x}\geqslant  \bm{0} \]

        The optimal value of $ (P_{aux}) $ is not zero. Therefore, (P) is
        infeasible. The basis in the last iteration was $ B=\{3,5\} $.

        \[ \bm{y}^\top =\bm{\tilde{c}_B}^\top\tilde{A}_B^{-1} \iff
            \bm{y}^\top
            =
            \underbrace{\begin{bmatrix}0 & -1\end{bmatrix}}_
            {\text{SEF of $ (P_{aux}) $ }}
            \begin{bmatrix}
                1 & 0 \\
                2 & 1
            \end{bmatrix}^{-1}
        \]
        $ \bm{\bar{y}}=(2,-1)^\top $
        is a certificate of infeasibility of (P).

        Compute $ \bm{\bar{y}}^\top  A
            = \begin{bmatrix}11 & 1 & 0 \end{bmatrix}\geqslant  \bm{0}^\top $ and
        $ \bm{\bar{y}}^\top \bm{b}=-3=\bm{c}^\top \bm{x} $
        where $ \bm{\bar{x}}=(0,0,0,1,3)^\top $.

        Thus, $ \bm{\bar{y}} $ is a certificate of optimality for $ (P_{aux}) $.
    \end{example}
\end{exbox}

\begin{thmbox}
    \begin{theorem}[Fundamental Theorem of LP (SEF)]
        Let (P) be an LP problem in SEF, where $ A\in \mathbb{R}^{m\times n} $ has $ \rank(A)=m $.
        \begin{enumerate}[(1)]
            \item if (P) does not have an optimal solution, then (P) is either infeasible or unbounded.
            \item if (P) has a feasible solution, then (P) has a basic feasible solution.
            \item if (P) has an optimal solution, then (P) has an optimal basic feasible solution.
        \end{enumerate}
    \end{theorem}
\end{thmbox}
