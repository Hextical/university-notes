\section{2019-09-19}
\begin{defbox}
    \subsection{Definition (Non-linear Program)}
    A \emph{non-linear program} has the form
    \[\min f(x)\]
subject to
\begin{align*}
    g_1(x)&\le 0\\
    g_2(x)&\le 0\\
    &\vdots\\
    g_m(x)&\le 0
\end{align*}
where
$f:\mathbb{R}^n\rightarrow \mathbb{R}$, 
$g_i:\mathbb{R}^n\rightarrow\mathbb{R}$, $\forall i\in\{1,\dots,m\}$.
\end{defbox}

Every LP problem is a very special case of a NLP problem. IP problems can
also be formulated as NLP problems.
\subsection{Example (Formulating LP problems as NLP problems)}
\begin{align*}
    x_i\in\mathbb{Z}&\iff \sin(\pi x_i)=0\\
    &\iff [\sin(\pi x_i)]^2\le 0
\end{align*}
NLP problems have huge modelling power, as a result, one must understand the
structure of the underlying problem and construct "good" NLP models that are
amendable to analysis and solution techniques.

\subsection{Example (Portfolio Optimization)}
There are $n$ stocks $1,\dots,n$ to invest in. We have a budget of $B$ dollars.
We have an expected return (for \$1 investment at the end of our planning
horizon) of $\mu_1,\dots,\mu_n$. We are also given 
$V\in \mathbb{R}^{n\times n}$, a variance coefficient matrix so that if
we invest in $x_1,\dots,x_n$ dollars in $n$ stocks, $1,\dots,n$ respectively,
then the expected risk of such an investment is given by 
$\symbfit{x}^\top  V\symbfit{x}$.
\[\sum\limits_{i = 1}^{n}\sum\limits_{j = 1}^{n}V_{ij}x_ix_j\]
$x_j:=$ amount of investment in stock $j$ in dollars.

Suppose we are also given a goal $G$ (a dollar amount we want as the value
of our portfolio at the end of the planning horizon).

\textbf{Data}
\begin{itemize}
    \item Budget (\$) $\rightarrow$ B
    \item Goal (\$) $\rightarrow$ G
    \item Expected return $\rightarrow$
    $\begin{bmatrix}
        \mu_1\\
        \vdots\\
        \mu_n
    \end{bmatrix}$
    \item Variance-covariance matrix $\rightarrow$ 
    $V\in \mathbb{R}^{n\times n}$
\end{itemize}
We want to minimize the risk of our portfolio while satisfying the budget and
the goal constraints. (NLP)
\[\min \symbfit{x}^\top  V\symbfit{x}\]
subject to
\begin{align*}
    \sum_{j = 1}^{n}x_j\ge B\\
    \sum_{j = 1}^{n}\mu_jx_j\le G\\
    \symbfit{x}\ge \mathbb{0}
\end{align*}
There are many variants of such models and extensions. For example, instead
of a goal $G$, we may given an upper bound on the risk, say 
$R\in\mathbb{R}_{>0}$. (NLP)
\[\max \sum_{j = 1}^{n}\mu_jx_j\]
subject to
\begin{align*}
    \sum_{j = 1}^{n}x_j\ge B\\
    \symbfit{x}^\top  V\symbfit{x}\le R\\
    \symbfit{x}\ge \mathbb{0}
\end{align*}
We can handle many more variants and extensions. Suppose investing in stock
$j$ below $l_j$ dollars is not allowed. For diversity of our portfolio, we
want to invest in at least 20 stocks, and for the sake of simplicity we want
to invest in at most 150 stocks. We introduce a binary-valued variable $z_j$.
\[z_j:=
\begin{cases}
    1 \text{, if we invest in stock $j$}\\
    0 \text{, otherwise}
\end{cases}
\forall j\in\{1,\dots,n\}\\
\]
Add these constraints:
\begin{align*}
    &l_jz_j\le x_j\le Bz_j\qquad \forall j\in\{1,\dots,n\}\\
    &20\le \sum_{j = 1}^{n}z_j\le 150\\
    &z_j\in\{0,1\}
\end{align*}