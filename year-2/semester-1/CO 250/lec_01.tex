\section{2019-09-05}
\subsection{Example (Manufacturing Tables and Chairs)}
Process: raw materials $\rightarrow$ machine $\rightarrow$ labour $\rightarrow$ final products

Rules:
\begin{itemize}
    \item Company has 30 workers and 40 machines available 40hrs/week.
    \item Manufacturing a table requires 2 machine-hours and 1 labour-hour.
    \item Manufacturing a chair requires 1 machine-hours and 3 labour-hours.
    \item Each manufacturer table yields \$10 of profit and each manufacturer chair yields \$15 of profit.
\end{itemize}

Goal: The company wants to prepare a weekly production plan which maximizes
total profit.

Variables:
\begin{itemize}
    \item $x_1:=$ the number of tables manufactured per week
    \item $x_2:=$ the number of chairs manufactured per week
\end{itemize}

The total profit per week can be modelled by $10x_1 + 15x_2$.

Constraints:
\begin{itemize}
    \item Machine-hours used per week $\le$  machine-hours available per week which
can be modelled by $2x_1 + x_2 \le  40 \times 40 = 1600$
    \item Labour-hours used per week $\le$ labour-hours available per week which can
    be modelled by $x_1 + 3x_2 \le 30 \times 40 = 1200$
\end{itemize}

We can then create a linear programming (LP) model.
\[\max 10x_1 + 15x_2\]
subject to
\begin{align*}
    2x_1 + x_2 \le 1600\\
    x_1 + 3x_2 \le 1200\\
    x_1, x_2 \ge 0
\end{align*}

\subsection{Example (A General Production Planning Problem)}
There are resources $I:=\{1,\dots,m\}$ and products $J:=\{1,\dots,n\}$. 
There are $b_i$ units of resource
$i$ available per week $\forall i\in I$. One unit of product $j$ yields $c_j$ of profit for
$\forall j\in J$. Manufacturing one unit of product $j$ requires $a_{ij}$ units of resource $i$.
We want to maximize the total profit of this manufacturing process.
$x_j :=$ amount of product $j$ manufactured per week. (LP)
\[\max c_1x_1 + \dots + c_nx_n = \sum\limits_{j=1}^n c_jx_j\]
subject to
\begin{align*}
    \sum\limits_{j=1}^n a_{ij}x_{ij}\le b_i \qquad\qquad\forall i\in\{1,\dots,m\}\\
    x_{j}\ge 0 \qquad\qquad\forall j\in\{1,\dots,n\}
\end{align*}

\begin{remark}
    If $\symbfit{x},\symbfit{y}\in\mathbb{R}^n$ and $\symbfit{x}\le \symbfit{y}$, then
    $x_1\le y_1, \dots x_n, \le y_n$.
\end{remark}

\begin{remark}
\[
\symbfit{c}:=\left[\begin{array}{c}{c_{1}} \\ {\vdots} \\ {c_{n}}\end{array}\right],
\symbfit{x}:=\left[\begin{array}{c}{x_{1}} \\ {\vdots} \\ {x_{n}}\end{array}\right],
A:=\left[\begin{array}{cccc}
    {a_{11}} & \cdots & {a_{1 n}} \\
    \vdots & & \vdots \\
     {a_{m 1}} & \cdots & {a_{m n}}
    \end{array}\right],
\symbfit{b}:=\left[\begin{array}{c}{b_{1}} \\ {\vdots} \\ {b_{n}}\end{array}\right]
\]
Given $A,\symbfit{b},\symbfit{c}$ with $\symbfit{x}\in\mathbb{R}^n$ as the variable vector, we realize that
$\symbfit{c}^\top  \symbfit{x}=\sum\limits_{j=1}^n c_jx_j$ is exactly the model that we wanted to maximize
in 1.2 such that it satisfies $A\symbfit{x}\le \symbfit{b}$, with $\symbfit{x}\ge \mathbb{0}$.
\end{remark}

\begin{defbox}
    \subsection{Definition (Affine Function)}
    Let $f:\mathbb{R}^n\rightarrow\mathbb{R}$. $f$ is an \emph{affine function} if
    $f(x)=\symbfit{a}^\top \symbfit{x}+\beta$ for some $\symbfit{a}\in\mathbb{R}^n$, and $\beta\in\mathbb{R}$.    
\end{defbox}

\begin{defbox}
    \subsection{Definition (Linear Function)}
    Let $f:\mathbb{R}^n\rightarrow\mathbb{R}$. $f$ is a \emph{linear function} if
    $f$ is an affine function such that $\beta=0$.
\end{defbox}

\begin{remark}
    Every linear function is affine, but the converse is not true.
\end{remark}

\begin{defbox}
    \subsection{Definition (Linear Constraint)}
    A \emph{linear constraint} is one of
    \begin{align*}
        f(x)\le \beta\\
        f(x)=\beta\\
        f(x)\ge \beta
    \end{align*}
    where $f$ is a linear function and $ \beta\in\mathbb{R} $
\end{defbox}

\begin{defbox}
    \subsection{Definition (Linear Program)}
    A \emph{linear program} (LP) is a problem of minimizing or
    maximizing an affine function subject to a finite number
    of constraints.
\end{defbox}