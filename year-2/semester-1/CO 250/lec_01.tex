\chapter{Introduction}
\makeheading{2019-09-05}
\begin{exbox}
    \begin{example}[\label{Manufacturing Tables and Chairs}Manufacturing Tables and Chairs]

        $ \; $

        \textsc{Process}: raw materials $\rightarrow$ machine $\rightarrow$ labour $\rightarrow$ final products

        \textsc{Rules}:
        \begin{itemize}
            \item Company has 30 workers and 40 machines available 40hrs/week.
            \item Manufacturing a table requires 2 machine-hours and 1 labour-hour.
            \item Manufacturing a chair requires 1 machine-hours and 3 labour-hours.
            \item Each manufacturer table yields \$10 of profit and each manufacturer chair yields \$15 of profit.
        \end{itemize}

        \textsc{Goal}: The company wants to prepare a weekly production plan which maximizes
        total profit.

        \textbf{Variables}
        \begin{itemize}
            \item $x_1:=$ the number of tables manufactured per week
            \item $x_2:=$ the number of chairs manufactured per week
        \end{itemize}

        \textbf{Objective function}

        The total profit per week can be modelled by $10x_1 + 15x_2$, which is what
        we want to maximize.

        \textbf{Constraints}
        \begin{itemize}
            \item Machine-hours used per week $\le$  machine-hours available per week which
                  can be modelled by $2x_1 + x_2 \leqslant  40 \times 40 = 1600$
            \item Labour-hours used per week $\le$ labour-hours available per week which can
                  be modelled by $x_1 + 3x_2 \leqslant 30 \times 40 = 1200$
        \end{itemize}

        We can then formulate the linear programming (LP) model:
        \begin{equation}
            \begin{aligned}
                 & \text{minimize}   & \quad & 10x_1 + 15x_2             \\
                 & \text{subject to} &       & 2x_1 + x_2 \leqslant 1600 \\
                 &                   &       & x_1 + 3x_2 \leqslant 1200 \\
                 &                   &       & x_1, x_2 \geqslant  0
            \end{aligned}\tag{LP}
        \end{equation}

        An \emph{optimal} solution to the LP using an algorithm later in
        in this course is $ \bm{x}:=(720,160)^\top $, which means
        that we want $ 720 $ tables, and $ 160 $ chairs.
    \end{example}
\end{exbox}

\begin{exbox}
    \begin{example}[A General Production Planning Problem]
        There are resources $I:=\{1,\dots,m\}$ and products $J:=\{1,\dots,n\}$.
        There are $b_i$ units of resource
        $i$ available per week $\forall i\in I$. One unit of product $j$ yields $c_j$ of profit for
        $\forall j\in J$. Manufacturing one unit of product $j$ requires $a_{ij}$ units of resource $i$.
        We want to maximize the total profit of this manufacturing process.
        $x_j :=$ amount of product $j$ manufactured per week.
        \begin{equation}
            \begin{aligned}
                 & \text{maximize}   & \quad & c_1x_1 + \cdots + c_n x_n = \sum\limits_{j=1}^n c_j x_j                             \\
                 & \text{subject to} &       & \sum\limits_{j=1}^n a_{ij}x_{ij}\leqslant b_i           & \forall i\in\{1,\dots,m\} \\
                 &                   &       & x_j\geqslant  0                                         & \forall j\in\{1,\dots,n\}
            \end{aligned}\tag{LP}\label{1.2}
        \end{equation}
    \end{example}
\end{exbox}

\begin{remark}
    If $\bm{x},\bm{y}\in\mathbb{R}^n$ and $\bm{x}\leqslant \bm{y}$, then
    $x_1\leqslant  y_1, \ldots, x_n\leqslant y_n$.
\end{remark}

\begin{remark}
    \[
        \bm{c}:=\left[\begin{array}{c}{c_{1}} \\ {\vdots} \\ {c_{n}}\end{array}\right]\quad
        \bm{x}:=\left[\begin{array}{c}{x_{1}} \\ {\vdots} \\ {x_{n}}\end{array}\right]\quad
        A:=\left[\begin{array}{cccc}
                {a_{11}}  & \cdots & {a_{1 n}} \\
                \vdots    &        & \vdots    \\
                {a_{m 1}} & \cdots & {a_{m n}}
            \end{array}\right]\quad
        \bm{b}:=\left[\begin{array}{c}{b_{1}} \\ {\vdots} \\ {b_{n}}\end{array}\right]
    \]
    Given $A,\bm{b},\bm{c}$ with $\bm{x}\in\mathbb{R}^n$ as the variable vector, we realize that
    $\bm{c}^\top  \bm{x}=\sum\limits_{j=1}^n c_j x_j$ is exactly the model that we wanted to maximize
    in~\ref{1.2} such that it satisfies $A\bm{x}\leqslant \bm{b}$, with $\bm{x}\geqslant  \bm{0}$.
\end{remark}

\begin{defbox}
    \begin{definition}
        Let $f:\mathbb{R}^n\rightarrow\mathbb{R}$. $f$ is an \textbf{\emph{affine function}} if
        $f(\bm{x})=\bm{a}^\top \bm{x}+\beta$ for some $\bm{a}\in\mathbb{R}^n$ and $\beta\in\mathbb{R}$.
    \end{definition}
\end{defbox}

\begin{defbox}
    \begin{definition}
        Let $f:\mathbb{R}^n\rightarrow\mathbb{R}$. $f$ is a \textbf{\emph{linear function}} if
        $f$ is an affine function with $\beta=0$.
    \end{definition}
\end{defbox}

\begin{remark}
    Every linear function is affine, but the converse is not true.
\end{remark}

\begin{defbox}
    \begin{definition}
        A \textbf{\emph{linear constraint}} is one of
        \begin{align*}
            f(\bm{x})\leqslant \beta \\
            f(\bm{x})=\beta          \\
            f(\bm{x})\geqslant  \beta
        \end{align*}
        where $f$ is a linear function and $ \beta\in\mathbb{R} $.
    \end{definition}
\end{defbox}

\begin{defbox}
    \begin{definition}
        A \textbf{\emph{linear program}} (LP) is the problem of minimizing or
        maximizing an affine function subject to a finite number
        of linear constraints.
    \end{definition}
\end{defbox}
