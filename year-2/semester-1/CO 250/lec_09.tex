\section{2019-10-03}
\begin{remark}
    The following lecture will not be 1-1 since the explanations in class were useless.
\end{remark}

\subsection{Example (Continuation of 8.3)}
So, the new basic feasible solution is $\symbfit{\bar{x}}:=(1,5,0,3,0)^\top$
determined by the basis $B:=\{1,2,5\}\cup\{4\}\setminus \{5\}=\{1,2,4\}$. Note that
we exclude $\{5\}$ since the index of which $t$ achieved the minimum was at
$\nicefrac{6}{2}$, i.e. index $5$ (row $x_5$). The canonical form determined by the new
basis is
\[\max \begin{bmatrix}
    0 & 0 & -\nicefrac{5}{2} & 0 & -\nicefrac{1}{2}
\end{bmatrix}\symbfit{x}+3\]
subject to
\[
    \begin{bmatrix}
        1 & 0 & \nicefrac{1}{2} & 0 & -\nicefrac{1}{2}\\
        0 & 1 & \nicefrac{1}{2} & 0 & \nicefrac{1}{2}\\
        0 & 0 & -\nicefrac{3}{2} & 1 & \nicefrac{1}{2}
    \end{bmatrix}\symbfit{x}
    =
    \begin{bmatrix}
        1\\
        5\\
        3
    \end{bmatrix}
\]
\begin{remark}
    $\symbfit{\bar{x}}$ is the optimal solution with optimal value $3$.
\end{remark}
\begin{remark}
    How did we arrive to this LP? Using the formulae in Proposition 8.2. If you didn't want to
    calculate $A_B^{-1}$, then follow the below instructions.
\end{remark}

\subsection{Example (Canonical form without computing the inverse)}
\begin{remark}
    The following was not taught in class or the textbook. 
    This method can be confusing and not intuitive.
\end{remark}


Write
\[A:=
    \left[\begin{array}{ccccc|c}  
        1 & 0 & -1 & 1 & 0 & 4\\
        0 & 1 & 2 & -1 & 0 & 2\\
        0 & 0 & -3 & 2 & 1 & 6
       \end{array}\right]
    \rightarrow
    \left[\begin{array}{ccccc|c}
        1 & 0 & \nicefrac{1}{2} & 0 & -\nicefrac{1}{2} & 1\\
        0 & 1 & \nicefrac{1}{2} & 0 & \nicefrac{1}{2} & 5\\
        0 & 0 & -\nicefrac{3}{2} & 1 & \nicefrac{1}{2} & 3
    \end{array}\right]
\]
and row reduce $A$ to make fourth column get a leading one as seen above.
The row-reduced matrix and the augment are your new constraints.\\
The objective function is tricky, we want a 0 in the fourth column of our $\symbfit{c}^\top $.
Also, we denote $x_1,x_2,x_4$ as the rows of the matrix respectively.
Using $x_4$ (which is our row-reduced $A$), we get
\[-1\left(
    \begin{bmatrix}
        0 & 0 & -\nicefrac{3}{2} & 1 & \nicefrac{1}{2}
    \end{bmatrix}\symbfit{x}-3\right)+
    \begin{bmatrix}
        0 & 0 & -4 & 1 & 0
    \end{bmatrix}\symbfit{x}
\]
The $-3$ right after the first matrix was the row of $\symbfit{b}$.

\newpage

\subsection{Simplex Algorithm}
\begin{algorithm}
    \caption{Simplex Algorithm}
    \SetKwInOut{Input}{Input}
    \SetKwInOut{Output}{Output}
    \Input{$A\in \mathbb{R}^{m\times n}$, $\symbfit{b}\in\mathbb{R}^m$, $\symbfit{c}\in\mathbb{R}^n$}
    Compute the canonical form for $B$, let $\symbfit{\bar{x}}$ be the basic feasible solution.\\
    If $\symbfit{c_N}\le 0$, then stop ($\symbfit{\bar{x}}$ is optimal).\\
    Choose $k\in N$ such that $c_k>0$.\\
    If $A_k\le \mathbb{0}$, then stop (the LP is unbounded).\\
    Let $r$ be the index which attains $t=\min\{\nicefrac{b_i}{A_{ik}}\mid A_{ik}>0\}$.\\
    Let $l\in B$ be the $r^{th}$ basis element.\\
    Set $B:=B\cup\{k\}\setminus\{l\}$.\\
    Go to step 1.
\end{algorithm}

\begin{thmbox}
    \subsection{Bland's Rule}
    In step 3, among all $k\in N$, with $c_k>0$ and in step 5, $r\in B$, 
    choose the smallest index for both $k$ and $r$.
\end{thmbox}

