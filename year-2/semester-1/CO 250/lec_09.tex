\section{2019-10-03}

\subsection{Example (Continuation of 8.3)}
So, the new basic feasible solution is $\bm{\bar{x}}:=(1,5,0,3,0)^\top$
determined by the basis $B:=\{1,2,5\}\cup\{4\}\setminus \{5\}=\{1,2,4\}$. Note that
we exclude $\{5\}$ since the index of which $t$ achieved the minimum was at
$\nicefrac{6}{2}$, i.e. index $5$ (row $x_5$). The canonical form determined by the new
basis is
\[\max \begin{bmatrix}
    0 & 0 & -\nicefrac{5}{2} & 0 & -\nicefrac{1}{2}
\end{bmatrix}\bm{x}+3\]
subject to
\[
\begin{bmatrix}
    1 & 0 & \nicefrac{1}{2} & 0 & -\nicefrac{1}{2}\\
    0 & 1 & \nicefrac{1}{2} & 0 & \nicefrac{1}{2}\\
    0 & 0 & -\nicefrac{3}{2} & 1 & \nicefrac{1}{2}
\end{bmatrix}\bm{x}
=
\begin{bmatrix}
    1\\
    5\\
    3
\end{bmatrix}
\]
\begin{remark}
    $\bm{\bar{x}}$ is the optimal solution with optimal value $3$.
\end{remark}
\begin{remark}
    How did we arrive to this LP? Using the formulae in Proposition 8.2. If you didn't want to
    calculate $A_B^{-1}$, then follow the below instructions.
\end{remark}

\subsection{Example (Canonical form without computing the inverse)}
\begin{remark}
    The following was not taught in class or the textbook. 
    This method can be confusing and not intuitive.
\end{remark}


Write
\[A:=
\left[\begin{array}{ccccc|c}  
    1 & 0 & -1 & 1 & 0 & 4\\
    0 & 1 & 2 & -1 & 0 & 2\\
    0 & 0 & -3 & 2 & 1 & 6
\end{array}\right]
\rightarrow
\left[\begin{array}{ccccc|c}
    1 & 0 & \nicefrac{1}{2} & 0 & -\nicefrac{1}{2} & 1\\
    0 & 1 & \nicefrac{1}{2} & 0 & \nicefrac{1}{2} & 5\\
    0 & 0 & -\nicefrac{3}{2} & 1 & \nicefrac{1}{2} & 3
\end{array}\right]
\begin{matrix}
    -x_1\\
    -x_2\\
    -x_4
\end{matrix}
\]
and row reduce $A$ to make fourth column get a leading one as seen above.
The row-reduced matrix and the augment are your new constraints.\\
The objective function is tricky, we want a 0 in the fourth column of our $\bm{c}^\top $.
Also, we denote $x_1,x_2,x_4$ as the rows of the matrix respectively as seen above.
Using $x_4$ (which is our row-reduced $A$), we get
\[(-1)\left(
\begin{bmatrix}
    0 & 0 & -\nicefrac{3}{2} & 1 & \nicefrac{1}{2}
\end{bmatrix}\bm{x}-3\right)+
(\begin{bmatrix}
    0 & 0 & -4 & 1 & 0
\end{bmatrix}\bm{x})
\]
The $-3$ right after the first matrix was the row of $\bm{b}$. General form:
\[ c([\text{Row}_i(A)]\bm{x}-\bm{b_i})+\text{original objective function} \]
where $ c $ is a constant.

\begin{algbox}
    \subsection{Algorithm (Simplex Algorithm)}
    \begin{algorithm}[H]
        \caption{Simplex Algorithm}
        \SetKwInOut{Input}{Input}
        \SetKwInOut{Output}{Output}
        \Input{$A\in \mathbb{R}^{m\times n}$, $\bm{b}\in\mathbb{R}^m$, $\bm{c}\in\mathbb{R}^n$ such that we have
        linear program (P): $\max \left\{\bm{c}^\top \bm{x},\,A \bm{x}=\bm{b},\bm{x}\ge \bm{0} \right\}$,
        and a feasible basis $ B $.}
        \Output{An optimal solution $ \bm{\bar{x}} $ of (P) or a certificate proving that the (P) is unbounded.}
        Compute the canonical form for the basis $B$. Let $\bm{\bar{x}}$ 
        be the basic feasible solution for $ B $.\\
        If $\bm{c_N}\le \bm{0}$, then stop ($\bm{\bar{x}}$ is optimal).\\
        Choose $k\in N$ such that $c_k>0$.\\
        If $\bm{a_k}\le \bm{0}$, then stop (the LP is unbounded).\\
        Let $r$ be any index $ i $ where the following minimum is attained:
        \[t=\min\left\{\frac{b_i}{a_{ik}} : a_{ik}>0\right\}\]\\
        Let $\ell$ be the $r^{th}$ basis element.\\
        Set $B:=B\cup\{k\}\setminus\{\ell\}$.\\
        Go to step 1.
    \end{algorithm}
\end{algbox}

\begin{thmbox}
    \subsection{Bland's Rule}
    Throughout the Simplex iterations with $ t=0 $, in Step 3, among
    all $ j\in N $, with $ c_j>0 $, choose $ k:=\min\left\{ j\in N:c_j>0\right\} $;
    also in Step 5, define $ t $ as before and choose the smallest
    $ r\in B $ with $ a_{rk}>0 $, and $ \nicefrac{b_r}{a_{rk}}=t $.
\end{thmbox}

\subsection{Example (Simplex Algorithm with Bland's Rule)}
Solve
(P)
\[ \max
\begin{bmatrix}
    0 & 3 & 1 & 0
\end{bmatrix}\bm{x} \]
subject to
\[ 
\begin{bmatrix}
    1 & 2 & -2 & 0\\
    0 & 1 & 3 & 1
\end{bmatrix}\bm{x}=
\begin{bmatrix}
    2\\
    5
\end{bmatrix} \]
\[ \bm{x}\ge \bm{0} \]
using the Simplex Algorithm with Bland's Rule. Give a certificate
of optimality or unboundedness for the problem, and verify it.

\emph{Solution.}

\myuline{Iteration 1}

Useful values computed:
\[ A_B=
\begin{bmatrix}
    1 & 0\\
    0 & 1\\
\end{bmatrix} \Rightarrow
A_B^{-1}=
\begin{bmatrix}
    1 & 0\\
    0 & 1\\
\end{bmatrix} \]
\[ \bm{y}^\top=\bm{c_B}^\top A_B^{-1}=
\begin{bmatrix}
    0 & 0
\end{bmatrix}
\begin{bmatrix}
    1 & 0\\
    0 & 1\\
\end{bmatrix}=
\begin{bmatrix}
    0 \\
    0
\end{bmatrix}\]

1. The LP is already in canonical form determined by $ B=\{1,4\} $.
Let $ \bm{\bar{x}}:=(2,0,0,5)^\top $ be the basic feasible solution for $ B $.

2. $ c_{\{2,3\}} \nleq \bm{0} $, so $ \bm{\bar{x}} $ is not optimal.

3. Using Bland's Rule we choose $ k=2\in N $ which enters the basis
as $ c_2\ge 0 $.

4.
$ 
a_2= \begin{bmatrix}
    2\\
    1
\end{bmatrix}\nleq \bm{0}
$, so the LP is not unbounded.

5.
$
\begin{bmatrix}
    x_1\\
    x_4
\end{bmatrix}
=
\begin{bmatrix}
    2\\
    5
\end{bmatrix}-t
\begin{bmatrix}
    2\\
    1
\end{bmatrix}\ge 0
$
so
\[ t=\min \left\{\frac{2}{2},\frac{5}{1} \right\} \]
Minimum is attained at index $ 1 $ ($ x_1 $). Let $ r=1 $ be the index which attains the smallest value of $ t $.

6. Let $ 1 $ be the $ 1^{st} $ basis element.

7. Set $ B:=\{1,4\}\cup \{2\}\setminus\{1\}=\{2,4\} $

\myuline{Iteration 2}

Useful values computed:
\[ A_B=
\begin{bmatrix}
    2 & 0\\
    1 & 1\\
\end{bmatrix} \Rightarrow
A_B^{-1}=
\begin{bmatrix}
    \nicefrac{1}{2} & 0\\
    -\nicefrac{1}{2} & 1\\
\end{bmatrix} \]
\[ \bm{y}^\top=\bm{c_B}^\top A_B^{-1}=
\begin{bmatrix}
    3 & 0
\end{bmatrix}
\begin{bmatrix}
    \nicefrac{1}{2} & 0\\
    -\nicefrac{1}{2} & 1\\
\end{bmatrix}=
\begin{bmatrix}
    \nicefrac{3}{2} \\
    0
\end{bmatrix}\]

1. Canonical form determined by $ B=\{2,4\} $ is
\[ \max 
\begin{bmatrix}
    -\nicefrac{3}{2} & 0 & 4 & 0
\end{bmatrix} + 3\]
subject to
\[ 
\begin{bmatrix}
    \nicefrac{1}{2} & 1 & -1 & 0\\
    -\nicefrac{1}{2} & 0 & 4 & 1
\end{bmatrix}\bm{x}=
\begin{bmatrix}
    1\\
    4
\end{bmatrix}
\]
\[ \bm{x}\ge \bm{0} \]

Let $ \bm{\bar{x}}:=(0,1,0,4)^\top $ be the basic feasible solution.

2. $ c_{\{1,3\}} \nleq \bm{0} $, so $ \bm{\bar{x}} $ is not optimal.

3. Using Bland's Rule we choose $ k=3\in N $ which enters the basis
as $ c_3\ge 0 $.

4.
$ 
a_3= \begin{bmatrix}
    -1\\
    4
\end{bmatrix}\nleq \bm{0}
$,
so the LP is not unbounded.

5.
$
\begin{bmatrix}
    x_2\\
    x_4
\end{bmatrix}
=
\begin{bmatrix}
    1\\
    4
\end{bmatrix}-t
\begin{bmatrix}
    -1\\
    4
\end{bmatrix}\ge 0
$
so
\[ t=\min \left\{\_,\frac{4}{4} \right\} \]
Minimum is attained at index $ 2 $ ($ x_4 $). Let $ r=2 $ be the index which attains the smallest value of $ t $.

6. Let $ 4 $ be the $ 2^{nd} $ basis element.

7. Set $ B:=\{1,4\}\cup \{3\}\setminus\{4\}=\{2,3\} $

\myuline{Iteration 3}

Useful values computed:
\[ A_B=
\begin{bmatrix}
    2 & -2\\
    1 & 3\\
\end{bmatrix} \Rightarrow
A_B^{-1}=
\begin{bmatrix}
    \nicefrac{3}{8} & \nicefrac{1}{4} \\
    -\nicefrac{1}{8} & \nicefrac{1}{4} 
\end{bmatrix} \]
\[ \bm{y}^\top=\bm{c_B}^\top A_B^{-1}=
\begin{bmatrix}
    3 & 1
\end{bmatrix}
\begin{bmatrix}
    \nicefrac{3}{8} & \nicefrac{1}{4} \\
    -\nicefrac{1}{8} & \nicefrac{1}{4} 
\end{bmatrix}=
\begin{bmatrix}
    1 & 1
\end{bmatrix}\]

1. Canonical form determined by $ B=\{2,3\} $ is
\[ \max
\begin{bmatrix}
    -1 & 0 & 0 & -1
\end{bmatrix} + 7\]

subject to
\[ 
\begin{bmatrix}
    \nicefrac{3}{8} & 1 & 0 & \nicefrac{1}{4} \\
    -\nicefrac{1}{8} & 0 & 1 & \nicefrac{1}{4}
\end{bmatrix}\bm{x}=
\begin{bmatrix}
    2\\
    1
\end{bmatrix}\]
\[ \bm{x}\ge \bm{0} \]

Let $ \bm{\bar{x}}:=(0,2,1,0)^\top $ be the basic feasible solution.

2. $ c_{\{1,4\}}\le \bm{0} $, stop $ \bm{\bar{x}} $ is optimal.

The certificate of optimality is $ \bm{\bar{y}}=(1,1)^\top $.

To verify that $ \bm{\bar{y}}=(1,1)^\top $ is the certificate of optimality.
We compute
\[ A ^\top \bm{\bar{y}}=
\begin{bmatrix}
    1\\
    3\\
    1\\
    1
\end{bmatrix}\ge
\begin{bmatrix}
    0\\
    3\\
    1\\
    0
\end{bmatrix}=\bm{c}\]
and
\[ \bm{c}^\top \bm{\bar{x}}=
\begin{bmatrix}
    0 & 3 & 1 & 0
\end{bmatrix}
\begin{bmatrix}
    0\\
    2\\
    1\\
    0
\end{bmatrix}
=
7
=
\begin{bmatrix}
    1 & 1
\end{bmatrix}
\begin{bmatrix}
    2\\
    5
\end{bmatrix}
=
\bm{\bar{y}}^\top \bm{b} \]

\begin{remark}
    This was obviously not done in class (in fact it's a textbook question!).
    It can be verified that $ \bm{\bar{y}} $ is indeed the certificate of optimality by
    using the summary of outcomes as seen above.
\end{remark}