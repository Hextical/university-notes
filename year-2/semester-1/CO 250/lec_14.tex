\makeheading{2019-10-31}
Last lecture, we defined the dual of LPs in SIF and found the dual of $ (P_1) $.
\[ (P_1):\, \max \{\bm{c}^\top \bm{x}: A \bm{x}\leqslant \bm{b},\,\bm{x}\geqslant  \bm{0}\}\]
\[ (D_1):\, \min\{ \bm{b}^\top \bm{y}: A ^\top \bm{y}\geqslant  \bm{c},\,\bm{y}\geqslant  \bm{0}\}\]

\begin{exbox}
    \begin{example}[Directly Writing the Dual of an LP]
        Suppose $ A\in\mathbb{R}^{3\times 4} $.

        (P)
        \[ \max 10x_1+20x_2+30x_3+40x_4 \]
        subject to
        \[ \begin{bmatrix}
                \bm{a_1}^\top \\
                \bm{a_2}^\top \\
                \bm{a_3}^\top
            \end{bmatrix}
            \begin{bmatrix}
                x_1 \\
                x_2 \\
                x_3 \\
                x_4
            \end{bmatrix}
            \begin{matrix}
                \le \\
                =   \\
                \geqslant
            \end{matrix}
            \begin{bmatrix}
                1 \\
                2 \\
                3
            \end{bmatrix}\]
        \[ x_1\geqslant  0, x_2\geqslant  0, x_3\leqslant 0, x_4 \text{ free} \]
        where $ \bm{a_1},\bm{a_2},\bm{a_3}\in \mathbb{R}^{4} $.

        (D)
        \[ \min \begin{bmatrix}
                1 & 2 & 3
            \end{bmatrix}
            \begin{bmatrix}
                y_1 \\
                y_2 \\
                y_3
            \end{bmatrix}\]
        subject to
        \[
            \begin{bmatrix}
                \bm{a_1} & \bm{a_2} & \bm{a_3}
            \end{bmatrix}
            \begin{bmatrix}
                y_1 \\
                y_2 \\
                y_3
            \end{bmatrix}
            \begin{matrix}
                \geqslant \\
                \geqslant \\
                \le       \\
                =
            \end{matrix}
            \begin{bmatrix}
                10 \\
                20 \\
                30 \\
                40
            \end{bmatrix}\]
        \[ y_1\geqslant  0, y_2 \text{ free}, y_3\leqslant 0 \]
        Dual of the dual is the original problem, the primal.

        Since:
        \begin{enumerate}
            \item constraint 1 in (P) is $\leqslant$, then $y_1 \geqslant  0$
            \item constraint 2 in (P) is $=$, then $y_2$ free
            \item constraint 3 in (P) is $\geqslant $, then $y_3 \leqslant 0$
            \item $x_1,x_2\geqslant  0$, then constraint 1, 2 in (D) is $\geqslant $
            \item $x_3\leqslant 0$, then constraint 3 in (D) is $\le$
            \item $x_4$ free, then constraint 4 in (D) is $=$
        \end{enumerate}
    \end{example}
\end{exbox}

\section{Weak Duality}
\begin{thmbox}
    \begin{theorem}[Weak Duality - Special Form]
        Consider (P)
        \[ \max \{\bm{c}^{\top} \bm{x}: A \bm{x} \leqslant  \bm{b},\, \bm{x} \geqslant qslant  \bm{0}\} \]
        and (P)'s dual (D)
        \[ \min \{\bm{b}^{\top} \bm{y}: A^{\top} \bm{y} \geqslant qslant  \bm{c},\, \bm{y} \geqslant qslant  \bm{0}\}\]
        Let $ \bm{\bar{x}} $ be a feasible solution for (P) and $ \bm{\bar{y}} $
        be a feasible solution for (D). Then
        \begin{enumerate}[(1)]
            \item $ \bm{c}^\top \bm{\bar{x}}\leqslant \bm{b}^\top \bm{\bar{y}} $
            \item if $ \bm{c}^\top \bm{\bar{x}}=\bm{b}^\top \bm{\bar{y}} $, then
                  $ \bm{\bar{x}} $ is an optimal solution for (P).
        \end{enumerate}
    \end{theorem}
\end{thmbox}

\begin{proof}
    Let $ \bm{\bar{x}} $ be a feasible solution of (P) and let $ \bm{\bar{y}} $
    be a feasible solution of (D). Then
    \begin{align*}
        \bm{c}^\top \bm{\bar{x}}
         & \leqslant (\bm{\bar{y}}^\top A)\bm{\bar{x}} \\
         & = \bm{\bar{y}}^\top (A\bm{\bar{x}})         \\
         & \leqslant \bm{\bar{y}}^\top \bm{b}
    \end{align*}
    If $ \bm{c}^\top \bm{\bar{x}}=\bm{b}^\top \bm{\bar{y}} $ it follows
    that $ \bm{\bar{x}} $ is optimal for (P).
\end{proof}

\begin{thmbox}
    \begin{theorem}
        Let (P) and (D) be a pair of primal-dual LPs. Then
        \begin{enumerate}[(1)]
            \item if (P) is unbounded, then (D) is infeasible
            \item if (D) is unbounded, then (P) is infeasible
            \item if (P) and (D) are both feasible, then they both
                  have optimal solutions
        \end{enumerate}
    \end{theorem}
\end{thmbox}

\begin{proof}
    (1) We prove the contrapositive, that is we prove:
    [(D) feasible $ \implies $ (P) not unbounded]

    Suppose that (D) is feasible. By Weak Duality theorem, we know that
    $ \bm{b}^\top \bm{y} $ is an upper bound on (P). Thus, (P) is not
    unbounded.

    (2) We prove the contrapositive, that is we prove:
    [(P) feasible $ \implies $ (D) not unbounded]

    Suppose that (P) is feasible. By Weak Duality theorem, we know that
    $ \bm{c}^\top \bm{x} $ is a lower bound on (D). Thus, (D) is not
    unbounded.

    (3) Assume (P) and (D) are both feasible. By (1) and (2) we know that
    (P) and (D) are both not unbounded. By Fundamental Theorem of Linear
    Programming, we know that exactly one of the following holds
    from (I)-(III):
    \begin{enumerate}[(I)]
        \item (P) and (D) are infeasible
        \item (P) and (D) are unbounded
        \item (P) and (D) both have optimal solutions
    \end{enumerate}
    Clearly, we know that (I) and (II) both do not hold in this part of the proof,
    thus (P) and (D) both have optimal solutions.
\end{proof}

\section{Strong Duality}
\begin{thmbox}
    \begin{theorem}[Strong Duality]
        Let (P) and (D) be a pair of primal-dual LPs. Then
        \begin{enumerate}[(1)]
            \item If there exists an optimal solution $ \bm{\bar{x}} $ of (P), then there exists an optimal solution $ \bm{\bar{y}} $ of (D).
            \item The value of $ \bm{\bar{x}} $ in (P) equals the value of $ \bm{\bar{y}} $ in (D).
        \end{enumerate}
    \end{theorem}
\end{thmbox}

\section{A Geometric Characterization of Optimality}

\subsection{Complementary Slackness}
Recall our proof of Weak Duality. Then for LPs in SEF:
$ \bm{\bar{x}}, \bm{\bar{y}} $ are feasible in (P) and (D) respectively.
$ \bm{\bar{x}} $ is optimal in (P), $ \bm{\bar{y}} $ is optimal in
(D)  if and only if
\[ \bm{c} ^\top \bm{\bar{x}}=(A^\top \bm{\bar{y}})^\top \bm{\bar{x}}=\bm{\bar{y}}^\top(A \bm{\bar{x}})=
    \bm{\bar{y}}^\top \bm{b} \]
The first equality came from $ (A ^\top \bm{y})^\top=\bm{c}^\top $, and the
last equality came from $ A \bm{\bar{x}}=\bm{b} $ (check Weak Duality Theorem -
Special Form).
That is, if and only if
\[ \bm{\bar{x}}^\top(A ^\top \bm{\bar{y}}-\bm{c})=0 \]
and
\[ \bm{\bar{y}}^\top(\bm{b}-A \bm{\bar{x}})=0 \]
That is, if and only if $ \forall j\in \{1,\ldots,n\} $ either
$ x_j=0 $ or $ (A ^\top \bm{\bar{y}}-\bm{c})_j=0 $ possibly
both, and $ \forall i\in \{1,\ldots,m\} $ either $ y_i=0 $
or $ (\bm{b}-A \bm{\bar{x}})_i=0 $ possibly both. We call these
the Complementary Slackness Conditions (CS).

\begin{thmbox}
    \begin{theorem}[Complementary Slackness]
        Let (P) and (D) be an arbitrary primal-dual pair. Let
        $ \bm{\bar{x}} $ be a feasible solution to (P) and let
        $ \bm{\bar{y}} $ be a feasible solution to (D). Then,
        $ \bm{\bar{x}} $ is an optimal solution to (P) and
        $ \bm{\bar{y}} $ is an optimal solution to (D) if and only if
        the complementary slackness conditions hold.
    \end{theorem}
\end{thmbox}

\begin{exbox}
    \begin{example}[Complementary Slackness]
        (P)
        \[ \max
            \begin{bmatrix}
                -2 & -1 & 0
            \end{bmatrix}\bm{x} \]
        subject to
        \[
            \begin{bmatrix}
                1  & 3 & 2 \\
                -1 & 4 & 2
            \end{bmatrix}\bm{x}
            \begin{matrix}
                \geqslant \\
                \le
            \end{matrix}
            \begin{bmatrix}
                5 \\
                7
            \end{bmatrix}\]
        \[ x_1\leqslant 0,\,x_2\geqslant  0,\,x_3\text{ free} \]
        \begin{enumerate}[(1)]
            \item Write the dual (D) of (P)
            \item Write the complementary slackness (CS) conditions for (P) and (D)
            \item Use weak duality to prove that $ \bm{\bar{x}} $ is optimal for (P)
                  and $ \bm{\bar{y}} $ is optimal for (D)
            \item Use CS to prove that $ \bm{\bar{x}} $ is optimal for (P) and
                  $ \bm{\bar{y}} $ is optimal for (D)
        \end{enumerate}
        and
        \[ \bm{\bar{x}}=(-1,0,3)^\top\qquad \bm{\bar{y}}=(-1,1)^\top \]
        \textbf{Solution.}

        (1)

        (D)
        \[ \min
            \begin{bmatrix}
                5 & 7
            \end{bmatrix} \]
        subject to
        \[
            \begin{bmatrix}
                1 & -1 \\
                3 & 4  \\
                2 & 2
            \end{bmatrix}\bm{y}
            \begin{matrix}
                \geqslant \\
                \le       \\
                =
            \end{matrix}
            \begin{bmatrix}
                -2 \\
                -1 \\
                0
            \end{bmatrix}\]
        \[ y_1\leqslant 0,\,y_2\geqslant  0,\,y_3\text{ free} \]

        (2)
        \begin{itemize}
            \item $ x_1=0 $ OR $ \boxed{y_1-y_2=-2} $
            \item $ \boxed{x_2=0} $ OR $ 3y_1+4y_2=-1 $
            \item $ y_1=0 $ OR $ \boxed{x_1+3x_2+2x_3=5} $
            \item $ y_2=0 $ OR $ \boxed{-x_1+4x_2+2x_3=7} $
        \end{itemize}

        (3) Verify that $ \bm{\bar{x}} $ and $ \bm{\bar{y}} $ are
        feasible for (P) and (D), and check
        $ \bm{c}^\top \bm{\bar{x}}=\bm{b}^\top \bm{\bar{y}} $.

        (4) By Complementary Slackness Theorem, this is trivially true as seen boxed
        above.
    \end{example}
\end{exbox}

\subsection{Geometry}
(P) \[ \max 2x_1+x_2 \]
subject to
\[ \begin{bmatrix}
        1  & 1  \\
        -1 & 0  \\
        0  & -1
    \end{bmatrix}\bm{x}\le
    \begin{bmatrix}
        4 \\
        0 \\
        0
    \end{bmatrix} \]

(D)
\[ \min 4y_1 \]
subject to
\[ \begin{bmatrix}
        1 & -1 & 0  \\
        1 & 0  & -1
    \end{bmatrix}
    \bm{y}=
    \begin{bmatrix}
        2 \\
        1
    \end{bmatrix} \]
\[ y_1,\,y_2,\,y_3\geqslant  0 \]

Let $ \bm{\bar{x}} $ be a feasible solution of (P) and $ J(\bm{\bar{x}}) $
denote the indices of the tight constraints at $ \bm{\bar{x}} $.

Then, $ \bm{\bar{x}} $ is optimal in (P) if and only if $ c $ is a
non-negative linear combination of tights rows ($ A^= $).

Suppose $ \bm{\bar{x}}=(4,0)^\top $. Then $ A^= =\begin{bmatrix}
        1 & 1  \\
        0 & -1
    \end{bmatrix} $. That is, $ \bm{\bar{x}} $ is tight at $ \row_1(A) $
and $ \row_3(A) $.

$ J(\bm{\bar{x}}):=\{1,3\} $, so $ \bm{\bar{x}} $
is optimal if and only if $ \exists \, \bar{y}_1,\,\bar{y}_2 $ such that
\[ \bm{c}^\top=\bar{y}_1\left[ \row_1(A) \right]+\bar{y}_2 \left[ \row_3(A) \right] \]

