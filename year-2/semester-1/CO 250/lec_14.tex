\section{2019-10-31}
Last lecture, we defined the dual of LPs in SIF and found the dual of $ (P_1) $.
\[ (P_1):\, \max \{\bm{c}^\top \bm{x}: A \bm{x}\le \bm{b},\,\bm{x}\ge \bm{0}\}\]
\[ (D_1):\, \min\{ \bm{b}^\top \bm{y}: A ^\top \bm{y}\ge \bm{c},\,\bm{y}\ge \bm{0}\}\]

\subsection{How we do directly write down the dual of an LP?}
Suppose $ A\in\mathbb{R}^{3\times 4} $.

(P)
\[ \max 10x_1+20x_2+30x_3+40x_4 \]
subject to
\[ \begin{bmatrix}
    \bm{a_1}^\top\\
    \bm{a_2}^\top\\
    \bm{a_3}^\top
\end{bmatrix}
\begin{bmatrix}
    x_1\\
    x_2\\
    x_3\\
    x_4
\end{bmatrix}
\begin{matrix}
    \le\\
    =\\
    \ge
\end{matrix}
\begin{bmatrix}
    1\\
    2\\
    3
\end{bmatrix}\]
\[ x_1\ge 0, x_2\ge 0, x_3\le 0, x_4 \text{ free} \]
where $ \bm{a_1},\bm{a_2},\bm{a_3}\in \mathbb{R}^{4} $.

(D)
\[ \min \begin{bmatrix}
    1 & 2 & 3
\end{bmatrix}
\begin{bmatrix}
    y_1\\
    y_2\\
    y_3
\end{bmatrix}\]
subject to
\[ 
\begin{bmatrix}
    \bm{a_1} & \bm{a_2} & \bm{a_3}
\end{bmatrix}
\begin{bmatrix}
    y_1\\
    y_2\\
    y_3
\end{bmatrix}
\begin{matrix}
    \ge\\
    \ge\\
    \le\\
    =
\end{matrix}
\begin{bmatrix}
    10\\
    20\\
    30\\
    40
\end{bmatrix}\]
\[ y_1\ge 0, y_2 \text{ free}, y_3\le 0 \]
Dual of the dual is the original problem, the primal.

Since:
\begin{enumerate}
    \item constraint 1 in (P) is $\le$, then $y_1 \ge 0$
    \item constraint 2 in (P) is $=$, then $y_2$ free
    \item constraint 3 in (P) is $\ge$, then $y_3 \le 0$
    \item $x_1,x_2\ge 0$, then constraint 1, 2 in (D) is $\ge$
    \item $x_3\le 0$, then constraint 3 in (D) is $\le$
    \item $x_4$ free, then constraint 4 in (D) is $=$
\end{enumerate}


\begin{thmbox}
    \subsection{Theorem (Weak Duality - Special Form)}
    Consider (P)
    \[ \max \{\bm{c}^{\top} \bm{x}: A \bm{x} \geq \bm{b},\, \bm{x} \geq \bm{0}\} \]
    and (P)'s dual (D)
    \[ \min \{\bm{b}^{\top} \bm{y}: A^{\top} \bm{y} \leq \bm{c},\, \bm{y} \geq \bm{0}\}\]
    Let $ \bm{\bar{x}} $ be a feasible solution for (P) and $ \bm{\bar{y}} $
    be a feasible solution for (D). Then
    \begin{enumerate}[(1)]
        \item $ \bm{c}^\top \bm{\bar{x}}\ge \bm{b}^\top \bm{\bar{y}} $
        \item if $ \bm{c}^\top \bm{\bar{x}}=\bm{b}^\top \bm{\bar{y}} $, then
        $ \bm{\bar{x}} $ is an optimal solution for (P).
    \end{enumerate}
\end{thmbox}
\begin{proof}
    Let $ \bm{\bar{x}} $ be a feasible solution of (P) and let $ \bm{\bar{y}} $
    be a feasible solution of (D). Then
    \begin{align*}
        \bm{b}^\top \bm{\bar{y}}&=\bm{\bar{y}}^\top \bm{b}\\
        &\le \bm{\bar{y}}^\top(A \bm{\bar{x}})\\
        &=(\bm{\bar{y}}^\top A) \bm{\bar{x}}\\
        &=(A ^\top \bm{\bar{y}})^\top \bm{\bar{x}}\\
        &\le \bm{c}^\top \bm{\bar{x}} 
    \end{align*}
    If $ \bm{c}^\top \bm{\bar{x}}=\bm{b}^\top \bm{\bar{y}} $ it follows
    that $ \bm{\bar{x}} $ is optimal for (P).
\end{proof}

\begin{thmbox}
    \subsection{Corollary}
    Let (P) and (D) be a pair of primal-dual LPs. Then
    \begin{enumerate}[(1)]
        \item if (P) is unbounded, then (D) is infeasible
        \item if (D) is unbounded, then (P) is infeasible
        \item if (P) and (D) are both feasible, then they both
        have optimal solutions
    \end{enumerate}
\end{thmbox}

\begin{thmbox}
    \subsection{Theorem (Strong Duality Theorem)}
    Let (P) and (D) be a pair of primal-dual LPs. Then
    \begin{enumerate}[(1)]
        \item If there exists an optimal solution $ \bm{\bar{x}} $ of (P), then there exists an optimal solution $ \bm{\bar{y}} $ of (D).
        \item The value of $ \bm{\bar{x}} $ in (P) equals the value of $ \bm{\bar{y}} $ in (D).
    \end{enumerate}
\end{thmbox}

\subsection{Complementary Slackness}
Recall our proof of Weak Duality. Then for LPs in SEF: 
$ \bm{\bar{x}}, \bm{\bar{y}} $ are feasible in (P) and (D) respectively.
$ \bm{\bar{x}} $ is optimal in (P), $ \bm{\bar{y}} $ is optimal in
(D)  if and only if
\[ \bm{c} ^\top \bm{\bar{x}}=(A^\top \bm{\bar{y}})^\top \bm{\bar{x}}=\bm{\bar{y}}^\top(A \bm{\bar{x}})=
\bm{\bar{y}}^\top \bm{b} \]
The first equality came from $ (A ^\top \bm{y})^\top=\bm{c}^\top $, and the
last equality came from $ A \bm{\bar{x}}=\bm{b} $ (check Weak Duality Theorem -
Special Form).
That is, if and only if
\[ \bm{\bar{x}}^\top(A ^\top \bm{\bar{y}}-\bm{c})=0 \]
and
\[ \bm{\bar{y}}^\top(\bm{b}-A \bm{\bar{x}})=0 \]
That is, if and only if $ \forall j\in \{1,\ldots,n\} $ either
$ x_j=0 $ or $ (A ^\top \bm{\bar{y}}-\bm{c})_j=0 $ possibly
both, and $ \forall i\in \{1,\ldots,m\} $ either $ y_i=0 $
or $ (\bm{b}-A \bm{\bar{x}})_i=0 $ possibly both. We call these
the Complementary Slackness Conditions (CS).

\begin{thmbox}
    \subsection{Theorem (Complementary Slackness Theorem)}
    Let (P) and (D) be an arbitrary primal-dual pair. Let
    $ \bm{\bar{x}} $ be a feasible solution to (P) and let
    $ \bm{\bar{y}} $ be a feasible solution to (D). Then,
    $ \bm{\bar{x}} $ is an optimal solution to (P) and
    $ \bm{\bar{y}} $ is an optimal solution to (D) if and only if
    the complementary slackness conditions hold.
\end{thmbox}

\subsection{Example (Complementary Slackness)}
(P)
\[ \max
\begin{bmatrix}
    -2 & -1 & 0
\end{bmatrix}\bm{x} \]
subject to
\[
\begin{bmatrix}
    1 & 3 & 2\\
    -1 & 4 & 2
\end{bmatrix}\bm{x}
\begin{matrix}
    \ge\\
    \le
\end{matrix}
\begin{bmatrix}
    5\\
    7
\end{bmatrix}\]
\[ x_1\le 0,\,x_2\ge 0,\,x_3\text{ free} \]
\begin{enumerate}[(1)]
    \item Write the dual (D) of (P)
    \item Write the complementary slackness (CS) conditions for (P) and (D)
    \item Use weak duality to prove that $ \bm{\bar{x}} $ is optimal for (P)
    and $ \bm{\bar{y}} $ is optimal for (D)
    \item Use CS to prove that $ \bm{\bar{x}} $ is optimal for (P) and
    $ \bm{\bar{y}} $ is optimal for (D)
\end{enumerate}
and
\[ \bm{\bar{x}}=(-1,0,3)^\top\qquad \bm{\bar{y}}=(-1,1)^\top \]
\emph{Solution.}

(1)

(D)
\[ \min
\begin{bmatrix}
    5 & 7
\end{bmatrix} \]
subject to
\[
\begin{bmatrix}
    1 & -1\\
    3 & 4\\
    2 & 2
\end{bmatrix}\bm{y}
\begin{matrix}
    \ge\\
    \le\\
    =
\end{matrix}
\begin{bmatrix}
    -2\\
    -1\\
    0
\end{bmatrix}\]
\[ y_1\le 0,\,y_2\ge 0,\,y_3\text{ free} \]

(2)
\begin{itemize}
    \item $ x_1=0 $ OR $ \boxed{y_1-y_2=-2} $
    \item $ \boxed{x_2=0} $ OR $ 3y_1+4y_2=-1 $
    \item $ y_1=0 $ OR $ \boxed{x_1+3x_2+2x_3=5} $
    \item $ y_2=0 $ OR $ \boxed{-x_1+4x_2+2x_3=7} $
\end{itemize}

(3) Verify that $ \bm{c}^\top \bm{\bar{x}}=\bm{b}^\top \bm{\bar{y}} $.

(4) By Complementary Slackness Theorem, this is trivially true as seen boxed
above.