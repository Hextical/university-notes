\makeheading{2019-09-10}
Recall the family of LP problems:
\[\max \{\bm{c}^\top \bm{x} : A\bm{x}\leqslant \bm{b},\,\bm{x}\geqslant  \bm{0}\} \]
An assignment of values to all variables such that every
constraint is satisfied is called a feasible solution.
A feasible region is the set of all feasible solutions.
An optimal solution is a feasible solution which has the best
possible objective value among all feasible solutions.
Note that an optimization problem may have many optimal
solutions, but it may have one optimal value.

\begin{exbox}
    \begin{example}[Refer to 1.1]
        Suppose an entrepreneur offers at most 500 machine hours/week
        (rental) at \$2.5/hour. Can we incorporate this new situation
        into our mathematical model? Can it still be a LP? Yes.
        $x_3:=$ the number of machine hours rented from the business
        person per week.
        \[\max 10x_1+15x_2-2.5x_3\]
        subject to
        \begin{align*}
            2x_1+x_2\leqslant 1600+x_3 \\
            x_1+3x_2\leqslant 1200     \\
            x_3\leqslant 500           \\
            x_1,x_2,x_3\geqslant  0
        \end{align*}
    \end{example}
\end{exbox}
\begin{exbox}
    \begin{example}[Constraints on Ratios, Percentages and Proportions]
        Suppose we are required to manufacture at least 10 tables and
        80 chairs per week. Also
        \[\nicefrac{\nicefrac{\text{\#of tables manufactured}}{\text{week}}}
            {\nicefrac{\text{\#of chairs manufactured}}{\text{week}}}
            \geqslant  6\]
        \[
            \left\{\begin{array}{r}
                x_1\geqslant  10 \\
                x_2\geqslant  80 \\
                \nicefrac{x_2}{x_1}\geqslant  6
            \end{array}\right\}
            \iff
            \left\{\begin{array}{r}
                x_1\geqslant  10 \\
                x_2\geqslant  80 \\
                x_2\geqslant  6x_1
            \end{array}\right\}
        \]
        In general suppose $f,g$ are affine functions
        \[
            b_1\leqslant \nicefrac{f(\bm{x})}{g(\bm{x})}\leqslant b_2
        \]
        provided that $g(\bm{x})>0$ for every feasible solution $\bm{x}$ we
        can equivalently write
        \begin{align*}
            f(\bm{x})\leqslant b_2 g(\bm{x}) \\
            f(\bm{x})\geqslant  b_1 g(\bm{x})
        \end{align*}
    \end{example}
\end{exbox}


\begin{exbox}
    \begin{example}[Multi-period, Multi-stage optimization problems]
        Consider planning for multiple periods where in each period we
        want to decide how much to produce, how much to keep in stock
        (inventory) for the upcoming periods.

        \textbf{Variables}

        For all $ i\in \{1,\ldots,T\} $, where $ T $ is the last period, we have:
        \begin{itemize}
            \item $ s_i:= $ the amount of units sold in period $ i $
            \item $ p_i:= $ the amount of units purchased/manufactured of period $ i $
            \item $ t_i:= $ the amount of units in stock at the end of period $ i $
            \item $ t_0:= $ the amount of units in stock at the beginning of the first period.
        \end{itemize}
        \textbf{Key Constraints}
        \begin{align*}
            p_i+t_{i-1}=s_{i}+t_i \qquad\qquad \forall i\in \{1,\dots,T\} \\
            p_t,s_t,d_t\geqslant  0 \qquad\qquad \forall i\in \{1,\dots,T\}
        \end{align*}
    \end{example}
\end{exbox}

\begin{remark}
    Typically we have additional constraints on $s_i,\,p_i,\,t_i,\,t_0$.
\end{remark}

\begin{defbox}
    \begin{definition}
        An \emph{integer program} (IP) is obtained from linear program
        by requiring a non-empty subset of variables to be integers.
    \end{definition}
\end{defbox}

\begin{remark}
    If all variables are restricted to be integers we say that we
    have a \emph{Pure IP}, and if at least some variables may take
    real values, we call it a \emph{Mixed IP}.
\end{remark}
