\section{2019-09-10}
Recall the family of LP problems:
\[\max \{\bm{c}^\top \bm{x} : A\bm{x}\le \bm{b},\,\bm{x}\ge \bm{0}\} \]
An assignment of values to all variables such that every
constraint is satisfied is called a feasible solution.
A feasible region is the set of all feasible solutions.
An optimal solution is a feasible solution which has the best
possible objective value among all feasible solutions.
Note that an optimization problem may have many optimal
solutions, but it may have one optimal value.
\subsection{Example (Refer to 1.1)}
Suppose an entrepreneur offers at most 500 machine hours/week
(rental) at \$2.5/hour. Can we incorporate this new situation
into our mathematical model? Can it still be a LP? Yes.
$x_3:=$ the number of machine hours rented from the business
person per week.
\[\max 10x_1+15x_2-2.5x_3\]
subject to
\begin{align*}
    2x_1+x_2\le 1600+x_3\\
    x_1+3x_2\le 1200\\
    x_3\le 500\\
    x_1,x_2,x_3\ge 0
\end{align*}

\subsection{Example (Constraints on Ratios, Percentages and Proportions)}
Suppose we are required to manufacture at least 10 tables and
80 chairs per week. Also 
$\nicefrac{\nicefrac{\text{\#of tables manufactured}}{\text{week}}}
{\nicefrac{\text{\#of chairs manufactured}}{\text{week}}}
\ge 6$.
\[
   \left\{\begin{array}{r}
        x_1\ge 10\\
        x_2\ge 80\\
        \nicefrac{x_2}{x_1}\ge 6
    \end{array}\right\}
    \iff
    \left\{\begin{array}{r}
        x_1\ge 10\\
        x_2\ge 80\\
        x_2\ge 6x_1
    \end{array}\right\}
\]
In general suppose $f,g$ are affine functions
\[
    b_1\le \nicefrac{f(x)}{g(x)}\le b_2
\]
provided that $g(x)>0$ for every feasible solution $x$ we
can equivalently write
\begin{align*}
    f(x)\le b_2 g(x)\\
    f(x)\ge b_1 g(x)
\end{align*}

\subsection{Example (Multi-period, Multi-stage optimization problems)}
Consider planning for multiple periods where in each period we
want to decide how much to produce, how much to keep in stock
(inventory) for the upcoming periods. Suppose we have just one period
(WLOG),
\begin{itemize}
    \item $d_t:=$ the demand for the end of period $t$ in \# of units (given)
    \item $s_t:=$ the \# of units of products in stock at beginning of period $t$
    \item $p_t:=$ the \# of units of products manufactured at period $t$
\end{itemize}
\textbf{Key constraints}
\begin{align*}
    p_t+s_t=d_t+s_{t+1} \qquad\qquad \forall t\in \{0,\dots,T\}\\
    p_t,s_t,d_t\ge 0 \qquad\qquad \forall t\in \{0,\dots,T\}
\end{align*}
\begin{remark}
    Typically we have additional constraints on $s_0$ and $s_{T+1}$.
\end{remark}

\begin{defbox}
    \subsection{Definition (Integer Program)}
    An \emph{integer program} (IP) is obtained from linear program 
    by requiring a non-empty subset of variables to be integers.
\end{defbox}

\begin{remark}
    If all variables are restricted to be integers $\rightarrow$ Pure IP,
    and if at least some variables may take real values $\rightarrow$ Mixed IP
\end{remark}
