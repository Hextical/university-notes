\makeheading{2019-10-01}
Let $A\in \mathbb{R}^{m\times n}, \bm{b}\in\mathbb{R}^m, \bm{c}\in\mathbb{R}^n$.
Consider (P)
\[\max \bm{c}^\top \bm{x}\]
subject to
\[ A\bm{x}=\bm{b} \]
\[ \bm{x}\geqslant \bm{0} \]
Suppose we are given
$\bm{\tilde{x}}\in\mathbb{R}^n$ such that, $A\bm{\tilde{x}}=\bm{b}$,
$\bm{\tilde{x}}\geqslant \bm{0}$
and
$\bm{y}\in\mathbb{R}^m$ such that $A^\top \bm{y}\geqslant \bm{c}$,
$\bm{y}^\top \bm{b}=\bm{c}^\top \bm{\bar{x}}$ with objective function
value $=\bm{c}^\top \bm{\bar{x}}$.


Computing $\bm{c}^\top \bm{\bar{x}}$ we get
\begin{align*}
    \bm{c}^\top \bm{\bar{x}} & =\bm{y}^\top \bm{b}                   \\
                             & =\bm{y}^\top (A\bm{\tilde{x}})        \\
                             & =
    \underbrace{(\bm{y}^\top A)}_{\geqslant  \bm{c}^\top }
    \underbrace{\bm{\tilde{x}}}_{\geqslant  \bm{0}}                  \\
                             & \geqslant  \bm{c}^\top \bm{\tilde{x}}
\end{align*}
Since $\bm{\bar{x}}$ achieves the objective value of
$\bm{c}^\top \bm{\bar{x}}$ and for every feasible solution the objective
value is at most $\bm{c}^\top \bm{\bar{x}}$, $\bm{\bar{x}}$ is an
optimal solution of (P).

\subsection{Canonical Forms}
\begin{defbox}
    \begin{definition}
        Consider the following LP in SEF:

        (P)
        \[\max \bm{c}^\top  \bm{x}+\bar{z}\]
        subject to
        \begin{align*}
            A\bm{x}=\bm{b} \\
            \bm{x}\geqslant  \bm{0}
        \end{align*}
        We say (P) is in \emph{canonical form} for a basis $B$ of $A$ if
        \begin{enumerate}[(C1)]
            \item $A_B$ is an identity matrix
            \item $\bm{c_B}=\bm{0}$
        \end{enumerate}
    \end{definition}
\end{defbox}

Now,
\begin{align*}
    A\bm{x}
     & =\sum_{j = 1}^{n}\bm{a_j}x_j                               \\
     & =\sum_{j\in B}^{n}\bm{a_j}x_j+\sum_{j\in N}^{n}\bm{a_j}x_j \\
     & =A_B \bm{x_B}+A_N \bm{x_N}
\end{align*}
Since $B$ is a basis of $A$, $A_B$ is non-singular,
\begin{align*}
     & A\bm{x}=\bm{b}                                \\
     & \iff
    A_B^{-1}A\bm{x}=A_B^{-1}\bm{b}                   \\
     & \iff
    A_B^{-1}(A_B\bm{x_B}+A_N\bm{x_N})=A_B^{-1}\bm{b} \\
     & \iff
    (\underbrace{A_B^{-1} A_B}_{I}\bm{x_B})+
    (A_B^{-1} A_N\bm{x_N})=A_B^{-1}\bm{b}            \\
     & \iff
    \bm{x_B}=A_B^{-1}\bm{b}-(A_B^{-1} A_N\bm{x_N})
\end{align*}
Consider (C2). For any $\bm{y}:=(y_1,\dots, y_m)^\top$
the equation
\[\bm{y}^\top A\bm{x}=\bm{y}^\top \bm{b}\]
can be written as
\[0=\bm{y}^\top \bm{b}-\bm{y}^\top A\bm{x}\]
Since  this  equation  holds  for  every  feasible  solution,  we  can  add  this
constraint to the objective function which is now:
\[\max \bm{c}^\top  \bm{x}+\bar{z}+\bm{y}^\top \bm{b}-\bm{y}^\top A\bm{x}
    \implies
    \max (\bm{c}^\top -\bm{y}^\top A)\bm{x}+\bm{y}^\top \bm{b}+\bar{z}\]
Let $\bm{\bar{c}}^\top :=\bm{c}^\top -\bm{y}^\top A$. For (C2) to be satisfied we need
$\bm{\bar{c}_B}=\bm{0}$, so we need to choose $\bm{y}$ accordingly, such as
\[\bm{\bar{c}_B}^\top =\bm{c_B}^\top -\bm{y}^\top A_B=\bm{0}^\top \]
equivalently,
\[\bm{y}^\top A_B=\bm{c_B}^\top
    \implies
    \bm{y}^\top =\bm{c_B}^\top A_B^{-1}\]
We have shown the following:

\begin{thmbox}
    \begin{theorem}[Canonical Form]
        Suppose an LP
        \[ \max \{\bm{c}^\top \bm{x}+\bar{z}:A \bm{x}=\bm{b},\, \bm{x}\geqslant  \bm{0}\} \]
        and a basis $ B $ of $ A $ are given. Then
        \[\max (\bm{c}^\top -\bm{y}^\top A)\bm{x}+\bm{y}^\top \bm{b}+\bar{z}\]
        subject to
        \[ A_B^{-1}A\bm{x}=A_B^{-1}\bm{b} \]
        \[ \bm{x}\geqslant  \bm{0} \]

        where $\bm{y}^\top =\bm{c_B}^\top A_B^{-1}$, is an equivalent LP in canonical form for the
        basis $B$ of $A$.
    \end{theorem}
\end{thmbox}

The canonical form is useful because it:
\begin{itemize}
    \item allows us to simply read a basic solution
    \item gives us easy ways to move in the feasible region to improve the current basic
          feasible solution
    \item gives us a way to obtain optimality certificates if
          $\bm{c}^\top -\bm{y}^\top A\leqslant \bm{0}^\top $
\end{itemize}

\begin{exbox}
    \begin{example}[Canonical Form]
        (P)
        \[\max \begin{bmatrix}
                0 & 0 & -4 & 1 & 0
            \end{bmatrix}\bm{x}\]
        subject to
        \[
            \begin{bmatrix}
                1 & 0 & -1 & 1  & 0 \\
                0 & 1 & 2  & -1 & 0 \\
                0 & 0 & -3 & 2  & 1
            \end{bmatrix}\bm{x}
            =
            \begin{bmatrix}
                4 \\
                2 \\
                6
            \end{bmatrix} \]
        \[ \bm{x}\geqslant  \bm{0} \]

        $B:=\{1,2,5\}$ is a basis of $A$. Thus, the basic solution corresponding to the
        basis is
        \[\bm{\bar{x}}:=(4,2,0,0,6)^\top\]

        $c_3=-4$, increasing the value of $x_3$ from $0$ will decrease the objective value by $-4$ units\\
        $c_4=1$, we want to increase the value of $x_4$, so
        \[
            \begin{bmatrix}
                x_1 \\
                x_2 \\
                x_5
            \end{bmatrix}
            =
            \begin{bmatrix}
                4 \\
                2 \\
                6
            \end{bmatrix}
            -x_4
            \begin{bmatrix}
                1  \\
                -1 \\
                2
            \end{bmatrix}
            \geqslant  0
        \]
        Let $t$ denote the maximum value we can assign to $x_4$ and stay feasible.\\
        So,
        $t=\min\{\nicefrac{4}{1},\_,\nicefrac{6}{2}\}=3$
    \end{example}
\end{exbox}
