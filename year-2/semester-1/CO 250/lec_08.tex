\section{2019-10-01}
Let $A\in M_{m\times n}(\mathbb{R}), \mathbf{b}\in\mathbb{R}^m, \mathbf{c}\in\mathbb{R}^n$.
Consider (P)
\[\max \mathbf{c}^T\mathbf{x}\]
subject to
\begin{align*}
    A\mathbf{x}=\mathbf{b}\\
    \mathbf{x}\ge\mathbf{0}
\end{align*}
Suppose we are given
$\mathbf{\tilde{x}}\in\mathbb{R}^n$ such that, $A\mathbf{\tilde{x}}=\mathbf{b}$,
$\mathbf{\tilde{x}}\ge\mathbf{0}$
and
$\mathbf{y}\in\mathbb{R}^m$ such that $A^T\mathbf{y}\ge\mathbf{c}$,
$\mathbf{y}^T\mathbf{b}=\mathbf{c}^T\mathbf{\bar{x}}$ with objective function
value $=\mathbf{c}^T\mathbf{\bar{x}}$.


Computing $\mathbf{c}^T\mathbf{\bar{x}}$ we get
\begin{align*}
    \mathbf{c}^T\mathbf{\bar{x}}&=\mathbf{y}^T\mathbf{b}\\
    &=\mathbf{y}^T(A\mathbf{\tilde{x}})\\
    &=
    \underbrace{(\mathbf{y}^TA)}_{\ge \mathbf{c}^T}
    \underbrace{\mathbf{\tilde{x}}}_{\ge \mathbf{0}}\\
    &\ge \mathbf{c}^T\mathbf{\tilde{x}}
\end{align*}
Since $\mathbf{\bar{x}}$ achieves the objective value of 
$\mathbf{c}^T\mathbf{\bar{x}}$ and for every feasible solution the objective
value is at most $\mathbf{c}^T\mathbf{\bar{x}}$, $\mathbf{\bar{x}}$ is an
optimal solution of (P).

\subsection{Definition (Canonical form)}
Consider the following LP in SEF: (P)
\[\max \mathbf{c}^T \mathbf{x}+\bar{z}\]
subject to
\begin{align*}
    A\mathbf{x}=\mathbf{b}\\
    \mathbf{x}\ge \mathbf{0}
\end{align*}
We say (P) is in \emph{canonical form} for a basis $B$ of $A$ if
\begin{enumerate}[(C1)]
    \item $A_B$ is an identity matrix
    \item $\mathbf{c_B}=\mathbf{0}$
\end{enumerate}
Now,
\begin{align*}
    A\mathbf{x}
    &=\sum_{j = 1}^{n}\mathbf{a_j}x_j\\
    &=\sum_{j\in B}^{n}\mathbf{a_j}x_j+\sum_{j\in N}^{n}\mathbf{a_j}x_j\\
    &=A_B \mathbf{x_B}+A_N \mathbf{x_N}
\end{align*}
Since $B$ is a basis of $A$, $A_B$ is non-singular,
\begin{align*}
    &A\mathbf{x}=\mathbf{b}\\
    &\iff
    A_B^{-1}A\mathbf{x}=A_B^{-1}\mathbf{b}\\
    &\iff
    A_B^{-1}(A_B\mathbf{x_B}+A_N\mathbf{x_N})=A_B^{-1}\mathbf{b}\\
    &\iff
    (\underbrace{A_B^{-1} A_B}_{I}\mathbf{x_B})+
    (A_B^{-1} A_N\mathbf{x_N})=A_B^{-1}\mathbf{b}\\
    &\iff
    \mathbf{x_B}=A_B^{-1}\mathbf{b}-(A_B^{-1} A_N\mathbf{x_N})
\end{align*}
Consider (C2). For any $\mathbf{y}:=\begin{bmatrix}
    y_1\\
    \vdots\\
    y_m
\end{bmatrix}$
the equation
\[\mathbf{y}^TA\mathbf{x}=\mathbf{y}^T\mathbf{b}\]
can be written as
\[0=\mathbf{y}^T\mathbf{b}-\mathbf{y}^TA\mathbf{x}\]
Since  this  equation  holds  for  every  feasible  solution,  we  can  add  this 
constraint to the objective function of (Q). The objective function is now
\[\max \mathbf{c}^T \mathbf{x}+\bar{z}+\mathbf{y}^T\mathbf{b}-\mathbf{y}^TA\mathbf{x}
\implies
\max (\mathbf{c}^T-\mathbf{y}^TA)\mathbf{x}+\mathbf{y}^T\mathbf{b}+\bar{z}\]
Let $\mathbf{\bar{c}}^T:=\mathbf{c}^T-\mathbf{y}^TA$. For (C2) to be satisfied we need
$\mathbf{\bar{c_B}}=\mathbf{0}$. We need to choose $\mathbf{y}$ accordingly, such as
\[\mathbf{\bar{c_B}}^T=\mathbf{c_B}^T-\mathbf{y}^TA_B=\mathbf{0}^T\]
equivalently,
\[\mathbf{y}^TA_B=\mathbf{c_B}^T
\implies
A_B^T\mathbf{y}=\mathbf{c_B}
\implies
\mathbf{y}^T=\mathbf{c_B}A_B^{-1}\]
We have shown the following:

\subsection{Proposition (Canonical Form)}
\[\max (\mathbf{c}^T-\mathbf{y}^TA)\mathbf{x}+\mathbf{y}^T\mathbf{b}+\bar{z}\]
subject to
\begin{align*}
    A_B^{-1}A\mathbf{x}=A_B^{-1}\mathbf{b}\\
    \mathbf{x}\ge 0
\end{align*}
where $\mathbf{y}^T=\mathbf{c_B}A_B^{-1}$, is an equivalent LP in canonical form for the 
basis $B$ of $A$.


The canonical form is useful because it:
\begin{itemize}
    \item allows us to simply read a basic solution
    \item gives us easy ways to move in the feasible region to improve the current basic
    feasible solution
    \item gives us a way to obtain optimality certificates if
    $\mathbf{c_N}^T-\mathbf{c_B}^TA_B^{-1}A_N\le \mathbf{0}^T$
\end{itemize}

\subsection{Example (Canonical Form)}
(P)
\[\max \begin{bmatrix}
    0 & 0 & -4 & 1 & 0
\end{bmatrix}\mathbf{x}\]
subject to
\begin{align*}
    \begin{bmatrix}
        1 & 0 & -1 & 1 & 0\\
        0 & 1 & 2 & -1 & 0\\
        0 & 0 & -3 & 2 & 1
    \end{bmatrix}\mathbf{x}
    =
    \begin{bmatrix}
        4\\
        2\\
        6
    \end{bmatrix}\\
    \mathbf{x}\ge \mathbf{0}
\end{align*}


Note that $B:=\{1,2,5\}$ is a basis of $A$, so the basic solution determined by $B$ is
\[\mathbf{\bar{x}}=
    \begin{bmatrix}
        4\\
        2\\
        0\\
        0\\
        6\\
    \end{bmatrix}
\]
$c_3=-4$, increasing the value of $x_3$ from $0$ will decrease the objective value by $-4$ units\\
$c_4=1$, we want to increase the value of $x_4$, so
\[
    \begin{bmatrix}
        x_1\\
        x_2\\
        x_5
    \end{bmatrix}
    =
    \begin{bmatrix}
        4\\
        2\\
        6
    \end{bmatrix}
    -x_4
    \begin{bmatrix}
        1\\
        -1\\
        2
    \end{bmatrix}
    \ge 0
\]
Let $t$ denote the maximum value we can assign to $x_4$ and stay feasible.\\
So,
$t=\min\{\nicefrac{4}{1},\_,\nicefrac{6}{2}\}=3$

