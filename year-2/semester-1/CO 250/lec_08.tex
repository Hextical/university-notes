\section{2019-10-01}
Let $A\in \mathbb{R}^{m\times n}, \symbfit{b}\in\mathbb{R}^m, \symbfit{c}\in\mathbb{R}^n$.
Consider (P)
\[\max \symbfit{c}^\top \symbfit{x}\]
subject to
\[ A\symbfit{x}=\symbfit{b} \]
\[ \symbfit{x}\ge\mathbb{0} \]
Suppose we are given
$\symbfit{\tilde{x}}\in\mathbb{R}^n$ such that, $A\symbfit{\tilde{x}}=\symbfit{b}$,
$\symbfit{\tilde{x}}\ge\mathbb{0}$
and
$\symbfit{y}\in\mathbb{R}^m$ such that $A^\top \symbfit{y}\ge\symbfit{c}$,
$\symbfit{y}^\top \symbfit{b}=\symbfit{c}^\top \symbfit{\bar{x}}$ with objective function
value $=\symbfit{c}^\top \symbfit{\bar{x}}$.


Computing $\symbfit{c}^\top \symbfit{\bar{x}}$ we get
\begin{align*}
    \symbfit{c}^\top \symbfit{\bar{x}}&=\symbfit{y}^\top \symbfit{b}\\
    &=\symbfit{y}^\top (A\symbfit{\tilde{x}})\\
    &=
    \underbrace{(\symbfit{y}^\top A)}_{\ge \symbfit{c}^\top }
    \underbrace{\symbfit{\tilde{x}}}_{\ge \mathbb{0}}\\
    &\ge \symbfit{c}^\top \symbfit{\tilde{x}}
\end{align*}
Since $\symbfit{\bar{x}}$ achieves the objective value of 
$\symbfit{c}^\top \symbfit{\bar{x}}$ and for every feasible solution the objective
value is at most $\symbfit{c}^\top \symbfit{\bar{x}}$, $\symbfit{\bar{x}}$ is an
optimal solution of (P).


\begin{defbox}
    \subsection{Definition (Canonical form)}
    Consider the following LP in SEF: (P)
    \[\max \symbfit{c}^\top  \symbfit{x}+\bar{z}\]
    subject to
    \begin{align*}
        A\symbfit{x}=\symbfit{b}\\
        \symbfit{x}\ge \mathbb{0}
    \end{align*}
    We say (P) is in \emph{canonical form} for a basis $B$ of $A$ if
    \begin{enumerate}[(C1)]
        \item $A_B$ is an identity matrix
        \item $\symbfit{c_B}=\mathbb{0}$
    \end{enumerate}
\end{defbox}

Now,
\begin{align*}
    A\symbfit{x}
    &=\sum_{j = 1}^{n}\symbfit{a_j}x_j\\
    &=\sum_{j\in B}^{n}\symbfit{a_j}x_j+\sum_{j\in N}^{n}\symbfit{a_j}x_j\\
    &=A_B \symbfit{x_B}+A_N \symbfit{x_N}
\end{align*}
Since $B$ is a basis of $A$, $A_B$ is non-singular,
\begin{align*}
    &A\symbfit{x}=\symbfit{b}\\
    &\iff
    A_B^{-1}A\symbfit{x}=A_B^{-1}\symbfit{b}\\
    &\iff
    A_B^{-1}(A_B\symbfit{x_B}+A_N\symbfit{x_N})=A_B^{-1}\symbfit{b}\\
    &\iff
    (\underbrace{A_B^{-1} A_B}_{I}\symbfit{x_B})+
    (A_B^{-1} A_N\symbfit{x_N})=A_B^{-1}\symbfit{b}\\
    &\iff
    \symbfit{x_B}=A_B^{-1}\symbfit{b}-(A_B^{-1} A_N\symbfit{x_N})
\end{align*}
Consider (C2). For any $\symbfit{y}:=(y_1,\dots, y_m)^\top$
the equation
\[\symbfit{y}^\top A\symbfit{x}=\symbfit{y}^\top \symbfit{b}\]
can be written as
\[0=\symbfit{y}^\top \symbfit{b}-\symbfit{y}^\top A\symbfit{x}\]
Since  this  equation  holds  for  every  feasible  solution,  we  can  add  this 
constraint to the objective function of (Q). The objective function is now
\[\max \symbfit{c}^\top  \symbfit{x}+\bar{z}+\symbfit{y}^\top \symbfit{b}-\symbfit{y}^\top A\symbfit{x}
\implies
\max (\symbfit{c}^\top -\symbfit{y}^\top A)\symbfit{x}+\symbfit{y}^\top \symbfit{b}+\bar{z}\]
Let $\symbfit{\bar{c}}^\top :=\symbfit{c}^\top -\symbfit{y}^\top A$. For (C2) to be satisfied we need
$\symbfit{\bar{c}_B}=\mathbb{0}$, so we need to choose $\symbfit{y}$ accordingly, such as
\[\symbfit{\bar{c}_B}^\top =\symbfit{c_B}^\top -\symbfit{y}^\top A_B=\mathbb{0}^\top \]
equivalently,
\[\symbfit{y}^\top A_B=\symbfit{c_B}^\top 
\implies
\symbfit{y}^\top =\symbfit{c_B}^\top A_B^{-1}\]
We have shown the following:


\begin{thmbox}
    \subsection{Proposition (Canonical Form)}
    \[\max (\symbfit{c}^\top -\symbfit{y}^\top A)\symbfit{x}+\symbfit{y}^\top \symbfit{b}+\bar{z}\]
    subject to
    \begin{align*}
        A_B^{-1}A\symbfit{x}=A_B^{-1}\symbfit{b}\\
        \symbfit{x}\ge \mathbb{0}
    \end{align*}
    where $\symbfit{y}^\top =\symbfit{c_B}A_B^{-1}$, is an equivalent LP in canonical form for the 
    basis $B$ of $A$.
\end{thmbox}

The canonical form is useful because it:
\begin{itemize}
    \item allows us to simply read a basic solution
    \item gives us easy ways to move in the feasible region to improve the current basic
    feasible solution
    \item gives us a way to obtain optimality certificates if
    $\symbfit{c_N}^\top -\symbfit{c_B}^\top A_B^{-1}A_N\le \mathbb{0}^\top $
\end{itemize}

\subsection{Example (Canonical Form)}
(P)
\[\max \begin{bmatrix}
    0 & 0 & -4 & 1 & 0
\end{bmatrix}\symbfit{x}\]
subject to
\[
    \begin{bmatrix}
        1 & 0 & -1 & 1 & 0\\
        0 & 1 & 2 & -1 & 0\\
        0 & 0 & -3 & 2 & 1
    \end{bmatrix}\symbfit{x}
    =
    \begin{bmatrix}
        4\\
        2\\
        6
    \end{bmatrix} \]
\[ \symbfit{x}\ge \mathbb{0} \]

Note that $B:=\{1,2,5\}$ is a basis of $A$, so the basic solution corresponding to the
basis is
$\symbfit{\bar{x}}:=(4,2,0,0,6)^\top$.

$c_3=-4$, increasing the value of $x_3$ from $0$ will decrease the objective value by $-4$ units\\
$c_4=1$, we want to increase the value of $x_4$, so
\[
    \begin{bmatrix}
        x_1\\
        x_2\\
        x_5
    \end{bmatrix}
    =
    \begin{bmatrix}
        4\\
        2\\
        6
    \end{bmatrix}
    -x_4
    \begin{bmatrix}
        1\\
        -1\\
        2
    \end{bmatrix}
    \ge 0
\]
Let $t$ denote the maximum value we can assign to $x_4$ and stay feasible.\\
So,
$t=\min\{\nicefrac{4}{1},\_,\nicefrac{6}{2}\}=3$