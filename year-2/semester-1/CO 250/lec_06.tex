\section{2019-09-24}
\subsection{Solving LP Problems}
Let $x_1,\dots,x_n$ be all the variables in an optimization problem. Then
assignment of values to all variables such that all constraints are satisfied,
gives a \emph{feasible solution}. An optimization problem is called \emph{feasible}
if it has at least one feasible solution, otherwise it is called \emph{infeasible}.

\subsection{Example (Infeasible LP)}
(LP)
\[\max x_1+2x_2+3x_3+4x_4+5x_5\]
subject to
\begin{align*}
    &
    \begin{matrix}
    1\\
    -2
    \end{matrix}
    \underbrace{
        \begin{bmatrix}
        -3 & 2 & 7 & 1 & -7 \\
        -2 & 1 & 2 & 0 & -4
        \end{bmatrix}}_{A}
    \underbrace{\begin{bmatrix}
        x_1\\
        x_2\\
        x_3\\
        x_4\\
        x_5
    \end{bmatrix}}_{\mathbf{x}}
    =
    \underbrace{\begin{bmatrix}
        6\\
        4
    \end{bmatrix}}_{\mathbf{b}}\\
    &\mathbf{x}\ge \mathbf{0}
\end{align*}
Let $\mathbf{y}:=
\begin{bmatrix}
    1\\
    -2
\end{bmatrix}$
and consider the facts
\begin{align*}
    &A\mathbf{x}=\mathbf{b}\\
    &\implies \mathbf{y}^TA\mathbf{x}=\mathbf{y}^T\mathbf{b}\\
    &\implies \underbrace{\begin{bmatrix}
        1 & 0 & 3 & 1 & 1
    \end{bmatrix}}_{\ge \mathbf{0}^T}
    \underbrace{\mathbf{x}}_{\ge \mathbf{0}}=\underbrace{6-8}_{< 0}=-2
\end{align*}
Therefore, $\nexists$ any solution to $A\mathbf{x}=\mathbf{b}$, $\mathbf{x}\ge 0$.
Thus, the LP is infeasible.

\subsection{Proposition (Infeasibility)}
Let $A\in M_{m\times n}(R), \mathbf{b}\in\mathbb{R}^m$. Suppose
$\exists \mathbf{y}\in\mathbb{R}^m$ such that
\begin{enumerate}
    \item $\mathbf{y}^TA\ge\mathbf{0}^T$
    \item $\mathbf{y}^T\mathbf{b}<0$
\end{enumerate}
For every $\mathbf{c}\in\mathbb{R}^n$, the LP
\[\max \{\mathbf{c}^T\mathbf{x} \mid A\mathbf{x}=\mathbf{b}\text{, }
\mathbf{x}\ge\mathbf{0}\}\]
is infeasible. In particular, we call a vector $\mathbf{y}$ a \emph{certificate of infeasibility}.

\begin{proof}
    Let $A\in M_{m\times n}(R), \mathbf{b}\in\mathbb{R}^m$ and suppose
$\exists \mathbf{y}\in\mathbb{R}^m$ such that
\begin{enumerate}
    \item $\mathbf{y}^TA\ge\mathbf{0}^T$
    \item $\mathbf{y}^T\mathbf{b}<0$
\end{enumerate}
Suppose for a contradiction that $\exists\mathbf{\bar{x}}\in\mathbb{R}^n$ 
such that
\[A\mathbf{\bar{x}}=\mathbf{b} \text{, }\mathbf{\bar{x}}\ge \mathbf{0}\]
\[
    A\mathbf{\bar{x}}=\mathbf{b}
    \implies
    \underbrace{\mathbf{y}^TA}_{\ge\mathbf{0}^T}
    \underbrace{\mathbf{\bar{x}}}_{\ge\mathbf{0}}
    =\underbrace{\mathbf{y}^T\mathbf{b}}_{\nless 0}
\]
a contradiction to 2.
\end{proof}

An optimization problem is called unbounded if $\forall M\in\mathbb{R}$, there
exists a feasible solution of the optimization problem with the objective 
value strictly better than $M$.
\subsection{Example (Unbounded LP)}
\[\max 
\begin{bmatrix}
    -1 & 3 & 0 & 0 & 1
\end{bmatrix}\mathbf{x}\]
subject to
\begin{align*}
    &\begin{bmatrix}
        -1 & 3 & -1 & 1 & 0\\
        -2 & 4 & 1 & 0 & 1
    \end{bmatrix}
    \mathbf{x}
    =
    \begin{bmatrix}
        2\\
        1
    \end{bmatrix}\\
    &\mathbf{x}\ge \mathbf{0}
\end{align*}
Consider
\[\mathbf{\tilde{x}}:=
\underbrace{\begin{bmatrix}
    0\\
    0\\
    0\\
    2\\
    1  
\end{bmatrix}}_{\mathbf{x}}
+
t
\underbrace{\begin{bmatrix}
    1\\
    0\\
    0\\
    1\\
    2
\end{bmatrix}}_{\mathbf{d}} \text{, } t\ge 0
\]

\[
    A\mathbf{x}=
    \begin{bmatrix}
        2\\
        1
    \end{bmatrix}, \bar{\mathbf{x}}\ge \mathbf{0}.\text{Therefore $\bar{\mathbf{x}}$ is a feasible solution.}
\]
\[
    A\mathbf{d}=\begin{bmatrix}
        0\\
        0
    \end{bmatrix}, \mathbf{d}\ge \mathbf{0}.\\
\]

\[A\tilde{\mathbf{x}}=A(\bar{\mathbf{x}}+t\mathbf{d})=A\bar{\mathbf{x}}+t(A\mathbf{d})=
\begin{bmatrix}
    2\\
    1
\end{bmatrix}\]
\[\tilde{\mathbf{x}}=\bar{\mathbf{x}}+t\mathbf{d}\]
Therefore, $\tilde{\mathbf{x}}$ is a feasible solution $\forall t\ge 0$.


\textbf{Objective function value of $\tilde{\mathbf{x}}$:}
\[
\begin{bmatrix}
    -1 & 3 & 0 & 0 & 1
\end{bmatrix}
\left(\begin{bmatrix}
    0\\
    0\\
    0\\
    2\\
    1  
\end{bmatrix}
+
t
\begin{bmatrix}
    1\\
    0\\
    0\\
    1\\
    2
\end{bmatrix}\right)
=
1+t(-1+2)=1+t\rightarrow+\infty \text{ as }t\rightarrow+\infty\]
Therefore the LP is unbounded.

\subsection{Proposition (Unboundedness)}
Suppose $\exists \mathbf{\bar{x}}\in\mathbb{R}^n$ such that
\[A\mathbf{\bar{x}}=\mathbf{b}, \mathbf{x}\ge \mathbf{0}.\]
and $\exists\mathbf{d}\in\mathbb{R}^n$ such that
\begin{enumerate}
    \item $A\mathbf{d}=\mathbf{0}$
    \item $\mathbf{d}\ge \mathbf{0}$
    \item $\mathbf{c}^T\mathbf{d}>0$
\end{enumerate}
For every $\mathbf{c}\in\mathbb{R}^n$, the LP
\[\max \{\mathbf{c}^T\mathbf{x} \mid A\mathbf{x}=\mathbf{b}\text{, }
\mathbf{x}\ge\mathbf{0}\}\]
is unbounded. In particular, we call a pair of vectors $\mathbf{\bar{x}}$, $\mathbf{d}$ a
\emph{certificate of unboundedness}.

\begin{proof}
    Suppose there exists such $\mathbf{d}$. Consider
    \[\tilde{\mathbf{x}}=\bar{\mathbf{x}}+t\mathbf{d}, t\ge 0\]
Then,
\[A\tilde{\mathbf{x}}=
\underbrace{A\bar{\mathbf{x}}}_{\mathbf{b}}+
t\underbrace{(A\mathbf{d})}_{\mathbf{0}}=\mathbf{b}\]
Therefore $\tilde{\mathbf{x}}$ is a feasible solution of the LP, $t\ge 0$.
The objective value of the function is
\[\mathbf{c}^T\tilde{\mathbf{x}}=\mathbf{c}^T\bar{\mathbf{x}}+t
\underbrace{(\mathbf{c}^T\mathbf{d})}_{>\mathbf{0}}\rightarrow +\infty\text{ as }t\rightarrow+\infty\]
Therefore, the LP is unbounded.
\end{proof}
\begin{remark}
    If the LP is $\min$, then flip the equality for 3.
\end{remark}

\subsection{Example (Optimal LP)}
\[\max 10x_1+15x_2\]
subject to
\begin{align*}
    2x_1+x_2+x_3=1600\\
    x_1+3x_2+x_4=1200\\
    \mathbf{x}\ge \mathbf{0}
\end{align*}
Consider
\[\bar{\mathbf{x}}:=
\begin{bmatrix}
    720\\
    160\\
    0\\
    0
\end{bmatrix}\]
\[\mathbf{y}:=
\begin{bmatrix}
    3\\
    4
\end{bmatrix}
\]
$A\bar{\mathbf{x}}=\mathbf{b}$, $\bar{\mathbf{x}}\ge \mathbf{0}$. So $\bar{\mathbf{x}}$ is a feasible solution.


Also, $\mathbf{c}^T\mathbf{x}=7200+2400=9600$.
Every feasible solution satisfies
\begin{align*}
    &A\mathbf{x}=\mathbf{b}\\
    &\implies \mathbf{y}^TA\mathbf{x}=\mathbf{y}^T\mathbf{b}
\end{align*}
\[\mathbf{y}^TA=
\begin{bmatrix}
    10 & 15 & 3 & 4
\end{bmatrix}
\ge
\begin{bmatrix}
    10 & 15 & 0 & 0
\end{bmatrix}=\mathbf{c}^T\]
\[\mathbf{y}^Tb=3\times 1600+4\times 1200=9600=\mathbf{c}^T\mathbf{x}\]
Therefore $\bar{\mathbf{x}}$ is an optimal solution.
