\section{2019-09-24}
\begin{defbox}
    \subsection{Feasible and Infeasible Solutions}
    Consider an LP with variables $x_1,\dots,x_n$.  Then the
    assignment of values to all variables such that all constraints are satisfied,
    gives a \emph{feasible solution}.

    An optimization problem is called \emph{feasible} if it has at least one feasible
    solution, otherwise it is called \emph{infeasible}.
\end{defbox}

\subsection{Example (Infeasible LP)}
(LP)
\[\max x_1+2x_2+3x_3+4x_4+5x_5\]
subject to
\begin{align*}
    &
    \begin{matrix}
    1\\
    -2
    \end{matrix}
    \underbrace{
        \begin{bmatrix}
        -3 & 2 & 7 & 1 & -7 \\
        -2 & 1 & 2 & 0 & -4
        \end{bmatrix}}_{A}
    \underbrace{\begin{bmatrix}
        x_1\\
        x_2\\
        x_3\\
        x_4\\
        x_5
    \end{bmatrix}}_{\bm{x}}
    =
    \underbrace{\begin{bmatrix}
        6\\
        4
    \end{bmatrix}}_{\bm{b}}\\
    &\bm{x}\ge \bm{0}
\end{align*}
Let $\bm{y}:=(1,-2)^\top$
and consider the facts
\begin{align*}
    &A\bm{x}=\bm{b}\\
    &\implies \bm{y}^\top A\bm{x}=\bm{y}^\top \bm{b}\\
    &\implies \underbrace{\begin{bmatrix}
        1 & 0 & 3 & 1 & 1
    \end{bmatrix}}_{\ge \bm{0}^\top }
    \underbrace{\bm{x}}_{\ge \bm{0}}=\underbrace{6-8}_{< 0}=-2
\end{align*}
Therefore, since $\nexists$ any solution to $A\bm{x}=\bm{b}$, $\bm{x}\ge 0$
the LP is infeasible.


\begin{thmbox}
    \subsection{Proposition (Infeasibility)}
    If
    $\exists \bm{y}\in\mathbb{R}^m$ such that
    \begin{enumerate}[(1)]
        \item $\bm{y}^\top A\ge\bm{0}^\top $
        \item $\bm{y}^\top \bm{b}<0$
    \end{enumerate}
    For every $\bm{c}\in\mathbb{R}^n$, the LP
    \[\max \{\bm{c}^\top \bm{x} : A\bm{x}=\bm{b},\,
    \bm{x}\ge\bm{0}\}\]
    is infeasible. In particular, we call a vector $\bm{y}$ a \emph{certificate of infeasibility}.
\end{thmbox}

\begin{proof}
Suppose there exists such a $ \bm{y} $.
Suppose for a contradiction that $\exists\bm{\bar{x}}\in\mathbb{R}^n$ 
(there is a feasible solution)
such that
\[A\bm{\bar{x}}=\bm{b} \text{, }\bm{\bar{x}}\ge \bm{0}\]
\[
    A\bm{\bar{x}}=\bm{b}
    \implies
    \underbrace{\bm{y}^\top A}_{\ge\bm{0}^\top }
    \underbrace{\bm{\bar{x}}}_{\ge\bm{0}}
    =\underbrace{\bm{y}^\top \bm{b}}_{\nless 0}
\]
a contradiction to (2).
\end{proof}

An optimization problem is called unbounded if $\forall M\in\mathbb{R}$, there
exists a feasible solution of the optimization problem with the objective 
value strictly better than $M$.
\subsection{Example (Unbounded LP)}
\[\max 
\begin{bmatrix}
    -1 & 3 & 0 & 0 & 1
\end{bmatrix}\bm{x}\]
subject to
\begin{align*}
    &\begin{bmatrix}
        -1 & 3 & -1 & 1 & 0\\
        -2 & 4 & 1 & 0 & 1
    \end{bmatrix}
    \bm{x}
    =
    \begin{bmatrix}
        2\\
        1
    \end{bmatrix}\\
    &\bm{x}\ge \bm{0}
\end{align*}
Consider
\[\bm{\tilde{x}}:=
\underbrace{\begin{bmatrix}
    0\\
    0\\
    0\\
    2\\
    1  
\end{bmatrix}}_{\bm{x}}
+
t
\underbrace{\begin{bmatrix}
    1\\
    0\\
    0\\
    1\\
    2
\end{bmatrix}}_{\bm{d}} \text{, } t\ge 0
\]

\[
    A\bm{x}=
    \begin{bmatrix}
        2\\
        1
    \end{bmatrix}, \bar{\bm{x}}\ge \bm{0}.\text{Therefore $\bar{\bm{x}}$ is a feasible solution.}
\]
\[
    A\bm{d}=\begin{bmatrix}
        0\\
        0
    \end{bmatrix}, \bm{d}\ge \bm{0}.\\
\]

\[A\tilde{\bm{x}}=A(\bar{\bm{x}}+t\bm{d})=A\bar{\bm{x}}+t(A\bm{d})=
\begin{bmatrix}
    2\\
    1
\end{bmatrix}\]
\[\tilde{\bm{x}}=\bar{\bm{x}}+t\bm{d}\]
Therefore, $\tilde{\bm{x}}$ is a feasible solution $\forall t\ge 0$.


\textbf{Objective function value of $\tilde{\bm{x}}$:}
\[
\begin{bmatrix}
    -1 & 3 & 0 & 0 & 1
\end{bmatrix}
\left(\begin{bmatrix}
    0\\
    0\\
    0\\
    2\\
    1  
\end{bmatrix}
+
t
\begin{bmatrix}
    1\\
    0\\
    0\\
    1\\
    2
\end{bmatrix}\right)
=
1+t(-1+2)=1+t\rightarrow+\infty \text{ as }t\rightarrow+\infty\]
Therefore the LP is unbounded.


\begin{thmbox}
    \subsection{Proposition (Unboundedness)}
    If $\exists \bm{\bar{x}}\in\mathbb{R}^n$ such that
    \[A\bm{\bar{x}}=\bm{b}, \bm{x}\ge \bm{0}.\]
    and $\exists\bm{d}\in\mathbb{R}^n$ such that
    \begin{enumerate}[(1)]
        \item $A\bm{d}=\bm{0}$
        \item $\bm{d}\ge \bm{0}$
        \item $\bm{c}^\top \bm{d}>0$
    \end{enumerate}
    For every $\bm{c}\in\mathbb{R}^n$, the LP
    \[\max \{\bm{c}^\top \bm{x} : A\bm{x}=\bm{b},\,
    \bm{x}\ge\bm{0}\}\]
    is unbounded. In particular, we call a pair of vectors $\bm{\bar{x}}$, $\bm{d}$ a
    \emph{certificate of unboundedness}.
\end{thmbox}

\begin{proof}
    Suppose there exists such $\bm{d}$. Consider
    \[\tilde{\bm{x}}=\bar{\bm{x}}+t\bm{d}, t\ge 0\]
Then,
\[A\tilde{\bm{x}}=
\underbrace{A\bar{\bm{x}}}_{\bm{b}}+
t\underbrace{(A\bm{d})}_{\bm{0}}=\bm{b}\]
Therefore $\tilde{\bm{x}}$ is a feasible solution of the LP, $t\ge 0$.
The objective value of the function is
\[\bm{c}^\top \tilde{\bm{x}}=\bm{c}^\top \bar{\bm{x}}+t
\underbrace{(\bm{c}^\top \bm{d})}_{>\bm{0}}\rightarrow +\infty\text{ as }t\rightarrow+\infty\]
Therefore, the LP is unbounded.
\end{proof}
\begin{remark}
    If the LP is $\min$, then flip the equality for (3).
\end{remark}

\subsection{Example (Optimal LP)}
\[\max 10x_1+15x_2\]
subject to
\begin{align*}
    2x_1+x_2+x_3=1600\\
    x_1+3x_2+x_4=1200\\
    \bm{x}\ge \bm{0}
\end{align*}
Consider $\bar{\bm{x}}:=(720,160,0,0)^\top$ and $\bm{y}:=(3,4)^\top$.

Note that
$A\bar{\bm{x}}=\bm{b}$, with $\bar{\bm{x}}\ge \bm{0}$, 
so $\bar{\bm{x}}$ is a feasible solution.


Also, $\bm{c}^\top \bm{\bar{x}}=7200+2400=9600$.
Every feasible solution satisfies
\begin{align*}
    &A\bm{x}=\bm{b}\\
    &\implies \bm{y}^\top A\bm{x}=\bm{y}^\top \bm{b}
\end{align*}
\[\bm{y}^\top A=
\begin{bmatrix}
    10 & 15 & 3 & 4
\end{bmatrix}
\ge
\begin{bmatrix}
    10 & 15 & 0 & 0
\end{bmatrix}=\bm{c}^\top \]
\[\bm{y}^\top \bm{b}
=3\times 1600+4\times 1200=9600
=\bm{c}^\top \bm{\bar{x}}\]
Therefore $\bar{\bm{x}}$ is an optimal solution.
