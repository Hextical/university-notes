\section{2019-09-24}
\begin{defbox}
    \subsection{Feasible and Infeasible Solutions}
    Consider an LP with variables $x_1,\dots,x_n$.  Then the
    assignment of values to all variables such that all constraints are satisfied,
    gives a \emph{feasible solution}.

    An optimization problem is called \emph{feasible} if it has at least one feasible
    solution, otherwise it is called \emph{infeasible}.
\end{defbox}

\subsection{Example (Infeasible LP)}
(LP)
\[\max x_1+2x_2+3x_3+4x_4+5x_5\]
subject to
\begin{align*}
    &
    \begin{matrix}
    1\\
    -2
    \end{matrix}
    \underbrace{
        \begin{bmatrix}
        -3 & 2 & 7 & 1 & -7 \\
        -2 & 1 & 2 & 0 & -4
        \end{bmatrix}}_{A}
    \underbrace{\begin{bmatrix}
        x_1\\
        x_2\\
        x_3\\
        x_4\\
        x_5
    \end{bmatrix}}_{\symbfit{x}}
    =
    \underbrace{\begin{bmatrix}
        6\\
        4
    \end{bmatrix}}_{\symbfit{b}}\\
    &\symbfit{x}\ge \mathbb{0}
\end{align*}
Let $\symbfit{y}:=(1,-2)^\top$
and consider the facts
\begin{align*}
    &A\symbfit{x}=\symbfit{b}\\
    &\implies \symbfit{y}^\top A\symbfit{x}=\symbfit{y}^\top \symbfit{b}\\
    &\implies \underbrace{\begin{bmatrix}
        1 & 0 & 3 & 1 & 1
    \end{bmatrix}}_{\ge \mathbb{0}^\top }
    \underbrace{\symbfit{x}}_{\ge \mathbb{0}}=\underbrace{6-8}_{< 0}=-2
\end{align*}
Therefore, since $\nexists$ any solution to $A\symbfit{x}=\symbfit{b}$, $\symbfit{x}\ge 0$
the LP is infeasible.


\begin{thmbox}
    \subsection{Proposition (Infeasibility)}
    If
    $\exists \symbfit{y}\in\mathbb{R}^m$ such that
    \begin{enumerate}[(1)]
        \item $\symbfit{y}^\top A\ge\mathbb{0}^\top $
        \item $\symbfit{y}^\top \symbfit{b}<0$
    \end{enumerate}
    For every $\symbfit{c}\in\mathbb{R}^n$, the LP
    \[\max \{\symbfit{c}^\top \symbfit{x} \mid A\symbfit{x}=\symbfit{b}\text{, }
    \symbfit{x}\ge\mathbb{0}\}\]
    is infeasible. In particular, we call a vector $\symbfit{y}$ a \emph{certificate of infeasibility}.
\end{thmbox}

\begin{proof}
Suppose there exists such a $ \symbfit{y} $.
Suppose for a contradiction that $\exists\symbfit{\bar{x}}\in\mathbb{R}^n$ 
(there is a feasible solution)
such that
\[A\symbfit{\bar{x}}=\symbfit{b} \text{, }\symbfit{\bar{x}}\ge \mathbb{0}\]
\[
    A\symbfit{\bar{x}}=\symbfit{b}
    \implies
    \underbrace{\symbfit{y}^\top A}_{\ge\mathbb{0}^\top }
    \underbrace{\symbfit{\bar{x}}}_{\ge\mathbb{0}}
    =\underbrace{\symbfit{y}^\top \symbfit{b}}_{\nless 0}
\]
a contradiction to (2).
\end{proof}

An optimization problem is called unbounded if $\forall M\in\mathbb{R}$, there
exists a feasible solution of the optimization problem with the objective 
value strictly better than $M$.
\subsection{Example (Unbounded LP)}
\[\max 
\begin{bmatrix}
    -1 & 3 & 0 & 0 & 1
\end{bmatrix}\symbfit{x}\]
subject to
\begin{align*}
    &\begin{bmatrix}
        -1 & 3 & -1 & 1 & 0\\
        -2 & 4 & 1 & 0 & 1
    \end{bmatrix}
    \symbfit{x}
    =
    \begin{bmatrix}
        2\\
        1
    \end{bmatrix}\\
    &\symbfit{x}\ge \mathbb{0}
\end{align*}
Consider
\[\symbfit{\tilde{x}}:=
\underbrace{\begin{bmatrix}
    0\\
    0\\
    0\\
    2\\
    1  
\end{bmatrix}}_{\symbfit{x}}
+
t
\underbrace{\begin{bmatrix}
    1\\
    0\\
    0\\
    1\\
    2
\end{bmatrix}}_{\symbfit{d}} \text{, } t\ge 0
\]

\[
    A\symbfit{x}=
    \begin{bmatrix}
        2\\
        1
    \end{bmatrix}, \bar{\symbfit{x}}\ge \mathbb{0}.\text{Therefore $\bar{\symbfit{x}}$ is a feasible solution.}
\]
\[
    A\symbfit{d}=\begin{bmatrix}
        0\\
        0
    \end{bmatrix}, \symbfit{d}\ge \mathbb{0}.\\
\]

\[A\tilde{\symbfit{x}}=A(\bar{\symbfit{x}}+t\symbfit{d})=A\bar{\symbfit{x}}+t(A\symbfit{d})=
\begin{bmatrix}
    2\\
    1
\end{bmatrix}\]
\[\tilde{\symbfit{x}}=\bar{\symbfit{x}}+t\symbfit{d}\]
Therefore, $\tilde{\symbfit{x}}$ is a feasible solution $\forall t\ge 0$.


\textbf{Objective function value of $\tilde{\symbfit{x}}$:}
\[
\begin{bmatrix}
    -1 & 3 & 0 & 0 & 1
\end{bmatrix}
\left(\begin{bmatrix}
    0\\
    0\\
    0\\
    2\\
    1  
\end{bmatrix}
+
t
\begin{bmatrix}
    1\\
    0\\
    0\\
    1\\
    2
\end{bmatrix}\right)
=
1+t(-1+2)=1+t\rightarrow+\infty \text{ as }t\rightarrow+\infty\]
Therefore the LP is unbounded.


\begin{thmbox}
    \subsection{Proposition (Unboundedness)}
    If $\exists \symbfit{\bar{x}}\in\mathbb{R}^n$ such that
    \[A\symbfit{\bar{x}}=\symbfit{b}, \symbfit{x}\ge \mathbb{0}.\]
    and $\exists\symbfit{d}\in\mathbb{R}^n$ such that
    \begin{enumerate}[(1)]
        \item $A\symbfit{d}=\mathbb{0}$
        \item $\symbfit{d}\ge \mathbb{0}$
        \item $\symbfit{c}^\top \symbfit{d}>0$
    \end{enumerate}
    For every $\symbfit{c}\in\mathbb{R}^n$, the LP
    \[\max \{\symbfit{c}^\top \symbfit{x} \mid A\symbfit{x}=\symbfit{b}\text{, }
    \symbfit{x}\ge\mathbb{0}\}\]
    is unbounded. In particular, we call a pair of vectors $\symbfit{\bar{x}}$, $\symbfit{d}$ a
    \emph{certificate of unboundedness}.
\end{thmbox}

\begin{proof}
    Suppose there exists such $\symbfit{d}$. Consider
    \[\tilde{\symbfit{x}}=\bar{\symbfit{x}}+t\symbfit{d}, t\ge 0\]
Then,
\[A\tilde{\symbfit{x}}=
\underbrace{A\bar{\symbfit{x}}}_{\symbfit{b}}+
t\underbrace{(A\symbfit{d})}_{\mathbb{0}}=\symbfit{b}\]
Therefore $\tilde{\symbfit{x}}$ is a feasible solution of the LP, $t\ge 0$.
The objective value of the function is
\[\symbfit{c}^\top \tilde{\symbfit{x}}=\symbfit{c}^\top \bar{\symbfit{x}}+t
\underbrace{(\symbfit{c}^\top \symbfit{d})}_{>\mathbb{0}}\rightarrow +\infty\text{ as }t\rightarrow+\infty\]
Therefore, the LP is unbounded.
\end{proof}
\begin{remark}
    If the LP is $\min$, then flip the equality for (3).
\end{remark}

\subsection{Example (Optimal LP)}
\[\max 10x_1+15x_2\]
subject to
\begin{align*}
    2x_1+x_2+x_3=1600\\
    x_1+3x_2+x_4=1200\\
    \symbfit{x}\ge \mathbb{0}
\end{align*}
Consider $\bar{\symbfit{x}}:=(720,160,0,0)^\top$ and $\symbfit{y}:=(3,4)^\top$.

Note that
$A\bar{\symbfit{x}}=\symbfit{b}$, with $\bar{\symbfit{x}}\ge \mathbb{0}$, 
so $\bar{\symbfit{x}}$ is a feasible solution.


Also, $\symbfit{c}^\top \symbfit{\bar{x}}=7200+2400=9600$.
Every feasible solution satisfies
\begin{align*}
    &A\symbfit{x}=\symbfit{b}\\
    &\implies \symbfit{y}^\top A\symbfit{x}=\symbfit{y}^\top \symbfit{b}
\end{align*}
\[\symbfit{y}^\top A=
\begin{bmatrix}
    10 & 15 & 3 & 4
\end{bmatrix}
\ge
\begin{bmatrix}
    10 & 15 & 0 & 0
\end{bmatrix}=\symbfit{c}^\top \]
\[\symbfit{y}^\top \symbfit{b}
=3\times 1600+4\times 1200=9600
=\symbfit{c}^\top \symbfit{\bar{x}}\]
Therefore $\bar{\symbfit{x}}$ is an optimal solution.
