\section{Tutorial 3}
\textbf{Problem 1.} Consider the set of non-negative integers
$ \mathbb{N}_0 $, but with a non-standard weight function
\[ w(a)=
    \begin{cases}
        \frac{3}{2} a + 1,\,\text{if $a$ is even}, \\
        2(a+1),\,\text{if $a$ is odd.}
    \end{cases} \]
Find the generating series for $ \mathbb{N}_0 $ with respect to this weight
function and express it as a simplified rational expression.

\textbf{Solution.}

$ \mathbb{N}_0=\mathbb{N}_{\text{even}}\cup\mathbb{N}_{\text{odd}} $
\begin{align*}
    \Phi_{\mathbb{N}_0}(x)
     & =\Phi_{\mathbb{N}_{\text{even}}}(x)+\Phi_{\mathbb{N}_{\text{odd}}}(x)\qquad\text{Sum Lemma} \\
     & =\sum\limits_{a \text{ even}}x^{3/2a+1}+\sum\limits_{a \text{ odd}}x^{2(a+1)}               \\
     & =x \sum\limits_{a \text{ even}}x^{3/2a}+x^2\sum\limits_{a \text{ odd}}x^{2a}                \\
     & =x \sum\limits_{i\ge 0} (x^{3/2})^{2i}+x^2\sum\limits_{i \ge 0} (x^{2})^{2i+1}              \\
     & =x \sum\limits_{i\ge 0} (x^3)^i + x^4 \sum\limits_{i\ge 0} (x^4)^i                          \\
     & =\frac{x}{1-x^3} + \frac{x^4}{1-x^4}
\end{align*}

\textbf{Problem 2.} Let $ m,\,n $ be positive integers and $ \alpha,\,\beta $
positive real numbers. Find the generating series for the cartesian product
\[ \{1,\ldots ,m\}\times \{1,\ldots ,n\} \]
with respect to the weight function
\[ w(a,b)=\alpha a+ \beta b \]
and express it as a simplified rational expression.

\textbf{Solution.}

\begin{thmbox}
    \begin{theorem}[Partial Geometric Series]
        \[ \sum\limits_{i=0}^{k} x^i=\frac{1-x^{k+1}}{1-x} \]
    \end{theorem}
\end{thmbox}

\begin{align*}
    \Phi_{A\times B}(x)
     & =\sum\limits_{(a,b)\in A\times B}x^{\alpha a+\beta B}                         \\
     & =\sum\limits_{a\in A}x^{\alpha a}\sum\limits_{b\in B}x^{\beta b}\qquad\text{Product Lemma}              \\
     & =x^{\alpha+\beta} \sum\limits_{i=1}^{m} (x^a)^i \sum\limits_{j=1}^{n} (x^b)^j \\
     & =x^{\alpha+\beta}\left( \frac{x^a(1-x^{am})}{1-x^a}  \right)
     \left( \frac{x^b(1-x^{bm})}{1-x^b}  \right)\qquad\text{Partial Geometric Series}                                                   \\
     & =x^{\alpha + \beta + a + b}
    \frac{(1-x^{am})(1-x^{bm})}{(1-x^a)(1-x^b)}
\end{align*}

\textbf{Problem 3.} Let $ a,\,b,\,n,\,k $ be positive integers with $ a\le b $
and $ k\le n $. How many compositions of $ n $ with $ k $ parts are there in which
all parts are elements of $ \{a,\ldots ,b\} $? Expressing the result as a finite
sum $ \sum\limits_{i=0}^{k} s_i $ is sufficient.

\emph{Rough.}

Start with a small example $ a=2,\,b=4,\,n=9,\,k=3 $.

Let $ C=\{\text{compositions with 3 parts where each part is 2, 3, or 4}\} $.
We want $ [x^9]\Phi_C(x) $.

Let $ P=\{\text{parts of value 2, 3, or 4}\} $. $ C=P\times P \times P $ (3 parts).
Thus, $ \Phi_C(x)=(\Phi_P(x))^3 $.

\begin{align*}
    (\Phi_P(x))^3 & =\left( \sum\limits_{i=2}^{4} x^i \right)^3
\end{align*}
We want $ [x^9](x^2+x^3+x^4)^3 $.

\textbf{Solution.}

So, in general we have
\begin{align*}
    \Phi_C(x) & =\left(\Phi_P(x)\right)^k                   \\
              & =\left( \sum\limits_{i=a}^{b} x^i \right)^k
\end{align*}
We want $ \left[x^n\right]\left( \sum\limits_{i=a}^{b} x^i \right)^k $.
\begin{align*}
    [x^n]\left( \sum\limits_{i=a}^{b} x^i \right)^k
     & =\left[x^n\right](x^a+\cdots+x^b)^k                                                                              \\
     & =\left[x^n\right]x^{ak}(1+\cdots+x^{b-a})^k                                                                      \\
     & =\left[x^{n-ak}\right]\left(\frac{1-x^{b-a+1}}{1-x}\right)^k                                                     \\
     & =\left[x^{n-ak}\right]\left(1+(-x^{b-a+1})\right)^k(1-x)^{-k}                                                    \\
     & =\left[x^{n-ak}\right]\sum\limits_{i\ge 0}\binom{k}{i}(-1)^ix^{i(b-a+1)}
    \sum\limits_{j\ge 0}\binom{j+k-1}{k-1}x^{j}                                                                         \\
     & =\left[x^{n-ak}\right]\sum\limits_{i\ge 0}\sum\limits_{j\ge 0}\binom{k}{i}\binom{j+k-1}{k-1}(-1)^ix^{i(b-a+1)+j}
\end{align*}
$ i(b-a+1)+j=n-ak $
\[ \sum\limits_{i=0}^{\lfloor \frac{n-ak}{b-a+1} \rfloor}
    \binom{k}{i}\binom{n-ak-i(b-a+1)+k-1}{k-1}(-1)^i \]
