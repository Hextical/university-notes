\makeheading{2020-03-25}

\textbf{Q}: Which graphs can be drawn on a mobius strip?

\begin{itemize}
    \item Non-planar graphs like $ K_5 $
\end{itemize}

\begin{thmbox}
    \begin{theorem}[Handshaking for Faces]
        If $ G $ is a planar graph and $ F $ is the set of
        faces in some drawing of $ G $, then
        \[ \sum\limits_{f\in F}\deg(f)=2|E(G)| \]
    \end{theorem}
\end{thmbox}

\begin{proof}
    \[ \sum\limits_{f\in F}\deg(f)
        =\sum\limits_{f\in F}(\text{length of boundary walk of }f) \]
    Each edge contributes $ 1 $ to the length of exactly
    two boundary walks of faces (one for each side).
    Therefore,
    \[ \sum\limits_{f\in F}(\text{length of boundary walk of }f)
        =2|E| \]
\end{proof}

In a planar drawing of a tree,
\begin{itemize}
    \item there is exactly one face
    \item its degree is $ 2|E| $
\end{itemize}

\begin{thmbox}
    \begin{theorem}[Euler's Formula]
        If $ F $ is the set of faces in a drawing of a connected
        planar graph $ G=(V,E) $, then
        \[ |V|-|E|+|F|=2 \]
    \end{theorem}
\end{thmbox}
Suppose we have a spanning tree; that is, we have a tree with
$ n $ vertices, $ n-1 $ edges, and $ 1 $ face. Now,
\[ |V|,|E|,|F|\longrightarrow_{\text{+e}} |V|, |E|+1,|F|+1 \]
\begin{proof}
    Suppose for a contradiction that the theorem is false.
    Let $ G $ be a counter-example with as few edges as possible.
    If $ G $ is a tree, then $ |E(G)|=|V(G)|-1 $ and any
    drawing of $ G $ has exactly one face, so for any
    drawing,
    \[ |V|-|E|+|F|=|V|-(|V|-1)+1=2 \]
    So, $ G $ is not a counter-example. Otherwise, $ G $
    is not a tree. Let $ e $ be an edge of $ G $ that is not
    a bridge; that is, $ G-e $ is connected. Now, a plane
    drawing of $ G $ gives rise to a plane drawing of $ G-e $
    with exactly one less face. So, the number of faces of
    (the drawing of) $ G-e $ is $ |F(G)|-1 $. By the minimality
    of $ G $, the graph $ G-e $ satisfies Euler's Formula.
    Therefore,
    \[ |V(G-e)|-|E(G-e)|+(|F(G)|-1)=2 \]
    \[ \implies |V(G)|-(|E(G)|-1)+(|F(G)|-1)=2 \]
    \[ \implies |V(G)|-|E(G)|+|F(G)|=2, \]
    contradicting the choice of $ G $.
\end{proof}

Let $ G=(V,E) $ be a connected planar graph, and $ F $ be the set of faces
in some drawing of $ G $.
\begin{itemize}
    \item Handshaking for Faces: $ \sum\limits_{f\in F}\deg(f)=2|E| $.
    \item Euler's Formula: $ |V|-|E|+|F|=2 $.
\end{itemize}

\textbf{Q}: What are the \emph{connected} planar drawings where:
\begin{itemize}
    \item every vertex has the same degree ($ d\geqslant 3 $), and
    \item every face has the same degree ($ k\geqslant 3 $)?
\end{itemize}

\textbf{A}:
\begin{itemize}
    \item $ K_4 $: vertices of degree $ 3 $ and faces of degree $ 3 $
    \item cube: vertices of degree $ 3 $ and faces of degree $ 4 $
    \item $ k $-cycle: vertices of degree $ 2 $ and faces of degree $ k $
\end{itemize}

\section{Platonic Solids}
\begin{defbox}
    \begin{definition}
        A graph is called \textbf{\emph{Platonic}} if it can be drawn in
        the plane so it is connected, every vertex has degree
        $ d\geqslant 3 $, and every face has degree $ k\geqslant 3 $.
    \end{definition}
\end{defbox}

Let $ G $ be a Platonic graph. Let $ n=|V(G)| $, $ m=|E(G)| $,
and $ \ell=|F| $ (in some plane drawing of $ G $).

By Euler's Formula,
\[ n-m+\ell=2 \]
By the Handshake Lemma,
\[ 2m=\sum\limits_{v\in V(G)}\deg(v)=n\times d \]
By Handshaking for Faces,
\[ 2m=\sum\limits_{f\in F}\deg(f)=\ell\times k \]
Solving for $ n $ and $ \ell $,
\[ n=\frac{2m}{d} \]
\[ \ell=\frac{2m}{k} \]
Plugging these into Euler's Formula,
\[ 2=\frac{2m}{d}-m+\frac{2m}{k}=m\left(\frac{2}{d}+\frac{2}{k}-1\right) \]
We have $ \frac{2}{d} +\frac{2}{k} -1=\frac{2}{m} >0 $.
We know $ d,k\geqslant 3 $. If $ d,k\geqslant 4 $, then
\[ \frac{2}{d} + \frac{2}{k} - 1\leqslant 0 \]
So, one of $ d,k $ is at most $ 3 $. If (say) $ d=3 $, then
\[ \frac{2}{d} +\frac{2}{k} -1>0 \]
so, $ \frac{2}{k} >\frac{1}{3} \implies k<6 $. This
only leaves $ 5 $ options:
\[ (d,k)\in \{(3,3),(3,4),(4,3),(3,5),(5,3)\} \]
The only equation that relates $ m,k, $ and $ d $ is:
\[ \frac{2}{m} = \frac{2}{k} +\frac{2}{d} -1 \]
If $ (k,d)=(3,3) $, then
\[ \frac{2}{m} =\frac{2}{3} +\frac{2}{3} -1\implies m=6 \]
\[ n=\frac{2m}{d} =4,\; \ell=\frac{2m}{k} =4 \]
If $ (k,d)=(3,3) $, solving gives $ n=4,\ell=4,m=6 $ which is
the \textbf{\emph{Tetrahedron}} ($ K_4 $); self-dual.

If $ (k,d)=(4,3) $, solving gives $ n=8,m=12,\ell=6 $ which is
the \textbf{\emph{Cube}}; dual: $ (k,d)=(3,4) $.

If $ (k,d)=(3,4) $, solving gives $ n=6,m=12,\ell=8 $ which is
the \textbf{\emph{Octahedron}}.

If $ (k,d)=(5,3) $, solving gives $ n=20,m=30,\ell=12 $ which is
the \textbf{\emph{Dodecahedron}}; dual: $ (k,d)=(3,5) $.

If $ (k,d)=(3,5) $, solving gives $ n=12,m=30,\ell=20 $ which is
the \textbf{\emph{Icosahedron}}.
