\section{Math 239 Tutorial 4}
\textbf{Problem 1.} Let $ S $ denote the set of strings of the form
$ \{1\}^* \{0\}^* \{1\}^* \{0\}^* $. Find the generating function for
$ \Phi_S(x) $, where the weight of a string is given by its length.

\emph{Solution.}

$ \{1\}^*\left( \{0\}\{0\}^* \{1\}\{1\}^* \right)\{0\}^*\cup \{1\}^* \{0\}^* $

\textbf{Problem 2.} Let $ S=\{00,111\}^* $. Find a formula for
$ \Phi_S(x) $.

\emph{Solution.}

\begin{align*}
    \Phi_S(x)
    &=\Phi_{\{00,111\}^*}\\
    &=\frac{1}{1-\Phi_{\{00,111\}}}\\
    &=\frac{1}{1-(x^2+x^3)}\\
\end{align*}

\textbf{Problem 3.} Let $ S $ be $ \{00,111\}^* $ and let $ S_n $
denote the set of strings of length $ n $ in $ S $. Give a combinatorial
proof that $ |S_n|=|S_{n-2}|+|S_{n-3}| $ for $ n\geqslant 3 $.

Look at $ n=0,...,6 $.

$ n=2 : |S_2|=1\rightarrow 00 $

$ n=3 : |S_3|=1 \rightarrow 111 $

$ n=4 : |S_4|=1 \rightarrow 0000$

$ n=5 : |S_5|=2 \rightarrow 00111, 11100 $

$ n=6 : |S_6|=2 \rightarrow 111111,000000 $

\emph{Solution.}

Let $ s\in S_n $, $ n\geqslant 3 $. What could $ s $ start with?

Case 1: $ 00t $, $ t\in S_{n-2} $

Case 2: $ 111t $, $ t\in S_{n-3} $

$ f: S_n\rightarrow S_{n-2}\cup S_{n-3} $, $\forall s\in S_n $, $ f(s)=t $
where
\[ s=
\begin{cases}
    00t,\,\text{if $s$ starts with }00\\
    111t,\,\text{if $s$ starts with }111
\end{cases} \]
$ g:S_{n-2}\cup S_{n-3}\rightarrow S_n $
\[ g(t)=
\begin{cases}
    00t,\,t\in S_{n-2}\\
    111t,\,t\in S_{n-3}
\end{cases} \]
Explain how $ f(g(t))=t $ and $ g(f(s))=s $.

\textbf{Problem 4.} Explain why $ \left( \{1\}^* \{0\}^* \right) $
is ambiguous.

\emph{Solution.}

Ambiguous means there are multiple ways to create a string. So, taking
$ \varepsilon $ works since,
\begin{align*}
    \left(\{1\}^0\{0\}^0\right)^{x}
    &=(\varepsilon)^{x}\\
    &=\varepsilon
\end{align*}