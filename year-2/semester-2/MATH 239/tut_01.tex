\chapter{Tutorials}

\section{Tutorial 1}
\textbf{Problem 1.} Give a combinatorial proof that the number of
subsets of $ \{1,\ldots ,n\} $ with even cardinality is the same as the
number with odd cardinality.

\textbf{Solution.}

Let $ X $ denote any subset of $ \{1,\ldots ,n\} $. The corresponding
set $ Y $ will be:

\[ Y = \begin{cases}
        X\cup \{1\},      & \text{if } 1\notin X \\
        X\setminus \{1\}, & \text{if } 1\in X
    \end{cases} \]

\textbf{Problem 2.} Let $ n $ be a positive integer. Give a combinatorial proof
of the identity
\[ \sum\limits_{i=0}^{n} i \binom{n}{i} = n2^{n-1}. \]

\textbf{Solution.}

Suppose we have a group of $ n $ people.

\emph{RHS}: Choose a committee of size $ i $, then choose one of
the $ i $ committee members to be a leader. There are $ \binom{n}{i} $
ways to pick the members of the committee and $ i $ ways to choose the leader,
which yields $ \sum\limits_{i=0}^{n} i\binom{n}{i} $.

\emph{LHS}: Pick a leader, then from the remaining $ (n-1) $ people
we choose them to either be in or out of the committee. There are
$ n $ ways to pick the leader and $ 2^{n-1} $ ways to pick
the remaining committee members.

Thus, since we are counting the same object twice in two different ways,
we have that
\[ \sum\limits_{i=0}^{n} i \binom{n}{i} = n2^{n-1}. \]

\textbf{Problem 3.} For any integers $ n,k,r $ where
$ n\geqslant k \geqslant r\geqslant 0 $, give a combinatorial proof
of the following identity.
\[ \binom{n}{k}\binom{k}{r}=\binom{n}{r}\binom{n-r}{k-r}. \]

\textbf{Solution.}

Suppose we have a group of $ n $ people, with a $ k $-person
committee and a $ r $-person subcommittee.

\emph{RHS}: Choose the committee in $ \binom{n}{k} $ ways,
then choose the subcommittee from the committee in $ \binom{k}{r} $ ways,
which yields $ \binom{n}{k}\binom{k}{r} $.

\emph{LHS}: Choose the $ r $ subcommittee members in $ \binom{n}{r} $ ways,
then fill in the remaining $ (k-r) $ committee members from the
remaining $ (n-r) $ people, which yields $ \binom{n}{r}\binom{n-r}{k-r} $.

Thus, since we are counting the same object twice in two different ways,
we have that
\[ \binom{n}{k}\binom{k}{r}=\binom{n}{r}\binom{n-r}{k-r}. \]

\textbf{Problem 4.} Let $ n\geqslant 5 $ be an integer. Give a combinatorial
proof of the following identity
\[ \sum\limits_{k=5}^{n} \binom{k-1}{4} = \sum\limits_{m=3}^{n-2} \binom{m-1}{2}
    \binom{n-m}{2}. \]
(Hint: Both sides are equal to $ \binom{n}{5} $.)

\textbf{Solution.}

Too hard for my poor soul.
