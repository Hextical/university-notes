\makeheading{2020-03-09}
\textbf{Recall}: Let $ C $ be an $ (n,k) $ code with generator polynomial
$ g(x) $. Suppose $ C $ is a $ t $-c.b.e.c.c. So, $ t\leqslant n-k $.
\[ H= \Bigl[\; I_{n-k}\mid -R^\top \;\Bigr] \]
is a PCM for $ C $; $ s(x)=r(x)\mod g(x) $.

\textbf{Idea}: Suppose $ e $ is a cyclic burst of length at most $ t $.
Compute shifts of $ e $, say $ e_i=x^i e $ has all its non-zero entries in the first
$ (n-k) $ positions. Then,
\[ s_i(x)=e_i(x)\mod g_i(x) \]
and we can recognize such an $ s_i(x) $ since it is a non-cyclic burst of length
at most $ t $. Them, $ e=x^{n-i}e_i $. How do we compute $ s_i(x) $? Recall,
$ \bm{r}=\bm{c}+\bm{e} $. So, $ x^i \bm{r}=x^i\bm{c}+x^i\bm{e} $, so
$ x^i \bm{r} $ and $ x^i\bm{e} $ have the same syndrome.

\section{Error Trapping Decoding (For Cyclic Burst Errors)}
Let $ r(x)= $ received polynomial. Let $ s_i(x)= $ syndrome of $ x^i r(x) $ for
each $ i\in[1,n-1] $ where $ s_0=r(x)\mod g(x) $.

\begin{algbox}
    \begin{algorithm}[H]
        \DontPrintSemicolon{}
        \caption{Error Trapping}

        \For{$ i=0$ \KwTo{} $ n-1 $} {
            Compute $ s_i(x) $ with~\ref{syndrome calculation}.

            \If{$ s_i(x) $ is a non-cyclic burst of length at most $ t $} {
                $ e_i(x)\gets(s_i(x),0) $\;
                $ e(x)\gets x^{n-i}e_i(x) $\;
                \Return{$ r(x)-e(x) $}
            }
        }
        \Return{}
    \end{algorithm}
\end{algbox}

\begin{exbox}
    \begin{example}
        $ g(x)=1+x+x^2+x^3+x^6 $ is the generator polynomial for $ (15,9) $-binary
        cyclic code with c.b.e.c.c $ 3 $. Decode $ r=(1110\; 1110\; 1100\; 000) $.

        \textbf{Solution.}
        Compute $ s_0(x)=r(x)\mod g(x)=x^5+x^4+x+1 $.
        \[ \begin{array}{|c|c|}
                \hline
                i & s_i(x) \\
                \hline
                0 & 110011 \\
                1 & 100101 \\
                2 & 101110 \\
                3 & 010111 \\
                4 & 110111 \\
                5 & 100111 \\
                6 & 101111 \\
                7 & 101011 \\
                8 & 101001 \\
                9 & 101000 \\
                \hline
            \end{array} \]
        \[ \implies \bm{e}_9=(101000\; 000000000) \]
        \[ \implies \bm{e}=x^6 \bm{e}_9=(000000\; 101000\;000 ) \]
        \[ \implies \bm{c}=\bm{r}-\bm{e}=(1110\;1100\;0100\;000) \]
        Check: $ H\bm{c}^\top=\bm{0} $ (bad) OR $ g(x)\mid c(x) $ via long division.
    \end{example}
\end{exbox}

\section{Interleaving}
\textbf{Goal}: Improve the c.b.e.c.c of a code.

Suppose $ C $ is an $ (n,k) $-code with c.b.e.c.c $ t $.

Suppose the following codewords are transmitted:
\[ \begin{array}{c}
        v_1=(v_{11},v_{12},\ldots ,v_{1n})\in C \\
        v_2=(v_{21},v_{22},\ldots ,v_{2n})\in C \\
        \vdots                                  \\
        v_s=(v_{s1},v_{s2},\ldots ,v_{sn})\in C
    \end{array} \]
Suppose $ v_1,\ldots ,v_s $ are transmitted in that order. If a cyclic burst error
of length at most $ t $ occurs in any codeword, that error can be corrected.

Instead, we transmit: the \underline{columns in order}:
\[ \left[ v_{11},v_{21},\ldots ,v_{s1},\ldots ,v_{1n},v_{2n},\ldots ,v_{sn} \right] \]
Now, if a cyclic burst error of length at most $ st $ occurs in this (fat) codeword,
this means that each original codeword suffered a cyclic error burst of length at most $ t $.

\begin{thmbox}
    \begin{theorem}
        Suppose $ C $ is an $ (n,k) $-cyclic code with generator polynomial $ g(x) $
        and cyclic burst error correcting capability $ t $. $ C^* $, the code
        obtained by \underline{interleaving $C$ to a depth $s$} is an
        $ (ns,ks) $-cyclic code with generator polynomial $ g^*(x)=g(x^s) $.
    \end{theorem}
\end{thmbox}
