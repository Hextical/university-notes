\makeheading{ 2020-01-31 }

\begin{thmbox}
    \begin{theorem}
        Let $ C $ be an $ (n,k) $-code over $ F $, and let $ H $ be a PCM
        for $ C $. Then $ d(C)\geqslant s $ if and only if every $ (s-1) $ columns
        of $ H $ are linearly independent over $ F $.
    \end{theorem} \end{thmbox}

\begin{proof}
    Let $ h_1,\ldots ,h_n $ be the columns of $ H $.

    $ (\impliedby) $ Suppose $ d(C)\leqslant s-1 $, so $ w(C)\leqslant s-1 $.
    Let $ c\in C $, with $ 1\leqslant w(c)\leqslant s-1 $. WLOG, suppose
    $ c_j=0 $ for all $ s\leqslant j\leqslant n $. Since $ c\in C $,
    we have $ Hc^{\top}=0 $. Therefore,
    $ c_1h_1+\cdots+c_{s-1}h_{s-1}=0 $.
    Since $ w(C)\geqslant 1 $, this is a non-trivial linear combination
    of $ h_1,\ldots ,h_{s-1} $ that equal $ 0 $. So,
    $ h_1,\ldots ,h_{s-1} $ are linearly dependent over $ F $.

    $ (\implies) $ Suppose there are $ s-1 $ columns of $ H $ that
    are linearly dependent over $ F $, say $ h_1,\ldots ,h_{s-1} $. So,
    we can write
    \[ c_1h_1+\cdots+c_{s-1}h_{s-1} \]
    where $ c_j\in F $ not all zero. Let $ c=(c_1,\ldots ,c_{s-1},
        \underbrace{0\cdots 0}_{n-s+1})\in V_n(F) $. Then,
    $ Hc^{\top}=0 $. So, $ c_i C $ and $ 1\leqslant w(c)\leqslant s-1 $,
    so $ d(C)\leqslant s-1 $.
\end{proof}

\begin{thmbox}
    \begin{corollary}
        Let $ C $ be an $ (n,k) $-code over $ F $ with PCM $ H $. Then,
        $ d(C) $ is the smallest number of columns of $ H $ that
        are linearly dependent over $ F $.
    \end{corollary} \end{thmbox}

\begin{exbox}
    \begin{example}
        Recall, we found a PCM
        \[ H=
            \left[\begin{array}{cc|ccc}
                    2 & 1 & 1 & 0 & 0 \\
                    1 & 2 & 0 & 1 & 0 \\
                    0 & 2 & 0 & 0 & 1
                \end{array}\right] \]
        for a $ (5,2) $-code $ C $ over $ \mathbb{Z}_3 $. Find $ d(C) $.

        \textbf{Solution.}
        \begin{itemize}
            \item No $ 0 $ column in $ H\implies d(C)\geqslant 2 $
            \item No two linearly dependent columns in $ H $ (since
                  no repeated columns, and no column is two times another
                  column $ \implies d(C)\geqslant 2 $)
        \end{itemize}
        \[ \begin{bmatrix} 2 & 1 & 0 \end{bmatrix}
            = 2\begin{bmatrix} 1 & 0 & 0 \end{bmatrix}
            + \begin{bmatrix} 0 & 1 & 0 \end{bmatrix} \]
        Therefore $ d(C)\ngeqslant 4 $, therefore $ d(C)=3 $.
    \end{example}
\end{exbox}

\begin{exbox}
    \begin{example}
        Let $ C $ be a binary code with PCM $ H $.
        \begin{itemize}
            \item $ d(C)=1\iff  $ $ H $ has a $ 0 $ column.
            \item $ d(C)=2\iff $ the columns of $ H $ are non-zero and two are
                  the same.
            \item $ d(C)=3\iff $ the columns of $ H $ are non-zero, distinct, and one
                  column is the sum of two other (distinct) columns.
        \end{itemize}
    \end{example}
\end{exbox}

\section{Hamming Codes and Perfect Codes}

\begin{exbox}
    \begin{example}
        Construct a $ (7,4,3) $-binary code $ C $.

        \textbf{Solution.}
        Consider a PCM for $ C $:
        \[H= \left[
                \begin{array}{ccc|cccc}
                    1 & 0 & 0 & 1 & 1 & 0 & 1 \\
                    0 & 1 & 0 & 1 & 0 & 1 & 1 \\
                    0 & 0 & 1 & 0 & 1 & 1 & 1
                \end{array} \right]_{3\times{} 7} \]
        This is a \textbf{Hamming Code} of order $ 3 $ over $ GF(2) $.
    \end{example}
\end{exbox}

\begin{defbox}
    \begin{definition}
        Let $ C $ be an $ [n,M] $-code with distance $ d $ over an
        alphabet $ A $ of size $ q $. Let $ e=\left\lfloor \frac{d-1}{2} \right\rfloor $.
        The \textbf{sphere packing bound} or \textbf{Hamming bound} is:
        \[ M \sum\limits_{i=0}^{e} \binom{n}{i}(q-1)^i\leqslant q^n \]
    \end{definition} \end{defbox}

\begin{defbox}
    \begin{definition}
        Let $ C $ be an $ [n,M] $-code over $ A $ of distance $ d $. Then,
        $ C $ is perfect if
        \[ M \sum\limits_{i=0}^{e} \binom{n}{i}(q-1)^i = q^n \]
    \end{definition} \end{defbox}

\textbf{Note:} If $ C $ is perfect, then IMLD=CMLD\@.
