\makeheading{ 2020-01-13 }

\section{Error Correcting \& Detecting Capabilities of a Code}
\begin{itemize}
    \item If $ C $ is used for error correction, the strategy is IMLD/CMLD\@.
    \item If $ C $ is used for error detection only, the strategy is
          to reject $ \bm{r} $ if $ \bm{r}\notin C $, otherwise accept $ \bm{r} $.
\end{itemize}
\begin{defbox}
    \begin{definition}
        A code $ C $ is called an \textbf{$\bm{e}$-error correcting code}
        if the decoder always makes the correct decision if
        at most $ e $ errors per codeword are introduced per transmission.
        We define \textbf{$\bm{e}$-error detecting code} similarly.
    \end{definition} \end{defbox}

\begin{exbox}
    \begin{example}[Error Detecting and Correcting Codes]$\;$
        \begin{itemize}
            \item $ C=\{0000,\,1111\} $ is a $ 1 $-error correcting code, but not a
                  $ 2 $-error correcting code.
            \item $ C=\{\underbrace{0\cdots 0}_{m},\,\underbrace{1\cdots 1}_{m}\} $
                  is a $ \lfloor \frac{m-1}{2} \rfloor $-error correcting code.
            \item $ C=\{0000,\,1111\} $ is a $ 3 $-error detecting code.
        \end{itemize}
    \end{example}
\end{exbox}

\begin{thmbox}
    \begin{theorem}
        If $ d(C)=d $, then $ C $ is a $ (d-1) $-error detecting code.
    \end{theorem} \end{thmbox}

\begin{proof}
    Suppose $ \bm{c}\in C $ is transmitted $ \bm{r} $ is received. If no
    errors occurred during transmission, then $ \bm{r}=\bm{c} $,
    so the decoder correctly accepts $ \bm{r} $. If at least $ 1 $
    and at most $ (d-1) $ errors occur, then
    $ 1\leqslant d(\bm{r},\bm{c})\leqslant d-1 $. Since $ d(C)=d $,
    we have $ \bm{r}\notin C $. Thus the decoder correctly rejects
    $ \bm{r} $. Thus $ C $ is a $ (d-1) $-error detecting code.
\end{proof}

\begin{thmbox}
    \begin{theorem}
        If $ d(C)=d $, then $ C $ is not a $ d $-error detecting code.
    \end{theorem} \end{thmbox}

\begin{proof}
    Since $ d(C)=d $, there exists codewords $ \bm{c}_1,\bm{c}_2\in C $
    with $ d(\bm{c}_1,\bm{c}_2)=d $. Hence it is possible that
    $ \bm{c}_1 $ is sent, $ d $ errors are introduced, and $ \bm{c}_2 $
    is received. In this case, the decoder incorrectly accepts
    $ \bm{c}_2 $. Thus $ C $ is not a $ d $-error detecting code.
\end{proof}

\begin{thmbox}
    \begin{theorem}
        If $ d(C)=d $, then $ C $ is a $ \lfloor \frac{d-1}{2} \rfloor $-error
        correcting code.
    \end{theorem} \end{thmbox}

\begin{proof}
    Suppose $ \bm{c}\in C $ is transmitted, at most $ \frac{d-1}{2}  $ errors
    are introduced, and $ \bm{r} $ is received.
    Let $ \bm{z}\in C $ with $ \bm{z}\neq \bm{c} $.
    By the triangle inequality, we have
    \begin{align*}
        d(\bm{c},\bm{z})\leqslant d(\bm{c},\bm{r})+d(\bm{r},\bm{z})\implies
        d(\bm{r},\bm{z})
         & \geqslant d(\bm{c},\bm{z})-d(\bm{c},\bm{r}) \\
         & \geqslant d-\frac{d-1}{2}                   \\
         & =\frac{d+1}{2}                              \\
         & >\frac{d-1}{2}
    \end{align*}
    So, $ \bm{c} $ is the unique codeword closest to $ \bm{r} $. Hence, IMLD/CMLD
    will decode $ \bm{r} $ to $ \bm{c} $. Thus, $ C $ is a
    $ \lfloor \frac{d-1}{2} \rfloor $-error correcting code.
\end{proof}

\begin{thmbox}
    \begin{theorem}
        If $ d(C)=d $, then $ C $ is not a $ \left( \lfloor \frac{d-1}{2}\rfloor +1 \right) $-
        error correcting code.
    \end{theorem} \end{thmbox}

\begin{proof}
    Exercise.
\end{proof}

Given $ q,\,n,\,M,\,d $, does there exist an $ [n,M] $-code over $ A $
with $ |A|=q $ such that $ d(C)=d $?

Let $ C=\{\bm{c}_1,\ldots ,\bm{c}_M\} $ and $ e=\lfloor \frac{d-1}{2} \rfloor $.
For any codeword $ \bm{c}\in C $, let $ S_{\bm{c}} $ be the sphere of radius $ e $ centred at
$ \bm{c} $;
that is,
\[ S_{\bm{c}}= \{\bm{r}\in A^n:d(\bm{r},\bm{c})\leqslant e\} \]
We proved that if $ \bm{c}_i,\bm{c}_j\in C $ with $ i\neq j $,
then $ S_{\bm{c}_i}\cap S_{\bm{c}_j}= \varnothing $ for each $ i\neq j $.
This question can be viewed as a \textbf{sphere packing problem}:
Can we place $ M $ spheres of radius $ e $ in $ A^n $ such that
no two spheres overlap? This is a purely combinatorial problem.

Does there exists a block code with parameters
$ q=2,\,n=128,\,M=2^{64},\,d\geqslant 22 $? Yes, we will see this in Chapter 6.

\textbf{Roadmap:} We'll view $ \{0,1\}^{n} $ as a vector space of
dimension $ n $ over $ \mathbb{Z}_q $ where $ |A|=q $. We will chose the code
$ C $ to be an $ M $-dimensional subspace of this vector space
and we will choose special subspaces that satisfy the $ d(C)=d $ requirement.
