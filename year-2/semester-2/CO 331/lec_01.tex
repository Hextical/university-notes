\chapter{Introduction and Fundamentals}
\makeheading{2020-01-06}
\section{An Introduction to Coding Theory}

Data Source
$ \rightarrow $
$ \underset{\text{(digital data)}}{\text{Source Encoder}} $
$ \rightarrow $
$ \underset{\text{(encoding algorithm)}}{\text{Channel Encoder}} $
$ \underset{\overset{\Big\uparrow}{\text{Noise}}}{\overset{\text{Channel}}{\rightarrow}} $
$ \underset{\text{(decoding algorithm)}}{\text{Channel Decoder}} $
$ \rightarrow $
Source Decoder $
    \rightarrow $
Data

\begin{exbox}
    \begin{example}[Repetition Code]
        \begin{center}
            \begin{tabular}{| *{5}{>{\centering\arraybackslash}p{3cm} |}}
                \hline
                source message $\rightarrow$ codeword & \# errors/codeword that can be detected & \# errors/codeword that can be corrected & rate                \\
                \hline
                0 $ \rightarrow $ 0                   & 0                                       & 0                                        & 1                   \\
                1 $ \rightarrow $ 1                   &                                         &                                          &                     \\
                \hline
                0 $ \rightarrow $ 00                  & 1                                       & 0                                        & $ \nicefrac{1}{2} $ \\
                1 $ \rightarrow $ 11                  &                                         &                                          &                     \\
                \hline
                0 $ \rightarrow $ 000                 & 2                                       & 1                                        & $ \nicefrac{1}{3} $ \\
                1 $ \rightarrow $ 111                 &                                         &                                          &                     \\
                \hline
                0 $ \rightarrow $ 00000               & 4                                       & 2                                        & $ \nicefrac{1}{5} $ \\
                1 $ \rightarrow $ 11111               &                                         &                                          &                     \\
                \hline
            \end{tabular}
        \end{center}
    \end{example}
\end{exbox}

\textbf{Goal of Coding Theory}

Design codes such that:
\begin{itemize}
    \item High information rate
    \item High error-correcting capability
    \item Efficient encoding and decoding algorithms
\end{itemize}

Codes $ \supset $ Block codes $ \supset $  Linear codes
$ \supset $  Cyclic codes $ \supset $ BCH Codes
$ \supset $ RS Codes

Codes not covered in this course:
\begin{itemize}
    \item Flamming codes
    \item Golay codes
    \item Raptor codes
    \item LDPC codes
    \item Turbo codes
\end{itemize}

Requirements for this course:
\begin{itemize}
    \item MATH 136
    \item Not required (but required to take the course): MATH 235
    \item Familiarity with: Groups, Fields, Ideals, Rings (these will be taught)
    \item Useful, if you have completed these you might be bored:
          PMATH 336, PMATH 334 [or the advanced equivalents]
\end{itemize}

\textbf{The big picture}

In its broadest sense, coding deals with the reliable, efficient, and secure
transmissions of data over channels that are subject to inadvertent noise and
malicious intrusion.

Data source
$ \rightarrow $
$ \underset{\text{(data compression)}}{\text{Source encoder}} $
$ \rightarrow $
$ \underset{\text{(cryptography; co 487)}}{\text{Encryptory authentication}} $
$ \rightarrow $
$ \underset{\text{(error correction codes)}}{\text{Channel encoder}} $
$ \underset
    {
        \overset{\Big\uparrow}{\overset{\text{noise}}{\text{(non-malicious)}}}\qquad
        \overset{\Big\uparrow}{\overset{\text{adversarial intrusion}}{\text{(malicious)}}}
    }
    {\overset{\text{channel}}{\longrightarrow}} $
$ \rightarrow $
Decriptory verification
$ \rightarrow $
$ \underset{\text{(data decompression)}}{\text{Source decoder}}  $
$ \rightarrow $
Data
