\chapter{Finite Fields}
\makeheading{ 2020-01-15 }

\section{Introduction}
\begin{Definition}{Field}{field}
    A \textbf{field} $ F $ is a set of elements under two binary operations,
    which we denote by $ + $ and $ \cdot $, such that
    $ {+\colon F\times F\to F} $ and $ {\cdot\colon F\times F\to F} $
    where all of the following axioms are satisfied:
    \begin{enumerate}[label=V\arabic*]
        \item $ a+(b+c)=(a+b)+c $
        \item $ a+b=b+a $
        \item $ \exists\ 0\in F $ such that $ a+0=a $
        \item $ \exists\ (-a)\in F $ such that $ a+(-a)=0 $
        \item $ a\cdot (b\cdot c)=(a\cdot b)\cdot c $
        \item $ a\cdot b=b\cdot a $
        \item $ \exists\ 1\in F $ such that $ a\cdot 1=a $
        \item $ \forall\ a\neq 0,\,\exists\ (a^{-1})\in F $ such that
              $ a\cdot (a^{-1})=1 $
        \item $ a\cdot (b+c)=a\cdot b+a\cdot c $
    \end{enumerate}
\end{Definition}

\begin{Definition}{Infinite Field}{infinite_field}
    A field $ F $ is \textbf{infinite} if $ \abs{F} $ is infinite.
\end{Definition}

\begin{Definition}{Finite Field}{finite_field}
    A field $ F $ is \textbf{finite} if $ \abs{F} $ is finite.
\end{Definition}

\begin{Definition}{Order}{order}
    The \textbf{order} of a field $ F $ denoted by $ \ord(F) $, is $ \abs{F} $.
\end{Definition}

\begin{Example}{Infinite and Finite Fields}{}
    \begin{itemize}
        \item $ \mathbb{Q},\mathbb{R},\mathbb{C} $ are infinite fields.
        \item $ \mathbb{Z} $ is \textbf{not} a field since $ 3\in\mathbb{Z} $, but
              $ (\sfrac{1}{3})\notin\mathbb{Z} $.
    \end{itemize}
\end{Example}

\textbf{Question:} For what $ n\in\mathbb{Z}_{\geqslant 2} $ does there exist
finite fields of order $ n $? If a field of order $ n $ exists, how do
we ``construct'' it?

\textbf{Recall:} Let $ n\in\mathbb{Z}_{\geqslant 2} $.
The integers modulo $ n $, denoted by
$ \mathbb{Z}_n $, is the set of all equivalence classes modulo $ n $.
\[ \mathbb{Z}_n=\set{[0],[1],[2],\ldots ,[n-1]} \]
where $ [a]+[b]=[a+b] $ and $ [a][b]=[ab] $.
More simply, $ \mathbb{Z}_{n}=\set{0,1,\ldots ,n-1} $
with addition and multiplication performed modulo $ n $.

\begin{Example}{Modular Arithmetic}{}
    Let $ \mathbb{Z}_9=\set{0,1,\ldots ,8} $.
    \begin{itemize}
        \item $ 5+7=3 $; that is, $ 5+7\equiv 3 \pmod{9} $
        \item $ 5\cdot 7=8 $; that is, $ 5+7\equiv 8 \pmod{9} $
    \end{itemize}
\end{Example}

\begin{Definition}{Commutative ring}{commutative_ring}
    A \textbf{commutative ring} satisfies field axioms V1-V9 except
    V8.
\end{Definition}

\begin{Theorem}{}{}
    $ \mathbb{Z}_n $ is a commutative ring.
\end{Theorem}

\begin{Theorem}{}{}
    $ \mathbb{Z}_n $ is a field if and only if $ n $ is prime.
\end{Theorem}

\begin{Proof}{}{}
    $ (\impliedby) $ Suppose $ n $ is prime. Let $ a\in\mathbb{Z}_n $, $ a\neq 0 $
    (i.e. $ 1\leqslant a\leqslant n-1 $). Since $ n $ is prime, $ \gcd(a,n)=1 $
    so $ \exists\ s,t\in\mathbb{Z} $ such that
    \[ as+nt=1 \]
    Reducing both sides modulo $ n $ gives
    \[ as\equiv 1 \pmod{n} \]
    Define $ a^{-1}=s $. Thus, V8 is satisfied and hence $ \mathbb{Z}_n $
    is a field of order $ n $.

    $ (\implies) $ Suppose for a contradiction that $ n $ is composite, say $ n=ab $
    where $ 2\leqslant a,b\leqslant n-1 $. Suppose $ a^{-1} $ exists, and define
    $ a^{-1}=s $. Then,
    \[ as\equiv 1 \pmod{n}\implies abs\equiv b\pmod{n}\implies ns\equiv b \pmod{n}
        \implies 0\equiv b\pmod{n} \]
    So, $ n\mid b $ which is impossible. Therefore, $ a^{-1} $ does not exist, and hence
    $ \mathbb{Z}_n $ is not a field.
\end{Proof}

\textbf{Question:} Do there exist finite fields of orders $ 4 $ and $ 6 $?

\begin{Definition}{Characteristic}{characteristic}
    The \textbf{characteristic} of a field denoted by $ \ch(F) $, is the smallest
    positive integer $ m $ such that
    \[ \underbrace{1+\cdots+1}_{m}=0 \]
    If no such $ m $ exists, then we define $ \ch(F)=0 $
\end{Definition}

\begin{Example}{Characteristic of Fields}{}
    \begin{itemize}
        \item $ \ch(\mathbb{Q})=0 $
        \item $ \ch(\mathbb{R})=0 $
        \item $ \ch(\mathbb{C})=0 $
        \item $ \ch(\mathbb{Z}_p)=p $ where $ p $ is prime.
    \end{itemize}
\end{Example}

\begin{Theorem}{}{}
    If $ \ch(F)=0 $, then $ F $ is infinite.
\end{Theorem}

\begin{Proof}{}{}
    Consider $ 1,1+1,\ldots,\underbrace{1+\cdots+1}_{a}\in F $.
    Suppose for a contradiction there exists distinct $ a,b\in\mathbb{Z} $
    such that
    \[ \underbrace{1+\cdots+1}_{a}=\underbrace{1+\cdots+1}_{b} \]
    where $ a>b $, then
    \[ \underbrace{1+\cdots+1}_{a}=\underbrace{1+\cdots+1}_{b}+
        \underbrace{1+\cdots+1}_{a-b}=\underbrace{1+\cdots+1}_{b} \]
    Hence, $ \underbrace{1+\cdots+1}_{a-b}=0\implies \ch(F)=(a-b) $
    which contradicts $ \ch(F)=0 $. Thus, $ F $ is infinite.
\end{Proof}

\begin{Theorem}{}{}
    If $ F $ is a finite field, then $ \ch(F) $ is prime.
\end{Theorem}

\begin{Proof}{}{}
    Suppose for a contradiction that $ \ch(F)=m $ is composite,
    say $ m=ab $ where
    $ 2\leqslant a,b\leqslant m-1 $. Now
    \[ (\underbrace{1+\cdots+1}_{a})(\underbrace{1+\cdots+1}_{b})
        =\underbrace{1+\cdots+1}_{m}=0 \]
    since $ \ch(F)=m $. Let $ \underbrace{1+\cdots+1}_{a}=s $
    and $ \underbrace{1+\cdots+1}_{b}=t $, so $ st=0 $ where $ s\neq 0 $.
    Since $ \ch(F)=m>a $, there exists $ c\in F $ such that
    $ cs=1 \implies c=s^{-1} $. Therefore $ s^{-1}st=0 $. Thus,
    $ t=0 $ which is a contradiction to $ \ch(F)=m $.
\end{Proof}

\textbf{Roadmap:} Let $ F $ be a finite field of order $ n $.
Then, $ \ch(F)=p $ where $ p $ is prime. Then, $ \mathbb{Z}_p $ is a subfield
of $ F $. $ F $ is a vector space over $ \mathbb{Z}_p $ of $ \dim=k $.
Then, order of $ F $ is $ p^k $.
