\chapter{Finite Fields}
\makeheading{ 2020-01-15 }

\section{Introduction}
\begin{defbox}
    \begin{definition}
        A \textbf{field} $ F $ is a set of elements under two binary operations,
        which we denote by $ + $ and $ \cdot $ such that
        $ +:F\times F\rightarrow F $ and $ \cdot:F\times F\rightarrow F $
        where all the following axioms are satisfied:
        \begin{enumerate}[V1]
            \item $ a+(b+c)=(a+b)+c $
            \item $ a+b=b+a $
            \item $ \exists\, 0\in F $ such that $ a+0=a $
            \item $ \exists\, (-a)\in F $ such that $ a+(-a)=0 $
            \item $ a\cdot (b\cdot c)=(a\cdot b)\cdot c $
            \item $ a\cdot b=b\cdot a $
            \item $ \exists\, 1\in F $ such that $ a\cdot 1=a $
            \item $ \forall\,a\neq 0,\,\exists (a^{-1})\in F $ such that
                  $ a\cdot (a^{-1})=1 $
            \item $ a\cdot (b+c)=a\cdot b+a\cdot c $
        \end{enumerate}
    \end{definition}
\end{defbox}

\begin{defbox}
    \begin{definition}
        A field $ F $ is \textbf{infinite} if $ |F| $ is infinite.
    \end{definition}
\end{defbox}

\begin{defbox}
    \begin{definition}
        A field $ F $ is \textbf{finite} if $ |F| $ is finite.
    \end{definition}
\end{defbox}

\begin{defbox}
    \begin{definition}
        The \textbf{order} of a field $ F $, denoted $ \ord(F) $ is $ |F| $.
    \end{definition}
\end{defbox}

\begin{exbox}
    \begin{example}[Infinite and Finite Fields]
        $ \; $
        \begin{itemize}
            \item $ \mathbb{Q},\,\mathbb{R},\,\mathbb{C} $ are infinite fields.
            \item $ \mathbb{Z} $ is \textbf{not} a field since $ 3\in\mathbb{Z} $, but
                  $ \left(\frac{1}{3}\right)\notin\mathbb{Z} $.
        \end{itemize}
    \end{example}
\end{exbox}

\textbf{Question:} For what $ n\in\mathbb{Z}_{\geqslant 2} $ do there exists
finite fields of order $ n $? If a field of order $ n $ exists, how do
we ``construct'' it?

\textbf{Recall:} Let $ n\geqslant 2 $. The integers modulo $ n $,
$ \mathbb{Z}_n $ is the set of all equivalence classes $ \mod n $.
\[ \mathbb{Z}_n=\left\{ [0],[1],[2],\ldots ,[n-1]\right\} \]
where $ [a]+[b]=[a+b] $ and $ [a][b]=[ab] $.
More simply, $ \mathbb{Z}_{n}=\left\{ 0,1,\ldots ,n-1\right\} $
with addition and multiplication performed $ \mod n $.

\begin{exbox}
    \begin{example}[Modulo]
        Let $ \mathbb{Z}_9=\{0,1,\ldots ,8\} $.
        \begin{itemize}
            \item $ 5+7=3 $ (i.e. $ 5+7\equiv 3 \mod 9 $)
            \item $ 5\cdot 7=8 $ (i.e. $ 5+7\equiv 8 \mod 9 $)
        \end{itemize}
    \end{example}
\end{exbox}

\begin{defbox}
    \begin{definition}
        A \textbf{commutative ring} satisfies field axioms V1-V9 except
        V8.
    \end{definition}
\end{defbox}

\begin{thmbox}
    \begin{theorem}
        $ \mathbb{Z}_n $ is a commutative ring.
    \end{theorem}
\end{thmbox}

\begin{thmbox}
    \begin{theorem}
        $ \mathbb{Z}_n $ is a field if and only if $ n $ is prime.
    \end{theorem}
\end{thmbox}

\begin{proof}
    $ (\impliedby) $ Suppose $ n $ is prime. Let $ a\in\mathbb{Z}_n $, $ a\neq 0 $
    (i.e. $ 1\leqslant a\leqslant n-1 $). Since $ n $ is prime, $ \gcd(a,n)=1 $
    so $ \exists\,s,t\in\mathbb{Z} $ such that
    \[ as+nt=1 \]
    Reducing both sides $ \mod n $ gives
    \[ as\equiv 1 \mod n \]
    Define $ a^{-1}=s $. Thus, V8 is satisfied and hence $ \mathbb{Z}_n $
    is a field of order $ n $.

    $ (\implies) $ Suppose for a contradiction that $ n $ is composite, say $ n=ab $
    where $ 2\leqslant a,b\leqslant n-1 $. Suppose $ a^{-1} $ exists, and define
    $ a^{-1}=s $. Then,
    \[ as\equiv 1 \mod n\implies abs\equiv b\mod n\implies ns\equiv b \mod n
        \implies 0\equiv b\mod n \]
    So, $ n\mid b $ which is impossible. Therefore, $ a^{-1} $ does not exist, and hence
    $ \mathbb{Z}_n $ is not a field.
\end{proof}

\textbf{Question:} Do there exist finite fields of orders $ 4 $ and $ 6 $?

\begin{defbox}
    \begin{definition}
        The \textbf{characteristic} of a field, denoted $ \ch(F) $, is the smallest
        possible integer $ m $ such that
        \[ \underbrace{1+\cdots+1}_{m}=0 \]
        If no such $ m $ exists, then we define $ \ch(F)=0 $
    \end{definition}
\end{defbox}

\begin{exbox}
    \begin{example}[Characteristic of Fields] $ \; $
        \begin{itemize}
            \item $ \ch(\mathbb{Q})=0 $
            \item $ \ch(\mathbb{R})=0 $
            \item $ \ch(\mathbb{C})=0 $
            \item $ \ch(\mathbb{Z}_p)=p $ where $ p $ is prime.
        \end{itemize}
    \end{example}
\end{exbox}

\begin{thmbox}
    \begin{theorem}
        If $ \ch(F)=0 $, then $ F $ is infinite.
    \end{theorem}
\end{thmbox}

\begin{proof}
    Consider $ 1,1+1,\ldots,\underbrace{1+\cdots+1}_{a}\in F $.
    Suppose for a contradiction there exists distinct $ a,b\in\mathbb{Z} $
    such that
    \[ \underbrace{1+\cdots+1}_{a}=\underbrace{1+\cdots+1}_{b} \]
    where $ a>b $, then
    \[ \underbrace{1+\cdots+1}_{a}=\underbrace{1+\cdots+1}_{b}+
        \underbrace{1+\cdots+1}_{a-b}=\underbrace{1+\cdots+1}_{b} \]
    Hence, $ \underbrace{1+\cdots+1}_{a-b}=0\implies \ch(F)=(a-b) $
    which contradicts $ \ch(F)=0 $. Thus, $ F $ is infinite.
\end{proof}

\begin{thmbox}
    \begin{theorem}
        If $ F $ is a finite field, then $ \ch(F) $ is prime.
    \end{theorem}
\end{thmbox}

\begin{proof}
    Suppose for a contradiction that $ \ch(F)=m $ is composite,
    say $ m=ab $ where
    $ 2\leqslant a,b\leqslant m-1 $. Now
    \[ (\underbrace{1+\cdots+1}_{a})(\underbrace{1+\cdots+1}_{b})
        =\underbrace{1+\cdots+1}_{m}=0 \]
    since $ \ch(F)=m $. Let $ \underbrace{1+\cdots+1}_{a}=s $
    and $ \underbrace{1+\cdots+1}_{b}=t $, so $ st=0 $ where $ s\neq 0 $.
    Since $ \ch(F)=m>a $, there exists $ c\in F $ such that
    $ cs=1 \implies c=s^{-1} $. Therefore $ s^{-1}st=0 $. Thus,
    $ t=0 $ which is a contradiction to $ \ch(F)=m $.
\end{proof}

\textbf{Roadmap:} Let $ F $ be a finite field of order $ n $.
Then, $ \ch(F)=p $ where $ p $ is prime. Then, $ \mathbb{Z}_p $ is a subfield
of $ F $. $ F $ is a vector space over $ \mathbb{Z}_p $ of $ \dim=k $.
Then, order of $ F $ is $ p^k $.
