\makeheading{ 2020-01-27 }

\begin{exbox}
\begin{example}
    Consider a $ \underbrace{\text{binary}}_{F=GF(2)=\mathbb{Z}_2} $
    $ (\underbrace{5}_{n},\underbrace{3}_{k}) $-code $ C $. Then
    $ M=q^k=2^3 $ and $ R=\frac{k}{n} =\frac{3}{5} $.

    $ C=\langle \underbrace{10010}_{v_1},\underbrace{01011}_{v_2},\underbrace{00101}_{v_3} \rangle$.

    \[ G=
        \left[\begin{array}{ccc|cc}
                1 & 0 & 0 & 1 & 0 \\
                0 & 1 & 0 & 1 & 1 \\
                0 & 0 & 1 & 0 & 1
            \end{array}\right]_{3\times 5} \]
    $ \rank(G)=3 $, thus $ G $ is a generator matrix for $ C $.

    \begin{center}
        \begin{tabular}{| *{3}{>{\centering\arraybackslash}p{4cm} |}}
            \hline
            $ M $ (source messages) & $ \rightarrow $ & $ C $ (codewords) \\
            \hline
            $ 000 $                 & $ \rightarrow $ & $ 00000 $         \\
            $ 001 $                 & $ \rightarrow $ & $ 00101 $         \\
            $ 010 $                 & $ \rightarrow $ & $ 01011 $         \\
            $ 011 $                 & $ \rightarrow $ & $ 01110 $         \\
            $ 100 $                 & $ \rightarrow $ & $ 10010 $         \\
            $ 101 $                 & $ \rightarrow $ & $ 10111 $         \\
            $ 110 $                 & $ \rightarrow $ & $ 11001 $         \\
            $ 111 $                 & $ \rightarrow $ & $ 11100 $         \\
            \hline
        \end{tabular}
    \end{center}
    $ d(C)=2 $, $ e=0 $.
\end{example}
\end{exbox}

\myuline{Note:} Any matrix equivalent to $ G $ is also a generator matrix
for $ C $, but yields a different encoding rule.

\begin{defbox}
    \begin{definition}
        Let $ \left[\, I_k\mid A \,\right]_{k\times n} $ be a generator matrix
        for an $ (n,k) $-code $ C $. If an $ (n,k) $-code has a generator
        matrix of this form, then $ C $ is \textbf{systematic}, and the generator
        matrix is in \textbf{standard form}.
    \end{definition} \end{defbox}

\begin{exbox}
    \begin{example}
    $ C=\langle 100011,101010,100110\rangle $
    is a non-systematic $ (6,3) $-code.
    Some generator matrices are:
    \[ G_1=\left[
            \begin{array}{ccc|ccc}
                1 & 0 & 0 & 0 & 1 & 1 \\
                1 & 0 & 1 & 0 & 1 & 0 \\
                1 & 0 & 0 & 1 & 1 & 0
            \end{array} \right] \]
    $ R_2+R_1 $:
    \[ G_2= \left[
            \begin{array}{ccc|ccc}
                1 & 0 & 0 & 0 & 1 & 1 \\
                0 & 0 & 1 & 0 & 0 & 0 \\
                1 & 0 & 0 & 1 & 1 & 0
            \end{array} \right] \]
    $ R_3+R_1 $:
    \[ G_3=\left[
            \begin{array}{ccc|ccc}
                1 & 0 & 0 & 0 & 1 & 1 \\
                0 & 0 & 1 & 0 & 0 & 0 \\
                0 & 0 & 0 & 1 & 0 & 1
            \end{array} \right] \]
    Clearly $ C $ is not systematic. However, if every codeword
    is permuted by moving the second bit to the fourth bit, we get $ C^{\prime} $
    that is linear and has the same length, dimension, and distance as $ C $.
\end{example}
\end{exbox}

\begin{defbox}
    \begin{definition}
        Let $ C $ be an $ (n,k) $-code. If $ \pi $ is a permutation on
        $ \{1,\ldots ,n\} $. Then $ \pi(C) $ (that is, apply $ \pi $ to each
        codeword) is an $ (n,k) $-code which is said to be an \textbf{equivalent code}
        for $ C $.
    \end{definition} \end{defbox}

\begin{thmbox}
    \begin{theorem}
        \begin{enumerate}[(1)]
            \item If $ C $ and $ C^{\prime} $ are equivalent codes, then
                  \[ d(C)=d(C^{\prime}) \]
            \item Every linear code is equivalent to a systematic code.
        \end{enumerate}
    \end{theorem} \end{thmbox}

\begin{proof}
    Let $ C $ be an $ (n,k) $ code. Let $ G $ be a generator matrix for $ C $
    in RREF. Them, one can permute the columns of $ G $ to get a matrix
    $ G^{\prime}=\left[ \,I_k\mid A\, \right] $ in standard form. Then,
    $ G^{\prime} $ is a generator matrix for a code $ C^{\prime} $ that is
    equivalent to $ C $.
\end{proof}

\begin{defbox}
    \begin{definition}
        Let $ x,y\in V_n(F) $. The \textbf{inner product} of $ x $ and $ y $
        is $ x\cdot y=\sum\limits_{i=1}^{n} x_iy_i\in F $
    \end{definition} \end{defbox}

\begin{thmbox}
    \begin{theorem}
        If $ x,y,z\in V_n(F) $ and $ \lambda\in F $, then
        \begin{enumerate}[(1)]
            \item $ x\cdot y=y\cdot x $
            \item $ x\cdot (y+z)=x\cdot y+x\cdot y $
            \item $ (\lambda x)\cdot y=\lambda(x\cdot y) $
            \item $ x\cdot x=0$ does \textbf{not} imply $ x=0 $
        \end{enumerate}
    \end{theorem} \end{thmbox}

\begin{exbox}
    \begin{example}
    Consider $ V_2(\mathbb{Z}_2) $. Then, $ (1,1)\cdot(1,1)=0 $.
\end{example}
\end{exbox}

\begin{defbox}
    \begin{definition}
        Let $ C $ be an $ (n,k) $-code over $ F $. The \textbf{dual code}
        of $ C $ is
        \[ C^{\perp}=\left\{ x\in V_n(F):x\cdot c=0\,\forall c\in C\right\} \]
    \end{definition} \end{defbox}

\begin{thmbox}
    \begin{theorem}
        Let $ x\in V_n(F) $.
        \[ x\in C^{\perp}\iff v_1\cdot x=\cdots =v_k\cdot x=0 \]
    \end{theorem} \end{thmbox}

\begin{proof}
    $ (\implies) $ If $ x\in C^{\perp} $, then $ x\cdot c=0 $ for all
    $ c\in C $. In particular,
    \[ x\cdot v_1=\cdots =x\cdot v_k=0 \]

    $ (\impliedby) $ Suppose $ x\cdot v_1=\cdots =x\cdot v_k=0 $. Let $ c\in C $.
    We can write
    \[ c=\lambda_1v_1+\cdots+\lambda_kv_k \]
    for all $ \lambda_i\in F $. Then,
    \[ x\cdot c=\lambda_1(x\cdot v_1)+\cdots+\lambda_k(x\cdot v_k)=0 \]
    Hence, $ x\in C^{\perp} $.
\end{proof}

\begin{thmbox}
    \begin{theorem}
        If $ C $ is an $ (n,k) $-code over $ F $, then $ C^{\perp} $ is an
        $ (n,n-k) $-code over $ F $.
    \end{theorem} \end{thmbox}

\begin{proof}
    Consider
    \[ G=\begin{bmatrix}
            v_1    \\
            \vdots \\
            v_k
        \end{bmatrix}_{k\times n} \]
    Then, $ x\in C^{\perp} $ if and only if $ Gx^{\top}=0 $. So, $ C^{\perp} $
    is the nullspace of $ G $. Hence, $ C^{\perp} $ is an
    $ (n-k) $-dimensional subspace of $ V_n(F) $.
\end{proof}