\makeheading{ 2020-01-27 }



\begin{exbox}
    \begin{example}
        Consider a $ \underbrace{\text{binary}}_{F=GF(2)=\mathbb{Z}_2} $
        $ (\underbrace{5}_{n},\underbrace{3}_{k}) $-code $ C $. Then
        $ M=q^k=2^3 $ and $ R=\frac{k}{n} =\frac{3}{5} $.

        $ C=\langle \underbrace{10010}_{v_1},\underbrace{01011}_{v_2},\underbrace{00101}_{v_3} \rangle$.

        \[ G=
            \spalignaugmatn{2}{
                1 0 0 1 0; 0 1 0 1 1; 0 0 1 0 1}_{3\times{}5}
        \]
        $ \rank(G)=3 $, thus $ G $ is a generator matrix for $ C $.

        \begin{table}[H]
            \centering
            \begin{tabular}{@{}ccc@{}}
                Messages $ (M) $ & \textrightarrow{} & Codewords $ (C) $ \\
                \midrule
                000              & \textrightarrow{} & 00000             \\
                001              & \textrightarrow{} & 00101             \\
                010              & \textrightarrow{} & 01011             \\
                011              & \textrightarrow{} & 01110             \\
                100              & \textrightarrow{} & 10010             \\
                101              & \textrightarrow{} & 10111             \\
                110              & \textrightarrow{} & 11001             \\
                111              & \textrightarrow{} & 11100
            \end{tabular}
        \end{table}
        $ d(C)=2 $, $ e=0 $.
    \end{example}
\end{exbox}

\textbf{Note:} Any matrix equivalent to $ G $ is also a generator matrix
for $ C $, but yields a different encoding rule.

\begin{defbox}
    \begin{definition}
        Let $ \left[\, I_k\mid A \,\right]_{k\times n} $ be a generator matrix
        for an $ (n,k) $-code $ C $. If an $ (n,k) $-code has a generator
        matrix of this form, then $ C $ is \textbf{systematic}, and the generator
        matrix is in \textbf{standard form}.
    \end{definition} \end{defbox}

\begin{exbox}
    \begin{example}
        $ C=\langle 100011,101010,100110\rangle $
        is a non-systematic $ (6,3) $-code.
        Some generator matrices are:
        \[ G_1=\spalignaugmatn{3}{
                1 0 0 0 1 1 ;
                1 0 1 0 1 0 ;
                1 0 0 1 1 0} \]
        $ R_2+R_1 $:
        \[ G_2=\spalignaugmatn{3}{
                1 0 0 0 1 1 ;
                0 0 1 0 0 1 ;
                1 0 0 1 1 0} \]
        $ R_3+R_1 $:
        \[ G_3=\spalignaugmatn{3}{
                1 0 0 0 1 1 ;
                0 0 1 0 0 1 ;
                0 0 0 1 0 1} \]
        Clearly $ C $ is not systematic. However, if every codeword
        is permuted by moving the second bit to the fourth bit, we get $ C^{\prime} $
        that is linear and has the same length, dimension, and distance as $ C $.
    \end{example}
\end{exbox}

\begin{defbox}
    \begin{definition}
        Let $ C $ be an $ (n,k) $-code. If $ \pi $ is a permutation on
        $ \{1,\ldots ,n\} $. Then $ \pi(C) $ (that is, apply $ \pi $ to each
        codeword) is an $ (n,k) $-code which is said to be an \textbf{equivalent code}
        for $ C $.
    \end{definition} \end{defbox}

\begin{thmbox}
    \begin{theorem}
        \begin{enumerate}[label=(\arabic*)]
            \item If $ C $ and $ C^{\prime} $ are equivalent codes, then
                  \[ d(C)=d(C^{\prime}) \]
            \item Every linear code is equivalent to a systematic code.
        \end{enumerate}
    \end{theorem} \end{thmbox}

\begin{proof}
    Let $ C $ be an $ (n,k) $ code. Let $ G $ be a generator matrix for $ C $
    in RREF\@. Then, one can permute the columns of $ G $ to get a matrix
    $ G^{\prime}=\left[ \,I_k\mid A\, \right] $ in standard form. Then,
    $ G^{\prime} $ is a generator matrix for a code $ C^{\prime} $ that is
    equivalent to $ C $.
\end{proof}

\begin{defbox}
    \begin{definition}
        Let $ \bm{x},\bm{y}\in V_n(F) $. The \textbf{inner product}
        of $ \bm{x} $ and $ \bm{y} $ is
        \[ \bm{x}\cdot \bm{y}=\sum\limits_{i=1}^{n} x_i y_i\in F \]
    \end{definition} \end{defbox}

\begin{thmbox}
    \begin{theorem}
        If $ \bm{x},\bm{y},\bm{z}\in V_n(F) $ and $ \lambda\in F $, then
        \begin{enumerate}[label=(\arabic*)]
            \item $ \bm{x}\cdot \bm{y}=\bm{y}\cdot \bm{x} $
            \item $ \bm{x}\cdot (\bm{y}+\bm{z})=\bm{x}\cdot \bm{y}+\bm{x}\cdot \bm{z} $
            \item $ (\lambda x)\cdot \bm{y}=\lambda(\bm{x}\cdot \bm{y}) $
            \item $ \bm{x}\cdot \bm{x}=\bm{0}$ does \textbf{not} imply $ \bm{x}=\bm{0} $
        \end{enumerate}
    \end{theorem} \end{thmbox}

\begin{exbox}
    \begin{example}
        Consider $ V_2(\mathbb{Z}_2) $. Then, $ (1,1)\cdot(1,1)=0 $.
    \end{example}
\end{exbox}

\begin{defbox}
    \begin{definition}
        Let $ C $ be an $ (n,k) $-code over $ F $. The \textbf{dual code}
        of $ C $ is
        \[ C^{\perp}=\left\{ \bm{x}\in V_n(F):\bm{x}\cdot \bm{c}=\bm{0}\,\forall \bm{c}\in C\right\} \]
    \end{definition} \end{defbox}

\begin{thmbox}
    \begin{theorem}
        Let $ \bm{x}\in V_n(F) $.
        \[ \bm{x}\in C^{\perp}\iff \bm{v}_1\cdot \bm{x}=\cdots =\bm{v}_k\cdot \bm{x}=0 \]
    \end{theorem} \end{thmbox}

\begin{proof}
    $ (\implies) $ If $ \bm{x}\in C^{\perp} $, then $ \bm{x}\cdot \bm{c}=0 $ for all
    $ \bm{c}\in C $. In particular,
    \[ \bm{x}\cdot \bm{v}_1=\cdots =\bm{x}\cdot \bm{v}_k=0 \]

    $ (\impliedby) $ Suppose $ \bm{x}\cdot \bm{v}_1=\cdots =\bm{x}\cdot \bm{v}_k=0 $. Let $ \bm{c}\in C $.
    We can write
    \[ \bm{c}=\lambda_1\bm{v}_1+\cdots+\lambda_k\bm{v}_k \]
    for all $ \lambda_i\in F $. Then,
    \[ \bm{x}\cdot \bm{c}=\lambda_1(\bm{x}\cdot \bm{v}_1)+\cdots+\lambda_k(\bm{x}\cdot \bm{v}_k)=0 \]
    Hence, $ \bm{x}\in C^{\perp} $.
\end{proof}

\begin{thmbox}
    \begin{theorem}
        If $ C $ is an $ (n,k) $-code over $ F $, then $ C^{\perp} $ is an
        $ (n,n-k) $-code over $ F $.
    \end{theorem} \end{thmbox}

\begin{proof}
    Consider
    \[ G=\begin{bmatrix}
            v_1    \\
            \vdots \\
            v_k
        \end{bmatrix}_{k\times n} \]
    Then, $ \bm{x}\in C^{\perp} $ if and only if $ G \bm{x}^{\top}=\bm{0} $. So, $ C^{\perp} $
    is the nullspace of $ G $. Hence, $ C^{\perp} $ is an
    $ (n-k) $-dimensional subspace of $ V_n(F) $.
\end{proof}
