\makeheading{ 2020-01-24 }

\begin{thmbox}
    \begin{theorem}
        If $ GF(q)^* $ has order $ t $, then
        \[ \alpha^1,\ldots,\alpha^{t-1} \]
        are pairwise distinct.
    \end{theorem}
\end{thmbox}

\begin{proof}
    Suppose for a contradiction that $ \alpha^i=\alpha^j $ where $ 0\leqslant i,j\leqslant t-1 $.
    WLOG suppose $ j>i $, then $ \alpha^{j-i}=1 $ which contradicts $ \ord(\alpha)=t $
    since $ 1\leqslant j-i\leqslant t-1 $.
\end{proof}

\section{Existence of Generators (Optional)}

\begin{thmbox}
    \begin{lemma}
        Let $ \alpha\in GF(q)^* $ with $ \ord(\alpha)=t $. Then $ \ord(\alpha^i)=
            t/\gcd(t,i) $.
    \end{lemma}
\end{thmbox}

\begin{proof}
    Let $ d=\gcd(t,i) $. The order of $ a^i $ is the smallest positive
    integer $ s $ such that $ \alpha^{is}=1 $. Now,
    \[ \alpha^{is}=1\iff t\mid is\iff \frac{t}{d}\mid \frac{i}{d}s
        \iff \frac{t}{d} \mid s \]
    Since the smallest positive integer $ s $ satisfying $ \frac{t}{d} \mid s $
    is $ s=\frac{t}{d} $, we have $ \ord(\alpha^i)=\frac{t}{d} $.
\end{proof}

\begin{thmbox}
    \begin{lemma}
        Let $ \alpha,\beta\in GF(q)^* $, with $ \ord(\alpha)=m $ and $ \ord(\beta)=n $.
        If $ \gcd(m,n)=1 $ then $ \ord(\alpha\beta)=mn $.
    \end{lemma}
\end{thmbox}
\begin{proof}
    Let $ t=\ord(\alpha\beta) $. Now,
    \[ (\alpha\beta)^{mn}=\alpha^{mn}\beta^{mn}=1, \]
    so $ t\mid mn $. Also,
    \[ 1=(\alpha\beta)^{tn}=\alpha^{tn}\beta^{tn}=\alpha^{tn}, \]
    so $ m\mid tn $. And, since $ \gcd(m,n)=1 $, we have $ m\mid t $. Similarly,
    \[ 1=(\alpha\beta)^{tm}=\alpha^{tm}\beta^{tm}=\beta^{tm}, \]
    so $ n\mid tm $. And, since $ \gcd(m,n)=1 $, we have $ n\mid t $. Hence, since $ \gcd(m,n)=1 $,
    we have $ mn\mid t $. Thus $ t=mn $.
\end{proof}

\begin{thmbox}
    \begin{theorem}
        Every finite field $ GF(q) $ has a generator.
    \end{theorem}
\end{thmbox}

\begin{proof}
    Let $ \alpha $ be an element of highest order in $ GF(q)^* $;
    say $ \ord(\alpha)=t $. Suppose that $ t<(q-1) $.

    If the order of every element in $ GF(q)^* $ were to divide $ t $ then the equation
    $ y^t-1=0 $ would have $ q-1 $ roots in $ GF(q) $, which is impossible
    since $ (q-1)>t $. Hence there exists an element $ \beta\in GF(q)^* $
    whose order $ b $ does not divide $ t $.

    Now, let $ \ell $ be a prime such that the highest power of $ \ell $
    which divides $ b $ (say $ \ell^e $) is greater than the highest
    power of $ \ell $ which divides $ t $ (say $ \ell^f $) --- such a prime
    $ \ell $ must exist since $ b $ does not divide $ t $.

    Consider the field elements $ \alpha^{\prime}=\alpha^{\ell^f} $
    and $ \beta^{\prime}=\beta^{b/\ell^e} $. We have
    \[ \ord(\alpha^{\prime})=\frac{t}{\gcd(t,\ell^f)} =\frac{t}{\ell^f} \]
    and
    \[ \ord(\beta^{\prime})=\frac{b}{\gcd(b,\ell^e)} =\frac{b}{b/\ell^e}=\ell^e \]
    Since $ \gcd(t/\ell^f,\ell^e)=1 $, we have $ \ord(\alpha^{\prime}\beta^{\prime})=
        (t/\ell^f)(\ell^e)=t\ell^{e-f}>t $. This contradicts the hypothesis
    that the highest order of any element in $ GF(q)^* $ is $ t $. Hence the
    hypothesis that $ t<(q-1) $ is wrong, and so $ t=q-1 $. Thus $ \alpha $
    is a generator of $ GF(q)^* $.
\end{proof}

\chapter{Linear Codes}
\section{Introduction}
Let $ F=GF(q) $. Let $ V_n(F)=\underbrace{F\times\cdots\times F}_{n}=F^n $.
Then, $ V_n(F) $ is an $ n $-dimensional vector space over $ F $ and
we have $ |V_n(F)|=q^n $.

\begin{defbox}
    \begin{definition}
        Let $ F=GF(q) $.
        A \textbf{linear $ \bm{(n,k)} $-code} over $ F $ is an $ n $-dimensional subspace
        of $ V_n(F) $.
    \end{definition}
\end{defbox}

\begin{defbox}
    \begin{definition}
        A \textbf{subspace} of a vector space $ V $ over $ F $ is a subset
        $ S\subseteq V $ such that
        \begin{enumerate}[V1]
            \item $ \bm{0}\in S \implies S\neq \varnothing $
            \item $ \bm{v}_1+\bm{v}_2\in S $, $ \forall \bm{v}_1,\bm{v}_2\in S $
            \item $ \lambda \bm{v}\in S $, $ \forall \lambda\in F $ and $ \bm{v}\in S $
        \end{enumerate}
        Note that $ S\subseteq V $ is also a vector space over $ F $.
    \end{definition}
\end{defbox}

Let $ C $ be an $ (n,k) $-code over $ F $. Let $ v_1,\ldots,v_k $ be
an ordered basis for $ C $.
\begin{enumerate}[(1)]
    \item The codewords in $ C $ are precisely:
          \[ m_1\bm{v}_1+\cdots +m_k\bm{v}_k \]
          where $ m_i\in F $. So, $ |C|=M=q^k $ since there are $ q $ choices for each $ m $.
          The length of $ C $ is $ n $ and has dimension $ k $.
    \item The rate of $ C $ is
          \[ R=\frac{\log_q(M)}{n} =\frac{k}{n} \]
\end{enumerate}

\begin{defbox}
    \begin{definition}
        The \textbf{Hamming weight} of $ \bm{v}\in V_n(F) $, denoted
        $ w(\bm{v}) $ is the number of non-zero coordinate positions in $ V $.
    \end{definition}
 \end{defbox}

\begin{defbox}
    \begin{definition}
        The \textbf{Hamming weight of an $ (n,k) $-code} $ C $ is:
        \[ w(C)=\min \left\{ w(\bm{c}):\bm{c}\in C,\,\bm{c}\neq \bm{0}\right\} \]
    \end{definition} 
\end{defbox}

\begin{thmbox}
    \begin{theorem}
        If $ C $ is a linear code, then $ d(C)=w(C) $.
    \end{theorem} 
\end{thmbox}

\begin{proof}
    \begin{align*}
        d(C)
         & =\min \left\{ d(\bm{x},\bm{y}):\bm{x},\bm{y}\in C,\,\bm{x}\neq \bm{y}\right\}                                     \\
         & =\min \left\{ w(\bm{x}-\bm{y}):\bm{x},\bm{y}\in C,\,\bm{x}\neq \bm{y}\right\}\qquad\text{by (A2Q1a)}              \\
         & =\min 
         \left\{ w(\bm{c}):\bm{c}\in C,\,\bm{c}\neq \bm{0}\right\}\qquad\text{since $C$ is a vector space} \\
         & =w(C)
    \end{align*}
\end{proof}

\section{Generator Matrices and the Dual Code}

Since $ M=q^k $, there are $ q^k $ source messages. We'll assume that the source
messages are elements of $ V_k(F) $. Then, a natural encoding rule is,
given $ (m_1,\ldots ,m_k)\in V_k(F) $ we'll encode the message as
\[ c=m_1\bm{v}_1+\cdots+m_k\bm{v}_k \]
The encoding rule depends on the basis chosen for $ C $.

If $ m=(m_1,\ldots ,m_k) $, then the encoding rule can be written as follows:
\begin{align*}
    C & =(m_1,\ldots ,m_k)
    \begin{bmatrix}
        -v_1-  \\
        -v_2-  \\
        \vdots \\
        -v_k-
    \end{bmatrix}_{k\times n} \\
      & =mG
\end{align*}
Note that $ v_i $ are row vectors in this course.

\begin{defbox}
    \begin{definition}
        Let $ C $ be an $ (n,k) $-code. A \textbf{generator matrix} $ G $
        for $ C $ is a $ k\times n $ matrix whose rows form a basis for $ C $.
    \end{definition} \end{defbox}

\textbf{Note:} An encoding rule for $ C $ with respect to $ G $ is $ C=mG $.
Performing elementary row operations on $ G $ gives
a different matrix for the same code $ C $ due to the order of the basis.
