\chapter{Error Correction Techniques and Digital Audio Recording}
\makeheading{2020-04-01}
\section{Reed-Solomon Codes}
Invented by Irving Reed and Gustave Solomon in 1960.
\begin{defbox}
    \begin{definition}
        A \textbf{\emph{Reed-Solomon (RS) code}} is a BCH code
        of length $ n $ over $ GF(q) $ where $ n\mid (q-1) $.

        \underline{Note}: $ m=1 $ since $ q^1\equiv 1\mod n $
    \end{definition}
\end{defbox}
\begin{exbox}
    \begin{example}
        Let $ q=2^4 $ and $ GF(2^4)=\mathbb{Z}_2/(\alpha^4+\alpha+1) $.
        Recall that $ \alpha $ is a generator of $ GF(2^4)^* $.

        Let $ \beta=\alpha^3 $, then $ \ord(\beta)=5 $, so $ q=16 $
        and $ n=5 $.

        Let
        \begin{align*}
            g(x) & =\lcm \left\{ m_{\beta}(x),m_{\beta^2}(x),m_{\beta^3}(x)\right\} \\
                 & =(x-\beta)(x-\beta^2)(x-\beta^3)                                 \\
                 & =x^3+\alpha^{11}x^2+\alpha^2x+\alpha^3
        \end{align*}
        Then, $ g(x) $ generates a $ (5,2) $-RS code $ C $ over $ GF(2^4) $
        with $ \delta=4 $. In fact, $ d(C)=4 $ since $ g(x) $
        is a codeword of weight $ 4 $. A generator matrix for $ C $ is
        \[ G=
            \begin{bmatrix}
                \alpha^3 & \alpha^2 & \alpha^{11} & 1           & 0 \\
                0        & \alpha^3 & \alpha^2    & \alpha^{11} & 1
            \end{bmatrix}_{2\times 5} \]
        Consider the code $ C^{\prime} $ obtained from $ C $
        by replacing each symbol in codewords of $ C $ by their binary vector representation.
        For example,
        \[ (\alpha^3,\alpha^2,\alpha^{11},1,0)\longleftrightarrow (0001\; 0010\; 0111\; 1000\; 0000) \]
        It is not hard to see that $ C^{\prime} $ is closed under vector addition and scalar
        multiplication over $ GF(2) $. Thus, $ C^{\prime} $ is a $ (20,8) $-binary code.
    \end{example}
\end{exbox}
\begin{defbox}
    \begin{definition}
        Suppose $ n\mid (q-1) $, and let $ \beta\in GF(q) $ be an element of order $ n $.
        Then, $ m_{\beta^i}(x)=x-\beta^i $ for all $ i $. A
        \textbf{\emph{RS code $ \bm{C} $ of length $ \bm{n} $ over $ \bm{GF(q)} $
                with designed distance $ \bm{\delta} $}} is a cyclic code over $ GF(q) $
        with generator polynomial
        \[ g(x)=(x-\beta^{a})(x-\beta^{a+1})(x-\beta^{a+2})\cdots
            (x-\beta^{a+\delta-2}) \]
        for some $ a $. Since $ \deg(g)=\delta-1 $, we have $ w(g)\leqslant \delta $,
        so $ d(C)\leqslant \delta $. By the BCH bound, $ d(C)\geqslant \delta $,
        hence $ d(C)=\delta $.

        Since $ \dim(C)=k=n-\deg(g)=n-\delta+1 $, we have $ k=n-d+1 $,
        so $ d=n-k+1 $. Recall that $ d\leqslant n-k+1 $ for any $ (n,k,d) $-code.
        Thus, \underline{RS are optimal} in the sense that, for any fixed $ n,k,q $,
        they achieve maximum distance among all $ (n,k,d) $-codes over $ GF(q) $.
    \end{definition}
\end{defbox}

\underline{RS codes have good (cyclic) burst error correcting capability}

Let $ C $ be a RS code of length $ n $ over $ GF(2^r) $ and designed
distance $ \delta $. Consider $ \bm{c}=(c_1,c_2,\ldots ,c_n)\in C $,
and let $ e=\lfloor\frac{d-1}{2} \rfloor =\lfloor\frac{n-k}{2} \rfloor $.
Note that $ \bm{c}_i\in GF(2^r) $.

By writing each $ \bm{c}_i $ as a binary vector of length
$ r $, we can view $ \bm{c} $ as a binary vector of length $ nr $ bits.

Now, if $ \bm{c} $ is transmitted and if a cyclic burst error of length
$ \leqslant 1+(e-1)r $ bits is introduced, then at most $ e $
symbols of $ \bm{c} $ are received incorrectly. Thus, the received
word can be decoded correctly.

\begin{thmbox}
    \begin{theorem}
        Let $ C $ be an $ (n,k) $-RS code over $ GF(2^r) $. Then
        $ C^{\prime} $, the code obtained by replacing each symbol
        in the codewords of $ C $ by the $ r $-bit binary representations,
        is a binary $ (nr,kr) $-code with c.b.e.c.c $ 1+(e-1)r $
        where $ e=\lfloor \frac{n-k}{2} \rfloor $.
    \end{theorem}
\end{thmbox}

\begin{exbox}
    \begin{example}
        Consider $ GF(2^8)=\mathbb{Z}_2[\alpha]/(\alpha^8+\alpha^4+\alpha^3+\alpha^2+1) $.
        Then $ \beta=\alpha $ has order $ n=255 $ (so $ q=256,n=255 $).
        Let
        \[ g(x)=\prod_{i=1}^{24}(x-\beta^i) \]
        Then $ g(x) $ is the generator polynomial for a $ (255,231,25) $-RS
        code $ C $ with e.c.c $ e=12 $. The related code $ C^{\prime} $
        is a $ (2040,1848) $-binary code with c.b.e.c.c $ 89 $.
    \end{example}
\end{exbox}
The code $ C $, and others derived from it, have widely been used in practice,
including in CDs, DVDs, and QR codes.
