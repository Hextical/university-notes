\makeheading{2020-02-03}
For fixed $ n,q,d $, a perfect code maximizes
\[ R=\frac{\log_q(M)}{n} \]

\begin{Example}{}{}
    \begin{itemize}
        \item $ GF(q)^n $ is a trivial perfect code with $ d=1 $.
        \item $ C=\set{\underbrace{0\cdots 0}_{n},\underbrace{1\cdots 1}_{n}} $
              over $ \mathbb{Z}_2 $ is a perfect code if and only if $ n $ is odd.
    \end{itemize}
\end{Example}

\begin{Exercise}{}{ex1_a2b}
    Prove that every perfect code must have odd distance
    (without referring to~\Cref{thm:tiet}).
\end{Exercise}

\begin{Proof}{\Cref{exercise:ex1_a2b}}{}
    Let $ C $ be an even code of even distance $ d=2t $. Then, $ e=\lfloor(d-1)/2\rfloor=t-1 $.
    Let $ \symbf{c}\in C $ and $ \symbf{r} $ be a vector such that $ d(\symbf{c},\symbf{r})=t $.
    Note that $ \symbf{r} $ is not in the sphere of radius $ e $ centred at $ \symbf{c} $.
    Now, if $ \symbf{r} $ were in the sphere of radius $ e $ centred at some codeword
    $ \symbf{c}^\prime\neq\symbf{c} $, then we would have
    \[ d(\symbf{c},\symbf{c}^\prime)\leqslant d(\symbf{c},\symbf{r})+d(\symbf{r},\symbf{c}^\prime)\leqslant t+e
        <d \]
    which is impossible since the distance of $ C $ is $ d $. Hence, $ \symbf{r} $ is not contained
    in any of the radius-$ e $ spheres centred at codewords, and so $ C $ is not
    a perfect code. It follows that a perfect code must have odd distance.
\end{Proof}

\begin{Theorem}{Tietäväinen, 1973}{tiet}
    The only perfect codes are:
    \begin{enumerate}[label=(\arabic*)]
        \item $ V_n(GF(q)) $.
        \item The binary replication code of odd length.
        \item The $ (23,12,7) $-binary Golay code and all codes equivalent to it.
        \item The $ (11,6,5) $-ternary Golay code and all codes equivalent to it.
              A generator matrix for this code is:
              \[ G=\spalignaugmatn{1}{
                  I_6
                  {\begin{matrix}
                      1 & 1 & 1 & 1 & 1 \\
                      0 & 1 & 2 & 2 & 1 \\
                      1 & 0 & 1 & 2 & 2 \\
                      2 & 1 & 0 & 1 & 2 \\
                      2 & 2 & 1 & 0 & 1 \\
                      1 & 2 & 2 & 1 & 0
                  \end{matrix}}}_{6\times{} 11}
              \]
        \item The Hamming codes and all codes of the same $ [n,M,d] $ parameters as them
              with $ d=3 $.
    \end{enumerate}
\end{Theorem}

\begin{Definition}{Hamming code of order $ \symbf{r} $ over $ \symbf{GF(q)} $}{}
    A \textbf{Hamming code of order $\symbf{r}$ over $\symbf{GF(q)}$}
    is a linear code over $ GF(q) $ with $ n=\dfrac{q^r-1}{q-1}, $ $ k=n-r $
    and an $ r\times n $ PCM matrix whose columns are non-zero,
    and no two columns are scalar multiples of each other.
\end{Definition}

\begin{Example}{}{}
    A Hamming code of order $ r=3 $ over $ GF(3) $ is a $ (13,10,3) $-code over $ GF(3) $ with
    PCM\@:
    \[ H=\spalignaugmat{
        I_3 {\begin{matrix}
            1 & 0 & 1 & 2 & 2 & 0 & 2 & 1 & 2 & 1 \\
            1 & 1 & 0 & 1 & 0 & 1 & 2 & 2 & 1 & 1 \\
            0 & 1 & 1 & 0 & 1 & 2 & 1 & 2 & 2 & 1
        \end{matrix}}}_{3\times{} 13} \]
\end{Example}

\textbf{Observations}:
\begin{enumerate}[label=(\roman*)]
    \item For every non-zero vector $ \symbf{v}\in V_r(GF(q)) $, exactly one scalar multiple of $ \symbf{v} $
          must be a column of a PCM (for the Hamming code of order $ r $ over $ GF(q) $).
    \item The dimension of the code is indeed $ k $ since $ \rank(\text{PCM})=r=n-k $
          since $ \lambda_i e_i $ are columns of the PCM\@.
    \item The Hamming codes have distance 3.
\end{enumerate}

\begin{Theorem}{}{hamming_perfect}
    Hamming codes are perfect.
\end{Theorem}

\begin{Proof}{\Cref{thm:hamming_perfect}}{}
    Recall that Hamming codes have $ e=1 $ and $ n=\frac{q^r-1}{q-1} $ with $ r=n-k $.
    \begin{align*}
        M \sum\limits_{i=0}^{e} \binom{n}{i}(q-1)^i
         & =q^{n-r}(1+n(q-1))                               \\
         & =q^{n-r}\left( 1+\frac{q^r-1}{q-1} (q-1) \right) \\
         & = q^{n-r}(1+q^r-1)                               \\
         & =q^{n-r}q^r                                      \\
         & =q^n
    \end{align*}
\end{Proof}

\begin{Definition}{Error vector}{error_vector}
    Suppose $ \symbf{c}\in C $ is transmitted. Suppose $ \symbf{r}\in V_n(F) $ is received.
    Then, the \textbf{error vector} is $ \symbf{e}=\symbf{r}-\symbf{c} $.
\end{Definition}

\begin{Example}{Error Vector}{}
    Over $ \mathbb{Z}_3 $, if $ \symbf{c}=(120212) $ is sent, and $ \symbf{r}=(122102) $ is received, then
    the error vector is $ \symbf{e}=(002220) $.
\end{Example}
