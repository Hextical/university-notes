\chapter{Some Special Linear Codes}
\makeheading{2020-02-07}
\begin{defbox}
    \begin{definition}
        A linear code $ C $ is \textbf{self-orthogonal} if $ C\subseteq C^{\perp} $.
    \end{definition}
\end{defbox}
\begin{defbox}
    \begin{definition}
        A linear code $ C $ is \textbf{self-dual} if $ C=C^{\perp} $.
    \end{definition}
\end{defbox}
For a binary $ (n,k) $-code $ C $, the syndrome table has
size $ 2^{n-k}\times n $ which is exponentially large.

\textbf{Goal}: Design decoding algorithm which require
very little space.

\begin{exbox}
    \begin{example}
        Use only the PCM $ H $ which is $ (n-k)\times n $ bits.
    \end{example}
\end{exbox}

\section{The Binary Golay Code C23 (1949)}
Let
\[
    \hat{B}=
    \left[
        \begin{array}{cccccccccccc}
            1 & 1 & 1 & 1 & 1 & 1 & 1 & 1 & 1 & 1 & 1 \\
            1 & 1 & 0 & 1 & 1 & 1 & 0 & 0 & 0 & 1 & 0 \\
            1 & 0 & 1 & 1 & 1 & 0 & 0 & 0 & 1 & 0 & 1 \\
            0 & 1 & 1 & 1 & 0 & 0 & 0 & 1 & 0 & 1 & 1 \\
            1 & 1 & 1 & 0 & 0 & 0 & 1 & 0 & 1 & 1 & 0 \\
            1 & 1 & 0 & 0 & 0 & 1 & 0 & 1 & 1 & 0 & 1 \\
            1 & 0 & 0 & 0 & 1 & 0 & 1 & 1 & 0 & 1 & 1 \\
            0 & 0 & 0 & 1 & 0 & 1 & 1 & 0 & 1 & 1 & 1 \\
            0 & 0 & 1 & 0 & 1 & 1 & 0 & 1 & 1 & 1 & 0 \\
            0 & 1 & 0 & 1 & 1 & 0 & 1 & 1 & 1 & 0 & 0 \\
            1 & 0 & 1 & 1 & 0 & 1 & 1 & 1 & 0 & 0 & 0 \\
            0 & 1 & 1 & 0 & 1 & 1 & 1 & 0 & 0 & 0 & 1
        \end{array}
        \right]_{12\times 11} \]
Then, $ \hat{G}=\left[\; I_{12}\mid \hat{B}\; \right]_{12\times 23} $
is a generator matrix for a $ (23,12) $-binary code called $ C_{23} $.

\textbf{Note}: In $ \hat{B} $,
\begin{itemize}
    \item $ R_1 $ in only contains $ 1 $'s.
    \item $ R_3 $ to $ R_{12} $ are left cyclic shifts of $ R_2 $.
\end{itemize}

\begin{thmbox}
    \begin{theorem}
        Facts:
        \begin{enumerate}
            \item $ d(C_{23})=7 $.
            \item $ C_{23} $ is perfect.
        \end{enumerate}
    \end{theorem}
\end{thmbox}

\begin{proof}
    We know that $ e=3 $, so
    $ 2^{12}\left[ \binom{23}{0}+\binom{23}{1}+\binom{23}{2}+\binom{23}{3} \right]=
        2^{23} $.
\end{proof}

\section{The Extended Golay Code C24}
Let
\[ B=
    \left[
        \begin{array}{c|c}
            0 & \hat{B} \\
            \bm{1}_{11}
        \end{array}
        \right]_{12\times 12}
\]
Then, $ G=\left[\; I_{12}\mid B \;\right]_{12\times 24} $ is a generator
matrix for a $ (24,12) $-binary code called $ C_{24} $.

Notes:
\begin{enumerate}[(i)]
    \item $ C_{24} $ is a $ (24,12,8) $-binary code ($ e=3 $)
    \item $ GG^\top=\bm{0} $
    \item $ C_{24}\subseteq C_{24}^\perp $, $ C_{24} $ is a self-orthogonal code.
    \item $ \dim (C_{24})=12 $ and $ d(C^{\perp})=12 $, so
          $ C_{24}=C_{24}^\perp $ ($ C_{24} $ is a self-dual code)
    \item $ B $ is symmetric
    \item PCM for $ C_{24} $ is $ H=\left[ -B ^\top \mid I_{12} \right]=
              \left[ B\mid I_{12} \right] $
    \item Since $ C_{24}=C_{24}^\perp $, $ H $ is also a GM for $ C_{24} $.
    \item $ G $ is also a PCM for $ C_{24}^\perp $.
\end{enumerate}

\subsection{Decoding Algorithm for C24}
Compute a syndrome of $ \bm{r} $. Find a vector $ \bm{e} $ with $ w(\bm{e})\leqslant 3 $,
that has the same syndrome as $ \bm{r} $. If no such $ \bm{e} $ exists, then
reject $ \bm{r} $, otherwise decode $ \bm{r} $ to $ \bm{c}=\bm{r}-\bm{e} $.

Let $ \bm{r}=(\bm{x},\bm{y}) $ and $ \bm{e}=(\bm{e}_1,\bm{e}_2) $.
There are five (not mutually exclusive) cases to consider. In the event that
$ w(\bm{e})\leqslant 3 $,
\begin{enumerate}[(A))]
    \item $ w(\bm{e}_1)=0 $, $ w(\bm{e}_2)=0 $
    \item $ 1\leqslant w(\bm{e}_1)\leqslant 3 $, $ w(\bm{e}_2)=0 $
    \item $ w(\bm{e}_1)=1 $ or $ 2 $, $ w(\bm{e}_2)=1 $
    \item $ w(\bm{e}_1)=0 $, $ 1\leqslant w(\bm{e}_2)\leqslant 3 $
    \item $ w(\bm{e}_1)=1 $, $ w(\bm{e}_2)=1 $ or $ 2 $
\end{enumerate}

\begin{thmbox}
    \begin{theorem}
        Let $ C $ be an $ (n,k,d) $-code over $ GF(q) $. Let $ \bm{x}=
            V_n(GF(q)) $ with $ w(\bm{x})\leqslant \lfloor \frac{d-1}{2} \rfloor $.
        Then $ \bm{x} $ is the unique vector of minimum weight in the coset
        of $ C $ containing $ \bm{x} $ (so, it must be a coset leader).
    \end{theorem}
\end{thmbox}

\begin{proof}
    Suppose for a contradiction that
    $ \bm{y} $ is a vector in the same coset of $ C $ as $ \bm{x} $
    with $ \bm{y}\neq \bm{x} $ and
    \[ w(\bm{y})\leqslant w(\bm{x})\leqslant \left\lfloor \frac{d-1}{2} \right\rfloor \]
    Then, $ \bm{y}-\bm{x}\neq 0 $, $ \bm{x}\equiv \bm{y}\mod C $,
    and so $ \bm{x}-\bm{y}\in C $. Now,
    \begin{align*}
        w(\bm{x}-\bm{y})=w(\bm{x}+(-\bm{y}))
         & \leqslant w(\bm{x})+w(\bm{y})                                                                 \\
         & =w(\bm{x})+w(\bm{y})                                                                          \\
         & \leqslant \left\lfloor \frac{d-1}{2} \right\rfloor + \left\lfloor \frac{d-1}{2} \right\rfloor \\
         & \leqslant d-1
    \end{align*}
    contradicting $ d(C)=d $.
\end{proof}
