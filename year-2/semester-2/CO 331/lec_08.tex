\section{2020-01-22}
\begin{exbox}
    \subsection{Example}
    Construct $ GF(2^4=16) $.
    
    \textbf{Solution:} Take $ f(x)=x^4+x+1\in\mathbb{Z}_2[x] $.
    \begin{itemize}
        \item $ f $ has no roots in $ \mathbb{Z}_2 $ and hence no linear factors
        \item long division shows that $ x^2+x+1\nmid x^4+x+1 $, so $ f $
        has no irreducible quadratic factors
        \item $ f $ is irreducible over $ \mathbb{Z}_2 $.
    \end{itemize}
    Thus, $ GF(16)=\mathbb{Z}_2[x]/(x^4+x+1) $.
\end{exbox}

\subsection{Properties of Finite Fields}
\begin{thmbox}
    \begin{theorem}[Frosh's Dream]
    Let $ \alpha,\beta\in GF(q) $ where $ \ch(GF(q))=p $.
    \[ (\alpha + \beta)^p=\alpha^p+\beta^p \]
\end{theorem} \end{thmbox}

\begin{proof}
    \[ (\alpha + \beta)^p=\alpha^p+\sum\limits_{i=1}^{p-1}
    \binom{p}{i}\alpha^i\beta^{p-i}+\beta^p \]
    Now,
    \[ \binom{p}{i}=\frac{p(p-1)(p-2)\cdots (p-i+1)}{(1\cdot 2\cdot 3\cdots i)}
    \in\mathbb{N} \]
    If $ 1\leqslant i\leqslant p-1 $ then $ p\mid $ numerator, but
    $ p \nmid $ denominator. Thus,
    \[ p\mid \binom{p}{i} \]
    \begin{align*}
        \binom{p}{i}\alpha^i\beta^{p-i}
        &=\underbrace{\alpha^i\beta^{p-i}+\cdots+\alpha^i\beta^{p-i}}_{\binom{p}{i}}\\
        &=\alpha^i\beta^{p-i}\left( \underbrace{1+\cdots+1}_{\binom{p}{i}} \right)\\
        &=\alpha^i\beta^{p-i}\\
        &=0
    \end{align*}
\end{proof}

\begin{thmbox}
    \begin{theorem}
    \subsection{Corollary}
    \[ (\alpha+\beta)^{p^m}=\alpha^{p^m}+\beta^{p^m} \]
    for all $ m\geqslant 1 $.
\end{theorem} \end{thmbox}

\begin{proof}
    Exercise. Hint: Induction on $ m $.
\end{proof}

\begin{thmbox}
    \begin{theorem}
    Let $ \alpha\in GF(q) $. Then
    \[ \alpha^q=\alpha \]
\end{theorem} \end{thmbox}

\begin{proof}
    If $ \alpha=0 $, then $ \alpha^q=0=\alpha $.

    Suppose $ \alpha\neq 0 $. Let $ \alpha_1,\ldots ,\alpha_{q-1} $ be the
    non-zero elements in $ GF(q) $. Consider
    \[ \alpha\alpha_1+\cdots+\alpha\alpha_k \]
    Note that the elements in this list are pairwise distinct because if
    $ \alpha\alpha_i=\alpha\alpha_j $ with $ i\neq j $, then
    \[ \alpha^{-1}\alpha\alpha_i=\alpha^{-1}\alpha\alpha_j \]
    which implies that $ \alpha_i=\alpha_j $ which is a contradiction.
    Also $ \alpha\alpha_i\neq 0 $ for all $ 1\leqslant i\leqslant q-1 $.
    Hence, $ \{\alpha_1,\ldots ,\alpha_{q-1}\}=\{\alpha\alpha_1,\ldots ,\alpha\alpha_{q-1}\} $
    Therefore, $ \alpha_1\cdots\alpha_{q-1}=(\alpha\alpha_1)\cdots(\alpha\alpha_{q-1}) $.
    Hence, $ \alpha^{q-1}=1 $. Thus, $ \alpha^q=\alpha $.
\end{proof}

\begin{defbox}
    \begin{definition}
    Let $ GF(q)^*=GF(q)/\{0\} $.
\end{definition} \end{defbox}

\begin{defbox}
    \begin{definition}
    Let $ \alpha\in GF(q)^* $. The \textbf{order of $\alpha$}, denoted
    $ \ord(\alpha) $ is the smallest positive integer $ t $ such that
    $ \alpha^t=1 $.
\end{definition} \end{defbox}

\begin{exbox}
    \subsection{Example}
    How many elements of order $ 1 $ are there in $ GF(q) $?

    \textbf{Solution:} $ \alpha=1 $
\end{exbox}

\begin{exbox}
    \subsection{Example}
    Find $ \ord(x) $ in $ GF(16)=\mathbb{Z}_2/(x^4+x+1) $.

    \textbf{Solution:}
    \begin{itemize}
        \item $ x^1=x $
        \item $ x^2=x^2 $
        \item $ x^3=x^3 $
        \item $ x^4=x+1 $
        \item $ x^5=x^2+x $
        \item $ x^6=x^3+x^2 $
        \item $ x^7=x^3+x+1 $
        \item $ x^8=x^2+1 $
        \item $ x^9=x^3+x $
        \item $ x^{10}=x^2+x+1 $
        \item $ x^{11}=x^3+x^2+x $
        \item $ x^{12}=x^3+x^2+x+1 $
        \item $ x^{13}=x^3+x^2+1 $
        \item $ x^{14}=x^3+1 $
        \item $ x^15\equiv 1\mod x^4+x+1 $
    \end{itemize}
    Since $ \ord(x)\neq 1,3,5 $ $ \ord(x)\mid 15 $, so we have $ \ord(x)=15 $.
\end{exbox}

\begin{thmbox}
    \begin{theorem}
    \subsection{Lemma}
    Let $ \alpha\in GF(q)^* $, $ \ord(\alpha)=t $ and $ s\in\mathbb{Z} $.
    \[ \alpha^s=1\iff t\mid s \]
\end{theorem} \end{thmbox}

\begin{proof}
    Let $ s\in\mathbb{Z} $. Long division gives $ s=\ell t+r $ where
    $ 0\leqslant r\leqslant t-1 $. Then
    \[ \alpha^s=\alpha^{\ell t+r}=(\alpha^t)^\ell \alpha^r=\alpha^r \]
    So,
    \begin{align*}
        \alpha^s=1&\iff a^r=1\\
        &\iff r=0 \qquad\text{since } 0\leqslant 1\leqslant t-1\\
        &\iff t\mid s
    \end{align*}
\end{proof}

\begin{thmbox}
\begin{theorem}
    \subsection{Corollary}
    If $ \alpha\in GF(q)^* $, then $ \ord(\alpha)\mid (q-1) $.
\end{theorem} \end{thmbox}

\begin{proof}
    We know $ \alpha^{q-1} $, so $ \ord(\alpha)\mid (q-1) $ by
    the previous Lemma.
\end{proof}

\begin{defbox}
    \begin{definition}
    An element $ \alpha\in GF(q) $ is a \textbf{generator} of
    $ GF(q)^* $ if $ \ord(\alpha)=q-1 $.
\end{definition} \end{defbox}

\begin{thmbox}
    \begin{theorem}
    \subsection{Lemma}
    If $ \alpha $ is a generator of $ GF(q)^* $, then
    \[ \{\alpha^0,\alpha^1,\cdots,\alpha^{q-2}\}=GF(q)^* \]
\end{theorem} \end{thmbox}