\makeheading{2020-03-04}
\textbf{Recall}: Let $ C $ be an $ (n,k) $-cyclic code over $ GF(q) $ with generator
polynomial $ g(x) $. One generator matrix for $ C $ is:
\[
    \begin{bmatrix}
        g(x)   \\
        x g(x) \\
        \vdots \\
        x^{k-1}g(x)
    \end{bmatrix}_{k\times n} \]
One PCM for $ C $ is:
\[ H=
    \begin{bmatrix}
        h^*(x)   \\
        x h^*(x) \\
        \vdots   \\
        x^{n-k-1}h^*(x)
    \end{bmatrix}_{(n-k)\times n} \]
Another generator matrix for $ C $ is:
\[ G=\spalignaugmat{R I_k}=
    \spalignaugmat{
        {\begin{matrix}
                    -r_0(x)     \\
                    -r_1(x)     \\
                    \vdots      \\
                    -r_{k-2}(x) \\
                    -r_{k-1}(x)
                \end{matrix}}
        I_k}
\]
where $ x^{n-k+i}=\ell_i(x)g(x)+r_i(x)\implies -r_i(x)+x^{n-k+i}=\ell_i(x)g(x) $ for each $ i\in[0,k-1] $.
Then, another PCM for $ C $ is: $ H=\left[ \; I_{n-k}\mid -R^\top \; \right]_{(n-k)\times n} $. So,
\[ H^\top=
    \spalignmat{I_{n-k}; -R}_{n\times (n-k)}
    =
    \begin{bmatrix}
        x^0\pmod{g(x)}       \\
        x \pmod{g(x)}        \\
        \vdots               \\
        x^{n-k-1}\pmod{g(x)} \\
        x^{n-k}\pmod{g(x)}   \\
        x^{n-k+1}\pmod{g(x)} \\
        x^{n-1}\pmod{g(x)}
    \end{bmatrix} \]
Hence, if $ \symbf{r}=(r_0,r_1,\ldots ,r_{n-1})\in V_{n-1}(F) $, then
\begin{align*}
    \symbf{s}
     & =H\symbf{r}^\top                                         \\
     & =(r_0 x^0\pmod{g(x)})+\cdots+(r_{n-1}x^{n-1}\pmod{g(x)}) \\
     & =(r_0x^0+r_1x+\cdots+r_{n-1}x^{n-1})\pmod{g(x)}          \\
     & =r(x)\pmod{g(x)}
\end{align*}

\begin{Theorem}{}{}
    Let $ C $ be a cyclic code with generator polynomial $ g(x) $, and $ \symbf{r}\in V_n(F) $.
    Then, the syndrome of $ \symbf{r} $ with respect to the previous PCM is:
    \[ s(x)=r(x)\pmod{g(x)} \]
\end{Theorem}

\begin{Example}{}{}
    $ g(x)=1+x+x^2+x^3+x^6 $ is the generator polynomial for a $ (15,9) $-binary cyclic code.
    Check $ g(x)\mid (x^{15}-1) $ over $ GF(2) $. Compute the syndrome of
    $ \symbf{r}=(1110\; 1110\; 1100\; 000) $.

    \textbf{Solution.} Long division of $ (x^9+x^8+x^6+x^5+x^4+x^2+x+1)/(x^6+x^3+x^2+x+1) $
    gives $ x^5+x^4+x+1 $ as the remainder. Thus,
    \[ s(x)=1+x+x^4+x^5\implies \symbf{s}=(110011) \]
\end{Example}

\begin{Remark}{}{}
    Given the syndrome $ \symbf{s} $ of $ \symbf{r} $, the syndromes of cyclic shifts of $ \symbf{r} $
    can be easily computed.
\end{Remark}

\begin{Theorem}{}{syndrome calculation}
    Let $ \symbf{r}\in V_n(F) $, and $ s(x) \equiv r(x)\pmod{g(x)} $ where
    $ s(x) = s_0+xs_1+\cdots+s_{n-k-1}x^{n-k-1} $.
    Then the syndrome of $ xr(x) $ is:
    \begin{enumerate}[label=(\roman*)]
        \item $ xs(x) $, if $ s_{n-k-1}=0 $.
        \item $ xs(x)-s_{n-k-1}g(x) $, if $ s_{n-k-1}\neq 0 $.
    \end{enumerate}
\end{Theorem}

\begin{Proof}{\Cref{thm:syndrome calculation}}{}
    We have
    \[ r(x)=\ell (x)g(x)+s(x) \]
    Multiply by $ x $,
    \[ xr(x)=x\ell(x)g(x)+xs(x) \]
    \begin{itemize}
        \item \textbf{Case 1}: If $ s_{n-k-1}=0 $, then $ \deg(s)\leqslant n-k-2 $,
              so $ \deg(xs(x))\leqslant n-k-1 $. So, $ xs(x) $ is \textbf{the} remainder upon dividing
              $ xr(x) $ by $ g(x) $. So, $ xs(x) $ is \textbf{the} syndrome of $ r(x) $.
        \item \textbf{Case 2}: If $ s_{n-k-1}\neq 0 $, then $ \deg(s)=n-k-1 $. Then
              \[ xr(x)=x\ell(x)g(x)+xs(x)+s_{n-k-1}g(x)-s_{n-k-1}g(x) \]
              \[ \implies xr(x)=(x\ell(x)+s_{n-k-1})g(x)+(xs(x)-s_{n-k-1}g(x)) \]
              Now,
              \[ xs(x)-s_{n-k-1}=(s_0+\cdots+s_{n-k-1}x^{n-k})-(\cdots+s_{n-k-1}x^{n-k})=xr(x) \]
              So, $ xs(x)-s_{n-k-1}g(x) $ is \textbf{the} syndrome of $ xr(x) $.
    \end{itemize}
\end{Proof}
