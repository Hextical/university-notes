\makeheading{2020-02-07}
\section{Decoding Linear Codes}
Let $ C $ be an $ (n,k) $-code over $ F=GF(q) $ with PCM $ H $.

\begin{defbox}
    \begin{definition}
        We write $ \bm{x}\equiv \bm{y} \mod C $, where $ \bm{x},\bm{y}\in V_n(F) $
        if $ \bm{x}-\bm{y}\in C $.
    \end{definition}
\end{defbox}

\textbf{Notes}:
\begin{enumerate}[label=(\arabic*)]
    \item $ \equiv\mod C $ is an equivalence relation.
    \item The set of equivalence classes partitions $ V_n(F) $.
    \item The equivalence classes containing $ \bm{x}\in V_n(F) $
          is called a \textbf{coset} of $ V_n(F) $. This class is:
          \begin{align*}
              \{\bm{y}\in V_n(F):\bm{y}\equiv \bm{x}\mod C\}
               & =\{\bm{x}+\bm{c}:\bm{c}\in C\} \\
               & =C+\bm{x}
          \end{align*}
          We call $ C+\bm{x} $ the coset of $ C $ represented by $ \bm{x} $.
\end{enumerate}

\begin{exbox}
    \begin{example}[Cosets]
        Consider a $ (5,2) $-binary code with generator matrix
        \[ G=\begin{bmatrix}
                1 & 0 & 1 & 1 & 1 \\
                0 & 1 & 1 & 1 & 0
            \end{bmatrix}_{2\times 5} \]
        with $ d(C)=3 $. Find all cosets of $ C $.

        \textbf{Solution.} The cosets of $ C $ are:
        \begin{enumerate}[label=(\arabic*)]
            \item $ C+00000=\{00000,10111,01110,11001\}=\{\bm{0},R_1,R_2, R_1+R_2\}
                      = C+10111 = C+01110 = C + 11001 $
            \item $ C+10000=\{10000,00111,11110,01001\} $
            \item $ C+01000=\{01000,11111,00000,10001\} $
            \item $ C+00100=\{00100,10011,01010,11101\} $
            \item $ C+00010=\{00010,10101,01100,11011\} $
            \item $ C+00001=\{00001,10110,01111,11000\} $
            \item $ C+00011=\{00011,10100,01101,11010\} $
            \item $ C+11100=\{11100,01011,10010,00101\} $
        \end{enumerate}
        In total, there are $ 8 $ cosets.
    \end{example}
\end{exbox}

\textbf{Notes}:
\begin{enumerate}[label=(\arabic*)]
    \item $ C+\bm{0}=C $
    \item If $ \bm{y}\in C+\bm{x} $, then $ C+\bm{y}=C+\bm{x} $ by definition
          of equivalence.
    \item The number of cosets is $ \sfrac{q^n}{q^k}=q^{n-k} $.
\end{enumerate}

\textbf{Recall}: If $ \bm{x}\in V_n(F) $, then it's syndrome is
\[ \bm{s}=H\bm{r}^\top\in V_{n-k}(F) \]

\begin{thmbox}
    \begin{theorem}
        Let $ \bm{x},\bm{y}\in V_n(F) $. Then $ \bm{x}\equiv \bm{y}\mod C $
        if and only if $ H\bm{x}^\top=H\bm{y}^\top $.
    \end{theorem}
\end{thmbox}

\begin{proof}
    \begin{align*}
        \bm{x}\equiv \bm{y}\mod C
         & \iff \bm{x}-\bm{y}\in C           \\
         & \iff H(\bm{x}-\bm{y})^\top=\bm{0} \\
         & \iff H\bm{x}^\top=\bm{y}^\top
    \end{align*}
\end{proof}
So, cosets are characterized by their syndromes.

\textbf{Decoding}
\begin{itemize}
    \item $ \bm{c}\in C $ is sent.
    \item $ \bm{r}\in V_n(F) $ is received.
    \item $ \bm{e}=\bm{r}-\bm{c}\in V_n(F) $
    \item $ H\bm{r}^\top=H\bm{e}^\top $.
\end{itemize}
So, $ \bm{r} $ and $ \bm{e} $ belong to the same coset of $ C $.

\textbf{CMLD}

Given $ \bm{r} $, find a vector $ \bm{e} $ of smallest weight in $ C+\bm{r} $
or equivalently, find a vector $ \bm{e} $ of smallest weight with the same
syndrome as $ \bm{r} $. Then, decode $ \bm{r} $ to $ \bm{c}=\bm{r}-\bm{e} $.

\textbf{IMLD}

Find the unique vector $ \bm{e} $ of smallest weight having the same syndrome
as $ \bm{r} $. If no such $ \bm{e} $ exists, then reject $ \bm{r} $.
Otherwise, decode $ \bm{r} $ to $ \bm{c}=\bm{r}-\bm{e} $.

\section{Syndrome Decoding Algorithm}
Given a PCM $ H $ for an $ (n,k) $-code $ C $ over $ F=GF(q) $.

\begin{defbox}
    \begin{definition}
        A vector of smallest weight is a coset of $ C $ is distinguished and called
        a \textbf{coset leader} (of that coset).
    \end{definition}
\end{defbox}

\begin{algbox}
    \begin{algorithm}[H]
        \DontPrintSemicolon{}
        \caption{Syndrome Decoding Algorithm}
        \SetKwInOut{Input}{Input}
        \SetKwInOut{Output}{Output}

        \Input{Table of cosets, parity-check matrix $ H $, and received vector $ \bm{r} $}
        \Output{Decoded vector}
        $ \bm{s}\gets H\bm{r}^\top $\;
        Look up the coset leader corresponding to $ \bm{s} $, say $ \bm{\ell} $.\;
        \Return{$ \bm{r}-\bm{\ell} $}
    \end{algorithm}
\end{algbox}

\begin{exbox}
    \begin{example}[Syndrome Decoding]
        \[ G=\begin{bmatrix}
                1 & 0 & 1 & 1 & 1 \\
                0 & 1 & 1 & 1 & 0
            \end{bmatrix}_{2\times 5} \]
        \[ H=\begin{bmatrix}
                1 & 1 & 1 & 0 & 0 \\
                1 & 1 & 0 & 1 & 0 \\
                1 & 0 & 0 & 0 & 1
            \end{bmatrix}_{3\times 5} \]
        \underline{Table of Cosets}
        \begin{center}
            \begin{multicols}{2}
                $ C+00000=\{00000,10111,01110,11001\} $\\
                $ C+10000=\{10000,00111,11110,01001\} $\\
                $ C+01000=\{01000,11111,00000,10001\} $\\
                $ C+00100=\{00100,10011,01010,11101\} $\\
                $ C+00010=\{00010,10101,01100,11011\} $\\
                $ C+00001=\{00001,10110,01111,11000\} $\\
                $ C+00011=\{00011,10100,01101,11010\} $\\
                $ C+11100=\{11100,01011,10010,00101\} $
            \end{multicols}
        \end{center}
        There are $ q^{n-k}=2^{5-2}=2^3=8 $ cosets in total.

        Coset Leaders $ \rightarrow $ Syndromes:
        \begin{itemize}
            \item $ 00000 \rightarrow 000 $
            \item $ 10000 \rightarrow 111 $
            \item $ 01000 \rightarrow 110 $
            \item $ 00100 \rightarrow 100 $
            \item $ 00010 \rightarrow 010 $
            \item $ 00001 \rightarrow 001 $
            \item $ 00011 \rightarrow 011 $
            \item $ 10010 \rightarrow 101 $
        \end{itemize}
        Suppose $ \bm{r}=10111 $ is received. Decode $ \bm{r} $.

        \textbf{Solution.}

        Compute $ \bm{s}=H\bm{r}^\top=(000)^\top $.

        The closest leader corresponding to $ \bm{s}=(000)^\top $ is $ \bm{\ell}=00000 $.

        Thus, we get the decoded vector $ \bm{r}-\bm{\ell}=10111 $.
    \end{example}
\end{exbox}
