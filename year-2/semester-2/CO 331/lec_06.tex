\makeheading{ 2020-01-17 }

\begin{defbox}
    \begin{definition}
        We say two fields $F$ and $S$ are \textbf{isomorphic} if they have the same
        binary operations and if there exists a bijection between them.
    \end{definition}
\end{defbox}

\begin{defbox}
    \begin{definition}
        Let $ F $ be a field. A subset $ S\subseteq F $ is called a \textbf{subfield}
        of $ F $ if $ S $ is a field itself with respect to the same operations
        of $ F $.
    \end{definition}
\end{defbox}

\begin{exbox}
    \begin{example}[Subfield]
        Let $ F $ be a finite field where $ \ch(F)=p $. Consider
        $ E=\{0,1,1+1,\ldots,\underbrace{1+\cdots+1}_{p-1}\}\subseteq F $.
        We see that $ E $ is a field with the same field operations as $ F $.
        Also, $ E $ has order $ p $. If we label the elements of $ E $
        in a natural way such that $ \underbrace{1+\cdots+1}_{p-1}
            \longleftrightarrow p-1 $,
        then
        \[ E=\{0,1,1+1,\ldots,\underbrace{1+\cdots+1}_{p-1}\}
            = \mathbb{Z}_p= \{0,1,2,\ldots ,p-1\}\subseteq F \]
        So $ E $ is isomorphic to $ \mathbb{Z}_p $.
    \end{example}
\end{exbox}

\begin{thmbox}
    \begin{theorem}
        If $ F $ is a finite field of characteristic $ p $, then
        $ \mathbb{Z}_p $ is a subfield of $ F $.
    \end{theorem}
\end{thmbox}

\begin{proof}
    Exercise.
\end{proof}

\begin{defbox}
    \begin{definition}
        Let $ F $ be a finite field, and consider $ \mathbb{Z}_p\subseteq F $.
        \begin{itemize}
            \item Each $ v\in F $ is vector.
            \item Each $ c\in\mathbb{Z}_p $ is a scalar.
            \item Addition in $ F $ is defined by vector addition.
            \item Multiplication in $ F $ by elements in $ \mathbb{Z}_p $
                  is defined by scalar multiplication.
        \end{itemize}
    \end{definition}
\end{defbox}

\begin{thmbox}
    \begin{theorem}
        If $ F $ is a finite field of characteristic $ p $, then $ F $
        is a vector space over $ \mathbb{Z}_p $.
    \end{theorem}
\end{thmbox}

\begin{proof}
    Exercise.
\end{proof}

\begin{thmbox}
    \begin{theorem}
        If $ F $ is a finite field of characteristic $ p $, then
        \[ \ord(F)=p^k \]
        for some $ k\in\mathbb{Z}_{\geqslant 1} $.
    \end{theorem}
\end{thmbox}

\begin{proof}
    Let $ k $ be the dimension of the vector space $ F $ over $ \mathbb{Z}_p $.
    Let $ \{\alpha_1,\ldots ,\alpha_k\} $ be a basis for $ F $. Then, every element
    in $ F $ can be written as
    \[ c_1\alpha_1+\cdots+c_k\alpha_k \]
    where $ c_i\in\mathbb{Z}_p $. For each $ \alpha_i $, there are $ p $
    possible choices for $ c_i $, hence $ \ord(F)=p^k $.
\end{proof}

\begin{exbox}
    \begin{example}
    There is no field of order $ 6 $.
    \end{example}
\end{exbox}

\textbf{Question:} Is there a finite field of order $ 4,\,8,\,9 $?

\section{Irreducible Polynomials}

\begin{defbox}
    \begin{definition}
        Let $ F $ be a field. The \textbf{set of all polynomials in $ x $ over $ F $}
        (polynomials with coefficients from $ F $) is denoted $ F[x] $. Addition
        and multiplication are both done in the usual way, with coefficient arithmetic
        in $ F $.
    \end{definition}
\end{defbox}

\begin{exbox}
    \begin{example}
    In $ \mathbb{Z}_{11} $, $ (2+5x+6x^2)+(3+9x+5x^2)=5+3x $.
    \end{example}
\end{exbox}

\begin{thmbox}
    \begin{theorem}
        Let $ F $ be a field. $ F[x] $ is a commutative ring.
    \end{theorem}
\end{thmbox}

\begin{defbox}
    \begin{definition}
        Let $ F $ be a field and let $ f\in F[x] $ with $ \deg(f)\geqslant 1 $.
        If $ g,h\in F[x] $ with $ f\mid (g-h) $, then we write
        \[ g\equiv h \mod f \]
        or equivalently, we can write $ g-h=\ell f $ for some $ \ell\in F[x] $.
    \end{definition}
\end{defbox}

\begin{thmbox}
    \begin{theorem}
        Congruence is an equivalence relation.
    \end{theorem}
\end{thmbox}

\begin{defbox}
    \begin{definition}
        For a given $ f\in F[x] $, the \textbf{equivalence class containing $\bm{g\in F[x]}$}
        is
        \[ [g]=\left\{h\in F[x]: h\equiv g \mod f\right\} \]
    \end{definition}
\end{defbox}

\begin{defbox}
    \begin{definition}
        For $ g,h\in F[x] $, we define addition and multiplication as follows:
        \begin{itemize}
            \item Addition: $ [g]+[h]=[g+h] $
            \item Multiplication: $ [g][h]=[gh] $
        \end{itemize}
    \end{definition}
\end{defbox}

\begin{thmbox}
    \begin{theorem}
        \begin{enumerate}[1.]
            \item The set of all equivalence classes, denoted $ F[x]/(f) $
                  where $ f\in F[x] $ and $ \deg(f)\geqslant 1 $ is a
                  commutative ring.
            \item The polynomials in $ F[x] $ of degree less than degree of $ f $
                  are a system of distinct representatives of equivalence classes in
                  $ F[x]/(f)$.
        \end{enumerate}
    \end{theorem}
\end{thmbox}
Proof of 5:
\begin{proof}
    Let $ g\in F[x] $. By division algorithm for polynomials we can write
    $ g=\ell f+r $ where $ \deg(r)<\deg(f) $. So, $ g-r=\ell f $. Hence,
    $ g\equiv r\mod f $. Thus, $ [g]=[r] $ and we have $ \deg(r)<\deg(f) $.
    Also, if $ r_1,r_2\in F[x] $ with $ r_1\neq r_2 $, and
    $ \deg(r_1),\deg(r_2)<\deg(f) $, then
    \[ f\nmid (r_1-r_2)\iff r_1\not\equiv r_2\mod f \]
    Thus, $ [r_1]\neq [r_2] $.
\end{proof}
