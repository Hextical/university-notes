\chapter{Cyclic Codes}
\makeheading{2020-02-14 $ \heartsuit $}
\section{Introduction}

\begin{Definition}{Cyclic subspace}{cyclic_subspace}
    A subspace $ S $ of $ V_n(F) $ is a \textbf{cyclic subspace}
    if $ (a_0,a_1,\ldots ,a_{n-1})\in S\implies
        (a_{n-1},a_0,\ldots , a_{n-2})\in S $.
\end{Definition}

\begin{Definition}{Cyclic code}{cyclic_code}
    A \textbf{cyclic code} is a cyclic subspace of $ V_n(F) $.
\end{Definition}

\section{Rings and Ideals}

Let $ R=F[x]/(x^n-1) $. We write
\[ \underbrace{(a_0,a_1,\ldots ,a_{n-1})}_{\in V_n(F)}
    \longleftrightarrow \underbrace{{a_0+a_1x+\cdots+a_{n-1}}x^{n-1}}_{\in R} \]
That is, there is an isomorphism between $ V_n(F) $ and $ R $.
\begin{itemize}
    \item Addition is preserved: $ \symbf{a}+\symbf{b}\longleftrightarrow a(x)+b(x) $
    \item Scalar multiplication is preserved: $ \lambda\symbf{a} \longleftrightarrow \lambda a(x) $
\end{itemize}
\subsection*{Why choose $ x^n -1$?}
Let $ \symbf{a}=(a_0,\ldots ,a_{n-1})\in V_n(F) $. Let $ a(x) $ be the associated
polynomial in $ R $. Then,
\begin{align*}
    x\cdot a(x)
     & =a_0x+a_1x^2+\cdots+a_{n-2}x^{n-1}+a_{n-1}x^n           \\
     & \equiv a_{n-1}+a_0x+\cdots+a_{n-2}x^{n-1} \pmod{x^n -1} \\
     & \longleftrightarrow (a_{n-1},a_0,\ldots ,a_{n-2})
\end{align*}
So, multiplying a polynomial in $ R $ by $ x $ corresponds
to a right cyclic shift of the associated vector.

We'll define $ \cdot : V_n(F)\times V_n(F)\to V_n(F) $
by
\[ a\cdot b \longleftrightarrow a(x)b(x)\pmod{x^n-1} \]

\begin{Definition}{Ideal}{ideal}
    Let $ R $ be a commutative finite ring. Then, the non-empty
    subset $ I $ of $ R $ is an \textbf{ideal} of $ R $ if
    \begin{enumerate}[label=(\arabic*)]
        \item For all $ a,b\in I $, $ a+b\in I $
        \item For all $ a\in I $, $ b\in R $, $ ab\in I $
    \end{enumerate}
    $ \set{0} $ and $ R $ are defined to be \textbf{trivial} ideals of $ R $.
\end{Definition}

\begin{Theorem}{}{cyclic_iff_ideal}
    Let $ S\subseteq V_n(F) $ be non-empty. Let $ I $ be the associated polynomials.
    Then $ S $ is a cyclic subspace of $ V_n(F) $ if and only if $ I $
    is an ideal of $ R=F[x]/(x^n-1) $.
\end{Theorem}

\begin{Proof}{\Cref{thm:cyclic_iff_ideal}}{}
    $ (\implies) $ Suppose $ S $ is a cyclic subspace of $ V_n(F) $.
    Since $ S $ is closed under addition, so is $ I $.
    Let $ a(x)\in I $, $ b(x)=b_0+\cdots+b_{n-1}x^{n-1}\in R $.
    Then $ xa(x)\in I $ since $ S $ is a cyclic subspace. So,
    $ x^i a(x)\in I $ for each $ i\in [0,n-1] $. Also,
    $ b_i x^i a(x)\in I $ since $ S $ is closed under scalar multiplication.
    Finally, $ a(x)b(x)=a(x)(b_0+\cdots+b_{n-1}x^{n-1}) $ which is in $ I $
    since $ I $ is closed under addition. Thus, $ I $ is an ideal.

    $ (\impliedby) $ Suppose $ I $ is an ideal of $ R $. Since $ I $
    is closed under addition, so is $ S $. Since $ I $ is closed
    under multiplication by constant polynomials, $ S $ is closed
    under scalar multiplication. Since $ I $ is closed under
    multiplication by $ x $, $ S $ is closed under (right) cyclic shifts.
    Thus, $ S $ is a cyclic subspace.
\end{Proof}

\begin{Definition}{Ideal generated by $\symbf{g(x)}$}{ideal_generated_by_g(x)}
    Let $ g(x)\in R $. Then $ \langle g(x) \rangle = \set{g(x)a(x):a(x)\in R} $
    is an ideal of $ R $ called the \textbf{ideal generated by $\symbf{g(x)}$}.
    If $ I $ is an ideal of $ R $, then $ I $ is a \textbf{principal}
    ideal if there exists a $ g(x)\in I $ such that $ I= \langle g(x) \rangle $.
    $ R $ is called the \textbf{principal ideal ring} if every ideal
    ring of $ R $ is principal.
\end{Definition}

\begin{Theorem}{}{fxn1_principal}
    $ R=F[x]/(x^n-1) $ is a principal ideal ring.
\end{Theorem}

\begin{Proof}{\Cref{thm:fxn1_principal}}{}
    Let $ I $ be an ideal of $ R $.

    Suppose $ I=\set{0} $, then $ I=\langle 0\rangle $ is principal.

    Suppose $ I\neq 0 $. Let $ g(x) $ be a polynomial of smallest
    degree in $ I $. Let $ a(x)\in I $. Long division
    gives
    \[ a(x)=\ell(x)g(x)+r(x) \]
    where $ \ell,r\in F[x] $ and $ \deg(r)<\deg(g) $, but $ \ell(x)g(x)\in I $
    since $ I $ is closed under multiplication by $ R $ and
    $ a(x)=\ell(x)g(x)\in I $. Therefore, $ r(x)\in I $.
    Since $ \deg(r) <\deg(g) $, we must have $ r(x)=0 $ (since we define
    $ \deg(0)=-\infty $). Hence, $ a(x)=\ell(x)g(x) $. Therefore,
    $ I=\langle g(x) \rangle $. Thus, $ R $ is a principal ideal ring.
\end{Proof}
