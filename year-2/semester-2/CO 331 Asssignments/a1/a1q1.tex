\usepackage{lmodern}
\usepackage[margin=1in]{geometry}
\usepackage[dvipsnames]{xcolor}
\usepackage[a-3u,pdf17]{pdfx}
\usepackage{hyperref}
\hypersetup{colorlinks=true,linkcolor=Blue}
\usepackage[theorems,breakable]{tcolorbox}
\usepackage[shortlabels]{enumitem}
\usepackage{xfrac}
\usepackage{mathtools}
\usepackage{amssymb}
\usepackage{cleveref}
\usepackage{booktabs}
\usepackage{derivative}
\usepackage{interval}
\intervalconfig{soft open fences,separator symbol={,}}
\usepackage{graphicx}
\graphicspath{{./figures/}}
\usepackage{tikz}
\usetikzlibrary{patterns,positioning,calc}
\usepackage{pgfplots}
\pgfplotsset{samples=100} % possibly uncomment this
\pgfplotsset{compat=1.18}
\usepgfplotslibrary{fillbetween}

\definecolor{myyellow}{RGB}{255,255,168}
\definecolor{mypurple}{RGB}{216,216,255}
\definecolor{mygreen}{RGB}{216,255,216}
\definecolor{myred}{RGB}{255,216,216}
\definecolor{mycyan}{RGB}{204,229,229}

\tcbset{
    common/.style={
            fonttitle=\bfseries,
            coltitle=black,
            boxrule=0pt,
            breakable
        },
    theorem/.style={
            common,
            colback=mypurple,
            colframe=mypurple!95!black,
            fontupper=\itshape{}
        },
}

\newtcbtheorem[number within=section, crefname={definition}{definitions}]
{Definition}{DEFINITION}{
    common,
    colback=myyellow,
    colframe=myyellow!95!black
}{def}

\newtcbtheorem[use counter from=Definition, crefname={remark}{remarks}]
{Remark}{REMARK}{
    common,
    colback=mycyan,
    colframe=mycyan!95!black,
}{remark}

\newtcbtheorem[use counter from=Definition, crefname={theorem}{theorems}]
{Theorem}{THEOREM}{
    theorem
}{thm}

\newtcbtheorem[no counter]
{Proof}{Proof of}{
    common,
    colframe=black!10,
    separator sign={\!\!}
}{pf}

\newtcbtheorem[use counter from=Definition, crefname={example}{examples}]
{Example}{EXAMPLE}{
    common,
    colback=mygreen,
    colframe=mygreen!95!black,
}{ex}

\newtcbtheorem[use counter from=Definition, crefname={corollary}{corollaries}]
{Corollary}{COROLLARY}{
    theorem
}{cor}

\newtcbtheorem[use counter from=Definition, crefname={exercise}{exercises}]
{Exercise}{EXERCISE}{
    common,
    colback=myred,
    colframe=myred!95!black,
}{exercise}

\newtcbtheorem[use counter from=Definition, crefname={proposition}{propositions}]
{Proposition}{PROPOSITION}{
    theorem
}{prop}

\DeclarePairedDelimiterX\Set[1]\{\}{#1}
\DeclarePairedDelimiterX\norm[1]\lVert\rVert{#1}
\DeclarePairedDelimiterX\abs[1]\lvert\rvert{#1}
\DeclarePairedDelimiterXPP{\LN}[1]{\operatorname{\mathrm{ln}}}(){}{#1}
\DeclarePairedDelimiterXPP{\EXP}[1]{\operatorname{\mathrm{exp}}}\{\}{}{#1}

\usepackage{nicematrix}
\newcommand{\R}{\mathbb{R}}
\newcommand{\ER}{\overline{\mathbb{R}}}
\newcommand{\N}{\mathbb{N}}
\newcommand{\Z}{\mathbb{Z}}
\newcommand{\glb}{\mathrm{glb}}
\newcommand{\lub}{\mathrm{lub}}
\newcommand{\LHR}{\stackrel{\text{\tiny L'R}}{=}}
\DeclarePairedDelimiter\sequence{\langle}{\rangle}
\DeclarePairedDelimiterXPP{\bigo}[1]{\mathcal{O}}(){}{#1}

\newcommand{\Dom}{\operatorname{Dom}}
\newcommand{\Cdm}{\operatorname{Cdm}}

\begin{document}
1. (10 marks) \textbf{Distance}\\
If $x_1$ and $x_2$ are binary $n$-tuples, then $x_1+x_2$ denotes the
bitwise modulo 2 sum of $x_1$ and $x_2$.
For example, $000111 + 011011 = 011100$.

\textbf{(a) Let $C$ be a binary $[n,M]$-code with distance $d$. Let $x \in \{0,1\}^n$,
and let $C+x = \{c+x\; : \; c \in C\}$. Prove that the distance of $C+x$ is
also $d$.}

\begin{proof}
    Suppose $ C $ is a binary $ [n,M] $-code with distance $ d $. That is,
    $ C=\{c_1c_2\cdots c_n:c_i\in \{0,1\},\, 1\le i \le n\} $
    where $ |C|=M $. Observe that when adding $ x $ to each codeword,
    where $ x\in \{0,1\}^n $, each codeword will differ in the exact same
    index $ i,\,\forall i\in \{1,\ldots ,n\} $ (that is, the indices 
    where any two codewords were different before adding $ x $,
    will still be different after adding $ x $).

    Suppose $ d(C_1=c_1c_2\cdots c_n,\,C_1^\prime=c_1^\prime c_2^\prime\cdots c_n^\prime)=d $,
    and suppose that $ C_1 $ and $ C_1^\prime $ differ in index $ i=1 $, (i.e.
    $ c_1\neq c_1^\prime $).

    \myuline{Case 1}

    $ c_1=1,\,c_1^\prime=0 $, and $ x_1=1 $. We get that $ c_1+x_1=0 $, and
    $ c_1^\prime+x_1=1 $. Thus, they still differ in the same index before
    and after being added by $ x $ (or $ x_1 $).

    \myuline{Case 2}

    $ c_1=1,\,c_1^\prime=0 $, and $ x_1=0 $. We get that $ c_1+x_1=1 $, and
    $ c_1^\prime+x_1=0 $. Thus, they still differ in the same index before
    and after being added by $ x $ (or $ x_1 $).

    Since $ d(c_1,c_1^\prime)=d(c_1^\prime,c_1) $, we can easily swap
    $ c_1 $ with $ c_1^\prime $ in each case above and the result will be the same.
    Also, we can do this for all $ M $ codewords to each index 
    (i.e. $ \binom{M}{2} $ comparisons for each pair of codewords,
    and $ \binom{n}{2} $ extra comparisons with Case 1 and Case 2).
    Thus, $ d(C)=d(C+x)=d $ as the difference between any two codewords
    will be the exact same after adding $ x $ as the indices where
    the difference is obtained will be the same.
\end{proof}


\textbf{(b) Construct a binary $[8,4]$-code with distance 5, or prove that no such
code exists.}

$ C=\{
    c_1=00000000,\,
    c_2=11111000,\,
    c_3=10101111,\,
    c_4=01010111
    \} $.

We have $ \binom{M}{2}=\binom{4}{2}=6 $ comparisons.
\begin{enumerate}
    \item $ d(c_1,c_2)=5 $
    \item $ d(c_1,c_3)=6 $
    \item $ d(c_1,c_4)=5 $
    \item $ d(c_2,c_3)=5 $
    \item $ d(c_2,c_4)=6 $
    \item $ d(c_3,c_4)=5 $
\end{enumerate}

Thus, we have constructed a binary $ [8,4] $-code with $ d(C)=5 $.

\textbf{(c) Construct a binary $[7,3]$-code with distance 5, or prove that no such
code exists.}

We will prove that no such code exists. Suppose $ C=\{c_1,\,c_2,\,c_3\} $
where $ c_1,\,c_2,\,c_3 $ are all codes of length $ 7 $. Suppose
that $ d(C)\ge 5 $ is attained by at least one pair of codewords,
$ c_1 $ and $ c_3 $. We try all possible cases.

\myuline{Case 1} $ d(c_1,c_3)=5 $

We know that $ c_1 $ and $ c_3 $ differ in five indices, and are the same
in two indices. We immediately see that we can pick $ c_2 $ to be different
in two indices (the same indices where $ c_1 $ and $ c_3 $ were the same).
Now $ c_2 $ has five indices left to pick.
If we were to pick $ c_2 $ to be different in three more indices when
compared to $ c_1 $ (to get $ d(c_1,c_2)=5\ge 5 $), then we only have two more
indices to select to be different when compared to $ c_3 $ (meaning we would get
$ d(c_2,c_3)=4<5\implies d(C)=4\neq 5 $). Thus, no such $ [7,3] $-code where
$ d(C)=5 $ can exist when $ d(c_1,c_3)=5 $.

\myuline{Case 2} $ d(c_1,c_3)=6 $

We know that $ c_1 $ and $ c_3 $ differ in six indices, and are the same
in one index. We immediately see that we can pick $ c_2 $ to be different
in one index (the same index where $ c_1 $ and $ c_3 $ were the same).
Now $ c_2 $ has six indices left to pick.
If we were to pick $ c_2 $ to be different in four more indices when
compared to $ c_1 $ (to get $ d(c_1,c_2)=5\ge 5 $), then we only have two more
indices to select to be different when compared to $ c_3 $ (meaning we would get
$ d(c_2,c_3)=3<5 \implies d(C)=3\neq 5$). Thus, no such $ [7,3] $-code where
$ d(C)=5 $ can exist when $ d(c_1,c_3)=6 $.

\myuline{Case 3} $ d(c_1,c_3)=7 $

We know that $ c_1 $ and $ c_3 $ differ in seven indices.
If we were to pick $ c_2 $ to be different in five indices when
compared to $ c_1 $ (to get $ d(c_1,c_2)=5\ge 5 $), then we only have two more
indices to select to be different when compared to $ c_3 $ (meaning we would get
$ d(c_2,c_3)=2<5 \implies d(C)=2\neq 5$). Thus, no such $ [7,3] $-code where
$ d(C)=5 $ can exist when $ d(c_1,c_3)=7 $.

Thus, no such $ [7,3] $-code exists with $ d(C)=5 $.

\end{document}