\section{2020-02-27}
\subsection{Invariants of Encapsulation}
Visibility: public/private

Node invariant: next is either \code{nullptr} or points to the heap

Strategy: create a wrapper list class which maintains full control over \code{Node}s.

\code{list.h}: \code{1201/lectures/se/iterator}
\begin{lstlisting}
    class List {
        struct Node; // private nested Node class
        Node *theList = nullptr;
        public:
            void addFront(int);
            int ith(int);
            ~List();
    };
\end{lstlisting}
\code{list.cc}
\begin{lstlisting}
    struct List::Node {
        int data;
        Node *next;
        Node(int data, Node *next) : data{data}, next{next} {};
        ~Node() {
            delete next;
        }
    };
    void List::addFront(int i) {
        theList = new Node{i, theList};
    }
    int List::ith(int index) { // precondition index is valid
        Node *curr = thisList;
        for (int j=0; j < index; ++j, curr = curr->next);
        return curr->data;
    }
    List::~List() {
        delete theList;
    }
    
    int main() {
        List l;
        l.addFront(1);
        l.addFront(2);
        l.addFront(3);
        for (int i=0; i < 3; ++i) {
            cout << list.ith(i); // 321
        }
    }
\end{lstlisting}
List traversal is $ O(n^2) $.
Previously,
\begin{lstlisting}
    Node *curr = theList;
    while(curr) {
        ...
        curr = curr->next;
    }
\end{lstlisting}
List traversal is $ O(n) $.

Idea: need to keep track of how much of the \code{List} has been traversed/iterated.
Create another class that wraps a \code{Node*}.

Design patterns: known good strategies to solve common design problems.

\subsection{Iterator Design Pattern}
Abstraction of a pointer to the container without exposing the pointer to the client.
\begin{lstlisting}
    // arr is an array
    for (int *p = arr; p!= arr + arraySize; ++p) {
        ...
    }
\end{lstlisting}
Need a begin and end to our iteration.
\begin{itemize}
    \item class should support
          \begin{itemize}
              \item \code{!=}
              \item \code{++} \textrightarrow{} prefix plus
              \item \code{*} \textrightarrow{} unary dereference
          \end{itemize}
\end{itemize}

\begin{lstlisting}
    class List {
        struct Node {...};
        Node *theList = nullptr;
        public:
            ... // methods from before
            class Iterator { // public nested class
                Node *curr;
                public:
                    explicit Iterator(Node *curr) : curr{curr} {}
                    bool operator!=(const Iterator &rhs) {
                        return curr != rhs.curr;
                    }
                    Iterator &operator++() {
                        curr = curr->next;
                        return *this;
                    }
                    // by reference to be able to write to the data field.
                    int &operator*() {
                        return curr->data;
                    }
            }; // end Iterator
            Iterator begin() {
                return Iterator{theList};
            }
            Iterator end() {
                return Iterator{nullptr};
            }
    };

    int main() {
        List ;
        l.addFront(1);
        l.addFront(2);
        l.addFront(3);
        // List::Iterator -> auto
        for (List::Iterator it = l.begin(); it != l.end(); ++it) {
            cout << *it; // 321
            *it = 5; // example of a write
        }
    }

\end{lstlisting}
\begin{itemize}
    \item \code{auto x = y;} \textrightarrow{} automatic type inference; define \code{x}
          to be the same type as \code{y}.
    \item C++ has built-in support for the Iterator design pattern
\end{itemize}
\subsection{C++ Range-based for loops}
We can use range-based for loops for a class \code{MyClass} if:
\begin{enumerate}[label=(\arabic*)]
    \item \code{MyClass} supports methods named \code{begin} and \code{end}
          that return objects of some type, \code{T}
    \item \code{class T} must support \code{!=}, \code{++}, \code{*}
\end{enumerate}
Example of a range-based for loop:
\begin{lstlisting}
    // First type, by value declaration of variable n, of type int
    for (auto n : l) {
        // implicit of n = *it;
        cout << n;
    }
    // Second type, take by reference to be able to mutate
    for (auto &n : l) {
        n = ...;
    }
\end{lstlisting}

To force clients to use \code{begin} and \code{end}, we should make the
\code{Iterator} private, but then \code{List::begin} and \code{List::end}
would lose access to this constructor. The \code{Iterator} class can declare
\code{List} to be a friend.
\begin{lstlisting}
    class List {
        ...
        public:
            class Iterator {
                ...
                friend class List;
            };
    };
\end{lstlisting}
Friendship weakens encapsulation. Advice: have as few friends as possible.
