\chapter{Linux}
\section{2020-01-07}

\subsection{Linux Shell}

Shell: interface to an OS

Graphical shell: click/touch, intuitive

Command line shell: type commands, not intuitive, more powerful

Stephen Bourne (70s): original UNIX shell

History: C shell (\code{csh}) \textrightarrow{} Turbo C shell (\code{tcsh}) \textrightarrow{}
KornShell (\code{ksh}) \textrightarrow{} Bourne Again Shell (\code{bash})

Check what command line shell: \code{echo \$0}

Go into bash: \code{bash}

\subsection{Linux File System}
Directories: files that contain files (called folders in Windows), e.g.
\code{usr}, \code{share}, \code{dict} are all directories

Root (literally a backslash) \code{/}: top directory

Path: location of a file in the file system, e.g. \code{/usr/share/dict/words}

Absolute path: path that starts at the root directory

Relative path: path relative to a directory

The path \code{dict/words}
relative to \code{/user/share} is \code{/usr/share/dict/words}

\subsection{Commands}
\begin{itemize}
      \item NAME\@: \code{ls} \textrightarrow{} list directory contents.
            \begin{itemize}
                  \item SYNOPSIS\@: \code{ls [OPTION]\textellipsis{} [FILE]}
                  \item DESCRIPTION\@: List information about the non-hidden \code{FILE}s
                        (current directory by default). Hidden files start with a \code{.}
                        \code{ls -a} or \code{ls -all} do not ignore entries starting
                        with a \code{.}; the \code{-a} is an argument
            \end{itemize}
      \item NAME\@: \code{pwd} \textrightarrow{} print name of current/working directory.
            \begin{itemize}
                  \item SYNOPSIS\@: \code{pwd [OPTION]\textellipsis}
                  \item DESCRIPTION\@: Print the full filename of the current working directory.
            \end{itemize}
      \item NAME\@: \code{cd} \textrightarrow{} change the shell working directory.
            \begin{itemize}
                  \item \code{..} \textrightarrow{} parent directory
                  \item \code{.} \textrightarrow{} current directory
                  \item \code{-} \textrightarrow{} previous directory
                  \item \code{\textasciitilde} \textrightarrow{} home directory
                  \item \code{\textasciitilde{} userid} \textrightarrow{} \code{userid}'s directory
            \end{itemize}
\end{itemize}
