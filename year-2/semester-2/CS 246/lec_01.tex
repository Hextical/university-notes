\section{2020-01-07}

\subsection{Linux Shell}

Shell: interface to an OS

Graphical shell: click/touch, intuitive

Command line shell: type commands, not intuitive, more powerful

Stephen Bourne (70s): original UNIX shell
Bourne Again Shell: bash

This course uses bash, a command line shell.

\# is a comment within the shell.

\begin{lstlisting}
    # check what command line shell
    echo $0
    # go into bash
    bash
\end{lstlisting}

\subsection{Linux File System}
Directories: files that contain files (called folders in Windows), e.g.
usr, share, dict are all directories

Root (a backslash) /: top directory

Path: location of a file in a file system, e.g. /usr/share/dict/words

Absolute path: path that starts at the root directory

Relative path: path relative to a directory.
The path dict/words relative to /user/share is /usr/share/dict/words

\subsection{Examples of Commands}

\begin{lstlisting}
    # view non-hidden files in the current directory
    ls
    # list all files (including hidden ones), the -a is an argument
    ls -a
    # print working directory (prints absolute path of current directory)
    pwd
    # change directory
    cd -argument
    # possible arguments for cd : explanation
    # .. : parent directory (back one, up one directory)
    # . : current directory
    # - : previous directory
    # ~ : home directory
    # ~userid : userid`s directory
\end{lstlisting}