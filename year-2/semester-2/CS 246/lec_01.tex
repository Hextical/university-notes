\section{2020-01-07}

\subsection{Linux Shell}

Shell: interface to an OS

Graphical shell: click/touch, intuitive

Command line shell: type commands, not intuitive, more powerful

Stephen Bourne (70s): original UNIX shell

History: C shell (\code{csh}) $ \rightarrow $ Turbo C shell (\code{tcsh}) $ \rightarrow $ 
KornShell (\code{ksh}) $ \rightarrow $ Bourne Again Shell (\code{bash})

Check what command line shell: \code{echo \$0}

Go into bash: \code{bash}

\subsection{Linux File System}
Directories: files that contain files (called folders in Windows), e.g.
\code{usr}, \code{share}, \code{dict} are all directories

Root (literally a backslash) \code{/}: top directory

Path: location of a file in the file system, e.g. \code{/usr/share/dict/words}

Absolute path: path that starts at the root directory

Relative path: path relative to a directory

The path \code{dict/words}
relative to \code{/user/share} is \code{/usr/share/dict/words}

\subsection{Commands}
\begin{itemize}
    \item NAME: \code{ls} $ \rightarrow $ list directory contents.
    \subitem SYNOPSIS: \code{ls [OPTION]$\ldots$ [FILE]}
    \subitem DESCRIPTION:
    List information about the non-hidden \code{FILE}s (current directory by default).
    Hidden files start with a \code{.}
    \subitem \code{ls -a} or \code{ls -all} do not ignore entries starting
    with a \code{.}; the \code{-a} is an argument
    \item NAME: \code{pwd} $ \rightarrow $ print name of current/working directory.
    \subitem SYNOPSIS: \code{pwd [OPTION]$\ldots$}
    \subitem DESCRIPTION: Print the full filename of the current working directory.
    \item NAME: \code{cd} $ \rightarrow $ change the shell working directory.
    \subitem \code{..} $ \rightarrow $ parent directory
    \subitem \code{.} $ \rightarrow $ current directory
    \subitem \code{-} $ \rightarrow $ previous directory
    \subitem \code{\textasciitilde} $ \rightarrow $ home directory
    \subitem \code{\textasciitilde userid} $ \rightarrow $ \code{userid}'s directory
\end{itemize}