\chapter{C++}
\section{2020-01-21}
\subsection{Testing}
A pizza shop allows users to order pizza online and earn 10 points for each pizza
ordered.

Ordering: A user types \code{O} followed by a number N to order N pizzas.
e.g. \code{O} 2 orders 2 pizzas

The system allows ordering between 1 to 10 pizzas.
If \code{N} is outside this range, the system prints ``Illegal order.''

On a successful order, the system display ``2 pizzas ordered''
followed by the total number of points they have.

Redeeming: At any time, users can type \code{R} to try to redeem free pizza.
If the user has enough points(50), ``Free Pizza!'' is printed.
If the user does not have enough points, ``No pizza for you!''
is printed followed by the number of points the user has.

Points: At any time, users can type \code{P} to print their current points balance.

Write exhaustive tests for this system.

\textbf{Solution.}

\begin{itemize}
    \item \code{O 1} $ \rightarrow $ 1 pizza ordered
    \item \code{O 10} $ \rightarrow $ 10 pizzas ordered
    \item \code{O 0} $ \rightarrow $ Illegal order
    \item \code{O 11} $ \rightarrow $ Illegal order
    \item \code{1} $ \rightarrow $ X
    \item \code{O 1 1} $ \rightarrow $ X
    \item \code{O 5} $ \rightarrow $ 5 pizzas ordered
          \begin{itemize}
              \item \code{P} $ \rightarrow $ 50
              \item \code{R} $ \rightarrow $ Free pizza!
              \item \code{P} $ \rightarrow $ 0
          \end{itemize}
    \item \code{O 7} $ \rightarrow $ 7 pizzas ordered
          \begin{itemize}
              \item \code{P} $ \rightarrow $ 70
              \item \code{R} $ \rightarrow $ Free pizza!
              \item \code{R} $ \rightarrow $ No free pizza for you! 20
          \end{itemize}
\end{itemize}

\subsection{C++ Introduction}
\lstset{frame=tb,
    language=C++,
    aboveskip=3mm,
    belowskip=3mm,
    showstringspaces=false,
    columns=flexible,
    basicstyle={\ttfamily},
    numbers=none,
    numberstyle=\tiny\color{gray},
    keywordstyle=\color{blue},
    commentstyle=\color{gray},
    stringstyle=\color{red},
    breaklines=true,
    breakatwhitespace=true,
    tabsize=2
}
Simula 64 $ \rightarrow $ first OO language (has classes)

C with classes $ \rightarrow $ C++

History: C++99 $ \rightarrow $ C++03 $ \rightarrow $ C++11 $ \rightarrow $ C++14

In C,
\begin{lstlisting}[language = C]
    # include <stdio.h>
    int main(void) {
        printf("Hello world\n");
        return 0;
    }
\end{lstlisting}
File: \code{hello.cc}
\begin{lstlisting}[language = C++]
    # include <iostream>
    using namespace std;

    int main() {
        cout << "Hello world" << endl;
        return 0;
    }
\end{lstlisting}

\subsection{iostream header}

\code{stdio.h}, \code{printf}, \code{scanf}, \code{read}
$ \rightarrow $ not allowed in C++ (although valid)

Instead use, \code{std::cout <{}< data1 <{}< data2;}

By placing \code{using namespace std;}, we can say
\begin{itemize}
    \item \code{cout} instead of \code{std::cout}
    \item \code{endl} instead of \code{std::endl}
\end{itemize}

\subsection{Compile C++}
Since we created an alias for \code{g++} in assignment 0, we can instead
compile with simply \code{g++14 hello.cc}. To rename the output file
we can specify the \code{-o} parameter.
\begin{itemize}
    \item \code{g++ -std=c++14 hello.cc} $ \rightarrow $ creates \code{a.out}
    \item \code{g++14 hello.cc -o prog.out} $ \rightarrow $ creates \code{prog.out}
\end{itemize}

\subsection{Run C++}
\begin{itemize}
    \item \code{./a.out} $ \rightarrow $ runs \code{a.out}
\end{itemize}

\subsection{C++ I/O}
\code{cout <{}< ``Hello'' <{}< ``World'' <{}< endl;}

When we create \code{iostream}, we get access to three stream variables.
\begin{enumerate}
    \item \code{stdin}
          \begin{itemize}
              \item \code{std::in}
              \item type: \code{istream}
              \item e.g. \code{cin <{}< x;}
          \end{itemize}
    \item \code{stdout}
          \begin{itemize}
              \item \code{std::out}
              \item type: \code{ostream}
              \item e.g. \code{cout <{}< ``Hello'';}
          \end{itemize}
    \item \code{stderr}
          \begin{itemize}
              \item \code{std::err}
              \item e.g. \code{cerr <{}< ``Error'';}
          \end{itemize}
\end{enumerate}

File: \code{add.cc}
\begin{lstlisting}[language = C++]
    #include <iostream>
    using namespace std;
    
    int main() {
      int x, y;
      cin >> x >> y;
      cout << x + y << endl;
    }
\end{lstlisting}
\begin{itemize}
    \item If a read fails, the expression \code{cin.fail()} is true
    \item If a read fails due to EOF, then expressions \code{cin.fail()}
          and \code{cin.eof()} are both true
\end{itemize}

File: \code{readInts.cc}
\begin{lstlisting}[language = C++]
    #include <iostream>
    using namespace std;
    
    int main() {
      int i;
      while (true) {
        cin >> i;
        if (cin.fail()) break;
        cout << i << endl;
      }
    }
\end{lstlisting}
\begin{itemize}
    \item Read as many \code{int}s from \code{stdin} and output to \code{stdout}
    \item Stop if a read fails
\end{itemize}

\begin{itemize}
    \item C++: an automatic conversion from iostream variables to bool.
    \item \code{cin} is true if \code{cin.fail()} is false
    \item \code{cin} is false if \code{cin.fail()} is true
\end{itemize}

\subsection{Summary of Files}
Files covered in this lecture found in \code{1201/lectures/c++}:
\begin{itemize}
    \item \code{hello.cc}
    \item \code{add.cc}
    \item \code{readInts.cc}
\end{itemize}
