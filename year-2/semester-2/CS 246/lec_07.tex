\section{2020-01-28}
\subsection{Short C++ Topics}
File: \code{readInts5.cc}
\begin{itemize}
    \item Read as many inputs from stdin, int output, bad int ignore
    \item Terminate when done receiving input
\end{itemize}
File: \code{readIntsSS.cc}
\begin{lstlisting}
    #include <iostream>
    #include <sstream>
    using namespace std;

    int main () {
      string s;
      while (cin >> s) {
        istringstream ss{s};
        int n;
        if (ss >> n) cout << n << endl;
      }
    }
\end{lstlisting}
\begin{itemize}
    \item Difference: ignores entire string if read fails, e.g. \code{he1lo}
    would ignore \code{helo}, but \code{readIntsSS.cc} discards entire string
\end{itemize}
\subsection{Default Arguments}
\begin{lstlisting}
    void print (string name = "suite.txt" ) { // set default value
        string s;
        ifstream f{ name };
        while (file >> s) {
            cout << s << endl;
        }
        print("filename");
        print(); // default value
    }
\end{lstlisting}
\begin{itemize}
    \item Default arguments must come last, e.g. \code{void test(int x = 0, string str);}
    $ \rightarrow $  illegal because \code{string} has no default value but comes last
    \item \code{void test(int x = 0, string str = ``hello'');} $ \rightarrow $ legal,
    can be called in three different ways as denoted ``legal'' below.
    \item When calling the function, we can use default values for the last
    $ N $ parameters, e.g. $ \rightarrow $ 
    \subitem \code{test();} $ \rightarrow $ legal;
    \subitem \code{test(s);} $ \rightarrow $ legal;
    \subitem \code{test(s, ``bla'');} $ \rightarrow $ legal;
    \subitem \code{test( ,``bla'')} $ \rightarrow $ illegal;
    \subitem \code{test(``bla'');} $ \rightarrow $ illegal;
    \item We cannot implement \code{void test();}, or \code{void test(int);}, or
    \code{void test(int, string)}
    \item In C, function names must be unique. In C++, we can have functions with
    the same name, but they must differ in the number of types of parameters,
    called \textbf{function overloading}. 
    \item \textbf{Signature}: name of function, types and number of parameters.
    \item The return type of a function is \textbf{not} part of the signature.
    \item A new function: \code{void test(int, int);} $ \rightarrow $ legal,
    does not conflict with \code{test(int, string);}
    \item A new function: \code{void test(string);} $ \rightarrow $ legal
\end{itemize}

\begin{lstlisting}
    int a = 21; // 10101
    int b = 3;
    a = a << b; // 10101000
\end{lstlisting}
\begin{itemize}
    \item left shift operator (bit shift): multiply by $ 2^b $;
    \item right sift operator: divide by $ 2^b $
    \item \code{<{}<} is a overloaded operator. Overloading is a type of function
    Overloading.
    \item C++ allows us to define the meaning of operators for user-define types
    (all types not built-in); \code{istream} $ \rightarrow $ \code{iostream}
    header.
\end{itemize}
\begin{lstlisting}
    int x;
    cout << x;
    string s{"hello"};
    cout << s;
\end{lstlisting}
\begin{itemize}
    \item these operators are \textbf{overloaded}, because \code{<{}<} can work
    differently for \code{string} and \code{int}
    \item example of function overloading since \code{int x}
    and \code{string s} call different functions 
\end{itemize}
\begin{lstlisting}
    int a, b;
    a + b;
    string c, d;
    c + d;
\end{lstlisting}
\begin{itemize}
    \item operators are overloaded as \code{+} differs depending on the type
\end{itemize}

\subsection{Structs}
In C,
\begin{lstlisting}
    struct Node {
        int data;
        struct Node *next;
    };
    struct Node n = {3, NULL};
\end{lstlisting}
\begin{itemize}
    \item In C++, you don't have to write \code{struct} before \code{Node} after defining
    \code{Node} as a struct
    \item Can remove the \code{=} before \{\} for uniform initialization
    \item Discouraged to use \code{NULL} constant, use \code{nullptr} instead
\end{itemize}

\subsection{Constants}
\code{const int MAX = 10;}
\begin{itemize}
    \item Constants must always be initialized
    \item \code{const Node n\{3, nullptr\}} $ \rightarrow $ \code{3} data,
    \code{nullptr} next
    \item \code{n.data} or \code{n.next} $ \rightarrow $ illegal
\end{itemize}
\begin{lstlisting}
    int n = 5;
    const int *p = &n;
\end{lstlisting}
\begin{itemize}
    \item \code{p} is a pointer to an \code{int} which is \code{const}; pointer
    is not \code{constant}
    \subitem \code{p = \&m;} $ \rightarrow $ legal
    \subitem \code{*p = 10;} $ \rightarrow $ illegal, but can still do \code{n = 10;}
    \item \code{int *const q = \&n;} $ \rightarrow $ \code{q} is a \code{const}
    ptr to an \code{int}
    \subitem \code{q = \&m;} $ \rightarrow $ illegal
    \subitem \code{*q = 10;} $ \rightarrow $ legal
    \item \code{const int *const r = \& n;} $ \rightarrow $ r is a \code{const int}
    to a ptr that is \code{const}
    \item \code{const} applies to the ``thing'' on the left, unless there is 
    nothing to the left, in which case it applies to the right
\end{itemize}
\subsection{Parameter Passing}
\begin{lstlisting}
    void inc(int x) { // copied passed by value
        x = x + 1
    }
    int x = 5;
    inc(x); // does not actually modify x
    cout << x; // prints 5
\end{lstlisting}
Passing pointers:
\begin{lstlisting}
    void inc(int *p) {
        *p = *p + 1;
    }
    int x = 5;
    inc(&x);
    cout << x; // prints 6
\end{lstlisting}
\begin{tabular}{|c|c|}
    \hline
    \code{scanf("\%d, \&x");} & \code{cin >{}> x;}\\
    \hline
    & \code{operator >{}> (cin, x);}\\
    & pass by reference
\end{tabular}
\subsection{Lvalue References}
Informally, an \textbf{lvalue} is anything that can appear on the left hand side of
an assignment.
\begin{itemize}
    \item \code{x = 5;} $ \rightarrow $ \code{x} is an lvalue
    \item \code{5 = 7} $ \rightarrow $ not an lvalue
    \item \code{x + y = 5} $ \rightarrow $ not an lvalue
    \item \code{str[i] = '3';} $ \rightarrow $ \code{str[i]} lvalue
\end{itemize}
Formally, an \textbf{lvalue} is a storage location, something whose addresses we
can obtain.
\begin{lstlisting}
    int y = 10;
    int &z = y;
\end{lstlisting}
\begin{itemize}
    \item \code{z} is an lvalue reference to \code{y}
    \item \code{z} acts as a constant pointer to \code{y} with
    \textbf{automatic dereferencing}
    \item \code{z = 20;} $ \rightarrow $ automatically dereferencing,
    don't need \code{*z} (actually a compile error); changes \code{y} to \code{20}
    \item \code{z} becomes an alias to \code{y}
    \item \code{int *p = \&z} $ \rightarrow $ gets \code{y}'s address
    \item \code{z} behaves like \code{y}
    \item \code{int \&} is a type, \textbf{not} an address
\end{itemize}
References must be initialized to lvalues.
\begin{itemize}
    \item \code{int \&z;} $ \rightarrow $ illegal
    \item \code{int \&z = 3;} $ \rightarrow $ illegal
    \item \code{int \&z = a + b;} $ \rightarrow $ illegal
\end{itemize}
\begin{itemize}
    \item Cannot create a pointer to a reference.
    \item Cannot create a pointer to an array of reference.
    \item Cannot create a pointer to a reference to a reference.
\end{itemize}
\begin{lstlisting}
    void inc(int &n) { // n is an alias for x
        n = n + 1; // n acts like a constant pointer to x with automatic dereferencing
    }
    int x = 5;
    inc(x); // x is passed by reference
    cout << x; // prints 6
\end{lstlisting}
\code{cin >{}> x} $ \rightarrow $ \code{x} is passed by reference
