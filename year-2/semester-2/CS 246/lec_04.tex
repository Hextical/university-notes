\section{2020-01-16}
\subsection{Shell Scripts}
Shell scripts: text files containing Linux commands executed as a program.
Information to a program: \textbf{arguments}, \code{stdin}

We can provide arguments to a script. Arguments are available in special
variables named \code{\$1}, \code{\$2}, etc.

File: \code{isItAWord}
\begin{lstlisting}
    #!/bin/bash
    egrep "^$1$" /usr/share/dict/words
\end{lstlisting}
\begin{itemize}
    \item \code{./isItAWord hello} \textrightarrow{} finds \code{hello} in
          \code{/usr/share/dict/words}
\end{itemize}

Every process sets a status code: $ 0 $ a success, non-zero for failure.
\code{\$?} \textrightarrow{} last status code

Run: \code{[ 1 -eq 2]} \code{echo \$?} \textrightarrow{} returns $ 0 $ because $ 1\neq 2 $

File: \code{goodPassword}
\begin{lstlisting}
    #!/bin/bash
    # Answers whether a word is in the dictionary (and therefore not a good
    #  password)

    egrep "^$1$" /usr/share/dict/words > /dev/null

    if [ $? -eq 0 ]; then
        echo Not a good password
    else
        echo Maybe a good password
    fi
\end{lstlisting}
\begin{itemize}
    \item \code{/dev/null} \textrightarrow{} equivalent to discarding output
\end{itemize}

\code{if} statement:
\begin{lstlisting}
    if [  ]; then
        ...
    elif [  ]; then
        ...
    else
        ...
    fi
\end{lstlisting}

\code{while} loop:
\begin{lstlisting}
    while [  ]; do
        ...
    done
\end{lstlisting}

File: \code{goodPasswordCheck} \textrightarrow{} same as \code{goodPassword},
but checks for the correct number of arguments by adding the following
(exits with a non-zero code if incorrect number of arguments are supplied):

\begin{lstlisting}
    if [ ${#} -ne 1 ]; then
        echo "Usage: $0 password" >&2
        exit 1
    fi
\end{lstlisting}

\begin{itemize}
    \item \code{\$\{\#\}} \textrightarrow{} number of arguments to the script
\end{itemize}

File: \code{count}
\begin{lstlisting}
    #!/bin/bash
    # count limit ---counts the numbers from 1 to limit

    x=1
    while [ $x -le $1 ]; do
        echo $x
        x=$((x + 1))
    done
\end{lstlisting}
\begin{itemize}
    \item \code{./count 10} \textrightarrow{} prints out numbers 1 to 10,
          each on a new line
    \item \code{\$((x+1))} is proper addition for \code{int} data type
\end{itemize}

Run: \code{x=1}

\code{echo \$((x+1))} \textrightarrow{} outputs \code{2}

\code{echo \$x+1} \textrightarrow{} outputs \code{1+1}

File: \code{renameC}

\begin{lstlisting}
    #!/bin/bash
    # Renames all .C files to .cc
    
    for name in *.C; do
        mv ${name} ${name%C}cc
    done
\end{lstlisting}
\begin{itemize}
    \item given a \code{file}, \code{mv \$\{file\} \$\{file\%C\}cc}
          \textrightarrow{} renames \code{file.C}
          to \code{file.cc} \textrightarrow{} removes \code{C}, adds \code{cc}; that is,
          anything after \code{\%} is removed
\end{itemize}

Files: \code{countWords}, \code{payday}

\subsection{Summary of Files}
Files covered in this lecture found in \code{1201/lectures/shell/scripts}:
\begin{itemize}
    \item \code{basic}
    \item \code{isItAWord}
    \item \code{goodPassword}
    \item \code{goodPasswordCheck}
    \item \code{renameC}
    \item \code{count}
    \item \code{countWords}
    \item \code{payday}
\end{itemize}
