\section{2020-03-18}
\subsection{Exceptions}
\begin{lstlisting}
    v.at(i) // checks whether the index i is in range
    // throws an out_of_range exception
    // if a thrown exception is not caught, the program terminates
    try {
        ...
    } catch (...) {
        ...
    } catch (...) {
        ...
    }
\end{lstlisting}
File: \code{callchain.cc}

Stack unwinding: call stack is repeatedly popped until an appropriate
catch block is found. If \code{main}'s stack frame is popped, then
the program terminates with an uncaught exception.

Error recovery can happen in stages
\begin{lstlisting}
    f() {
        try {...}
        catch (someExn e) { // the exception that is caught might be a subtype of someExn
            // partial recovery
            throw SomeOtherExn("...");
        }
    }
\end{lstlisting}
A few options when throwing an exception with a catch block:
\begin{itemize}
    \item \code{throw SomeOtherExn(``\textellipsis'');}
    \item \code{throw e;} $ \rightarrow $ e might have been sliced
    \item \code{throw;} $ \rightarrow $ throw the original exception (want this in most cases)
\end{itemize}

\section{2020-03-19}
C++ exceptions all inherit from an ``exception'' class.
\begin{lstlisting}
    try {...}
    catch (exception &e) { // by reference
        ...
    }
\end{lstlisting}
In C++ you can throw any value. How do I catch any exception? C++ has special syntax for this:
\begin{lstlisting}
    try {
        ...
    } catch (...) { // ... is the literal syntax for ANYTHING (not the usual ... used in notes)
        ...
    }
\end{lstlisting}
While we can throw anything, we should use good design.
\begin{itemize}
    \item use c++ exception class
          \begin{itemize}
              \item \code{out\_of\_range}
              \item \code{bad\_alloc}
          \end{itemize}
    \item create your own classes
          \begin{itemize}
              \item \code{class InvalidMove\{\};}
              \item \code{class BadInput\{\};}
          \end{itemize}
\end{itemize}
\begin{lstlisting}
    int readInt() {
        int n;
        if (!(cin >> n)) throw BadInput{};
    }
    try {
        n = readInt();
    } catch (BadInput &) {
        n = 0;
        // use default
    }
\end{lstlisting}
Destructors and Exceptions: By default if the destructor throws an exception, the program
terminates.
\section{2020-03-22: Factory Method Pattern}
