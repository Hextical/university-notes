\section{2020-02-04}
\subsection{Preprocessor, Separate Compilation}
Source code $ \rightarrow $ Preprocessor $ \rightarrow $ Compiler
$ \rightarrow $ \code{a.out} executable

\code{\#include} is a direct copy/paste for the preprocessor directive

\code{\#include} with quotes $ \rightarrow $ look in current directory,
e.g. \code{\# include ``file''}

\code{g++ -E FILE} shows what the \code{include} copy/pastes

\code{1201/lectures/c++/4-preprocess}

File: \code{hello.cc}

\begin{itemize}
    \item \code{\#define VAR VALUE} $ \rightarrow $ searches and replaces
    \code{VAR} with \code{VALUE}
    \subitem \code{\#define MAX 10} $ \rightarrow $ replaces \code{MAX} with \code{10}
    \subitem Obsolete because now we have \code{const}
\end{itemize}

\subsection{Conditional Compilation}
File: \code{course.cc}

\code{g++14 -DVAR=VALUE FILE} $ \rightarrow $ changes type in command-line
for \code{course.cc}

Preprocessor comment; nests perfectly:
\begin{lstlisting}
    #if 0
        ...
    #endif
\end{lstlisting}

Block comments (less powerful):
\begin{lstlisting}
    /*
    ...
    */
\end{lstlisting}

\begin{lstlisting}
    #define VAR // VAR gets empty string
    #ifdef VAR // if VAR is defined, true
    #ifndef VAR // if VAR is not defined, true
\end{lstlisting}

File: \code{debug.cc}

\code{g++14 -DDEBUG debug.cc} $ \rightarrow $ defines \code{DEBUG}, so
one can see full verbose due to the \code{\#ifdef}s in \code{debug.cc}

Note: \code{-D} can be used to define multiple variables

\subsection{Separate Compilation}
\begin{itemize}
    \item Header files (\code{.h}): Declarations of functions, Global Variables,
    and Type Definitions
    \item Implementation files (\code{.cc}): Definitions of functions
\end{itemize}

\code{1201/lectures/c++/5-separate}

File: \code{example1}

Compile the files with either:
\begin{itemize}
    \item \code{g++14 main.cc vec.cc}
    \item \code{g++14 *.cc}
\end{itemize}
Notes:
\begin{itemize}
    \item Implementation files (\code{.cc}) are never \code{include}d
    \item Implementation files (\code{.cc}) are compiled
    \item Header (\code{.h}) files are \textbf{never compiled}, they are \code{include}d
\end{itemize}

Want to compile each file separately to produce a position of the executable,
then finally merge these positions.

By default, \code{g++} will compile and link to produce the executable.

\code{g++14 -c vec.cc}, \code{g++14 -c main.cc} $ \rightarrow $ compiles each
separately, without the executable (\code{.o} (Object) files are produced).

\code{g++14 main.o vec.o -o myprog} $ \rightarrow $ merges them into
the \code{myprog} executable

\subsection{Build Tools}
Don't memorize any of this section.

\textbf{make}: Automatically use ``last modified timestamp'' (uses \code{ls -l})

Specify dependencies in a \code{Makefile}.

\code{1201/lectures/tools/1-make}

File: \code{example1}
\begin{itemize}
    \item \code{.PHONY} $ \rightarrow $ specifies that \code{clean} is not a file, but
    a command
    \subitem \code{clean} $ \rightarrow $ \code{make clean} will delete any \code{.o} files
    in the current directory
\end{itemize}

File: \code{example2}
\begin{itemize}
    \item Uses variables for compilation
    \item \code{-Wall} $ \rightarrow $ warn all, compiler will give errors for warnings
\end{itemize}

File: \code{example3}
\begin{itemize}
    \item \code{main.o}, \code{vec.o}, \code{myprog} within
    the Makefile is all one needs to change for A2Q5
\end{itemize}