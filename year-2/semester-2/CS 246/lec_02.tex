\section{2020-01-09}
It's strongly recommend that you \textbf{do not} memorize these commands presented,
you should try them out on your own to see what the output is.

\begin{itemize}
    \item \code{CTRL + C} $ \rightarrow $ send kill signal
    \item \code{CTRL + D} $ \rightarrow $ send EOF (end-of-file)
    \item NAME: \code{cat} $ \rightarrow $ concatenate files and print on the standard output
    \subitem SYNOPSIS: \code{cat [OPTION]$\ldots$ [FILE]}
    \subitem DESCRIPTION: Concatenate FILE(s) to standard output.
    With no \code{FILE}, or when \code{FILE} is \code{-}, read standard input.
    \item \code{>} $ \rightarrow $ output redirection, overwrites files
    \subitem \code{cat > out.txt} $ \rightarrow $ redirects output produced by
    \code{cat} to the file \code{out.txt}
    \subitem \code{cat t1.txt > t2.txt} $ \rightarrow $ redirects all text from 
    \code{t1.txt} to \code{t2.txt}
    \item \code{>{}>} $ \rightarrow $ appends at the end of the file instead of overwriting like 
    \code{>}
    \item \code{<} $ \rightarrow $ input redirection
    \subitem \code{cat < sample.txt} $ \rightarrow $
    input redirection, the shell handles this
    \subitem \code{cat sample.txt} $ \rightarrow $
    \code{cat} handles this
    \subitem \code{cat -n < in > out} $ \rightarrow $
    \code{-n} numbers all output lines. Input redirect from file \code{in} to
    \code{cat}, then output redirect with numbered lines to file \code{out}.
\end{itemize}

\subsection{Linux Streams}

\begin{itemize}
    \item \myuline{1. Standard input (stdin)}
    \subitem keyboard
    \subitem use \code{<} to change to file
    \item \myuline{2. Standard output (stdout)}
    \subitem terminal
    \subitem use \code{1>} to change to file; the \code{1} before the \code{>} 
    is optional
    \subitem buffered
    \item \myuline{3. Standard error (stderr)}
    \subitem terminal
    \subitem use \code{2>} to change to file
    \subitem non-buffered
\end{itemize}

We use the non-buffered stream when we immediately want to output an error
so that it does not take extra CPU cycles (extra material).

Within the stream,

\code{stdin} $ \rightarrow $ program $ \rightarrow $ 
1. \code{stdout} and 2. \code{stderr}

\code{prog arg1 < in > out 2>\&1} $ \rightarrow $
\code{\&1} is the location of \code{stdout}, so any errors will be redirected to
\code{stdout}.

\subsection{Wildcard Matching}
$ \underbrace{\code{ls *.txt}}_{\text{globbing pattern}} \rightarrow $
match anything that ends with \code{.txt}. The shell performs this operation.

Using single/double quotes will suppress globbing patterns.

\code{\textbackslash} is the escape character

\myuline{Example}

Count the number of words in the first 15 lines of \code{sample.txt}.

\emph{Solution.}

\begin{itemize}
    \item \code{wc -w} $ \rightarrow $ print number of words in entire text
    \item \code{head -15 sample.txt} $ \rightarrow $ get only the first 15 lines
    of \code{sample.txt}
    \item \code{head -15 sample.txt > temp.txt wc -w temp.txt} $ \rightarrow $
    doing both, with output in a \code{temp.txt} file.
\end{itemize}

What if we didn't want \code{temp.txt} to be produced? We use Linux pipes.

\subsection{Linux Pipe}
Connect \code{stdout} of prog1 to \code{stdin} of prog2.

\code{head -15 sample.txt | wc -w} $ \rightarrow $
the first program, \code{head} runs with \code{sample.txt}, then the output is fed into
the second program, \code{wc}.

\myuline{Example}

Suppose \code{words*.txt} contains one word per line. Produce
a list of words sorted, with no duplicates from \code{words*.txt}.

\emph{Solution.}

\code{cat words*.txt | sort -u} OR 
\code{cat words*.txt | sort | uniq} $ \rightarrow $ \code{sort -u} will sort
and remove any duplicate words. \code{uniq} removes duplicates.

\code{echo Today is \$(date)} $ \rightarrow $
\code{\$(date)} is embedding a command date

Double quotes: does not supresses embedded commands

Single quotes: supresses embedded commands