\section{2020-03-03}
\subsection{Accessors/Mutators (Getters/Setters), System modelling}
Advice: keep fields \code{private}

\begin{lstlisting}
    class Vec {
        int x, y;
        public:
            // accessors/getters
            int getX() const { return x; }
            int getY() const { return y; }
            // mutators/setters
            void setX(int _x) { x = _x; }
            void setY(int _y) { y = _y; }
    };
\end{lstlisting}

\subsection{I/O Operators}
\begin{itemize}
    \item standalone functions
    \item need to access fields
          \subitem Option 1: provide provide accessors/mutators
          \subitem Option 2: make I/O operators \code{friend}s
\end{itemize}

\begin{lstlisting}
    class Vec {
        int x, y;
        friend ostream &operator<<(ostream &, const Vec &);
    };
\end{lstlisting}

\subsection{System Modelling}
\begin{itemize}
    \item identifying main entities (what are the main classes in the program)
    \item relationship between entities
    \item UML: Unified Modelling Language
\end{itemize}

A class in UML: (box with three sections)
\begin{enumerate}[(1)]
    \item Class
          \subitem \code{Vec}
    \item (Optional) Fields
          \subitem -\code{x}: Integer
          \subitem -\code{y}: Integer
    \item (Optional) Methods
          \subitem +\code{getX()}: Integer
          \subitem +\code{setX(Integer)}
\end{enumerate}
Constructors and the Big 5 are not shown.

\emph{Note}:
\begin{itemize}
    \item \code{-} $ \rightarrow $ private
    \item \code{+} $ \rightarrow $ public
\end{itemize}

Relationship 1: Composition (OWNS-A)
\begin{lstlisting}
    class Vec {
        int x, y;
        public:
            Vec(int, int);
            // default constructor `could' go here
    };
    class Basis {
        Vec v1, v2;
        // let's put it here instead to avoid calling Vec
        public:
            Basis() : v1{0, 1}, v2{1, 1} {}
    };
\end{lstlisting}

Basis is composed of 2 \code{Vec} objects. Instead, we write:
`Basis OWNS-A \code{Vec}'. Typically, if A OWNS-A B:
\begin{enumerate}[(1)]
    \item copying \code{A} copies \code{B} (deep)
    \item destroying \code{A} destroys \code{B}
\end{enumerate}

\begin{lstlisting}
    class List {
        Node *theList; // List OWNS-A Node
    };
\end{lstlisting}

\code{Basis} OWNS-A \code{Vec}:

TODO: Diamond is shaded
$ \diamond\longrightarrow^2_{\code{v1}, \code{v2}} $
\begin{itemize}
    \item Draw a diamond from \code{Basis} with the arrow head
          pointing \code{Vec}
    \item Above the arrow, write the number of how many objects are owned
    \item Below the arrow, write the objects that are owned
\end{itemize}

Relationship 2: Aggregation (HAS-A)

Typically, \code{A} HAS-A \code{B} if:
\begin{itemize}
    \item copying \code{A} does not copy \code{B} (shallow)
    \item destroying \code{A} does not destroy \code{B}
\end{itemize}

Same drawing as OWNS-A, but diamond is not shaded.

Relationship 3: Inheritance (IS-A)

Three classes:
\begin{enumerate}[(1)]
    \item Class
          \subitem \code{Book}
    \item Fields
          \subitem -\code{title}: String
          \subitem -\code{author}: String
          \subitem -\code{numPages}: Integer
\end{enumerate}

\begin{enumerate}[(1)]
    \item Class
          \subitem \code{Text}
    \item Fields
          \subitem -\code{title}: String
          \subitem -\code{author}: String
          \subitem -\code{numPages}: Integer
          \subitem -\code{topic}: String
\end{enumerate}

\begin{enumerate}[(1)]
    \item Class
          \subitem \code{Comic}
    \item Fields
          \subitem -\code{title}: String
          \subitem -\code{author}: String
          \subitem -\code{numPages}: Integer
          \subitem -\code{hero}: String
\end{enumerate}

Observe:
\begin{itemize}
    \item \code{Text} IS-A \code{Book} with an additional \code{topic} field
    \item \code{Comic} IS-A \code{Book} with an additional \code{hero} field
\end{itemize}

We represent \code{Text} IS-A \code{Book} by drawing an arrow from \code{Text}
with the head pointing to \code{Book}. Similarly, \code{Comic}.
The new \code{Text} and \code{Comic} class will only have the class name
with the extra field.

We refer to the \code{Book} class as a:
\begin{itemize}
    \item Superclass
    \item Parent
\end{itemize}

We refer to the \code{Comic} and \code{Text} class as a:
\begin{itemize}
    \item Subclass
    \item Child
\end{itemize}

\begin{lstlisting}
    class Book {
        string title, author;
        int numPages;
        public:
            Book(string, string, int);
    };
    // public inheritance (similar to extends keyword in Java)
    class Text : public Book {
        // only need to write new field
        string topic;
        ...
    };
\end{lstlisting}

\code{Text} inherits \textbf{all} (public \textbf{and} private)
members from \code{Book}.
\begin{itemize}
    \item any method that could be called on \code{Book} can also
          be called on \code{Text}
\end{itemize}
\code{Text} has inherited the \code{private} fields.
\begin{lstlisting}
    int main() {
        Text t{...};
        t.author // cannot access this
    }
\end{lstlisting}

\begin{lstlisting}
    // will not compile (even if we changed all fields to public)
    Text::Text(string t, string a, int n, string topic) :
        title{t}, author{a}, numPages{n}, topic{topic} {}
\end{lstlisting}
\begin{itemize}
    \item Inherited fields: private in base class (\code{title}, \code{author},
          \code{numPages})
    \item MIL can only refer to fields declared by the class
\end{itemize}

\textbf{Steps of Object Construction}:
\begin{enumerate}[(1)]
    \item Space is allocated
    \item Superclass part is constructed
    \item Subclass' field initialization/MIL
    \item Subclass constructor runs
\end{enumerate}

\begin{lstlisting}
    Text::Text(string t, string a, int numPages, String topic) : // (1)
        Book(t, a, n), // (2) and (3)
        topic{topic} // (4)
        {}
\end{lstlisting}

Visibility: \code{protected} (UML \#)
\begin{itemize}
    \item members that are \code{protected} are visible to the class
          and its subclasses.
    \item breaks encapsulation
          \subitem child classes can break invariants
\end{itemize}
Compromise: keep fields \code{private} but provide \code{protected}
accessors and mutators.

