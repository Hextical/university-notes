\section{2020-03-10}
\subsection{Templates, STL, Design Patterns}
\begin{lstlisting}
    class Stack {
        int *content;
        int capacity; // capacity of array
        int length; // useful items in array
        public:
            Stack();
            void push(int);
            int top() const; // get the top of the stack
            void pop();
            ~Stack();
    };
\end{lstlisting}
Suppose we wanted a different data type from \code{int}, a \code{string}. One option
would be to manually copy the \code{.cc} and \code{.h} file and replace the types.
A better option, is to use \textbf{C++ Templates}.

\textbf{C++ template class}: a class parameterized on one or more types.
\begin{lstlisting}
    template <typename T> // T is a parameter with a type
    class Stack {
        T *content;
        int capacity; // capacity of array
        int length; // useful items in array
        public:
            Stack();
            void push(T);
            T top() const; // get the top of the stack
            void pop();
            ~Stack();
    };
\end{lstlisting}
There are only minor changes for this simple class.
\begin{lstlisting}
    stack<int> s;
    s.push(s);
    // stack where each entry of the element of the stack is another stack of type string
    stack<stack<string>> bla;
\end{lstlisting}
For the \code{List} class:
\begin{lstlisting}
    template <typename T>
    // not parameterizing the node class
    class List {
        struct Node {
            T data;
            Node *next;
        };
        ...
        public:
            class Iterator {
                Node *curr;
                Iterator(Node *);
                public:
                    T &operator*();
                    Iterator &operator++();
                    bool operator!=(Iterator &);
                    friend class List<T>;
            };
            T ith(int idx);
            void addToFront(T &);
            ~List();
    };
\end{lstlisting}
\begin{lstlisting}
    List<int> l1;
    l1.addToFront(1);
    List<List<int>> l2;
    l2.addToFront(l1);
    // a is a list of ints that are copied by value
    for(auto a : l2) { ... }
    // let's take it by reference
    for (auto &a : l2) {
        for (auto b : a) {
            cout << b;
        }
    }
\end{lstlisting}

\textbf{STL} (Standard Template Library)

\code{std::vector} $ \longleftrightarrow $ dynamic length arrays
\begin{itemize}
    \item automatically resize as needed (possibly even shrink)
\end{itemize}

\begin{lstlisting}
    // Can even be used as a Queue
    #include <vector> // ArrayList from Java
    ...
    vector<int> v; // stack allocated, empty vector
    vector<int> v{3,4}; // [3, 4]
    v.emplace_back(5); // [3, 4, 5] // automatically resizes
    // emplace_back can be more efficient than push_back as it uses move whenever
    // possible; place_back will do a copy
    v.pop_back(); // pop
    v.erase(v.begin()); // vectors have iterators
    v.erase(v.begin() + 3);
    // Suppose we wanted to traverse the vector; there are a lot of options.
    // Option 1
    for (int i = 0; i < v.size(); ++i) {
        cout << v[i]; // overloaded []
    }
    // Option 2
    for (vector<int>::iterator it = v.begin(); it != v.end(); ++it) {
        ...
    }
    // Option 3 (don't need access to the iterator)
    for (auto &i : v) {
        ...
    }
    // Option 4 (reverse iterate), no shortcut based for loop
    // r.begin() (reverse begin)
    // r.end() (reverse end)
    for (vector<int>::reverse.iterator it = v.rbegin(); it != v.rend(); ++it) {
        ...
    }
    // Option 2 and 4 can use auto in the beginning of the for loop
\end{lstlisting}

Code to an interface not to an implementation.
\begin{itemize}
    \item Create abstract classes that define the interface
    \item (Destructor needs \code{virtual} A4Q1), else memory leaks will occur
    \item Work with pointers of this abstract type
    \subitem call the interface methods
    \subitem use \code{virtual} methods where behaviour differs
\end{itemize}

\subsection{Abstract Iterator Design Pattern}
\begin{lstlisting}
    class AbsIter {
        public:
            // need the virtual keyword, else breaks
            virtual int &operator*() const = 0; // pure virtual method
            virtual AbsIter &operator++() = 0;
            virtual bool operator!=(const AbsIter &) = 0;
            virtual ~AbsIter();
    };
\end{lstlisting}

\begin{lstlisting}
    class List {
        ...
        public:
            class Iterator : public AbsIter {
                ...
                public:
                    // override inherited pure virtual methods
            };
    };
\end{lstlisting}

\begin{lstlisting}
    class Set {
        public:
            class Iterator : public AbsIter {
                ...
            };
    };
\end{lstlisting}

\begin{lstlisting}
    void each(AbsIter &start, cons AbsIter &end) {
        while (start != end) {
            // do something with *start
            ++start;
        }
    }
\end{lstlisting}

\begin{lstlisting}
    template <typename Fn>
    // template function
    void foreach(AbsIter &start, const AbsIter &end, Fn f) {
        while (start != end) {
            f(*start); // can only take one parameter (possible to do more)
            ++start;
        }
    }
    void addOne(int &n) { n = n+1; }
    List l;
    // done add to front
    List::Iterator begin = l.begin();
    foreach(begin, l.end(), addOne);
\end{lstlisting}
