\section{2020-03-05}
\subsection{Method Overloading}
\code{bool isHeavy()}
\begin{itemize}
    \item Books $ > 200 $
    \item Test $ > 500 $
    \item Comic $ > 30 $
\end{itemize}

\code{c++/8-inheritance/example1}
\begin{lstlisting}
    class Book {
        string title, author;
        int numPages;
        public:
            int getPages() const { return numPages; }
            bool isHeavy() const { return numPages > 200; }
            class Text : public Book {
                string topic;
                public:
                    bool isHeavy() const { return getPages() > 500; }
            };
            bool Comic::isHeavy() const { return getPages() > 30; }
    };
    
    Book b{..., ..., 100};
    b.isHeavy(); // returns false
    Comic c{..., ..., 40, ...};
    c.isHeavy(); // Comic::isHeavy, returns true
    // b2 will not contain hero; `object slicing/cohesion'
    Book b2 = Comic{..., ..., 40, ...}; // all comics are books, (converse false)
    b2.isHeavy(); // Book::isHeavy, returns false
    Comic c{..., ..., 40, ...};
    Comic *pc{&c};
    pc->isHeavy(); // Comic::isHeavy, returns true
    // Note: auto will make the type Comic
    Book *pb{&c}; // legal b/c all Comics are Books; no slicing
    pb->isHeavy(); // Book:isHeavy
\end{lstlisting}
The compiler looks at the declared type of the pointer to choose the method.
\code{pb} is a \code{Book} pointer, so \code{Book::isHeavy} runs. We
would rather the decision be made on the actual type of object.

\textbf{Rice's Property}
\begin{lstlisting}
    // legal
    if (...) bp = &c;
    else bp = &book;
\end{lstlisting}

\code{virtual} methods: the choice of which method to run is decided
based on the runtime type of the object a pointer is pointing to.

\begin{lstlisting}
    // only needed in parent
    class Book {
        ...
        public:
            virtual bool isHeavy() const;
    };
\end{lstlisting}

\begin{lstlisting}
    bool Book::isHeavy() const {...}
    // override: prompts compiler to check the method that you 
    // override had an implementation in the base class
    // e.g. isheavy() or forgetting const -> override catches this
    // override can only be used for virtual methods
    bool Comic::isHeavy() const override {...}
    bool Text::isHeavy() const override {...}

    Comic c{..., ..., 40, ...};
    Comic *pc{&c};
    Book *pb{&c};
    Book &br{c};
    // Java behaviour (due to virtual)
    pc->isHeavy(); // Comic::isHeavy
    pb->isHeavy(); // Comic::isHeavy
    br.isHeavy(); // Comic::isHeavy
\end{lstlisting}
Since the program makes the decision, we call this \textbf{dynamic dispatch}.

\code{example4}
\begin{lstlisting}
    // array that store 20 Book pointers, can point to: Comic, Book, Text
    Book *collection[20]; // polymorphic array
    for (int i = 0, i < 20; ++i) {
        collection[i]->isHeavy();
        // dynamically dispatch to the appropriate isHeavy
    }
\end{lstlisting}
\textbf{Polymorphism}: ability to accommodate multiple types
under one abstraction.

\code{inheritance/example5}: Destructors
\begin{lstlisting}
    struct x {
        int *x;
        X(int n) : x{new int[n]} {}
        ~x() { delete[] x; }
    };
    // y IS-A x
    struct y : public x {
        int *y;
        Y(int n, int m) : X{n}, y{new int[m]} {}
        ~Y() { delete[] y; }
    };
\end{lstlisting}

\textbf{Steps for Destroying Subclass Objects}:
\begin{enumerate}[(1)]
    \item Subclass constructor
    \item Subclass fields that are objects are destroyed
    \item Superclass part is destroyed (destructor is called)
    \item Space is reclaimed
\end{enumerate}
\begin{lstlisting}
    Y myY{10, 20}; // 2 does not happen for this when it goes out of scope
    X *myX = new Y{10, 20};
    delete myX; // calls ~X(), leaks memory
    // we need to make ~X() virtual, then this problem is fixed
\end{lstlisting}
If a class \textbf{might} have subclasses, declare the destructor virtual,
even if the destructor body is empty.

If a class should not have subclasses, declare it \code{final}.
\begin{lstlisting}
    class Abc final {
        ...
    };
    class Xyz : public Abc {}; // does not compile
\end{lstlisting}

Pure Virtual methods (P.V):
\begin{enumerate}[(1)]
    \item Class
          \subitem \emph{\code{Student}} (italics for abstract)
    \item Fields
          \subitem +\code{fees()}: int
\end{enumerate}

\code{Regular} \& \code{Co-op} IS-A \code{Student} (both point towards \code{Student} box)
\begin{enumerate}[(1)]
    \item Class
          \subitem \code{Regular}
    \item Fields
          \subitem +\emph{\code{fees()}: int}
\end{enumerate}

\begin{enumerate}[(1)]
    \item Class
          \subitem \code{Co-op}
    \item Fields
          \subitem +\emph{\code{fees()}: int}
\end{enumerate}

\emph{Note}: \code{virtual} $ \rightarrow $ \emph{italics}

\begin{lstlisting}
    class Student {
        public:
            // pure virtual method (bizarre syntax)
            virtual int fees() = 0;
    };
    class Regular : public Student {
        public:
            int fees() override {...}
    }
\end{lstlisting}

\code{Student} is an \textbf{abstract} class. A class is \textbf{abstract}
if:
\begin{enumerate}[(1)]
    \item it declares a pure virtual method
    \item it inherits a pure virtual method that it does not override
\end{enumerate}
We cannot instantiate abstract classes.
\begin{lstlisting}
    Student s{..., ..., ...}; // does not compile (attempt to instantiate abstract class)
\end{lstlisting}
We say a class is \textbf{concrete} if it is not abstract.
\begin{itemize}
    \item organize entities (place common attributes, behaviours)
          in the base class
    \item can still use \code{Student} pointers
    \subitem \code{Student *arr[200]; for (...) { arr[i]->fees; }}
\end{itemize}
