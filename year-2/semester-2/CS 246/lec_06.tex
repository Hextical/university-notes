\section{2020-01-23}
\subsection{C++ I/O}
\code{cin >{}> x >{}> y;} \textrightarrow{}
\code{cin >{}> y;} \textrightarrow{}  \code{cin;}

If a read fails, all subsequent attempts to read fail, unless
you acknowledge that failure.

Read and print all ints from \code{stdin}. Terminate on EOF\@.
Ignore ``bad input'' (non-int).

File: \code{readInts5.cc}
\begin{lstlisting}
    #include <iostream>
    using namespace std;

    int main() {
      int i;
      while (true) {
        if (!(cin >> i)) {
          if (cin.eof()) {
            break;         // nothing left to read
          } else {         // read failed, infinite loop without below
            cin.clear();   // set flags to false
            cin.ignore();  // ignore next character
          }
        } else {
          cout << i << endl;
        }
      }
    }

\end{lstlisting}

\subsection{C++ Strings}
In C, we used null-terminated character arrays. In C++, we have
a string type, header: \code{<string>}

\code{string str = ``hello'';} \textrightarrow{} creates a null terminated
string in both C and C++

C++ strings automatically resize.

\begin{tabular}{c|c|c}
                & C                    & C++             \\
  \hline
  comparisons   & \code{strcmp(s1,s2)} & \code{s1 == s2} \\
  length        & \code{strlen(s)}     & \code{s.length} \\
  concatenation & \code{strcat(s1,s2)} & \code{s1 + s2}
\end{tabular}
When we do \code{s1 + s2}, we create a new string. If we wanted to
append s1 to s2, we can do \code{s1 = s1 + s2}.

File: \code{readStrings.cc}

\begin{lstlisting}
    #include <iostream>
    #include <string>
    using namespace std;
    
    int main() {
      string s;
      cin >> s;
      cout << s << endl;
    }    
    
\end{lstlisting}

\begin{itemize}
  \item reads until first whitespace (ignore all whitespace until first
        character)
\end{itemize}
\code{getline(cin,s)} \textrightarrow{} reads until a new line

In C, we used \code{\%d}, \code{\%x}, etc.\ for \code{printf}. In C++, we
can use the \code{<iomanip>} header as follows:
\begin{lstlisting}[language = C++]
    int x = 24;
    cout << x; // prints 24 in decimal
    cout << hex; // switch cout to hexadecimal
    cout << x; // prints 24 in hexadecimal
    cout << dec; // switch cout to decimal
\end{lstlisting}

\code{<fstream>} \textrightarrow{} input file stream

\code{<ofstream>} \textrightarrow{} output file stream

File: \code{fileInput.cc}
\begin{lstlisting}
    #include <fstream>
    #include <iostream>
    using namespace std;
    
    int main() {
      ifstream file{"t.txt"};
      string s;
      while (file >> s) {
        cout << s << endl;
      }
    }    
\end{lstlisting}
\code{$\underbrace{\text{ifstream}}_{\text{type}} \underbrace{\text{file}}_{\text{variable}}
    \underbrace{\{\text{``t.txt''\}}}_{\text{initialization of variable}}$}
\textrightarrow{} uniform initialization syntax, only in $ \geqslant\text{C++11} $

Other examples of uniform initialization syntax:
\begin{itemize}
  \item \code{int x\{5\}};
  \item \code{string s\{``hello''\}};
\end{itemize}

\code{<sstream>}
\begin{itemize}
  \item \code{istringstream}
  \item \code{ostringstream}
\end{itemize}

File: \code{buildString.cc}, \code{getNum.cc}

\subsection{Summary of Files}
Files covered in this lecture found in \code{1201/lectures/c++/2-io}:
\begin{itemize}
  \item \code{readInts5.cc}
  \item \code{readStrings.cc}
  \item \code{fileInput.cc}
  \item \code{buildString.cc}
  \item \code{getNum.cc}
\end{itemize}
