\section{2020-01-30}
\subsection{More Short C++ Topics}
Why does \code{cin >{}> x} work?

\code{operator >{}> (cin, x);} $ \rightarrow $ \code{x} is passed by reference

\code{istream \&operator >{}> (istream \&in, int \&n);}

We take the stream by reference since 
\begin{enumerate}
    \item we want changes to the stream be changes to \code{cin}.
    \item streams cannot be copied.
\end{enumerate}
Pass by value vs pass by reference
\begin{lstlisting}
    struct ReallyBig {...};
    void f(ReallyBig rb); // pass by value, costly
    void g(ReallyBig &rb); // pass by reference: avoid copy
    void h(const ReallyBig &rb); // avoid copy, h cannot change what rb refers to
\end{lstlisting}
Advice: prefer to pass arguments as reference to \code{const} for anything bigger
than an \code{int}.
\begin{lstlisting}
    void f(int &n);
    f(5); // 5 is not an lvalue; illegal
    f(x + y); // illegal
    int temp = x + y;
    f(temp); // legal
    // pass by reference to constant
    void g(const int &n);
    g(5); // legal
    g(x + y); // legal
\end{lstlisting}
\subsection{Dynamic Memory Allocation}
\begin{itemize}
    \item In C, \code{int *p = malloc (sizeof(int) * length);}, \code{free(p);}.
    \item In C++, \code{malloc, free, realloc} $ \rightarrow $ banned
\end{itemize}
\begin{lstlisting}
    // {data, next};
    Node n{5, nullptr}; // n is on stack
    // np is on stack, pointing to the heap
    Node *np = new Node {2, nullptr}; // on heap, new figures out how much memory is needed
    n.next = np;
    delete np;
\end{lstlisting}
Stack allocated nodes:
\begin{lstlisting}
    Node myNodes[10];
    Node *np = new Node[10];
    delete np; // undefined behaviour
    delete []np; // syntax to deallocate an array
\end{lstlisting}
\begin{lstlisting}
    // return by value, copy is made
    Node getNode() {
        Node n;
        ...
        return n;
    };
    // compiles, "Dangling Pointer", but incorrect
    // to correct: heap allocate + return pointer to heap node
    Node *getNode() {
        Node n;
        ...
        return &n;
    }
\end{lstlisting}

\subsection{Operator Overloading}
Define the meaning of C++ operators for user-defined types.
\begin{lstlisting}
    struct Vec {
        int x;
        int y;
    };
    Vec v1{1, 2};
    Vec v2{1, 2};
    Vec v3 = add(v1, v2); // In C, we would do this

    // Overload the + operator such that `v1 + v2' works
    Vec operator+(const Vec &vec1, const Vec &vec2) {
        Vec toRet{vec1.x + vec2.x, vec1.y + vec2.y};
        return toRet;
    }

    Vec v4 = v1 + v2 + v3; // works, wouldn't work without const
    
    // Overload the * operator such that `c * v' works
    Vec operator*(int c, const Vec &vec) {
        return {c * vec.x, c * vec.y};
    }

    Vec v = 3 * v2; // works
    Vec v = v2 * 3; // doesn't work
    
    Vec operator*(const Vec &vec, int c) {
        return c * vec;
    }
\end{lstlisting}

\begin{lstlisting}
    struct Student {
        int grade;
    };
    Student s{15};
    cout << s.grade << "%";
    
    // Overload the << operator such that `cout << s' works
    ostream operator<<(ostream &out, const Student &s) {
        out << s.grade << "%";
        return out;
    }

    // Overload the >> such that `cin >> s' works
    istream &operator>>(istream &in, Student &s) {
        int n;
        in >> n;
        s.grade = (n < 0) ? 0 : n; // short hand if statement
        s.grade = (s.grade < 100) ? 100 : s.grade;
        return in;
    }
\end{lstlisting}
