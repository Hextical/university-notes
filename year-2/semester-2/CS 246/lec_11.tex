\section{2020-02-11}
\subsection{C++ Classes (Cont.)}
Last time: Initializing objects

\begin{lstlisting}
    struct Student {
        const int id;
    };
\end{lstlisting}
How to initialize fields that are constants or lvalue references?

Fields must always already be initialized before constructor body runs;
i.e. step 2 occurs before step 3 below

\textbf{Steps for Object Construction}
\begin{enumerate}[(1)]
    \item allocate space: could be on stack or could be on heap if new
    \item field initialization: only fields that are \textbf{objects} are initialized
    \subitem their default constructor gets called
    \item constructor body runs
\end{enumerate}
`Hijack' (2): \textbf{Use Member Initialization List} (MIL)

Example (4 fields, 4 parameter constructor):
\begin{lstlisting}
    Student::Student(int id, int assigns, int mt, int final)
        // pairs separated by commas
        // before curly brace are field names
        // initialize id to id, assigns to assigns, ...
        : id{id}, assigns{assigns}, mt{mt}, final{final} {
            ... // constructor body
        }
\end{lstlisting}
MIL should use field declaration order; will compile with a warning if you do not
do this, but it will initialize according to field order.

Not that we did not need to use \code{this} to disambiguate field/parameter.

MIL is necessary to initialize constants and references that can be used for
all fields.

Using the MIL can be more efficient than using the constructor body.

Special constructor that takes a single parameter: Constructing objects
as copies of other objects

\begin{lstlisting}
    Student billy{75, 50, 65}; // three parameter constructor {assigns, mt, final}
    Student bobby{billy}; // Uses copy constructor
\end{lstlisting}
Constructing an object as a copy of another called the \textbf{copy constructor}.
There is always a built-in constructor for any class.

\code{.cc} (You get this for free already)
\begin{lstlisting}
    // crucial to be by reference; will not compile if done by value
    Student::Student(const Student &other)
    : assigns{others.assigns}, mt{others.mt}, final{others.final} {}
\end{lstlisting}
When is a copy constructor called?
\begin{enumerate}[(1)]
    \item Explicitly constructing an object as a copy of another
    \item Pass by value
    \item Return by value
\end{enumerate}
Every class comes with (gets for free):
\begin{enumerate}[(I)]
    \item default constructor
    \item copy constructor
    \item copy assignment operator
    \item destructor
    \item move constructor
    \item move assignment operator
\end{enumerate}
The \textbf{Big 5}: (II)-(VI).

\begin{lstlisting}
    struct Node {
        int data;
        Node *next;
        // two parameter constructor
        Node(int data, Node *next);
        // free copy constructor
        Node(const Node &other);
    };
    Node::Node(int data, Node *next) : data{data}, next{next} {}
    Node::Node(const Node &other) : data{other.data}, next{other.next} {}
\end{lstlisting}

\textbf{Shallow copy}:
\begin{lstlisting}
    Node *n = new Node{1, new Node{2, new Node{3, nullptr}}};
    // Three heads with shared tails, this is bad
    // stack allocated node
    // m on stack with head same as n's head, but tail points to n's tail (on heap)
    Node m{*n}; // make a copy of *n and make it m
    // heap allocate the copy
    // p on stack, same as m, but with first node in heap
    Node *p = new Node{*n};
\end{lstlisting}
Sometimes we want a \textbf{deep copy} (need to write your own copy constructor):
\begin{lstlisting}
    Node::Node(const Node &other) {
        data = others.data;
        // INCORRECT, segementation fault if next is nullptr (dereferencing nullptr)
        next = new Node{*other.next};
        // CORRECT
        if (others.next) {
            next = new Node{*other.next};
        } else {
            next = nullptr;
        }
    }
\end{lstlisting}
More compactly,
\begin{lstlisting}
    Node::Node(const Node &other) :
    data{other.data},
    next{other.next ? new Node {*other.next} : nullptr} {}
\end{lstlisting}

One parameter constructors create ``implicit'' conversions.
\begin{lstlisting}
    Node::Node(int data) : data{data}, next{nullptr} {}
\end{lstlisting}

\begin{lstlisting}
    void foo(Node n) {
        ...
    }
    Node n{4};
    Node n = 4;
    // Both possible, implicit conversion
    foo(n);
    foo(5);
\end{lstlisting}
We can declare a constructor ``explicit'' to disable implicit conversions:
\begin{lstlisting}
    struct Node {
        ...
        explicit Node(int);
    };
\end{lstlisting}

\subsection{Destructor}
\textbf{Destructor:} method that gets called when objects are destroyed.
\begin{itemize}
    \item Stack: when stack is popped
    \item Heap: when delete is called on a pointer to a heap allocated object
\end{itemize}
A class can only have \textbf{one} destructor, they cannot be overloaded.

\textbf{Steps for Object Deconstruction}
\begin{enumerate}[(1)]
    \item Destructor body runs
    \item Fields that are objects are destroyed (reverse declaration order)
    \item Space is reclaimed
\end{enumerate}
Free destructor has an empty destructor body.
\begin{lstlisting}
    // Allocate everything on heap
    Node *p = new Node{1, new Node{2, new Node{}, nullptr}};
    delete p; // leaks Node 2 and Node 3
\end{lstlisting}

\code{.h}
\begin{lstlisting}
    struct Node {
        ~Node();
    };
\end{lstlisting}
\code{(.c)}
\begin{lstlisting}
    Node::~Node() {
        delete next; // recursively deletes next (including nullptr which is safe)
    }
\end{lstlisting}

\subsection{Copy Assignment Operator}
\begin{lstlisting}
    Student billy{..., ..., ...};
    // Bobby was born a cheater
    Student bobby{billy}; // done already
    // Jane existed, then started cheating off Billy
    Student jane; // default constructor
    jane = billy;
    jane.operator=(billy); // equivalent to line above (don't do this)
\end{lstlisting}
The \textbf{Copy Assignment Operator} is used to assign to existing objects.
