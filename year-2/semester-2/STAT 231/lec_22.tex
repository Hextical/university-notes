\chapter{Lectures}
\section{2020-03-02}
\underline{Roadmap}:
\begin{enumerate}[(i)]
    \item 5 min recap
    \item Confidence for Normal with unknown variance
    \item Prediction Intervals
    \item Relationship between likelihood intervals and confidence intervals
\end{enumerate}
\[ W \thicksim \chi^2_n\iff W=Z_1^2+Z_2^2+\cdots+Z_n^2 \]
where each $ Z_i \thicksim N(0,1) $ and $ Z_i $'s independent. We know
$ E(W)=n $ and $ Var(W)=2n $.

Let $ W_1 \thicksim \chi^2_{n_1} $ and $ W_2 \thicksim \chi^2_{n_2} $
be independent, then
\[ W_1+W_2 \thicksim \chi^2_{n_1+n_2} \]

\underline{Student's T-distribution}

We say $ T \thicksim T_n $ if
\[ T=\frac{Z}{\sqrt{\nicefrac{W}{n}}}  \]
where $ Z \thicksim N(0,1) $ and $ W \thicksim \chi^2_n $ are independent. Note
that $ E(T)=0 $ and $ T $ is symmetric. Also, as $ n \to \infty $, then
$ T \to Z \thicksim N(0,1) $.

\begin{thmbox}
    \begin{theorem}
        Let $ Y_1,\ldots ,Y_n $ be iid $ N(\mu,\sigma^2) $ where $ \mu $ and $ \sigma $ are
        unknown. Let
        \[ \overline{Y}=\frac{1}{n} \sum\limits_{i=1}^{n} Y_i \]
        and
        \[ S^2=\frac{1}{n-1} \sum\limits_{i=1}^{n} (Y_i-\overline{Y})^2 \]
        Then,
        \begin{enumerate}[(i)]
            \item The pivotal quantity for $ \mu $ is:
                  \[ \frac{\overline{Y}-\mu}{\frac{S}{\sqrt{n}}} \thicksim T_{n-1}  \]
            \item The pivotal quantity for $ \sigma^2 $ is:
                  \[ \frac{(n-1)S^2}{\sigma^2} \thicksim \chi^2_{n-1}  \]
        \end{enumerate}
    \end{theorem}
\end{thmbox}
\begin{remark}
    (i) Shows that if we replace $ \sigma $ by its estimator $ S $, then it follows a $ T $-distribution
    with $ (n-1) $ degrees of freedom.
\end{remark}

\begin{exbox}
    \begin{example}
        An independent sample of $ 25 $ students are taken and STAT 231 scores are recorded.
        \begin{itemize}
            \item $ \overline{y}=75 $
            \item $ s^2=\frac{1}{n-1} \sum\limits_{i=1}^{n} (y_i-\overline{y})^2=64 $
        \end{itemize}
        \begin{enumerate}[(a)]
            \item Find the $ 99\% $ confidence interval for $ \mu $.
            \item Find the $ 95\% $ confidence interval for $ \sigma^2 $.
            \item Find the $ 99\% $ prediction interval for $ Y_{26} $.
        \end{enumerate}
        \textbf{Solution.} We know $ Y_1,\ldots ,Y_{25} \thicksim N(\mu,\sigma^2) $
        where $ Y_i= $ STAT 231 score of the $ i^{\text{th}} $ student.

        (a) We know
        \[ \frac{\overline{Y}-\mu}{\frac{S}{\sqrt{n}}} \thicksim T_{24} \]
        We want a $ t $ such that
        \[ P(|T_{24}|\leqslant t)=0.99 \]
        Using the table we see that $ t=2.80 $. Now,
        \[ P(-2.8\leqslant T_{24}\leqslant 2.8)=0.99 \]
        \[ \implies P\left(-2.8\leqslant \frac{\overline{Y}-\mu}{\frac{S}{\sqrt{n}}}
            \leqslant 2.8\right)=0.99 \]
        \[ \implies P\left(\overline{Y}-2.8 \frac{S}{\sqrt{n}}\leqslant \mu\leqslant \overline{Y}+
            2.8 \frac{S}{\sqrt{n}}\right)=0.99 \]
        Thus, the $ 99\% $ confidence interval is:
        \[ \overline{y}\pm 2.8 \frac{s}{\sqrt{n}}\implies \left[ 62.2, 87.8 \right] \]

        (b) We know
        \[ \frac{(n-1)S^2}{\sigma^2} \thicksim \chi^2_{24}  \]
        We want any value $ a $ and $ b $ such that
        \[ P(a\leqslant \chi^2_{24}\leqslant b)=0.95 \]
        We choose the symmetric solution with $ a=0.025\rightarrow 13.120 $ and $ b=0.975\rightarrow 40.646 $.
        Now,
        \[ P\left( 13.120\leqslant \chi^2_{24}\leqslant 40.646 \right)=0.95 \]
        \[ \implies P\left( 13.120\leqslant \frac{(n-1) S^2}{\sigma^2}\leqslant 40.646 \right)=0.95 \]
        \[ \implies P\left( \frac{(n-1)S^2}{40.646}\leqslant \sigma^2 \leqslant \frac{(n-1)S^2}{13.120} \right)=0.95 \]
        Thus, the $ 95\% $ confidence interval for $ \sigma^2 $ is:
        \[ \left[ \frac{(n-1)s^2}{40.646} , \frac{(n-1)s^2}{13.120} \right]\implies
            \left[ 37.79, 117.07 \right] \]

        (c) Prediction interval.
        \[ Y_{26} \thicksim N(\mu,\sigma^2) \]
        \[ \overline{Y} \thicksim N(\mu,\nicefrac{\sigma^2}{n}) \]
        \[ \implies Y_{26}-\overline{Y} \thicksim N\left(0,\sigma^2\left( 1+\frac{1}{n}  \right)\right) \]
        Therefore, the pivotal quantity is:
        \[ \frac{Y_{26}-\overline{Y}}{\sigma \sqrt{1+\frac{1}{n}}} =Z \thicksim N(0,1) \]
        we replace $ \sigma $ by its estimator and get
        \[ \frac{Y_{26}-\overline{Y}}{S \sqrt{1+\frac{1}{n}}} \thicksim T_{24} \]
        Thus,
        \[ P(|T_{24}|\leqslant 2.8)=0.99 \]
        yields the general $ 99\% $ prediction interval:
        \[ \overline{y}\pm t^* s \sqrt{1+\frac{1}{n}} \]
    \end{example}
\end{exbox}
We make the following remark:
\begin{remark} Let $ Y_1,\ldots ,Y_{n} $ be iid $ N(\mu,\sigma^2) $. Then,
    \begin{enumerate}[(i)]
        \item The general confidence interval for $ \mu $ is:
              \[ \overline{y}\pm z^* \frac{\sigma}{\sqrt{n}} \quad\text{if $\sigma$ is known} \]
              \[ \overline{y}\pm t^* \frac{s}{\sqrt{n}} \quad\text{if $\sigma$ is unknown} \]
        \item The general confidence interval for $ \sigma^2 $ is:
              \[ \left[ \frac{(n-1)s^2}{b} ,\frac{(n-1)s^2}{a} \right] \]
              where $ a $ and $ b $ come from the $ \chi^2_{n-1} $ table
              and $ b-a= $ RHS.
        \item The general prediction interval for $ Y_{n+1} $ is:
              \[ \overline{y}\pm t^* s \sqrt{1+\frac{1}{n}} \]
    \end{enumerate}
\end{remark}

\begin{thmbox}
    \begin{theorem}
        As $ n\to \infty $,
        \[ \Lambda(\theta)=-2\ln\left[  \frac{L(\theta)}{L(\tilde{\theta})} \right] \thicksim \chi^2_1  \]
        where $ \tilde{\theta} $ is the maximum likelihood estimator. We call the random
        variable $ \Lambda(\theta) $ the likelihood ratio statistic.
    \end{theorem}
\end{thmbox}

\begin{exbox}
    \begin{example}
        Suppose $ n $ is large, and we have a $ 10\% $ likelihood interval. What is the corresponding
        coverage probability?

        \textbf{Solution.} $ 10\% $ likelihood interval $ \implies R(\theta)\geqslant 0.1 $
        \[ \implies \frac{L(\theta)}{L(\hat{\theta})}\geqslant 0.1  \]
        \[ \implies -2\ln\left[  \frac{L(\theta)}{L(\hat{\theta})} \right]\leqslant -2\ln(0.1)  \]
        \[ \implies \lambda(\theta)\leqslant -2\ln(0.1) \]
        Thus, the corresponding coverage:
        \begin{align*}
            P(\Lambda(\theta)\leqslant -2\ln(0.1))
             & = P(Z^2\leqslant -2\ln(0.1))        \\
             & = P(|Z|\leqslant \sqrt{-2\ln(0.1)}) \\
             & \approx 97\%
        \end{align*}
    \end{example}
\end{exbox}
