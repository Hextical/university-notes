\section{2020-01-22}
\underline{STAT 231}: Characteristics of the population
are unknown.

\subsection*{Data Summary}
\begin{itemize}
    \item Extract important properties.
    \item Fit the right model.
\end{itemize}

\subsection*{Disappearance of the 400 hitter}
\begin{itemize}
    \item Batting average $ \stackrel{?}{=} $ proportion of successes.
    \item Battling champion = person with the highest batting average.
    \item Before $ 1950 $: 3 champions $ \geqslant 400 $.
    \item Since $ 1953 $: 0.
\end{itemize}
\emph{Question}: Why?

\subsection*{Arguments}
\begin{itemize}
    \item Absolute.
    \item Relative.
    \item Better pitchers: Relief.
    \item Better fielding: Glove sizes.
    \item Better managing.
\end{itemize}
All these arguments are incorrect.

The average points of the generic batter is roughly the same over time,
but the standard deviation decreases by a lot. Thus, we have a tighter Gaussian
distribution for the model today compared to back then since the average
player is pretty good (before there was huge variability).

\textbf{``The median isn't the message''---Stephen Jay Gould}
\subsection*{Intro to Statistical Models}
\begin{Definition}{}{}
    A \textbf{\emph{statistical model}} is a specification of the
    distribution from which the data set is drawn, where the attribute of interest
    is a parameter of that distribution.
\end{Definition}
\begin{Example}{}{}
    A coin is tossed $ 200 $ times with $ y=110 $ heads. What can we say
    about the ``fairness'' of the coin?

    The attribute of interest is
    \[ \Prob{H}=\text{probability of heads}=\theta=\text{unknown} \]
    Based on our sample, we try to ``estimate'' $ \theta $.
    Let $ Y $ be the number of heads when we toss a coin $ 200 $ times,
    then our statistical model is: $ Y \sim \bin{200,\theta} $
    with $ y=110 $.
\end{Example}



\begin{Example}{}{}
    How good are Canadians on Jeopardy? Let $ \{y_1,\ldots ,y_{10}\} $
    be our data set where $ y_i $ is the number of shows that the
    $ i^{\text{th}} $ Canadian appeared on.
    \[ \theta=\Prob{\text{Canadian wins Jeopardy}} \]
    Is $ \hat{\theta}\gg 1/3 $?
    \[ \{y_1=2,y_2=3,y_3=1,y_4=5\} \]
    \begin{itemize}
        \item $ y_1=\theta(1-\theta) $
        \item $ y_4=\theta^4(1-\theta) $
    \end{itemize}
    Then, our statistical model is $ Y_i \sim \geo{1-\theta} $
    for $ i=1,\ldots ,10 $.
\end{Example}

\subsection*{Objective}
The average salary of a UW co-op student is $ \$10000 $ per term.
Is this claim true? Suppose $ \{y_1,\ldots ,y_{100}\} $ is given and
$ Y_i \sim \N{\mu,\sigma^2} $
where each $ i\in[1,100] $ are independent. We will answer this question later in the course.
