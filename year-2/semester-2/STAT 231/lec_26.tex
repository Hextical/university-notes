\section{2020-03-11}
\underline{Roadmap}:
\begin{enumerate}[label=(\roman*)]
    \item Testing for normal problems
    \item How to test for a ``bias'' of a scale
    \item One-sided tests
    \item Relationship between C.I and H.T
    \item Other distributions
\end{enumerate}
\underline{Problem}: $ Y_1,\ldots ,Y_n \sim N(\mu,\sigma^2) $ iid.
\begin{itemize}
    \item $ H_0 $: $ \mu=\mu_0 $
    \item $ H_1 $: $ \mu\neq \mu_0 $
\end{itemize}
\underline{Steps involved}:
\begin{enumerate}[label=(\roman*)]
    \item Construct the Discrepancy measure $ D $ (satisfying the properties), this
          measures how much the data disagrees with $ H_0 $
    \item Calculate the value of $ D $ from your sample ($ d $)
    \item $ p\text{-value}=P(D\geqslant d; H_0\text{ is true}) $
    \item Draw appropriate conclusions based on your $ p\text{-value} $
\end{enumerate}
\begin{exbox}
    \begin{example}
        The STAT 231 scores are normally distributed with mean $ \mu $
        and variance $ \sigma^2=49 $.
        \begin{itemize}
            \item $ H_0 $: $ \mu=75 $
            \item $ H_1 $: $ \mu\neq 75 $
        \end{itemize}
        A random sample of $ 25 $ students are taken $ \bar{y}=72 $.
        Find the $ p $-value.

        \textbf{Solution.}
        From Chapter $ 4 $ we know that
        \[ \frac{\bar{Y}-\mu}{\frac{\sigma}{\sqrt{n}}} =Z \sim N(0,1) \]
        \[ D=\left|\frac{\bar{Y}-\mu_0}{\frac{\sigma}{\sqrt{n}}}\right| \]
        where we can see that $ D $ is a legitimate test statistic as
        it satisfies all the required properties since:
        \begin{enumerate}
            \item $ D\geqslant 0 $ for all $ d $
            \item $ D=0\implies $ best news for $ H_0 $
            \item High values of $ D \implies $ bad news for $ H_0 $
            \item Probabilities can be calculated if $ H_0 $ is true
        \end{enumerate}
        Thus, we have
        \[ d=\left|\frac{\bar{y}-\mu_0}{\frac{\sigma}{\sqrt{n}}}\right|=
            \left|\frac{72-75}{\frac{7}{\sqrt{5}}} \right|=\frac{15}{7} =2.14 \]
        \begin{align*}
            p\text{-value}
             & =P(D\geqslant d)      \\
             & =P(|Z|\geqslant 2.14) \\
             & <0.05
        \end{align*}
        Evidence against $ H_0 $.
    \end{example}
\end{exbox}

\begin{exbox}
    \begin{example}
        UW brochure claims that the average starting salary of UW graduates
        is $ \$60000\text{/year} $. We assume normality. We want to test this claim.
        Let $ \bar{y}=58000 $ and $ s=5000 $. What should you conclude?

        \textbf{Solution.}
        \begin{itemize}
            \item $ H_0 $: $ \mu=60000 $
            \item $ H_1 $: $ \mu\neq 60000 $
        \end{itemize}
        \[ D=\left|\frac{\bar{Y}-\mu_0}{\frac{S}{\sqrt{n}}}\right| \]
        where all the properties of $ D $ are satisfied.
        \[ d=\left|\frac{\bar{y}-60000}{\frac{5000}{\sqrt{25}}}\right|=2 \]
        \begin{align*}
            p\text{-value}
             & =P(D\geqslant d)        \\
             & =P(|T_{24}|\geqslant 2)
        \end{align*}
        The $ p $-value for this test is between $ 5\% $ and $ 10\% $. Weak evidence
        against $ H_0 $.
    \end{example}
\end{exbox}
