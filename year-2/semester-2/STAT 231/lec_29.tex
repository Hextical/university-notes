\makeheading{2020-03-18}
\underline{Roadmap}:
\begin{enumerate}[(i)]
    \item 5 min recap
    \item LTRS for large $ n $
    \item An example
\end{enumerate}

$ Y_1,\ldots ,Y_n $ iid $ \thicksim N(\mu,\sigma^2) $
\begin{itemize}
    \item $ H_0 $: $ \sigma^2=\sigma_0^2 $
    \item $ U=\frac{(n-1)s^2}{\sigma_0^2} \thicksim \chi^2_{n-1} $
\end{itemize}
We calculated the $ p $-value:
\[ U=\frac{(n-1)s^2}{\sigma_0^2}  \]
If
\begin{itemize}
    \item $ U > $ median $ \chi^2_{n-1} \implies $ $ p\text{-value}=2P(U\geqslant u) $
    \item $ U < $ median $ \chi^2_{n-1} \implies $ $ p\text{-value}=2P(U\leqslant u) $
\end{itemize}
\underline{Exercise}: Construct the $ 95\% $ confidence interval for $ \sigma^2 $.
Then, check if $ \sigma_0^2(4)\in 95\% $ confidence interval.
\begin{itemize}
    \item $ H_0 $: $ \sigma^2=4 $ (more than $ 10\% $, so it is in the $ 95\% $
          confidence interval)
\end{itemize}

\underline{Likelihood Ratio Test Statistic} (one parameter)

$ Y_1,\ldots ,Y_n $ iid $ f(y_i;\theta) $ with $ n $ large.
\begin{itemize}
    \item Sample: $ \{y_1,\ldots ,y_n\} $
    \item $ \theta= $ unknown parameter
    \item $ H_0 $: $ \theta=\theta_0 $
    \item $ H_1 $: $ \theta=\theta_0 $
\end{itemize}
\underline{Step 1}: Test statistic:
\[ \Lambda=-2\ln\left[ \frac{L(\theta)}{L(\tilde{\theta})}  \right] \]
If $ H_0 $ is true:
\[ \Lambda=-2\ln\left[ \frac{L(\theta)}{L(\tilde{\theta})}  \right]
    \thicksim \chi^2_{1} \]
\underline{Step 2}: Calculate $ \lambda $
\[ \lambda=-2\ln\left[ \frac{L(\theta_0)}{L(\hat{\theta})}  \right]=-2\ln
    \left[ R(\theta_0) \right] \]
\begin{align*}
    p\text{-value}
     & =P(\Lambda\geqslant \lambda) \\
     & =P(Z^2\geqslant \lambda)     \\
     & =1-P(|Z|\leqslant \lambda)
\end{align*}

\begin{exbox}
    \begin{example}
        Suppose $ Y_1,\ldots ,Y_n \thicksim f(y_i;\theta) $ iid. where
        \[ f(y,\theta)=\frac{2y}{\theta}e^{-\nicefrac{y^2}{\theta}} \]
        Data: $ n=20 $, $ \sum y_i^2=72 $

        We want to test $ H_0 $: $ \theta=5 $ (two sided alternative).
        \begin{itemize}
            \item $ \hat{\theta}=\frac{1}{n} \sum y_i^2 =3.6 $
            \item $ R(\theta_0)=\frac{\hat{\theta}}{\theta_0}e^{(1-\nicefrac{\hat{\theta}}{\theta_0})^n} $
            \item $ \lambda(\theta_0) = \cdots $
        \end{itemize}
        We know $ \lambda=-2\ln \left[ R(\theta_0) \right]=1.9402 $ and so
        \[ R(\theta_0)=\frac{L(\theta_0)}{L(\hat{\theta})}=0.3791 \]
        also $ \theta_0=5 $. Lastly, calculate the $ p $-value.
        \begin{align*}
            p\text{-value}
             & =P(\Lambda\geqslant \lambda) \\
             & =P(Z^2\geqslant 1.9402)      \\
             & \approx 16.5\%
        \end{align*}
        Thus, no evidence against null-hypothesis ($ H_0 $).
    \end{example}
\end{exbox}
A few final points:
\begin{enumerate}[(i)]
    \item Careful about the previous example.
    \item $ \lambda $ and the relationship with $ R $
    \item Next video
\end{enumerate}
\begin{itemize}
    \item $ n=20 $ is not large
    \item $ \lambda=-2\ln\left[ R(\theta_0) \right] $: high values of $ \lambda\implies $
          low values of $ R(\theta_0) $
\end{itemize}
