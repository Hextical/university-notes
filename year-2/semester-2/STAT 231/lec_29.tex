\section{2020-03-18: Likelihood Ratio Test Statistic Example}
\underline{Roadmap}:
\begin{enumerate}[(i)]
    \item 5 min recap
    \item LTRS for large $ n $
    \item An example
\end{enumerate}

(i) \underline{5 min recap}

$ Y_1,\ldots ,Y_n $ iid $ \thicksim N(\mu,\sigma^2) $
\begin{itemize}
    \item $ H_0 $: $ \sigma^2=\sigma_0^2 $
    \item $ U=\frac{(n-1)S^2}{\sigma_0^2} \thicksim \chi^2_{n-1} $
\end{itemize}
We calculated the $ p $-value:
\[ u=\frac{(n-1)s^2}{\sigma_0^2}  \]
\begin{itemize}
    \item If $ u > $ median $ \chi^2_{n-1} \implies $ $ p\text{-value}=2P(U\geqslant u) $ (twice right tail)
    \item If $ u < $ median $ \chi^2_{n-1} \implies $ $ p\text{-value}=2P(U\leqslant u) $ (twice left tail)
\end{itemize}
\underline{Exercise} For \ref{small u variance},
\begin{itemize}
    \item Construct the $ 95\% $ confidence interval for $ \sigma^2 $.
    \item Check if $ \sigma_0^2(4)\in 95\% $ confidence interval.
\end{itemize}
We already know that $ H_0 $: $ \sigma^2=4 $ yields a $ p\text{-value}>0.1 $, so it should
be in the $ 90\% $ confidence interval $ \implies $ it's in the $ 95\% $ confidence interval.

(ii) \underline{LTRS for large $ n $}

$ Y_1,\ldots ,Y_n $ iid $ f(y_i;\theta) $ with $ n $ large.
\begin{itemize}
    \item Sample: $ \{y_1,\ldots ,y_n\} $
    \item $ \theta= $ unknown parameter
    \item $ H_0 $: $ \theta=\theta_0 $
    \item $ H_1 $: $ \theta\neq\theta_0 $
\end{itemize}
\underline{Step 1}: Test statistic:
\[ \Lambda(\theta)=-2\ln\left[ \frac{L(\theta)}{L(\tilde{\theta})}  \right] \]
If $ H_0 $ is true:
\[ \Lambda(\theta_0)=-2\ln\left[ \frac{L(\theta_0)}{L(\tilde{\theta})}  \right]
    \thicksim \chi^2_{1} \]
\underline{Step 2}: Calculate $ \lambda(\theta_0) $
\[ \lambda(\theta_0)=-2\ln\left[ \frac{L(\theta_0)}{L(\hat{\theta})}  \right]=-2\ln
    \left[ R(\theta_0) \right] \]
\begin{align*}
    p\text{-value}
     & =P(\Lambda\geqslant \lambda)      \\
     & =P(Z^2\geqslant \lambda)          \\
     & =1-P(|Z|\leqslant \sqrt{\lambda})
\end{align*}

(iii) \underline{An example}
\begin{exbox}
    \begin{example}
        Suppose $ Y_1,\ldots ,Y_n \thicksim f(y_i;\theta) $ iid where
        \[ f(y,\theta)=\frac{2y}{\theta}e^{-\nicefrac{y^2}{\theta}} \]
        \begin{itemize}
            \item $ n=20 $
            \item $ \sum\limits_{i=1}^{n} y_i^2=72 $
        \end{itemize}

        We want to test $ H_0 $: $ \theta=5 $ (two sided alternative).
        \begin{itemize}
            \item $ \hat{\theta}=\frac{1}{n} \sum\limits_{i=1}^{n}y_i=\frac{1}{20}(72)=3.6 $
            \item $ R(\theta_0)=\left( \frac{\hat{\theta}}{\theta_0} \right)^n
                      e^{\left(1-\frac{\hat{\theta}}{\theta_0}\right)n} = 0.379052 $
            \item $ \lambda(\theta_0) =-2\ln \left[ R(\theta_0) \right]=1.94016 $
        \end{itemize}
        \begin{align*}
            p\text{-value}
             & =P(\Lambda\geqslant \lambda)                       \\
             & =P(Z^2\geqslant 1.94016)                           \\
             & =1-\left[ 2 P(Z\leqslant \sqrt{1.94016})-1 \right] \\
             & = 1-\left[ 2(0.97381)-1 \right]                    \\
             & = 0.16452                                          \\
             & \approx 16.5\%
        \end{align*}
        Thus, no evidence against null-hypothesis ($ H_0 $).
    \end{example}
\end{exbox}
A few final points:
\begin{enumerate}[(i)]
    \item Careful about the previous example:
          \subitem $ n=20 $ is not large
    \item $ \lambda $ and the relationship with $ R $:
          \subitem high values of $ \lambda\implies $ low values of $ R(\theta_0) $
    \item Next video
\end{enumerate}

