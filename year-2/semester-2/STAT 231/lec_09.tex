\section{2020-01-24}
\underline{Roadmap}:
\begin{itemize}
    \item Statistical models
    \item Notations and Definitions
    \item Likelihood function for discrete data
    \item MLE (Maximum Likelihood Estimate)
\end{itemize}

A coin is tossed $ 100 $ times with $ y=40 $ heads. What can we say
about the fairness of the coin?

\underline{Step 1}: Identify the attribute of interest.
\begin{align*}
    \theta & =P(\text{H})                           \\
           & =\text{population proportion of heads} \\
           & =\text{population parameter}           \\
           & =\text{unknown constant}
\end{align*}
\underline{Step 2}: Estimate $ \theta $ using your data. Based on your data
set, what is the ``likely'' value of $ \theta $?
\begin{align*}
    \hat{\theta}(y_1,\ldots ,y_n) & =
    \text{number that can be calculated using our data set}           \\
                                  & =\text{point estimate of } \theta
\end{align*}
\underline{Step 3}: Given $ \hat{\theta} $, is $ \theta=0.5 $ ``reasonable''?

\underline{Notation}:
\begin{itemize}
    \item Population parameters are denoted with greek letter such as:
          $ \theta,\,\mu,\,\sigma^2,\,\tilde{n} $
    \item Data sets are denoted with English letter such as:
          $ y,\,y_1,\ldots ,y_n $ when the data set is unknown or
          $ \hat{\theta},\,\hat{\mu} $ if your data set is known.
    \item Random variables are denoted with upper case English letters such as:
          $ Y_1,\ldots ,Y_n,\,Y,\,Z $
    \item $ y=40 $ heads where $ y $ is an outcome of a Binomial experiment. Model:
          \[ Y \sim \bin(100,\theta) \]
\end{itemize}

\begin{exbox}
    \begin{example}
        Question: Will trump win Wisconsin in 2020? A sample of 500 people
        are picked up and 200 of them said that they will vote for Trump. Based
        on this data will Trump win in 2020?

        Let $ \theta=\text{proportion of the population that vote for Trump} $
        \[ Y \sim \bin(500,\theta) \]
    \end{example}
\end{exbox}

\begin{exbox}
    \begin{example}
        Suppose we are interested in the average number of texts a UW math student
        receives every half hour and $ n $ students were interviewed.

        Let $ \mu $ be the population average of texts received by a UW student.
        \[ Y_i \sim \poi(\mu) \]
        for $ i=1,\ldots ,n $.
    \end{example}
\end{exbox}

\begin{defbox}
    \begin{definition}
        A \textbf{\emph{point estimate}} of a parameter is the value of a function
        of the observed data $ y_1,\ldots ,y_n $ and other known quantities such as
        the sample size $ n $. We use $ \hat{\theta} $ to denote an estimate
        of the parameter $ \theta $.
    \end{definition}
\end{defbox}

\begin{defbox}
    \begin{definition}
        The \textbf{\emph{likelihood function}} for $ \theta $ is defined as
        \[ L(\theta)=L(\theta;\bm{y})=P(\bm{Y}=\bm{y}; \theta) \]
        for $ \theta\in\Omega $ where the \textbf{\emph{parameter space}} $ \Omega $
        is the set of all possible values for $ \theta $.
    \end{definition}
\end{defbox}

\begin{defbox}
    \begin{definition}
        The value of $ \theta $ which maximizes $ L(\theta) $ for given data $ \bm{y} $
        is called the \textbf{\emph{maximum likelihood estimate}} (m.l.\ estimate) of $ \theta $.
        It is the value of $ \theta $ which maximizes the probability of observing the data
        $ \bm{y} $. This value is denoted $ \hat{\theta} $.
    \end{definition}
\end{defbox}

\begin{exbox}
    \begin{example}
        A coin is tossed 100 times and we get $ y=40 $ heads. Let $ \theta $
        be the probability of heads. Find the MLE of $ \theta $.

        \[ L(\theta)=\binom{100}{40}\theta^{40}(1-\theta)^{60} \]
        \[ \ell(\theta)=\ln\left[ \binom{100}{40} \right]+40\ln(\theta)+60\ln(1-\theta) \]
        \[ \frac{d\ell}{d\theta}=\frac{40}{\theta} -\frac{60}{1-\theta} :=0 \]
        \[ \implies \hat{\theta}=0.4 \]
    \end{example}
\end{exbox}
We can generalize this further.
