\section{2020-03-13}
\underline{Roadmap}:
\begin{enumerate}[(i)]
    \item Recap and the relationship between Confidence and Hypothesis
    \item Example: Bias Testing
    \item Testing for variance (Normal)
    \item What if we don't know how to construct a Test-Statistic?
\end{enumerate}

\begin{exbox}
    \begin{example}
        $ Y_1,\ldots Y_n $ iid $ N(\mu,\sigma^2) $
        \begin{itemize}
            \item $ \sigma^2= $ known
            \item $ \mu= $ unknown
            \item Sample: $ \{y_1,\ldots ,y_n\} $
            \item $ \overline{y}= $ sample mean
            \item $ H_0 $: $ \mu=\mu_0 $ where $ \mu_0 $ is given
            \item $ H_1 $: $ \mu \neq \mu_0 $
        \end{itemize}
        \[
            \begin{aligned}
                D= & \left|\frac{\overline{Y}-\mu_0}{\frac{\sigma}{\sqrt{n}}} \right|
                   & \quad                                                            & \rightarrow &  & \text{Test-Statistic (r.v.)}       \\
                d= & \left|\frac{\overline{y}-\mu_0}{\frac{\sigma}{\sqrt{n}}} \right|
                   &                                                                  & \rightarrow &  & \text{Value of the Test-Statistic}
            \end{aligned}
        \]
        \[
            \begin{aligned}
                p\text{-value}
                 & =P(D\geqslant d)   & \quad & \text{assuming }H_0\text{ is true} \\
                 & =P(|Z|\geqslant d) &       & Z \thicksim N(0,1)
            \end{aligned}
        \]
    \end{example}
\end{exbox}
\underline{Question}: Suppose the $ p $-value for the test $ >0.05 $
if and only if $ \mu_0 $ belongs in the $ 95\% $ confidence interval for $ \mu $?

YES.

Suppose $ \mu_0 $ is in the $ 95\% $ confidence interval for $ \mu $, i.e.
\[ \overline{y}\pm 1.96 \frac{\sigma}{\sqrt{n}} \]
\[ \begin{aligned}
        \mu_0\leqslant \overline{y}+1.96 \frac{\sigma}{\sqrt{n}} \\
        \mu_0\geqslant \overline{y}-1.96 \frac{\sigma}{\sqrt{n}}
    \end{aligned}
\]
These two equations yield
\[ d=\left|\frac{\overline{y}-\mu_0}{\frac{\sigma}{\sqrt{n}}} \right|\leqslant 1.96 \]
\[ p\text{-value}=P(|Z|\geqslant d)>0.05 \]

\underline{General result} (assuming same pivot)

$ p $-value of a test $ H_0 $: $ \theta=\theta_0 $ vs $ H_1 $: $ \theta\neq \theta_0 $
is more than $ q\% $, then $ \theta_0 $ belongs to the $ 100(1-q)\% $
confidence interval and vice versa.

\begin{exbox}
    \begin{example}[Bias]
        A $ 10 $ kg weight is weighed $ 20 $ times ($ y_1,\ldots ,y_n $).
        \begin{itemize}
            \item $ \overline{y}=10.5 $
            \item $ s=0.4 $
            \item $ H_0 $: The scale is unbiased
            \item $ H_1 $: The scale is biased
        \end{itemize}
        If the scale was unbiased,
        \[ Y_1,\ldots ,Y_n \thicksim N(10,\sigma^2) \]
        If the scale was biased,
        \[ Y_1,\ldots ,Y_n \thicksim N(10+\delta,\sigma^2) \]
        \begin{itemize}
            \item $ H_0 $: $ \delta=0 $ (unbiased)
            \item $ H_1 $: $ \delta\neq 0 $ (biased)
        \end{itemize}
        is equivalent to
        \begin{itemize}
            \item $ H_0 $: $ \mu=10 $
            \item $ H_1 $: $ \mu\neq 10 $
        \end{itemize}
        Test-statistic:
        \[ D=\left|\frac{\overline{Y}-10}{\frac{S}{\sqrt{n}}} \right| \]
        Compute $ d $.
        \[ d=
            \left|\frac{\overline{y}-10}{\frac{s}{\sqrt{n}}} \right|=
            \left|\frac{10.5-10}{\frac{0.4}{\sqrt{20}}} \right|=5.59017 \]
        \[
            \begin{aligned}
                p\text{-value}
                 & =P(D\geqslant d)                             \\
                 & =P(|T_{19}|\geqslant 5.59)                   \\
                 & = 1-P(|T_{19}|\leqslant 5.59)                \\
                 & =1-\left[ 2P(T_{19}\leqslant 5.59)-1 \right] \\
                 & \approx 1-(2-1)                              \\
                 & =0
            \end{aligned}
        \]
        Very strong evidence against $ H_0 $.
    \end{example}
\end{exbox}

\begin{exbox}
    \begin{example}[Draw Conclusions]
        $ Y_1,\ldots ,Y_n = $ co-op salaries. $ Y_1,\ldots ,Y_n \thicksim N(\mu,\sigma^2) $
        \begin{itemize}
            \item $ H_0 $: $ \mu=3000 $
            \item $ H_1 $: $ \mu<3000 $ ($ \mu\neq 3000 $)
        \end{itemize}
        \[ D=\left|\frac{\overline{Y}-\mu_0}{\frac{S}{\sqrt{n}}} \right| \]
        \[ D=
            \begin{cases}
                0                                             & \overline{Y}>\mu_0 \\
                \frac{\overline{Y}-\mu_0}{\frac{S}{\sqrt{n}}} & \overline{Y}<\mu_0
            \end{cases} \]
    \end{example}
\end{exbox}
If $ n $ is large, then
\[ Y_1,\ldots ,Y_n \thicksim f(y_i;\theta) \]
\begin{itemize}
    \item $ H_0 $: $ \theta=\theta_0 $
    \item $ H_1 $: $ \theta\neq \theta_0 $
\end{itemize}
\[ \Lambda(\theta)=-2\ln\left[ \frac{L(\theta_0)}{L(\tilde{\theta})} \right] \]
where $ \Lambda $ satisfies all the properties of $ D $. Also,
\[ \lambda(\theta)=-2\ln\left[ \frac{L(\theta_0)}{L(\hat{\theta})}\right] =-2ln\left[ R(\theta_0) \right] \]
and
\[ p\text{-value}=P(\Lambda\geqslant \lambda)=P(Z^2\geqslant \lambda) \]
