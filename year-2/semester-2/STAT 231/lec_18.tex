\section{2020-02-14 \texorpdfstring{$\heartsuit$}{[Valentine's Day!]}}

\underline{Roadmap}:
\begin{itemize}
    \item Confidence interval for a Normal problem with known variance
    \item The Q-Q-plot, and how to interpret it?
\end{itemize}

\begin{Definition}{}{}
    A $ 100p\% $ confidence interval for $ \theta $ is an interval $
        [\ell,u] $ where $ \ell=\ell(y_1,\ldots ,y_n) $ and $ u=u(y_1,\ldots
        ,y_n) $ which is an estimate of the random interval (coverage interval)
    \[ \left[ \mathcal{L}(Y_1,\ldots ,Y_n),U(Y_1,\ldots ,Y_n) \right] \] such that
    \[ P\left( \mathcal{L}(Y_1,\ldots ,Y_n)\leqslant \theta\leqslant U(Y_1,\ldots
        ,Y_n) \right)=p \] where $ p $ is the coverage probability.
\end{Definition}

\underline{Problem}:  $ Y_1,\ldots ,Y_n $ are iid $ N(\mu,\sigma^2) $
\begin{itemize}
    \item $ \sigma^2= $ known
    \item $ \mu= $ unknown parameter of interest
    \item a probability is pre-specified
    \item Sample: $ \{y_1,\ldots ,y_n\} $
\end{itemize}
\underline{Objective}: To construct a $ 95\% $ confidence interval for $ \mu $.

\underline{Step 1}: Identify the sampling distribution of the estimator
\begin{itemize}
    \item $ \mu= $ attribute
    \item $ \bar{y}= $ sample mean = estimate
    \item $ \bar{Y}= $ estimator $ = \tilde{\mu} $
    \item If $ Y_1,\ldots ,Y_n \sim N(\mu,\sigma^2) $, then
          \[ \bar{Y} \sim N\left( \mu,\sigma^2/n \right) \]
\end{itemize}
\underline{Step 2}: Construct the pivotal quantity $ Q $

\begin{Definition}{}{}
    A \textbf{\emph{pivotal quantity}} $ Q((Y_1,\ldots ,Y_n);\theta) $ is a
    function of $ \left( Y_1,\ldots ,Y_n;\theta \right) $ (a random
    variable) whose probabilities can be calculated without knowing what $
        \theta $ is
    \[ P(Q\geqslant a)\; P(Q\leqslant b) \] can be calculated without
    knowing $ \theta $.
\end{Definition}

For example, if $ \bar{Y} \sim N(\mu,\sigma^2/n) $, then the pivotal
quantity is
\[ \frac{\bar{Y}-\mu}{\sigma/n} \] and the pivotal distribution is $ Z $.

\underline{Step 3}: Find the coverage interval using the pivotal distribution.
For $ 95\% $ we got
\[ \left[ \bar{Y}-1.96\frac{\sigma}{\sqrt{n}},
        \bar{Y}+1.96\frac{\sigma}{\sqrt{n}}\right] \] \underline{Step 4}: Estimate
the coverage interval using your data.

Confidence Interval:
\[ \left[ \bar{y}-1.96\frac{\sigma}{\sqrt{n}},
        \bar{y}+1.96\frac{\sigma}{\sqrt{n}}\right] \]

\underline{Notes}:
\begin{enumerate}[label=(\roman*)]
    \item Interpretation of a confidence interval.

          Coverage: $ \left[ \bar{Y}-1.96\frac{\sigma}{\sqrt{n}},
                  \bar{Y}+1.96\frac{\sigma}{\sqrt{n}}\right] $

          Confidence:$ \left[ \bar{y}-1.96\frac{\sigma}{\sqrt{n}},
                  \bar{y}+1.96\frac{\sigma}{\sqrt{n}}\right] $

          If we did this experiment many times, approximately $ 95\% $ of the
          intervals will contain $ \mu $.

    \item As the confidence level increases, the interval is wider.
    \item Unrealistic example since $ \sigma $ is known
    \item Can we choose the length of the interval? Yes.
\end{enumerate}

\underline{The Q-Q-Plot}

\underline{Model Selection}

The Q-Q plot is given by $ (y_{(\alpha)},z_{(\alpha)}) $ where
\begin{itemize}
    \item $ y_{(\alpha)} = \alpha^{\text{th}} $ quantile of your data set
    \item $ z_{(\alpha)} = \alpha^{\text{th}}  $ quantile of $ Z \sim
              N(0,1) $
\end{itemize}
If the Q-Q plot is linear, then there is evidence of normality.

Let $ Y \sim N(\mu,\sigma^2) $. Show that the Q-Q plot is a straight line.
\begin{Proof}{}{}
    \begin{align*}
         & P(Y\leqslant y_{(\alpha)})=\alpha                                                      \\
         & P\left( \frac{Y-\mu}{\sigma} \leqslant \frac{y_{(\alpha)}-\mu}{\sigma}  \right)=\alpha \\
         & P(Z\leqslant W)=\alpha                                                                 \\
         & F(W)=\alpha                                                                            \\
         & \implies W=z_{(\alpha)}                                                                \\
         & \implies \frac{y_{(\alpha)}-\mu}{\sigma}=z_{(\alpha)}                                  \\
         & \implies y_{(\alpha)}=\mu+\sigma z_{(\alpha)}
    \end{align*}
\end{Proof}

\underline{Clicker Question}:
\begin{itemize}
    \item $ n=100 $
    \item Confidence level: $ 95\% $
\end{itemize}
We want to half the length of the interval.
\[ \bar{y}\pm a\rightarrow \bar{y}\pm \frac{a}{2} \] How many
\underline{more} sample points do you need.
\begin{enumerate}[label=(\alph*)]
    \item $ 100 $
    \item $ 300 $
\end{enumerate}
