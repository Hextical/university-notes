\section{2020-02-28}
\begin{itemize}
    \item 5 min recap
    \item The Chi-squared and the T-distribution
    \item Normal problem with unknown variance
    \item Clicker questions
\end{itemize}
\underline{Confidence intervals}

Case I\@: Confidence interval for the mean for normal when $ \sigma $ is known
\[ \bar{y}\pm z^* \frac{\sigma}{\sqrt{n}}  \]
\begin{itemize}
    \item $ \bar{y}= $ sample mean
    \item $ \sigma= $ population standard deviation
    \item $ n= $ sample size
    \item $ z^* $ depends on the level of confidence
          \begin{itemize}
              \item $ z^*=1.96 $ if confidence level is $ 95\% $ for every $ n $
          \end{itemize}
\end{itemize}

Case II\@: Binomial Confidence
\[ Y \sim \bin(n,\theta) \]
\begin{itemize}
    \item $ \theta= $ probability of success (unknown)
\end{itemize}
Confidence interval is given by
\[ \hat{\theta}\pm \underbrace{z^* \sqrt{\frac{\hat{\theta}\left( 1-\hat{\theta} \right)}{n}}}_{\text{
            margin of error
        }} \]
\begin{itemize}
    \item $ \hat{\theta}= $ sample proportion
    \item $ n= $ sample size
\end{itemize}
If we want the margin of error to be $ \leqslant \ell $, then
\[ n\geqslant \left( \frac{z^*}{\ell} \right)^2\left( \frac{1}{4} \right) \]

\underline{The Chi-Squared Distriubtion}
\begin{defbox}
    \begin{definition}
        $ W $ is a continuous random variable taking all non-negative values.
        $ W $ is said to follow a \textbf{\emph{Chi-Squared}} distribution
        with $ n $ degrees of freedom (d.f), denoted $ W \sim \chi^2(n) $,
        if
        \[ W=Z_1^2+\cdots+Z_n^2 \]
        where $ Z_i \sim N(0,1) $ with $ Z_i $'s independent.
    \end{definition}
\end{defbox}
\underline{Properties of the Chi-Squared}
\begin{enumerate}[label=(\roman*)]
    \item $ n= $ d.f. = parameter of the Chi-squared. Once $ n $ is specified, the d.f.\ is known
    \item Density function looks like a gaussian distribution as $ df\rightarrow\infty $
    \item If $ W \sim \chi^2_n $, then $ E(W)=n $ and $ Var(W)=2n $
\end{enumerate}
Cases:
\begin{itemize}
    \item Case I\@: $ n=1 $, then $ W=Z^2 $
    \item Case II\@: $ n=2 $, then $ W \sim \exponential(2) $
    \item Case III\@: $ n $ is ``large'', then $ W \sim N(n,2n) $ approximately
    \item Case IV\@: $ n $ is intermediate, then we use the table
\end{itemize}

Let $ (X,Y) $ be a random point on a Cartesian plane. Assuming $ X $ and $ Y $
have independent $ G(0,1) $ distributions, the probability that a point is greater
than $ 1.96 $ away from the origin is

\textbf{Hint}: The distance formula is $ x^2+y^2=d^2 $.
\begin{enumerate}[label=(\Alph*)]
    \item \textbf{less than $ \bm{40\%} $}
    \item at least $ 40\% $ but less than $ 60\% $
    \item at least $ 60\% $ but less than $ 80\% $
    \item at least $ 80\% $
\end{enumerate}
Why? We know that $ D^2 \sim \exponential(2) $, then we compute the following.
\[ P(D\geqslant 1.96)=1-F(1.96)=1-\left( 1-\frac{1}{2} e^{-1.96/2} \right)\approx 0.19=19\% \]

\underline{The Student's T-distribution}
\begin{defbox}
    \begin{definition}
        $ T $ is said to follow a \textbf{\emph{Student's T-distribution}} with
        $ n $ degrees of freedom, denoted $ T \sim t(n) $, if
        \[ T=\frac{Z}{\sqrt{W/n}}  \]
        where $ Z \sim N(0,1) $ and $ W \sim \chi^2(n) $.
    \end{definition}
\end{defbox}
\underline{Properties}
\begin{enumerate}[label=(\roman*)]
    \item $ T $ can take all possible values
    \item $ T $ is symmetric around zero
    \item Similar to $ Z $, but with flatter tails
    \item As $ n\rightarrow+\infty $, then $ T\rightarrow Z $
\end{enumerate}
\underline{Clicker Question}:
\begin{itemize}
    \item $ Z \sim N(0,4) $
    \item $ T \sim t(15) $
    \item $ W \sim \chi^2(3) $
    \item $ Z,\;T,\;W $ are all independent
\end{itemize}
$ \E{W+T+\left( \frac{Z}{2} \right)^2} = $
\begin{enumerate}[label=(\Alph*)]
    \item $ 3 $
    \item $ \bm{4} $
    \item $ 5 $
    \item None of the above.
\end{enumerate}
Why?
\begin{itemize}
    \item $ \E{W}=3 $
    \item $ \E{T}=0 $ since $ T $ is symmetric around zero for $ n>1 $
    \item Let $ Y=\frac{Z}{2} $. Then,
          \[ \E{Y^2}=\Var{Y}+\E{Y}^2
              =\left( \frac{1}{2} \right)^2 \Var{Z}+\frac{1}{2} \E{Z}
              =\frac{1}{4}(4)+0
              =1 \]
\end{itemize}
Thus, $ \E{W+T+Y}=3+0+1=4 $.
