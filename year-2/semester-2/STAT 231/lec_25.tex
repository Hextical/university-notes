\section{2020-03-09}
\underline{Roadmap}:
\begin{enumerate}[(i)]
    \item Binomial testing
    \item Review for the midterm (excluded from these notes)
\end{enumerate}
\begin{defbox}
    \begin{definition}
        $ p $-value: Probability of observing as extreme an observation of your data,
        given the null hypothesis is true.
    \end{definition}
\end{defbox}

\begin{defbox}
    \begin{definition}
        A test statistic (discrepancy measure) is a random variable that measures
        the level of disagreement of your data with the null hypothesis. Typically,
        it satisfies the following properties:
        \begin{enumerate}[(i)]
            \item $ D\geqslant 0 $
            \item $ D=0 \implies $ best news for $ H_0 $
            \item High values of $ D \implies $ bad news for $ H_0 $
            \item Probabilities can be calculated if $ H_0 $ is true
        \end{enumerate}
    \end{definition}
\end{defbox}

\underline{Steps for a Statistical test}

\underline{Step 1}: Construct the test-statistic $ D $
\begin{exbox}
    \begin{example}
        Test whether a coin is fair (against the two sided alternative).
        Let $ n=100 $ and $ y=52 $ heads.
        \begin{itemize}
            \item $ H_0 $: $ \theta=\frac{1}{2} $
            \item $ H_1 $: $ \theta\neq \frac{1}{2} $
        \end{itemize}
        where $ \theta=P(H) $.

        \underline{Model}: $ Y \thicksim \bin(100,\theta) $.
        \[ D=|Y-50| \]
        as it satisfies (i)-(iv).
    \end{example}
\end{exbox}
\underline{Step 2}: Find $ d $ from your data set.
\[ p\text{-value}=P(D\geqslant d;H_0\text{ is true}) \]
\underline{Step 3}: Make conclusions based on your p-value

For our Binomial problem,
\[ D=|Y-50|\implies d=|52-50|=2 \]
Thus,
\[ p\text{-value}=P(|Y-50|\geqslant 2) \]
but this is difficult to calculate. For $ n $ large enough, we can use
\[ D=\left| \frac{Y-n\theta}{\sqrt{n\theta(1-\theta)}} \right| \]
as a possible test statistic.
