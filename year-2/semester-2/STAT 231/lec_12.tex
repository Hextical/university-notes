\section{2020-01-31}
\underline{Roadmap}:
\begin{itemize}
    \item 5 min recap
    \item Likelihood function for multinomial
    \item Testing for the model
          \subitem Observed vs Expected frequencies
    \item Likelihood function and the MLE for the uniform distribution
\end{itemize}

\begin{exbox}
    \begin{example}
        The MLE of $ \theta $ for
        \[ f(y;\theta)=\frac{1}{\theta}e^{-y/\theta}  \]
        is $ \hat{\theta}=\overline{y} $. Find the corresponding MLE for
        $ \lambda $ for
        \[ f(y;\lambda)=\lambda e^{-\lambda y}. \]

        \textbf{Solution.} Since $ \lambda=\frac{1}{\theta} $, we have
        \[ \hat{\theta}=\overline{y}\implies \frac{1}{\lambda}=\overline{y} \]
        by the invariance property.
        Thus, the MLE for $ \lambda $ is
        \[ \hat{\lambda}=\frac{1}{\overline{y}}. \]
    \end{example}
\end{exbox}

\begin{exbox}
    \begin{example}
        Suppose $ 4 $ people $ (A,B,C,D) $ run a 100 meter race every week.
        Let $ \theta_i $ be the probability person $ i $ wins a race for $ i\in\left\{ A,B,C,D\right\} $.
        Suppose also the following data is given to us.
        \begin{itemize}
            \item $ n=20 $
            \item $ y_A=8 $
            \item $ y_B=6 $
            \item $ y_C=4 $
            \item $ y_D=2 $
        \end{itemize}
        \underline{Model}: $ Y \thicksim \mult(n,\theta_A,\ldots ,\theta_D) $

        \textbf{Questions}:
        \begin{enumerate}[(a)]
            \item What is the likelihood function?
            \item What are the MLEs?
        \end{enumerate}
        The likelihood function is given by
        \[ L(\theta_A,\ldots ,\theta_D)=\frac{20!}{8!6!4!2!} \theta_A^8\theta_B^6\theta_C^4\theta_D^2 \]
        Intuitively, the MLEs are given by
        \begin{itemize}
            \item $ \hat{\theta}_A=\frac{8}{20} $
            \item $ \hat{\theta}_B=\frac{6}{20} $
            \item $ \hat{\theta}_C=\frac{4}{20} $
            \item $ \hat{\theta}_D=\frac{2}{20} $
        \end{itemize}
    \end{example}
\end{exbox}

The Multinomial joint probability function is
\[ f(y_1,\ldots ,y_k;\bm{\theta})=\frac{n!}{y_1!\cdots y_k!}\prod_{i=1}^k \theta_i^{y_i} \]
for $ y_i=0,1,\ldots $ where $ \sum\limits_{i=1}^{k} y_i = n $.
The likelihood function for $ \bm{\theta}=\bm(\theta_1,\ldots ,\theta_k) $ based on data
$ y_1,\ldots ,y_k $ is given by
\[ L(\bm{\theta})=L(\theta_1,\ldots ,\theta_k)=\frac{n!}{y_1!\cdots y_k!} \prod_{i=1}^k
    \theta_i^{y_i} \]
or more simply
\[ L(\bm{\theta})=\prod_{i=1}^k \theta_i^{y_i} \]
The log likelihood is
\[ \ell(\bm{\theta})=\sum\limits_{i=1}^{k} \left[ y_i\ln(\theta_i) \right] \]
If $ y_i $ represents the number of times outcome $ i $ occurred in the $ n $ ``trials''
for $ i=1,\ldots ,k $, then the following result holds.
\begin{thmbox}
    \begin{prop}
        Suppose $ Y \thicksim \mult(n,\theta_1,\ldots ,\theta_k) $, then the MLE for
        $ \bm{\theta}=(\theta_1,\ldots ,\theta_k) $ is
        \[ \hat{\theta}_i=\frac{y_i}{n} \]
        for $ i=1,\ldots ,k $.
    \end{prop}
\end{thmbox}
\begin{proof}
    Use Lagrange multiplier method for $ \ell(\bm{\theta}) $ satisfying the linear
    constraint $ \sum\limits_{i=1}^{k} \theta_i=1 $.
\end{proof}

\begin{exbox}
    \begin{example}
        Let $ Y $ be a discrete random variable taking values in $ \{0,1,2,3\} $ and
        \[ P(Y=0)=\theta^3,\; P(Y=1)=3\theta(1-\theta)^2,\;P(Y=2)=3\theta^2(1-\theta),\;
            P(Y=3)=(1-\theta)^3 \]
        where $ \theta $ is an unknown parameter, with $ 0<\theta<1 $.
        We make a table of $ 80 $ independent observations from the distribution above.
        \[
            \begin{array}{c|c}
                Y & \text{Observed Frequency} \\
                \hline
                0 & 10                        \\
                1 & 30                        \\
                2 & 30                        \\
                3 & 10
            \end{array}
        \]
        (a) Determine the likelihood function, $ L(\theta) $.

        \textbf{Solution.}
        \begin{align*}
            L(\theta)
             & =\left( \theta^3 \right)^{10}\left[ 3\theta(1-\theta)^2 \right]^{30}
            \left[ 3\theta^2(1-\theta) \right]^{30}\left[ (1-\theta)^3 \right]^{10}                        \\
             & =3^{30}3^{30}\theta^{30}\theta^{30}\theta^{60}(1-\theta)^{60}(1-\theta)^{30}(1-\theta)^{30} \\
             & =3^{30}3^{30}\theta^{120}(1-\theta)^{120}
        \end{align*}
        or more simply
        \[ L(\theta)=\theta^{120}(1-\theta)^{120} \]

        (b) Determine the log likelihood function, $ \ell(\theta) $.

        \textbf{Solution.}
        \[ \ell(\theta)=120\ln(\theta)+120\ln(1-\theta) \]
        or more simply
        \[ \ell(\theta)=\ln(\theta)+\ln(1-\theta) \]

        (c) Using the function $ \ell(\theta) $ in (b) in order to derive the maximum
        likelihood estimate of $ \theta $.

        \textbf{Solution.}
        \[ \frac{d\ell}{d\theta}=\frac{1}{\theta}-\frac{1}{1-\theta}=\frac{1-2\theta}{\theta(1-\theta)}:=0 \]
        \[ \implies \hat{\theta}=\frac{1}{2}=0.5 \]
    \end{example}
\end{exbox}

\begin{exbox}
    \begin{example}[Using the likelihood functions to test models]
        Suppose $ W_1,\ldots ,W_n $ are iid. We collect data $ \bm{w}=(w_1,\ldots ,w_n) $.

        \underline{Model}: $ W_i \thicksim \poi(\theta) $
        \[
            \begin{array}{c|c|c|}
                W           & \text{Observed Frequency} & \text{Expected Frequency} \\
                \hline
                0           & y_0                       & e_1                       \\
                1           & y_1                       & e_2                       \\
                2           & y_2                       & e_3                       \\
                3           & y_3                       & e_4                       \\
                4           & y_4                       & e_5                       \\
                \geqslant 5 & y_5                       & e_6
            \end{array}
        \]
        To calculate the expected $ e_i $'s we use the formula
        \[ e_i=n\cdot p_i \]
        where
        \[ p_i=P(Y=i). \]
        for $ i\in[0,4] $ where $ n $ is the total number of observations (observed frequencies summed).
        For example, $ e_i $ would be the following.
        \[ e_i=n\cdot \left( \frac{e^{-\hat{\theta}}\cdot\hat{\theta}^{i}}{i!} \right) \]
        for $ j\in[0,4] $. Note that $ \hat{\theta}=\overline{y} $.
        To estimate $ e_5 $, we write
        \[ e_5=n\cdot P(Y\geqslant 5)=n\cdot \left( 1-\sum\limits_{i=0}^{4}P(Y=i) \right) \]
        Then, we compare the observed frequencies to the expected frequencies.
    \end{example}
\end{exbox}
