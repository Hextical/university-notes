\section{2020-04-01: Large Samples and Paired Data}
\underline{Roadmap}:
\begin{enumerate}[(i)]
    \item Independent population, unequal variance
    \item Paired Data
    \item Housekeeping: \emph{evaluate.uwaterloo.ca}
    \item Recap
\end{enumerate}

The following are equivalent:
\begin{itemize}
    \item $ H_1 $: $ \mu_1=\mu_2 $
    \item Confidence interval: $ \mu_1-\mu_2=0 $
\end{itemize}

\underline{Recap}: \underline{Equal variances}:

\[ Y_{1i} \thicksim N(\mu_1,\sigma^2),\; Y_{2j} \thicksim N(\mu_2,\sigma^2) \]
Pivotal Quantity:
\[ \frac{(\overline{Y}_1-\overline{Y}_2)-(\mu_1-\mu_2)}{S_p \sqrt{\frac{1}{n_1}+\frac{1}{n_2}}}
    \thicksim T_{n_1+n_2-2}\implies (\overline{y}_1+\overline{y}_2)\pm t^* s_p\sqrt{\frac{1}{n_1}+\frac{1}{n_2}}  \]
where
\[ s_p^2=\frac{(n_1-1)s_1^2+(n_2-1)s_2^2}{n_1+n_2-2}  \]
Test statistic is the absolute value of above.

Unequal variances, large samples, independent population

\[ Y_{1i}\thicksim N(\mu,\sigma_1^2),\; Y_{2j} \thicksim N(\mu_2,\sigma_2^2) \]
where $ i=1,\ldots ,n_1 $ and $ j=1,\ldots ,n_2 $.
\begin{thmbox}
    \begin{theorem}
        If $ n_1 $ and $ n_2 $ are large, then
        \[ \frac{(\overline{Y}_1-\overline{Y}_2)-(\mu_1-\mu_2)}{\sqrt{\frac{S_1^2}{n_1}+\frac{S_2^2}{n_2}}}
            \thicksim Z  \]
    \end{theorem}
\end{thmbox}
The $ 95\% $ confidence interval; that is, we solve $ P(-1.96\leqslant Z \leqslant 1.96)=0.95 $
where $ Z $ is defined as in the theorem is:
\[ (\overline{y}_1-\overline{y}_2)\pm z^*\sqrt{\frac{s_1^2}{n_1}+\frac{s_2^2}{n_2}} \]
where $ z^*=1.96 $. To test $ H_0 $: $ \mu_1=\mu_2 $, check if $ 0 $ is within the interval.

\begin{exbox}
    \begin{example}\;
        \begin{itemize}
            \item $ n_1=278 $
            \item $ n_2=345 $
            \item $ \overline{y}_1=60.2 $
            \item $ \overline{y}_2=58.1 $
            \item $ s_1=10.16 $
            \item $ s_2=9.02 $
        \end{itemize}
        Find the $ 95\% $ confidence interval for $ \mu_1-\mu_2 $.

        \textbf{Solution.}
        \[ (\overline{y}_1-\overline{y}_2)\pm z^*\sqrt{\frac{s_1^2}{n_1}+\frac{s_2^2}{n_2}} \]
        yields
        \[ \left[ 0.57,3.63 \right] \]
        Suppose we are given $ H_0 $: $ \mu_1=\mu_2 \iff \mu_1-\mu_2=0 $ at $ 5\% $, is this reasonable? No,
        since $ 0 $ is not within the interval above $ \implies $ $ p $-value $ <0.05 $.
    \end{example}
\end{exbox}

\underline{Paired Data}: Natural $ 1-1 $ map between the units of the population.
\begin{enumerate}[(i)]
    \item \underline{Examples}
    \item \underline{Idea of Pivotal Quantity}
    \item \underline{Example}
\end{enumerate}
(i)
\begin{itemize}
    \item Before and after
    \item Same car, same driver, number of miles travelled between fuel A and fuel B (not independent)
\end{itemize}
\[ \begin{pmatrix}
        b_1 \\
        a_1
    \end{pmatrix},\ldots,
    \begin{pmatrix}
        b_n \\
        a_n
    \end{pmatrix} \]
where each $ b_i $ are before data and each $ a_i $ are after data.
\[ B_i \thicksim N(\mu_1,\sigma_1^2) \]
\[ A_i \thicksim N(\mu_2,\sigma_2^2) \]
these are pairs, so let's subtract them
\[ (B_i-A_i)=Y_i \thicksim N(\mu_1-\mu_2,\sigma^2) \]
for some $ \sigma^2 $ (there will be covariance within there).
We are testing $ H_0 $: $ \mu=0 $. Population of differences ($ B_i $'s vs $ A_i $'s)

\begin{exbox}
    \begin{example} See Table 6.3 in the course notes for the data.
        Step 1: Construct $ y_i=b_i-a_i $ for each $ i\in[1,n] $.
        \[ Y_i \thicksim N(\mu,\sigma^2) \]
        and test $ H_0 $: $ \mu=0 $.
        \begin{itemize}
            \item $ \overline{y}=-0.020 $
            \item $ s=0.411 $
            \item $ d=\frac{\overline{y}}{\nicefrac{s}{\sqrt{n}}}\thicksim T_{n-1} $ where $ n-1=19 $
            \item Confidence interval: $ \left[ -0.212,0.172 \right] $
                  \subitem $ \overline{y}+t^* \nicefrac{s}{\sqrt{n}} $, $ t^*= $ column 19, row 0.975.
        \end{itemize}
        $ 0 $ falls within the confidence interval, so the $ p $-value is less than $ 5\% $.
    \end{example}
\end{exbox}

\underline{Final points}
\begin{enumerate}[(i)]
    \item Case I: Equal variance, independent samples
    \item Case II: Unequal variance, independent samples, large sample sizes
    \item Case III: Paired data
\end{enumerate}
We ignored one case: small sample sizes, unequal variances (we don't worry about it in this course).

Typically, in paired data the two variables are not independent, but positively correlated,
however the variance is $ \sigma_1^2+\sigma_2^2-2\text{Cov}(b_i,a_i) $
where $ \text{Cov}(b_i,a_i)>0 $ if the variance is lower, the variances are more accurate.
We should always go for the paired method iff the covariance is positively correlated.
