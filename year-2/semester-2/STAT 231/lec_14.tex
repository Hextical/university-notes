\section{2020-02-05}
\underline{Roadmap}:
\begin{itemize}
    \item Two examples
          \begin{itemize}
              \item Likelihood and the MLE for $ \uniform{0,\theta} $
              \item Discrete example
          \end{itemize}
    \item PPDAC
          \begin{itemize}
              \item Example and definitions
          \end{itemize}
\end{itemize}

\begin{Example}{}{}
    $ Y_1,\ldots ,Y_n $ are iid random variables with $ \uniform{0,\theta} $
    where $ \theta= $ unknown parameter (attribute) of interest.
    \begin{itemize}
        \item Data set: $ (y_1,\ldots ,y_n) $ where $ y_i>0 $ for each $ i\in[1,n] $
    \end{itemize}
    What is the MLE for $ \theta $.

    \textbf{Solution.}
    \[ f(y_i;\theta)=\text{density function} \]
    \[ f(y_i;\theta)=
        \begin{cases}
            \frac{1}{\theta} & 0\leqslant y_i \leqslant \theta\quad\forall i\in[1,n] \\
            0                & \text{otherwise}
        \end{cases} \]
    Therefore, the likelihood function is
    \[ \mathcal{L}(\theta)=
        \begin{cases}
            \frac{1}{\theta^n} & 0\leqslant y_i\leqslant \theta\quad\forall i\in[1,n] \\
            0                  & \text{otherwise}
        \end{cases} \]
    Note that $ 0\leqslant y_i\leqslant \theta\quad\forall i\in[1,n]\iff
        \theta>\max\{y_1,\ldots ,y_n\} $, thus
    \[ \mathcal{L}(\theta)=
        \begin{cases}
            \frac{1}{\theta^n} & \theta>\max \{y_1,\ldots ,y_n\} \\
            0                  & \text{otherwise}
        \end{cases} \]
    Thus, the MLE is
    \[ \hat{\theta}=\max(y_1,\ldots ,y_n) \]
\end{Example}



\begin{Example}{}{}
    \begin{itemize}
        \item Students come out of a classroom with equal probability
        \item There are $ N $ students in the class identified as $ \{1,\ldots ,N\} $, where
              $ N $ is unknown
        \item We observe $ 3 $ students come out ($ 1,2,7 $)
    \end{itemize}
    What is $ \hat{N} $ given your data?

    \textbf{Solution.}
    \[ \mathcal{L}(N;(1,2,7))=
        \begin{cases}
            0            & N<7          \\
            \binom{N}{3} & N\geqslant 7
        \end{cases} \]
    Given this likelihood,
    \[ \hat{N}=7 \]
    can be thought of as a discrete version of TODO
\end{Example}


\underline{PPDAC}
A step-by-step, algorithmic approach to a statistical question.
\begin{itemize}
    \item P\@: Problem
    \item P\@: Plan
    \item D\@: Data
    \item A\@: Analysis
    \item C\@: Conclusion
\end{itemize}

\begin{Example}{}{}
    We are interested in the attitude of Canadian residents to climate change
    (whether or not climate change is the number one issue facing the world).

    The area of Kitchener-Waterloo and Wellington County were selected
    and $ 200 $ people were randomly selected and interviewed.

    $ 126 $ of them agreed that climate change is the number one issue.
\end{Example}

\underline{Problem}
\begin{itemize}
    \item What question are we trying to answer?
    \item Types of problems:
          \begin{itemize}
              \item Descriptive: Estimating attributes of the population
              \item Causative: Check whether there is a relationship between $ x $ and $ y $
              \item Predictive: Predicting (forecasting) future values of a variate
          \end{itemize}
    \item Target population: The population of interest
          \begin{itemize}
              \item All Canadian residents
          \end{itemize}
    \item Variate: The property of the unit of the population we are interested in
          \[ y_i=
              \begin{cases}
                  0 & \text{climate change is not the number one issue} \\
                  1 & \text{otherwise}
              \end{cases} \]
    \item Attribute: A function of the variate
          \begin{itemize}
              \item $ \theta= $ proportion of Canadians who believe climate change is the number one issue
          \end{itemize}
\end{itemize}
\underline{Plan}
\begin{itemize}
    \item Study population: The population from which the sample is drawn
          \begin{itemize}
              \item The study population is \emph{usually} a subset of the target population, but
                    \textbf{does not} have to be, e.g.\ medical tests on mice.
          \end{itemize}
\end{itemize}
