\section{2020-02-05}
\underline{Roadmap}:
\begin{itemize}
    \item Two examples
          \subitem Likelihood and the MLE for $ \uniform[0,\theta] $
          \subitem Discrete example
    \item PPDAC
          \subitem Example and definitions
\end{itemize}
\begin{exbox}
    \begin{example}\label{uniform mle}
        $ Y_1,\ldots ,Y_n $ are iid random variables with $ \uniform[0,\theta] $
        where $ \theta= $ unknown parameter (attribute) of interest.
        \begin{itemize}
            \item Data set: $ (y_1,\ldots ,y_n) $ where $ y_i>0 $ for each $ i\in[1,n] $
        \end{itemize}
        What is the MLE for $ \theta $.

        \textbf{Solution.}
        \[ f(y_i;\theta)=\text{density function} \]
        \[ f(y_i;\theta)=
            \begin{cases}
                \frac{1}{\theta} & 0\leqslant y_i \leqslant \theta\quad\forall i\in[1,n] \\
                0                & \text{otherwise}
            \end{cases} \]
        Therefore, the likelihood function is
        \[ L(\theta)=
            \begin{cases}
                \frac{1}{\theta^n} & 0\leqslant y_i\leqslant \theta\quad\forall i\in[1,n] \\
                0                  & \text{otherwise}
            \end{cases} \]
        Note that $ 0\leqslant y_i\leqslant \theta\quad\forall i\in[1,n]\iff
            \theta>\max\{y_1,\ldots ,y_n\} $, thus
        \[ L(\theta)=
            \begin{cases}
                \frac{1}{\theta^n} & \theta>\max \{y_1,\ldots ,y_n\} \\
                0                  & \text{otherwise}
            \end{cases} \]
        Thus, the MLE is
        \[ \hat{\theta}=\max(y_1,\ldots ,y_n) \]
    \end{example}
\end{exbox}

\begin{exbox}
    \begin{example} $ \; $
        \begin{itemize}
            \item Students come out of a classroom with equal probability
            \item There are $ N $ students in the class identified as $ \{1,\ldots ,N\} $, where
                  $ N $ is unknown
            \item We observe $ 3 $ students come out ($ 1,2,7 $)
        \end{itemize}
        What is $ \hat{N} $ given your data?

        \textbf{Solution.}
        \[ L(N;(1,2,7))=
            \begin{cases}
                0            & N<7          \\
                \binom{N}{3} & N\geqslant 7
            \end{cases} \]
        Given this likelihood,
        \[ \hat{N}=7 \]
        can be thought of as a discrete version of \ref{uniform mle}.
    \end{example}
\end{exbox}

\underline{PPDAC}
A step-by-step, algorithmic approach to a statistical question.
\begin{itemize}
    \item P: Problem
    \item P: Plan
    \item D: Data
    \item A: Analysis
    \item C: Conclusion
\end{itemize}
\begin{exbox}
    \begin{example}
        We are interested in the attitude of Canadian residents to climate change
        (whether or not climate change is the number one issue facing the world).

        The area of Kitchener-Waterloo and Wellington County were selected
        and $ 200 $ people were randomly selected and interviewed.

        $ 126 $ of them agreed that climate change is the number one issue.
    \end{example}
\end{exbox}
\underline{Problem}
\begin{itemize}
    \item What question are we trying to answer?
    \item Types of problems:
          \subitem Descriptive: Estimating attributes of the population
          \subitem Causative: Check whether there is a relationship between $ x $ and $ y $
          \subitem Predictive: Predicting (forecasting) future values of a variate
    \item Target population: The population of interest
          \subitem All Canadian residents
    \item Variate: The property of the unit of the population we are interested in
          \subitem \[ y_i=
              \begin{cases}
                  0 & \text{climate change is not the number one issue} \\
                  1 & \text{otherwise}
              \end{cases} \]
    \item Attribute: A function of the variate
          \subitem $ \theta= $ proportion of Canadians who believe climate change is the number one issue
\end{itemize}
\underline{Plan}
\begin{itemize}
    \item Study population: The population from which the sample is drawn
          \subitem The study population is \emph{usually} a subset of the target population, but
          \textbf{does not} have to be, e.g. medical tests on mice.
\end{itemize}
