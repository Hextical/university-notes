\section{2020-02-12}
\underline{Roadmap}:
\begin{itemize}
    \item 5 min recap
    \item Recap of STAT 230
          \subitem (Strong) Law of Large \#'s
          \subitem CLT
    \item Confidence Interval for the Normal problem with known variance
\end{itemize}
\[ Y_i \thicksim f(y_i;\theta) \]
$ i=1,\ldots ,n $ and $ Y_i $'s independent with $ \theta= $ unknown parameter.

\underline{Likelihood Interval}
A $ 10\% $ likelihood interval:
\[ \{\theta:R(\theta)\geqslant 0.1\} \]
\underline{Notes}
\begin{enumerate}[(i)]
    \item The MLE $ \theta $ is in every likelihood interval for all $ p\in[0,1] $
    \item Suppose $ \theta $ belongs to the $ 100p\% $ likelihood interval, then
          $ \theta $ belongs to the $ 100q\% $ likelihood interval, where $ q<p $.
    \item As $ n $ becomes large, the intervals become narrower, for given $ p $.
    \item Plausibility
          \subitem $ R(\theta)\geqslant 0.5\implies $ very plausible
          \subitem $ \vdots $
          \subitem $ R(\theta)<0.01\implies $ very implausible
    \item $ \{\theta:R(\theta)\geqslant p\}\iff \{\theta:r(\theta)\geqslant \ln(p)\} $,
          where $ r(\theta) = $ log relative likelihood function
\end{enumerate}
\underline{Confidence Interval}
\begin{exbox}
    \begin{example}
        The STAT 231 scores are $ N(\mu,64) $. A sample of $ 25 $ students are
        taken
        \begin{itemize}
            \item $ \bar{y}=75 $
            \item $ s^2=81 $
        \end{itemize}
        Given this data, find the $ 95\% $ confidence interval for $ \mu $.
    \end{example}
\end{exbox}
\underline{Central Limit Theorem}

\underline{Law of Large Numbers}: $ Y_1,\ldots ,Y_n $ are iid random
variables with mean $ \mu $ and variance $ \sigma^2 $.
\[ \bar{Y}_n=\frac{1}{n} \sum\limits_{i=1}^{n} Y_i \]
Then, $ \bar{Y}_n\to \mu $ as $ n\to \infty $.

\underline{CLT}: If $ Y_1,\ldots ,Y_n $ are iid random variables
with mean $ \mu $ and variance $ \sigma^2 $, and
\[ S_n=\sum\limits_{i=1}^{n} Y_i \]
\[ \bar{Y}_n=\frac{1}{n} \sum\limits_{i=1}^{n} Y_i \]
Then, $ S_n \thicksim N(n\mu,n\sigma^2) $ and $ \bar{Y}_n
    \thicksim N(\mu,\sigma^2/n) $ approximately as $ n\to \infty $.
\begin{exbox}
    \begin{example}
        $ Y_1,\ldots ,Y_n \thicksim \exponential(100) $ with $ n=50 $.
        \[ P\left(\bar{Y}>102\right) \]
        \[ \bar{Y} \thicksim N(100,100^2/50) \]
    \end{example}
\end{exbox}
\begin{exbox}
    \begin{example}
        $ Y \thicksim \bin(n,\theta) $. If $ n $ is large, then
        \[ Y \thicksim N(n\theta,n\theta(1-\theta)) \]
        where $ Y=Y_1+\cdots+Y_n $ where $ Y_i \thicksim \text{Bernoulli}(p) $.
    \end{example}
\end{exbox}
\begin{exbox}
    \begin{example}
        For any iid Normal variables, the result is true for any $ n $ (not just large).
        $ Y_1,\ldots ,Y_n $ iid $ N(\mu,\sigma^2) $, then $ S_n \thicksim N(n\mu,n\sigma^2) $
        and $ \bar{Y}_n \thicksim N(\mu,\sigma^2/n) $ for all $ n $.
    \end{example}
\end{exbox}
\underline{Back to the Confidence Interval problem}:

\underline{Steps}

\underline{Step 1}: Identify the sampling distribution of your estimator.

\underline{Step 2}: Construct the Pivotal Quantity.

\underline{Step 3}: Use the pivot to construct the coverage interval.

\underline{Step 4}: Estimate this interval using your data (confidence interval).

\begin{exbox}
    \begin{example}
        $ Y_1,\ldots ,Y_n \thicksim N(\mu,64) $ with
        \begin{itemize}
            \item $ n=25 $
            \item $ \bar{y}=75 $
            \item $ s^2=81 $
        \end{itemize}
        \underline{Objective}: To construct a $ 95\% $ confidence interval.

        \underline{Step 1}: $ \hat{\mu}=\bar{y}=75 $, then
        \[ \bar{Y} \thicksim N(\mu,64/25) \]
        where $ \bar{Y} $ is the sampling distribution of the sample mean.

        \underline{Step 2}: The pivotal quantity is given by
        \[ \frac{\bar{Y}-\mu}{8/5} =Z \thicksim N(0,1) \]

        \underline{Step 3}:
        \[ P\left(-1.96\leqslant Z\leqslant 1.96\right)=0.95 \]
        \[ \implies P\left( \bar{Y}-1.96\times \frac{8}{5} \leqslant \mu
            \leqslant \bar{Y}+1.96\times \frac{8}{5} \right)=0.95 \]

        \underline{Step 4}: The confidence interval is:
        \[ \left[ \bar{y}-1.96\times \frac{8}{5},\;\bar{y}+1.96\times \frac{8}{5} \right] \]
    \end{example}
\end{exbox}

\underline{Clicker Question}: The sample population is always a subset of the target
population.
\begin{enumerate}[(a)]
    \item True
    \item \textbf{False}
\end{enumerate}
