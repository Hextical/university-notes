\section{2020-02-26}
\underline{Roadmap}:
\begin{itemize}
    \item Recap
    \item Confidence interval for the Binomial problem
    \item How to choose the ``right'' sample size?
\end{itemize}
\underline{Confidence Intervals}
\begin{itemize}
    \item $ \theta= $ unknown parameter
    \item $ Y_i \sim f(y_i;\theta) $ for $ i=1,\ldots ,n $ with $ Y_i $'s independent
          \begin{itemize}
              \item $ f= $ distribution (density) function
          \end{itemize}
    \item Data set: $ \{y_1,\ldots ,y_n\} $
    \item $ [a,b] $ which is an estimate of the random interval $ [A,B] $
          which contain $ \theta $ with the given probability
\end{itemize}

\underline{Step 1}: Estimate $ \theta \longleftrightarrow \hat{\theta} $ and
find the sampling distribution of $ \tilde{\theta} $.
\begin{itemize}
    \item $ \hat{\theta}= $ estimate
    \item $ \tilde{\theta}= $ estimator
    \item $ Y_1,\ldots ,Y_n \sim N(\mu,\sigma^2) $ where $ \sigma^2 $ is known
    \item $ \hat{\mu}=\bar{y} \longleftrightarrow \hat{\theta} $
    \item $ \bar{Y} \longleftrightarrow \tilde{\theta} $
    \item Sampling distribution: $ \bar{Y}\sim N(\mu,\sigma^2/n) $
\end{itemize}
\underline{Step 2}: Construct the pivotal quantity.
\[ \frac{\bar{Y}-\mu}{\sigma/\sqrt{n}}=Z \]
where $ Z $ is the pivotal distribution.

\underline{Step 3}: Construct the coverage interval. The $ 95\% $ coverage interval is given by
\begin{align*}
     & P\left( -1.96\leqslant Z\leqslant 1.96 \right)=0.95                                                                 \\
     & P\left( -1.96\leqslant \frac{\bar{Y}-\mu}{\frac{\sigma}{\sqrt{n}}}\leqslant 1.96  \right)=0.95                      \\
     & P\left( \bar{Y}-1.96 \frac{\sigma}{\sqrt{n}}\leqslant \mu\leqslant \bar{Y}+1.96 \frac{\sigma}{\sqrt{n}}\right)=0.95
\end{align*}
\underline{Step 4}: Construct the confidence interval.
\[ \left[ \bar{y}-1.96 \frac{\sigma}{\sqrt{n}} ,\; \bar{y}+1.96 \frac{\sigma}{\sqrt{n}}  \right] \]
For the normal problem with $ \sigma= $ known, the confidence interval is given by
\[ \bar{y}\pm z^* \frac{\sigma}{\sqrt{n}}  \]
\begin{exbox}
    \begin{example}[Binomial Distribution]
        In the 2020 US election, CNN does an exit poll in Wisconsin of 1200 voters.
        \begin{itemize}
            \item $ 56\% $ voted for Trump
            \item $ 44\% $ voted for Bernie Sanders
        \end{itemize}
        Find the $ 95\% $ confidence interval for $ \theta= $ proportion of votes that
        Trump gets.

        \underline{Model}: $ Y \sim \bin(1200,\theta) $ with $ \theta= $ probability of
        voting for Trump.

        \textbf{Solution.}

        \underline{Step 1}: $ \hat{\theta}=y/n=0.56 $, $ \tilde{\theta}=Y/n $
        where $ Y \sim N(n\theta,n\theta(1-\theta)) $
        \[ \frac{Y-n\theta}{\sqrt{n\theta(1-\theta)}}=Z  \]
        \[ \implies \frac{\tilde{\theta}-\theta}{\sqrt{\frac{\theta(1-\theta)}{n}}}=Z  \]
        However, if we use this pivotal quantity separating $ \theta $ could be problematic.
        Thus, using version 2 of CLT we get
        \[ \frac{\tilde{\theta}-\theta}{\sqrt{\frac{\tilde{\theta}(1-\tilde{\theta})}{n}}}=Z \]
        is a better pivotal quantity.
        Thus, the general confidence interval for Binomial is
        \[ \hat{\theta}\pm z^*\sqrt{\frac{\hat{\theta}\left(1-\hat{\theta}\right)}{n} } \]
        \[ \implies 0.56\pm 1.96\sqrt{\frac{0.56\times 0.44}{1200}}\iff \left[ 0.53,0.59 \right] \]
        Even in the worst case scenario, Trump wins (call the election for CNN).

        Note that
        \[ z^*\sqrt{\frac{\hat{\theta}\left(1-\hat{\theta}\right)} {n}} \]
        is called the \textbf{\emph{margin of error}}.

        Suppose we want the margin of error to be $ \leqslant 0.03 $ for a $ 95\% $ interval, then
        \[ z^*\sqrt{\frac{\hat{\theta}\left(1-\hat{\theta}\right)}{n}}\leqslant 0.03\iff
            n\geqslant \left( \frac{z^*}{0.03}  \right)^2\hat{\theta}\left(1-\hat{\theta}\right) \]
        We note that $ \hat{\theta}=0.5 $ is the maximum, so we choose $ n $ such that
        \[ n\geqslant \left( \frac{z^*}{0.03}  \right)^2(0.5)(0.5) \]
        Thus, for the $ 95\% $ confidence interval we get
        \[ n\geqslant \left( \frac{1.96}{0.03} \right)^2(0.5)(0.5)\approx 1048 \]
    \end{example}
\end{exbox}
