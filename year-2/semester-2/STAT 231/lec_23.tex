\section{2020-03-04}
\begin{defbox}
    \begin{definition}
        An estimator $ \tilde{\theta} $ is called \textbf{\emph{unbiased}} for $ \theta $ if
        \[ E(\tilde{\theta})=\theta \]
    \end{definition}
\end{defbox}
\begin{exbox}
    \begin{example}
        Let $ W=\frac{(n-1)S^2}{\sigma^2} $. Prove $ S^2 $ is an unbiased
        estimator for $ \sigma^2 $.

        \textbf{Solution.}
        \begin{align*}
             & E(W)=n-1                                              \\
             & \implies E\left(\frac{(n-1)S^2}{\sigma^2} \right)=n-1 \\
             & \implies \frac{n-1}{\sigma^2} E(S^2)=n-1              \\
             & \implies E(S^2)=\sigma^2
        \end{align*}
        Thus, $ S^2 $ is an unbiased estimator for $ \sigma^2 $ by definition.
    \end{example}
\end{exbox}

\underline{Other Confidence Intervals}

\underline{Poisson}
Suppose $ Y_1,\ldots ,Y_n \sim \poi(\mu) $
are independent and $ n $ is large. Find the $ 95\% $ confidence interval.
\[ \overline{Y} \sim N(\mu,\sigma^2=\sfrac{\mu}{n}) \]
Find the pivotal quantity now.

\underline{Exponential}
Suppose $ Y_1,\ldots ,Y_n \sim \exp(\theta) $ are independent and $ n $ is small.

\begin{thmbox}
    \begin{theorem}
        If $ Y \sim \exponential(\theta) $, then
        \[ \frac{2Y}{\theta} \sim \exponential(2) \]
        If $ W_i=\sfrac{2Y_i}{\theta} $, then
        \[ \sum\limits_{i=1}^{n} W_i \sim \chi^2_{2n} \]
    \end{theorem}
\end{thmbox}
\begin{proof}
    Let $ F_W(w) $ be the cumulative distribution function of $ W $. Then,
    \begin{align*}
        F_W(w)
         & =P(W\leqslant w)                               \\
         & =P\left( \frac{2Y}{\theta} \leqslant w \right) \\
         & =P\left( Y\leqslant \frac{w\theta}{2} \right)  \\
         & =1-e^{-\frac{w\theta/2}{\theta}}               \\
         & =1-e^{-\sfrac{w}{2}}
    \end{align*}
    Therefore,
    \[ f(w)=\frac{1}{2} e^{-\sfrac{w}{2}} \]
\end{proof}
Using this theorem, we can find the confidence interval for $ \theta $.
\begin{align*}
     & P\left(a\leqslant \chi^2_{2n}\leqslant b\right)=0.95                                         \\
     & \implies P\left(a \leqslant \sum\limits_{i=1}^{n} W_i \leqslant b\right)=0.95                \\
     & \implies P\left(a\leqslant \sum\limits_{i=1}^{n} \frac{2Y_i}{\theta} \leqslant b\right)=0.95 \\
     & \implies P\left(a\leqslant \frac{2}{\theta} \sum\limits_{i=1}^{n} Y_i\leqslant b\right)=0.95
\end{align*}
yields
\[ \left[ \frac{2 \sum\limits_{i=1}^{n} Y_i}{b} , \frac{2 \sum\limits_{i=1}^{n} Y_i}{a} \right] \]
where $ a $ and $ b $ are from the $ \chi^2 $ table.

\begin{thmbox}
    \begin{theorem}
        If we have a $ p\% $ coverage interval with $ Z $ as a pivot, and $ n $ is large, then
        the corresponding likelihood is given by
        \[ \exp\left[-\sfrac{(z^*)^2}{2}\right] \]
    \end{theorem}
\end{thmbox}

\begin{exbox}
    \begin{example}
        If $ p=0.95 $ and $ z^*=1.96 $, then the corresponding likelihood is:
        \[ \exp\left[-\sfrac{(1.96)^2}{2}\right] \approx 0.15 \]
    \end{example}
\end{exbox}
