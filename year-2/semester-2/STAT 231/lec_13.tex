\section{2020-02-03}
\underline{Roadmap}:
\begin{itemize}
    \item Review for the midterm
    \item Likelihood and the MLE for Uniform distribution
\end{itemize}
\begin{exbox}
    \begin{example}
        The average number of typos in an academic journal. A random
        sample of $ 100 $ pages are taken. Let $ y_1,\ldots ,y_{100} $
        be the observed data where $ y_i $ is the number of typos
        in page $ i $.
    \end{example}
\end{exbox}
\begin{exbox}
    \begin{example}
        Average score in STAT 231 and whether STAT 231 scores
        are correlated with STAT 230 scores.
        Let $ (x_1,y_1),\ldots ,(x_n,y_n) $ be the observed data
        where
        \begin{itemize}
            \item $ x_i= $ STAT 230 score of the $ i^{\text{th}} $ student
            \item $ y_i= $ STAT 231 score of the $ i^{\text{th}} $ student
        \end{itemize}
    \end{example}
\end{exbox}
\underline{Step 1}: Identify the population, the parameter of interest,
the type of study, variates, attributes (function of the variates), etc.

\underline{Step 2}: Collect data
\begin{itemize}
    \item Observational: None of the variables are controlled
    \item Experimental: Some variables are under the control of the person
          doing the experiment
\end{itemize}
\underline{Types of problems}
\begin{itemize}
    \item Estimation: We are trying to estimate a population attribute
    \item Hypothesis testing: Testing a claim made about the population
    \item Prediction: Predict the ``future'' value of a variate
\end{itemize}

\underline{Step 3}: Summarize data (to identify the model)
\begin{itemize}
    \item Numerical
    \item Graphical
    \item Test whether the model is appropriate
          \subitem Compare the CDF to the ECDF
          \subitem Compare the theoretical properties
          \subitem Compare the observed vs expected frequencies
\end{itemize}

\underline{Step 4}: Do the statistical analysis based on your final model
\begin{itemize}
    \item Parameter: Unknown constant, e.g. $ \theta= $ population mean
    \item Estimate: A number that can be computed from the data set, e.g.
          $ \hat{\theta}= $ (sample mean)
    \item Estimator: The random variable from which $ \hat{\theta} $ is drawn,
          denoted $ \tilde{\theta} $.
\end{itemize}

\underline{Likelihood function}
\[ L(\theta)=\prod_{i=1}^n f(y_i;\theta) \]
where $ f= $ distribution/density function.
\[ \ell(\theta)=\ln\left[ L(\theta) \right] \]
$ \hat{\theta} $ is the MLE of $ \hat{\theta} $ that maximizes $ L(\theta) $

\underline{Measures of Association}
\begin{itemize}
    \item Data set: $ (x_1,y_1),\ldots ,(x_n,y_n) $
          \subitem $ x_i= $ number of bears you drink per week
          \subitem $ y_i= $ STAT 231 score in MT 1
\end{itemize}
If $ x_i>\bar{x} $ and $ y_i<\bar{y} $, then
\[ (x_i-\bar{x})(y_i-\bar{y})<0 \]
\underline{Sample Correlation}
\[ r_{xy}=\frac{\sum\limits_{i=1}^{n} (x_i-\bar{x})(y_i-\bar{y})}{
        \sqrt{\sum\limits_{i=1}^{n} (x_i-\bar{x})^2\sum\limits_{i=1}^{n}
            (y_i-\bar{y})^2}
    }=\frac{S_{xy}}{\sqrt{S_{xx}S_{yy}}}  \]
Note that we always have $ -1\leqslant r_{xy}\leqslant 1 $.
\begin{itemize}
    \item If $ |r_{xy}|\approx 1 $, then there is evidence of a strong linear relationship
    \item If $ |r_{xy}|\approx 0 $, then there is no evidence of a linear relationship
\end{itemize}
Note that
\begin{align*}
    \sum\limits_{i=1}^{n} (x_i-\bar{x})(y_i-\bar{y})
     & =\sum\limits_{i=1}^{n} x_iy_i-n\bar{x}\bar{y} \\
     & =\sum\limits_{i=1}^{n} (x_i-\bar{x})y_i
\end{align*}
\[ \begin{array}{c|c|c}
                          & \text{Rich}              & \text{Poor}              \\
        \hline
        \text{Smoker}     & \underbrace{20}_{n_{11}} & \underbrace{80}_{n_{12}} \\
        \hline
        \text{Non-smoker} & \underbrace{50}_{n_{21}} & \underbrace{50}_{n_{22}} \\
        \hline
    \end{array} \]
\begin{align*}
    \text{Relative Risk}
     & =\frac{\frac{20}{20+80}}{\frac{50}{50+50}}                         \\
     & =\frac{\frac{n_{11}}{n_{11}+n_{12}}}{\frac{n_{21}}{n_{21}+n_{22}}}
\end{align*}
