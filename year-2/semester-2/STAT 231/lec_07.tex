\makeheading{2020-01-20}
\underline{Roadmap}:
\begin{itemize}
    \item Intro
    \item Big picture of STAT 230 and STAT 231
    \item Quiz Recap
\end{itemize}
\begin{exbox}
    \begin{example}[STAT 230]
        A \textbf{fair} die is rolled $ 60 $ times. What is the
        probability that $ 12 $ of them are sixes?
        
        \textbf{Solution.}
        Let $ X= $ the number of successes (sixes), then $ X \sim \bin(60,\sfrac{1}{6}) $.
        \[ P(X=12)
            =\binom{60}{12}\left( \frac{1}{6}  \right)^{12}\left( 1-\frac{1}{6} \right)^{60-12}
            \approx 0.11\]
    \end{example}
\end{exbox}
\begin{exbox}
    \begin{example}[STAT 231]
        A die is rolled $ 60 $ times and $ 12 $ of them were sixes.
        What can we say about the ``fairness'' of the die?
        
        \textbf{Solution.} We will solve this answer later.
    \end{example}
\end{exbox}
\begin{enumerate}
    \item STAT 230: Population $ \rightarrow $ Sample
    \item STAT 231: Sample $ \rightarrow $ Population
\end{enumerate}
Think of STAT 231 as the ``reverse'' of STAT 230.

\underline{Errors are inevitable}

Data collection is extremely important. Why do we summarize data?

\begin{enumerate}[label=(\alph*)]
    \item To identify the ``model''.
    \item To extract important properties.
\end{enumerate}

\section{Data Summaries}

We summarize our data into two categories:
\begin{enumerate}[label=(\arabic*)]
    \item Numerical
    \item Graphical
\end{enumerate}

\subsection{Numerical Summaries}

\begin{itemize}
    \item Location: mean, median, and mode
    \item Variability: variance and standard deviation
    \item Shape: skewness and kurtosis
    \item Quantiles and Percentiles
\end{itemize}

\underline{Location}
\begin{itemize}
    \item Sample mean $ \left( \bar{y} \right) $:
          \[ \bar{y}=\frac{1}{n} \sum_{i=1}^{n} y_{i} \]
    \item Sample median $ \left( \hat{m} \right) $:
          \begin{itemize}
              \item Odd $ n $:
                    \[ \hat{m}=y_{\left(n/2\right)} \]
              \item Even $ n $:
                    \[ \hat{m}=\frac{1}{2} \left( y_{\left(\left\lfloor n/2\right\rfloor\right)} +
                        y_{\left(\left\lceil n/2\right\rceil\right)}\right) \]
          \end{itemize}
    \item Sample mode: value of $ y $ which appears in the sample with the highest frequency
          (not necessarily unique)
\end{itemize}
The sample mean, median and mode describe the ``center'' of the distribution
of variate values in a data set. Since the median is less affected by a few extreme
observations, it is a more robust measure of location.

\underline{Variability}
\begin{itemize}
    \item Sample variance $ \left( s^2 \right) $:
          \[ s^{2}=\frac{1}{n-1} \sum_{i=1}^{n}\left(y_{i}-\bar{y}\right)^{2}
              =\frac{1}{n-1}\left[\sum_{i=1}^{n} y_{i}^{2}-\frac{1}{n}
                  \left(\sum_{i=1}^{n} y_{i}\right)^{2}\right]
              =\frac{\sum_{i=1}^{n} y_{i}^{2}-n \bar{y}^{2}}{n-1} \]
    \item Sample standard deviation $ \left( s \right) $:
          \[ s=\sqrt{s^2} \]
    \item Range:
          \[ \text{Range}=y_{(n)}-y_{(1)} \]
    \item IQR\@:
          \[ \text{IQR}=q_{(0.75)}-q_{(0.25)} \]
\end{itemize}
The sample variance and standard deviation measure the variability or spread of the
variate values in a data set. Since the interquartile range is less affected
by a few extreme observations, it is a more robust measure of variability.

\underline{Shape}
\begin{itemize}
    \item Sample skewness $ (g_1) $:
          \[g_{1}=\frac{\frac{1}{n}
                  \sum_{i=1}^{n}\left(y_{i}-\bar{y}\right)^{3}}{\left[\frac{1}{n} \sum_{i=1}^{n}\left(y_{i}-\bar{y}\right)^{2}\right]^{3 / 2}}\]
    \item Sample kurtosis $ (g_2) $:
          \[g_{2}=\frac{\frac{1}{n} \sum_{i=1}^{n}\left(y_{i}-\bar{y}\right)^{4}}{\left[\frac{1}{n} \sum_{i=1}^{n}\left(y_{i}-\bar{y}\right)^{2}\right]^{2}}\]
\end{itemize}
Measures of shape generally indicate how the data, in terms of a relative frequency
histogram, differ from the Normal bell-shaped curve. The sample skewness is a measure
of (lack of) symmetry in the data. If the relative frequency histogram
of the data has a long right tail, then the sample skewness will be positive.
The sample kurtosis measures the heaviness of the tails and the peakedness of the data
relative to to data that are Normally distributed. If the sample kurtosis is greater than 3,
then this indicates heavier tails (and a more peaked center) than data that are Normally distributed.
For data that arise from a model with no tails, for example the Uniform distribution, the sample
kurtosis will be less than 3.

\begin{exbox}
    \begin{example}
        Suppose we have $ 20 $ observations and the following data is given.
        \begin{itemize}
            \item $ \bar{y}=50 $
            \item $ s^2=5000 $
        \end{itemize}
        Suppose one observation is unreliable, say $ y_i=60 $. Calculate
        the new mean.
        
        \textbf{Solution.}
        \begin{align*}
            \bar{y}_{\text{new}}
             & =\frac{\text{New Total}}{19}          \\
             & =\frac{\text{Old Total $ - 60 $}}{19} \\
             & =\frac{50\times 20-60}{19}            \\
             & =\frac{940}{19}                       \\
             & \approx 49.47
        \end{align*}
    \end{example}
\end{exbox}

\underline{Sample Quantiles and Percentiles}
\begin{defbox}
    \begin{definition}
        Let $ \{y_{(1)},\ldots ,y_{(n)}\} $ where $ y_{(1)}\leqslant \cdots y_{(n)} $
        be the \textbf{\emph{order statistic}} for the data set $ \{y_1,\ldots ,y_n\} $.
        For $ 0<p<1 $, the $ p^{\text{th}} $ sample quantile (also called the $ 100p^{\text{th}} $
        sample percentile), is a value, call it $ q(p) $, determined as follows:
        \begin{itemize}
            \item Let $ m=(n+1)p $ where $ n $ is the sample size.
            \item If $ m $ is an integer and $ 1\leqslant m\leqslant n $, then $ q(p)=y_{(m)} $.
            \item If $ m $ is not an integer, but $ 1<m<n $, then we determine the closest integer $ j $ such
                  that $ j < m < j+1 $ and then $ q(p)=\frac{1}{2} \left[ y_{(j)}+y_{(j+1)} \right] $.
        \end{itemize}
    \end{definition}
\end{defbox}

\begin{defbox}
    \begin{definition}
        The quantiles $ q(0.25) $, $ q(0.5) $ and $ q(0.75) $ are called the
        \textbf{\emph{lower (first) quartile}}, the \textbf{\emph{median}},
        and the \textbf{\emph{upper (third) quartile}} respectively.
    \end{definition}
\end{defbox}

\begin{defbox}
    \begin{definition}
        The \textbf{\emph{interquartile range}} is $ \text{IQR}=q(0.75)-q(0.25) $.
    \end{definition}
\end{defbox}

\begin{defbox}
    \begin{definition}
        The \textbf{\emph{five number summary}} of a data set consist of the smallest observation,
        the lower quartile, the median, the upper quartile, and the largest value, that is,
        the five values: $ y_{(1)},\;q(0.25),\;q(0.5)=\hat{m},\;q(0.75),\;y_{(n)} $.
    \end{definition}
\end{defbox}

\subsection{Graphical Summaries}

\begin{itemize}
    \item Frequency histogram
    \item Empirical cumulative distribution function
    \item Box plots
    \item Scatter plot
    \item Run charts
\end{itemize}
\begin{defbox}
    \begin{definition}
        For a data set $ \{y_1,\ldots ,y_n\} $, the \textbf{\emph{empirical cumulative distribution function}}
        (e.c.d.f) is defined by
        \[ \hat{F}(y)=\frac{\text{number of values in $ \{y_1,y_2,\ldots ,y_n\} $ which are $ \leqslant y $}}{n} \]
        for all $ y\in\mathbb{R} $.
        The e.c.d.f is an estimate, based on the data, of the population cumulative distribution function.
    \end{definition}
\end{defbox}
