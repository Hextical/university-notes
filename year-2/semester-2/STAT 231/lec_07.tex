\section{2020-01-20}
\underline{Roadmap}:
\begin{itemize}
    \item Intro
    \item Big picture of STAT 230 and STAT 231
    \item Quiz Recap
\end{itemize}
\begin{exbox}
    \begin{example}[STAT 230]
        A fair die is rolled $ 60 $ times. What is the
        probability that $ 12 $ of them are sixes?
        Let $ X= $ the number of successes, thus $ X \thicksim \bin(60,\nicefrac{1}{6}) $.
        Then, we want $ P(X=12) $.
    \end{example}
\end{exbox}
\begin{exbox}
    \begin{example}[STAT 231]
        A die is rolled $ 60 $ times, $ 12 $ of them were sixes.
        What can we say about the ``fairness'' of the die?
    \end{example}
\end{exbox}
\begin{enumerate}
    \item STAT 230: Population $ \rightarrow $ Sample
    \item STAT 231: Sample $ \rightarrow $ Population
\end{enumerate}
Think of STAT 231 as the ``reverse'' of STAT 230.

\underline{Errors are inevitable} Data collection
is extremely important. Why do we summarize data?
\begin{enumerate}[(a)]
    \item To identify the ``model''.
    \item To extract important properties.
\end{enumerate}
How can we summarize data? There are two categories
\begin{enumerate}[(1)]
    \item Numerical: Discrete ``count'' \& Continuous ``measure''
    \item Categorical ``ordinal'': Underlying order
\end{enumerate}
Summary
\begin{enumerate}[(a)]
    \item Numerical
    \item Graphical
\end{enumerate}
\underline{Numerical}
\begin{itemize}
    \item Location: mean, median, and mode
    \item Variability: variance and standard deviation
    \item Skewness: right-tailed or left-tailed
    \item Kurtosis: how frequent extreme observations are
\end{itemize}
\underline{Location}

\begin{itemize}
    \item Mean
          \[ \overline{y}=\frac{1}{n} \sum_{i=1}^{n} y_{i} \]

\end{itemize}

\underline{Variability}
\begin{itemize}
    \item Variance
          \[ s^{2}=\frac{1}{n-1} \sum_{i=1}^{n}\left(y_{i}-\overline{y}\right)^{2}=\frac{1}{n-1}\left[\sum_{i=1}^{n} y_{i}^{2}-\frac{1}{n}
                  \left(\sum_{i=1}^{n} y_{i}\right)^{2}\right]=\frac{\sum_{i=1}^{n} y_{i}^{2}-n \overline{y}^{2}}{n-1} \]
    \item Standard deviation
          \[ s=\sqrt{s^2} \]
\end{itemize}
\begin{exbox}
    \begin{example}
        Suppose we have $ 20 $ observations and the following data is given.
        \begin{itemize}
            \item $ \overline{y}=50 $
            \item $ s^2=5000 $
        \end{itemize}
        Suppose one observation is unreliable, say $ y_i=60 $. Calculate
        the new mean.

        \textbf{Solution.}
        \begin{align*}
            \overline{y}_{\text{new}}
             & =\frac{\text{New Total}}{19}          \\
             & =\frac{\text{Old Total $ - 60 $}}{19} \\
             & =\frac{50\times 20-60}{19}            \\
             & =\frac{940}{19}                       \\
             & \approx 49.47
        \end{align*}
    \end{example}
\end{exbox}
\underline{5 Number Summary}
Let $ \{y_{(1)},\ldots ,y_{(n)}\} $ be the sorted data set of $ \{y_1,\ldots ,y_n\} $
where $ y_{(1)} $ is the smallest number, and $ y_{(n)} $ is the largest number.
\begin{enumerate}[(1)]
    \item $ \min $
    \item $ q(0.25) $
    \item $ q(0.5) $
    \item $ q(0.75) $
    \item $ \max $
\end{enumerate}
You can use the rule below to determine the location of $ q(p) $ in the sorted list
\[ m=(n+1)p \]
\begin{itemize}
    \item If $ m $ is an integer and $ 1\leqslant m\leqslant n $, then $ q(p)=y_{(m)} $.
    \item If $ m $ is not an integer, but $ 1<m<n $, then we determine the closest integer $ j $ such
          that $ j < m < j+1 $ and then $ q(p)=\frac{1}{2} \left( y_{(j)}+y_{(j+1)} \right) $.
\end{itemize}

\underline{Graphical}
\begin{itemize}
    \item Histogram
    \item Empirical CDF
    \item Box Plot
\end{itemize}
The empirical cumulative distribution function is
\[ F(y)=\frac{\text{number of values in $ \{y_1,y_2,\ldots ,y_n\} $ which are $ \leqslant y $}}{n} \]
