\section{2020-03-16: Testing for Variances}
\underline{Roadmap}:
\begin{enumerate}[label=(\roman*)]
    \item General info
    \item Testing for variance for Normal
    \item An example
\end{enumerate}
The general problem:
\begin{itemize}
    \item $ Y_1,\ldots ,Y_n \sim N(\mu,\sigma^2)$ iid
          where $ \mu $ and $ \sigma $ are both unknown.
    \item Sample: $ \{y_1,\ldots ,y_n\} $
    \item $ H_0 $: $ \sigma^2=\sigma_0^2 $ vs two sided alternative.
\end{itemize}

\begin{enumerate}[label=(\roman*)]
    \item Test statistic? Problem
    \item Convention?
\end{enumerate}
The pivot is:
\[ U=\frac{(n-1)S^2}{\sigma_0^2} \sim \chi^2_{n-1} \]
can we use this as our test statistic? We will calculate
\[ u=\frac{(n-1)s^2}{\sigma_0^2} \]
We want to compare $ u $ to the median of $ \chi^2_{n-1} $:
\begin{itemize}
    \item If $ u> $ median, then $ p\text{-value}=2P(U\geqslant u) $.
    \item If $ u< $ median, then $ p\text{-value}=2P(U\leqslant u) $.
\end{itemize}

\begin{exbox}
    \begin{example}\label{small u variance} $ \; $
        \begin{itemize}
            \item Normal population: $ \{y_1,\ldots ,y_n\} $
            \item $ n=20 $
            \item $ \sum\limits_{i=1}^{n} y_i=888.1 $
            \item $ \sum\limits_{i=1}^{n} y_i^2=39545.03 $
            \item $ H_0 $: $ \sigma=\sigma_0=2 \iff \sigma^2=\sigma_0^2=4 $
            \item $ H_1 $: $ \sigma\neq \sigma_0=2 \iff \sigma^2\neq \sigma_0^2=4 $
        \end{itemize}
        What is the $ p $-value?
        We know
        \[ s^2=\frac{1}{n-1} \left[ \sum\limits_{i=1}^{n} y_i^2-n\bar{y}^2 \right]
            =
            \frac{1}{19} \left[ (39545.03)-(20)\left( \frac{888.1}{20}  \right)  \right]
            =
            5.7342 \]
        Compute $ u $:
        \[ u=\frac{(n-1)s^2}{\sigma_0^2}
            =
            \frac{(19)(5.7342)}{4}
            =
            27.24 \]
        We need to determine if $ u $ is to the right or left of the median $ \chi^2_{19} $.
        We know it will be to the right since the mean of $ \chi^2_{19} $ is $ 19 $.
        $ \chi^2 $ is right-skewed, so the mean must be bigger than the median,
        thus the median must be less than $ 19 $. Therefore, $ u> $ median. Alternatively,
        we can use the table and look at $ p=0.5,\,df=19\rightarrow 18.338<u $.
        \begin{align*}
            p\text{-value}
             & =2P(U\geqslant u)               \\
             & =2P(U\geqslant 27.24)           \\
             & =2P(\chi^2_{19}\geqslant 27.24)
        \end{align*}
        We see that $ 27.24 $ falls between $ p=0.9 $ and $ p=0.95 $. The area to the right of
        $ p=0.9 $ is $ 10\% $ and the area to the right of $ p=0.95 $ is $ 5\% $.
        Thus, $ 2P(5\%\text{and }\% 10)=10\%\text{ and} 20\% $,
        which implies $ p>0.1 $ and we conclude there is no evidence against null-hypothesis.
    \end{example}
\end{exbox}
