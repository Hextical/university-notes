\chapter{Online Lectures}
\section{2020-03-16: Testing for Variances}
\underline{Roadmap}:
\begin{enumerate}[(i)]
    \item General info
    \item Testing for variance for Normal
    \item An example
\end{enumerate}
The general problem: $ Y_1,\ldots ,Y_n \thicksim N(\mu,\sigma^2)$ iid
where $ \mu $ and $ \sigma^2 $ are both unknown.
$ H_0 $: $ \sigma^2=\sigma_0^2 $ vs two sided alternative.
\begin{enumerate}[(i)]
    \item Test statistic? Problem
    \item Convention?
\end{enumerate}
The pivot is:
\[ U=\frac{(n-1)s^2}{\sigma_0^2} \thicksim \chi^2_{n-1} \]
can we use this as our test statistic?

\begin{exbox}
    \begin{example} $ \; $
        \begin{itemize}
            \item Normal population: $ \{y_1,\ldots ,y_n\} $
            \item $ n=20 $, $ \sum y_i=888.1 $, $ \sum y_i^2=39545.03 $
            \item $ H_0 $: $ \sigma=2 $
            \item $ H_1 $: $ \sigma\neq 2 $
        \end{itemize}
        What is the $ p $-value?
        We know
        \[ s^2=\frac{1}{n-1} \left[ \sum y_i^2-n\overline{y}^2 \right]=5.7342 \]
        Compute $ U $:
        \[ U=\frac{(n-1)s^2}{\sigma_0^2}=27.24 \]
        $ \chi^2_{19} $
        \begin{align*}
            p\text{-value}
             & =2P(U\geqslant 27.24)           \\
             & =2P(\chi^2_{19}\geqslant 27.24) \\
             & =10\%\text{ and }20\%
        \end{align*}
        so, $ p>0.1 $ means there is no evidence against null-hypothesis.
    \end{example}
\end{exbox}
