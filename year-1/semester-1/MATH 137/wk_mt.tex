\section{The Intermediate Value Theorem}
\subsection{Approximating Solutions to Equations}
\subsection{The Bisection Method}
\section{The Extreme Value Theorem}
\chapter{Derivatives}
\section{Instantaneous Velocity}
Suppose you are driving down a highway. Every 30 minutes you record your distance:
\begin{center}
    \begin{tabular}{cccccccc}
        Time (min)    & 0 & 30 & 60  & 90  & 120 & 150 & 180 \\
        \midrule
        Distance (km) & 0 & 55 & 100 & 130 & 200 & 250 & 300
    \end{tabular}
\end{center}
\begin{itemize}
    \item What was your average speed in these three hours?
          \[ \text{Average speed}=\frac{\text{distance}}{\text{time}}=\frac{300\text{ km}}{1.5\text{ h}}=100\text{ km/h}. \]
    \item First 1.5 hours?
          \[ \frac{130}{1.5}\approx 86.6\text{ km/h}. \]
    \item Last 1.5 hours?
          \[ \frac{300-130}{1.5}\approx 113\text{ km/h}. \]
\end{itemize}
In general, the formula for the \textbf{average velocity}, $ V_{\text{ave}} $ from $ t=t_0 $ to $ t=t_1 $ is
\[ V_{\text{ave}}=\frac{s(t_1)-s(t_0)}{t_1-t_0}, \]
where $ s(t) $ is the distancea at time $ t $. To get the instantaneous velocity, we need to use limits!
The instantaneous velocity at $ t=t_0 $ is
\[ \lim\limits_{{t} \to {t_0}}\frac{s(t)-s(t_0)}{t-t_0} \]
or
\[ \lim\limits_{{h} \to {0}}\frac{s(t_0+h)-s(t_0)}{h}. \]
\begin{Example}{}{}
    Find the instantaneous velocity for $ s(t)=t^2+3t $ at $ t=1 $, $ t=2 $, and $ t_0\in\R $.
    \tcblower{}
    \textbf{Solution}.
    \begin{itemize}
        \item $\begin{aligned}[t]
                      \lim\limits_{{h} \to {0}}\frac{s(1+h)-s(1)}{h}
                       & =\lim\limits_{{h} \to {0}}\frac{(1+h)^2+3(1+h)-(1^2+3(1))}{h} \\
                       & =\lim\limits_{{h} \to {0}}\frac{5h+h^2}{h}                    \\
                       & =\lim\limits_{{h} \to {0}}(5+h)                               \\
                       & =5.
                  \end{aligned}$
        \item $\begin{aligned}[t]
                      \lim\limits_{{h} \to {0}}\frac{(2+h)^2+3(2+h)-(2^3+3(2))}{h}
                       & =\lim\limits_{{h} \to {0}}\frac{7h+h^2}{h} \\
                       & =7.
                  \end{aligned}$
        \item $\begin{aligned}
                      \lim\limits_{{h} \to {0}}\frac{(t_0+h)^2+3(t_0+h)-(t_0^2+3t_0)}{h}
                       & =\lim\limits_{{h} \to {0}}(2 t_0+3+h) \\
                       & =2t_0+3.
                  \end{aligned}$
    \end{itemize}
    The instantaneous velocity is a special case of a derivative!
\end{Example}
\section{Definition of the Derivative}
We can perform the same analysis that we did on $ s(t) $ in the previous section on any function!
\begin{Definition}{}{}
    The \textbf{average rate of change of $ f(x) $} from $ x=a $ to $ x=b $ is
    \[ f_{\text{ave}}=\frac{f(b)-f(a)}{b-a}. \]
\end{Definition}
\begin{Definition}{}{}
    The \textbf{instantaneous rate of change of $ f(x) $} at $ x=a $, or the derivative of $ f(x) $ at $ x=a $, denoted $ f'(a) $
    is defined as
    \[ f'(a)=\lim\limits_{{h} \to {0}}\frac{f(a+h)-f(a)}{h}=\lim\limits_{{x} \to {a}}\frac{f(x)-f(a)}{x-a}. \]
    If this limit exists, we say that $ f $ is \textbf{differentiable} at $ x=a $.
\end{Definition}
\subsection{The Tangent Line}
\begin{Definition}{}{}
    The \textbf{tangent line} to the graph of $ f $ at $ x=a $ is the line passing through $ (a,f(a)) $ with slope $ m=f'(a) $.
    It follows that the equation of the tangent line is
    \[ y=f(a)+f'(a)(x-a). \]
\end{Definition}
\begin{Example}{}{}
    Find the equation of the tangent line to $ f(x)=x^2+x+1 $ at $ x=3 $.
    \tcblower{}
    \textbf{Solution}. First, we should compute $ f'(3) $:
    \begin{align*}
        f'(3)
         & =\lim\limits_{{h} \to {0}}\frac{f(3+h)-f(3)}{h}               \\
         & =\lim\limits_{{h} \to {0}}\frac{(3+h)^2+(3+h)+1-(3^2+3+1)}{h} \\
         & =\lim\limits_{{h} \to {0}}\frac{9+6h+h^2+3+h+1-9-3-1}{h}      \\
         & =\lim\limits_{{h} \to {0}}\frac{7h+h^2}{h}                    \\
         & =\lim\limits_{{h} \to {0}}(7+h)                               \\
         & =7.
    \end{align*}
    So, $ f'(3)=7 $. The point on the graph is $ (3,f(3))=(3,13) $. So, the tangent line is
    \[ y=13+7(x-3)=13+7x-21=7x-8. \]
\end{Example}
\begin{Remark}{}{}
    Can't define the derivative as the slope of the tangent line! Without knowing what the derivative is first, we can't
    even define the tangent line!
\end{Remark}
\subsection{Differentiability versus Continuity}
\begin{itemize}
    \item Q\@: Does continuity imply differentiability?
    \item A\@: No! Consider $ f(x)=\abs{x} $ at $ x=0 $. Clearly,
          \[ \lim\limits_{{x} \to {0}}\abs{x}=0=\abs{0}, \]
          so $ f $ is continuous at $ x=0 $, but
          \[ \lim\limits_{{h} \to {0}}\frac{f(0+h)-f(0)}{h}=\lim\limits_{{h} \to {0}}\frac{\abs{h}}{h} \]
          does not exist. Therefore, $ f $ is not differentiable at $ x=0 $. Therefore,
          continuity \underline{does not} imply differentiability.
    \item Q\@: Does differentiability imply continuity?
    \item A\@: Yes!
\end{itemize}
\begin{Theorem}{Differentiability Implies Continuity}{}
    Let $ A\subseteq \R $ be open, let $ f\colon A\to\R $ and let $ a\in A $. If $ f $
    is differentiable at $ a $, then $ f $ is continuous at $ a $.
    \tcblower{}
    \textbf{Proof}: We have
    \[ f(x)-f(a)=\frac{f(x)-f(a)}{x-a}(x-a)\to f'(a)\cdot 0=0\text{ as $ x\to a $} \]
    and so
    \[ f(x)=\bigl(f(x)-f(a)\bigr)+f(a)\to 0+f(a)=f(a)\text{ as $ x\to a $}. \]
    This proves that $ f $ is continuous at $ a $.
\end{Theorem}