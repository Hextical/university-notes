\chapter{Sequences and Convergence}
\section{Absolute Values}
What is an absolute value? We commonly think of it as an operation that removes negative signs.
\begin{Example}{}{}
    $ \abs{-2}=2 $, $ \abs{-17}=17 $, $ \abs{3}=3 $, etc.
\end{Example}
So, is $ \abs{-x}=x $ for all $ x\in\R$? Not always! Let's give the definition to avoid ambiguity.
\begin{Definition}{\href{https://proofwiki.org/wiki/Definition:Absolute_Value/Definition_1}{Absolute Value}}{}
    Let $ x\in\R $. The \textbf{absolute value} of $ x $ is denoted $ \abs{x} $,
    and is defined as follows:
    \[ \abs{x}=\begin{cases}
            x,  & x>0, \\
            0,  & x=0, \\
            -x, & x<0.
        \end{cases} \]
\end{Definition}
This also tells us the distance from $ x $ to $ 0 $, or the \underline{magnitude} (size of $ x $).
\begin{Example}{}{}
    How do we get the distance between two arbitrary numbers using absolute values? For example, what is
    the distance from $ 3 $ to $ 7 $? $ 4 $ units. Also, $ \abs{7-3}=4=\abs{3-7} $.
\end{Example}
So, the distance from $ a $ to $ b $ is $ \abs{b-a} $ for all $ a,b\in\R $. Also,
$ \abs{b-a}=\abs{a-b} $, which makes sense since the distance from $ a $ to $ b $ should be the same as
the distance from $ b $ to $ a $.
\subsection{Inequalities Involving Absolute Values}
The main focus of this course is \textbf{approximation}. We will seek ways to approximate
roots, curves, limits, etc., but if we make an approximation it will be useless unless we can talk
about how close it is to the actual object! So, we will look for ways to determine the maximum
size of the \underline{\textbf{error}}. Before we do this, we will need to examine \underline{\textbf{inequalities}}.
Let's start with the triangle inequality.
\begin{Theorem}{Triangle Inequality}{}
    Let $ x,y,z\in\R $. Then
    \[ \abs{x-y}\le \abs{x-z}+\abs{z-y}. \]
    \tcblower{}
    \textbf{Proof}:
    Since $ \abs{x-y}=\abs{y-x} $, we can assume without loss of generality (WLOG) that $ x\le y $.
    Hence, we consider three cases.

    \underline{Case 1} ($ z<x $): Clearly, $ \abs{x-y}\le \abs{z-y} $, which means
    $ \abs{x-y}\le \abs{x-z}+\abs{z-y} $.

    \underline{Case 2} ($ x\le z\le y $): In this case, $ \abs{x-y}=\abs{x-z}+\abs{z-y} $, which means
    $ \abs{x-y}=\abs{x-z}+\abs{z-y} $, as desired.

    \underline{Case 3} ($ y<z $): This time, $ \abs{x-y}\le \abs{x-z} $, so $ \abs{x-y}\le \abs{x-z}+\abs{z-y} $.
\end{Theorem}
We consider a useful variant of the triangle inequality.
\begin{Theorem}{Triangle Inequality II}{}
    Let $ x,y\in\R $. Then
    \[ \abs{x+y}\le \abs{x}+\abs{y}. \]
    \tcblower{}
    \textbf{Proof}:
    \begin{align*}
        \abs{x+y}
         & =\abs{x-(-y)}                                                           \\
         & \le \abs{x-0}+\abs{0-(-y)} &  & \text{triangle inequality with $ z=0 $} \\
         & =\abs{x}+\abs{y}.
    \end{align*}
\end{Theorem}
If we want to prove $ \abs{x}<\delta $, we just need to prove $ x<\delta $ and $ x>-\delta $, that is,
$ -\delta<x<\delta $. So, what do the inequalities of the form $ \abs{x-a}<\delta $ for
$ a,\delta\in\R $ look like? What set does this represent? Well, it's the set of all
$ x\in\R $ that are less than $ \delta $ units away from $ a $. So, starting at $ a $,
we move $ \delta $-units to the left and right, which means
\[ \abs{x-a}<\delta\iff -\delta<x-a<\delta\iff a-\delta<x<a+\delta. \]
So, it is the interval $ (a-\delta,a+\delta) $, where we do not include the endpoints as the inequality is strict.

What about $ \abs{x-a}\le \delta $? In this case,
\[ \abs{x-a}\le \delta\iff -\delta\le x-a\le \delta\iff a-\delta\le x\le a+\delta. \]
So, it is the interval $ [a-\delta,a+\delta] $.

What about $ 0<\abs{x-a}<\delta $? Now, the distance can't be zero which means $ x\ne a $. So,
it translates to $ (a-\delta,a+\delta)\setminus\{a\} $ or $ (a-\delta,a)\cup (a,a+\delta) $.

\begin{Example}{}{}
    Find the corresponding sets for the inequalities.
    \begin{enumerate}[(1)]
        \item $ \abs{x-4}<3 $.
        \item $ 2\le \abs{x-4}<4 $.
        \item $ \abs{x-1}+\abs{x+2}\ge 4 $.
    \end{enumerate}
    \tcblower{}
    \textbf{Solution}.
    \begin{enumerate}[(1)]
        \item $ \abs{x-4}<3\iff -3<x-4<3\iff 1<x<7 $, so $ (1,7) $ is the corresponding interval.
        \item $ 2\le \abs{x-4}<4 $ means $ 2\le \abs{x-4} $ and $ \abs{x-4}<4 $, so
              \[ (2\le x-4)\lor (x-4\le -2)\iff (6\le x)\lor (x\le 2) \]
              and
              \[ -4<x-4<4\iff 0<x<8. \]
              Putting these together, we get $ 0<x\le 2 $ or $ 6\le x<8 $, so $ (0,2]\cup [6,8) $ is the corresponding interval.
        \item We consider three cases.
              \begin{enumerate}[(i)]
                  \item If $ x> 1 $, then both $ x-1> 0 $ and $ x+2> 0 $, so we get
                        \[ x-1+x+2> 4\iff 2x+1> 4\iff 2x> 3\iff x>3/2. \]
                  \item If $ -2\le x\le 1 $, then $ \abs{x-1}=1-x $, but $ \abs{x+2}=x+2 $, so we get
                        \[ 1-x+x+2\ge 4\iff 3\ge 4, \] which is not true for \underline{any} $ x $.
                  \item If $ x<-2 $, then $ \abs{x-1}=1-x $ and $ \abs{x+2}=-x-2 $, so we get
                        \[ 1-x+(-x-2)\ge 4\iff -1-2x\ge 4\iff -5\ge 2x\iff -5/2\ge x. \]
              \end{enumerate}
              Putting it all together, we get $ x>3/2 $ or $ x\le -5/2 $, that is,
              $ (-\infty,-5/2]\cup (3/2,\infty) $.
    \end{enumerate}
\end{Example}