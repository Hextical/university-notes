\section{Week 1}
To begin with, our goal is to develop methods for determining the area under a curve.
We know we can approximate the area using rectangles (or other geometric shapes), but
we want the \textbf{exact} area. For this we will need \textbf{Riemann sums}.

\subsection{Definition (Partition)}
A \emph{partition}, $P$ for the interval $ [a,b] $ is a finite sequence of increasing
numbers of the form
\[ a=t_0<\dots <t_{n-1}=b \]
This partition subdivides the interval $ [a,b] $ into $ n $ subintervals:
\[ [t_0,t_1],\ldots [t_{n-1},t_n] \]
These subintervals may \textbf{not} all have the same length.
Denote the length of the $ i^{th} $ subinterval $ [t_{i-1},t_i] $ by $ \Delta t_i $
(so $ \Delta t_i=t_i-t_{i-1} $)

\subsection{Definition (Norm)}
The \textbf{norm} of a partition is the length of the widest subinterval:
\[ ||P||=\max\{\Delta t_1,\dots,\Delta t_{n}\} \]

\subsection{Definition (Riemann Sum)}
Given a bounded function $ f $ on $ [a,b] $, a partition $ P $ of $ [a,b] $, and a set
$ \{c_1,\dots,c_n\} $ where $ c_i\in[t_{i-1},t_i] $, then a \textbf{Riemann Sum} for
$f$ with respect to $ P $ is
\[ S=\sum\limits_{i=1}^{n} f(c_i)\Delta t_i \]

Again, we want the \textbf{exact} area, and for that we will need to use infinitely
many points!

But we do need to make sure that the norm of our partitions is getting smaller,
and that the area we get doesn't depend on the choise of Riemann Sum.

\subsection{Definition (Integrable)}
We say that $ f $ is \textbf{integrable} on $ [a,b] $ if there exists a unique number
$ I\in\mathbb{R} $ such that if whenever $ \{P_n\} $ is a sequence of partitions with
$ \lim\limits_{{n} \to {\infty}} ||P_n||=0 $ and $ \{S_n\} $ is any sequence of
Riemann Sums associated to the $ P_n $'s, we have $ \lim\limits_{{n} \to {\infty}} S_n=I $.

In this case, we call $ I $ the \textbf{integral of f over $ [a,b] $ } and denote it by
\[ \int\limits_{a}^{b} f(x) dx \]
where $ a,b $ are the bounds of integration, $ f(x) $ is the integrand, $ x $ is the
variable of integration. The complete thing is called a definite integral.

It represents the exact (signed) area under $ f $.

\begin{remark}
    The variable of integration is a \textbf{dummy variable} since we can change it into
    whatever we wand and it won't change the value of the integral.

    That is:
    \[
        \int\limits_{a}^{b} f(x) dx =
        \int\limits_{a}^{b} f(t) d{t} = \text{etc.}
    \]
\end{remark}

This looks \textbf{horrible} to compute in practice (and it is). The good news is if
$ f $ is continuous, it's not so bad! (still bad though)

\subsection{Theorem (Integrability Theorem for Cont. Functions)}
Let $ f $ be continuous on $ [a,b] $. Then $ f $ is integrable on $ [a,b] $.

(Proof is omitted).

This is fantastic! This means that we can \textbf{choose} any collection of Riemann Sums
we want when computing the integral of a continuous function!

Let's examine a "nice" choice: one where the partition is regular and where we just
pick the $ c_i $'s to be the right-hand endpoints!

\subsection{Definition (Regular n-partition)}
For the interval $ [a,b] $, the \emph{regular $n-$partition} where all $ n $ subintervals
have the same length, that is $ \Delta t=\frac{b-a}{n} $ and $ t_i=t_o+i\Delta t $.

Using this, we define the \textbf{regular right-hand Riemann Sum} by taking $ c_i=t_i $ for
all $ i $:
\[ S_n=\sum\limits_{i=1}^{n} f(t_i)\Delta t=\sum\limits_{i}^{n} f(t_i)\left(\frac{b-a}{n}\right) \]

\begin{remark}
    We can also define the left-hand Riemann Sum.
\end{remark}

Now, we can write a nicer formula for integrating continuous functions!

If $ f $ is continuous, then
\[ \int\limits_{a}^{b} f(x) d{x} =
    \lim\limits_{{n} \to {\infty}} \sum\limits_{i=1}^{n} f(t_i)\left(\frac{b-a}{n}\right) \]

\textbf{Example}

Find
\[ \int\limits_{0}^{4} x+x^3 d{x} \]

Since $ f(x)=x+x^3 $ is continuous, we can use the above formula.

In our case: $ \frac{b-a}{n} = \frac{4}{n} $, and $ t_i = 0+\frac{4i}{n} = \frac{4i}{n} $.

So, $ f(t_i) = \frac{4i}{n} + \frac{64i^3}{n^3} $.
Then, we get:
\begin{align}
    \int\limits_{0}^{4} x+x^3 d{x}
     & = \lim\limits_{{n} \to {\infty}} \sum\limits_{i=1}^{n}
    \left( \frac{4i}{n} +\frac{64i^3}{n^3} \right)\left( \frac{4}{n} \right)                  \\
     & = \lim\limits_{{n} \to {\infty}} \frac{16}{n^2} \sum\limits_{i=1}^{n} i +
    \frac{256}{n^4} \sum\limits_{i=1}^{n} i^3                                                 \\
     & = \lim\limits_{{n} \to {\infty}} \frac{16}{n^2} \left( \frac{n(n+1)}{2} \right) +
    \frac{256}{n^4} \left( \frac{n^2(n+1)^2}{4} \right)                                       \\
     & = \lim\limits_{{n} \to {\infty}} \frac{8n+8}{n} +64 \left(\frac{n^2+2n+1}{n^2} \right) \\
     & = 8+64                                                                                 \\
     & =72
\end{align}
where from (2) to (3) we used both of the following:
\[ \sum\limits_{i=1}^{n} i=\frac{n(n+1)}{2} \]
\[ \sum\limits_{i=1}^{n} i^3=\frac{n^2(n+1)^2}{4} \]

Since a definite integral is the limit of a sequence, many limit laws also hold!

\subsection{Theorem (Properties of Integrals)}
If $ f $ is integrable on $ [a,b] $, then:
\begin{enumerate}[(1)]
    \item For any $ c\in\mathbb{R} $,
          \[ \int\limits_{a}^{b} cf(x) d{x} = c \int\limits_{a}^{b} f(x) d{x} \]
    \item r
          \[ \int\limits_{a}^{b} (f+g)(x) d{x} = \int\limits_{a}^{b} f(x) d{x} +
              \int\limits_{a}^{b} g(x) d{x} \]
    \item If $ m\le f(x)\le M $ for $ x\in[a,b] $, then
          \[ m(b-a)\le \int\limits_{a}^{b} f(x) d{x} \le M(b-a) \]
    \item If $ 0\le f(x) $ for $ x\in[a,b] $, then
          \[ 0\le \int\limits_{a}^{b} f(x) d{x} \]
    \item If $ f(x)\le g(x) $ for $ x\in[a,b] $, then
          \[ \int\limits_{a}^{b} f(x) d{x} \le \int\limits_{a}^{b} g(x) d{x} \]
    \item If $ |f| $ is integrable on $ [a,b] $, then
          \[ \left|\int\limits_{a}^{b} f(x) d{x}\right|\le \int\limits_{a}^{b} |f(x)| d{x} \]
\end{enumerate}

\textbf{Proof}: (1) and (2) follow from limit laws for sequences. (3) implies (4),
(1), (2), and (4) imply (5). (6) follows from the triangle inequality

\begin{proof} (3)

    Suppose $ m\le f(x)\le M $ and partition the interval
    \[ a=t_0<\dots<t_n=b \]

    Note that
    \[ \sum\limits_{i=1}^{n} \Delta t=\frac{b-a}{n}(b)=b-a \]
    Then, since $ m\le f(x)\le M $, we get
    \[ m(b-a)=\sum\limits_{i=1}^{n} m\Delta t\le \sum\limits_{i=1}^{n} f(t_i)\Delta t
        \le \sum\limits_{i=1}^{n} M\Delta t=M(b-a) \]
    So, taking limits gives
    \[ m(b-a)\le \int\limits_{a}^{b} f(x) d{x} \le M(b-a) \]
\end{proof}

\subsection{Definition (More Properties)}
\begin{enumerate}[(1)]
    \item If $ f(a) $ is defined, then
          \[ \int\limits_{a}^{a} f(x) d{x} =0 \]
    \item If $ f $ is integrable on $ [a,b] $, then
          \[ \int\limits_{a}^{b} f(x) d{x}=-\int\limits_{b}^{a} f(x) d{x} \]
    \item (Theorem) If $ f $ is integrable on an interval $ I $ containing $ a,b,c $, then
          \[ \int\limits_{a}^{b} f(x) d{x}=\int\limits_{a}^{c} f(x) d{x}+\int\limits_{c}^{b} f(x) d{x} \]
\end{enumerate}

\begin{remark}
    $ c $ does \textbf{not} need to be between $ a $ and $ b $!
\end{remark}

\subsection{Geometric Interpretation of the Integral}
So far, we have only examined positive functions, but we should note that $ \int\limits_{a}^{b} f(x)dx $
returns the \textbf{signed} area between $ f $ and the $ x $-axis. That is, if $ f(x)\le 0 $, then
$ \int\limits_{a}^{b} f(x)dx\le 0 $ too.

\subsection{Definition (Average Value)}
If $ f $ is continuous on $ [a,b] $, the \textbf{average value} of $ f $ on $ [a,b] $ is defined as
\[ \frac{1}{b-a} \int\limits_{a}^{b} f(x)dx \]

\subsection{Geometric Interpretation}
If $ f $ is continuous on $ [a,b] $, EVT says there exists $ m,M\in\mathbb{R} $ such that
\[m\le f(x) \le M\]
for $ x\in[a,b] $ and $ f(c_1)=m $, $ f(c_2)=M $ for some $ c_1,c_2\in[a,b] $
Also, we know \[ m(b-a)\le \int\limits_{a}^{b} f(x) d{x} \le M(b-a) \]
So,
\[ m(b-a)\le \frac{1}{b-a} \int\limits_{a}^{b} f(x) d{x} \le M(b-a) \]
\[ f(c_1)\le \frac{1}{b-a} \int\limits_{a}^{b} f(x) d{x} \le f(c_2) \]
IVT says there exists $ c $ between $ c_1 $ and $ c_2 $, so that
\[ f(c)=\frac{1}{b-a} \int\limits_{a}^{b} f(x) d{x} \]

We have proven:
\subsection{Theorem (Average Value Theorem, AVT)}
Assume $ f $ is continuous on $ [a,b] $. There exists $ c\in[a,b] $ such that
\[ f(c)=\frac{1}{b-a} \int\limits_{a}^{b} f(x) d{x} \]

\begin{remark}
    Note that this theorem holds even if $ b<a $ since
    \begin{align*}
        f(c) & =\frac{1}{b-a} \int\limits_{a}^{b} f(x) d{x}              \\
             & =\frac{1}{a-b}\left(-\int\limits_{a}^{b} f(x) d{x}\right) \\
             & =\frac{1}{b-a} \int\limits_{a}^{b} f(x) d{x}
    \end{align*}
\end{remark}

The big problem we face now is that evaluating $ \int\limits_{a}^{b} f(x) d{x} $ is
monstrously difficult for all but the simplest of functions...

IF ONLY THERE WAS A BETTER WAY!

(spoilers: there's a better way! It's the Fundamental Theorem of Calculus!)
