\section{Sequences and Their Limits}
\subsection{Introduction to Subsequences}
\begin{Definition}{}{}
    An infinite sequence of numbers is a list of numbers in a definite order, e.g.,
    \[ a_1,a_2,a_3,a_4,\ldots,a_n,\; a_i\in\R. \]
    \underline{Notation}: $ \Set{a_1,a_2,\ldots,a_n} $ or $ \Set{a_n}_{n=1}^{\infty} $ or $ \Set{a_n} $.
\end{Definition}
Sequences can be defined explicitly (in terms of $ n $) or recursively (in terms of previous terms).
\begin{Example}{Explicit Sequences}{}
    \begin{itemize}
        \item $ \Set*{\frac{1}{n+1}}_{n=1}^{\infty} $: $ 1/2,1/3,1/4,1/5,\ldots $.
        \item $ \Set{\sqrt{n+2}}_{n=2}^{\infty} $: $ \sqrt{4},\sqrt{5},\sqrt{6},\ldots $.
        \item $ \Set{(-1)^n}_{n=1}^\infty $: $ -1,1,-1,1,\ldots $.
    \end{itemize}
\end{Example}
\subsection{Recursively Defined Sequences}
\begin{Example}{Recursive Sequences}{}
    \begin{itemize}
        \item $ a_1=1 $, $ a_{n+1}=\sqrt{1+a_n} $, so $ a_1=1 $, $ a_2=\sqrt{2} $, $ a_3=\sqrt{1+\sqrt{2}} $, and so on for $ n\ge 1 $.
        \item Fibonacci's sequence: $ a_1=1 $, $ a_2=1 $, $ a_{n+2}=a_{n+1}+a_n $ for $ n\ge 1 $, i.e.,
              $ 1,1,2,3,5,8,13,\ldots $.
    \end{itemize}
\end{Example}
We can plot sequences on a number line, or we could think of a sequence as a function $ f\colon \N\to\R $, writing $ f(n)=a_n $, e.g.,
for $ a_n=1/2 $ we would write $ f(n)=1/2 $.

\underline{Why study sequences?}
\begin{itemize}
    \item Lots of continuous processes can be modelled with discrete data, as we will see.
    \item We can use recursive sequences to approximate solutions to equations that can't be solved explicitly (Newton's Method).
    \item For another (ancient) application, see page 14 of the course notes about calculating square roots.
\end{itemize}
Our goal now will be to determine how to find the limit of a sequence, that is, find what the value of the terms of the sequence
are approaching (if it exists).

We may want to build new sequences out of old ones or only discuss what happens to a sequence eventually, that is, after a certain index.
\begin{Example}{}{}
    For $ \Set{\frac{1}{n}}_{n=1}^{\infty} $, if we consider only the odd terms, we get $ 1,1/3,1/5 $, or the $ k\textsuperscript{th} $
    term is
    \[ \frac{1}{2k-1} \]
    for $ k\in\N $.
    This is called a subsequence.
\end{Example}
\subsection{Subsequences and Tails}
\begin{Definition}{Subsequence}{}
    If $ \Set{a_n} $ is a sequence and $ {n_1,n_2,\ldots} $ is a sequence of natural numbers, where
    $ n_1<n_2<n_3<\cdots $, then the sequence
    \[ \Set{a_{n_1},a_{n_2},\ldots}=\Set{a_{n_k}} \]
    is a \textbf{subsequence} of $ \Set{a_n} $.
\end{Definition}
One particular subsequence is $ \Set{a_k,a_{k+1},a_{k+2}} $ for some $ k\in\N $.
This is called the tail of $ \Set{a_n} $ with cut-off $ k $.

\subsection{Limits of Sequences}
We are going to see lots of different limits this term, but we will start with sequences.
\begin{Example}{}{}
    $ \Set{\frac{1}{n}} $ seems like it converges to $ 0 $, or that $ 0 $ is the limit of the sequence.
    We saw this when we plotted the sequence. We will eventually want a formal definition, but let's start intuitively.
\end{Example}
Given a sequence $ \Set{a_n} $, what does it mean to say that $ \Set{a_n} $ converges to $ L $
as $ n $ goes to infinity?

What about ``as $ n $ gets larger, $ a_n $ gets closer to $ L $?'' Unfortunately, this isn't a good definition. For example, as $ n $
gets larger $ \frac{1}{n} $ gets closer to $ 0 $, but it also gets closer to $ -1 $, $ -2 $, and so on. But, $ 0 $
is \underline{the} limit! What makes it different? Well, the sequence gets infinitely close to $ 0 $,
unlike the other numbers! Let's try to define this again: ``the limit of $ \Set{a_n} $ is $ L $
if, as $ n $ gets infinitely large, $ a_n $ gets infinitely close to $ L $.'' This is much better!
But how can we formalize the idea of ``infinitely close?''
\begin{Definition}{Formal Definition of the Limit of a Sequence I}{}
    Let $ \seq $ be a sequence in $ \R $. For $ L\in\R $, we say that the sequence $ \seq $ \textbf{converges} to $ L $
    (or that the \textbf{limit} of $ \seq $ is equal to $ L $), and we write $ a_n\to L $ (as $ n\to\infty $), or we write
    $ \lim\limits_{{n} \to {\infty}}a_k=L $, when
    \[ \forall \varepsilon\in\R_{>0}\; \exists N\in \N\; \forall n\in\N\;\bigl(n\ge N\implies \abs{a_n-L}<\varepsilon\bigr). \]
    We say that the sequence $ \seq $ \textbf{diverges to infinity}
    (in $ \R $) when there exists $ L\in \R $ such that $ \seq $ converges to $ L $.
    We say that the sequence \textbf{diverges} (in $ \R $) when it does not converge (to any $ L\in\R $).
\end{Definition}
\begin{Example}{}{}
    Consider $ a_n=\frac{1}{n^2} $. We'd guess that the limit is $ 0 $. Say $ \varepsilon=\frac{1}{100} $, can we find a large enough $ n\in N $
    so that $ \abs*{\frac{1}{n^2}-0}<\frac{1}{100}  $ if $ n\ge N $? Well, we need
    \[ \abs*{\frac{1}{n^2}-0}<\frac{1}{100}\implies \frac{1}{n^2}<\frac{1}{100}\implies n^2>100, \]
    so $ n>10 $. Let $ N=11 $, then if $ n\ge N $, we get
    $ \abs*{\frac{1}{n^2}-0}<\frac{1}{100} $. But wait! We aren't done yet! The definition says we need to prove it for any $ \varepsilon>0 $,
    but we only proved it for $ \varepsilon=\frac{1}{100} $. Let's adapt the proof to work for any $ \varepsilon>0 $.
\end{Example}
\underline{Proof that $ \lim\limits_{{n} \to {\infty}}\frac{1}{n^2}=0 $}.
Let $ \varepsilon>0 $ be given. Let $ N>\frac{1}{\sqrt{\varepsilon}} $ for $ N\in \N $. Then, if $ n\ge N $,
we get
\[ \abs*{\frac{1}{n^2}-0}=\frac{1}{n^2}\le \frac{1}{N^2}<\frac{1}{(1/\sqrt{\varepsilon})^2}=\frac{1}{1/\varepsilon}=\varepsilon \]
as desired.

The point is: we have to give a method for choosing $ N $ that works for \underline{any} $ \varepsilon>0 $. Also, the logical order of the
proof is important, so let's do some more examples.

\begin{Example}{}{}
    Prove that $ \lim\limits_{{n} \to {\infty}}\frac{n}{2n+3}=\frac{1}{2} $.
    \tcblower{}
    \textbf{Proof}: Let $ \varepsilon>0 $ be given. Let $ N>\frac{1}{4}\bigl(\frac{3}{\varepsilon}-6\bigr) $ for $ N\in\N $.
    Then, if $ n\ge N $, we get:
    \[ \abs{a_n-L}=\abs*{\frac{n}{2n+3}-\frac{1}{2}}=\frac{3}{4n+6}\le \frac{3}{4N+6}<\frac{3}{4\bigl(\frac{1}{4}(\frac{3}{\varepsilon}-6)\bigr)+6}=\varepsilon \]
    as desired.

    \underline{Aside}: We want
    \[ \frac{3}{4n+6}<\varepsilon\iff \frac{3}{\varepsilon}<4n+6\iff \frac{3}{\varepsilon}-6<4n\iff \frac{1}{4}\biggl(\frac{3}{\varepsilon}-6\biggr)<n. \]
\end{Example}
\begin{Example}{}{}
    Prove that $ \lim\limits_{{n} \to {\infty}}\frac{n^2}{3n^2+7n}=\frac{1}{3} $.
    \tcblower{}
    \textbf{Proof}: Let $ \varepsilon>0 $ be given. Let $ N>\frac{7}{9\varepsilon} $ for $ N\in\N $.
    Then, if $ n\ge N $, we get:
    \[ \abs*{a_n-L}=\abs*{\frac{n^2}{3n^2+7n}-\frac{1}{3}}=\frac{7n}{9n^2+21n}\le \frac{7n}{9n^2}=\frac{7}{9n}\le \frac{7}{9(\frac{7}{9\varepsilon})}=\varepsilon. \]
    \underline{Aside}: We want
    \[ \frac{7}{9n}<\varepsilon\iff \frac{7}{9\varepsilon}<n. \]
\end{Example}
\begin{Remark}{Avoid Common Mistakes}{}
    \begin{itemize}
        \item Don't choose $ \varepsilon $! Let it be arbitrary.
        \item Never assume $ \abs{a_n-L}<\varepsilon $, make sure you only do work in an aside with that inequality since it is what you
              are proving.
        \item In practice, unless you are asked to, do not use this formal definition. We will now try to develop better methods for finding limits.
    \end{itemize}
\end{Remark}
\subsection*{Equivalent Definitions of the Limit}
When proving $ \lim\limits_{{n} \to {\infty}}a_n=L $, we are given $ \varepsilon>0 $, and we try to find $ N\in\N $
so that if $ n\ge N $, then $ \abs{a_n-L}<\varepsilon $. But, this is the same as saying
$ a_n\in (L-\varepsilon,L+\varepsilon) $. Also, the collection of $ \Set{a_n} $ for which
$ n\ge N $ is the tail of the sequence with cut-off $ N $. So, here's another definition.
\begin{Definition}{}{}
    $ \lim\limits_{{n} \to {\infty}}a_n=L $ if for any $ \varepsilon>0 $, the interval
    $ (L-\varepsilon,L+\varepsilon) $ contains a tail of the sequence $ \Set{a_n} $.
\end{Definition}
Let's push it further! Since the above is true for any $ \varepsilon>0 $, if we pick any
open interval $ (a,b) $ containing $ L $, then we can find a small enough $ \varepsilon>0 $
so that $ (L-\varepsilon,L+\varepsilon)\subseteq (a,b) $. Therefore, any interval containing
$ L $ also contains a tail of $ \Set{a_n} $. Let's collect all of these alternate (but equivalent) definitions together.
\begin{Theorem}{}{}
    The following are equivalent:
    \begin{enumerate}[(1)]
        \item $ \lim\limits_{{n} \to {\infty}}a_n=L $.
        \item For any $ \varepsilon>0 $, $ (L-\varepsilon,L+\varepsilon) $ contains a tail of $ \Set{a_n} $.
        \item For any $ \varepsilon>0 $, $ (L-\varepsilon,L+\varepsilon) $ contains all but finitely many terms of $ \Set{a_n} $.
        \item Every interval $ (a,b) $ containing $ L $ contains a tail of $ \Set{a_n} $.
        \item Every interval $ (a,b) $ containing $ L $ contains all but finitely many terms of $ \Set{a_n} $.
    \end{enumerate}
    Clearly, changing finitely many terms of $ \Set{a_n} $ does not affect the convergence or the limit.
\end{Theorem}
\begin{Example}{}{}
    Can a sequence have more than one limit?
    Consider $ \Set{(-1)^n}=-1,1,-1,1,\ldots $, it equals to both $ 1 $ and $ -1 $ infinitely often. Could both $ 1 $ and $ -1 $
    be the limits? No! Let's prove $ -1 $ isn't a limit.
    \tcblower{}
    \textbf{Proof}: Consider the interval $ (-2,0) $. Clearly $ -1\in(-2,0) $, but
    this interval does not contain any of the infinitely many $ 1 $'s in the sequence. So, $ -1 $
    is not a limit by (5) above. A similar argument can be used with the interval $ (0,2) $ to show
    $ 1 $ is also not a limit. So, does $ \Set{(-1)^n} $ have a limit at all? Let's prove it doesn't!
    Let $ \varepsilon=1/2 $, and supposed for a contradiction that the sequence converges to $ L\in\R $. That means
    the interval $ (L-1/2,L+1/2) $ must contain all but finitely many terms of the sequence, that is, but $ 1 $ and $ -1 $
    must lie in that interval. But the interval is only $ 1 $ unit long! So there is not $ L\in\R $ for which both $ 1 $ and $ -1 $
    lie inside $ (L-1/2,L+1/2) $. So, $ \Set{(-1)^n} $ diverges.
\end{Example}
A similar argument can be used to prove limits are unique.
\begin{Theorem}{}{}
    Let $ \seq $ be a sequence in $ \R $. If $ \seq $ has a limit (finite or infinite), then the limit is unique.
    \tcblower{}
    \textbf{Proof}: Suppose for a contradiction that $ L $ and $ M $
    are both limits of $ \Set{a_n} $ and $ L\ne M $ and WLOG that $ L<M $. Consider two intervals:
    \[ (L-1,\tfrac{L+M}{2})\ni L,\quad (\tfrac{L+M}{2},M+1)\ni M. \]
    This means, by definition, only finitely many terms of the sequence are not in the first interval and only finitely
    many terms are not in the second interval. But the sequence has infinitely many terms! So, at least one term is in both intervals
    which is impossible. This is a contradiction, so $ L=M $.

    \underline{Note}: This does not cover the cases where the limit is infinite.
\end{Theorem}
\begin{Remark}{A Remark on Possible Limits}{}
    If $ a_n\ge 0 $ for all $ n $, then $ \Set{a_n} $ can't converge to a negative number! If it did, say to $ L<0 $,
    then the interval $ (L-1,0) $ would contain $ L $ but no terms of the sequence.
\end{Remark}
\begin{Theorem}{}{}
    If $ a_n\ge 0 $ for all $ n $ and $ \lim\limits_{{n} \to {\infty}}a_n=L $, then $ L\ge 0 $.
    More generally, if $ \alpha\le a_n\le \beta $ for all $ n $ and $ \lim\limits_{{n} \to {\infty}}a_n=L $,
    then $ \alpha\le L\le \beta $.
\end{Theorem}
\begin{itemize}
    \item Q\@: If $ a_n>0 $ for all $ n $ and $ \lim\limits_{{n} \to {\infty}}a_n=L $ is $ L>0 $?
    \item A\@: Not necessarily! Consider $ a_n=\frac{1}{n}>0 $, but $ L=0 $.
\end{itemize}
\subsection{Divergence to Infinity}
Consider $ a_n=n $. It is clear that the sequence is getting larger without bound, so $ \lim\limits_{{n} \to {\infty}}a_n $
does not exist. That is, $ \Set{a_n} $ diverges. But we can say more! Since it does not get infinitely large,
we can make a definition to capture this.
\begin{Definition}{}{}
    \begin{itemize}
        \item We say that $ \seq $ \textbf{diverges to $ \infty $}, or that the limit of $ \seq $ is equal to \textbf{infinity}, and we write
              $ a_n\to \infty $ (as $ n\to\infty $), or we write $ \lim\limits_{{n} \to {\infty}}a_n=\infty $, when
              \[ \forall M\in\R_{>0}\; \exists N\in\N\;\forall n\in \N\; \bigl(n\ge N\implies a_n>M\bigr). \]
              Equivalently, any interval of the form $ (M,\infty) $ contains a tail of $ \Set{a_n} $.
        \item We say that $ \seq $ \textbf{diverges to $ -\infty $}, or that the limit of $ \seq $ is equal to
              \textbf{negative infinity}, and we write
              $ a_n\to -\infty $ (as $ n\to\infty $), or we write $ \lim\limits_{{n} \to {\infty}}a_n=-\infty $, when
              \[ \forall M\in\R_{<0}\; \exists N\in\N\;\forall n\in \N\; \bigl(n\ge N\implies a_n<M\bigr). \]
              Equivalently, any interval of the form $ (-\infty,M) $ contains a tail of $ \Set{a_n} $.
    \end{itemize}
\end{Definition}
\begin{Example}{}{}
    Show $ \lim\limits_{{n} \to {\infty}}(1-n)=-\infty $.
    \tcblower{}
    \textbf{Proof}: Let $ M<0 $ be given, pick $ N>1-M $ for $ N\in\N $. Then, if $ n\ge N $, we have
    \[ a_n=1-n\le 1-N<1-(1-M)=M. \]
    \underline{Aside}: Want $ 1-n<M\iff 1-M<n $.
\end{Example}