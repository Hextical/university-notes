\section{Sequences and Their Limits}
\subsection{Introduction to Subsequences}
\begin{Definition}{\href{https://proofwiki.org/wiki/Definition:Mapping\#Definition_1}{Mapping}}{}
    A \textbf{mapping} from $ S $ to $ T $ is a binary relation
    on $ S\times T $ which associates each element of $ S $ with exactly one element of $ T $.
\end{Definition}
\begin{Definition}{
        \href{https://proofwiki.org/wiki/Definition:Domain_(Relation_Theory)/Mapping}{Domain},
        \href{https://proofwiki.org/wiki/Definition:Codomain_(Relation_Theory)/Mapping}{Codomain}
    }{}
    Let $ S $ and $ T $ be sets.\smallskip

    Let $ f\colon S\to T $ be a mapping.\bigskip

    The \textbf{domain} of $ f $ is $ S $, and can be denoted $ \Dom(f) $.\smallskip

    The \textbf{codomain} of $ f $ is $ T $, and can be denoted $ \Cdm(f) $
\end{Definition}
\begin{Definition}{\href{https://proofwiki.org/wiki/Definition:Subset}{Subset}}{}
    Let $ S $ and $ T $ be sets.\bigskip

    $ S $ is a \textbf{subset} of $ T $ if and only if all the elements of $ S $ are also elements of $ T $.
    \[ S\subseteq T\iff \forall x:(x\in S\implies x\in T). \]
\end{Definition}
\begin{Definition}{\href{https://proofwiki.org/wiki/Definition:Sequence\#Formal_Definition}{Sequence (Formal Definition)}}{}
    A \textbf{sequence} is a mapping whose domain is a subset of the set of natural numbers $ \N $. For the rest of these notes,
    $ \N=\Set{1,2,\ldots} $.
\end{Definition}
\begin{Definition}{\href{https://proofwiki.org/wiki/Definition:Real_Sequence}{Real Sequence}}{}
    A \textbf{real sequence} is a sequence (usually infinite) whose codomain is $ \R $.
    \tcblower{}
    \underline{Notation}: The notation for a real sequence is conventionally of the form
    $ \sequence{a_n} $, where it is understood that:
    \begin{itemize}
        \item the domain of $ \sequence{a_n} $ is $ \N $;
        \item the codomain of $ \sequence{a_n} $ is $ \R $.
    \end{itemize}
    We could think of a sequence as a function $ f\colon \N\to\R $, writing $ f(n)=a_n $.
\end{Definition}
Sequences can be defined explicitly (in terms of $ n $) or recursively (in terms of previous terms).
\begin{Example}{Real Sequence}{}
    \begin{itemize}
        \item The first few terms of the real sequence $ \sequence{\frac{1}{n+1}} $ are $ 1/2,1/3,1/4,1/5,\ldots $.
        \item The first few terms of the real sequence $ \sequence{\sqrt{n+2}}_{n\ge 2} $ are $ \sqrt{4},\sqrt{5},\sqrt{6},\ldots $.
        \item The first few terms of the real sequence $ \sequence{(-1)^n} $ are $ -1,+1,-1,+1,\ldots $.
    \end{itemize}
\end{Example}
\subsection{Recursively Defined Sequences}
\begin{Definition}{\href{https://proofwiki.org/wiki/Definition:Recursive_Sequence\#Definition}{Recursive Sequence}}{}
    A \textbf{recursive sequence} is a sequence where each term is defined from earlier terms in the sequence.
\end{Definition}
\begin{Example}{Recursive Sequence}{}
    \begin{itemize}
        \item $ a_1=1 $, $ a_{n+1}=\sqrt{1+a_n} $, so $ a_1=1 $, $ a_2=\sqrt{2} $, $ a_3=\sqrt{1+\sqrt{2}} $, and so on for $ n\in\N $.
        \item Fibonacci's sequence: $ a_1=1 $, $ a_2=1 $, $ a_{n+2}=a_{n+1}+a_n $ for $ n\in \N $, i.e.,
              $ 1,1,2,3,5,8,13,\ldots $.
    \end{itemize}
\end{Example}
\underline{Why study sequences?}
\begin{itemize}
    \item Lots of continuous processes can be modelled with discrete data, as we will see.
    \item We can use recursive sequences to approximate solutions to equations that can't be solved explicitly (Newton's Method).
    \item For another (ancient) application, see page 14 of the course notes about calculating square roots.
\end{itemize}
Our goal now will be to determine how to find the limit of a sequence, that is, find what the value of the terms of the sequence
are approaching (if it exists).
\subsection{Subsequences and Tails}
We may want to build new sequences out of old ones or only discuss what happens to a sequence eventually, that is, after a certain index.
\begin{Example}{}{}
    For $ \sequence{\frac{1}{n}} $, if we consider only the odd terms, we get $ 1,1/3,1/5 $, or the $ k\textsuperscript{th} $
    term is
    \[ \frac{1}{2k-1} \]
    for $ k\in\N $.
    This is called a subsequence.
\end{Example}
\begin{Definition}{\href{https://proofwiki.org/wiki/Definition:Strictly_Increasing/Sequence/Real_Sequence\#Definition}{Strictly Increasing (Sequence)}}{}
    Let $ \sequence{a_n} $ be a real sequence.\bigskip

    Then $ \sequence{a_n} $ is \textbf{strictly increasing} if and only if:
    \[ \forall n\in\N:a_n<a_{n+1}. \]
\end{Definition}
\begin{Definition}{\href{https://proofwiki.org/wiki/Definition:Subsequence}{Subsequence}}{}
    Let $ \sequence{a_n} $ be a sequence in a set $ S $.\smallskip

    Let $ \sequence{n_k} $ be a strictly increasing sequence in $ \N $.\bigskip

    Then the composition $ \sequence{a_{n_k}} $ is called a \textbf{subsequence} of $ \sequence{a_n} $.
\end{Definition}
\begin{Example}{Subsequence}{}
    One particular subsequence is $ \sequence{a_k,a_{k+1},a_{k+2}} $ for some $ k\in\N $.
    This is called the tail of $ \sequence{a_n} $ with cut-off $ k $.
\end{Example}
\subsection{Limits of Sequences}
We are going to see lots of different limits this term, but we will start with sequences.
\begin{Example}{}{}
    $ \sequence{\frac{1}{n}} $ seems like it converges to $ 0 $, or that $ 0 $ is the limit of the sequence.
    We saw this when we plotted the sequence. We will eventually want a formal definition, but let's start intuitively.
\end{Example}
\begin{Definition}{\href{https://proofwiki.org/wiki/Definition:Convergent_Sequence/Real_Numbers}{Convergent (Real) Sequence}}{conv_seq}
    Let $ \sequence{a_n} $ be a real sequence.\bigskip

    $ \sequence{a_n} $ \textbf{converges} to $ L\in\R$ if and only if:
    \[ \exists L\in\R:\forall \varepsilon\in\R_{>0}: \exists N\in \R_{>0}:\forall n\in\N:(n\ge N\implies \abs{a_n-L}<\varepsilon) \]
    In this case, we write $ \lim\limits_{{n} \to {\infty}}a_n=L $.
\end{Definition}
\begin{Remark}{}{}
    The negation of~\Cref{def:conv_seq} would be the following.\smallskip

    \[ \forall L\in\R:\exists \varepsilon\in\R_{>0}: \forall N\in \R_{>0}:\exists n\in\N:(n\ge N \land \abs{a_n-L}\ge \varepsilon) \]
    In this case, we write $ \lim\limits_{{n} \to {\infty}}a_n\ne L $.
\end{Remark}
\begin{Remark}{\href{https://proofwiki.org/wiki/Definition:Convergent_Sequence/Note_on_Domain_of_N}{Note on Domain of $ N $}}{}
    Some sources insist that $ N\in\N $, but this is not strictly necessary and can make proofs more cumbersome.
\end{Remark}
\begin{Example}{}{}
    Prove that $ \displaystyle \lim\limits_{{n} \to {\infty}}\frac{1}{n^2}=0 $.
    \tcblower{}
    \textbf{Proof}: Let $ \varepsilon>0 $ be arbitrary, and pick $ N>\frac{1}{\sqrt{\varepsilon}} $. If $ n\ge N $,
    we get
    \[ \abs*{\frac{1}{n^2}-0}=\frac{1}{n^2}\le \frac{1}{N^2}<\frac{1}{(1/\sqrt{\varepsilon})^2}=\frac{1}{1/\varepsilon}=\varepsilon \]
    as desired.
\end{Example}
\begin{Example}{}{}
    Prove that $ \displaystyle \lim\limits_{{n} \to {\infty}}\frac{n}{2n+3}=\frac{1}{2} $.
    \tcblower{}
    \textbf{Proof}: Let $ \varepsilon>0 $ be arbitrary, and pick $ N>\frac{1}{4}\bigl(\frac{3}{\varepsilon}-6\bigr) $.
    If $ n\ge N $, we get:
    \[ \abs{a_n-L}=\abs*{\frac{n}{2n+3}-\frac{1}{2}}=\frac{3}{4n+6}\le \frac{3}{4N+6}<\frac{3}{4\bigl(\frac{1}{4}(\frac{3}{\varepsilon}-6)\bigr)+6}=\varepsilon. \]
    \underline{Aside}: We want
    \[ \frac{3}{4n+6}<\varepsilon\iff \frac{3}{\varepsilon}<4n+6\iff \frac{3}{\varepsilon}-6<4n\iff \frac{1}{4}\biggl(\frac{3}{\varepsilon}-6\biggr)<n. \]
\end{Example}
\begin{Example}{}{}
    Prove that $ \displaystyle \lim\limits_{{n} \to {\infty}}\frac{n^2}{3n^2+7n}=\frac{1}{3} $.
    \tcblower{}
    \textbf{Proof}: Let $ \varepsilon>0 $ be arbitrary, and pick $ N>\frac{7}{9\varepsilon} $.
    If $ n\ge N $, we get:
    \[ \abs*{a_n-L}=\abs*{\frac{n^2}{3n^2+7n}-\frac{1}{3}}=\frac{7n}{9n^2+21n}\le \frac{7n}{9n^2}=\frac{7}{9n}\le \frac{7}{9(\frac{7}{9\varepsilon})}=\varepsilon. \]
    \underline{Aside}: We want
    \[ \frac{7}{9n}<\varepsilon\iff \frac{7}{9\varepsilon}<n. \]
\end{Example}
\begin{Remark}{Avoid Common Mistakes}{}
    \begin{itemize}
        \item Don't choose $ \varepsilon $! Let it be arbitrary.
        \item Never assume $ \abs{a_n-L}<\varepsilon $, make sure you only do work in an aside with that inequality since it is what you
              are proving.
        \item In practice, unless you are asked to, do not use this formal definition. We will now try to develop better methods for finding limits.
        \item One can think of~\Cref{def:conv_seq} as a nested logical statement, hence the order is (mostly) important.
    \end{itemize}
\end{Remark}
\subsection*{Equivalent Definitions of the Limit}
Note that in~\Cref{def:conv_seq}, $ \abs{a_n-L}<\varepsilon \iff a_n\in(L-\varepsilon,L+\varepsilon) $.
The collection of $ \sequence{a_n} $ for which
$ n\ge N $ is the tail of the sequence with cut-off $ N $. So, here's another definition.
\begin{Definition}{}{}
    $ \lim\limits_{{n} \to {\infty}}a_n=L \iff \forall \varepsilon\in\R_{>0} $ the interval
    $ (L-\varepsilon,L+\varepsilon) $ contains a tail of the sequence $ \sequence{a_n} $.
\end{Definition}
Let's push it further! Since the above is true for any $ \varepsilon>0 $, if we pick any
open interval $ (a,b) $ containing $ L $, then we can find a small enough $ \varepsilon>0 $
so that $ (L-\varepsilon,L+\varepsilon)\subseteq (a,b) $. Therefore, any interval containing
$ L $ also contains a tail of $ \sequence{a_n} $. Let's collect all of these alternate (but equivalent) definitions together.
\begin{Theorem}{}{}
    The following are equivalent:
    \begin{enumerate}[(1)]
        \item $ \lim\limits_{{n} \to {\infty}}a_n=L $.
        \item For any $ \varepsilon>0 $, $ (L-\varepsilon,L+\varepsilon) $ contains a tail of $ \sequence{a_n} $.
        \item For any $ \varepsilon>0 $, $ (L-\varepsilon,L+\varepsilon) $ contains all but finitely many terms of $ \sequence{a_n} $.
        \item Every interval $ (a,b) $ containing $ L $ contains a tail of $ \sequence{a_n} $.
        \item Every interval $ (a,b) $ containing $ L $ contains all but finitely many terms of $ \sequence{a_n} $.
    \end{enumerate}
    Clearly, changing finitely many terms of $ \sequence{a_n} $ does not affect the convergence or the limit.
\end{Theorem}
\begin{Example}{}{}
    Can a sequence have more than one limit?
    Consider $ \sequence{(-1)^n}=-1,1,-1,1,\ldots $, it equals to both $ 1 $ and $ -1 $ infinitely often. Could both $ 1 $ and $ -1 $
    be the limits? No! Let's prove $ -1 $ isn't a limit.
    \tcblower{}
    \textbf{Proof}: Consider the interval $ (-2,0) $. Clearly $ -1\in(-2,0) $, but
    this interval does not contain any of the infinitely many $ 1 $'s in the sequence. So, $ -1 $
    is not a limit by (5) above. A similar argument can be used with the interval $ (0,2) $ to show
    $ 1 $ is also not a limit. So, does $ \sequence{(-1)^n} $ have a limit at all? Let's prove it doesn't!
    Let $ \varepsilon=1/2 $, and supposed for a contradiction that the sequence converges to $ L\in\R $. That means
    the interval $ (L-1/2,L+1/2) $ must contain all but finitely many terms of the sequence, that is, but $ 1 $ and $ -1 $
    must lie in that interval. But the interval is only $ 1 $ unit long! So there is not $ L\in\R $ for which both $ 1 $ and $ -1 $
    lie inside $ (L-1/2,L+1/2) $. So, $ \sequence{(-1)^n} $ diverges.
\end{Example}
A similar argument can be used to prove limits are unique.
\begin{Theorem}{\href{https://proofwiki.org/wiki/Convergent_Real_Sequence_has_Unique_Limit}{Convergent Real Sequence has Unique Limit}}{}
    Let $ \sequence{a_n} $ be a sequence in $ \R $.\bigskip

    Then $ \sequence{a_n} $ can have at most one limit.
    \tcblower{}
    Click on the title's link for a quality proof.
\end{Theorem}
\begin{Remark}{A Remark on Possible Limits}{}
    If $ a_n\ge 0 $ for all $ n $, then $ \sequence{a_n} $ can't converge to a negative number! If it did, say to $ L<0 $,
    then the interval $ (L-1,0) $ would contain $ L $ but no terms of the sequence.
\end{Remark}
\begin{Theorem}{\href{https://proofwiki.org/wiki/Limit_of_Bounded_Convergent_Sequence_is_Bounded}{Limit of Bounded Convergent Sequence is Bounded}}{}
    Let $ \sequence{x_n} $, $ \sequence{a_n} $, and $ \sequence{b_n} $ be convergent real sequences.\smallskip

    Let $ \sequence{x_n} $, $ \sequence{a_n} $, and $ \sequence{b_n} $ converge to $ x,a,b\in\R $, respectively.\bigskip

    \[ \exists N\in\N:\forall n\in\N:(n\ge N\implies a_n\le x_n\le b_n)
        \implies a\le x\le b. \]
\end{Theorem}
\begin{itemize}
    \item Q\@: If $ x_n>0 $ for all $ n $ and $ \lim\limits_{{n} \to {\infty}}x_n=x $ is $ x>0 $?
    \item A\@: Not necessarily! Consider $ x_n=\frac{1}{n}>0 $, but $ x=0 $.
\end{itemize}
\subsection{Divergence to Infinity}
Consider $ \sequence{a_n}=\sequence{n} $. It is clear that the sequence is getting larger without bound, so $ \lim\limits_{{n} \to {\infty}}a_n $
does not exist, i.e., $ \sequence{a_n} $ diverges.
\begin{Definition}{\href{https://proofwiki.org/wiki/Definition:Unbounded_Divergent_Sequence\#Definition}{Divergent Sequence}}{}
    Let $ \sequence{a_n} $ be a real sequence.\bigskip

    $ \sequence{a_n} $ \textbf{diverges to} $ \infty $ if and only if:
    \[ \forall M\in\R_{>0}: \exists N\in \R_{>0}:\forall n\in\N:(n\ge N\implies a_n>M) \]
    In this case, we write $ \lim\limits_{{n} \to {\infty}}a_n=\infty $.\smallskip

    $ \sequence{a_n} $ \textbf{diverges to} $ -\infty $ if and only if:
    \[ \forall M\in\R_{<0}: \exists N\in \R_{>0}:\forall n\in\N:(n\ge N\implies a_n<M) \]
    In this case, we write $ \lim\limits_{{n} \to {\infty}}a_n=-\infty $.
\end{Definition}
\begin{Example}{}{}
    Show $ \lim\limits_{{n} \to {\infty}}(1-n)=-\infty $.
    \tcblower{}
    \textbf{Proof}: Let $ M<0 $ be arbitrary, and pick $ N>1-M $. If $ n\ge N $, we have
    \[ a_n=1-n\le 1-N<1-(1-M)=M. \]
    \underline{Aside}: We want $ 1-n<M\iff 1-M<n $.
\end{Example}