\section{Implicit Differentiation}
So far, we have examined derivatives of explicitly-defined functions (e.g., $ y=f(x) $), but what about implicitly-defined functions?
\begin{Example}{}{}
    If $ x^2+y^2=1 $, then this isn't even a function (as it does not pass the vertical line test). But, if we divide
    up the curve into positive and negative parts on the $ y $-axis then it can be a function. Then,
    we could find the derivative of each piece! The good news is that it doesn't matter if we break it up first!
    We can differentiate both sides of an implicit equation using the chain rule and solve for $ y $. We do need to assume
    that the equation defines an implicit function though, more on this later.
\end{Example}
\begin{Example}{}{}
    Find $ y' $ if $ 3x^3 y^3+x^2 y+13x=12 $.
    \tcblower{}
    \textbf{Solution}. We let $ y=y(x) $, take the derivative with respect to $ x $ on both sides, and then solve for $ y'(x) $:
    \begin{align*}
        \odv*{[3x^3 y(x)^3+x^2 y(x)+13x]}{x}                                           & =\odv*{[12]}{x}                               \\
        \odv*{[3x^3 y(x)^3]}{x}+\odv*{[x^2 y(x)]}{x}+\odv*{[13x]}{x}                   & =0                                            \\
        3\bigl( 3x^2y(x)^3+3x^3(3)y(x)^2 y'(x) \bigr) +\bigl(2xy(x)+x^2 y'(x)\bigr)+13 & =0                                            \\
        9 x^2 y(x)^3 + 9 x^3 y(x)^2 y'(x) + 2 x y(x)+ x^2 y'(x) + 13                   & =0                                            \\
        9 x^3 y(x)^2 y'(x)+x^2 y'(x)                                                   & =-13-9 x^2 y(x)^3-2 x y(x)                    \\
        y'(x)\bigl( 9 x^3+x^2\bigr)                                                    & =-13-9 x^2 y(x)^3-2 x y(x)                    \\
        y'(x)                                                                          & =\frac{-13-9 x^2 y(x)^3-2 x y(x)}{9 x^3+x^2}.
    \end{align*}
    Therefore,
    \[ y'=\frac{-13-9 x^2 y^3-2 x y}{9 x^3+x^2}. \]
\end{Example}
\begin{Remark}{}{}
    We can't always find the derivative of both sides of an equation unless we have a function!
\end{Remark}
\begin{Example}{}{}
    If $ x^2+y^2+1 $, we can show that $ y'=-\frac{x}{y} $, but for which $ (x,y)\in\R^2 $ is this valid for? None!
    $ x^2+y^2+1\ne 0 $ for \underline{any} $ (x,y)\in \R^2 $, so we differentiated nothing! Another example is
    if $ 2x=x $, we would differentiate to get $ 2=1 $ (nonsense). The issue is
    $ 2x=x $ is only true if $ x=0 $, so we can't compute the derivative as we can't take a limit! So be careful, use this power wisely!
\end{Example}
\subsection*{Logarithmic Differentiation}
We can use implicit differentiation to find the derivative of functions of the form
\[ y=(f(x))^{g(x)},\; f(x)>0 \]
by taking the ``$ \ln $'' of both sides.
\begin{Example}{}{}
    Let $ y=(\ln x)^{\sin x} $ for $ x>1 $. Find $ y' $.
    \tcblower{}
    \textbf{Solution}. Let $ y=y(x) $ so that $ y(x)=(\ln x)^{\sin x} $.
    Taking the logarithm (and then the derivative with respect to $ x $) on both sides gives
    \begin{align*}
        \ln y(x)             & =(\sin x) \ln(\ln x)                                                      \\
        \odv*{[\ln y(x)]}{x} & =\odv*{[(\sin x) \ln(\ln x)]}{x}                                          \\
        \frac{y'(x)}{y(x)}   & =(\cos x)\ln(\ln x)+\sin x \frac{1}{\ln x}\frac{1}{x}                     \\
        \implies y'(x)       & =y(x)\biggl[(\cos x)\ln(\ln x)+ \frac{\sin x}{x\ln x}\biggr]              \\
        \implies y'(x)       & =(\ln x)^{\sin x}\biggl[(\cos x)\ln(\ln x)+ \frac{\sin x}{x\ln x}\biggr].
    \end{align*}
\end{Example}
\begin{Example}{}{}
    Let $ y=x^{\arctan x} $. Find $ y' $.
    \tcblower{}
    \textbf{Solution}. Let $ y=y(x) $ so that $ y(x)=x^{\arctan x} $.
    Taking the logarithm (and then the derivative with respect to $ x $) on both sides gives
    \begin{align*}
        \ln y(x)             & =\arctan(x)\ln x                                                     \\
        \odv*{[\ln y(x)]}{x} & =\odv*{[\arctan(x)\ln x]}{x}                                         \\
        \frac{y'(x)}{y(x)}   & =\frac{1}{1+x^2}\ln x+\arctan(x) \frac{1}{x}                         \\
        \implies y'(x)       & =y(x)\biggl[\frac{\ln x}{1+x^2}+\frac{\arctan x}{x}\biggr]           \\
        \implies y'(x)       & =x^{\arctan x}\biggl[\frac{\ln x}{1+x^2}+\frac{\arctan x}{x}\biggr]. \\
    \end{align*}
\end{Example}
\section{Local Extrema}
\begin{Definition}{Local Maximum, Local Minimum}{}
    Let $ A\subseteq\R $ be open, let $ f\colon A\to\R $, and let $ a\in A $. Then $ f $
    has a \textbf{local maximum} at $ a $ if and only if
    \[ \forall x\in A: f(x)\le f(a). \]
    Similarly, we say $ f $ has a \textbf{local minimum} at $ a $ if and only if
    \[ \forall x\in A: f(x)\ge f(a). \]
\end{Definition}
We also present an equivalent definition.
\begin{Definition}{Local Maximum, Local Minimum}{}
    Let $ A\subseteq\R $ be open, let $ f\colon A\to\R $, and let $ a\in A $. Then $ f $
    has a \textbf{local maximum} at $ a $ if and only if
    \[ \exists \delta>0:\forall x\in A: \abs{x-a}\le \delta\implies f(x)\le f(a). \]
    Similarly, we say $ f $ has a \textbf{local minimum} at $ a $ if and only if
    \[ \exists \delta>0:\forall x\in A: \abs{x-a}\le \delta\implies f(x)\ge f(a). \]
\end{Definition}
\begin{Remark}{}{}
    Local maximum/minimum means max/min \underline{nearby} a point (i.e., in a small neighbourhood).
    Global max/min means max/min over the entire interval in question. So, global max/mins that occur inside the interval
    are also local max/mins.
\end{Remark}
How do we find local extrema? We will use the following theorem.
\subsection{The Local Extrema Theorem}
\begin{Theorem}{Fermat's Theorem/Local Extrema Theorem}{}
    Let $ A\subseteq \R $ be open, let $ f\colon A\to\R $, and let $ a\in A $. Suppose that $ f $
    is differentiable at $ a $ and that $ f $ has a local maximum or minimum value at $ a $. Then
    $ f'(a)=0 $.
    \tcblower{}
    \textbf{Proof}: We suppose that $ f $ has a local maximum value at $ a $ (the case that $ f $ has a local minimum
    value at $ a $ is similar). Choose $ \delta>0 $ so that $ \abs{x-a}\le \delta\implies f(x)\le f(a) $. For
    $ x\in A $ with $ a<x<a+\delta $, since $ x>a $ and $ f(x)\ge a $ we have $\frac{f(x)-f(a)}{x-a}\ge 0$,
    and so
    \[ f'(a)=\lim\limits_{{x} \to {a^+}}\frac{f(x)-f(a)}{x-a}\ge 0 \]
    by the Comparison Theorem. Similarly, for $ x\in A $ with $ a-\delta\le x<a $, since $ x<a $
    and $ f(x)\ge f(a) $ we have $ \frac{f(x)-f(a)}{x-a}\le 0 $, and so
    \[ f'(a)=\lim\limits_{{x} \to {a^-}}\frac{f(x)-f(a)}{x-a}\le 0. \]
\end{Theorem}
\begin{itemize}
    \item Q\@: Is the converse true?
    \item A\@: No! $ f(x)=x^3 $ has a critical point at $ x=0 $, but $ 0 $ is neither a local max nor a local min.
    \item Q\@: If $ c $ is a local max/min, then is $ f'(c)=0 $?
    \item A\@: No! $ f(x)=\abs{x} $ has a local min at $ x=0 $, but $ f'(0) $ does not exist.
\end{itemize}
\subsection*{Finding Global Extrema}
We just saw that if we want to find a local extrema, we should look at points where $ f'=0 $ or $ f' $ does not exist.
Let's give a name to points like this.
\begin{Definition}{Critical Point}{}
    A point $c$ in the domain of a function $f$ is called a \textbf{critical point} for $f$ if either
    $ f'(c)=0 $ or $ f'(c) $ does not exist.
\end{Definition}
Now, the EVT guarantees a continuous function
has a global max/min on a closed interval. Either these are at the endpoints or they are inside, and therefore local max/mins,
and hence critical points!

So here is the algorithm for finding the global max/min
of a continuous function $ f(x) $ on $ [a,b] $.
\begin{enumerate}[(i)]
    \item Find all critical points of $ f $ in $ [a,b] $.
    \item Evaluate $ f(a) $, $ f(b) $, and $ f(c) $, where $ c $ are all the critical points.
    \item The largest value tells you where the global maximum is, and the smallest tells you what the global minimum is.
\end{enumerate}
\begin{Example}{}{}
    Find the global maximum and minimum for $ f(x)=x^3-3x+2 $ on $ [-3,3] $.
    \tcblower{}
    \textbf{Solution}. $ f'(x)=3x^2-3=3(x-1)(x+1)=0 $ if $ x=\pm 1 $.
    These critical points are both inside $ [-3,3] $. Now, we check
    $ f(-3)=-16 $, $ f(-1)=4 $, $ f(1)=0 $, $ f(3)=20 $. Therefore, the global maximum is at $ (3,20) $ and
    the global minimum is at $ (-3,-16) $.
\end{Example}
\begin{Example}{}{}
    Find the global maximum and minimum for $ f(x)=1/x $ on $ [3,7] $.
    \tcblower{}
    \textbf{Solution}. $ f'(x)=-1/x^2 $ and $ f'(x) $ does not exist if $ x=0 $.
    However, $ 0 $ is not a critical point of $ f $ since $ 0\notin [3,7] $.
    So, $ f $ has no critical points. Now, $ f(3)=1/3 $ and $ f(7)=1/7 $, so
    the global maximum is at $ (3,1/3) $ and the global minimum is at $ (7,1/7) $.
\end{Example}
We will re-visit this when we discuss curve sketching.

\chapter{The Mean Value Theorem}
\section{The Mean Value Theorem}
\section{Applications of the Mean Value Theorem}
\subsection{Antiderivatives}
