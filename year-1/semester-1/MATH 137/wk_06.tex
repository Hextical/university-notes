\section{The Derivative Function}
\begin{Definition}{The Derivative Function}{}
    We say that $ f $ is \textbf{differentiable} on an interval $ I $ if $ f'(a) $ exists
    for each $ a\in I $. In this case, we define the \textbf{derivative function} as
    \[ f'(x)=\lim\limits_{{h} \to {0}}\frac{f(x+h)-f(x)}{h},\; x\in I. \]
    Alternative (Leibniz) notation:
    \[ f'(x)=\odv{f}{x}=\odv*{(f)}{x}, \]
    where ``$\odv{}{x}$'' is called a \textbf{differential operator}.

    If $ y=f(x) $, write $ \odv{y}{x} $. For $ f'(a) $, write $ \odv{f}{x}_{x=a} $.
\end{Definition}
Let's look at some examples!
\begin{Example}{}{}
    For $ f(x)=7 $, find $ f'(x) $ for $ x\in\R $.
    \tcblower{}
    \textbf{Solution}.
    \[ f'(x)=\lim\limits_{{h} \to {0}}\frac{f(x+h)-f(x)}{h}=\lim\limits_{{h} \to {0}}\frac{7-7}{h}=0. \]
    Therefore, $ f'(x)=0 $ for all $ x\in\R $.
\end{Example}
\begin{Example}{}{}
    Find the equation of the tangent line to $ f(x)=x^2+3x+2 $ at $ x=2 $.
    \tcblower{}
    \textbf{Solution}. The tangent line passes through $ (a,f(a))=(2,f(2))=(2,12) $ since
    $ f(2)=2^2+3(2)+2=12 $. Next,
    \begin{align*}
        f'(x)
         & =\lim\limits_{{h} \to {0}}\frac{f(x+h)-f(x)}{h}               \\
         & =\lim\limits_{{h} \to {0}}\frac{(x+h)^2+3(x+h)+2-x^2-3x-2}{h} \\
         & =\lim\limits_{{h} \to {0}}\frac{x^2+2xh+h^2+3x+3h-x^2-3x}{h}  \\
         & =\lim\limits_{{h} \to {0}}\frac{2xh+h^2+3h}{h}                \\
         & =\lim\limits_{{h} \to {0}}(2x+h+3)                            \\
         & =2x+3,
    \end{align*}
    which gives $ f'(2)=2(2)=3=7 $. Therefore, the tangent line to $ f $ at $ x=2 $ is
    \[ y=f(2)+f'(2)(x-2)=12+7(x-2)=12+7x-14=7x-2. \]
\end{Example}
\begin{Remark}{}{}
    \begin{itemize}
        \item Much faster than computing $ f'(a) $ each time!
        \item We will soon learn ways to find $ f'(x) $ \underline{much faster}, but if asked to use the
              \underline{definition}, then you must use the formula
              \[ f'(x)=\lim\limits_{{h} \to {0}}\frac{f(x+h)-f(x)}{h}. \]
    \end{itemize}
\end{Remark}
\begin{Example}{}{}
    Using the definition, find $ f'(x) $ where
    \begin{enumerate}[(1)]
        \item $ f(x)=x $;
        \item $ f(x)=x^2 $;
        \item $ f(x)=x^3 $;
        \item $ f(x)=\sqrt{x} $.
    \end{enumerate}
    \tcblower{}
    \textbf{Solution}.
    \begin{enumerate}[(1)]
        \item $ \begin{aligned}[t]
                      f'(x)
                       & =\lim\limits_{{h} \to {0}}\frac{f(x+h)-f(x)}{h} \\
                       & =\lim\limits_{{h} \to {0}}\frac{x+h-x}{h}       \\
                       & =\lim\limits_{{h} \to {0}}\frac{h}{h}           \\
                       & =1.
                  \end{aligned} $
        \item $ \begin{aligned}[t]
                      f'(x)
                       & =\lim\limits_{{h} \to {0}}\frac{f(x+h)-f(x)}{h}     \\
                       & =\lim\limits_{{h} \to {0}}\frac{(x+h)^2-x^2}{h}     \\
                       & =\lim\limits_{{h} \to {0}}\frac{x^2+2xh+h^2-x^2}{h} \\
                       & =\lim\limits_{{h} \to {0}}\frac{2xh+h^2}{h}         \\
                       & =\lim\limits_{{h} \to {0}}(2x+h)                    \\
                       & =2x.
                  \end{aligned} $
        \item $ \begin{aligned}[t]
                      f'(x)
                       & =\lim\limits_{{h} \to {0}}\frac{f(x+h)-f(x)}{h}             \\
                       & =\lim\limits_{{h} \to {0}}\frac{(x+h)^3-x^3}{h}             \\
                       & =\lim\limits_{{h} \to {0}}\frac{x^3+3x^2h+3xh^2+h^3-x^3}{h} \\
                       & =\lim\limits_{{h} \to {0}}\frac{3x^2h+3xh^2+h^3}{h}         \\
                       & =\lim\limits_{{h} \to {0}}(3x^2+3xh+h^2)                    \\
                       & =3x^2.
                  \end{aligned} $
        \item $ \begin{aligned}[t]
                      f'(x)
                       & =\lim\limits_{{h} \to {0}}\frac{f(x+h)-f(x)}{h}                                                              \\
                       & =\lim\limits_{{h} \to {0}}\frac{\sqrt{x+h}-\sqrt{x}}{h}\cdot \frac{\sqrt{x+h}+\sqrt{x}}{\sqrt{x+h}+\sqrt{x}} \\
                       & =\lim\limits_{{h} \to {0}}\frac{x+h-x}{h(\sqrt{x+h}+\sqrt{x})}                                               \\
                       & =\lim\limits_{{h} \to {0}}\frac{h}{h(\sqrt{x+h}+\sqrt{x})}                                                   \\
                       & =\lim\limits_{{h} \to {0}}\frac{1}{\sqrt{x+h}+\sqrt{x}}                                                      \\
                       & =\frac{1}{2\sqrt{x}}.
                  \end{aligned} $
    \end{enumerate}
\end{Example}
\subsection*{Higher-Order Derivatives}
\begin{Definition}{}{}
    If $ f $ is differentiable with derivative $ f' $ and $ f' $ is also
    differentiable, then we call $ \odv*{(f')}{x} $ the \textbf{second derivative} of $ f $,
    denoted $ f''(x) $ or $ f^{(2)}(x) $, or $ \odv[order=2]{f}{x} $.

    In general, $ f^{(n+1)}(x)=\odv*{(f^{(n)}(x))}{x} $, where $ f^{(n)} $ is the $ n\textsuperscript{th} $ derivative.
\end{Definition}
\begin{Exercise}{}{}
    Prove the following with the limit definition, where $ f(x)=x^4 $.
    \begin{itemize}
        \item $ f'(x)=4x^3 $.
        \item $ f''(x)=12x^2 $.
        \item $ f'''(x)=24x $.
        \item $ f^{(4)}=24 $.
        \item $ f^{(5)}=0 $.
    \end{itemize}
\end{Exercise}
Note that using the limit definition is very inefficient (not to mention awful and ugly). So, let's develop some rules
to help us calculate derivatives more quickly!
\section{Derivatives of Elementary Functions}
Now that we know the definition of the derivative, let's work on finding
derivatives of elementary functions to speed up the process.
\begin{itemize}
    \item \textbf{Constants}: If $ f(x)=c $ where $ c\in\R $, then $ f'(x)=0 $.
    \item \textbf{Lines}: If $ f(x)=mx+b $ where $ m,b\in\R $, then $ f'(x)=m $.
    \item \textbf{Quadratics}: If $ f(x)=ax^2+bx+c $ where $ a,b,c\in\R $ and $ a\ne 0 $, then $ f'(x)=2ax+b $.
\end{itemize}
\subsection{The Derivative of $ \sin x $ and $ \cos x $}
First, we need to prove a different claim:
\[ \displaystyle \lim\limits_{{x} \to {0}}\frac{\cos x-1}{x}=0. \]
\begin{align*}
    \lim\limits_{{x} \to {0}} \frac{\cos x-1}{x}\cdot \frac{\cos x+1}{\cos x+1}
     & =\lim\limits_{{x} \to {0}}\frac{\cos^2 x-1}{x(\cos x+1)}                 \\
     & =\lim\limits_{{x} \to {0}}\frac{-\sin^2 x}{x(\cos(x+1))}                 \\
     & =\lim\limits_{{x} \to {0}}\frac{\sin x}{x}\cdot \frac{-\sin x}{\cos x+1} \\
     & =1\cdot 0                                                                \\
     & =0,
\end{align*}
using the fundamental trigonometry limit. Now, we can compute $ (\sin x)' $.
\begin{align*}
    (\sin x)'
     & =\lim\limits_{{h} \to {0}}\frac{\sin(x+h)-\sin x}{h}                                      \\
     & =\lim\limits_{{h} \to {0}}\frac{\sin x\cos(h)+\cos x\sin(h)-\sin x}{h}                    \\
     & =\lim\limits_{{h} \to {0}}\frac{\sin(h)}{h}\cos x+\biggl(\frac{\cos(h)-1}{h}\biggr)\sin x \\
     & =1\cdot \cos x+0\cdot \sin x                                                              \\
     & =\cos x.
\end{align*}
\begin{Exercise}{}{}
    Show that $ (\cos x)'=-\sin x $.
\end{Exercise}
\subsection{The Derivative of $ e^x $}
First, what is the number $ e $? There are lots of ways to define it, for example:
$ \lim\limits_{{x} \to {\infty}}(1+\frac{1}{x})^{x}=e $ or $ \sum_{n=0}^{\infty}\frac{1}{n!}=e $.
But for us, we will define $ e $ to be the unique number $ a\in\R $ such that the tangent line to $ a^x $
has slope $ 1 $ at $ x=0 $. That is,
\[ \lim\limits_{{h} \to {0}}\frac{e^h-e^0}{h}=1\implies \lim\limits_{{h} \to {0}}\frac{e^h-1}{h}=1. \]
So, we get $ (e^x)'=\lim\limits_{{h} \to {0}}\frac{e^{x+h}-e^x}{h}=\lim\limits_{{h} \to {0}}e^x(\frac{e^h-1}{h})=e^x $.
So, $ (e^x)'=e^x $.

\setcounter{section}{6}
\section{Arithmetic Rules for Differentiation}
Now that we know how to find the derivatives of certain basic functions,
let us look at some rules that tell us how to differentiate combinations
of functions.
\begin{Theorem}{Arithmetic Rules for Differentiation}{}
    Suppose $ f $ and $ g $ are differentiable at $ x=a $.
    \begin{enumerate}[(1)]
        \item \textbf{Constant Multiple Rule}. Let $ h(x)=cf(x) $. Then $ h $ is differentiable at $ x=a $ and
              \[ h'(a)=c f'(a). \]
        \item \textbf{Sum Rule}. Let $ h(x)=f(x)+g(x) $. Then $ h $ is differentiable at $ x=a $ and
              \[ h'(a)=f'(a)+g'(a). \]
        \item \textbf{Product Rule}. Let $ h(x)=f(x)g(x) $. Then $ h $ is differentiable at $ x=a $ and
              \[ h'(a)=f'(a)g(a)+f(a)g'(a). \]
        \item \textbf{Reciprocal Rule}. Let $ h(x)=\frac{1}{g(x)} $. If $ g(a)\ne 0 $, then $ h $ is differentiable at $ x=a $ and
              \[ h'(a)=-\frac{g'(a)}{[g(a)]^2}. \]
        \item \textbf{Quotient Rule}: Let $ h(x)=\frac{f(x)}{g(x)} $. If $ g(a)\ne 0 $, then $ h $ is differentiable at $ x=a $ and
              \[ h'(a)=\frac{f'(a)g(a)-f(a)g'(a)}{[g(a)]^2}. \]
    \end{enumerate}
    \tcblower{}
    \textbf{Proof}:
    \begin{enumerate}[(1)]
        \item Easy exercise.
        \item Easy exercise.
        \item $ \begin{aligned}[t]
                      (fg)'(a)
                       & =\lim\limits_{{h} \to {0}}\frac{f(a+h)g(a+h)-f(a)g(a)}{h}                        \\
                       & =\lim\limits_{{h} \to {0}}\frac{f(a+h)g(a+h)-f(a+h)g(a)+f(a+h)g(a)-f(a)g(a)}{h}  \\
                       & =\lim\limits_{{h} \to {0}}f(a+h)\frac{g(a+h)-g(a)}{h}+g(a) \frac{f(a+h)-f(a)}{h} \\
                       & =f(a)g'(a)+g(a)f'(a).
                  \end{aligned} $
        \item $ \begin{aligned}[t]
                      \biggl(\frac{1}{f}\biggr)'(a)
                       & =\lim\limits_{{h} \to {0}}\frac{\frac{1}{f(a+h)}-\frac{1}{f(a)}}{h}    \\
                       & =\lim\limits_{{h} \to {0}}\frac{f(a)-f(a+h)}{h f(a+h)f(a)}             \\
                       & =\lim\limits_{{h} \to {0}}\frac{-(f(a+h)-f(a))}{h}\frac{1}{f(a+h)f(a)} \\
                       & =\frac{-f'(a)}{[f(a)]^2}.
                  \end{aligned} $
        \item We can combine the product and reciprocal rules!
              $ \begin{aligned}[t]
                      \biggl(\frac{f}{g}\biggr)'(a)
                       & =\biggl(f \frac{1}{g}\biggr)'(a)                       \\
                       & =f'(a)\frac{1}{g(a)}+f(a)\biggl(\frac{1}{g}\biggr)'(a) \\
                       & =\frac{f'(a)}{g(a)}-\frac{f(a)g'(a)}{[g(a)]^2}         \\
                       & =\frac{f'(a)g(a)-f(a)g'(a)}{[g(a)]^2}.
                  \end{aligned} $
    \end{enumerate}
\end{Theorem}
\begin{Theorem}{The Power Rule for Differentiation}{}
    Assume that $ \alpha\in\R $, $ \alpha\ne 0 $, and $ f(x)=x^{\alpha} $. Then $ f $ is differentiable and
    \[ f'(a)=\alpha x^{\alpha-1}, \]
    where $ x^{\alpha-1} $ is defined.
\end{Theorem}
In general, the proof is difficult. If $ \alpha\in\N $, then it is a simple application of the Binomial Theorem.
For $ \alpha\in\mathbf{Q} $, it is possible with more tools (chain rule and inverse function theorem). But for general
$ \alpha\in\R $, we would need more tools, and it outside the scope of this course. So, we omit the proof. Let's look at some examples!
\begin{Example}{}{}
    \begin{enumerate}[(1)]
        \item $ f(x)=x^2\sin x $.
              \[ f'(x)=(x^2)'\sin x+x^2(\sin x)'=2x\sin x+x^2\cos x. \]
        \item $ f(x)=\frac{x^4-1}{x-7} $.
              \[ f'(x)=\frac{(x-7)(x^4+1)'-(x^4+1)(x-7)'}{(x-7)^2}=\frac{(x-7)(4x^3)-(x^4+1)(1)}{(x-7)^2}. \]
        \item $ f(x)=\sec x=\frac{1}{\cos x} $.
              \[ f'(x)=\frac{-(\cos x)'}{\cos^2 x}=\frac{\sin x}{\cos^2 x}=\frac{\sin x}{\cos x}\frac{1}{\cos x}=\tan x\sec x. \]
        \item $ f(x)=e^x\cos x $.
              \[ f'(x)=e^x\cos x-e^x\sin x. \]
        \item $ f(x)=3x^5 $.
              \[ f'(x)=15x^4,\; f''(x)=60x^3,\; f^{(3)}(x)=180x^2,\; f^{(4)}(x)=360x,\; f^{(5)}(x)=360,\; f^{(\ge 6)}(x)=0. \]
    \end{enumerate}
\end{Example}
\section{The Chain Rule}
\begin{Theorem}{Chain Rule}{}
    Let $ A,B\subseteq \R $ be open, let $ f\colon A\to \R $, let $ g\colon B\to \R $, and let
    $ h=g\circ f\colon C\to \R $, where $ C=A\cap f^{-1}(B) $. Let $ a\in C $ and let $ b=f(a)\in B $. Suppose
    that $ f $ is differentiable at $ a $ and $ g $ is differentiable at $ b $. Then $ h $ is differentiable at $ a $ with
    \[ h'(a)=g'(f(a))f'(a). \]
    In Leibniz notation, if $ z=g(y) $ and $ y=f(x) $, then
    \[ \odv{z}{x}=\odv{z}{y}\odv{y}{x}. \]
    \tcblower{}
    The proof is quite involved, for a geometric argument see the course notes.
\end{Theorem}
\begin{Corollary}{Generalized Power Rule}{}
    If $ g(x)=f(x)^\alpha $ for $ \alpha\in\R\setminus \{0\} $, then
    \[ g'(x)=\alpha f(x)^{\alpha-1}f'(x). \]
\end{Corollary}
\begin{Example}{}{}
    Find $ f'(x) $.
    \begin{enumerate}[(1)]
        \item $ f(x)=(3x^2+2x+7)^{19} $.
        \item $ f(x)=\sin(e^x+x^e) $.
        \item $ f(x)=e^{\sin(x^2)} $.
    \end{enumerate}
    \tcblower{}
    \textbf{Solution}.
    \begin{enumerate}[(1)]
        \item $ f'(x)=38(3x+1)(3x^2+2x+7)^{18} $.
        \item $ f'(x)=\cos(e^x+x^e)(e^x+e xe^{e-1}) $.
        \item $ f'(x)=e^{\sin(x^2)}(\sin(x^2))'=e^{\sin(x^2)}\cos(x^2)(x^2)'=e^{\sin(x^2)}\cos(x^2)(2x) $.
    \end{enumerate}
\end{Example}
Also, with the chain rule and the derivative of $ e^x $, we can get the derivative of $ a^x $ for $ a>0 $.
\[ a^x=e^{x\ln(a)}\implies (a^x)'=(e^{x\ln(a)})'=e^{x\ln(a)}(x\ln(a))'=a^x\ln(a).  \]
\begin{Example}{}{}
    $ f(x)=2^{3x}+5^{\cos x} $. $ f'(x)=2^{3x}\ln(2)(3)+5^{\cos x}\ln(5)(-\sin x) $.
\end{Example}
\section{Derivatives of Other Trigonometric Functions}
So far, we've seen:
\begin{align*}
    (\sin x)' & =\cos x        \\
    (\cos x)' & =-\sin x       \\
    (\sec x)' & =\sec x\tan x.
\end{align*}
\begin{Example}{}{}
    \begin{align*}
        (\tan x)'
         & =\biggl(\frac{\sin x}{\cos x}\biggr)'
         & =\frac{\cos x(\sin x)'-\sin x(\cos x)'}{\cos^2 x} \\
         & =\frac{\cos^2 x+\sin^2 x}{\cos^2 x}               \\
         & =\frac{1}{\cos^2 x}                               \\
         & =\sec^2 x .
    \end{align*}
\end{Example}
\begin{Exercise}{}{}
    Prove that $ (\cot x)'=-\csc^2 x $ and $ (\csc x)'=-\csc x\cot x $.
\end{Exercise}
Recap:
\[ \begin{array}{cc}
        f(x)   & f'(x)         \\
        \midrule
        \sin x & \cos x        \\
        \cos x & -\sin x       \\
        \tan x & \sec^2 x      \\
        \cot x & -\csc^2 x     \\
        \sec x & \sec x\tan x  \\
        \csc x & -\csc x\cot x \\
    \end{array} \]
