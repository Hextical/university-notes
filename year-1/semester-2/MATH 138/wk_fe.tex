\setcounter{section}{8}
\section{Binomial Series}
Let's fine one more series: Binomial series!

We know the Binomial Theorem for $ (1+x)^k $ where
$ n\in\mathbb{N} $:
\[ (1+x)^n=\sum\limits_{k=0}^{n} \binom{n}{k}x^k \]
where $ \displaystyle \binom{n}{k}=\frac{n!}{k!(n-k)!} $.

The question is: can we extend this to
$ (1+x)^n $ for any $ n\in\mathbb{R} $? Yes! We can find
its Maclaurin series!
\begin{itemize}
    \item $ f(x)=(1+x)^n\implies f(0)=1 $
    \item $ f^\prime(x)=n(1+x)^{n-1}\implies f^\prime(0)=n $
    \item $ f^{\prime\prime}(n)=n(n-1)(1+x)^{n-2}\implies f^{\prime\prime}(0)=n(n-1) $
    \item $ \vdots $
    \item $ f^{(k)}(x)=n(n-1)\cdots\left[ n-(k-1) \right](1+x)^{n-k}
              \implies f^{(k)}(0)=n(n-1)\cdots(n-k+1) $
\end{itemize}
So, we get
\[ \sum\limits_{k=0}^{\infty}\frac{n(n-1)\cdots(n-k+1)}{k!} x^k  \]
for the Maclaurin series.

First, let's determine the radius of convergence, for $ n\neq 0,1,2,\ldots $:
Ratio Test:
\[ \lim\limits_{{k} \to {\infty}}
    \abs*{\left( \frac{n(n-1)\cdots(n-k+1)(n-k)x^{k+1}}{(k+1)!}  \right)
    \left( \frac{k!}{n(n-1)\cdots(n-k+1)x^k}  \right)}
    =\lim\limits_{{k} \to {\infty}} \abs*{\frac{n-k}{k+1}}\abs{x}=\abs{x} \]
Need $ \abs{x}<1 $, so $ R=1 $, and the open interval is
$ \interval[open]{-1}{1} $.

What about endpoint convergence? Here is the answer,
but you won't be expected to know this:

Interval of convergence:
\begin{itemize}
    \item $ \interval{-1}{1} $ if $ n>0 $, $ n\notin\mathbb{N} $
    \item $ \interval[open left]{-1}{1} $ if $ -1<n< 0$
    \item $ \interval[open]{-1}{-1} $ if $ n\leqslant -1 $
    \item $ \mathbb{R} $ if $ n=0,1,2,\cdots $
\end{itemize}
Notation:
\[ \binom{n}{k}=
    \frac{n(n-1)\cdots(n-k+1)}{k!} \]
called the \textbf{Binomial coefficients} with
$ k $-terms in the numerator. Keep in mind that $ \binom{n}{0}=1 $.

The bigger question is: does $ (1+x)^n $ equal to its Maclaurin
series on $ \interval[open]{-1}{1} $? The answer is yes! Let's prove it.
We could try to use the Convergence Theorem, but instead we will prove it directly!

First, we claim that
\[ \binom{n}{k+1}(k+1)+\binom{n}{k}k=\binom{n}{k}n \]
for $ k\geqslant 1 $.

\underline{Proof}:
\begin{align*}
    \binom{n}{k+1}(k+1)+\binom{n}{k}k
     & =\frac{n(n-1)\cdots(n-k+1)(n-k)}{(k+1)!}(k+1)
    +\frac{n(n-1)\cdots(n-k+1)}{k!}(k)               \\
     & =\frac{n(n-1)\cdots(n-k+1)}{n!}(n-k)
    +\frac{n(n-1)\cdots(n-k+1)}{k!}(k)               \\
     & =\binom{n}{k}(n-k)+\binom{n}{k}k              \\
     & =\binom{n}{k}(n-k+k)                          \\
     & =\binom{n}{k}n
\end{align*}

Next, let $ f(x)=\sum\limits_{k=0}^{\infty} \binom{n}{k}x^k $. We claim that
\[ f^\prime(x)+xf^\prime(x)=nf(x) \]
for all $ x\in\interval[open]{-1}{1} $.

\underline{Proof}:
\begin{align*}
    f^\prime(x)+xf^\prime(x)
     & =\sum\limits_{k=1}^{\infty} \binom{n}{k}k x^{k-1}+\sum\limits_{k=1}^{\infty}\binom{n}{k}
    k x^k                                                                                       \\
     & =\binom{n}{1}+\sum\limits_{k=2}^{\infty}
    \binom{n}{k}k x^{k-1}+\sum\limits_{k=1}^{\infty} \binom{n}{k}k x^k                          \\
     & =\binom{n}{1}+
    \sum\limits_{k=1}^{\infty} \binom{n}{k+1}(k+1)x^k+
    \sum\limits_{k=1}^{\infty} \binom{n}{k}k x^k                                                \\
     & =\binom{n}{1}+\sum\limits_{k=1}^{\infty}
    \left[ \binom{n}{k+1}(k+1)+\binom{n}{k}k \right]x^n                                         \\
     & =\binom{n}{1}+\sum\limits_{k=1}^{\infty} \binom{n}{k}n x^k                               \\
     & =n+\sum\limits_{k=1}^{\infty}\binom{n}{k}n x^k                                           \\
     & =n\left[ 1+\sum\limits_{k=1}^{\infty} \binom{n}{k}x^k \right]                            \\
     & =n \sum\limits_{k=0}^{\infty} \binom{n}{k}x^k                                            \\
     & =n f(x)
\end{align*}

Finally, let $ g(x)=\dfrac{f(x)}{(1+x)^n} $. Let's show $ g^\prime(x)=0 $
for $ x\in\interval[open]{-1}{1} $:
\begin{align*}
    g^\prime(x)
     & =\frac{f^\prime(x)(1+x)^n-f(x)n(1+x)^{n-1}}{(1+x)^{2n}}                                       \\
     & =\frac{f^\prime(x)(1+x)^n-(1+x)f^\prime(x)(1+x)^{n-1}}{(1+x)^{2n}} & \text{by previous claim} \\
     & =\frac{f^\prime(x)(1+x)^{n}-f^\prime(x)(1+x)^n}{(1+x)^{2n}}                                   \\
     & =0
\end{align*}
So, $ g^\prime(x)=0 $ for all $ x\in\interval[open]{-1}{1} $, which means
$ g $ is constant on $ \interval[open]{-1}{1} $.

Since $ f(0)=1 $, we get $ g(0)=\sfrac{1}{1} =1 $, so $ g(x)=1 $
for all $ x\in\interval[open]{-1}{1} $. This implies
$ f(x)=(1+x)^n $ for $ x\in\interval[open]{-1}{1} $. We have finally proven:

\begin{Theorem}{Generalized Binomial Theorem}{}
    Let $ n\in\mathbb{R} $, then for all $ x\in\interval[open]{-1}{1} $:
    \[ (1+x)^n=
        \sum\limits_{k=0}^{\infty} \binom{n}{k}x^k \]
    where
    \[ \binom{n}{k}=\frac{n(n-1)\cdots(n-k+1)}{k!} \]
    and $ \binom{n}{0}=1 $.
\end{Theorem}

\begin{Example}{}{}
    Find the Maclaurin series for $ \arcsin(x) $.

    \textbf{Solution.}

    \underline{Step 1}: Find the Maclaurin series for $ (1+x)^{-\sfrac{1}{2}} $.
    \[ (1+x)^{-\sfrac{1}{2}}
        =\sum\limits_{k=0}^{\infty} \frac{\left( -\sfrac{1}{2} \right)\left( -\sfrac{3}{2}  \right)
            \left( -\sfrac{5}{2} \right)\cdots\left( -\sfrac{1}{2} -k +1\right)}{k!}x^k
        =\sum\limits_{k=0}^{\infty} \frac{(-1)^k(1)(3)(5)\cdots(2k-1)}{2^k (k!)}x^k   \]
    for $ x\in\interval[open]{-1}{1} $.

    \underline{Step 2}: Find the Maclaurin series for $ (1-x^2)^{-\sfrac{1}{2} } $.
    \[ (1-x^2)^{-\sfrac{1}{2}}
        =\left[ 1+(-x^2) \right]^{-\sfrac{1}{2}}
        =\sum\limits_{k=0}^{\infty} \frac{(-1)^k(1)(3)(5)\cdots(2k-1)}{2^k (k!)}(-x^2)^k
        =\sum\limits_{k=0}^{\infty} \frac{(1)(3)(5)\cdots(2k-1)}{2^k}x^{2k}  \]
    for $ \abs{-x^2}<1\implies\abs{x}<1 $ with $ x\in\interval[open]{-1}{1} $.

    \underline{Step 3}: Integrate!
    \[ \arcsin(x)
        =\sum\limits_{k=0}^{\infty}\frac{(1)(3)(5)\cdots(2k-1)x^{2k+1}}{2^k(k!)(2k+1)}   \]
    for $ x\in\interval[open]{-1}{1} $ with $ C=0 $ since $ \arcsin(0)=0 $.
\end{Example}

\section{Additional Examples and Applications of Taylor Series}
The applications we will examine are:
\begin{enumerate}[label=(\Roman*)]
    \item Finding sums
    \item Evaluating limits
    \item Evaluating and approximating integrals
\end{enumerate}
\underline{Recap of Known Series}:
\begin{itemize}
    \item $ \displaystyle \frac{1}{1-x} =\sum\limits_{n=0}^{\infty} x^n $ ($ R=0 $)
    \item $ \displaystyle e^x=\sum\limits_{n=0}^{\infty} \frac{x^n}{n!}  $ ($ R=\infty $)
    \item $ \displaystyle \sin(x)=\sum\limits_{n=0}^\infty \frac{(-1)^n x^{2n+1}}{(2n+1)!} $
          ($ R=\infty $)
    \item $ \displaystyle \cos(x)=\sum\limits_{n=0}^\infty \frac{(-1)^n x^{2n}}{(2n)!} $
          ($ R=\infty $)
    \item $ \displaystyle (1+x)^n=\sum\limits_{k=0}^{\infty}\binom{n}{k}x^k
              =\sum\limits_{k=0}^{\infty} \frac{n(n-1)\cdots(n-k+1)}{k!} x^k  $ ($ R=1 $)
          ($ R=\infty $)
\end{itemize}


\subsection*{Finding Sums}
Given a series, we may be able to manipulate
it into one of the above series and find the sum that way. Alternatively, we could
manipulate a known series into the given series!

\begin{Example}{}{}
    \begin{enumerate}[label=(\roman*)]
        \item Find the sum of $ \displaystyle \sum\limits_{n=0}^{\infty} \left( \frac{n+1}{n!} \right)x^n =S(x)$.

              \textbf{Solution.} This is almost $ e^x $, but it has an extra ``$ n+1 $'' Let's integrate!
              \[ \int S(x)\odif{x} =
                  \sum\limits_{n=0}^{\infty} \left( \frac{x^{n+1}}{n!}  \right) +C
                  =x \sum\limits_{n=0}^{\infty} \frac{x^n}{n!} +C
                  =x e^x+C \]
              So $ S(x)=(x+e^x+C)^\prime=e^x+xe^x $.

        \item $ \displaystyle \sum\limits_{n=0}^{\infty} \left[ \frac{(-1)^n x^{2n+1}}{2n+1} \right]+4
                  =S_2(x) $.

              \textbf{Solution.} Differentiate:
              \[ S_2^\prime(x)=\sum\limits_{n=0}^{\infty} (-1)^n x^{2n}=\sum\limits_{n=0}^{\infty} (-x^2)^n=
                  \frac{1}{1+x^2} \]
              so $ \displaystyle S_2(x)=\int \frac{1}{1+x^2} \odif{x} =\arctan(x)+C $. But $ S_2(0)=4 $,
              so $ C=4 $. Thus, $ S_2(x)=\arctan(x)+ 4 $.

        \item $ \displaystyle \sum\limits_{n=0}^{\infty} \frac{(-1)^n\pi^{2n}}{2^{2n}(2n)!}=S_3(x) $.

              \textbf{Solution.}
              \[ S_3(x)=\sum\limits_{n=0}^{\infty} \frac{(-1)^n\left( \sfrac{\pi}{2} \right)^{2n}}{(2n)!}
                  =\cos\left( \frac{\pi}{2} \right)=0  \]

        \item Starting with $ \displaystyle \frac{1}{1-x} =\sum\limits_{n=0}^{\infty} x^n $,
              find $ \displaystyle\sum\limits_{n=1}^{\infty} \frac{n x^n}{7} $.

              \textbf{Solution.}
              \begin{align*}
                   & \left( \frac{1}{1-x} \right)^\prime=\frac{1}{(1-x)^2}=
                  \sum\limits_{n=1}^{\infty} n x^{n-1}                                      \\
                   & \implies \frac{x}{(1-x)^2}=\sum\limits_{n=1}^{\infty} n x^n            \\
                   & \implies \frac{x}{7(1-x)^2} =\sum\limits_{n=1}^{\infty} \frac{nx^n}{7}
              \end{align*}
        \item $ \displaystyle S_5(x)=\sum\limits_{n=0}^{\infty}
                  \frac{e(e-1)\cdots(e-n+1)}{3^n (n!)}  $.

              \textbf{Solution.}
              \[ S_5(x)=\sum\limits_{n=0}^{\infty}
                  \frac{e(e-1)\cdots(e-n+1)}{n!}\left( \frac{1}{3}  \right)^n
                  =\left( 1+\frac{1}{3}  \right)^e
                  =\left( \frac{4}{3}  \right)^e \]

    \end{enumerate}
\end{Example}

\subsection*{Evaluating Limits}
We can use Taylor series to evaluate limits, instead
of L'Hopital's Rule. This idea is similar to how we used Taylor
polynomials and Taylor's Approximation Theorem I to evaluate
limits in MATH 137.

\begin{Example}{}{}
    Evaluate with series and not L'Hopital's Rule.

    \begin{enumerate}[label=(\roman*)]
        \item $ \displaystyle \lim\limits_{{x} \to {0}} \frac{e^x-1}{x} $.

              \textbf{Solution.}
              \begin{align*}
                  \lim\limits_{{x} \to {0}} \frac{e^x-1}{x}
                   & =\lim\limits_{{x} \to {0}} \frac{\left( 1+x+\sfrac{x^2}{2!} +\cdots \right)-1}{x} \\
                   & =\lim\limits_{{x} \to {0}}
                  \frac{x+\sfrac{x^2}{2!} +\cdots}{x}                                                  \\
                   & =\lim\limits_{{x} \to {0}} \left[ 1+\sfrac{x}{2!}+\cdots \right]                  \\
                   & =1
              \end{align*}

        \item $ \displaystyle \lim\limits_{{x} \to {0}} \frac{1-\cos(x)}{x^2} $.

              \textbf{Solution.}
              \begin{align*}
                  \lim\limits_{{x} \to {0}} \frac{1-\cos(x)}{x^2}
                   & =\lim\limits_{{x} \to {0}}
                  \frac{1-\left( 1-\sfrac{x^2}{2!} +\sfrac{x^4}{4!} -\cdots \right)}{x^2}             \\
                   & =\lim\limits_{{x} \to {0}} \frac{\sfrac{x^2}{2!}+\sfrac{x^4}{4!} }{x^2}          \\
                   & +\lim\limits_{{x} \to {0}} \left[ \sfrac{1}{2!} -\sfrac{x^2}{4!} +\cdots \right] \\
                   & =\frac{1}{2}
              \end{align*}

        \item $ \displaystyle \lim\limits_{{x} \to {0}} \frac{e^x-\sfrac{x^2}{2} -x-1}{\sin(x)-x} $.

              \textbf{Solution.}
              \begin{align*}
                  \lim\limits_{{x} \to {0}} \frac{e^x-\sfrac{x^2}{2} -x-1}{\sin(x)-x}
                   & =\lim\limits_{{x} \to {0}}
                  \frac{\left( 1+x+\sfrac{x^2}{2!}+\sfrac{x^3}{3!}+\cdots \right)-
                  \sfrac{x^2}{2} -x-1}{\left( x-\sfrac{x^3}{3!} +\sfrac{x^5}{5!} -\cdots \right)-x}       \\
                   & =\lim\limits_{{x} \to {0}}
                  \frac{\sfrac{x^3}{3!} +\sfrac{x^4}{4!}+\cdots}{-\sfrac{x^3}{3!}+\sfrac{x^5}{5!}-\cdots} \\
                   & =\lim\limits_{{x} \to {0}}
                  \frac{\sfrac{1}{3!}+\sfrac{x}{4!}+\cdots}{-\sfrac{1}{3!}+\sfrac{x^2}{5!} -\cdots}       \\
                   & =\frac{\sfrac{1}{3!}  }{-\sfrac{1}{3!}  }                                            \\
                   & =-1
              \end{align*}
    \end{enumerate}
\end{Example}

\subsection*{Evaluating Integrals as Series}

\begin{Example}{}{}
    Evaluate $ \displaystyle \int e^{-x^2}\odif{x} $ as a series.

    \textbf{Solution.}
    \[ \int e^{-x^2}\odif{x}
        =\int \sum\limits_{n=0}^{\infty} \frac{(-1)^n x^{2n}}{n!} \odif{x}
        =\sum\limits_{n=0}^{\infty} \left[ \frac{(-1)^n x^{2n+1}}{n!(2n+1)} \right]+C \]
\end{Example}

\begin{Example}{}{}
    How many terms would we need to use to approximate
    $ \displaystyle \int_{0}^{1} e^{-x^2}\odif{x} $
    to an accuracy of $ \dfrac{1}{10!(21)} $?

    \textbf{Solution.}
    \[ \int_{0}^{1} e^{-x^2}\odif{x}
        = \sum\limits_{n=0}^{\infty}\left[ \frac{(-1)^n x^{2n+1}}{n!(2n+1)} \right]_0^1
        =\sum\limits_{n=0}^{\infty} \frac{(-1)^n}{n!(2n+1)}   \]
    This converges by AST\@! Let's use AST estimation. First, write out some terms:
    \[ 1-\frac{1}{1!(3)} +\frac{1}{2!(5)} -\frac{1}{3!(7)} +\frac{1}{4!(9)}-
        \frac{1}{5!(11)}+\frac{1}{6!(13)} -\frac{1}{7!(15)} +\frac{1}{8!(17)} -\frac{1}{9!(19)} +
        \underset{\text{error}}{\boxed{\frac{1}{10!(21)}}} \]
    So, the estimate needs at least 10 terms.
\end{Example}

\subsection*{Recap of Power Series}
\underline{Strategy for Solving Questions}:
\begin{itemize}
    \item Given a \emph{series}, to find radius and interval of
          convergence:
          \begin{itemize}
              \item Ratio test for $ R $ and open interval
              \item Check endpoints with other tests
          \end{itemize}
    \item Given a \emph{series}, to find its sum:
          \begin{itemize}
              \item Relate it to a known Series
              \item Might need to integrate/differentiate
          \end{itemize}
    \item Given a \emph{function},
          to get its Taylor/Maclaurin series, we can:
          \begin{itemize}
              \item Use the Taylor series formula $ \displaystyle \sum\limits_{n=0}^{\infty}
                        \frac{f^{(n)}(a)}{n!} (x-a)^n $ where $ R $ and $ I $ need to be found from scratch
                    in this case.
              \item Manipulate/Integrate/Differentiate a known series. $ R $ will be known,
                    but endpoints need to be checked to find $ I $.
              \item If asked for a Taylor series about $ x=a $, try $ f(x)=f(x-a+a) $,
                    manipulate, and use a known series.
          \end{itemize}
    \item Stuff we can do with Taylor series:
          \begin{enumerate}[label=(\Roman*)]
              \item Find sums
              \item Evaluate limits
              \item Evaluate and approximate integrals
          \end{enumerate}
\end{itemize}
