\section{Separable Differential Equations}
In this course, we will consider first-order ODEs that can be written
in the form
\[ y^\prime=f(x,y) \]
That is, we can solve for $ y^\prime $.

\begin{Definition}{Separable}{}
    A separable ODE is a first-order ODE that can be written as
    \[ \frac{dy}{dx} =g(y)h(x), \]
    that is, we can factor the RHS into a product of functions, one containing
    only $ x $'s and one containing only $ y $'s.
\end{Definition}
To solve a separable ODE, move $ g(y) $ to the LHS and integrate both sides with respect
to $ x $.
\begin{align*}
    \frac{dy}{dx} =g(y)h(x)
     & \implies \frac{1}{g(y)}\frac{dy}{dx} =h(x)                          \\
     & \implies \int \frac{1}{g(y)}\frac{dy}{dx} \, d{x} =\int h(x)\, d{x} \\
     & \implies \int \frac{1}{g(y)} \, d{y} =\int h(x)\, d{x}              \\
\end{align*}
Now, integrate each side!
\begin{Remark}{}{}
    What is going on in the last step? A substitution! Say $ y=f(x) $,
    then $ dy=f^\prime(x)\,dx $. Therefore, the LHS is:
    \[ \int \frac{1}{g(f(x))} f^\prime(x)\, d{x} =\int \frac{1}{g(y)} \, d{y} \]
\end{Remark}

\begin{Example}{}{}
    Solve
    \[ \frac{dy}{dx} =\frac{x}{y}, \]
    find the general solution.

    \textbf{Solution.}
    \begin{align*}
        \frac{dy}{dx} =\frac{x}{y}
         & \implies \int y\, d{y} =\int x\, d{x}   \\
         & \implies \frac{y^2}{2} =\frac{x^2}{2}+C
    \end{align*}
    Solve for $ y $ if possible: $ y=\pm\sqrt{x^2+2C} $ or $ y=\pm\sqrt{x^2+C_1} $
\end{Example}

\begin{Example}{}{}
    Find the particular solution to the following IVP\@.
    \[ \frac{dy}{dx} =\frac{3x^2+4x+2}{2y-2} \]
    with $ y(0)=-1 $.

    \textbf{Solution.}
    \[ \int 2y-2\, d{y} =\int 3x^2+4x+2\, d{x} \implies y^2-2y=x^3+2x^2+2x+C \]
    Next, get $ C $ by using $ y(0)=-1 $.
    \[ (-1)^2-2(-1)=0+0+0+C\implies C=3 \]
    So, $ y^2-2y=x^3+2x^2+2x+3 $. We can solve for $ y $ if we complete the square
    on the LHS\@:
    \begin{align*}
        y^2-2y    & =  x^3+2x^2+2x+3           \\
        (y-1)^2-1 & = x^3+2x^2+2x+3            \\
        y         & = 1\pm\sqrt{x^3+2x^2+2x+4}
    \end{align*}
    but, only one satisfies $ y(0)=-1 $:
    \[ y=1-\sqrt{x^3+2x^2+2x+4} \]
\end{Example}

\begin{Remark}{}{}
    Watch out for dividing by zero! If you see a possible divide
    by zero (with $ y $), then deal with that case separately.
\end{Remark}

\begin{Example}{}{}
    Find the general solution to
    \[ \frac{dy}{dx}=\frac{y\cos(x)}{1+2y^2} \]
    \textbf{Solution.} We get
    \[ \int \frac{1+2y^2}{y} \, d{y} =\int \cos(x)\, d{x} \]
    if $ y\not\equiv 0 $ ``not identically'' meaning not the constant function $ 0 $.
    \[ \int \frac{1}{y} +2y\, d{y} =\sin(x)+C\implies \ln(y)+y^2=\sin(x)+C \]
    (can't solve for $ y $).

    But what if $ y\equiv 0 $? Then, $ \frac{dy}{dx} =0 $, and the ODE becomes
    $ 0=0 $ which is true for all $ x $! So, $ y\equiv 0 $ is also a solution.

    Therefore, the general solution is
    \[ \ln(y)+y^2=\sin(x)+C \text{ or }y\equiv 0 \]
    Note that the $ y\equiv 0 $ solution (which is an equilibrium solution) is also
    called a \textbf{singular solution} since you can't get it by choosing $ C $.
\end{Example}

\begin{Example}{}{}
    Find a particular solution for the IVP
    \[ \frac{dy}{dx} =xy \]
    with $ y(0)=1 $.

    \textbf{Solution.}
    \[ \int \frac{1}{y} \, d{y} =\int x\, d{x} \]
    if $ y\not\equiv 0 $, but note that $ y(0)=1 $, so $ y\not\equiv 0 $!
    \begin{align*}
        \ln\abs{y} & =\frac{x^2}{2} +C                                                    \\
        \abs{y}    & =e^{\sfrac{x^2}{2}}+C=e^C e^{\sfrac{x^2}{2}}                         \\
        y          & =\pm e^C e^{\sfrac{x^2}{2}}                  & \text{say } A=\pm e^C \\
        y          & =Ae^{\sfrac{x^2}{2}}
    \end{align*}
    Use $ y(0)=1 $ to get $ A $: $ 1=Ae^0=A $, so $ y=e^{\sfrac{x^2}{2}} $.
\end{Example}

Sometimes an ODE isn't separable, but a substitution will make it separable.

Common substitutions: $ V=y+x $, $ V=\sfrac{y}{x} $, $ V=y^\prime $, etc.

\begin{Example}{}{}
    Solve
    \[ \frac{dy}{dx} =(x+y)^2-1 \]

    \textbf{Solution.} This ODE is not separable, but let $ V=x+y $, so
    $ V^\prime=1+y^\prime $ or $ y^\prime=V^\prime-1 $. So,
    \[ V^\prime-1=V^2-1\implies V^\prime=V^2 \]
    Now it's separable! Therefore,
    \[ \int \frac{1}{V^2} \, d{V} =\int \, d{x} \]
    if $ V\not\equiv 0 $.
    \[ -\frac{1}{V} =x+C\implies V=-\frac{1}{x+C} \]
    but $ V=x+y $, so
    \[ x+y=-\frac{1}{x+C} \]
    so we get
    \[ y=-x-\frac{1}{x+C} \]
    What if $ V\equiv 0 $? Then $ y=-x $, and $ \sfrac{dy}{dx}=-1, $ the ODE becomes
    $ -1=-1 $ which is true for all $ x $! So, $ y=-x $ is another solution. Thus,
    \[ y=-x\text{ or }y=-x-\frac{1}{x+C} \]
\end{Example}

\subsection*{Application: Mixing Problems}
Suppose a tank has 1000 L of salt water at an initial concentration of
0.1 kg/L.

Salt water of concentration 0.3 kg/L flows in at a rate of 10 L/min. The solution is
kept well-mixed, and drains out at the same rate.

Let $ X(t)= $ the amount of salt in the tank at time $ t $.

Then,
\begin{align*}
    \frac{dx}{dt}
     & =(\text{rate in})-(\text{rate out})                                         \\
     & =(10\text{ L/min})(0.3\text{kg/L})-(\text{conc.\ in tank})(10\text{ L/min}) \\
\end{align*}
Therefore,
\begin{align*}
    (\text{conc.\ in tank})
     & =\frac{\text{Amount of salt}}{\text{volume}}                                      \\
     & =\frac{x}{1000}                              & \text{since volume doesn't change}
\end{align*}
So,
\begin{align*}
    \frac{dx}{dt}
     & =3-\frac{x}{100}                       \\
     & =\frac{300-x}{100} & \text{separable!} \\
\end{align*}
\[ \int \frac{1}{300-x} \, d{x}=\int \frac{1}{100} \, d{t} \implies -\ln\abs{300-x}=\frac{t}{100} +C \]
What is $ X(0) $? Starts at 100 L at 0.1 kg/L, so $ X(0) $ = 100 kg.

Find $ C $: \[ -\ln\abs{300-100}=C\implies C=-\ln(200) \]
So,
\[ -\ln\abs{300-x}=\frac{t}{100}-\ln(200) \]
Solve for $ x $:
\[ \abs{300-x}=200e^{-\sfrac{t}{100}} \]
but $ 300-x\geqslant 0 $ since $ X(0)=100 $ and $ x $ is increasing
to 300. So, we get:
\[ X(t)=300-200e^{-\sfrac{t}{100}} \]

\section{Linear First-Order Differential Equations}
The general form for a linear first-order ODE is
\[ A(x)y^\prime+B(x)y=C(x) \]
where $ A(x)\not\equiv 0 $. Or, dividing by $ A(x) $, we can write it as:
\[ y^\prime+P(x)y=Q(x) \]
\begin{Example}{Preliminary Example}{}
    \[ \frac{dy}{dx} +\frac{1}{x} y=1 \]
    The trick is: multiply by $ x $!
    \[ x \frac{dy}{dx} +y=x \]
    Notice that now the LHS is the derivative of $ xy $. So,
    \[ \frac{d}{dx} (xy)=x \]
    Now, integrate both sides!
    \[ xy=\int x\, d{x}=\frac{x^2}{2} +C \]
    Thus,
    \[ y=\frac{x}{2} +\frac{C}{x} \]
\end{Example}
This is the general strategy: find a clever function to multiply
the ODE by so that the LHS collapses into the derivative of a product.
colour
Then, we just need to integrate both sides and solve for $ y $.

Let's find the useful function:

Say the ODE is
\[ \frac{dy}{dx} +P(x)y=Q(x) \]
and the desired function is $ \mu(x) $. Multiplying, we get
\[ \mu(x)\frac{dy}{dx} +\mu(x)P(x)y=\mu(x)Q(x) \]
We \textbf{want} the LHS to be
\[ \frac{d}{dx} \left[ \mu(x)y \right]=\mu^\prime(x)y+\mu(x)\frac{dy}{dx}  \]
Solving, we get:
\[ \mu(x)\frac{dy}{dx} +\mu(x)P(x)y=\mu(x)\frac{dy}{dx}+\mu^\prime(x)y \]
\begin{align*}
     & \implies \mu(x)P(x)y=\mu^\prime(x)y & \text{should hold for all } y \\
     & \implies \mu(x)P(x)=\mu^\prime(x)
\end{align*}
or
\[ \mu(x)P(x)=\frac{d\mu}{dx}  \]
which is separable!
\[ \int \frac{1}{\mu} \, d{\mu} =\int P(x)\, d{x}\implies \ln\abs{\mu}=\int P(x)\, d{x} \]
Thus, the final form is
\[ \mu=e^{\int P(x)\, d{x}} \]
\begin{Remark}{}{}
    We can ignore the ``$ +C $'' and absolute values since we only need to find one $ \mu $
    that works, not all of them.
\end{Remark}
So, we get an algorithm to solve a linear first-order ODE\@:
\begin{enumerate}
    \item Write it in the form
          \[ \frac{dy}{dx}=P(x)y=Q(x) \]
    \item Find
          \[ \mu(x)=e^{\int P(x)\, d{x}} \]
    \item Multiply the ODE by $ \mu $, collapse LHS into a product rule.
    \item Integrate both sides (add $ +C $) and solve for $ y $.
\end{enumerate}

\begin{Example}{}{}
    Solve
    \[ \frac{dy}{dx} +2xy=x \]
    \textbf{Solution.}
    \[ \mu(x)=e^{\int P(x)\, d{x}}=e^{\int 2x\, d{x}}=e^{x^2} \]
    Multiply by $ e^{x^2} $:
    \[ e^{x^2}\frac{dy}{dx} +2xe^{x^2}y=xe^{x^2} \]
    \[ \implies \frac{d}{dx} \left[ e^{x^2}y \right]=xe^{x^2} \]
    Integrate:
    \begin{align*}
        e^{x^2}y
         & =\int x e^{x^2}\, d{x}       & u=x^2\iff du=2x\,dx \\
         & =\frac{1}{2} \int e^u\, d{u}                       \\
         & =\frac{1}{2} e^{u}+C                               \\
         & =\frac{1}{2}e^{x^2}+C
    \end{align*}
    So,
    \[ e^{x^2}y=\frac{1}{2} e^{x^2}+C\implies y=\frac{1}{2} +\frac{C}{e^{x^2}} \]
\end{Example}

\begin{Example}{}{}
    Find a particular solution to
    \[ x \frac{dy}{dx} +2xy=1 \]
    with $ y(1)=0 $.

    \textbf{Solution.} It's not in the correct form! First, divide by $ x^2 $:
    \[ \frac{dy}{dx} +\frac{2}{x} y=\frac{1}{x^2}  \]
    Now,
    \begin{align*}
        \mu(x)
         & =e^{\int \sfrac{2}{x} \, d{x} } \\
         & =e^{2\ln\abs{x}}                \\
         & =e^{\ln(x)^2}                   \\
         & =x^2
    \end{align*}
    So multiply by $ x^2 $, so our original ODE was actually what we wanted! Cool!
    \[ x^2 \frac{dy}{dx} +2xy=1 \]
    \[ \frac{d}{dx} \left[ x^2y \right]=1 \]
    Integrating gives
    \begin{align*}
        x^2 y=x+C \\
        y=\frac{1}{x} +\frac{C}{x^2}
    \end{align*}
    Finally, we know $ y(1)=0 $, so $ 0=1+C\implies C=-1 $. Thus,
    \[ y=\frac{1}{x} -\frac{1}{x^2} \]
\end{Example}

\begin{Remark}{}{}
    There is a formula you can use as well! The solution to
    \[ \frac{dy}{dx}+P(x)y=Q(x) \]
    is
    \[ y=\frac{1}{\mu(x)}\Bigl[\int \mu(x)Q(x)\, d{x} \Bigr]  \]
    where
    \[ \mu(x)=e^{\int P(x)\, d{x}} \]
\end{Remark}

\begin{Example}{}{}
    \[ x \frac{dy}{dx} +2xe^x y=xe^x \]
    First, divide by $ x $:
    \[ \frac{dy}{dx}+2e^x y=e^x \]
    \[ \mu(x)=e^{\int 2e^x\, d{x}}=e^{2e^x} \]
    Now, using the formula:
    \begin{align*}
        y
         & =\frac{1}{\mu(x)}\int \mu(x)Q(x)\, d{x}                 \\
         & =\frac{1}{e^{2e^x}}\Bigl[\int e^{2e^x}e^x\, d{x} \Bigr] \\
         & =\frac{1}{e^{2e^x}} \Bigl[e^{2u}\Bigr]                  \\
         & =\frac{1}{e^{2e^x}}\left( \frac{e^{2u}}{2} +C \right)   \\
         & =\frac{1}{e^{2e^x}}\left( \frac{e^{2e^x}}{2} +C \right) \\
         & =\frac{1}{2} +\frac{C}{e^{2e^x}}
    \end{align*}
\end{Example}

There is an important result regarding these ODEs:
\begin{Theorem}{}{}
    Assume $ P $ and $ Q $ are continuous functions on an interval $ I $.
    Then, for each $ x_0\in I $ and any $ y_o\in\mathbb{R} $, the IVP
    \[ y^\prime +P(x)y=Q(x) \]
    with $ y(x_0)=y_0 $ has exactly one solution on $ I $.
\end{Theorem}

\begin{Remark}{}{}
    This is not true for other IVPs, some have no solutions, and some have more than
    one (some even have $ \infty $-many!).
\end{Remark}

\setcounter{section}{6}
\section{Newton's Law of Cooling}
The law states that an object's temperature changes at a rate that is proportional
to the difference between the temperature of
the object and the ambient temperature; that is, the temperature of the room,
$ T_{\text{room}} $. The formula is:
\[ \frac{dT}{dt}=-K\left( T-T_{\text{room}} \right)  \]
where
\begin{itemize}
    \item $ \sfrac{dT}{dt} = $ rate of change of temperature
    \item $ -K =$ a constant
    \item $ T= $ temperature at time $ t $
    \item $ T_{\text{room}}= $ temperature of surroundings (constant)
\end{itemize}
Q\@: Why is the constant negative?

A\@: If $ T>T_{\text{room}} $, we would expect the object to be \underline{cooling},
so $ \sfrac{dT}{dt} <0 $. This means the coefficient needs to be negative since
$ T-T_{\text{room}}>0 $. On the other hand, if $ T<T_{\text{room}} $,
then the object is heating up, so $ \sfrac{dT}{dt}>0 $, but
$ T-T_{\text{room}}<0 $, so again we need a negative constant.

\underline{Exercise}: The solution to this separable and linear ODE is
\[ T(t)=Ce^{-Kt}+T_{\text{room}} \]
for $ C,K\in\mathbb{R} $. Also, $ K $ can be determined with extra info.

Notice that $ \lim\limits_{{t} \to {\infty}} T(t)=T_{\text{room}} $,
as expected.

\begin{Example}{}{}
    For
    \[ \frac{dT}{dt}=-K(T-25)  \]
    if the object was initially at $ 0^\circ C $, and after
    10 minutes it was at $ 5^\circ C $, solve the ODE\@. Here,
    $ T $ is in $ ^\circ C $ and $ t $ is in minutes.

    \textbf{Solution.} We know $ T=Ce^{-Kt}+25 $, and $ T(0)=0 $,
    $ T(10)=5 $. First, $ T(0)=0\implies 0=C+25 $, so $ C=-25 $.
    Therefore,
    \[ T=-25e^{-Kt}+25 \]
    Next,
    \begin{align*}
         & T(10)=5                                               \\
         & \implies \frac{-20}{-25} =e^{-10K}                    \\
         & \implies \frac{4}{5} =e^{-10K}                        \\
         & \implies K=-\frac{1}{10}\ln\left( \frac{4}{5} \right)
    \end{align*}
    So,
    \[ T=-25e^{\sfrac{1}{10} \ln\left( \sfrac{4}{5} \right)t}+25 \]
\end{Example}

\section{Models of Population Growth}
The two models we will examine are:
\begin{itemize}
    \item Natural Growth (Exponential Growth)
    \item Logistic Growth
\end{itemize}
\textbf{Natural Growth}: It makes sense to assume that
population grows at a rate proportional to the size of the population.
So, if $ P= $ population, and $ t= $ time, then
\[ \frac{dP}{dt}=kP  \]
where $ k= $ constant, roughly equal to $ \text{birth rate}-\text{death rate} $.

It's separable! If $ p\not\equiv 0 $:
\[ \int \frac{1}{P} \, d{P}=\int k\, d{t}  \]
\begin{align*}
     & \implies \ln\abs{P}=kt+C    \\
     & \implies \abs{P}=e^C e^{kt} \\
     & \implies P=\pm e^{C}e^{kt}
\end{align*}
Say $ \pm e^C =A\in\mathbb{R} $, then
\[ P=Ae^{kt} \]
Note that $ P(0)=Ae^0=A $, so $ A=\text{initial population} $. So,
the solution to the IVP $ \sfrac{dP}{dt} =kP $ with $ P(0)=P_0 $ is
\[ P=P_0 e^{kt} \]

Is natural growth a good model? Say we have 1000 bacteria in a petri
dish, and we observe that 300 new bacteria are formed after 1 hour.
It is reasonable to assume that 2000 bacteria would spawn 600 in one hour,
isn't it? Well, yes! That is, until they run out of food!

Shouldn't we also take the environment into account? After
a certain population, there won't be enough food/space to
support any more growth. So it seems like natural growth is fine
as long as there are lots of resources (that is, for small populations).
But once population nears its limit, it won't be a good model any more.

This limit is called the Carrying Capacity; that is, the maximum population
that the environment can support in the long run.

Denote the carrying capacity by $ M $.

\begin{Definition}{}{}
    The \textbf{logistical differential equation} is
    \[ \frac{dP}{dt}=kP\left( 1-\frac{P}{M}  \right) \]
    where $ k =$ same constant as in natural growth.
\end{Definition}
\begin{Remark}{}{}
    Sometimes written
    \[ \frac{dP}{dt}=kP(M-P)  \]
    but $ K $ is different here.
\end{Remark}

Let's examine some cases:
\begin{enumerate}
    \item If $ P\ll M $, then $ 1-\sfrac{P}{M}\approx 1 $, so $ \sfrac{dP}{dt} \approx kP $
          (natural growth!)
    \item If $ P\approx M $, then $ 1-\sfrac{P}{M}\approx 0 $, so $ \sfrac{dP}{dt}\approx 0 $,
          which makes sense as population has reached its limit.
    \item If $ P>M $, then $ \sfrac{dP}{dt} <0 $, as expected. (The population
          shrinks as there are not enough resources).
\end{enumerate}
So, this equation appears to be a good model! Let's solve it
\[ \sfrac{dP}{dt} =kP(1-\sfrac{P}{M}) \]
with $ P(0)=P_0 $. It's separable!
\[ \int \frac{1}{P(1-\sfrac{P}{M})} \, d{P}=\int k\, d{t} =kt+C \]
Use partial fractions! Notice
\[ \frac{1}{P(1-\sfrac{P}{M})}=\frac{M}{P(M-P)} =\frac{1}{P} +\frac{1}{M-P} \]
So we get
\begin{align*}
    \int \frac{1}{P} +\frac{1}{M-P} \, d{P} & =kt+C                                                   \\
    \implies  \ln\abs{P}-\ln{M-P}           & =kt+C                                                   \\
    \implies  \ln\abs*{\frac{P}{M-P}}       & =kt+C                                                   \\
    \implies \ln\abs*{\frac{M-P}{P}}        & =-kt-C                                                  \\
    \implies  \abs*{\frac{M-P}{P}}          & =e^{-kt-C}=e^{-C}e^{-kt}                                \\
    \implies  \frac{M-P}{P}                 & =\pm e^{-C}e^{-kt}=Ae^{-kt} & \text{where }A=\pm e^{-C} \\
    \implies  \frac{M}{P} -1                & =Ae^{-kt}                                               \\
    \implies  P                             & =\frac{M}{1+Ae^{-kt}}
\end{align*}
Next, use $ P(0)=P_0 $ to get $ A $:
\[ P_0=\frac{M}{1+A}\implies 1+A=\frac{M}{P_0}\implies A=\frac{M-P_0}{P_0}  \]
So, the solution to $ \sfrac{dP}{dt}=kP\left( 1-\sfrac{P}{M} \right) $, $ P(0)=P_0 $
is
\[ P=\frac{M}{1+Ae^{-kt}}  \]
where $ A=\sfrac{(M-P_0)}{P_0} $.

\begin{Example}{}{}
    Scientists took 100 wolves and let them go in a walled-off nature preserve. They
    estimated the carrying capacity to be 1500, and after one year, there were 150 wolves.

    Assuming logistic growth, find $ P(t) $.

    \textbf{Solution.}
    \[ \frac{dP}{dt}=kP\left(1-\frac{P}{1500}\right)  \]
    $ P(0)=100 $, and $ P(1)=150 $. So,
    \[ P=\frac{1500}{1+Ae^{-kt}}  \]
    with $ A=\sfrac{(1500-100)}{100}=14  $. So,
    \[ P=\frac{1500}{1+14e^{-kt}}  \]
    Lastly, use $ P(1)=150 $ to get $ k $.
    \[ 150=\frac{1500}{1+14e^{-k}}\implies 1+14e^{-k}=10\implies e^{-k}=\frac{9}{14} \implies
        -k=\ln\left( \frac{9}{14} \right) \]
    So,
    \[ P=\frac{1500}{1+14e^{\ln\left( \sfrac{9}{14}  \right)t}} =\frac{1500}{1+14\left( \sfrac{9}{14}  \right)^t} \]
    Q\@: How long until there are 1000 wolves?

    A\@: Find $ t $:
    \[ 1000=\frac{1500}{1+14\left( \sfrac{9}{14}  \right)^t}
        \implies \frac{3}{2} =1+14\left( \frac{9}{14}  \right)^t
        \implies \frac{1}{28} =\left( \frac{9}{14} \right)^t
        \implies t=\frac{\ln\left( \sfrac{1}{28}  \right)}{\ln\left( \sfrac{9}{14}\right)}
        \approx 754\text{ years}\]
\end{Example}

\underline{For Fun}: There are other models too:
\begin{enumerate}
    \item Taking harvesting/hunting into account:
          \[ \frac{dP}{dt}=kP\left( 1-\frac{P}{M} \right)-C \]
          where $ C $ is the harvesting constant.
    \item If a population is \textbf{too sparse}, they may go extinct:
          \[ \frac{dP}{dt}=kP\left( 1-\frac{P}{M} \right)\left( 1-\frac{N}{P}  \right)  \]
          where $ N= $ minimum population to prevent extinction.
\end{enumerate}
