\section{Improper Integrals}
So far, we have only examined integrals of continuous, or at
least bounded functions. Let's see how to deal with a more general
collection of functions!

In particular, we will examine two types:
\begin{enumerate}[label=\arabic*)]
    \item Continuous functions over infinite intervals
    \item Functions with infinite discontinuities
\end{enumerate}

In particular:
\begin{itemize}
    \item Type I: Infinite Intervals. Integrals of the form
          \[ \int_{-\infty}^{a} f(x)\, d{x},\,\int_{a}^{\infty} f(x)\, d{x},\,
              \int_{-\infty}^{\infty} f(x)\, d{x} \]
    \item Type II: Infinite Discontinuity. For example,
          \[ \int_{-1}^{1} \frac{1}{x} \, d{x}  \]
          as there is an issue at $ x=0 $.
\end{itemize}

In all cases, the idea is to replace the problematic point with a
letter and take a limit.

Let's see them in more detail now!

\subsection{Type I}
We replace the infinite endpoint with a letter and take a limit
\begin{itemize}
    \item \[ \int_{-\infty}^{a} f(x)\, d{x} =\lim\limits_{{b} \to {-\infty}}
              \int_{b}^{a} f(x)\, d{x} \]
    \item \[ \int_{\infty}^{a} f(x)\, d{x} =\lim\limits_{{b} \to {\infty}}
              \int_{a}^{b} f(x)\, d{x} \]
    \item \[ \int_{-\infty}^{\infty} f(x)\, d{x}=
              \lim\limits_{{b_1} \to {-\infty}} \int_{b_1}^{0} f(x)\, d{x}+
              \lim\limits_{{b_2} \to {\infty}} \int_{0}^{b_2} f(x)\, d{x}  \]
\end{itemize}
Don't use
\[ \int_{-\infty}^{\infty} f(x)\, d{x}=\lim\limits_{{b} \to {\infty}}
    \int_{-b}^{b} f(x)\, d{x}  \]
This is called the ``Cauchy Principal Value'' and it is something else!

We say that the integral \textbf{converges} if all the limits exist
(and are finite). The integral \textbf{diverges} if even one limit does
not exist (or is $ \pm\infty $).

\begin{Example}{Type I Integrals}{}
    Evaluate the following or show they diverge.
    \begin{enumerate}[label=(\roman*)]
        \item \begin{align*}
                  \int_{2}^{\infty} \frac{1}{x^2} \, d{x}
                   & =\lim\limits_{{b} \to {\infty}} \int_{2}^{b} \frac{1}{x^2} \, d{x}       \\
                   & = \lim\limits_{{b} \to {\infty}} \left[-\frac{1}{x}  \right]_{2}^b       \\
                   & =\lim\limits_{{b} \to {\infty}} \left( -\frac{1}{b} +\frac{1}{2} \right) \\
                   & =\frac{1}{2}
              \end{align*}
              Thus, the integral converges.
        \item \begin{align*}
                  \int_{-\infty}^{\infty} \sin(x)\, d{x}
                   & =\lim\limits_{{b_1} \to {-\infty}} \int_{b_1}^{0} \sin(x)\, d{x}
                  +\lim\limits_{{b_2} \to {\infty}} \int_{0}^{b_2} \sin(x)\, d{x}
              \end{align*}
              Let's evaluate the first one:
              \[ \lim\limits_{{b_1} \to {-\infty}}\int_{b_1}^{0} \sin(x)\, d{x}
                  = \lim\limits_{{b_1} \to {-\infty}}\bigl[-\cos(x) \bigr]_{b_1}^0
                  =\lim\limits_{{b_1} \to {-\infty}} \left[-\cos(0)+\cos(b_1)\right] \]
              which does not exist. Therefore, this integral
              diverges, there is no need to check the second limit!
        \item \begin{align*}
                  \int_{0}^{\infty} \frac{1}{1+x^2}\, d{x}
                   & =\lim\limits_{{b} \to {\infty}} \int_{0}^{b} \frac{1}{1+x^2} \, d{x} \\
                   & = \lim\limits_{{b} \to {\infty}} \bigl[\arctan(x) \bigr]_{0}^b       \\
                   & =\lim\limits_{{b} \to {\infty}} \left[\arctan(b)-\arctan(0) \right]  \\
                   & =\frac{\pi}{2} -0                                                    \\
                   & =\frac{\pi}{2}
              \end{align*}
              Thus, the integral converges.
    \end{enumerate}
\end{Example}
Question: For which $ p\in\mathbb{R} $ does $ \int_{1}^{\infty} \frac{1}{x^p} \, d{x} $
converge?

Let's find out!
\begin{enumerate}[label=C\arabic*]
    \item $ p>1 $.
          \begin{align*}
              \lim\limits_{{b} \to {\infty}} \int_{1}^{b} x^{-p}\, d{x}
               & = \lim\limits_{{b} \to {\infty}} \left[\frac{x^{-p+1}}{p+1}  \right]_{1}^b \\
               & =\lim\limits_{{b} \to {\infty}} \left( \frac{b^{-p+1}}{-p+1}-
              \frac{1^{-p+1}}{-p+1}  \right)                                                \\
               & =\frac{1}{p-1}
          \end{align*}
          since $ -p+1<0 $, so $ b^{-p+1}\to 0 $. So, the integral converges if $ p>1 $.
    \item $ p<1 $. The calculation is the same as C1, until:
          \[ \lim\limits_{{b} \to {\infty}} \left( \frac{b^{-p+1}}{-p+1} -\frac{1}{-p+1} =\infty \right) \]
          since $ -p+1>0 $, so $ b^{-p+1}\to\infty $. So, the integral diverges if $ p<1 $.
    \item $ p=1 $.
          \begin{align*}
              \lim\limits_{{b} \to {\infty}} \int_{1}^{b} \frac{1}{x} \, d{x}
               & =\lim\limits_{{b} \to {\infty}} \bigl[\ln\abs{x}\bigr]_{1}^b         \\
               & =\lim\limits_{{b} \to {\infty}} \left( \ln\abs{b}-\ln\abs{1} \right) \\
               & =\infty
          \end{align*}
          So, the integral diverges if $ p=1 $.
\end{enumerate}
Therefore, we have proven:

\begin{Theorem}{$p$-Integrals}{}
    The improper integral
    \[ \int_{1}^{\infty} \frac{1}{x^p} \, d{x} \]
    converges if and only if $ p>1 $.

    If $ p>1 $,
    \[ \int_{1}^{\infty} \frac{1}{x^p} \, d{x} =\frac{1}{p-1} \]
\end{Theorem}

Next, let's examine some properties of Type I improper integrals.

\begin{Theorem}{Properties of Type I Improper Integrals}{}
    Suppose $ \int_{a}^{\infty} f(x)\, d{x}  $ and $ \int_{a}^{\infty} g(x)\, d{x}  $
    both converge.
    \begin{enumerate}[label=(\arabic*)]
        \item $ \int_{a}^{\infty} cf(x)\, d{x} $
              converges for any $ c\in\mathbb{R} $, and
              \[ \int_{a}^{\infty} cf(x)\, d{x} =c \int_{a}^{\infty} f(x)\, d{x} \]
        \item $ \int_{a}^{\infty} f(x)+g(x)\, d{x} $
              converges, and
              \[ \int_{a}^{\infty} f(x)+g(x)\, d{x}=
                  \int_{a}^{\infty} f(x)\, d{x} +\int_{a}^{\infty} g(x)\, d{x} \]
        \item If $ f(x)\leqslant g(x) $ for all $ x\geqslant a $, then
              \[ \int_{a}^{\infty} f(x)\, d{x} \leqslant \int_{a}^{\infty} g(x)\, d{x} \]
        \item If $ a<c<\infty $, then $ \int_{c}^{\infty} f(x)\, d{x} $ converges, and
              \[ \int_{a}^{\infty} f(x)\, d{x}=
                  \int_{a}^{c} f(x)\, d{x} +\int_{c}^{\infty} f(x)\, d{x} \]
    \end{enumerate}
\end{Theorem}
Evaluating integrals in general is hard, and determining if an improper integral converges
may be even harder! However, we do have a way of comparing a difficult
integral to a simpler one (for example, a $ p $-Integral!)
