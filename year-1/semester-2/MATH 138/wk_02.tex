\section{The Fundamental Theorem of Calculus (Part 1)}
The FTC is, essentially, a simple derivative rule. But
its consequences are very valuable. The reason is that it
provides the link between integral calculus and
differential calculus!

We start with integral functions: let $ f $ be
continuous on $ \interval{a}{b} $.

Define
$ \displaystyle G(x)=\int_{a}^{x} f(t)\, d{t} $
for $ x\in\interval{a}{b} $.

What is $ G(x) $? it's the function that returns the signed
area under $ f $ from $ a $ to $ x $.

\begin{Example}{}{}
    Compute the area of $ f(x)=x $ on $ \interval{0}{5} $.

    \textbf{Solution.}
    \[
        G(x)
        =\int_{0}^{x} t\, d{t}
        =\frac{1}{2}(\text{base})(\text{height})
        =\frac{1}{2}(x)(x)
        =\frac{x^2}{2}
    \]
    Wait a minute! $ G^\prime(x)=x=f(x) $. Is this always true?!
\end{Example}

\begin{Theorem}{Fundamental Theorem of Calculus I (FTC I)}{FTC_I}
    If $ f $ is continuous on an open interval $ I $ containing
    $ x=a $, and if
    \[ G(x)=\int_{a}^{x} f(t)\, d{t}  \]
    Then $ G $ is differentiable for all $ x\in I $ and
    $ G^\prime(x)=f(x) $; that is,
    $ \displaystyle \frac{d}{dx}\int_{a}^{x} f(t)\, d{t}=f(x) $.
\end{Theorem}

\begin{Proof}{\ref{thm:FTC_I}}{}
    Let $ f $ be continuous on $ I $, $ \displaystyle G(x)=\int_{a}^{x} f(t)\, d{t} $,
    and fix $ x_0\in I $.

    Let $ \varepsilon>0 $ be given. Since $ f $ is continuous at $ x_0 $,
    there exists a $ \delta>0 $ such that if $ 0<\abs{c-x_0}<\delta $, then
    \[ \abs{f(c)-f(x_0)}<\varepsilon \]
    Let $ 0<\abs{x-x_0}<\delta $. Then,
    \begin{align*}
        \frac{G(x)-G(x_0)}{x-x_0}
         & =\frac{\displaystyle\int_{a}^{x} f(t)\, d{t} -\int_{a}^{x_0} f(t)\, d{t} }
        {x-x_0}                                                                        \\
         & =\frac{\displaystyle\int_{a}^{x_0} f(t)\, d{t} +\int_{x_0}^{x} f(t)\, d{t}-
        \int_{a}^{x_0} f(t)\, d{t} }{x-x_0}                                            \\
         & =\frac{1}{x-x_0} \int_{x_0}^{x} f(t)\, d{t}
    \end{align*}
    The AVT says there exists $ c $ between $ x $ and $ x_0 $ such that
    \[ f(c)=\frac{1}{x-x_0} \int_{x_0}^{x} f(t)\, d{t}  \]
    Since $ 0<\abs{x-x_0}<\delta $, we get $ 0<\abs{c-x_0}<\delta $ too,
    so
    \[
        \abs*{\frac{G(x)-G(x_0)}{x-x_0}-f(x_0)}=\abs{f(c)-f(x_0)}<\varepsilon
    \]
    This says
    $ \displaystyle G^\prime(x_0)=\lim\limits_{{x} \to {x_0}} \frac{G(x)-G(x_0)}{x-x_0}=f(x_0) $.
\end{Proof}

\begin{Example}{}{}
    Compute $ \displaystyle \frac{d}{dx}\int_{5}^{x} \sin(t^2)\, d{t} $

    \textbf{Solution.}
    $ \displaystyle \frac{d}{dx}\int_{5}^{x} \sin(t^2)\, d{t}=\sin(x^2) $
    since $ f(t)=\sin(t^2) $ is continuous, by FTC I.
\end{Example}

\begin{Example}{}{}
    Compute
    $ \displaystyle \frac{d}{dx}\int_{5}^{x^2} \sin(t^2)\, d{t} $.

    \textbf{Solution.} Let $ \displaystyle G(x)=\int_{5}^{x} \sin(t^2)\, d{t} $,
    then $ G(x^2)=\displaystyle \int_{5}^{x^2} \sin(t^2)\, d{t} $. So,
    \[
        \frac{d}{dx} \int_{5}^{x^2} \sin(t^2)\, d{t}
        =\frac{d}{dx} \left[ G(x^2) \right]
        =G^\prime(x^2) (2x)
        =f(x^2)(2x)
        =\sin(x^4) (2x)
    \]
\end{Example}

We will see a more general formula next week!

\section{The Fundamental Theorem of Calculus (Part 2)}
It seems like integrating is the opposite operation to
differentiation, and it is! We can use antiderivatives
to evaluate integrals, as we will see. But first, let's
quickly recap what we know about antidifferentiation.

\begin{Definition}{Antiderivative}{}
    Given a function $ f $, an \textbf{antiderivative} of
    $ f $ is a function $ F $ such that $ F^\prime(x)=f(x) $.
\end{Definition}

\begin{Remark}{}{}
    Antiderivatives are not unique! For example, let $ f(x)=2x $, some
    antiderivatives of $ f(x) $ include:
    \begin{itemize}
        \item $ F_1(x)=x^2 $
        \item $ F_2(x)=x^2+4 $
        \item $ F_3(x)=x^2-\pi $
    \end{itemize}
\end{Remark}

\begin{Definition}{Indefinite integral}{}
    The collection of all antiderivatives of $ f(x) $ is denoted
    by $ \displaystyle\int f(x)\, d{x} $ and
    \[ \int f(x)\, d{x} =F(x)+C \]
    where $ C\in\mathbb{R} $ and $ F $ is any antiderivative.
    This is called the \textbf{indefinite integral}.
\end{Definition}

\begin{Remark}{}{}
    By the Antiderivative Theorem, we know any two antiderivatives
    of $ f $ differ by a constant.
\end{Remark}

Here are a bunch of antiderivatives:
\begin{itemize}
    \item $ \displaystyle \int x^n\, d{x}=\frac{x^{n+1}}{n+1}+C $ if $ n\neq -1 $
    \item $ \displaystyle \int \frac{1}{x}\, d{x}=\ln\abs{x}+C $
    \item $ \displaystyle \int e^x\,d{x}=e^x+C $
    \item $ \displaystyle \int \sin(x)\,d{x}=-\cos(x)+C $
    \item $ \displaystyle \int \cos(x)\,d{x}=\sin(x)+C $
    \item $ \displaystyle \int \sec^2(x)\,d{x}=\tan(x)+C $
    \item $ \displaystyle \int \frac{1}{1+x^2}\,d{x}=\arctan(x)+C $
    \item $ \displaystyle \int \frac{1}{\sqrt{1-x^2}}\,d{x}=\arcsin(x)+C $
    \item $ \displaystyle \int -\frac{1}{\sqrt{1-x^2}}\, d{x}=\arccos(x)+C $
    \item $ \displaystyle \int \sec(x)\tan(x)\,d{x}=\sec(x)+C $
    \item $ \displaystyle \int a^x\,d{x}=\frac{a^x}{\ln(a)}+C $ for $ a>0 $
\end{itemize}

By FTC I, we know every continuous function has an antiderivative,
but how can we use them to actually evaluate definite integrals?
Well\textellipsis{}

\begin{Theorem}{Fundamental Theorem of Calculus II (FTC II)}{FTC_II}
    If $ f $ is continuous on $ \interval{a}{b} $ and $ F $
    is any antiderivative of $ f $, then
    \[ \int_{a}^{b} f(x)\, d{x} =F(b)-F(a)
        =\bigl[F(x)\bigr]_{a}^b \]
\end{Theorem}

\begin{Proof}{\ref{thm:FTC_II}}{}
    Let $ F $ be any antiderivative of $ f $ and define
    $ \displaystyle G(x)=\int_{a}^{x} f(t)\, d{t} $.

    By FTC I, we know $ G^\prime(x)=f(x) $ as well, so by the
    Antiderivative Theorem, $ G(x)=F(x)+C $ for some $ C\in\mathbb{R} $.

    But then,
    \[ G(b)-G(a)=[F(b)+C]-[F(a)+C]=F(b)-F(a) \]
    Also,
    \begin{align*}
        \int_{a}^{b} f(t)\, d{t}
         & =G(b)                                    \\
         & =G(b)-G(a) & \quad & \text{since }G(a)=0 \\
         & =F(b)-F(a)
    \end{align*}
\end{Proof}

Now, we can evaluate definite integrals \emph{without} Riemann sums!

\begin{Example}{}{}
    \begin{enumerate}[label=(\roman*)]
        \item Compute $ \displaystyle \int_{1}^{3} x^2+x\, d{x} $.

              \textbf{Solution.}
              \[
                  \int_{1}^{3} x^2+x\, d{x}
                  =\left[\frac{x^3}{3}+\frac{x^2}{2}\right]_{1}^3
                  = \left( \frac{3^3}{3} +\frac{3^2}{2} \right)-\left( \frac{1}{3}+\frac{1}{2} \right)
                  =\frac{27}{2} -\frac{5}{6}
                  =\frac{38}{3}
              \]
        \item Compute $ \displaystyle  \int_{0}^{2\pi} \sin(x)\, d{x} $.

              \textbf{Solution.}
              \[
                  \int_{0}^{2\pi} \sin(x)\, d{x}
                  =\bigl[-\cos(x)\bigr]_0^{2\pi}
                  =-\cos(2\pi)+\cos(0)
                  =1+1
                  =0
              \]
              This makes sense since the signed area is zero.
        \item Compute $
                  \displaystyle \int_{2}^{8} \frac{x^2+2x+1}{x} \, d{x} $.

              \textbf{Solution.}
              \begin{align*}
                  \int_{2}^{8} \frac{x^2+2x+1}{x} \, d{x}
                   & =\int_{2}^{8} x+2+\frac{1}{x}\, d{x}              \\
                   & =\left[\frac{x^2}{2} +2x+\ln\abs{x} \right]_{2}^8 \\
                   & =[32+16+\ln(8)]-[2+4+\ln(2)]                      \\
                   & =42+\ln(8)-\ln(2)                                 \\
                   & =42+\ln(4)
              \end{align*}
    \end{enumerate}
\end{Example}

This is fantastic! We are only limited by our ability to find antiderivatives!
As we will see, finding antiderivatives is hard in general, but in the next couple of
weeks we will learn a few techniques.

But first, let's look at the extended version of FTC I\@:

\begin{Corollary}{Extended Version of the Fundamental Theorem of Calculus}{ftc_ext}
    If $ f $ is continuous, and $ g, h $ are both differentiable, then
    \[ \frac{d}{dx}\Bigr[\int_{g(x)}^{h(x)} f(t)\, d{t} \Bigl]
        =f(h(x))h^\prime(x)-f(g(x))g^\prime(x) \]
    (also called the Leibniz Formula).
\end{Corollary}

\begin{Proof}{\ref{cor:ftc_ext}}{}
    Let $ F $ be an antiderivative of $ f $, then by FTC II\@:
    \[ \int_{g(x)}^{h(x)} f(t)\, d{t}=F(h(x))-F(g(x)) \]
    for each $ x $. So,
    \begin{align*}
        \frac{d}{dx}\Bigr[\int_{g(x)}^{h(x)} f(t)\, d{t} \Bigl]
         & =\frac{d}{dx}\bigl[F(h(x))-F(g(x))\bigr]             \\
         & =F^\prime(h(x))h^\prime(x)-F^\prime(g(x))g^\prime(x) \\
         & =f(h(x))h^\prime(x)-f(g(x))g^\prime(x)
    \end{align*}
\end{Proof}

\begin{Example}{}{}
    Compute $ \displaystyle \frac{d}{dx}\int_{5x}^{\ln(x)} \cos(t^2-3t)\, d{t} $.

    \textbf{Solution.}
    \[ \frac{d}{dx}\int_{5x}^{\ln(x)} \cos(t^2-3t)\, d{t}
        =\cos\bigl[\ln(x)^2-3\ln(x)\bigr]\left(\frac{1}{x}\right)-\cos(25x^2-15x)(5) \]
\end{Example}

\section{Change of Variables}
The first integration technique we will examine is the reverse chain rule:
Change of Variable, also called Substitution.

The rule is:
$ \displaystyle \int f(g(x))g^\prime(x)\, d{x}=\int f(u)\, d{u}  $;
that is, we ``substitute'' $ u=g(x) $.

\begin{Proof}{Change of Variable (Sketch)}{}
    Let $ f $ and $ g $ be functions and let $ h $ be an antiderivative of $ f $,
    so $ h^\prime(x)=f(x) $.

    Let $ H(x)=h(g(x)) $, so
    \[ H^\prime(x)=h^\prime(g(x))g^\prime(x)=f(g(x))g^\prime(x) \]
    so $ h(g(x)) $ is an antiderivative of $ f(g(x))g^\prime(x) $.

    Therefore,
    \begin{align*}
        \int f(g(x))g^\prime(x)d{x}\,
         & =h(g(x))+C        &  & \text{for some }c\in\mathbb{R} \\
         & =h(u)+C           &  & \text{if }u=g(x)               \\
         & =\int f(u)\, d{u} &  & \text{if }u=g(x)
    \end{align*}
    So, if $ u=g(x) $, then $ du=g^\prime(x)\,dx $.
\end{Proof}

General strategy: let $ u=\ldots $, $ du=\ldots\,dx $, then solve for
$ dx $, substitute in $ u $ and $ dx $, try to transform the integral into one
in terms of only $ u $.

\textbf{Good choices for $ \bm{u} $}:
\begin{itemize}
    \item $ u= $ a function whose derivative is present.
    \item $ u= $ base of an ugly power
    \item $ u= $ function inside another function; that is, inside $ \sin/\cos/\ln $,
          or in the exponent of $ e $.
\end{itemize}

\begin{Example}{}{}
    \begin{enumerate}[label=(\roman*)]
        \item Compute $ \displaystyle  \int \frac{\ln(x)}{x} \, d{x} $.

              \textbf{Solution.}
              \begin{align*}
                  \int \frac{\ln(x)}{x} \, d{x}
                   & =\int \frac{u}{x} (x)\, d{u} &  & u=\ln(x)\iff du=\frac{1}{x}\, dx \\
                   & =\int u\, d{u}                                                     \\
                   & =\frac{u^2}{2} +C                                                  \\
                   & =\frac{[\ln(x)]^2}{2} +C
              \end{align*}
        \item Compute $ \displaystyle \int \frac{\cos(\sqrt{x})}{\sqrt{x}} \, d{x} $.

              \textbf{Solution.}
              \begin{align*}
                  \int \frac{\cos(\sqrt{x})}{\sqrt{x}} \, d{x}
                   & = \int \frac{\cos(u)}{u}(2u) \, d{u} &  & u=\sqrt{x}\iff du=\frac{1}{2\sqrt{x}}\, dx \\
                   & =2 \int \cos(u)\, d{u}                                                               \\
                   & =2\sin(u)+C                                                                          \\
                   & =2\sin(\sqrt{x})+C
              \end{align*}
        \item Compute
              $ \displaystyle \int \frac{x^2}{\sqrt{x+1}} \, d{x} $.
              Don't forget to eliminate all of the $ x $'s!

              \textbf{Solution.}
              \begin{align*}
                  \int \frac{x^2}{\sqrt{x+1}} \, d{x}
                   & =\int \frac{(u-1)^2}{\sqrt{u}} \, d{u}                                                     &  & u=x+1\iff du=dx \\
                   & =\int \frac{u^2-2u+1}{\sqrt{u}} \, d{u}                                                                         \\
                   & =\int u^{\sfrac{3}{2}}-2u^{\sfrac{1}{2}}+u^{-\sfrac{1}{2}}\, d{u}                                               \\
                   & =\frac{2}{5} u^{\sfrac{5}{2} }-\frac{4}{3}u^{\sfrac{3}{2}}+2u^{\sfrac{1}{2}}+C                                  \\
                   & =\frac{2}{5} (x+1)^{\sfrac{5}{2} }-\frac{4}{3}(x+1)^{\sfrac{3}{2}}+2(x+1)^{\sfrac{1}{2}}+C
              \end{align*}
        \item Compute
              $ \displaystyle \int \sin^6(x)\cos(x)\, d{x} $.

              \textbf{Solution.}
              \begin{align*}
                  \int \sin^6(x)\cos(x)\, d{x}
                   & =\int u^6\, d{u}        &  & u=\sin(x)\iff du=\cos(x)\,dx \\
                   & =\frac{u^7}{7}+C                                          \\
                   & =\frac{\sin^7(x)}{7} +C
              \end{align*}
        \item Compute $ \displaystyle  \int x e^{5x^2}\, d{x} $.

              \textbf{Solution.}
              \begin{align*}
                  \int x e^{5x^2}\, d{x}
                   & =\int \frac{xe^u}{10x} \, d{u} &  & u=5x^2\iff du=10x\,dx \\
                   & =\frac{1}{10} \int e^u\, d{u}                             \\
                   & =\frac{e^u}{10} +C                                        \\
                   & =\frac{e^{5x^2}}{10} +C
              \end{align*}
    \end{enumerate}
\end{Example}

\subsection*{Substitution and Definite Integrals}
Q\@: What should we do with the limits of integration when making a substitution?

A\@: We should change them as well!

\begin{Theorem}{Change of Variable}{ch_of_var}
    If $ g^\prime(x) $ is continuous on $ \interval{a}{b} $ and $ f(x) $
    is continuous between $ g(a) $ and $ g(b) $, then
    \[ \int_{x=a}^{x=b} f(g(x))g^\prime(x)\, d{x} =
        \int_{u=g(a)}^{u=g(b)} f(u)\, d{u}  \]
\end{Theorem}

\begin{Proof}{\ref{thm:ch_of_var}}{}
    Let $ h(u) $ be an antiderivative of $ f(u) $. Then $ h(g(x)) $
    is an antiderivative of $ f(g(x))g^\prime(x) $.

    By FTC II,
    $ \displaystyle\int_{a}^{b} f(g(x))g^\prime(x)\, d{x} =h(g(b))-h(g(a)) $.

    But also,
    $ \displaystyle\int_{g(a)}^{g(b)} f(u)\, d{u} =h(g(b))-h(g(a)) $,
    so we get
    $ \displaystyle\int_{a}^{b} f(g(x))g^\prime(x)\, d{x} =\int_{g(a)}^{g(b)} f(u)\, d{u} $.
\end{Proof}

\begin{Example}{}{}
    \begin{enumerate}[label=(\roman*)]
        \item Compute $ \displaystyle  \int_{0}^{1} e^x\cos(e^x)\, d{x} $.

              \textbf{Solution.}
              \begin{align*}
                  \int_{0}^{1} e^x\cos(e^x)\, d{x}
                   & =\int_{1}^{e}\frac{u\cos(u)}{u} \, d{u} &  & u=e^x\iff du=e^x\,dx \\
                   & =\int_{1}^{e} \cos(u)\, d{u}                                      \\
                   & =\bigl[\sin(u)\bigr]_{1}^e                                        \\
                   & =\sin(e)-\sin(1)
              \end{align*}
        \item Compute $ \displaystyle \int_{0}^{1} \frac{x^3}{1+x^4} \, d{x} $.

              \textbf{Solution.}
              \begin{align*}
                  \int_{0}^{1} \frac{x^3}{1+x^4} \, d{x}
                   & =\int_{1}^{2} \frac{x^3}{u\cdot4x^3}\, d{u}   &  & u=1+x^4\iff du=4x^3\,dx \\
                   & =\frac{1}{4} \int_{1}^{2} \frac{1}{u} \, d{u}                              \\
                   & =\left[ \frac{\ln\abs{u}}{4}  \right]_{1}^2                                \\
                   & =\frac{\ln(2)}{4} -\frac{\ln(1)}{4}                                        \\
                   & =\frac{\ln(2)}{4}
              \end{align*}
    \end{enumerate}
\end{Example}

\begin{Remark}{}{}
    You can also leave the limits of integration in terms of $ x $ as long as
    you make it clear and don't forget to switch back to $ x $ at the end before
    plugging numbers in!
\end{Remark}

\begin{Example}{Tricky Change of Variable}{tricky_change_of_var}
    \begin{align*}
        \int \sec(x)\, d{x}
         & =\int \sec(x)\frac{\sec(x)+\tan(x)}{\sec(x)+\tan(x)}\, d{x}    \\
         & =\int \frac{\sec^2(x)+\sec(x)\tan(x)}{\sec(x)+\tan(x)} \, d{x} \\
         & =\int \frac{1}{u}\, d{u}                                       \\
         & =\ln\abs{u}+C                                                  \\
         & =\ln\abs{\sec(x)+\tan(x)}+C
    \end{align*}
    We made the substitution $ u=\sec(x)+\tan(x)\iff du=\sec(x)\tan(x)+\sec^2(x)\,dx $.
\end{Example}

\begin{Remark}{}{}
    The trick used in~\Cref{ex:tricky_change_of_var} only works for $ \sec(x) $ and
    $ \csc(x) $, so it's not useful to memorize.
\end{Remark}

\begin{Exercise}{}{}
    Compute:
    $ \displaystyle\int \tan(x)\, d{x} $, $ \displaystyle\int \cot(x)\, d{x} $.
\end{Exercise}
