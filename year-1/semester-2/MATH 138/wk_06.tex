\chapter{Numerical Series}
\section{Introduction to Series}
\begin{Definition}{Infinite series}{}
    Let $ \set{a_n}_{n=1}^\infty $ be a sequence. An
    \textbf{infinite series} is an expression of the form
    \[ a_1+a_2+a_3+a_4+\cdots=\sum\limits_{n=1}^{\infty}a_n \]
    This is a formal expression since we don't know what this means numerically.
\end{Definition}

\begin{Example}{Infinite series}{}
    \begin{enumerate}[label=(\roman*)]
        \item $ \displaystyle\sum\limits_{n=1}^{\infty} \frac{1}{n} =1+\frac{1}{2} +\frac{1}{3} +\frac{1}{4}+\cdots $
        \item $ \displaystyle\sum\limits_{n=1}^{\infty} \frac{n}{n+1} =\frac{1}{2} +\frac{2}{3} +\frac{3}{4} +\cdots $
        \item $ \displaystyle\sum\limits_{n=0}^{\infty} (-1)^n=1-1+1-1+1-\cdots $
    \end{enumerate}
\end{Example}
\underline{Overall Goal}: Determine of a given series of numbers \textbf{converges}
or \textbf{diverges}.

Wait a minute! What do these words mean for an infinite sum?!

Well, we need to somehow convert a series to a sequence, which we know how to take a limit of!

\begin{Definition}{Sequence of Partial Sums}{}
    If $ \sum\limits_{n=1}^{\infty} a_n $ is a series, define its
    \textbf{sequence of partial sums}, $ \set{S_n} $, as $ S_n=a_1+a_2+\cdots+a_n $
\end{Definition}

\begin{Example}{}{}
    For $ \sum\limits_{n=1}^{\infty} \sfrac{1}{n} $, $ S_1=1 $, $ S_2=1+\sfrac{1}{2}=\sfrac{3}{2} $,
    etc.
\end{Example}

Now, we can define convergence/divergence.

\begin{Definition}{Convergence, Divergence}{}
    A series $ \sum\limits_{n=1}^{\infty} a_n $ \textbf{converges} to $ S\in\mathbb{R} $
    if $ \lim\limits_{{n} \to {\infty}} S_n=S $. Here $ S $ is called the \textbf{sum}
    of the series.

    If $ \set{S_n} $ diverges, we say the series \textbf{diverges}.
\end{Definition}

\begin{Example}{}{}
    Using partial sums, determine if $ \displaystyle \sum_{n=0}^{\infty}(-1)^n $ converges
    or diverges.

    \textbf{Solution.} Partial sums:
    \begin{itemize}
        \item $ S_1=1 $
        \item $ S_2=0 $
        \item $ S_3=1 $
        \item $ S_4=0 $
        \item etc.
    \end{itemize}
    Clearly, $ \lim\limits_{{n} \to {\infty}} S_n $ does not exist, so
    the series diverges.
\end{Example}

\begin{Example}{}{}
    Using partial sums, determine if $ \displaystyle
        \sum\limits_{n=1}^{\infty} \left[\frac{1}{n} -\frac{1}{n+1}\right] $ converges
    or diverges.

    \textbf{Solution.} Partial sums:
    \begin{itemize}
        \item $ S_1=1-\sfrac{1}{2} $
        \item $ S_2=1-\sfrac{1}{2} +\sfrac{1}{2}-\sfrac{1}{3}=1-\sfrac{1}{3} $
        \item $ S_3=1-\sfrac{1}{2} +\sfrac{1}{2} -\sfrac{1}{3} -\sfrac{1}{3} -\sfrac{1}{4}=1-\sfrac{1}{4} $
    \end{itemize}
    There's a pattern! $ S_n=1-\dfrac{1}{n+1} $, so $
        \lim\limits_{{n} \to {\infty}} S_n=
        \lim\limits_{{n} \to {\infty}} \left[1-\dfrac{1}{n+1}\right]=1 $. Therefore, the series
    converges to 1.
\end{Example}

\begin{Remark}{}{}
    The above series is called a \textbf{Telescoping series}.
\end{Remark}

Let's examine a famous series: The \textbf{Harmonic series}.
\[ \sum\limits_{n=1}^{\infty}\frac{1}{n} \]
Does it converge or diverge? Say it converges to $ S $, so
\begin{align*}
    S
     & =\sum\limits_{n=1}^{\infty} \frac{1}{n}                                              \\
     & =\left( 1+\frac{1}{2} \right)+\left( \frac{1}{3} +\frac{1}{4} \right)+
    \left( \frac{1}{5} +\frac{1}{6} \right)+\left( \frac{1}{7} +\frac{1}{8}  \right)+\cdots \\
     & >\left( \frac{1}{2} +\frac{1}{2}  \right)+\left( \frac{1}{4}+\frac{1}{4}  \right) +
    \left( \frac{1}{6}+\frac{1}{6} \right)+\left( \frac{1}{8} +\frac{1}{8}  \right)+\cdots  \\
     & =1+\frac{1}{2} +\frac{1}{3} +\frac{1}{3} +\frac{1}{4} +\cdots                        \\
     & =S
\end{align*}
So $ S>S $, a contradiction. Thus, the series diverges.

\begin{Remark}{}{}
    There are many proofs that the Harmonic series diverges, but this is my favourite.
    First published in 1976!
\end{Remark}
It turns out that finding sums of series is hard in general. The partial
sum method rarely works, but it does work for telescoping series, as we saw.

It also works for \emph{geometric series}, let's explore these now!

\section{Geometric Series}
\begin{Definition}{Geometric Series}{}
    A \textbf{geometric series} is a series of the form
    \[ \sum\limits_{n=0}^{\infty} r^n=1+r+r^2+\cdots+r^n+\cdots \]
    for some $ r\in\mathbb{R} $.
\end{Definition}
Let's figure out where the series converges!

\underline{Case 1}: $ r=1 $. Then the series is
$ \sum\limits_{n=0}^{\infty} 1=1+1+1+1+\cdots $.
So $ S_n=n $, and $ \lim\limits_{{n} \to {\infty}} S_n =\infty $, so the series
diverges in this case.

\underline{Case 2}: $ r=-1 $. Then the series is $ \sum\limits_{n=0}^{\infty} (-1)^n $
which we know diverges from before.

\underline{Case 3}: $ r\neq 1 $. Then
\[ S_n=1+r+r^2+\cdots+r^n \]
and
\[ rS_n=r+r^2+\cdots+r^{n+1} \]
So, $ S_n-rS_n=1-r^{n+1} $, therefore
\[ S_n=\frac{1-r^{n+1}}{1-r} \]
Thus, $ \lim\limits_{{n} \to {\infty}} S_n=\lim\limits_{{n} \to {\infty}} \dfrac{1-r^{n+1}}{1-r}
    =
    \left( \dfrac{1}{1-r} \right) \lim\limits_{{n} \to {\infty}} \left[1-r^{n+1}\right] $.

Clearly, $ \lim\limits_{{n} \to {\infty}} \left[1-r^{n+1}\right] $ diverges if $ \abs{r}>1 $,
but if $ \abs{r}<1 $ then $ \lim\limits_{{n} \to {\infty}} \left[1-r^{n+1}\right]=1 $.
Thus,
\[ \boxed{ \sum\limits_{n=0}^{\infty} r^n=\frac{1}{1-r}} \]
if $ \boxed{\abs{r}<1} $ and diverges otherwise.

\section{Arithmetic of Series}
Before we explore geometric series further, let's look at some arithmetic properties of series.

\begin{Theorem}{}{sum_properties}
    Suppose $ \sum\limits_{n=1}^{\infty} a_n=A $ and $ \sum\limits_{n=1}^{\infty} b_n=B $
    and $ k\in\mathbb{R} $.
    \begin{enumerate}[label=(\arabic*)]
        \item $ \sum\limits_{n=1}^{\infty} ka_n=kA $
        \item $ \sum\limits_{n=1}^{\infty} a_n\pm b_n=A\pm B $
    \end{enumerate}
\end{Theorem}

\begin{Proof}{\ref{thm:sum_properties}}{}
    Follows from limit properties.
\end{Proof}

Also, if we know something about the tail of a series, then we can draw conclusions about
the whole series!

\begin{Theorem}{}{convergence_tail}
    If $ \sum\limits_{n=1}^{\infty} a_n $ converges, then $ \sum\limits_{n=j}^{\infty} a_n $
    also converges for each $ j\geqslant 1 $.

    If $ \sum\limits_{n=j}^{\infty} a_n $ converges for some $ j $,
    then $ \sum\limits_{n=1}^{\infty} a_n $ converges.
\end{Theorem}

\begin{Proof}{\ref{thm:convergence_tail}}{}
    \[ \sum\limits_{n=1}^{\infty} a_n=a_1+a_2+\cdots+a_{j-1}+\sum\limits_{n=j}^{\infty} a_n \]
    and $ a_1+a_2+\cdots+a_{j-1}\in\mathbb{R} $, does not affect convergence.
    So, convergence only depends on the tail! Changing finitely many terms will not
    affect convergence.
\end{Proof}

\setcounter{section}{2}
\begin{Example}{}{}
    Determine whether the following series is convergent or divergent. If a series is
    convergent, find its sum.
    \begin{enumerate}[label=(\roman*)]
        \item $ \displaystyle\sum\limits_{n=0}^{\infty} \left( \frac{2}{3}  \right)^n $

              \textbf{Solution.} Converges to
              $ \displaystyle \frac{1}{1-\sfrac{2}{3}}=3 $.
        \item $ \displaystyle \sum\limits_{n=1}^{\infty} \frac{1}{2^n} $

              \textbf{Solution.} Converges to
              \[ \sum\limits_{n=1}^{\infty} \frac{1}{2^n}
                  =\frac{1}{2} \sum\limits_{n=1}^{\infty} \frac{1}{2^{n-1}}
                  =\frac{1}{2} \sum\limits_{n=0}^{\infty} \frac{1}{2^n}
                  =\frac{1}{2} \sum\limits_{n=0}^{\infty} \left( \frac{1}{2}  \right)^n
                  =\frac{1}{2} \left( \frac{1}{1-\sfrac{1}{2}}  \right)
                  =1. \]
        \item $ \displaystyle\sum\limits_{n=0}^{\infty} 3\left( \frac{3}{2}  \right)^n $

              \textbf{Solution.} Diverges since $ \abs{r}=\sfrac{3}{2} >1 $.
        \item $ \displaystyle\sum\limits_{n=0}^{\infty} 2^{3n}3^{-2n} $

              \textbf{Solution.} Converges to
              \[
                  \sum\limits_{n=0}^{\infty} 2^{3n}3^{-2n}
                  =\sum\limits_{n=0}^{\infty} \frac{8^n}{9^n}
                  =\sum\limits_{n=0}^{\infty}
                  \left( \frac{8}{9}  \right)^n
                  =\frac{1}{1-\sfrac{8}{9}}
                  =9. \]
        \item $ \displaystyle\sum\limits_{n=1}^{\infty} (25) 8^n5^{-n} $

              \textbf{Solution.} Diverges since $ \abs{r}=\sfrac{8}{5} >1 $.
    \end{enumerate}
\end{Example}

\subsection*{Interesting Application: Decimals to Fractions}
We can use geometric series to write an infinite repeating decimal as a fraction!

\begin{Example}{}{}
    Use the geometric series to write the decimal
    $ 3.2131313\cdots=3.2\overline{13} $ as a fraction (it doesn't need to be in lowest terms).

    \textbf{Solution.}
    \begin{align*}
        3.2131313\cdots
         & =3.2+\frac{13}{10^3} +\frac{13}{10^5} +\frac{13}{10^7} +\cdots                            \\
         & =\frac{32}{10} +\frac{13}{10^3} \left( 1+\frac{1}{100} +\frac{1}{100^2}+\cdots  \right)   \\
         & =\frac{32}{10} +\frac{13}{10^3} \sum\limits_{n=0}^{\infty} \left( \frac{1}{100} \right)^n \\
         & =\frac{32}{10} +\frac{13}{1000} \left( \frac{1}{1-\sfrac{1}{100} } \right)                \\
         & =\frac{3181}{900}
    \end{align*}
\end{Example}

\subsection*{A Real-World Application}
Suppose a spaceship is firing a laser beam at a planet with two layers of shields. These shields reflect
one-third of the beam, absorb five-ninths of the beam, and transmit one-ninth of the beam.
If the beam has initial intensity $ I $, what fraction is transmitted to the other side?

\textbf{Solution.} The total that gets through is:
\[
    \frac{I}{81} +\frac{I}{9(81)} +\frac{I}{9^2(81)} +\cdots
    =\frac{I}{81} \sum\limits_{n=0}^{\infty} \left( \frac{1}{9}  \right)^n
    =\frac{I}{81} \left( \frac{1}{1-\sfrac{1}{9}}  \right)
    =\frac{I}{81} \left( \frac{9}{8}  \right)
    =\frac{I}{72}
\]

Finding sums of series, in general, is hard. While we can do it for geometric
series and telescoping series, we can't usually get a nice formula for $ S_n $,
the partial sums

Soon, we will prove that $ \displaystyle \sum\limits_{n=1}^{\infty} \frac{1}{n^2}  $ converges,
but what is the sum?

First a few partial sums:
\begin{itemize}
    \item $ S_1=1 $
    \item $ S_2=1+\sfrac{1}{4} =1.25 $
    \item $ S_3=1+\sfrac{1}{4} +\sfrac{1}{9} =1.36111\ldots $
    \item $ S_4=1.4236\ldots $
\end{itemize}
Sum? $ 1.75 $? $ 2 $? $ \sfrac{\pi^2}{6} $? Yes!
\[ \boxed{\sum\limits_{n=1}^{\infty} \frac{1}{n^2} =\frac{\pi^2}{6}} \]

This is difficult to prove though!

So, from now on, we will focus on determining
if a series converges or diverges, and not actually finding
the sum.

Let's start developing some tests for convergence/divergence.

\section{Divergence Test}
First, let us  prove a theorem:
\begin{Theorem}{}{div_test_contrapositive}
    If $ \sum\limits_{n=1}^{\infty} a_n $ converges, then $ \lim\limits_{{n} \to {\infty}} a_n=0 $.
\end{Theorem}
\begin{Proof}{\ref{thm:div_test_contrapositive}}{}
    Suppose $ \sum\limits_{n=1}^{\infty} a_n $ converges, say $ \sum\limits_{n=1}^{\infty} a_n=S $.

    Let $ \set{S_n} $ be a sequence of partial sums, so
    $ S_n=a_1+a_2+\cdots+a_n $
    and $ \lim\limits_{{n} \to {\infty}} S_n=S $. By sequence of limit properties we get
    $ \lim\limits_{{n} \to {\infty}} S_{n-1}=S $ too, and $ S_n-S_{n-1}=a_n $. Thus,
    $ \lim\limits_{{n} \to {\infty}} a_n=\lim\limits_{{n} \to {\infty}} S_{n}-S_{n-1}=S-S=0 $.
\end{Proof}

The Divergence Test is the contrapositive of the above theorem:

\begin{Theorem}{Divergence Test}{}
    If $ \lim\limits_{{n} \to {\infty}} a_n\neq 0 $ (or DNE), then
    $ \sum\limits_{n=1}^{\infty} a_n $ diverges.
\end{Theorem}

\begin{Remark}{}{}
    This test tells only ever tells us a series \textbf{diverges}, never that a series
    converges. So, if you are checking $ \sum\limits_{n=1}^{\infty} a_n $, if
    $ \lim\limits_{{n} \to {\infty}} a_n $
    \begin{itemize}
        \item $ =0 $, get no info;
        \item $ \neq 0 $, series diverges.
    \end{itemize}
\end{Remark}

Q\@: Can you think of a series such that $ \displaystyle \sum\limits_{n=1}^{\infty} a_n $ diverges,
even though $ \displaystyle \lim\limits_{{n} \to {\infty}} a_n=0 $?

A\@: Yes! The Harmonic Series $ \displaystyle \sum\limits_{n=1}^{\infty}\frac{1}{n} $.

So, be careful!

When to use the Divergence Test: First! It is a good idea to see if the test works before
moving on to more complicated tests!

\begin{Example}{}{}
    Use Divergence Test to draw a conclusion
    for the following series.
    \begin{enumerate}[label=(\roman*)]
        \item $ \displaystyle\sum\limits_{n=1}^{\infty} \frac{n}{n+1} $

              \textbf{Solution.}

              $ \displaystyle \lim\limits_{{n} \to {\infty}} \frac{n}{n+1} =1\neq 0 $, so the series diverges
              by the Divergence Test.
        \item $ \displaystyle\sum\limits_{n=1}^{\infty} \frac{2^{n}+3^n}{2^n} $

              \textbf{Solution.}
              $ \displaystyle  \lim\limits_{{n} \to {\infty}} \frac{2^n+3^n}{2^n} =\lim\limits_{{n} \to {\infty}}
                  \left[ 1+\left( \frac{3}{2} \right)^n  \right]=\infty\neq 0 $,
              so the series diverges by the Divergence Test.
        \item $ \displaystyle\sum\limits_{n=1}^{\infty} \frac{1}{n^2} $

              \textbf{Solution.}
              $ \displaystyle \lim\limits_{{n} \to {\infty}}
                  \frac{1}{n^2} =0 $, so the Divergence Test fails!
        \item $ \displaystyle\sum\limits_{n=1}^{\infty} \arctan(n) $

              \textbf{Solution.} $ \displaystyle \lim\limits_{{n} \to {\infty}}
                  \arctan(n)=\frac{\pi}{2}\neq 0 $, so the series diverges
              by the Divergence Test.
    \end{enumerate}
\end{Example}
