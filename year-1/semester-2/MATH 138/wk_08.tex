\section{Absolute versus Conditional Convergence}
So far we've examined tests for positive series, and a test
for alternating series, but what can we do about the series that have
non-alternating assortment of positive and negative terms?
Is there a way to make our tests for positive series work in this case too?
Yes! Just like for improper integrals, we use absolute values
and discuss absolute convergence!

\begin{Definition}{Absolutely convergent}{}
    A series $ \sum\limits_{n=1}^{\infty} a_n $ is \textbf{absolutely convergent}
    if $ \sum\limits_{n=1}^{\infty} \abs{a_n} $ converges.
\end{Definition}

\begin{Remark}{}{}
    Note that if a series only has positive terms then absolute convergence
    is the same as convergence because $ \sum\limits_{n=1}^{\infty} \abs{a_n}=
        \sum\limits_{n=1}^{\infty} a_n $.
\end{Remark}

\begin{Example}{}{}
    \begin{enumerate}[label=(\roman*)]
        \item Is $ \displaystyle \sum\limits_{n=1}^{\infty} \frac{(-1)^n}{n^2} $ absolutely
              convergent?

              \textbf{Solution.} Yes, $ \displaystyle \sum\limits_{n=1}^{\infty} \abs*{\frac{(-1)^n}{n^2}}
                  =\sum\limits_{n=1}^{\infty} \frac{1}{n^2} $ converges. ($ p $-series, $ p=2>1 $)
        \item Is $ \displaystyle \sum\limits_{n=1}^{\infty} \frac{(-1)^{n+1}}{n} $ absolutely
              convergent?

              \textbf{Solution.} No, $ \displaystyle \sum\limits_{n=1}^{\infty} \abs*{\frac{(-1)^{n+1}}{n}}=
                  \sum\limits_{n=1}^{\infty} \frac{1}{n} $ diverges. (Harmonic Series)
    \end{enumerate}
\end{Example}

\begin{Remark}{}{}
    Don't say ``absolutely divergent,'' that makes it sound worse than it is!
\end{Remark}

We know $ \displaystyle \sum\limits_{n=1}^{\infty} \frac{(-1)^{n+1}}{n} $ converges by AST,
and we have a name for the series that behave this way:

\begin{Definition}{Conditionally convergent}{}
    A series is \textbf{conditionally convergent} if it is convergent, but
    not absolutely convergent.
\end{Definition}

We also have an analogue of the ACT for series:

\begin{Theorem}{ACT}{act}
    If $ \sum\limits_{n=1}^{\infty} \abs{a_n} $ converges, then $ \sum\limits_{n=1}^{\infty}a_n $
    converges.
\end{Theorem}

\begin{Remark}{}{}
    Note that unless $ a_n\geqslant 0 $ for all $ n $, $ \sum\limits_{n=1}^{\infty} \abs{a_n} $
    and $ \sum\limits_{n=1}^{\infty} a_n $ will converge to different values!
\end{Remark}

\begin{Proof}{\ref{thm:act}}{}
    (Similar to the proof of ACT for integrals). Suppose $ \sum\limits_{n=1}^{\infty}\abs{a_n} $
    converges. Note that $ 0\leqslant a_n+\abs{a_n}\leqslant 2\abs{a_n} $. Since
    $ \sum\limits_{n=1}^{\infty} \abs{a_n} $ converges, $ \sum\limits_{n=1}^{\infty} 2\abs{a_n} $
    converges, so by comparison $ \sum\limits_{n=1}^{\infty} \left( a_n+\abs{a_n} \right) $
    converges, too. But then
    \[ \sum\limits_{n=1}^{\infty}a_n
        =\sum\limits_{n=1}^{\infty} \left[ (a_n+\abs{a_n}) -\abs{a_n}\right]
        =\underset{\text{converges}}{\sum\limits_{n=1}^{\infty} \left( a_n+\abs{a_n} \right)}
        -\underset{\text{converges}}{\sum\limits_{n=1}^{\infty} \abs{a_n}}\]
    So $ \sum\limits_{n=1}^{\infty} a_n $ converges.
\end{Proof}

Cool! So, to prove a series converges we can prove it is absolutely convergent
instead, which allows us to use tests like Integral/Comparison/Limit comparison!

\begin{Example}{}{}
    Is $ \displaystyle \sum\limits_{n=1}^{\infty} \frac{\sin(n^3)}{n^3} $ convergent?

    \textbf{Solution.} Let's check absolute convergence!
    $ \displaystyle \sum\limits_{n=1}^{\infty} \abs*{\frac{\sin(n^3)}{n^3}} $: We know
    $ \displaystyle 0\leqslant \abs*{\frac{\sin(n^3)}{n^3}}\leqslant \frac{1}{n^3} $
    and $ \displaystyle \sum\limits_{n=1}^{\infty} \frac{1}{n^3} $ converges ($ p $-series, $ p=3>1 $),
    so $ \displaystyle \sum\limits_{n=1}^{\infty} \abs*{\frac{\sin(n^3)}{n^3} } $
    converges by comparison. So, the given series converges by ACT\@.
\end{Example}

A typical question will ask ``are the following series absolutely convergent,
conditionally convergent, or divergent,'' then you should:
\begin{itemize}
    \item Step 1: Try the Divergence Test
    \item Step 2: Check absolute convergence with old tests for positive series.
    \item Step 3: Check conditional convergence with AST\@.
\end{itemize}

\begin{Remark}{}{}
    Doing Step 2 before Step 3 is a good idea since if the series converges
    absolutely you're done, while if it converges by AST you still need
    to check absolute convergence.
\end{Remark}

\begin{Example}{}{}
    Do the following series converge absolutely, conditionally, or diverge?
    \begin{enumerate}[label=(\roman*)]
        \item $ \displaystyle \sum\limits_{n=1}^{\infty} \frac{(-1)^n4^n}{3^n} $

              \textbf{Solution.} Divergence Test: $ \displaystyle \lim\limits_{{n} \to {\infty}}
                  \frac{(-1)^n4^n}{3^n} $ does not exist, so the series diverges.
        \item $ \displaystyle \sum\limits_{n=1}^{\infty} \frac{(-1)^n\sqrt{n^2+n}}{n^{\sfrac{3}{2}}} $

              \textbf{Solution.} Note the Divergence Test fails. Check absolute convergence:
              \[ \sum\limits_{n=1}^{\infty} \frac{(-1)^n\sqrt{n^2+n}}{n^{\sfrac{3}{2}}}=
                  \sum\limits_{n=1}^{\infty}\frac{\sqrt{n^2+n}}{n^{\sfrac{3}{2}}} \]
              LCT with $ \displaystyle \sum\limits_{n=1}^{\infty} \frac{1}{\sqrt{n}} $:
              \[ \lim\limits_{{n} \to {\infty}}
                  \frac{\left( \dfrac{\sqrt{n^2+n}}{n^{\sfrac{3}{2}}} \right)}{
                      \left( \dfrac{1}{\sqrt{n}} \right)
                  }
                  =\lim\limits_{{n} \to {\infty}} \frac{\sqrt{n^2+n}}{n}
                  =\lim\limits_{{n} \to {\infty}}\sqrt{1+\frac{1}{n}}=1 \]
              So since $ \displaystyle \sum\limits_{n=1}^{\infty} \frac{1}{\sqrt{n}} $ diverges, so does
              the given series, so our series is \emph{not} absolutely convergent. Let's check
              AST\@: Clearly $ \displaystyle \lim\limits_{{n} \to {\infty}} \frac{\sqrt{n^2+n}}{n^{\sfrac{3}{2}}}=0 $,
              but is the sequence decreasing? We could use derivatives to check, but instead let's try
              it directly:
              \begin{align*}
                  \frac{\sqrt{(n+1)^2+(n+1)}}{(n+1)^{\sfrac{3}{2}}}<\frac{\sqrt{n^2+n}}{n^{\sfrac{3}{2}}}
                   & \iff n^{\sfrac{3}{2}}\sqrt{n^2+2n+1+n+1}<\sqrt{n^2+n}(n+1)^{\sfrac{3}{2}} \\
                   & \iff n^3(n^2+3n+2)<(n^2+n)(n+1)^3                                         \\
                   & \iff n^5+3n^4+2n^3<(n^2+n)(n^3+3n^2+3n+1)                                 \\
                   & \iff 2n^3<3n^3+n^2+n^4+3n^3+3n^2+n                                        \\
                   & \iff 0<n^4+4n^3+4n^2+n
              \end{align*}
              which is true for all $ n\geqslant 1 $. So the sequence is decreasing, which means
              the series converges by AST\@. Thus, our given series is conditionally convergent.
        \item $ \displaystyle \sum\limits_{n=1}^{\infty} \frac{e^n}{\pi^n} $

              \textbf{Solution.} Note the Divergence Test fails. Wait a minute! This series only has positive
              terms! So absolute convergence = convergence! Also, it's a geometric series
              ($ \abs{r}=\sfrac{e}{\pi} <1 $), so the series \emph{converges absolutely}.
    \end{enumerate}
\end{Example}

\subsection*{Aside on Rearranging The Order of a Series}
We have been discussing ``sums'' of infinite series, but it turns out
that these can behave very strangely!

Q\@: Can we rearrange the order in which we sum the series?

A\@: Sometimes! If $ \sum\limits_{n=1}^{\infty} a_n $ is absolutely convergent
with sum $ S $, then any rearrangement of the terms will also have the sum $ S $.

What about conditional convergence? Let's see! Soon, we will prove that the sum
of the alternating Harmonic Series is $ \ln(2) $.

$ (\star) $:
\[
    1-\frac{1}{2} +\frac{1}{3} -\frac{1}{4} +\frac{1}{5} -\cdots=\ln(2)
\]

Divide by $ 2 $:
\[ \frac{1}{2} -\frac{1}{4} +\frac{1}{6} -\frac{1}{8} +\frac{1}{10} -\cdots=\frac{1}{2}\ln(2) \]

$ (\star\star) $ Add in some $ 0 $'s:
\[ 0+\frac{1}{2}+0-\frac{1}{4} +0+\frac{1}{6} +0-\frac{1}{8} +\cdots=
    \frac{1}{2} \ln(2) \]
$ (\star\star\star) $ Computing $ (\star) + (\star\star)$:
\[ 1+\frac{1}{3} -\frac{1}{2} +\frac{1}{5} +\frac{1}{7} -\frac{1}{4} +\cdots
    =\frac{3}{2} \ln(2)
\]
But $ (\star\star\star) $ is a rearrangement of $ (\star) $. So changing the order changes the sum!

In fact, Riemann proved that if $ \sum\limits_{n=1}^{\infty} a_n $ is conditionally
convergent, then by rearranging we can make it add up to \emph{any} real number
(or $ \pm\infty $)!

\section{Ratio and Root Tests}
We have two more tests to examine!

\begin{Theorem}{Ratio Test}{ratio}
    Let $ \sum\limits_{n=1}^{\infty} a_n $ be a series, and assume
    $ \displaystyle \lim\limits_{{n} \to {\infty}} \abs*{\frac{a_{n+1}}{a_n} }=L $,
    with $ L\in\mathbb{R} $ or $ L=\infty $.
    \begin{enumerate}[label=(\arabic*)]
        \item\label{ratio_1} If $ L<1 $, then $ \sum\limits_{n=1}^{\infty} a_n $ is absolutely convergent.
        \item\label{ratio_2} If $ L>1 $ (or $ L=\infty $), then $ \sum\limits_{n=1}^{\infty} a_n $
              diverges.
        \item\label{ratio_3} If $ L=1 $, we get no info.
    \end{enumerate}
\end{Theorem}
\begin{Proof}{\ref{thm:ratio}}{}
    \underline{Proof of~\ref{ratio_1}} Suppose $ L<1 $ (Idea: compare to a geometric series!)
    Since $ L<1 $, we can pick $ r\in\mathbb{R} $ with $ L<r<1 $. Since
    $ \displaystyle \lim\limits_{{n} \to {\infty}} \abs*{\dfrac{a_{n+1}}{a_n}}=L<r $,
    for large enough $ n $, say $ n\geqslant N $, $ \abs*{\dfrac{a_{n+1}}{a_n} }<r $,
    or $ \abs{a_{n+1}}<r\abs{a_n} $. So
    \begin{align*}
         & \abs{a_{N+1}}<r\abs{a_N}                  \\
         & \abs{a_{N+2}}<r\abs{a_{N+1}}<r^2\abs{a_N} \\
         & \abs{a_{N+3}}<\cdots r^3\abs{a_N}
    \end{align*}
    $ (\star) $ In general: $ \abs{a_{N+K}}<\cdots<r^k\abs{a_N} $.

    Furthermore, $ \sum\limits_{n=1}^{\infty} \abs{a_N}r^n $ converges (geometric series,
    $ r<1 $). So, by $ (\star) $ and comparison, $ \sum\limits_{n=N+1}^{\infty} \abs{a_n} $
    converges, but then so does $ \sum\limits_{n=1}^{\infty} \abs{a_n} $. Thus,
    $ \sum\limits_{n=1}^{\infty} a_n $ converges absolutely.

    \underline{Proof of~\ref{ratio_2}} Suppose $ \displaystyle \lim\limits_{{n} \to {\infty}} \abs*{\frac{a_{n+1}}{a_n} }=L>1 $
    (or $ L=\infty $). Then eventually $ \displaystyle \abs*{\frac{a_{n+1}}{a_n}}>1 $,
    or $ \abs{a_{n+1}}>\abs{a_n}>0 $. Hence, the size of the terms is increasing eventually,
    so $ \lim\limits_{{n} \to {\infty}} a_n\neq 0 $. Therefore, the
    series diverges by the Divergence Test.

    \underline{Proof of~\ref{ratio_3}} Consider
    $ \displaystyle \sum\limits_{n=1}^{\infty} \frac{1}{n} $ and
    $ \displaystyle \sum\limits_{n=1}^{\infty}
        \frac{1}{n^2}  $. In both cases $ L=1 $, but one converges and the one diverges. So,
    if $ L=1 $ we get no info.
\end{Proof}

\begin{Remark}{}{}
    When to use:
    \begin{itemize}
        \item Factorials! (see what I did there?!)
        \item Also works on some ``almost'' geometric series
    \end{itemize}
    If you see a factorial (after simplifying), use the Ratio Test first! Even
    before the Divergence Test.
\end{Remark}

\begin{Example}{}{}
    Determine whether the following series are absolutely convergent,
    conditionally convergent, or divergent.
    \begin{enumerate}[label=(\roman*)]
        \item $ \displaystyle \sum\limits_{n=1}^{\infty} \frac{3^n}{n!} $

              \textbf{Solution.} Ratio Test:
              \[ \lim\limits_{{n} \to {\infty}} \abs*{\frac{a_{n+1}}{a_n}}=
                  \lim\limits_{{n} \to {\infty}} \frac{3^{n+1}}{(n+1)!}\left( \frac{n!}{3^n} \right)
                  =\lim\limits_{{n} \to {\infty}} \frac{3}{n+1}=0<1 \]
              So the given series \emph{converges absolutely}.
        \item $ \displaystyle \sum\limits_{n=1}^{\infty} \frac{(-1)^n9^n}{n2^n} $

              \textbf{Solution.} Ratio Test:
              \[ \lim\limits_{{n} \to {\infty}} \abs*{\frac{a_{n+1}}{a_n}}=
                  \lim\limits_{{n} \to {\infty}}
                  \abs*{\frac{(-1)^{n+1}9^{n+1}}{(n+1)2^{n+1}}
                  \left[ \frac{n2^n}{(-1)^n9^n} \right]}
                  =\lim\limits_{{n} \to {\infty}} (9)\left( \frac{1}{2} \right)
                  \left( \frac{n}{n+1} \right)
                  =\frac{9}{2} >1 \]
              So, the series \emph{diverges}.
        \item $ \displaystyle \sum\limits_{n=1}^{\infty}\frac{n^n}{n!} $

              \textbf{Solution.} Ratio Test:
              \[ \lim\limits_{{n} \to {\infty}} \abs*{\frac{(n+1)^{n+1}}{(n+1)!}
                      \left( \frac{n!}{n^n} \right)}
                  =\lim\limits_{{n} \to {\infty}} \frac{(n+1)^{n+1}}{(n+1)n^n}
                  =\lim\limits_{{n} \to {\infty}} \frac{(n+1)^n}{n^n}
                  =\lim\limits_{{n} \to {\infty}} \left( \frac{n+1}{n}  \right)^n=e>1  \]
              So, the series \emph{diverges}. This shows $ n^n\gg n! $ as $ n\to\infty $.
        \item $ \displaystyle \sum\limits_{n=1}^{\infty} \frac{n^2+3n}{5^n}  $

              \textbf{Solution.} ``Almost geometric,'' Ratio Test:
              \[ \lim\limits_{{n} \to {\infty}}
                  \abs*{\frac{(n+1)^2+3(n+1)}{5^{n+1}}
                      \left( \frac{5^n}{n^2+3n} \right)}
                  =\lim\limits_{{n} \to {\infty}} \frac{1}{5} \left( \frac{n^2+2n+1+3n+3}{n^2+3n} \right)
                  =\frac{1}{5}
                  <1 \]
              So, the series \emph{converges absolutely}.
        \item $ \displaystyle \sum\limits_{n=1}^{\infty} \frac{n^2+2n+1}{3n^4+4} $

              \textbf{Solution.} Should use LCT (exercise), but what if we use the Ratio Test?
              \begin{align*}
                  \lim\limits_{{n} \to {\infty}}  &
                  \abs*{\frac{(n+1)^2+2(n+1)+1}{3(n+1)^4+4}
                  \left( \frac{3n^4+4}{n^2+2n+1} \right)}                                    \\
                  =\lim\limits_{{n} \to {\infty}} & \left[ \frac{(n+1)^2+2(n+1)+1}{n^2+2n+1}
                      \left( \frac{3n^4+4}{3(n+1)^4+4} \right)  \right] =1
              \end{align*}
              Ratio Test fails! We have one more test to examine: the Root Test!
    \end{enumerate}
\end{Example}

We have one more test to examine: the Root Test!

\begin{Theorem}{Root Test}{}
    Let $ \sum\limits_{n=1}^{\infty} a_n $ be a series and assume $ \lim\limits_{{n} \to {\infty}}
        \sqrt[n]{\abs{a_n}}=L\in\mathbb{R} $ or $ L=\infty $.
    \begin{enumerate}[label=(\arabic*)]
        \item If $ L<1 $, then $ \sum\limits_{n=1}^{\infty} a_n $ converges absolutely.
        \item If $ L>1 $, then $ \sum\limits_{n=1}^{\infty} a_n $ diverges.
        \item If $ L=1 $, then we get no info.
    \end{enumerate}
\end{Theorem}
The proof is similar to the Ratio Test proof.

\begin{Remark}{}{}
    When to use:
    \begin{itemize}
        \item When all the terms of a series are raised to the power of $ n $
    \end{itemize}
    Warning:
    \begin{itemize}
        \item If the Ratio Test fails ($ L=1 $), then the Root Test will also fail
              and vice versa.
    \end{itemize}
\end{Remark}

\begin{Example}{}{}
    $ \displaystyle \sum\limits_{n=1}^{\infty}\left( \frac{n+1}{3n+7} \right)^n $

    \textbf{Solution.} Root Test:
    \[ \lim\limits_{{n} \to {\infty}} \abs*{\left( \frac{n+1}{3n+7} \right)^n}^{\sfrac{1}{n}}
        =\lim\limits_{{n} \to {\infty}} \frac{n+1}{3n+7}
        =\frac{1}{3}
        <1 \]
    So the series \emph{converges absolutely}.
\end{Example}

To help us predict if a series will converge or diverge, let's examine the
relative sizes of common functions:
\[ \boxed{\left[ \ln(n) \right]^p\ll n^p\ll x^n\ll n!\ll n^n} \]
for $ \abs{x}>1 $.

We have seen most of these already, but let's prove one more:
\begin{Theorem}{}{xn_grow}
    For any $ x\in\mathbb{R} $,
    \[ \lim\limits_{{n} \to {\infty}} \frac{x^n}{n!} =0 \]
\end{Theorem}

\begin{Proof}{\ref{thm:xn_grow}}{}
    Using the Ratio Test, we can show that $ \displaystyle\sum\limits_{n=1}^{\infty} \frac{x^n}{n!} $
    converges for any $ x\in\mathbb{R} $. Therefore, by the Divergence Test,
    $ \displaystyle \lim\limits_{{n} \to {\infty}} \frac{x^n}{n!} =0 $.
\end{Proof}
This shows that $ x^n\ll n! $.

\subsection*{Series Test Recap}
\begin{itemize}
    \item Sums of Geometric and Telescoping Series
          \begin{itemize}
              \item Try to spot these series.
              \item If the question says ``find the sum,'' it's likely one of these.
          \end{itemize}
    \item Divergence Test (Any Series)
          \begin{itemize}
              \item Try this first, unless there is a factorial.
          \end{itemize}
    \item Integral Test (Positive Series)
          \begin{itemize}
              \item Last resort when all else fails.
              \item Don't forget continuous, positive, and decreasing.
          \end{itemize}
    \item $ p $-series $ \left( \sum \dfrac{1}{n^p} \right) $
          \begin{itemize}
              \item Good for Comparison and Limit Comparison Test.
          \end{itemize}
    \item Comparison Test (Positive Series)
          \begin{itemize}
              \item Also a last resort, LCT is usually better.
          \end{itemize}
    \item LCT (Positive Series)
          \begin{itemize}
              \item Series of the form $ \dfrac{\text{powers of }n}{\text{powers of }n} $.
              \item ``Almost'' geometric series.
              \item Don't forget: $ L=0 $ or $ L=0 $ are more complicated!
          \end{itemize}
    \item Ratio Test (Any Series)
          \begin{itemize}
              \item Factorials!
              \item ``Almost'' geometric series.
              \item $ L=1 $ gives no info.
          \end{itemize}
    \item Root Test (Any Series)
          \begin{itemize}
              \item When all terms have a power of $ n $.
          \end{itemize}
          All the above tests can only discuss absolute convergence or divergence.
    \item AST (Alternating Series)
          \begin{itemize}
              \item For proving conditional convergence.
          \end{itemize}
\end{itemize}

\begin{Exercise}{Series Practice}{series_practice}
    Determine whether the following series are absolutely convergent,
    conditionally convergent, or divergent.
    \begin{enumerate}[label=(\roman*)]
        \item $ \displaystyle \sum\limits_{n=1}^{\infty}\frac{5^{2n}}{n!} $.
        \item $ \displaystyle \sum\limits_{n=1}^{\infty} \frac{n^2+n}{n^3-3n+1} $
        \item $ \displaystyle \sum\limits_{n=1}^{\infty} \frac{n^3+7}{n^3+n^2} $.
        \item $ \displaystyle \sum\limits_{n=1}^{\infty} \frac{\left[ \ln(n) \right]^3}{\sqrt{n}} $.
        \item $ \displaystyle \sum\limits_{n=1}^{\infty} (\sqrt[n]{2}-1)^n $.
        \item $ \displaystyle \sum\limits_{n=1}^{\infty} \frac{(-1)^n(n^2+1)}{n^3+3} $.
    \end{enumerate}
\end{Exercise}
\textbf{Solutions to~\ref{exercise:series_practice}}:
\begin{enumerate}[label=(\roman*)]
    \item Ratio Test (absolutely convergent)
    \item LCT (diverges)
    \item Divergence Test (diverges)
    \item LCT or Comparison (diverges)
    \item Root Test (converges absolutely)
    \item LCT and AST (converges conditionally)
\end{enumerate}
