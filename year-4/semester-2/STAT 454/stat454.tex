\documentclass[oneside]{book}\usepackage[]{graphicx}\usepackage[svgnames]{xcolor}
% maxwidth is the original width if it is less than linewidth
% otherwise use linewidth (to make sure the graphics do not exceed the margin)
\makeatletter
\def\maxwidth{ %
  \ifdim\Gin@nat@width>\linewidth
    \linewidth
  \else
    \Gin@nat@width
  \fi
}
\makeatother

\definecolor{fgcolor}{rgb}{0.345, 0.345, 0.345}
\newcommand{\hlnum}[1]{\textcolor[rgb]{0.686,0.059,0.569}{#1}}%
\newcommand{\hlstr}[1]{\textcolor[rgb]{0.192,0.494,0.8}{#1}}%
\newcommand{\hlcom}[1]{\textcolor[rgb]{0.678,0.584,0.686}{\textit{#1}}}%
\newcommand{\hlopt}[1]{\textcolor[rgb]{0,0,0}{#1}}%
\newcommand{\hlstd}[1]{\textcolor[rgb]{0.345,0.345,0.345}{#1}}%
\newcommand{\hlkwa}[1]{\textcolor[rgb]{0.161,0.373,0.58}{\textbf{#1}}}%
\newcommand{\hlkwb}[1]{\textcolor[rgb]{0.69,0.353,0.396}{#1}}%
\newcommand{\hlkwc}[1]{\textcolor[rgb]{0.333,0.667,0.333}{#1}}%
\newcommand{\hlkwd}[1]{\textcolor[rgb]{0.737,0.353,0.396}{\textbf{#1}}}%
\let\hlipl\hlkwb

\usepackage{framed}
\makeatletter
\newenvironment{kframe}{%
 \def\at@end@of@kframe{}%
 \ifinner\ifhmode%
  \def\at@end@of@kframe{\end{minipage}}%
  \begin{minipage}{\columnwidth}%
 \fi\fi%
 \def\FrameCommand##1{\hskip\@totalleftmargin \hskip-\fboxsep
 \colorbox{shadecolor}{##1}\hskip-\fboxsep
     % There is no \\@totalrightmargin, so:
     \hskip-\linewidth \hskip-\@totalleftmargin \hskip\columnwidth}%
 \MakeFramed {\advance\hsize-\width
   \@totalleftmargin\z@ \linewidth\hsize
   \@setminipage}}%
 {\par\unskip\endMakeFramed%
 \at@end@of@kframe}
\makeatother

\definecolor{shadecolor}{rgb}{.97, .97, .97}
\definecolor{messagecolor}{rgb}{0, 0, 0}
\definecolor{warningcolor}{rgb}{1, 0, 1}
\definecolor{errorcolor}{rgb}{1, 0, 0}
\newenvironment{knitrout}{}{} % an empty environment to be redefined in TeX

\usepackage{alltt}
\usepackage[svgnames]{xcolor}
\usepackage[british]{babel}
\usepackage[protrusion,expansion,babel,final]{microtype}
\usepackage[margin=1in]{geometry}
\usepackage[pdfversion=1.7]{hyperref}
\usepackage[shortlabels]{enumitem}
\usepackage{graphicx}
\usepackage{mathtools}
\usepackage{cleveref}
\usepackage{booktabs}
\usepackage{nicematrix}
\usepackage{derivative}
\usepackage{etoolbox}
\usepackage{siunitx}
\usepackage{lmodern}
\usepackage[T1]{fontenc}
\usepackage[scaled=.98]{XCharter}
\usepackage[scaled=1.04,varqu,varl]{inconsolata}% inconsolata typewriter
\usepackage{amssymb}
\makeatletter
\@namedef{T1/zi4/m/it}{<->ssub*lmr/m/it}
\makeatother

\usepackage{bm}
\usepackage{tikz}
\usepackage{float}

% Functions
\providecommand\given{} % just to make sure it exists
\DeclarePairedDelimiterXPP{\E}[1]{\operatorname{\mathbb{E}}}[]{}{%
    \renewcommand\given{\nonscript\:\delimsize\vert\nonscript\:\mathopen{}}%
    \ifblank{#1}{\:\cdot\:}%
    #1}%
\DeclarePairedDelimiterXPP{\V}[1]{\operatorname{\textsf{V}}}(){}{%
    \renewcommand\given{\nonscript\:\delimsize\vert\nonscript\:\mathopen{}}%
    \ifblank{#1}{\:\cdot\:}%
    #1}%
\DeclarePairedDelimiterXPP{\Var}[1]{\operatorname{\textsf{Var}}}(){}{%
    \renewcommand\given{\nonscript\:\delimsize\vert\nonscript\:\mathopen{}}%
    \ifblank{#1}{\:\cdot\:}%
    #1}%
\DeclarePairedDelimiterXPP{\Cov}[1]{\operatorname{\textsf{Cov}}}(){}{%
    \renewcommand\given{\nonscript\:\delimsize\vert\nonscript\:\mathopen{}}%
    \ifblank{#1}{\:\cdot\:}%
    #1}%
\DeclarePairedDelimiterXPP{\Corr}[1]{\operatorname{\textsf{Corr}}}(){}{%
    \renewcommand\given{\nonscript\:\delimsize\vert\nonscript\:\mathopen{}}%
    \ifblank{#1}{\:\cdot\:}%
    #1}%
\DeclarePairedDelimiterXPP{\Covadj}[1]{\operatorname{\textsf{Cov}_{\text{adj}}}}(){}{%
    \renewcommand\given{\nonscript\:\delimsize\vert\nonscript\:\mathopen{}}%
    \ifblank{#1}{\:\cdot\:}%
    #1}%
\DeclarePairedDelimiterXPP\Prob[1]{\operatorname{\mathbb{P}}}(){}{%
    \renewcommand\given{\nonscript\:\delimsize\vert\nonscript\:\mathopen{}}%
    \ifblank{#1}{\:\cdot\:}%
    #1}%
\DeclarePairedDelimiterXPP\ProbM[1]{\operatorname{\mathcal{P}}}(){}{%
    \renewcommand\given{\nonscript\:\delimsize\vert\nonscript\:\mathopen{}}%
    \ifblank{#1}{\:\cdot\:}%
    #1}%
\DeclarePairedDelimiterXPP\Ind[1]{\operatorname{\mathbb{I}}}\{\}{}{%
    \renewcommand\given{\nonscript\:\delimsize\vert\nonscript\:\mathopen{}}%
    \ifblank{#1}{\:\cdot\:}%
    #1}%
\DeclarePairedDelimiterXPP{\se}[1]{\operatorname{\textsf{se}}}(){}{%
    \ifblank{#1}{\:\cdot\:}%
    #1}%
\DeclarePairedDelimiterXPP{\seadj}[1]{\operatorname{\textsf{se}_{\text{adj}}}}(){}{%
    \renewcommand\given{\nonscript\:\delimsize\vert\nonscript\:\mathopen{}}%
    \ifblank{#1}{\:\cdot\:}%
    #1}%
\DeclarePairedDelimiterXPP{\estseadj}[1]{\operatorname{\widehat{\textsf{se}}_{\text{adj}}}}(){}{%
    \renewcommand\given{\nonscript\:\delimsize\vert\nonscript\:\mathopen{}}%
    \ifblank{#1}{\:\cdot\:}%
    #1}%
\DeclarePairedDelimiterXPP{\estse}[1]{\widehat{\operatorname{\textsf{se}}}}(){}{%
    \ifblank{#1}{\:\cdot\:}%
    #1}%
\DeclarePairedDelimiterXPP{\estV}[1]{\widehat{\operatorname{\textsf{V}}}}(){}{
    \renewcommand\given{\nonscript\:\delimsize\vert\nonscript\:\mathopen{}}%
    \ifblank{#1}{\:\cdot\:}%
    #1}%
\DeclarePairedDelimiterXPP{\estVar}[1]{\widehat{\operatorname{\textsf{Var}}}}(){}{
    \renewcommand\given{\nonscript\:\delimsize\vert\nonscript\:\mathopen{}}%
    \ifblank{#1}{\:\cdot\:}%
    #1}%
\let\exp\relax%
\let\log\relax%
\let\ln\relax%
\DeclarePairedDelimiterXPP{\exp}[1]{\operatorname{\textsf{exp}}}\{\}{}{#1}%
\DeclarePairedDelimiterXPP{\log}[1]{\operatorname{\textsf{log}}}(){}{#1}%
\DeclarePairedDelimiterXPP{\ln}[1]{\operatorname{\textsf{ln}}}(){}{#1}%
\DeclarePairedDelimiterXPP{\diag}[1]{\operatorname{\textsf{diag}}}(){}{#1}%
\DeclarePairedDelimiterXPP{\sign}[1]{\operatorname{\textsf{sign}}}(){}{#1}%

\DeclarePairedDelimiterXPP{\expit}[1]{\operatorname{\textsf{expit}}}(){}{#1}%
\DeclarePairedDelimiterXPP{\logit}[1]{\operatorname{\textsf{logit}}}(){}{#1}%
\newcommand{\HN}{\textsl{H}_{\textsl{0}}}%
\newcommand{\HA}{\textsl{H}_{\textsl{A}}}%

% Distributions
\DeclarePairedDelimiterXPP{\N}[1]{\mathcal{N}}(){}{#1}%
\DeclarePairedDelimiterXPP{\POI}[1]{\text{POI}}(){}{#1}%
\DeclarePairedDelimiterXPP{\BIN}[1]{\text{BIN}}(){}{#1}%
\DeclarePairedDelimiterXPP{\BERN}[1]{\text{BERN}}(){}{#1}%
\DeclarePairedDelimiterXPP{\MVN}[1]{\text{MVN}}(){}{#1}%
\DeclarePairedDelimiterXPP{\NB}[1]{\text{NB}}(){}{#1}%
\DeclarePairedDelimiterXPP{\GAM}[1]{\text{GAM}}(){}{#1}%
\DeclarePairedDelimiterXPP{\BetaDist}[1]{\text{Beta}}(){}{#1}%

\newcommand{\iid}{\overset{\text{iid}}{\sim}}%
\newcommand{\ind}{\overset{\text{ind}}{\sim}}%
\newcommand{\OR}{\text{OR}}%
\newcommand{\RR}{\text{RR}}%
\newcommand{\cOR}{\text{cOR}}%

\DeclarePairedDelimiter\abs{\lvert}{\rvert}
% can be useful to refer to this outside \Set
\newcommand\SetSymbol[1][]{%
    \nonscript\:#1\vert{}
    \allowbreak\nonscript\:
    \mathopen{}}
\DeclarePairedDelimiterX\Set[1]\{\}{%
    \renewcommand\given{:}
    #1
}
\DeclareMathOperator*{\argmax}{arg\,max}
\DeclareMathOperator*{\argmin}{arg\,min}
\DeclareMathOperator*{\arginf}{arg\,inf}
\DeclareMathOperator*{\argsup}{arg\,sup}

\providecommand{\RandomVector}[1]{\bm{#1}}% general vectors in bold italic
\providecommand{\Vector}[1]{\bm{#1}}% general vectors in bold italic
\providecommand{\Matrix}[1]{\bm{#1}}
\providecommand{\MatrixCal}[1]{\bm{\mathcal{#1}}}
\providecommand{\Field}[1]{\bm{#1}}

\usepackage{stackengine}
\usepackage[british]{isodate}
\newcommand{\makeheading}[2]%
{%
\begin{center}%
    \makebox[\linewidth]{\raisebox{-.5ex}[0cm][0cm]{\stackanchor{\textcolor{Gray}{\textsc{#1}}}{\scriptsize\itshape\printyearoff#2}\;}\color{Crimson!50}\hrulefill}%
\end{center}%
}%

\usepackage[breakable]{tcolorbox}
\tcbset{
    regular/.style={
        boxrule=0pt,
        breakable,
        sharp corners
    }
}

\newtcolorbox{Example}[1]{regular,colframe=Green!20!white,colback=Green!10!white,coltitle=Green,title={#1}}%
\newtcolorbox{Regular}[1]{regular,colframe=Navy!15!white,colback=Navy!5!white,coltitle=Navy,title={#1}}%
\newtcolorbox{Result}[1]{regular,colframe=Red!15!white,colback=Red!5!white,coltitle=Red,title={#1}}%

\hypersetup{colorlinks=true,%
linkcolor=[rgb]{0,0.5,1},%
pdftitle={Sampling Theory and Practice (STAT 454/854)},%
pdfauthor={Cameron Roopnarine, Changbao Wu},%
pdfsubject={Statistics},%
pdfkeywords={University of Waterloo, Winter 2022 (1221)}}%

\title{%
\LARGE Sampling Theory and Practice\\%
\large STAT 454\thanks{STAT 454$\equiv$ STAT 854}\\%
\normalsize Winter 2022 (1221)\thanks{Online Course until January 27\textsuperscript{th}, 2022}}%
\author{Cameron Roopnarine\thanks{\LaTeX{}er}\and Changbao Wu\thanks{Instructor}}%
\date{\today}%
\usepackage{pgfplots}
\pgfplotsset{compat=1.18}
\usetikzlibrary{petri,decorations.pathreplacing,calc}
\IfFileExists{upquote.sty}{\usepackage{upquote}}{}
\begin{document}


\maketitle
\tableofcontents

\chapter{Review of Basic Concepts in Survey Sampling}
\makeheading{Week 1}{\daterange{2022-01-05}{2022-01-07}}%chktex 8
\begin{Example}{}
      \textbf{Example 1.1}. The Mathematics Faculty plans to conduct a survey to study
      the well-being of recent graduates from the faculty.
      \tcblower{}
      \begin{itemize}
            \item \emph{Target population}: Who is the group to be studied?
            \item \emph{Sample data} (variables to be measured): What information should we collect?
            \item \emph{Sampling frame(s)}: From what can we select individuals to be surveyed?
            \item \emph{Sampling methods/procedures}: How do we select individuals to be surveyed?
            \item \emph{Method of data collection}: What method(s) can we use to collect data?
                  \begin{itemize}
                        \item Examples: face-to-face, telephone, mail, questionnaire.
                  \end{itemize}
            \item \emph{Statistical analysis}: How do we use the data to draw conclusions?
      \end{itemize}
\end{Example}
\section*{Survey Populations}
\begin{Regular}{}
      \begin{itemize}
            \item \textbf{Target population}: The set of all units covered by the main
                  objective of the study.
                  \begin{Example}{Target Population of Example 1.1}
                        All students who received a formal degree from Waterloo
                        between 2016 and 2019.
                  \end{Example}
            \item \textbf{Frame population}: The set of all units covered by the
                  sampling frame(s).
                  \begin{Example}{Sampling Frame of Example 1.1}
                        The list of personal email addresses of
                        students who graduated between 2016 and 2019.
                  \end{Example}
            \item \textbf{Sampled/study population}:
                  The population represented by the sample. Under probability sampling, the
                  sampled population is the set of all units which have a non-zero
                  probability to be selected in the sample.
                  \begin{itemize}
                        \item The sampled population is not the set of sampled units!
                        \item Units which cannot be reached or do not respond to surveys
                              (non-response) are not part of the sampled population.
                  \end{itemize}
      \end{itemize}
\end{Regular}
\section*{Population Structures and Sampling Frames}
\begin{Regular}{}
      A general \textbf{population} is $ U=\Set{1,2,\ldots,N} $,
      where $ N $ is the population size, and the labels $ 1,2,\ldots,N $ represent
      the $ N $ units.
      \tcblower{}
      \begin{itemize}
            \item \textbf{Unstructured population}:
                  There exists a single complete list of all $N$ units, which can be
                  used as the sampling frame.
            \item \textbf{Stratified population}:
                  The population $U$ has a stratified structure if it is divided into $H$
                  non-overlapping subpopulations:
                  \[ U=U_1\cup U_2\cup\cdots\cup U_H, \]
                  where the subpopulation $ U_h $ is called stratum $ h $, with stratum
                  population size $ N_h $ for $ h=1,2,\ldots,H $. It follows that
                  \[ N=\sum_{h=1}^{H}N_h. \]
                  Sampling frames for stratified sampling:
                  $H$ separate lists, each list consists of all units in one stratum.
            \item \textbf{Clustered population}:
                  If the survey population can be divided into groups, called
                  \emph{clusters}, such that every unit in the population belongs to one
                  and only one group, we say the population is clustered.

                  First stage sampling frame for cluster sampling: A complete list of clusters (but not all the units within each
                  cluster).
            \item Stratified sampling versus cluster sampling:
                  \begin{itemize}
                        \item Under stratified sampling, sample data are collected from every
                              stratum.
                        \item Under cluster sampling, only a portion of the clusters has
                              members in the final sample.
                  \end{itemize}
      \end{itemize}
\end{Regular}
\begin{Example}{}
      \textbf{Example 1.2}. Survey of the population of high school students in the
      Waterloo region. There are a total of $15$ high schools. Take a sample
      of $300$ students from the population.
      \tcblower{}
      \begin{itemize}
            \item \textbf{Stratified sampling}: Randomly select $20$ students from each high school.
            \item \textbf{Two-stage cluster sampling}: Randomly select $5$ high schools from the list of $15$ schools,
                  and then randomly select $60$ students from each of the $5$ selected
                  schools.
            \item \textbf{Stratified two-stage cluster sampling}: The Waterloo region can be divided into KW area ($8$ high
                  schools) and non-KW area ($7$ high schools). First, randomly select $3$
                  schools from the KW area and 2 schools from the non-KW area, then
                  randomly select $60$ students from each of the $5$ selected schools.
      \end{itemize}
\end{Example}
\section*{Sampling Units and Observational Units}
\begin{Regular}{}
      \begin{itemize}
            \item \textbf{Sampling units}: Units used to select the survey sample.
                  \begin{itemize}
                        \item Under clustering sampling, sampling units are the clusters.
                        \item Under non-clustering sampling, sampling units are the individual
                              units.
                  \end{itemize}
            \item \textbf{PSU and SSU}: Under two-stage cluster sampling, the first stage
                  sampling units are clusters, called the \emph{primary sampling unit}
                  (PSU); the second stage sampling units are individual units,
                  called the \emph{secondary sampling unit} (SSU).
            \item \textbf{Observational units}: Observational units are always the individual units from which
                  measurements are taken.
      \end{itemize}
\end{Regular}
\begin{Example}{}
      \textbf{Example 1.3}. An educational worker wanted to find out the average
      number of hours each week (of a certain month and year) spent on
      watching television by four and five-year-old children in the Waterloo
      Region. She conducted a survey using the list of $123$ pre-school
      kindergartens administered by the Waterloo Region District School
      Board. She first randomly selected $10$ kindergartens from the list.
      Within each selected kindergarten, she was able to obtain a complete
      list of all four and five-year-old children, with contact information for
      their parents/guardians. She then randomly selected $50$ children from
      the list and mailed the survey questionnaire to their parents/guardians.
      The planned sample size is $ 10\times 50=500 $ and the sample data were
      compiled from those who completed and returned the questionnaires.
      \tcblower{}
      \begin{itemize}
            \item \emph{Target population}:
                  All four and five-year-old children in the Region of Waterloo at
                  the time of the survey. This is defined by the overall objective of
                  the study.
            \item \emph{Sampling frames}: Two-stage cluster sampling methods were used (further details to
                  follow). The first stage sampling frame is the list of $123$
                  kindergartens administered by the school board. The second
                  stage sampling frames are the complete lists of all four and five-year-old
                  children for the $10$ selected kindergartens.
            \item \emph{Sampling units and observational units}: The first stage sampling units are the kindergartens; the second
                  stage sampling units are the individual children (or equivalently,
                  their parents); observational units are individual children.
            \item \emph{Frame population}: All four and five-year-old children who attend one of the 123
                  kindergartens in the Region of Waterloo. It is apparent that
                  children who are homeschooled are not covered by the frame
                  population. Thus, as is frequently the case, the frame population
                  is not the same as the target population.
            \item \emph{Sampled population}: All four and
                  five-year-old children who attend one of the $123$
                  kindergartens in the Region of Waterloo and whose
                  parents/guardians would complete and return the survey
                  questionnaire if the child was selected for the survey.
      \end{itemize}
\end{Example}

\makeheading{Week 2}{\daterange{2022-01-10}{2022-01-14}}%chktex 8
\section*{Survey Samples}
\begin{Regular}{}
      A \textbf{survey sample} $S$, is a subset of the population
      $ U=\Set{1,2,\ldots,N} $.
      \tcblower{}
      The sample size $ n=\abs{S} $ is the number
      of units in the sample (a set of $ n $ ``unordered'' units):
      \[ S=\Set{i_1,i_2,\ldots,i_n}. \]
      We could simply use $ S=\Set{1,2,\ldots,n} $.
\end{Regular}
\begin{Example}{Survey Sample}
      If $ N=10 $ and $n=3 $, then $U=\Set{1,2,3,4,5,6,7,8,9,10}$,
      and some possible samples are:
      \begin{itemize}
            \item $S=\Set{7,4,9}=\Set{i_1,i_2,i_3}$, or
            \item $S=\Set{1,2,3}$.
      \end{itemize}
\end{Example}
\section*{Non-probability Samples versus Probability Samples}
\begin{Regular}{}
      \textbf{Non-probability samples} are selected by subjective or any
      convenient methods.
      \tcblower{}
      We will list some examples of non-probability sampling.
      \begin{itemize}
            \item \textbf{Quota sampling}: The sample is obtained by a number of
                  interviewers, each of whom is required to sample certain
                  numbers of units with certain types or characteristics. How to
                  select the units is completely left in the hands of the interviewers.
            \item \textbf{Judgement or purposive sampling}: The sample is selected based
                  on what the sampler believes to be ``typical'' or ``most
                  representative'' of the population.
            \item \textbf{Restricted sampling}: The sample is restricted to certain parts of
                  the population which are readily accessible.
            \item \textbf{Sample of convenience}: The sample is taken from those who are
                  easy to reach.
            \item \textbf{Sample of volunteers}: The sample consists of those who
                  volunteer to participate.
            \item \textbf{Web panels}: The sample is selected from a panel of people who
                  signed up to do surveys in order to receive cash or other
                  incentives.
      \end{itemize}
      \underline{Remarks}:
      \begin{itemize}
            \item The most serious issue with non-probability survey samples is that
                  the sample is \emph{biased}. A sample is \textbf{biased}
                  if it has unknown inclusion probabilities.
            \item Non-probability survey samples are not the focus of this course. But
                  the topic is becoming important in recent years, since data from
                  non-probability survey samples become useful sources.
            \item Yilin Chen's PhD thesis research is on statistical analysis with
                  non-probability survey samples, to be introduced in the last lecture.
      \end{itemize}
\end{Regular}
\begin{Regular}{}
      \textbf{Probability samples}, theoretically speaking, are selected through a
      probability measure over a pool of candidate samples.
      \begin{itemize}
            \item Let $ \Omega=\Set{S\given S\subseteq U} $
                  be the set of all possible subsets of the survey population $U$.
            \item Let $ \mathcal{P} $ be a probability measure over $ \Omega $ such that
                  \begin{enumerate}[(i)]
                        \item $ \mathcal{P}(S)\ge 0 $ for any $ S\in \Omega $, and
                        \item $ \sum_{S:S\in\Omega}\mathcal{P}(S)=1 $
                  \end{enumerate}
      \end{itemize}
      A probability sample $ S $ is selected based on the \textbf{probability sampling design}, $ \mathcal{P} $.
\end{Regular}
\begin{Example}{}
      \textbf{Example 1.4}. $ N=3 $; $ U=\Set{1,2,3} $, $ n=1 $ or $ 2 $.

      Candidate samples:
      \begin{itemize}
            \item $ n=1 $: $ S_1=\Set{1} $, $ S_2=\Set{2} $, $ S_3=\Set{3} $.
            \item $ n=2 $: $ S_4=\Set{1,2} $, $ S_5=\Set{1,3} $, $ S_6=\Set{2,3} $, and
                  $ S_7=\Set{1,2,3} $ (census!).
      \end{itemize}
      \[ \begin{NiceArray}{c|ccccccc}
                  S              & S_1 & S_2 & S_3 & S_4 & S_5 & S_6 & S_7 \\
                  \midrule
                  \mathcal{P}(S) & 1/6 & 1/6 & 1/6 & 1/6 & 1/6 & 1/6 & 0
            \end{NiceArray} \]
      \[ \begin{NiceArray}{c|ccccccc}
                  S              & S_1 & S_2 & S_3 & S_4 & S_5 & S_6 & S_7 \\
                  \midrule
                  \mathcal{P}(S) & 0   & 0   & 0   & 1/3 & 1/3 & 1/3 & 0
            \end{NiceArray} \]
      \[ \begin{NiceArray}{c|ccccccc}
                  S              & S_1 & S_2 & S_3 & S_4 & S_5 & S_6 & S_7 \\
                  \midrule
                  \mathcal{P}(S) & 0   & 0   & 0   & 1/2 & 1/4 & 1/4 & 0
            \end{NiceArray} \]
      \[ \mathcal{P}(S)\ge 0 \]
      \[ \sum_{S}\mathcal{P}(S)=1. \]
\end{Example}
\textbf{Sampling design $ \mathcal{P} $ with fixed sample size}: $ \ProbM{S}=0 $
if $ \abs{S}\ne n $. The probability measure is defined over
\[ \Omega_n=\Set{S\given S\subseteq U\text{ and }\abs{S}=n}. \]
\begin{Example}{}
      \textbf{The cumulative sum method for generating a discrete random
            variable}:
      \[ X \sim f(x):\quad p_i=f(x_i)=\Prob{X=x_i},i=1,2,\ldots. \]
      \begin{itemize}
            \item Step 1. Probability cumulation.
                  \begin{align*}
                        b_0 & =0                 \\
                        b_1 & =p_1               \\
                        b_2 & =p_1+p_2           \\
                        b_3 & =p_1+p_2+p_3       \\
                            & \vdotswithin{=}    \\
                        b_j & =\sum_{i=1}^{j}p_i \\
                            & \vdotswithin{=}
                  \end{align*}
            \item Step 2. Generate $ r\sim U(0,1) $.
            \item Step 3. Let $ X=x_j $ if $ b_{j-1}<r\le b_j $.
      \end{itemize}
      Can show $ X \sim f(x) $.
\end{Example}
\textbf{Survey variables and population parameters}:
\begin{itemize}
      \item $ y $: the response variable; $ \Vector{x} $
            the vector of auxiliary variables.
      \item $ (y_i;\Vector{x}_i) $: the values of $ (y,\Vector{x}) $
            associated with unit $ i $, $ i=1,2,\ldots,N $.
      \item A common assumption in survey sampling: the values $ (y_i,\Vector{x}_i) $
            can be measured without error if $ i $ is selected in the sample.
      \item Population totals:
            \[ T_y=\sum_{i=1}^{N}y_i\text{ and }T_{\Vector{x}}=\sum_{i=1}^{N}\Vector{x}_i. \]
      \item Population means:
            \[ \mu_y=\frac{1}{N}\sum_{i=1}^{N}y_i\text{ and }\mu_{\Vector{x}}=\frac{1}{N}\sum_{i=1}^{N}\Vector{x}_i. \]
      \item Population variance of $ y $:
            \[ \sigma_y^2=\frac{1}{N-1}\sum_{i=1}^{N}(y_i-\mu_y)^2. \]
\end{itemize}
\textbf{An important special case: $y$ is a binary variable}:
\[ y_i=\begin{cases}
            1, & \text{if unit $ i $ has attribute ``$A$''}, \\
            0, & \text{otherwise}.
      \end{cases} \]
$ N $: the total number of units in the population (population size).
$ M $: the total number of units in the population having attribute ``$A$.''
\begin{itemize}
      \item Population total:
            \[ T_y=\sum_{i=1}^{N}y_i=M. \]
      \item Population mean:
            \[ \mu_y=\frac{T_y}{N}=\frac{M}{N}=p. \]
      \item Population variance:
            \[ \sigma_y^2=\frac{1}{N-1}\biggl(\sum_{i=1}^{N}y_i^2-N\mu_y^2\biggr). \]
\end{itemize}
\textbf{Probability sampling and design-based inference}:
\begin{itemize}
      \item The survey population $ U=\Set{1,2,\ldots,N} $ is viewed as fixed.
      \item The values $ y_i $ and $ \Vector{x}_i $ attached to unit $ i $
            and the population parameters such as $ T_y $ and $ \mu_y $
            are also viewed as fixed.
      \item The values of the population parameters can be determined without error by conducting
            a census.
      \item The sample $ S $ is selected according to a probability sampling design $ \mathcal{P} $.
      \item The sample $ S $ is a random set under $ \mathcal{P} $.
      \item Each unit in the population has a probability to be included in the
            sample.
      \item Randomization is induced by the probability sampling design for
            the selection of the survey sample.
\end{itemize}
\textbf{Basic sampling techniques and advanced topics}:
\begin{itemize}
      \item Basic sampling techniques and theory are developed for the
            estimation of the population total $T_y$ and the population mean $ \mu_y $.\\
            (Chapters 1--5 in the textbook)
      \item The basic methods and theory can be extended to handle more
            advanced topics, such as design-based regression analysis using
            survey data.\\
            (Chapters 6--11 in the textbook)
\end{itemize}

\chapter{Review of Simple Random Sampling}
\section{Simple Random Sampling Without Replacement (SRSWOR)}
The task: Select a sample of size $n$ from a population of size $N$ with
equal probability among all candidate samples.

The total number of candidate samples:
\[ \binom{N}{n}=\frac{N(N-1)\cdots(N-n+1)}{n!}. \]
The probability measure for the sampling design:
\begin{align*}
      \mathcal{P}(S)
       & =\begin{cases}
                \frac{1}{\binom{N}{n}}, & \text{if $\abs{S}=n$}     \\
                0,                      & \text{if $\abs{S}\ne n$}.
          \end{cases}
\end{align*}
$ \mathcal{P}(S) $ cannot be used to select a sample in practice.
$ N=1000 $, $ n=3 $:
\[ \binom{N}{n}=\frac{1000\times 999\times 998}{6}. \]
$ \mathcal{P}(S) $ is a theoretical tool.
\textbf{Sampling scheme or sampling procedure}:
Select the survey sample through a sequential draw-by-draw method;
select units from the sampling frame, one-at-a-time, until the final
sample is chosen.

\textbf{SRSWOR} is a sampling procedure to select a sample of size n with
equal probability among all candidate samples.

\textbf{The sampling frame for SRSWOR}:
A complete list of N units in the population.

\textbf{The SRSWOR sampling procedure}:
\begin{enumerate}[(1)]
      \item Select the first unit from the $N$ units on the sampling frame with
            equal probabilities $1/N$; denote the selected unit as $i_1$;
      \item Select the second unit from the remaining $ N-1 $ units on the
            sampling frame with equal probabilities $ 1/(N-1) $; denote the
            selected unit as $i_2$;
      \item Continue the process and select the $n\textsuperscript{th}$ unit from the remaining
            $ N-n+1 $ units on the sampling frame with equal probabilities
            $ 1/(N-n+1) $; denote the selected unit as $ i_n $.
\end{enumerate}
\begin{Result}{}
      \textbf{Theorem 2.1}. Under simple random sampling without replacement,
      the selected sample satisfies the probability measure $ \mathcal{P} $ specified as
      \[ \ProbM{S}=\begin{cases}
                  1/\binom{N}{n}, & \text{if $ \abs{S}=n $}, \\
                  0,              & \text{otherwise}.
            \end{cases} \]
      \tcblower{}
      \textbf{Proof}: Let $ S=\Set{i_1,i_2,\ldots,i_n} $ be the final sample.
      \[ \mathcal{P}(S)=\frac{n(n-1)\cdots(2)(1)}{N(N-1)\cdots(N-n+1)}=\frac{1}{\binom{N}{n}}. \]
\end{Result}
\begin{itemize}
      \item Survey sample selection always focuses on units, that is, the labels.
      \item Survey sample data: $ \Set{(y_i,x_i),i\in S} $.
\end{itemize}
\textbf{Sample mean and sample variance}:
\[ \bar{y}=\frac{1}{n}\sum_{i\in S}y_i. \]
\[ s_y^2=\frac{1}{n-1}\sum_{i\in S}(y_i-\bar{y})^2=\frac{1}{n-1}\biggl(\sum_{i\in S}y_i^2-n(\bar{y})^2\biggr). \]
\textbf{Remarks}:
\begin{itemize}
      \item The sample mean $ \bar{y} $ and $ s_y^2 $ are useful
            statistics under simple random sampling, but not necessarily under
            other sampling methods.
      \item The notation $ \sum_{i\in S} $ is preferred over
            $ \sum_{i=1}^{n} $.
      \item The form of estimators for population parameters depends on the
            sampling methods.
      \item The combination of ``sampling design'' and ``estimation method''
            is called a ``sampling strategy'' (Thompson, 1997; Rao, 2005).
\end{itemize}
\textbf{Expectation and variance under design-based inferences}:
In classic statistics: $ X_1,X_2,\ldots,X_n $ are iid with $ \E{X_i}=\mu $,
$ \V{X_i}=\sigma^2 $.

Sample mean: $ \bar{X}=\frac{1}{n}\sum_{i=1}^{n}X_i $.
\[ \E{\bar{X}}=\frac{1}{n}\sum_{i=1}^{n}\E{X_i}=\frac{1}{n}\sum_{i=1}^{n}\mu=\mu. \]
\[ \V{\bar{X}}=\frac{1}{n^2}\sum_{i=1}^{n}\V{X_i}=\frac{1}{n^2}\sum_{i=1}^{n}\sigma^2=\frac{\sigma^2}{n}. \]
Under SRSWOR:
\[ \E{\bar{y}}=\E*{\frac{1}{n}\sum_{i\in S}y_i}\ne \frac{1}{n}\sum_{i\in S}\E{y_i}. \]
$ S $: a random set.
$ \sum_{i\in S} $: a random ``sum.''
$ y_i $: a fixed quantity for the given $ i $.
\textbf{Three fundamental results in survey sampling under SRSWOR}:
\begin{enumerate}[(a)]
      \item The sample mean $ \bar{y}=n^{-1}\sum_{i\in S}y_i $
            is a design-unbiased estimator for the population
            mean $ \mu_y=N^{-1}\sum_{i=1}^{N}y_i $: $ \boxed{\E{\bar{y}}=\mu_y} $.

            There are three possible ways to prove (a), depending on how
            the randomization under SRSWOR is handled.

            Method 1. Use the probability measure $ \mathcal{P}(S) $
            for the survey design.
            \[ \mathcal{P}(S)=\frac{1}{\binom{N}{n}} \]
            for $ \abs{S}=n $. $ \bar{y} $ depends only on $ S $.
            \[ \bar{y}=\frac{1}{n}\sum_{i\in S}y_i=\bar{y}(S), \]
            that is, $ \bar{y} $ is a function of $ S $.
            \begin{align*}
                  \E{\bar{y}}
                   & =\sum(\text{value})(\text{prob})                                      \\
                   & =\sum_S \bar{y}(S)\mathcal{P}(S)                                      \\
                   & =\sum_{S:\abs{S}=n}\frac{1}{n}\sum_{i\in S}y_i \frac{1}{\binom{N}{n}} \\
                   & =\frac{1}{n}\frac{1}{\binom{N}{n}}\sum_{S:\abs{S}=n} \sum_{i\in S}y_i \\
                   & =\frac{1}{n}\frac{1}{\binom{N}{n}}\sum_{i=1}^{N}t_i y_i               \\
                   & =\frac{1}{N}\sum_{i=1}^{N}y_i                                         \\
                   & =\mu_y,
            \end{align*}
            where $ t_i = $\#S which includes the unit $ i $:
            \[ t_i=\binom{N-1}{n-1}. \]
            \begin{Example}{}
                  $ N=3 $, $ n=2 $: $ S_1=\Set{1,2} $, $ S_2=\Set{1,3} $, $ S_3=\Set{2,3} $.
                  \begin{align*}
                        \sum_{S:\abs{S}=n} \sum_{i\in S}y_i
                         & =(y_1+y_2)+(y_1+y_3)+(y_2+y_3) \\
                         & =2 y_1+2y_2+2y_3.
                  \end{align*}
            \end{Example}
            Method 2. Use the sampling scheme, the sequential
            draw-by-draw procedure. Let $ Z_k $ be the $ y $-value from the
            $ k\textsuperscript{th} $ draw:
            \[ S=\Set{i_1,i_2,\ldots,i_n}. \]
            \[ Z_k=y_{ik},\;k=1,2,\ldots,n. \]
            \[ \bar{y}=\frac{1}{n}\sum_{i\in S}y_i=\frac{1}{n}\sum_{k=1}^{n}Z_k. \]
            \[ \E{\bar{y}}=\E*{\frac{1}{n}\sum_{k=1}^n Z_k}
                  =\frac{1}{n}\sum_{k=1}^{n}\E{Z_k}. \]
            What's the probability function of $ Z_k $?
            \[ \begin{NiceArray}{c|cccc}
                        Z_k          & y_1 & y_2 & \cdots & y_N \\
                        \midrule
                        f(\:\cdot\:) & 1/N & 1/N & \cdots & 1/N
                  \end{NiceArray} \]
            \[ \E{Z_k}=\sum_{i=1}^{N}y_i \frac{1}{N}=\mu_y. \]
            Method 3. Use the sample inclusion indicator variables.
            \[ A_i=\begin{cases}
                        1, & \text{if $ i\in S $},    \\
                        0, & \text{if $ i\notin S $}.
                  \end{cases}\qquad i=1,2,\ldots,N. \]
            The $ A_i $'s are random variables.
            \begin{align*}
                  \Prob{A_i=1} & =p=\Prob{i\in S}=\frac{1\times \binom{N-n}{n-1}}{\binom{N}{n}}=\frac{n}{N}.
                  \Prob{A_i=0} & =1-p.                                                                       \\
                  \E{A_i}      & =p=\frac{n}{N}.                                                             \\
                  \V{A_i}      & =p(1-p)=\frac{n}{N}\biggl(1-\frac{n}{N}\biggr).                             \\
                  \E{\bar{y}}  & =\E*{\frac{1}{n}\sum_{i\in S}y_i}                                           \\
                               & =\E*{\frac{1}{n}\sum_{i=1}^{N}A_i y_i}                                      \\
                               & =\frac{1}{n}\sum_{i=1}^{n}y_i\E{A_i}                                        \\
                               & =\frac{1}{N}\sum_{i=1}^{N}y_i                                               \\
                               & =\mu_y.
            \end{align*}

      \item The design-based variance of $ \bar{y} $
            under SRSWOR is given by
            \[ \V{\bar{y}}=\biggl(1-\frac{n}{N}\biggr)\frac{\sigma_y^2}{n}, \]
            where $ \sigma_y^2 $ is the population variance. The term
            $ (1-n/N) $ is called the \emph{finite population correction} (fpc)
            factor; The ratio $ n/N $ is called the \emph{sampling fraction}.

            This result can be proved using different methods. Use the indicator variables:
            \begin{align}
                  \V{\bar{y}}
                   & =\V*{\frac{1}{n}\sum_{i=1}^{N}A_i y_i}                            \\
                   & =\frac{1}{n^2}\biggl[\sum_{i=1}^{N}y_i\V{A_i}+
                  \sum_{i=1,i\ne j}^{N}\sum_{j=1,i\ne j}^{N}y_iy_j\Cov{A_i,A_j}\biggr] \\
                   & =\V{A_i}=\frac{n}{N}\biggl(1-\frac{n}{N}\biggr).
            \end{align}
            \[ \Cov{A_i,A_j}=\E{A_iA_j}-\E{A_i}\E{A_j}. \]
            \begin{align*}
                  \E{A_iA_j}
                   & =\sum_i\sum_j a_ia_j \Prob{A_i=a_i}\Prob{A_j=a_j}      \\
                   & =\Prob{A_i=1,A_j=1}                                    \\
                   & =\Prob{i\in S,j\in S}                                  \\
                   & =\frac{1\times 1\times \binom{N-2}{n-2}}{\binom{N}{n}} \\
                   & =\frac{n(n-1)}{N(N-1)}.
            \end{align*}
            \[ \mu_y^2=\frac{1}{N^2}\biggl(\sum_{i=1}^{N}y_i\biggr)^2=\frac{1}{N^2}\sum_{i=1}^{N}\sum_{j=1}^{N}y_iy_j
                  =\frac{1}{N^2}\biggl[\sum_{i=1}^{N}y_i^2+\sum_{i}^{N}\sum_{j}^{N}y_iy_j\biggr]. \]
      \item The sample variance $ s_y^2 $ is an unbiased estimator for the
            population variance $ \sigma_y^2 $ under SRSWOR, i.e.,
            $ \boxed{\E{s_y^2}=\sigma_y^2} $.
      \item An unbiased variance estimator for $ \bar{y} $ is given by
            \[ v(\bar{y})=\biggl(1-\frac{n}{N}\biggr)\frac{s_y^2}{n}, \]
            which satisfies
            \[ \E{v(\bar{y})}=\V{\bar{y}}, \]
            where
            \[ \V{\bar{y}}=\biggl(1-\frac{n}{N}\biggr)\frac{\sigma_y^2}{n}. \]
            \begin{align*}
                  s_y^2
                   & =\frac{1}{n-1}\biggl[\sum_{i\in S}y_i^2-n(\bar{y})^2\biggr]                         \\
                   & =\frac{n}{n-1}\biggl(\frac{1}{n}\sum_{i\in S}y_i^2\biggr)-\frac{n}{n-1}(\bar{y})^2.
            \end{align*}
            $ \E{\bar{y}}=\mu_y $ implies
            \begin{align*}
                  \E*{\frac{1}{n}\sum_{i\in S}y_i^2}=\frac{1}{N}\sum_{i=1}^{N}y_i^2.
            \end{align*}
            \begin{align*}
                  \E{\bar{y}^2}
                   & =\V{\bar{y}}+\bigl(\E{\bar{y}}\bigr)^2                    \\
                   & =\biggl(1-\frac{n}{N}\biggr)\frac{\sigma_y^2}{n}+\mu_y^2.
            \end{align*}
            Homework: Show that $ \E{s_y^2}=\sigma_y^2 $.
\end{enumerate}
\textbf{Summary of the main theoretical results under SRSWOR}:
\begin{itemize}
      \item The population mean $ \mu_y $ and the population variance
            $ \sigma_y^2 $ are fixed (but unknown) population parameters.
      \item The sample mean $ \bar{y} $ and the sample variance $ s_y^2 $
            are random variables under the survey design.
      \item The $ \bar{y} $ is an unbiased estimator $ \mu_y $: $ \E{\bar{y}}=\mu_y $.
      \item $ \V{\bar{y}}=\biggl(1-\frac{n}{N}\biggr)\frac{\sigma_y^2}{n} $
            is the theoretical variance of $ \bar{y} $ and is a fixed, but unknown
            quantity depending on the population variance $ \sigma_y^2 $.
      \item $ v(\bar{y})=\biggl(1-\frac{n}{N}\biggr)\frac{s_y^2}{n} $
            is unbiased estimator for $ \bar{y} $ (computable with the given sample data).
      \item The population size $ N $ is known under SRSWOR\@. (As part of the sampling
            frame information).
\end{itemize}
The R function for SRSWOR and SRSWR (next section) with
specified $N$ and $n$: \texttt{sample(N,n)}
%\begin{noindent}
\begin{knitrout}
\definecolor{shadecolor}{rgb}{0.969, 0.969, 0.969}\color{fgcolor}\begin{kframe}
\begin{alltt}
\hlstd{N} \hlkwb{=} \hlnum{10}
\hlstd{n} \hlkwb{=} \hlnum{4}
\hlstd{sam} \hlkwb{=} \hlkwd{sample}\hlstd{(N, n)}
\hlstd{sam}
\end{alltt}
\begin{verbatim}
[1] 1 8 6 2
\end{verbatim}
\begin{alltt}
\hlstd{sam} \hlkwb{=} \hlkwd{sample}\hlstd{(N, n,} \hlkwc{replace} \hlstd{= T)}
\hlstd{sam}
\end{alltt}
\begin{verbatim}
[1] 3 2 2 2
\end{verbatim}
\begin{alltt}
\hlstd{N} \hlkwb{=} \hlnum{100}
\hlstd{n} \hlkwb{=} \hlnum{4}
\hlstd{sam} \hlkwb{=} \hlkwd{sample}\hlstd{(N, n)}
\hlstd{sam}
\end{alltt}
\begin{verbatim}
[1] 36 77 11 15
\end{verbatim}
\begin{alltt}
\hlstd{sam} \hlkwb{=} \hlkwd{sample}\hlstd{(N, n,} \hlkwc{replace} \hlstd{= T)}
\hlstd{sam}
\end{alltt}
\begin{verbatim}
[1] 17  5 32 66
\end{verbatim}
\end{kframe}
\end{knitrout}
%\end{noindent}
\end{document}
