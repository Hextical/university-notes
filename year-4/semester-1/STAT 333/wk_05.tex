\makeheading{Week 5}{\daterange{2021-10-06}{2021-10-20}}%chktex 8
\section{Transience and Recurrence}
\subsection*{First Visit Probability}
\begin{Regular}
    We now wish to take a closer look at the likelihood of a DTMC beginning in some state of
    returning to that particular state. To that end, let us consider the probability that, starting
    from state $i$, the first visit of the DTMC to state $j$ occurs at time $ n\in\mathbb{Z}^+ $, to be denoted by
    \[ f_{i,j}^{(n)}=\Prob{X_n=j,X_{n-1}\ne j,\ldots,X_2\ne j, X_1\ne j\given X_0=i}\;\forall i,j\in\mathbb{N}. \]
    Clearly, we see that $ f_{i,j}^{(1)}=P_{i,j} $.

    For $n \geq 2$, however, the determination of $f_{i, j}^{(n)}$ becomes more complicated,
    and so we wish to construct a procedure which will enable us to compute
    $f_{i, j}^{(n)}$ for such $n$. To do so, we consider the quantity $P_{i, j}^{(n)}$, $n \in \mathbb{Z}^{+}$,
    and condition on the time that the first visit to state $j$ is made:

    \begin{align*}
        P_{i, j}^{(n)}
         & =\Prob{X_{n}=j \given X_{0}=i}                                                                                                  \\
         & =\sum_{k=1}^{n} \Prob{X_{n}=j, \text{first visit to state $ j $ occurs at time $k$} \given X_{0}=i}                             \\
         & =\sum_{k=1}^{n} \Prob{X_{n}=j, X_{k}=j, X_{k-1} \neq j, \ldots, X_{2} \neq j, X_{1} \neq j \given X_{0}=i}                      \\
         & =\sum_{k=1}^{n} \Prob{X_{k}=j, X_{k-1} \neq j, \ldots, X_{2} \neq j, X_{1} \neq j \given X_{0}=i} \Prob{X_{n}=j \given X_{k}=j} \\
         & =\sum_{k=1}^{n} f_{i,j}^{(k)} P_{j,j}^{(n-k)}, \label{eq3.2}\tag*{(3.2)}
    \end{align*}
    where we applied the Markov property in the second last equality.

    \textbf{We have}: $ P_{i,j}^{(n)}=\sum_{k=1}^{n} f_{i,j}^{(k)}P_{j,j}^{(n-k)}\leftarrow $~\ref{eq3.2}.

    From~\ref{eq3.2}, we can write
    \[ P_{i,j}^{(n)}=f_{i,j}^{(n)}P_{j,j}^{(0)}+\sum_{k=1}^{n-1} f_{i,j}^{(k)}P_{j,j}^{(n-k)}=f_{i,j}^{(n)}+\sum_{k=1}^{n-1}f_{i,j}^{(k)}P_{j,j}^{(n-k)}, \]
    implying that
    \[ f_{i, j}^{(n)}=P_{i, j}^{(n)}-\sum_{k=1}^{n-1} f_{i, j}^{(k)} P_{j, j}^{(n-k)},\; n \in \mathbb{Z}^{+}. \label{eq3.3}\tag*{(3.3)} \]
    For $ n\ge 2 $, note that~\ref{eq3.3} yields a recursive means to compute $ f_{i,j}^{(n)} $.
\end{Regular}
\subsection*{Transience and Recurrence}
\begin{Regular}
    Define the related quantity:
    \[ f_{i,j}=\Prob{\text{DTMC ever makes a future visit to state $ j $}\given X_0=i}. \]
    Note that
    \begin{align*}
         & f_{i,j}                                                                                                                                                                \\
         & =\sum_{k=1}^{\infty} \Prob{\text{DTMC ever makes a future visit to state $ j $, DTMC visits state $ j $ for $ 1\textsuperscript{th} $ time at time $ k $}\given X_0=i} \\
         & =\sum_{k=1}^{\infty} \Prob{\text{DTMC visits state $ j $ for $ 1\textsuperscript{th} $ time at time $ k $}\given X_0=i}                                                \\
         & =\sum_{k=1}^{\infty} f_{i,j}^{(k)}\le 1.
    \end{align*}
\end{Regular}
This leads to the following important concept in the study of Markov chains.
\begin{Regular}
    \textbf{Definition}: State $ i $ is said to be \emph{transient} if $ f_{i,i}<1 $. On the other hand, state $ i $ is said to be
    \emph{recurrent} if $ f_{i,i}=1 $.
\end{Regular}
\begin{Regular}
    \begin{center}
        \begin{tikzpicture}
            \begin{axis}[
                    axis lines=middle,
                    xmin=0, xmax=7,
                    ymin=0, ymax=5,
                    xtick={0,1,2,3,4,5,6},
                    ytick={3},
                    xticklabels={$0$, \stackanchor{visit}{1}, \stackanchor{visit}{2}, \stackanchor{}{$\cdots$}, \stackanchor{visit}{$k-1$}, \stackanchor{visit}{$k$}, \stackanchor{}{$\cdots$}},
                    yticklabels={$i$},
                    xlabel = Time,
                    ylabel = $X_n$,
                ]
                \addplot[color=black,mark=*,only marks]
                coordinates {
                        (0,3)(1,3)(2,3)(4,3)(5,3)
                    };
                \draw [decoration={brace,raise=3pt},decorate,thick] (0,3) -- (1,3) node [above=3pt] at ($(0,3)!.5!(1,3)$) {$f_{i,i}$};
                \draw [decoration={brace,raise=3pt},decorate,thick] (1,3) -- (2,3) node [above=3pt] at ($(1,3)!.5!(2,3)$) {$f_{i,i}$};
                \node at ($(2,3)!.5!(4,3)$) {\ldots};
                \draw [decoration={brace,raise=3pt},decorate,thick] (4,3) -- (5,3) node [above=3pt] at ($(4,3)!.5!(5,3)$) {$f_{i,i}$};
                \draw [->,thick] (5,3) -- (7,3) node [above=3pt] at ($(5,3)!.5!(7,3)$) {$1-f_{i,i}$};
            \end{axis}
        \end{tikzpicture}
    \end{center}
    In what follows, we proceed to look at alternative ways
    of characterizing the notions of transience and recurrence.
    As such, let us first define $M_i$ to be a rv which
    counts the number of (future) times the DTMC visits
    state $i$ (disregarding the possibility of starting in state
    $i$ at time $0$). If we assume that $f_{i,i} < 1$, then the Markov
    property and stationary assumption jointly imply that
    \[ \Prob{M_i=k\given X_0=i}=\biggl(\prod_{n=1}^k f_{i,i}\biggr)(1-f_{i,i})=f_{i,i}^k(1-f_{i,i}),\; k=0,1,2,\ldots, \label{eq3.4}\tag*{(3.4)} \]
    as the DTMC will return to state $i$, $k$ times with probability $f_{i,i}$, and then never return with
    probability $ 1-f_{i,i} $.

    \textbf{We have}: $ \Prob{M_i=k\given X_0=i}=f_{i,i}^k(1-f_{i,i}),\; k=0,1,2,\ldots\leftarrow $~\ref{eq3.4}.

    We recognize~\ref{eq3.4} as the pmf of a $ \GEOf{1-f_{i,i}} $ rv, thereby implying that
    \[ \E{M_i\given X_0=i}=\frac{1-(1-f_{i,i})}{1-f_{i,i}} =\frac{f_{i,i}}{1-f_{i,i}} <\infty\text{ since }f_{i,i}<1. \]
    However, if $ f_{i,i}=1 $, then $ \Prob{M_i=\infty\given X_0=i}=1 $, immediately implying that $ \E{M_i\given X_0=i}=\infty $.
    Therefore, an equivalent way of viewing transience/recurrence is as follows:
    \[ \E{M_i\given X_0=i}\begin{cases}
            <\infty, & \text{iff state $ i $ is transient}, \\
            =\infty, & \text{iff state $ i $ is recurrent}.
        \end{cases} \]
    Following further on the notion of $ M_i $, define a sequence of indicator rvs $ \Set{A_n}_{n=1}^\infty $ such that
    \[ A_n=\begin{cases}
            0, & \text{if $ X_n\ne i $}, \\
            1, & \text{if $ X_n=i $}.
        \end{cases} \]
    With this definition, note that $ M_i=\sum_{n=1}^{\infty} A_n $. Now,
    \begin{align*}
        \E{M_i\given X_0=i}
         & =\E*{\sum_{n=1}^{\infty} A_n\given X_0=i}                                            \\
         & =\sum_{n=1}^{\infty} \E{A_n\given X_0=i}                                             \\
         & =\sum_{n=1}^{\infty} \Bigl[0\Prob{A_n=0\given X_0=i}+1\Prob{A_n=1\given X_0=i}\Bigr] \\
         & =\sum_{n=1}^{\infty} \Prob{X_n=i\given X_0=i}                                        \\
         & =\sum_{n=1}^{\infty} P_{i,i}^{(n)}.
    \end{align*}
    \textbf{We have}: $ \E{M_i\given X_0=i}=\sum_{n=1}^{\infty} P_{i,i}^{(n)} $.

    Therefore, yet another equivalent way of characterizing transience/recurrence is as follows:
    \[ \sum_{n=1}^{\infty} P_{i,i}^{(n)}\begin{cases}
            <\infty, & \text{iff state $ i $ is transient}, \\
            =\infty, & \text{iff state $ i $ is recurrent}.
        \end{cases} \]
    \underline{Remark}: A simple way of viewing these concepts is as follows: a recurrent state will be visited
    \emph{infinitely often}, whereas a transient state will only be visited \emph{finitely often}.
\end{Regular}
As was the case concerning the periodicity of states within the same communication class, the
next theorem indicates that transience/recurrence is \emph{also} a class property.
\begin{Result}
    \textbf{Theorem 3.2}. If $ i\leftrightarrow j $ and state $ i $ is recurrent, then state $ j $ is recurrent.
    \tcblower{}
    \textbf{Proof}: Since $ i\leftrightarrow j $, $ \exists m,n\in\mathbb{N} $
    such that $ P_{i,j}^{(m)}>0 $ and $ P_{j,i}^{(n)}>0 $. Also,
    since state $ i $ is recurrent we know that $ \sum_{\ell=1}^{\infty} P_{i,i}^{(\ell)}=\infty $.
    Suppose that $ s\in\mathbb{Z}^+ $. Note that
    \[ P_{j,j}^{(n+s+m)}\ge P_{j,i}^{(n)}P_{i,i}^{(s)}P_{i,j}^{(m)}. \]
    Look at
    \begin{align*}
        \sum_{k=1}^{\infty} P_{j,j}^{(k)}
         & \ge \sum_{k=n+m+1}^{\infty} P_{j,j}^{(k)}                                                                                                         \\
         & =\sum_{s=1}^{\infty} P_{j,j}^{(n+s+m)}                                                                                  &  & \text{let $s=k-n-m$} \\
         & \ge \sum_{s=1}^{\infty} P_{j,i}^{(n)}P_{i,i}^{(s)}P_{i,j}^{(m)}                                                                                   \\
         & =\underbrace{P_{j,i}^{(n)}}_{>0}\underbrace{P_{i,j}^{(m)}}_{>0}\underbrace{\sum_{s=1}^{\infty} P_{i,i}^{(s)}}_{=\infty}                           \\
         & =\infty.
    \end{align*}
    Therefore, $ \sum_{k=1}^{\infty} P_{j,j}^{(k)}=\infty $ and state $ j $
    is recurrent.
\end{Result}
\underline{Remark}: An obvious by-product of this theorem is that if $ i\leftrightarrow j $ and state $ i $ is transient,
then state $ j $ must also be transient.

The following theorem serves as a companion result to Theorem 3.2.
\begin{Result}
    \textbf{Theorem 3.3}. If $ i\leftrightarrow j $ and state $ i $ is recurrent, then
    \[ f_{i,j}=\Prob{\text{DTMC ever makes a future visit to state $ j $}\given X_0=i}=1. \]
    \tcblower{}
    \textbf{Proof}: Clearly, the result is true if $ i=j $. Therefore, suppose
    that $ i\ne j $. Since $ i\leftrightarrow j $, the fact that state $ i $
    is recurrent implies that state $ j $ is recurrent by Theorem 3.2,
    and $ f_{j,j}=1 $. To prove that $ f_{i,j}=1 $, suppose that $ f_{i,j}<1 $
    and try to get a contradiction. Since $ i\leftrightarrow j $,
    $ \exists n\in\mathbb{Z}^+ $, such that $ P_{j,i}^{(n)}>0 $. Let
    $ n_i $ be the \underline{smallest} such $ n $ satisfying
    $ P_{j,i}^{(n)}>0 $. Thus, each time the DTMC visits state $ j $,
    there is the possibility of being in state $ i $, $ n_i $ time
    units later (with probability $ P_{j,i}^{(n_i)}>0 $). If we suppose that
    $ f_{i,j}<1 $, then this implies that the probability of
    returning to state $ j $ after visiting state $ i $ in the
    future is not guaranteed (as $1-f_{i,j}>0$). Therefore, we have:
    \begin{align*}
        1-f_{j,j}
         & =\Prob{\text{DTMC \underline{never} makes a future visit to state $ j $}\given X_0=j} \\
         & \ge \underbrace{P_{j,i}^{(n_i)}}_{>0}\underbrace{(1-f_{i,j})}_{>0}                    \\
         & >0\implies 1-f_{j,j}>0\implies f_{j,j}<1,
    \end{align*}
    which is a contradiction since $ f_{j,j}=1 $. Thus, it must hold true
    that $ f_{i,j}=1 $ when state $ i $ is recurrent and $ i\leftrightarrow j $.
\end{Result}
\underline{Remark}: Based on the above result, we know that, starting from any state of a recurrent class,
a DTMC will visit each state of that class infinitely many times.

At this stage, a natural question to ask is ``\emph{What do the results that we have accumulated thus
    far tell us about the behaviour of states within the same communication class?}'' The answer
would be:
\begin{enumerate}[(i)]
    \item these states communicate with each other,
    \item these states all have the same period,
    \item these states are all either recurrent or all transient.
\end{enumerate}
In fact, in an irreducible DTMC, there is only one communication class and so all the states
are either recurrent or transient. When the assumption that the DTMC has a \underline{finite} number of
states is included, we obtain the following important result.
\begin{Result}
    \textbf{Theorem 3.4}. A finite-state DTMC has at least one recurrent state.
    \tcblower{}
    \textbf{Proof}: We wish to prove the existence of at least
    one recurrent state in a finite-state DTMC, or equivalently,
    that not all states can be transient. Suppose that $ \Set{0,1,\ldots,N} $
    represents the states of the DTMC where $ N<\infty $. To prove
    that not all states can be transient, suppose they
    are all transient and try to get a contradiction. Now, for each $ i=0,1,2,\ldots N $,
    if state $ i $ is assumed to be transient, then we know that after
    a \underline{finite} amount of time (denoted by $ T_i $),
    state $ i $ will never be visited again. As a result, after a finite amount
    of time
    \[ T=\max{T_0,T_1,\ldots,T_N} \]
    has gone by, none of these states will be visited ever again. However,
    the DTMC must be in some state after time $ T $, but we have exhausted
    all states for the DTMC to be in. This is a contradiction. Thus,
    not all states can be transient in a finite-state DTMC\@.
\end{Result}
\underline{Remarks}:
\begin{enumerate}[(1)]
    \item Looking at the above result, it is useful to think of it in the following way. There must be
          at least one recurrent state. After all, the DTMC has to spend its time somewhere, and if
          it visits each of its \emph{finitely many states finitely many times}, then where else could it
          possibly go?
    \item As an immediate consequence of Theorem 3.4, an irreducible, finite-state DTMC must be
          recurrent (i.e., all states of the DTMC are recurrent).
\end{enumerate}
\begin{Example}
    \textbf{Example 3.3}. (\emph{continued}) Recall that we considered the irreducible DTMC with TPM
    \[ P=\begin{bNiceMatrix}[first-row,first-col]
              & 0   & 1 & 2   & 3 \\
            0 & 0   & 1 & 0   & 0 \\
            1 & 0   & 0 & 1   & 0 \\
            2 & 0   & 0 & 0   & 1 \\
            3 & 0.5 & 0 & 0.5 & 0
        \end{bNiceMatrix}. \]
    Determine whether each state is transient or recurrent.
    \tcblower{}
    \textbf{Solution}: Since this is a finite-state DTMC as well, each
    of states $ 0 $, $ 1 $, $ 2 $, and $ 3 $ is therefore recurrent.
\end{Example}
Another interesting property concerning recurrence can also be deduced.
\begin{Result}
    \textbf{Theorem 3.5}. If state $i$ is recurrent and state $i$ does not communicate with state $j$, then
    $ P_{i,j}^{(k)}=0\;\forall k\in\mathbb{Z}^+ $.
    \tcblower{}
    \textbf{Proof}: Let us assume that $ P_{i,j}^{(k)}>0 $ for some $ k\in\mathbb{Z}^+ $.
    Let $ k_i $ be the smallest $ k $ that satisfies $ P_{i,j}^{(k_i)}>0 $.
    Then, $ P_{j,i}^{(n)} $ would be equal to $ 0 \;\forall n\in\mathbb{Z}^+ $,
    since otherwise, states $ i $ and $ j $ would communicate. However,
    the DTMC, starting in state $ i $, would be a positive probability
    of at least $ P_{i,j}^{(k_i)} $ of never returning to state $ i $
    (by the nature of how $ k_i $ was chosen). This contradicts the recurrence
    of state $ i $. Hence, we must have $ P_{i,j}^{(k)}=0\; \forall k\in\mathbb{Z}^+ $.
\end{Result}
\begin{Example}
    \textbf{Example 3.4}. (\emph{continued}) Recall our earlier DTMC with TPM
    \[ P=\begin{bNiceMatrix}[first-row,first-col,cell-space-limits=1pt]
              & 0           & 1           & 2           & 3           \\
            0 & \frac{1}{3} & \frac{2}{3} & 0           & 0           \\
            1 & \frac{1}{2} & \frac{1}{4} & \frac{1}{8} & \frac{1}{8} \\
            2 & 0           & 0           & 1           & 0           \\
            3 & \frac{3}{4} & \frac{1}{4} & 0           & 0
        \end{bNiceMatrix}. \]
    Determine whether each state is transient or recurrent.
    \tcblower{}
    \textbf{Solution}: We previously found the communication
    classes for this DTMC were $ \Set{0,1,3} $ and $ \Set{2} $.
    \[ \sum_{n=1}^{\infty} P_{2,2}^{(n)}=\sum_{n=1}^{\infty} 1=\infty\implies\text{state $2$ is recurrent}. \]
    Looking at the possible transitions that can take place among states $ 0 $,
    $ 1 $, and $ 3 $, we strongly suspect state $ 1 $ to be \underline{transient}
    (since there is a positive probability of never returning to state $ 1 $ if a
    transition to state $ 2 $ occurs). To prove this formally,
    assume instead that state $ 1 $ is recurrent and try to get a contradiction.
    Assuming that state $ 1 $ is recurrent, note that
    state $ 1 $ does not communicate with state $ 2 $. By Theorem 3.5,
    we have $ P_{1,2} $ must be equal to $ 0 $. But in fact, we have
    $ P_{1,2}=1/8\ne 0 $. This is a contradiction. Thus, state $ 1 $
    must indeed be transient. Thus, state $ 1 $ must be transient,
    and so $ \Set{0,1,3} $ is a transient class.
\end{Example}
\underline{Remark}: As the previous example illustrates, the contrapositive of Theorem 3.5 also provides a
test for transience, in that if $ \exists k\in\mathbb{Z}^+ $ such that $ P_{i,j}^{(k)}>0 $
and states $i$ and $j$ do not
communicate, then state $i$ must be transient. Moreover, this result implies that once a process
enters a recurrent class of states, it can never leave that class. For this reason, a recurrent
class is often referred to as a \emph{closed class}.
\begin{Example}
    \textbf{Example 3.8}. Consider a DTMC with TPM
    \[ P=\begin{bNiceMatrix}[first-row,first-col,cell-space-limits=1pt]
              & 0           & 1           & 2           & 3           \\
            0 & \frac{1}{4} & 0           & \frac{3}{4} & 0           \\
            1 & 0           & \frac{1}{3} & 0           & \frac{2}{3} \\
            2 & 0           & 1           & 0           & 0           \\
            3 & 0           & \frac{2}{5} & 0           & \frac{3}{5}
        \end{bNiceMatrix}. \]
    Determine whether each state is transient or recurrent.
    \tcblower{}
    \textbf{Solution}: There are three communication classes,
    namely $ \Set{0} $, $ \Set{1,3} $, and $ \Set{2} $.
    \[ \sum_{n=1}^{\infty} P_{0,0}^{(n)}=\sum_{n=1}^{\infty} \biggl(\frac{1}{4}\biggr)^{\!n}=\frac{1/4}{1-1/4}=\frac{1}{3} <\infty,   \]
    and hence state $ 0 $ is transient.
    \[ \sum_{n=1}^{\infty} P_{2,2}^{(n)}=\sum_{n=1}^{\infty} 0=0<\infty, \]
    and hence state $ 2 $ is transient.

    On the other hand, concerning $ \Set{1,3} $,
    we observe that
    \[ f_{1,1}^{(1)}=P_{1,1}=\frac{1}{3}, \]
    and
    \[ f_{1,1}^{(n)}=\frac{2}{3}\biggl(\frac{3}{5}\biggr)^{\!n-2}\biggl(\frac{2}{5} \biggr),\; n\ge 2. \]
    Thus,
    \begin{align*}
        f_{1,1}
         & =\sum_{n=1}^{\infty} f_{1,1}^{(n)}                                                                      \\
         & =\frac{1}{3} +\sum_{n=2}^{\infty}\frac{2}{3}\biggl(\frac{3}{5}\biggr)^{\!n-2}\biggl(\frac{2}{5} \biggr) \\
         & =\frac{1}{3} +\frac{2}{3}\biggl(\frac{2}{5} \biggr) \frac{1}{1-3/5}                                     \\
         & =\frac{1}{3} +\frac{2}{3}\biggl(\frac{2}{5} \biggr)\biggl(\frac{5}{2} \biggr)                           \\
         & =1.
    \end{align*}
    By definition, state $ 1 $ is recurrent, and hence the class $ \Set{1,3} $ is recurrent.
\end{Example}
\underline{Remark}: Instead of showing that $ f_{1,1}=1 $ in the previous example, we could have used an even
simpler argument to conclude that $ \Set{1,3} $ is a recurrent class. In particular, after establishing
that $ \Set{0} $ and $ \Set{2} $ are transient classes, this DTMC has a finite number of states, and so $\Set{1, 3}$
must be recurrent due to Theorem 3.4.
\subsection*{Random Walk}
\begin{Example}
    \textbf{Example 3.9}. Consider a DTMC $ \Set{X_n,n\in\mathbb{N}} $ whose state space $ \mathcal{S} $
    is the set of \underline{all} integers (i.e., $ \mathcal{S}=\mathbb{Z} $). Furthermore, suppose
    that the TPM for this DTMC satisfies
    \[ P_{i,i-1}=1-p\text{ and }P_{i,i+1}=p\;\forall i\in\mathbb{Z}\text{ where }0<p<1. \]
    In other words, from any state, either a jump up by one unit or a jump down by one unit takes
    place in the next transition. As such, $ X_n $ is expressible as $ X_n=\sum_{k=0}^{n} Y_k $ where $ \Set{Y_k}_{k=0}^\infty $
    is an independent sequence of rvs with $ Y_0=X_0 $ and $ \Prob{Y_k=-1}=1-p $ and $ \Prob{Y_k=1}=p $,
    $ k\in\mathbb{Z}^+ $. This DTMC is well-studied in the literature and is the basis for many applications
    in a variety of areas (particularly in finance). It is often referred to as the \emph{Random Walk} or
    \emph{Drunkard's Walk}. Characterize the behaviour of this DTMC in terms of its communication
    classes, periodicity, and transience/recurrence.
    \tcblower{}
    \textbf{Solution}: \underline{State Transition Diagram}
    \begin{center}
        \begin{tikzpicture}[thick]
            \node (-3) at (-6,0) {$\cdots$};
            \node [place] (-2) at (-4,0) {$-2$};
            \node [place] (-1) at (-2,0) {$-1$};
            \node [place] (0) at ( 0,0) {$0$};
            \node [place] (1) at (2,0) {$1$};
            \node [place] (2) at (4,0) {$2$};
            \node (3) at (6,0) {$\cdots$};
            \draw[->] (-3.north east) to [out=45,in=135] (-2.north west);
            \draw[->] (-2.north east) to [out=45,in=135] (-1.north west);
            \draw[->] (-1.north east) to [out=45,in=135] (0.north west);
            \draw[->] (0.north east) to [out=45,in=135] (1.north west);
            \draw[->] (1.north east) to [out=45,in=135] (2.north west);
            \draw[->] (2.north east) to [out=45,in=135] (3.north west);
            \draw[->,rotate=180] (3.south west) to [out=45,in=135] (2.south east);
            \draw[->,rotate=180] (2.south west) to [out=45,in=135] (1.south east);
            \draw[->,rotate=180] (1.south west) to [out=45,in=135] (0.south east);
            \draw[->,rotate=180] (0.south west) to [out=45,in=135] (-1.south east);
            \draw[->,rotate=180] (-1.south west) to [out=45,in=135] (-2.south east);
            \draw[->,rotate=180] (-2.south west) to [out=45,in=135] (-3.south east);
        \end{tikzpicture}
    \end{center}


    Since $ 0<p<1 $, all states clearly communicate with
    each other. This implies that $ \Set{X_n,n\in\mathbb{N}} $
    is an irreducible DTMC\@. Hence, we can determine its
    periodicity (and likewise transience/recurrence) by analysing
    any state we wish. Let us select state $0$. Starting from state $0$,
    note that we cannot possibly be visited in an odd number of transitions,
    since we are guaranteed to have the number of up (down) jumps
    exceed the number of down (up) jumps. Thus,
    \[ P_{0,0}^{(1)}=P_{0,0}^{(3)}=\cdots=0, \]
    or equivalently
    \[ P_{0,0}^{(2n-1)}=0\;\forall n\in\mathbb{Z}^+. \]
    However, since it is clearly possible to return to state $ 0 $ in an \underline{even}
    number of transitions, it immediately follows that
    \[ P_{0,0}^{(2n)}>0\;\forall n\in\mathbb{Z}^+. \]
    Hence,
    \begin{align*}
        d(0)
         & =\gcd{n\in\mathbb{Z}^+:P_{0,0}^{(n)}>0} \\
         & =\gcd{2,4,6,\ldots}                     \\
         & =2.
    \end{align*}
    Finally, to determine whether state $ 0 $ is transient \underline{or}
    recurrent, let us consider
    \begin{align*}
        \sum_{n=1}^{\infty} P_{0,0}^{(n)}
         & =\underbrace{P_{0,0}^{(1)}}_{=0}+P_{0,0}^{(2)}+\underbrace{P_{0,0}^{(3)}}_{=0}+P_{0,0}^{(4)}+\cdots \\
         & =\sum_{n=1}^{\infty} P_{0,0}^{(2n)}                                                                 \\
         & =\sum_{n=1}^{\infty} \binom{2n}{n}p^n(1-p)^{n},
    \end{align*}
    where the last equality follows from the fact that in order for
    the DTMC to return to state $ 0 $ from state $ 0 $ in $ 2n $ steps,
    there must be an equal number ($ n $) of up and down jumps,
    and $ \binom{2n}{n} $ represents the number of ways these jumps could be arranged
    among the $ 2n $ steps.

    \underline{Recall}: (\emph{Ratio Test for Series}).
    Suppose that $ \sum_{n=1}^{\infty} a_n $ is a series of positive
    terms and $ L=\lim\limits_{{n} \to {\infty}} \frac{a_{n+1}}{a_n} $.
    \begin{enumerate}[(i)]
        \item If $ L<1 $, the series converges.
        \item If $ L>1 $, the series diverges.
        \item If $ L=1 $, the test is inconclusive.
    \end{enumerate}
    In our case, $ a_n=\binom{2n}{n}p^n(1-p)^n $.
    \begin{align*}
        L
         & =\lim\limits_{{n} \to {\infty}} \frac{a_{n+1}}{a_n}                                                                    \\
         & =\lim\limits_{{n} \to {\infty}} \frac{\frac{(2(n+1))!}{(n+1)!(n+1)!}p^{n+1}(1-p)^{n+1}}{\frac{(2n)!}{n!n!}p^n (1-p)^n} \\
         & =\lim\limits_{{n} \to {\infty}} \frac{(2n+2)(2n+1)}{(n+1)(n+1)} p(1-p)                                                 \\
         & =4p(1-p).
    \end{align*}
    A plot of $ L=4p(1-p) $ reveals the following shape:
    \begin{center}
        \begin{tikzpicture}
            \begin{axis}[
                    xmin=0, xmax=1.1,
                    ymin=0, ymax=1.1,
                    xtick={0,0.5,1},
                    ytick={0,1},
                    legend pos=north west,
                    grid style=dashed,
                    axis lines=middle,
                    xlabel = $p$,
                    ylabel = $L$,
                ]
                \addplot[smooth,domain=0:1]{4*x*(1-x)};
            \end{axis}
        \end{tikzpicture}
    \end{center}
    Note that if $ p\ne 1/2 $, then $ L<1 $. By the ratio test, this implies that
    \[ \sum_{n=1}^{\infty} \binom{2n}{n}p^n(1-p)^n<\infty, \]
    and so state $0$ is therefore \underline{transient}. Thus, the entire DTMC
    is transient when $ p\ne 1/2 $.
    On the other hand, if $ p=1/2 $, then $ L=1 $, and so the ratio test is
    inconclusive. To determine what is happening when $ p=1/2 $,
    we consider an alternative approach in which $ p=1/2 $ and $ p\ne 1/2 $
    can both be handled. First, recall that
    \[ f_{i,j}=\Prob{\text{DTMC ever makes a future visit to state $ j $}\given X_0=i}. \]
    Let $ q=1-p $. Condition on the state of the DTMC at time $ 1 $ to get
    \begin{align*}
        f_{0,0}
         & =\Prob{\text{DTMC ever makes a future visit to state $ 0 $}\given X_0=0}                                    \\
         & =\Prob{X_1=-1\given X_0=0}\Prob{\text{DTMC ever makes a future visit to state $ 0 $}\given X_1=-1,X_0=0}    \\
         & \quad+\Prob{X_1=1\given X_0=0}\Prob{\text{DTMC ever makes a future visit to state $ 0 $}\given X_1=1,X_0=0} \\
         & =q f_{-1,0}+p f_{1,0}
    \end{align*}
    We have
    \begin{equation}
        \boxed{f_{0,0}=q f_{-1,0}+p f_{1,0}}.\label{eq3.5}\tag*{(3.5)}
    \end{equation} If we let $ \mathcal{F}_0 $
    represent the event that the DTMC ever makes a future visit to state $ 0 $,
    then
    \[ \mathcal{F}_0=\bigcup_{i=1}^\infty\Set{X_i=0}. \] So,
    \begin{align*}
        f_{1,0}
         & =\Prob{\mathcal{F}_0\given X_0=1}                                                                                              \\
         & =\Prob[\big]{\mathcal{F}_0\cap\Set{X_1=0}\given X_0=1}+\Prob[\big]{\mathcal{F}_0\cap\Set{X_1=2}\given X_0=1}                   \\
         & =\Prob{\mathcal{F}_0\given X_1=0,X_0=1}\Prob{X_1=0\given X_0=1}+\Prob{\mathcal{F}_0\given X_1=2,X_0=1}\Prob{X_1=2\given X_0=1} \\
         & =\Prob{X_1=0\given X_0=1}+\Prob{X_1=2\given X_0=1}\Prob{\mathcal{F}_0\given X_1=2,X_0=1}                                       \\
         & =q+p\Prob{\mathcal{F}_0\given X_1=2}\text{ due to the Markov property}                                                         \\
         & =q+p\Prob[\bigg]{\bigcup_{i=2}^\infty\Set{X_i=0}\cup \Set{X_1=0}\given X_1=2}                                                  \\
         & =q+p\Prob[\bigg]{\bigcup_{i=2}^\infty\Set{X_i=0}\given X_1=2}                                                                  \\
         & =q+p\Prob{\mathcal{F}_0\given X_0=2}\text{ due to the stationary assumption}                                                   \\
         & =q+p f_{2,0}\text{ by definition}.
    \end{align*}
    Moreover, it follows that
    \begin{align*}
        f_{1,0}
         & =q+pf_{2,0}                             \\
         & =q+pf_{2,1}f_{1,0}                      \\
         & =q+pf_{1,0}f_{1,0}                      \\
         & =q+pf_{1,0}^2.\label{eq3.6}\tag*{(3.6)}
    \end{align*}
    Rewriting~\ref{eq3.6}, we end up with
    \[ pf_{1,0}^2-f_{1,0}+q=0, \]
    which is a quadratic equation in $ f_{1,0} $. Applying the quadratic formula,
    we get that
    \begin{align*}
        f_{1,0}
         & =\frac{1\pm \sqrt{1-4pq}}{2p}                                     \\
         & =\frac{1\pm \sqrt{(p+q)^2-4pq}}{2p}     &  & \text{since $p+q=1$} \\
         & =\frac{1\pm \sqrt{p^2+2pq+q^2-4pq}}{2p}                           \\
         & =\frac{1\pm \sqrt{p^2+2pq+q^2}}{2p}                               \\
         & =\frac{1\pm \sqrt{(p-q)^2}}{2p}                                   \\
         & =\frac{1\pm \abs{p-q}}{2p}.
    \end{align*}
    Let
    \[
        r_1=\frac{1+\abs{p-q}}{2p},\text{ and }r_2=\frac{1-\abs{p-q}}{2p}
    \]
    denote the two roots, and let us consider the case we are mostly
    interested in, which is when $p=q$ (i.e., $ p=1/2 $). In this case,
    $ r_1 $ and $ r_2 $ yield the same value, namely
    \[ \frac{1\pm \abs{1/2-1/2}}{2(1/2)}=1,  \]
    and so it must be that $ f_{1,0}=1 $. Similarly, it follows
    (via a symmetry argument) that $ f_{-1,0}=1 $ when $ p=1/2 $.
    Therefore, for $ p=1/2 $,~\ref{eq3.5} simplifies to become
    \[ f_{0,0}=\frac{1}{2}(1)+\frac{1}{2} (1)=1, \]
    implying that state $ 0 $ is recurrent by definition. Thus, we conclude
    that the DTMC is recurrent only when $ p=1/2 $. Out of mathematical interest,
    let us now attempt to determine $ f_{0,0} $ for $ p\ne q $. Consider the
    special case when $ p<q $. Then $ \abs{p-q}=-(p-q) $ and the roots
    $ r_1 $ and $ r_2 $ simplify to become
    \[ r_1=\frac{1-(p-q)}{2p}=\frac{1-p+q}{2p}=\frac{2q}{2p}=\frac{q}{p}>1,  \]
    and
    \[ r_2=\frac{1+(p-q)}{2p}=\frac{1-q+p}{2p}=\frac{2p}{2p}=1. \]
    Since $ 0\le f_{1,0}\le 1 $, the root $ r_1 $ must be inadmissible,
    thereby implying that $ r_2 $ is the correct root to use in this case.
    Moreover, by interchanging the up and down jump probabilities and applying
    the same symmetry argument above, it readily follows that
    \[ f_{-1,0}=1\text{ for $p>q$}. \]
    \underline{Remark}: For $ p>q $, it can be shown that $ r_2 $
    is again the admissible root for $ f_{1,0} $, ultimately leading to
    $ f_{1,0}=q/p $ (see Exercise 3.2.3). As an immediate consequence,
    we also have $ f_{-1,0}=p/q $ for $ p<q $.
    Combining our findings for $ p<q $ and $ p>q $, we have that
    \[ f_{1,0}=\frac{1-\abs{p-q}}{2p} \text{ and }f_{-1,0}=\frac{1-\abs{q-p}}{2q}. \]
    Therefore,~\ref{eq3.5} gives rise to
    \begin{align*}
        f_{0,0}
         & =q\biggl(\frac{1-\abs{q-p}}{2q} \biggr)+p\biggl(\frac{1-\abs{p-q}}{2p} \biggr) \\
         & =1-\frac{1}{2} \abs{q-p}-\frac{1}{2} \abs{p-q}                                 \\
         & =1-\frac{1}{2} \bigl(\abs{q-p}+\abs{p-q}\bigr).
    \end{align*}
    If $ p>q $ (i.e., $ 2q<1 $), then
    \begin{align*}
        f_{0,0}
         & =1-\frac{1}{2} \bigl(-(q-p)+(p-q)\bigr) \\
         & =1-\frac{1}{2}(2p-2q)                   \\
         & =1-p+q                                  \\
         & =2q<1.\label{eq3.7}\tag*{(3.7)}
    \end{align*}
    Therefore, state $ 0 $ is transient. On the other hand, if $ p<q $ (i.e., $ 2p<1 $),
    it can be shown that $ f_{0,0}=2p<1 $ (see Exercise 3.2.4), implying also
    that state $ 0 $ is transient. Thus, if both cases are combined, we end up
    obtaining
    \[ f_{0,0}=2\min{p,q}<1\text{ for $ p\ne q $ (i.e., $ p\ne 1/2 $)}. \]
    However, this formula even gives the correct result
    when $ p=q=1/2 $. In general,
    \[ f_{0,0}=2\min{p,1-p},\; 0<p<1. \]
\end{Example}
