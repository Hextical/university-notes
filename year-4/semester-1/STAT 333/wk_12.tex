\makeheading{Week 12}{\daterange{2021-12-01}{2021-12-07}}%chktex 8
\section{Two Important Generalizations}
\subsection*{The Non-homogeneous Poisson Process}
Oftentimes, we find the Poisson process difficult to apply in applications of real-life
phenomena, largely due to the fact that the Poisson process assumes a \emph{constant} arrival rate of
$ \lambda $ for all time. In what follows, we consider a more general type of process in which the arrival
rate is allowed to vary as a function of time.
\begin{Regular}
    \textbf{Definition}: The counting process $ \Set[\big]{N(t),t\ge 0} $ is a \emph{non-homogeneous} (or non-stationary)
    Poisson process with \emph{rate function} $ \lambda(t) $ if the following three conditions hold true:
    \begin{enumerate}[(1)]
        \item $ \Set[\big]{N(t),t\ge 0} $ has independent increments.
        \item $ \Prob[\big]{N(t+h)-N(t)=1}=h\lambda(t)+\order{h} $.
        \item $ \Prob[\big]{N(t+h)-N(t)\ge 2}=\order{h} $.
    \end{enumerate}
\end{Regular}
Many applications that generate random points in time are modelled more realistically with\\
non-homogeneous processes. For instance:
\begin{itemize}
    \item The rate of customers entering a supermarket is not the same during the entire day.
    \item The average arrival rate of vehicles on a highway fluctuates between its maximum during
          rush hours and its minimum during low traffic times.
\end{itemize}
The mathematical cost of this generalization, however, is that we lose the stationary
increments property.

For $ s_1,s_2\ge 0 $, the following theorem specifies the distribution of $ N(s_1+s_2)-N(s_1) $.
\begin{Result}
    \textbf{Theorem 4.8}. If $ \Set[\big]{N(t),t\ge 0} $ is a non-homogeneous Poisson process with rate function
    $ \lambda(t) $, then $ N(s_1+s_2)-N(s_1)\sim \POI[\big]{m(s_1+s_2)-m(s_1)} $ where the \emph{mean value function}
    $ m(t) $ is given by
    \[ m(t)=\int_{0}^{t}\lambda(\tau)\odif{\tau},\; t\ge 0. \]
    \tcblower{}
    \textbf{Proof}: Let $ \phi_u(s_1,s_2)=\E[\Big]{e^{u\bigl(N(s_1+s_2)-N(s_1)\bigr)}} $, which is the mgf of $ N(s_1+s_2)-N(s_1) $,
    where $ u $ serves as the argument of the mgf. For $ h\ge 0 $, we first note that
    \begin{align*}
        \phi_u(s_1,s_2+h)
         & =\E[\Big]{e^{u\bigl(N(s_1+s_2+h)-N(s_1)\bigr)}}                                                                                      \\
         & =\E[\Big]{e^{u\bigl(N(s_1+s_2+h)-N(s_1+s_2)+N(s_1+s_2)-N(s_1)\bigr)}}                                                                \\
         & =\E[\Big]{e^{u\bigl(N(s_1+s_2+h)-N(s_1+s_2)\bigr)}e^{u\bigl(N(s_1+s_2)-N(s_1)\bigr)}}                                                \\
         & =\E[\Big]{e^{u\bigl(N(s_1+s_2+h)-N(s_1+s_2)\bigr)}}\E[\big]{e^{u\bigl(N(s_1+s_2)-N(s_1)\bigr)}}\text{ due to independent increments} \\
         & =\phi_u(s_1+s_2,h)\phi_u(s_1,s_2).\label{eq4.13}\tag*{(4.13)}
    \end{align*}
    Applying a similar approach to that used in the proof of Theorem 4.3, it can be shown that~\ref{eq4.13} ultimately give rise
    to the first-order differential equation (see Exercise 4.4.1).
    \begin{align*}
        \odv*{\phi_u(s_1,s_2)}{s_2}
                                                             & =\lambda(s_1+s_2)(e^u-1)\phi_u(s_1,s_2)                \\
        \odv*{\phi_u(s_1,t)}{t}
                                                             & =\lambda(s_1+t)(e^u-1)\phi_u(s_1,t)\text{ let $s_2=t$} \\
        \frac{\odv*{\phi_u(s_1,t)}{t}}{\phi_u(s_1,t)}        & =\lambda(s_1+t)(e^u-1)                                 \\
        \odv*{\LN[\big]{\phi_u(s_1,t)}}{t}                   & =\lambda(s_1+t)(e^u-1)                                 \\
        \odif{\LN[\big]{\phi_u(s_1,t)}}                      & =\lambda(s_1+t)(e^u-1)\odif{t}                         \\
        \int_{0}^{s_2}\odif{\LN[\big]{\phi_u(s_1,t)}}        & =\int_{0}^{s_2}\lambda(s_1+t)(e^u-1)\odif{t}           \\
        \biggl[\LN[\big]{\phi_u(s_1,t)}\biggr]_{t=0}^{t=s_2} & =(e^u-1)\int_{0}^{s_2}\lambda(s_1+t)(e^u-1)\odif{t}    \\
        \LN[\big]{\phi_u(s_1,s_2)}-\LN[\big]{\phi_u(s_1,0)}  & =(e^u-1)\int_{s_1}^{s_1+s_2}\lambda(\tau)\odif{\tau}.
    \end{align*}
    However,
    \[ \phi_u(s_1,0)=\E[\Big]{e^{u\bigl(N(s_1)-N(s_1)\bigr)}}=\E{e^{u\cdot 0}}=\E{e^0}=1. \]
    \begin{align*}
        \LN[\big]{\phi_u(s_1,s_2)} & =(e^u-1)\int_{s_1}^{s_1+s_2}\lambda(\tau)\odif{\tau}                        \\
        \phi_u(s_1,s_2)            & =e^{(e^u-1)\int_{s_1}^{s_1+s_2}\lambda(\tau)\odif{\tau}},\; u\in\mathbb{R}.
    \end{align*}
    Since
    \begin{align*}
        \int_{s_1}^{s_1+s_2}\lambda(\tau)\odif{\tau}
         & =\int_{0}^{s_1+s_2}\lambda(\tau)\odif{\tau}-\int_{0}^{s_1}\lambda(\tau)\odif{\tau} \\
         & =m(s_1+s_2)-m(s_1).
    \end{align*}
    Thus, we have
    \[
        \phi_u(s_1,s_2)=e^{(e^u-1)\bigl(m(s_1+s_2)-m(s_1)\bigr)},\; u\in\mathbb{R},
    \]
    which is the mgf of a $ \POI[\big]{m(s_1+s_2)-m(s_1)} $ rv. By the mgf uniqueness property,
    \[  N(s_1+s_2)-N(s_1)\sim \POI[\big]{m(s_1+s_2)-m(s_1)}. \]
\end{Result}
\underline{Remarks}:
\begin{enumerate}[(1)]
    \item As a direct consequence of Theorem 4.8, for all $ s,t\ge 0 $, we have
          \[ \Prob[\big]{N(s+t)-N(s)=n}=e^{-\bigl(m(s+t)-m(s)\bigr)}\frac{\bigl(m(s+t)-m(s)\bigr)^n}{n!},\; n=0,1,2,\ldots. \]
    \item If the rate function $ \lambda(\tau)=\lambda\; \forall \tau\ge 0 $, then note that
          \[ \int_{s}^{s+t}\lambda(\tau)\odif{\tau}=\int_{s}^{s+t}\lambda\odif{\tau}=\lambda(s+t-s)=\lambda t \]
          and
          \[ \Prob[\big]{N(s+t)-N(s)=n}=\frac{e^{-\lambda t}(\lambda t)^n}{n!},\; n=0,1,2,\ldots. \]
          This is expected, since $ \Set[\big]{N(t),t\ge 0} $ simplifies to become the standard (i.e.,
          stationary) Poisson process.
\end{enumerate}
\begin{Example}
    \textbf{Example 4.7}. Requests for technical support within the statistics department occur according
    to a non-homogeneous Poisson process $ \Set[\big]{N(t),t\ge 0} $ having rate function
    \[ \lambda(t)=\begin{cases}
            5t/2,       & \text{if $0\le t<1$},    \\
            t^2/2+2,    & \text{if $1\le t<4$},    \\
            (9-t)(t-2), & \text{if $4\le t\le 9$},
        \end{cases} \]
    where $t$ is measured in hours from the start of the workday. What is the probability that four
    requests occur in the first two hours of the workday and ten more occur in the final two hours
    of the workday?
    \tcblower{}
    \textbf{Solution}: The plot below provides a visual depiction of the rate function in use:
    \begin{center}
        \begin{tikzpicture}
            \begin{axis}[
                    xmin=0, xmax=10,
                    ymin=0, ymax=13,
                    xtick={0,1,4,5,9},
                    ytick={0,5/2,10,12},
                    yticklabels={$0$,$5/2$,$10$,$12$},
                    legend pos=north west,
                    axis lines=middle,
                    xlabel = $t$,
                    ylabel = $\lambda(t)$,
                    ymajorgrids=true,
                    xmajorgrids=true,
                    grid style=dashed,
                ]
                \addplot[domain=0:1]{5*x/2};
                \addplot[domain=1:4]{x^2/2+2};
                \addplot[domain=4:9]{(9-x)*(x-2)};
                %\addplot[name path=null, draw=none]{5/6};
                %\addplot[fill=blue,fill opacity=0.05]
                %fill between[of=f and null,soft clip={domain=0:5/12},];
            \end{axis}
        \end{tikzpicture}
    \end{center}
    We wish to calculate
    \[ \Prob[\big]{N(2)=4,N(9)-N(7)=10}. \]
    Using the independent increments property, we first have
    \[ \Prob[\big]{N(2)=4,N(9)-N(7)=10}=
        \Prob[\big]{N(2)=4}\Prob[big]{N(9)-N(7)=10}. \]
    Now, we need to calculate
    \begin{align*}
        m(2)-m(0)
         & =\int_{0}^{2}\lambda(t)\odif{t}                                                      \\
         & =\int_{0}^{1}\frac{5t}{2}\odif{t}+\int_{1}^{2}\frac{t^2}{2}+2\odif{t}                \\
         & =\biggl[\frac{5t^2}{4}\biggr]_{t=0}^{t=1}+\biggl[\frac{t^3}{6}+2t\biggr]_{t=1}^{t=2} \\
         & =\frac{53}{12}.
    \end{align*}
    Also,
    \begin{align*}
        m(9)-m(7)
         & =\int_{7}^{9}\lambda(t)\odif{t}                               \\
         & =\int_{7}^{9}(9-t)(t-2)\odif{t}                               \\
         & =\int_{7}^{9}(-18+11t-t^2)\odif{t}                            \\
         & =\biggl[-18t+\frac{11t^2}{t}-\frac{t^3}{3}\biggr]_{t=7}^{t=9} \\
         & =\frac{34}{3}.
    \end{align*}
    As a result, it follows that
    \begin{align*}
        \Prob[\big]{N(2)=4}
         & =e^{-\bigl(m(2)-m(0)\bigr)}\frac{\bigl(m(2)-m(0)\bigr)^4}{4!} \\
         & =\frac{e^{-53/12}(53/12)^4}{4!},
    \end{align*}
    and
    \begin{align*}
        \Prob[\big]{N(9)-N(7)=10}
         & =e^{-\bigl(m(9)-m(7)\bigr)}\frac{\bigl(m(9)-m(7)\bigr)^{10}}{10!} \\
         & =\frac{e^{-34/3}(34/3)^{10}}{10!}.
    \end{align*}
    Finally,
    \begin{align*}
        \Prob[\big]{N(2)=4,N(9)-N(7)=10}
         & =\Prob[\big]{N(2)=4}\Prob[big]{N(9)-N(7)=10}                          \\
         & =\frac{e^{-53/12}(53/12)^4}{4!}\cdot \frac{e^{-34/3}(34/3)^{10}}{10!} \\
         & \simeq 0.0221.
    \end{align*}
\end{Example}
\subsection*{The Compound Poisson Process}
Another restriction we might wish to relax concerns the assumption that arrivals in a Poisson
process occur strictly one after the other.

For instance, vehicles crossing the Canada-USA border would usually have more than just one
passenger on board. Individuals arriving to a sporting event often arrive in groups of two or
more.

There are also applications where each arrival might generate a random monetary amount.
For example, in an insurance company where claims occur according to a Poisson process, the
claim sizes themselves are (random) amounts of money, and one would typically be interested
in the total amount of money paid out by the insurance company by time $t$.

In any of the above scenarios, each arrival in a Poisson process comes with an associated
real-valued rv that represents the \emph{value} of the arrival in a sense.

This gives rise to the following definition.
\begin{Regular}
    \textbf{Definition}: Let $ \Set{Y_i}_{i=1}^\infty $ be an iid sequence of rvs. Let $ \Set[\big]{N(t),t\ge 0} $ be a
    Poisson process at rate $ \lambda $, independent of each $ Y_i $, $ i=1,2,3,\ldots $. If $ X(t)=\sum_{i=1}^{N(t)}Y_i $, then the
    process $ \Set[\big]{X(t),t\ge 0} $ is a \emph{compound} Poisson process.
    \tcblower{}
    \underline{Remarks}:
    \begin{enumerate}[(1)]
        \item The above definition is another way of generalizing the Poisson process, since if we
              choose the distribution of each $Y_i$ to be degenerate at $1$ (i.e., $Y_i = 1$ with probability $1$
              $ \forall i\in\mathbb{Z}^+ $), then $ \Set[\big]{X(t),t\ge 0} $ and $ \Set{N(t),t\ge 0} $ are identical processes.
        \item The compound Poisson process $ \Set[\big]{X(t),t\ge 0} $ inherits the independent and stationary increments
              properties from the Poisson processes $ \Set[\big]{N(t),t\ge 0} $. To see this formally,
              let $ (s_1,t_1] $ and $ (s_2,t_2] $ be time intervals such that $ t_1\le s_2 $ (resulting in $ (s_1,t_1] \cap (s_2,t_2]=\emptyset $).
                  Making use of the assumptions concerning $ \Set{Y_i}_{i=1}^\infty $ and $ \Set{N(t),t\ge 0} $, note that
              \begin{align*}
                   & \Prob[\big]{X(t_{1})-X(s_{1}) \leq a_{1}, X(t_{2})-X(s_{2}) \leq a_{2}}                                                                              \\
                   & =\Prob[\bigg]{\sum_{i=1}^{N(t_{1})} Y_{i}-\sum_{i=1}^{N(s_{1})}Y_{i} \leq a_{1}, \sum_{i=1}^{N(t_{2})} Y_{i}-\sum_{i=1}^{N(s_{2})} Y_{i} \leq a_{2}} \\
                   & =\Prob[\bigg]{\sum_{i=N(s_{1})+1}^{N(t_{1})} Y_{i}\leq a_{1}, \sum_{i=N(s_{2})+1}^{N(t_{2})} Y_{i} \leq a_{2}}                                       \\
                   & =\sum_{m_{1}=0}^{\infty} \sum_{n_{1}=0}^{\infty} \sum_{m_{2}=0}^{\infty} \sum_{n_{2}=0}^{\infty}                                                     \\
                   & \phantom{{}={}} \Prob[\bigg]{\sum_{i=N(s_{1})+1}^{N(t_{1})} Y_{i} \leq a_{1}, \sum_{i=N(s_{2})+1}^{N(t_{2})} Y_{i} \leq a_{2} \given
                  \begin{array}{l}
                          N(s_{1})=m_{1}, N(t_{1})-N(s_{1})=n_{1}, \\
                          N(s_2)-N(t_1)=m_2,N(t_2)-N(s_2)=n_2
                      \end{array}}                                                                                                                \\
                   & \phantom{{}={}}\times
                  \Prob[\big]{N(s_{1})=m_{1}, N(t_{1})-N(s_{1})=n_{1}, N(s_{2})-N(t_{1})=m_{2}, N(t_{2})-N(s_{2})=n_{2}}                                                  \\
                   & =\sum_{m_{1}=0}^{\infty} \sum_{n_{1}=0}^{\infty} \sum_{m_{2}=0}^{\infty} \sum_{n_{2}=0}^{\infty}
                  \Prob[\bigg]{\sum_{i=m_1+1}^{m_1+n_1}Y_i\le a_1,\sum_{i=m_1+n_1+m_2+1}^{m_1+n_1+m_2+n_2}Y_i\le a_2}                                                     \\
                   & \phantom{{}={}}\times \Prob[\big]{N(s_1)=m_1}\Prob[\big]{N(t_1)-N(s_1)=n_1}\Prob[\big]{N(s_2)-N(t_1)=m_2}\Prob[\big]{N(t_2)-N(s_2)=n_2}              \\
                   & =\Set*{\sum_{n_1=0}^{\infty}\Prob{\sum_{i=1}^{n_1}Y_i\le a_1}\Prob[\big]{N(t_1-s_1)=n_1}}
                  \Set*{\sum_{n_2=0}^{\infty}\Prob{\sum_{i=1}^{n_2}Y_i\le a_2}\Prob[\big]{N(t_2-s_2)=n_2}}                                                                \\
                   & =\Prob[\bigg]{\sum_{i=1}^{N(t_1-s_1)}Y_i\le a_1}\Prob[\bigg]{\sum_{i=1}^{N(t_2-s_2)}Y_i\le a_2}                                                      \\
                   & =\Prob[\big]{X(t_1-s_1)\le a_1}\Prob[\big]{X(t_2-s_2)\le a_2}.
              \end{align*}
        \item For $t > 0$, determining the probability distribution of $X(t)$ is, generally speaking, a
              challenging mathematical problem. On the other hand, the mean and variance of $X(t)$
              are readily accessible. In particular, by making use of the results for the mean and
              variance of a random sum (see Example 2.9), we immediately obtain
              \[ \E[\big]{X(t)}=\E[\big]{N(t)}\E{Y_1}=\lambda t\E{Y_1} \]
              and
              \begin{align*}
                  \Var[\big]{X(t)}
                   & =\Var{Y_1}\E[\big]{N(t)}+\E{Y_1}^2\Var[\big]{N(t)} \\
                   & =\lambda t\bigl(\Var{Y_1}-\E{Y_1}^2\bigr)          \\
                   & =\lambda t\E{Y_1^2}.
              \end{align*}
    \end{enumerate}
\end{Regular}
\begin{Example}
    \textbf{Example 4.8}. Claims received by an insurance company occur according to a Poisson process
    at a rate of $30$ claims per year. Individual claim amounts, which are assumed to be
    independent, are known to be either $\$1000$, $\$2000$, or $\$3000$. Company records indicate that
        one year ago, the average total amount paid out was $\$56000$ and that the standard deviation
        of the total amount paid out was $\$11000$. Based on this information, how likely was it that an
        individual claim of $\$3000$ took place last year?
        \tcblower{}
        \textbf{Solution}: Let $ N(t) $ represent the number of claims received by the company
        by time $ t $ (measured in years), which is known to have a $ \POI{30t} $
        distribution in the past year. In addition, let $ Y_i $, $ i\in\mathbb{Z}^+ $,
        be the size of the $ i\textsuperscript{th} $ individual claim amount (measured in thousands of dollars),
        having pmf of the form
        \[ \Prob{Y_i=y}=\begin{cases}
                \alpha,         & \text{if $y=1$}, \\
                \beta,          & \text{if $y=2$}, \\
                1-\alpha-\beta, & \text{if $y=3$}.
            \end{cases} \]
        Therefore, $ X(t)=\sum_{i=1}^{N(t)}Y_i $ represents the total amount paid out (measured in thousands of dollars) by time $ t $.
        We have:
        \[ \E[\big]{X(1)}=30\E{Y_1}=56\implies \E{Y_1}=\frac{56}{30}=\frac{28}{15}. \]
        That is,
        \[ \alpha+2\beta+3(1-\alpha-\beta)=\frac{28}{15}\implies 30\alpha+15\beta=17.\label{eq4.14}\tag*{(4.14)} \]
        Furthermore,
        \[ \Var[\big]{X(1)}=(30)\E{Y_1^2}=11^2\implies \E{Y_1^2}=\frac{121}{30}. \]
        That is,
        \[ \alpha+4\beta+9(1-\alpha-\beta)=\frac{121}{30}\implies 240\alpha+150\beta=149.\label{eq4.15}\tag*{(4.15)} \]
        The above equations~\ref{eq4.14} and~\ref{eq4.15} are linear, and can be solved to find $ \alpha $ and $ \beta $ as follows:
    \[ 10\times\ref{eq4.14}-\ref{eq4.15}\implies 60\alpha=21\implies \alpha=\frac{7}{20}.\label{eq4.16}\tag*{(4.16)} \]
    Plugging~\ref{eq4.16} into~\ref{eq4.14} leads to
    \[ 30\times \frac{7}{20}+15\beta=7\implies \beta=\frac{13}{30}. \]
    From above, the desired probability is simply
    \[ 1-\alpha-\beta=1-\frac{7}{20}-\frac{13}{30}=\frac{13}{60}\simeq 0.217. \]
\end{Example}