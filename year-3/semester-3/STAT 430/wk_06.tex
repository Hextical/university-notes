\makeheading{Week 6}
\section{Balanced Incomplete Block Designs}
\begin{itemize}
    \item Randomized Complete Block Designs (RCBD) were a tool for the exploration of \emph{one} design factor ($ m $ levels)
          while controlling for the effect of \emph{one} nuisance factor ($ b $ blocks).
          \begin{itemize}
              \item In a RCBD, we carry out \emph{every} experimental condition inside \emph{every} block.
              \item But sometimes, due to practical constraints, this is not possible.
          \end{itemize}
    \item The Gap is experimenting with $ m=3 $ promotional offers:
          \begin{itemize}
              \item Version 1: $ 50\% $ discount on one item.
              \item Version 2: $ 20\% $ discount on your entire order.
              \item Version 3: Spend $ \$ 50 $ and get a $ \$ 10 $ gift card.
          \end{itemize}
    \item Experimenters would like to control for a possible day-of-week effect (block by day).
          \begin{itemize}
              \item Naturally, one might consider a RCBD\@; that is, suppose we observe data in \emph{every} block-condition combination
                    as seen in~\Cref{rcbd_GAP}.
                    \begin{table}[!htbp]
                        \centering
                        \caption{Complete Block Design}\label{rcbd_GAP}
                        \begin{NiceTabular}{|cc|cccccc|}
                            \toprule         &   & \multicolumn{6}{c} {\emph{Day}}                                                                            \\
                            &   & 1                               & 2            & 3            & 4            & 5            & 6            \\
                            \midrule         & 1 & $\checkmark$ & $\checkmark$ & $\checkmark$ & $\checkmark$ & $\checkmark$ & $\checkmark$     \\
                            \emph{Promotion} & 2 & $\checkmark$ & $\checkmark$ & $\checkmark$ & $\checkmark$ & $\checkmark$ & $\checkmark$ \\
                            & 3 & $\checkmark$ & $\checkmark$ & $\checkmark$ & $\checkmark$ & $\checkmark$ & $\checkmark$ \\
                            \bottomrule
                        \end{NiceTabular}
                    \end{table}
          \end{itemize}
    \item But The Gap may only offer two of the three promotions in a single day.
          \begin{itemize}
              \item So we must consider an \textbf{incomplete} block design; that is, suppose we observe data in only \emph{some} block-condition combinations.
          \end{itemize}
          \begin{table}[!htbp]
              \centering
              \caption{Incomplete Block Design}\label{bibd_GAP}
              \begin{NiceTabular}{|cc|cccccc|}
                  \toprule         &   & \multicolumn{6}{c} {\emph{Day}}                                                                            \\
                  &   & 1                               & 2            & 3            & 4            & 5            & 6            \\
                  \midrule         & 1 & $\checkmark$                    & $\checkmark$ & $\times$     & $\checkmark$ & $\checkmark$ & $\times$     \\
                  \emph{Promotion} & 2 & $\checkmark$                    & $\times$     & $\checkmark$ & $\checkmark$ & $\times$     & $\checkmark$ \\
                  & 3 & $\times$                        & $\checkmark$ & $\checkmark$ & $\times$     & $\checkmark$ & $\checkmark$ \\
                  \bottomrule
              \end{NiceTabular}
          \end{table}
    \item We refer to~\Cref{bibd_GAP} as a \textbf{balanced incomplete block design} (BIBD).
          \begin{Remark}{Notation}{}
              \begin{itemize}
                  \item $ m $: number of experimental conditions. In our previous example, $ m=3 $.
                  \item $ b $: number of blocks. In our previous example, $ b=6 $.
                  \item $ m^\star $: number of experimental conditions that can be run in each block. Also known as ``block size.'' In our previous example, $ m^\star=2 $.
                        \begin{itemize}
                            \item RCBD\@: $ m^\star=m $.
                            \item BIBD\@: $ m^\star<m $.
                        \end{itemize}
                  \item $ r $: number of blocks in which each condition appears. In our previous example, $ r=4 $.
                  \item $ \lambda $: number of blocks that \emph{any} pair of conditions appear in together. In our previous example, $ \lambda=2 $.
              \end{itemize}
          \end{Remark}
    \item The BIBD is ``balanced'' in the sense that:
          \begin{itemize}
              \item The number of conditions in each block is the same for every block ($ m^\star $).
              \item The number of blocks each condition appears in is the same for every condition ($ r $).
              \item The number of blocks each pair of conditions appear in together is the same for every possible
                    condition pairing ($ \lambda $).
          \end{itemize}
    \item This balance allows for the comparison of a metric of interest across $m$ conditions while still accounting
          for a nuisance factor with $b$ levels
          \begin{itemize}
              \item But despite this balance, the ``incompleteness'' requires some sacrifice.
          \end{itemize}
\end{itemize}
\subsection{General Comments on the Design of a BIBD}
\begin{itemize}
    \item Not just any haphazard combination of $ (m,b,m^\star,r,\lambda) $ values will yield a BIBD\@.
    \item Great care must go into planning a BIBD to ensure all forms of balance.
    \item A variety of restrictions must be met:
          \begin{itemize}
              \item Consequences of ``incompleteness:''
                    \begin{itemize}
                        \item $ m^\star<m $.
                        \item $ r<b $.
                        \item $ \lambda<r $.
                    \end{itemize}
              \item Number of block-condition combinations for which we observe data:
                    \begin{itemize}
                        \item $ mr=bm^\star $
                    \end{itemize}
              \item For condition $ X $ (doesn't matter which condition this is), $ r(m^\star-1)=\lambda(m-1) $ is the
                    total number of conditions that condition $ X $ appears within the same blocks.
                    \begin{itemize}
                        \item Condition $ X $ appears in $ r $ blocks, and in each, it's grouped with $ m^\star-1 $ \emph{other} conditions.
                        \item We pair each of the other $ m-1 $ conditions with condition $ X $ $ \lambda $ times.
                    \end{itemize}
          \end{itemize}
    \item We use these restrictions as follows:
          \begin{enumerate}
              \item Specify $ m $, $ m^\star $, and $ \lambda $.
              \item Calculate $ r=\lambda(m-1)/(m^\star-1) $, noting that it must be an integer.
              \item Calculate $ b=mr/m^\star $, noting that it must be an integer.
          \end{enumerate}
          \begin{Example}{}{}
              Let $ m=3 $, $ m^\star=2 $, and $ \lambda=1 $. We have
              $ r=(1)(2)/(1) =2 $, and
              $ b=(3)(2)/2 =3 $. See~\Cref{bibd1}.
          \end{Example}
          \begin{Example}{Pizza Table}{}
              Let $ m=3 $, $ m^\star=2 $, and $ \lambda=2 $. We have
              $ r=(2)(2)/(1) =4 $, and
              $ b=(3)(4)/2 =6 $.
          \end{Example}
          \begin{Example}{}{}
              Let $ m=3 $, $ m^\star=2 $, and $ \lambda=3 $. We have
              $ r=(3)(2)/(1) =6 $, and
              $ b=(3)(6)/2 =9 $. See~\Cref{bibd2}.
          \end{Example}
    \item We select the design based on a trade-off between larger $ \lambda $ values and smaller $ b $ values.
          \begin{itemize}
              \item Larger $ \lambda $ provides more information about pairwise comparisons.
              \item Smaller $ b $ corresponds to fewer blocks and hence a smaller experiment.
          \end{itemize}
\end{itemize}
\begin{table}[!htbp]
    \centering
    \caption{Incomplete Block Design}\label{bibd1}
    \begin{NiceTabular}{|cc|ccc|}
        \toprule         &   & \multicolumn{3}{c} {\emph{Block}}                               \\
        &   & 1                                 & 2            & 3            \\
        \midrule         & 1 & $\checkmark$                      & $\checkmark$ & $\times$     \\
        \emph{Condition} & 2 & $\checkmark$                      & $\times$     & $\checkmark$ \\
        & 3 & $\times$                          & $\checkmark$ & $\checkmark$ \\
        \bottomrule
    \end{NiceTabular}
\end{table}
\begin{table}[!htbp]
    \centering
    \caption{Incomplete Block Design}\label{bibd2}
    \begin{NiceTabular}{|cc|ccccccccc|}
        \toprule         &   & \multicolumn{9}{c} {\emph{Block}}                                                                                                                         \\
        &   & 1                                 & 2            & 3            & 4            & 5            & 6            & 7            & 8            & 9            \\
        \midrule         & 1 & $\checkmark$                      & $\checkmark$ & $\checkmark$ & $\checkmark$ & $\checkmark$ & $\checkmark$ & $\times$     & $\times$     & $\times$     \\
        \emph{Condition} & 2 & $\checkmark$                      & $\checkmark$ & $\checkmark$ & $\times$     & $\times$     & $\times$     & $\checkmark$ & $\checkmark$ & $\checkmark$ \\
        & 3 & $\times$                          & $\times$     & $\times$     & $\checkmark$ & $\checkmark$ & $\checkmark$ & $\checkmark$ & $\checkmark$ & $\checkmark$ \\
        \bottomrule
    \end{NiceTabular}
\end{table}
\subsection{General Comments on the Analysis of a BIBD}
\begin{itemize}
    \item Primary analysis goal:
          \begin{itemize}
              \item Determine whether there exist significant differences among the expected response values from
                    one experimental condition to another.
          \end{itemize}
    \item In a RCBD, we do this by comparing the condition-specific means $ \bar{y}_{\bullet j\bullet} $ to the overall mean $ \bar{y}_{\bullet\bullet\bullet} $.
          This isn't fair in a BIBD because $ \bar{y}_{\bullet\bullet\bullet} $ is calculated from data from blocks that condition $ j $ didn't appear in.
    \item In a BIBD, due to incompleteness, we compare $ \bar{y}_{\bullet j\bullet} $ with the average response from the blocks that
          condition $j$ appeared in:
          \[ \frac{\sum_{k\in\mathcal{B}_j}\sum_{i=1}^{n_{jk}} y_{ijk}}{\sum_{k\in\mathcal{B}_j}n_{jk} }  \]
          where $ \mathcal{B}_j\subset\set{1,2,\ldots,B} $ is the subset of indices indicating which blocks condition $ j $ appeared in.
    \item In general, the analysis of BIBDs involves an \emph{adjustment} of this form when evaluating the effect of the
          design factor.
\end{itemize}
\section{Latin Square Designs}
\begin{itemize}
    \item Until now, we have discussed experimental designs that employ blocking to control for \emph{one} nuisance
          factor:
          \begin{itemize}
              \item If we want to control for \emph{two} nuisance factors, we should use a \textbf{Latin square design}.
              \item If we want to control for \emph{three} nuisance factors, we should use a \textbf{Graeco-Latin square design}.
              \item If we want to control for \emph{four} nuisance factors, we should use a \textbf{Hyper-Graeco-Latin square design}.
          \end{itemize}
    \item A Latin square of order $ p $ is a $ p\times p $ grid containing $ p $ unique symbols.
          \begin{itemize}
              \item Each of these symbols occurs exactly once in each column.
              \item Each of these symbols occurs exactly once in each row.
              \item These ``symbols'' are typically denoted by Latin letters.
          \end{itemize}
          \begin{table}[!htbp]
              \centering
              \caption{$ 3\times 3 $, $ 4\times 4 $, and $ 5\times 5 $ Latin Square Examples}
              \begin{NiceTabular}{|ccc|}
                  \toprule
                  A & C & B \\
                  C & B & A \\
                  B & A & C\\
                  \bottomrule
              \end{NiceTabular}\quad
              \begin{NiceTabular}{|cccc|}
                  \toprule
                  A & B & C & D\\
                  C & D & A & B\\
                  B & C & D & A\\
                  D & A & B & C\\
                  \bottomrule
              \end{NiceTabular}\quad
              \begin{NiceTabular}{|ccccc|}
                  \toprule
                  A & B & C & D & E\\
                  E & A & B & C & D\\
                  D & E & A & B & C\\
                  C & D & E & A & B\\
                  B & C & D & E & A\\
                  \bottomrule
              \end{NiceTabular}
          \end{table}
    \item A Sudoku puzzle is a special example of a $ 9\times 9 $ Latin square.
    \item We exploit this combinatorial structure to help us design experiments that facilitate blocking by two
          nuisance factors.
          \begin{itemize}
              \item We arbitrarily associate the $p$ rows with the levels of the first nuisance factor.
              \item We arbitrarily associate the $p$ columns with the levels of the second nuisance factor.
              \item We arbitrarily associate the $p$ Latin letters with the levels of the design factor.
          \end{itemize}
    \item We present an example with $ p=4 $ in~\Cref{latinsquarex1}.
          \begin{table}[!htbp]
              \centering
              \caption{$ 4\times 4 $ Latin Square Design}\label{latinsquarex1}
              \begin{NiceTabular}{|cc|cccc|}
                  \toprule            &   & \multicolumn{4}{c} {\emph{NF 2}}             \\
                  &   & 1                                          & 2 & 3 & 4 \\
                  \midrule            & 1 & A                                          & B & C & D \\
                  \multirow{2}{*}{\emph{NF 1}} & 2 & D                                          & A & B & C \\
                  & 3 & C                                          & D & A & B \\
                  & 4 & B                                          & C & D & A \\
                  \bottomrule
              \end{NiceTabular}
          \end{table}
          \begin{itemize}
              \item \textbf{Limitation of LSD's}: we need to experiment with \emph{all} of these factors at $ p $ levels.
              \item $ (3,2) $ element represents the block where NF 1 is at level 3, NF 2 is at level 2, and DF is at level D.
          \end{itemize}
    \item Each cell in this table represents a ``block'' in which we fix the nuisance factors' levels, and the
          Latin letter indicates the execution of an experimental condition.
    \item Rows, columns, and letters are all orthogonal, allowing us to separately estimate the effects of the
          design factor and each of the two nuisance factors.
    \item We may informally summarize these effects with the overall average and level-specific averages of the
          response variables.
          \begin{itemize}
              \item Average response in a particular condition:
                    \[ \bar{y}_{\bullet j\bullet\bullet}=\frac{1}{np}\sum_{(j,k,\ell)\in \mathcal{S}_j}\sum_{i=1}^{n} y_{ijk\ell}  \]
              \item Average response in a given row:
                    \[ \bar{y}_{\bullet\bullet k\bullet}=\frac{1}{np}\sum_{(j,k,\ell)\in \mathcal{S}_k}\sum_{i=1}^{n} y_{ijk\ell}  \]
              \item Average response in a given column:
                    \[ \bar{y}_{\bullet\bullet\bullet\ell}=\frac{1}{np}\sum_{(j,k,\ell)\in \mathcal{S}_\ell}\sum_{i=1}^{n} y_{ijk\ell}  \]
              \item Overall average:
                    \[ \bar{y}_{\bullet\bullet\bullet\bullet}=\frac{1}{N}\sum_{(j,k,\ell)\in \mathcal{S}}\sum_{i=1}^{n} y_{ijk\ell}  \]
                    \begin{itemize}
                        \item $ y_{ijk\ell} $ is the response observation for unit $ i=1,2,\ldots,n $ in block $ (k,\ell) $ and hence condition $ j $.
                        \item $ j,k,\ell=1,2,\ldots,p $.
                        \item $ n $ is the number of units in each block.
                        \item $ N=np^2 $.
                    \end{itemize}
          \end{itemize}
    \item A comment about notation:
          \begin{itemize}
              \item Each block contains just one condition, so each pair $(k, \ell)$ uniquely determines the value of $j$.
              \item Consequently, there exist just $p^2$ tuples $(j, k, \ell)$.
              \item Denote them by the set $ \mathcal{S} $.
              \item From~\Cref{latinsquarex1}, we have:
                    \begin{table}[!htbp]
                        \centering
                        \begin{NiceTabular}{|cccc|}
                            \toprule
                            $ (1,1,1) $ & $ (2,1,2) $ & $ (3,1,3) $ & $ (4,1,4) $\\
                            $ (4,2,1) $ & $ (1,2,2) $ & $ (2,2,3) $ & $ (3,2,4) $\\
                            $ (3,3,1) $ & $ (4,3,2) $ & $ (1,3,3) $ & $ (2,3,4) $\\
                            $ (2,4,1) $ & $ (3,4,2) $ & $ (4,4,3) $ & $ (1,4,4) $\\
                            \bottomrule
                        \end{NiceTabular}
                    \end{table}
                    \begin{itemize}
                        \item $ S_{j=1}=\set{(1,1,1),(1,2,2),(1,3,3),(1,4,4)} $.
                        \item We also define:
                              \begin{itemize}
                                  \item $ \mathcal{S}_j\subset \mathcal{S} $: all tuples for which the design factor is level $ j $.
                                  \item $ \mathcal{S}_k\subset \mathcal{S} $: all tuples for which the nuisance factor $ 1 $'s is level $ k $.
                                  \item $ \mathcal{S}_\ell\subset \mathcal{S} $: all tuples for which the nuisance factor $ 2 $'s is level $ \ell $.
                              \end{itemize}
                    \end{itemize}
          \end{itemize}
    \item The primary analysis goal in a Latin square design is to determine whether the expected response differs significantly
          from one condition to another.
          \begin{itemize}
              \item If so, to identify the optimal condition while controlling for the potential effect of the nuisance factors.
          \end{itemize}
    \item We've previously done this with gatekeeper tests of the form:
          \begin{tightcenter}
              $ \mathbf{H}_0 $: $ \theta_1=\theta_2=\cdots=\theta_p $ versus $ \mathbf{H}_\text{A} $: $ \theta_j\ne \theta_{j^\prime} $ for some $ j\ne j^\prime $.
          \end{tightcenter}
    \item We do the same thing here, while accounting for the nuisance factors, with \emph{appropriately defined} linear
          or logistic regression models which contain:
          \begin{itemize}
              \item An intercept.
              \item $ p-1 $ indicator variables for the design factor's levels.
              \item $ p-1 $ indicator variables for nuisance factor 1's levels.
              \item $ p-1 $ indicator variables for nuisance factor 2's levels.
          \end{itemize}
    \item We write the linear predictor as:
          \[ \alpha+\sum_{j=1}^{p-1} \beta_j x_{ij}+\sum_{k=1}^{p-1} \gamma_j z_{ik}+\sum_{\ell=1}^{p-1} \delta_\ell w_{i\ell} \]
          \begin{itemize}
              \item $ x_{ij}=1 $ if unit $ i $ is in condition $ j=1,2,\ldots,p-1 $ (zero otherwise).
              \item $ z_{ik}=1 $ if unit $ i $ is in a block for which nuisance factor 1 is at level $ k=1,2,\ldots,p-1 $ (zero otherwise).
              \item $ w_{i\ell}=1 $ if unit $ i $ is in a block for which nuisance factor 2 is at level $ \ell=1,2,\ldots,p-1 $ (zero otherwise).
              \item The $\beta$'s jointly quantify the effect of the design factor.
              \item The $\gamma$'s jointly quantify the effect of nuisance factor 1.
              \item The $\delta$'s jointly quantify the effect of nuisance factor 2.
          \end{itemize}
    \item Three relevant hypotheses are:
          \begin{tightcenter}
              $ \mathbf{H}_0 $: $ \beta_1=\beta_2=\cdots=\beta_{p-1}=0 $ versus $ \mathbf{H}_\text{A} $: $ \beta_j\ne 0$ for some $ j $.
              \begin{itemize}
                  \item Provides insight into whether DF is important.
              \end{itemize}
              $ \mathbf{H}_0 $: $ \gamma_1=\beta_2=\cdots=\gamma_{p-1}=0 $ versus $ \mathbf{H}_\text{A} $: $ \gamma_k\ne 0$ for some $ k $.
              \begin{itemize}
                  \item Provides insight into whether NF 1 is important.
              \end{itemize}
              $ \mathbf{H}_0 $: $ \delta_1=\delta_2=\cdots=\delta_{p-1}=0 $ versus $ \mathbf{H}_\text{A} $: $ \delta_\ell\ne 0$ for some $ \ell $.
              \begin{itemize}
                  \item Provides insight into whether NF 2 is important.
              \end{itemize}
          \end{tightcenter}
    \item We test these hypotheses by comparing a \emph{full} (linear predictor) and \emph{reduced} ($ \mathbf{H}_0 $ is true) model.
          \begin{itemize}
              \item We try to determine whether the full model fits the data significantly better than the reduced one.
          \end{itemize}
\end{itemize}
\subsection{Latin Squares to Compare Means}
\begin{itemize}
    \item Here we're interested in testing the following hypothesis (while accounting for the influence of the
          nuisance factors):
          \begin{tightcenter}
              $ \mathbf{H}_0 $: $ \mu_1=\mu_2=\cdots=\mu_p $ versus $ \mathbf{H}_\text{A} $: $ \mu_j\ne \mu_{j^\prime} $ for some $ j\ne j^\prime $
          \end{tightcenter}
          where $ \mu_j $ is the expected response in condition $ j=1,2,\ldots,p $.
    \item We do this by testing:
          \begin{tightcenter}
              $ \mathbf{H}_0 $: $ \beta_1=\beta_2=\cdots=\beta_{p-1}=0 $ versus $ \mathbf{H}_\text{A} $: $ \beta_j\ne 0$ for some $ j $
          \end{tightcenter}
          with an ANOVA in the context of the following linear regression model:
          \[ Y_i=\alpha+\sum_{j=1}^{p-1} \beta_j x_{ij}+\sum_{k=1}^{p-1} \gamma_k z_{ik}+\sum_{\ell=1}^{p-1} \delta_\ell w_{i\ell}+\varepsilon_i\quad \text{(Full Model)} \]
          \begin{itemize}
              \item $ Y_i $ is the response observation for unit $ i=1,2,\ldots,N=np^2 $.
              \item $ \varepsilon_i \stackrel{\text{iid}}{\sim} \N{0,\sigma^2} $ is the random error term.
          \end{itemize}
    \item The relevant sums of squares are:
          \begin{itemize}
              \item The total sum of squares, which quantifies overall variation in response values:
                    \[ \text{SS}_\text{T}=\sum_{(j,k,\ell)\in \mathcal{S}}\sum_{i=1}^{n} (y_{ijk\ell}-\bar{y}_{\bullet\bullet\bullet\bullet})^2=\text{SS}_{\text{C}}+\text{SS}_{\text{B}_1}+\text{SS}_{\text{B}_2}+\text{SS}_{\text{E}}  \]
              \item The condition sum of squares, which quantifies variability in the response from one condition to another:
                    \[ \text{SS}_{\text{C}}=\sum_{(j,k,\ell)\in \mathcal{S}}\sum_{i=1}^{n} (\bar{y}_{\bullet j\bullet\bullet}-\bar{y}_{\bullet\bullet\bullet\bullet})^2=np \sum_{j=1}^{p} (\bar{y}_{\bullet j\bullet\bullet}-\bar{y}_{\bullet\bullet\bullet\bullet})^2  \]
              \item The first block sum of squares, which quantifies variability in the response from one level of nuisance factor 1 to another:
                    \[ \text{SS}_{\text{B}_1}=\sum_{(j,k,\ell)\in \mathcal{S}}\sum_{i=1}^{n} (\bar{y}_{\bullet\bullet k\bullet}-\bar{y}_{\bullet\bullet\bullet\bullet})^2=np \sum_{k=1}^{p} (\bar{y}_{\bullet\bullet k\bullet}-\bar{y}_{\bullet\bullet\bullet\bullet})^2  \]
              \item The second block sum of squares, which quantifies variability in the response from one level of nuisance factor 2 to another:
                    \[ \text{SS}_{\text{B}_2}=\sum_{(j,k,\ell)\in \mathcal{S}}\sum_{i=1}^{n} (\bar{y}_{\bullet\bullet\bullet\ell}-\bar{y}_{\bullet\bullet\bullet\bullet})^2=np \sum_{\ell=1}^{p} (\bar{y}_{\bullet\bullet\bullet\ell}-\bar{y}_{\bullet\bullet\bullet\bullet})^2  \]
              \item The error sum of squares, which quantifies variability in the response that was not explained by conditions or blocks (i.e., the design and nuisance factors):
                    \[ \text{SS}_\text{E}=\sum_{(j,k,\ell)\in \mathcal{S}}\sum_{i=1}^{n} (y_{ijk\ell}-\bar{y}_{\bullet j\bullet\bullet}-\bar{y}_{\bullet\bullet k\bullet}-\bar{y}_{\bullet\bullet\bullet\ell}-2\bar{y}_{\bullet\bullet\bullet\bullet})^2 \]
          \end{itemize}
    \item We show the corresponding ANOVA table in~\Cref{three_way_anova_LS}.
          \begin{table}[!htbp]
              \centering
              \caption{Three-Way ANOVA Table Associated with a Latin Square Design}\label{three_way_anova_LS}
              \begin{NiceTabular}{|l|c|c|c|c|}
                  \toprule
                  Source    & SS                     & d.f.        & MS                                                                   & Test Statistic                                           \\
                  \midrule
                  Design Factor & $ \text{SS}_\text{C} $ & $ p-1 $     & $ \text{MS}_\text{C}=\text{SS}_\text{C}/(p-1) $     & $ t_\text{C}=\text{MS}_\text{C}/\text{MS}_\text{E} $ \\
                  Nuisance Factor 1     & $ \text{SS}_{\text{B}_1} $ & $ p-1 $     & $ \text{MS}_{\text{B}_1}=\text{SS}_{\text{B}_1}/(p-1) $     & $ t_{\text{B}_1}=\text{MS}_{\text{B}_1}/\text{MS}_\text{E} $ \\
                  Nuisance Factor 2     & $ \text{SS}_{\text{B}_2} $ & $ p-1 $     & $ \text{MS}_{\text{B}_2}=\text{SS}_{\text{B}_2}/(p-1) $     & $ t_{\text{B}_2}=\text{MS}_{\text{B}_2}/\text{MS}_\text{E} $ \\
                  Error     & $ \text{SS}_\text{E} $ & $ N-3p+2 $ & $ \text{MS}_\text{E}=\text{SS}_\text{E}/(N-3p+2) $ &                                                      \\
                  \midrule
                  Total     & $ \text{SS}_\text{T} $ & $ N-1 $\\
                  \bottomrule
              \end{NiceTabular}
          \end{table}
    \item So, how do we use this table?
          \begin{itemize}
              \item We test $ \mathbf{H}_0 $: $ \beta_1=\beta_2=\cdots=\beta_{p-1}=0 $ using $ t_\text{C}=\text{MS}_\text{C}/\text{MS}_\text{E} $.
                    \begin{itemize}
                        \item $ p $-value: $ \Prob[\big]{T\ge t_\text{C}} $ where $ T \sim F(p-1,N-3p+2) $.
                    \end{itemize}
              \item We test: $ \mathbf{H}_0 $: $ \gamma_1=\gamma_2=\cdots=\gamma_{p-1}=0 $ using $ t_{\text{B}_1}=\text{MS}_{\text{B}_1}/\text{MS}_{\text{E}} $.
                    \begin{itemize}
                        \item $ p $-value: $ \Prob*{T\ge t_{\text{B}_1}} $ where $ T \sim F(p-1,N-3p+2) $.
                    \end{itemize}
              \item We test: $ \mathbf{H}_0 $: $ \delta_1=\delta_2=\cdots=\delta_{p-1}=0 $ using $ t_{\text{B}_2}=\text{MS}_{\text{B}_2}/\text{MS}_{\text{E}} $.
                    \begin{itemize}
                        \item $ p $-value: $ \Prob*{T\ge t_{\text{B}_2}} $ where $ T \sim F(p-1,N-3p+2) $.
                    \end{itemize}
          \end{itemize}
\end{itemize}
\subsection{Example: Netflix Latency}
\begin{Example}{Netflix Latency}{}
    Consider the latency experiment described at the beginning of the chapter in which Netflix is experimenting
    with server-side modifications to improve (reduce) the latency of \href{netflix.com}{netflix.com}. In particular, they have four
    different experimental conditions (A, B, C, D) that are intended to reduce average latency (in milliseconds).
    Two nuisance factors that may also influence latency are browser (Chrome, Microsoft Edge, Firefox,
    Safari), and time of day ([00:01,06:00], [06:01,12:00], [12:01,18:00], [18:01,00:00]). The design of the experiment
    is the $ 4\times 4 $ Latin square shown in~\Cref{netflixex1}. In order to determine whether the expected latency in each condition
    differs significantly, we randomize $n = 500$ users to each of the $p^2 = 16$ blocks.
    \begin{center}
        \captionsetup{type=table}\captionof{table}{$ 4\times 4 $ Latin Square Design for the Netflix Experiment}\label{netflixex1}
        \begin{NiceTabular}{|cc|cccc|}
            \toprule            &   & \multicolumn{4}{c} {\emph{Browser}}             \\
            &   & Chrome                                          & Edge & Firefox & Safari \\
            \midrule            & [00:01,06:00] & A                                          & B & C & D \\
            \multirow{2}{*}{\emph{Time}} & [06:01,12:00] & D                                          & A & B & C \\
            & [12:01,18:00] & C                                          & D & A & B \\
            & [18:01,00:00] & B                                          & C & D & A \\
            \bottomrule
        \end{NiceTabular}
    \end{center}
    We analyze the data with the following linear regression model:
    \[ Y_i=\alpha+\beta_2 x_{i2}+\beta_3 x_{i3}+\beta_4 x_{i4}+\gamma_1 z_{i1}+\gamma_2 z_{i2}+\gamma_3 z_{i3}+\delta_2 w_{i2}+\delta_3 w_{i3}+\delta_4 w_{i4}+\varepsilon_i \]
    \begin{itemize}
        \item $ x_{i2} $, $ x_{i3} $, $ x_{i4} $ are indicators for conditions B, C, D, where A is the baseline.
        \item $ z_{i1} $, $ z_{i2} $, $ z_{i3} $ are browser indicators for Microsoft Edge, Firefox, Safari, where Chrome is the baseline.
        \item $ w_{i2} $, $ w_{i3} $, $ w_{i4} $ are time indicators for time periods [06:01,12:00], [12:01,18:00], [18:01,00:00], where [00:01,06:00] is the baseline.
    \end{itemize}
    The ANOVA table associated with this model is~\Cref{netflixex2}.
    \begin{center}
        \captionsetup{type=table}\captionof{table}{Netflix Latin Square ANOVA Table}\label{netflixex2}
        \begin{NiceTabular}{|l|c|c|c|c|}
            \toprule
            Source    & SS                     & d.f.        & MS                                                                   & Test Statistic                                           \\
            \midrule
            Condition & $ 203903.38 $ & $ 3 $     & $ 67967.79 $     & $ 679.14 $ \\
            Browser     & $ 32.95 $ & $ 3 $     & $ 10.98 $     & $ 0.1097 $ \\
            Time     & $ 333242.01 $ & $ 3 $     & $ 111080.67 $     & $ 1109.92 $ \\
            Error     & $ 799636.18 $ & $ 7990 $ & $ 100.08 $ &                                                      \\
            \midrule
            Total     & $ 1336815 $ & $ 7999 $\\
            \bottomrule
        \end{NiceTabular}
    \end{center}
    \begin{itemize}
        \item $ \mathbf{H}_0 $: $ \beta_2=\beta_3=\beta_4=0 $.
              \begin{itemize}
                  \item $ p\text{-value}=\Prob*{T\ge t_\text{C}}=\Prob*{T\ge 679.14}\approx 0 $ where $ T \sim F(3,7990) $.
                  \item Therefore, we reject $ \mathbf{H}_0 $ and conclude that the design factor significantly influences the response
                        and hence the expected response is \underline{not} the same in each condition.
              \end{itemize}
        \item $ \mathbf{H}_0 $: $ \gamma_1=\gamma_2=\gamma_3=0 $.
              \begin{itemize}
                  \item $ p\text{-value}=\Prob*{T\ge t_{\text{B}_1}}=\Prob*{T\ge 0.1097}=0.9545 $ where $ T \sim F(3,7990) $.
                  \item Therefore, we do not reject $ \mathbf{H}_0 $ and conclude that ``browser'' does not
                        significantly influence average latency, and so blocking by browser was probably not necessary.
              \end{itemize}
        \item $ \mathbf{H}_0 $: $ \delta_2=\delta_3=\delta_4=0 $.
              \begin{itemize}
                  \item $ p\text{-value}=\Prob*{T\ge t_{\text{B}_2}}=\Prob*{T\ge 1109.92}\approx 0 $ where $ T \sim F(3,7990) $.
                  \item Therefore, we reject $ \mathbf{H}_0 $ and conclude that the time of day significantly influences average latency
                        and so blocking by it was sensible.
              \end{itemize}
    \end{itemize}
    \href{https://github.com/Hextical/university-notes/blob/master/year-3/semester-3/STAT 430/code/W6/Latin_square_means.R}{[R Code] \texttt{Latin\_square\_means}}
\end{Example}
\subsection{Latin Squares to Compare Proportions}
\begin{itemize}
    \item Here we're interested in testing the following hypothesis (while accounting for the influence of the
          nuisance factors):
          \begin{tightcenter}
              $ \mathbf{H}_0 $: $ \pi_1=\pi_2=\cdots=\pi_p $ versus $ \mathbf{H}_\text{A} $: $ \pi_j\ne \pi_{j^\prime }$ for some $ j\ne j^\prime $
          \end{tightcenter}
          where $ \pi_j $ is the expected response in condition $ j=1,2,\ldots,p $.
    \item We do this by testing:
          \begin{tightcenter}
              $ \mathbf{H}_0 $: $ \beta_1=\beta_2=\cdots=\beta_{p-1}=0 $ versus $ \mathbf{H}_\text{A} $: $ \beta_j\ne 0$ for some $ j $
          \end{tightcenter}
          with a likelihood ratio test (LRT) in the context of the following logistic regression model:
          \[ \log*{\frac{\pi_i}{1-\pi_i}}=\alpha+\sum_{j=1}^{p-1} \beta_j x_{ij}+\sum_{k=1}^{p-1} \gamma_k z_{ik}+\sum_{\ell=1}^{p-1} \delta_\ell w_{i\ell} \]
          \begin{itemize}
              \item $ Y_i=1 $ if unit $ i $ performs some action of interest, and $ Y_i=0 $ otherwise.
              \item $ \pi_i=\E{Y_i}=\text{expected response of unit $i$} $.
              \item The likelihood ratio test compares the full model to the one without the $ x $'s.
          \end{itemize}
    \item Similarly, we test:
          \begin{tightcenter}
              $ \mathbf{H}_0 $: $ \gamma_1=\gamma_2=\cdots=\gamma_{p-1}=0 $ versus $ \mathbf{H}_\text{A} $: $ \gamma_k\ne 0$ for some $ k $
          \end{tightcenter}
          with a LRT that compares the full model to the reduced one without the $ z $'s.
    \item And we test:
          \begin{tightcenter}
              $ \mathbf{H}_0 $: $ \delta_1=\delta_2=\cdots=\delta_{p-1}=0 $ versus $ \mathbf{H}_\text{A} $: $ \delta_\ell\ne 0$ for some $ \ell $
          \end{tightcenter}
          with a LRT that compares the full model to the reduced one without the $ w $'s.
    \item The observed test statistic for all of these tests is:
          \begin{align*}
              t & =2\,\log*{\frac{\text{Likelihood}_{\text{Full Model}}}{\text{Likelihood}_{\text{Reduced Model}}}}    \\
                & =2\Bigl[\text{Log-Likelihood}_{\text{Full Model}}-\text{Log-Likelihood}_{\text{Reduced Model}}\Bigr]
          \end{align*}
          which, if $ \mathbf{H}_0 $ is true, follows an approximate $ \chi^2(p-1) $.
    \item $ p\text{-value}=\Prob{T\ge t} $ where $ T \sim \chi^2(p-1) $.
\end{itemize}
\subsection{Example: Uber Weekend Promos}
\begin{Example}{Uber Weekend Promos}{}
    Consider an experiment in which Uber is investigating the influence of three different promotional offers on
    ride-booking-rate (RBR).
    \begin{itemize}
        \item Promo A\@: None.
        \item Promo B\@: One free ride today.
        \item Promo C\@: Book a ride today and get 50\% off your next 2 rides.
    \end{itemize}
    The experimenters would like to control for a possible day-of-week effect, and so they want to block by day.
    They would also like to control for possible city-to-city differences, and so they also want to block by city.
    To do so they run a $3\times 3$ Latin square design as illustrated in~\Cref{uberex1}. Interest lies in determining whether
    the different promotions perform similarly with respect to RBR --- and they wish to determine which
    one maximizes RBR --- while controlling for the effects of day and city. In order to do this they randomize
    $n = 1000$ users to each of the $p^2 = 9$ blocks.
    \begin{center}
        \captionsetup{type=table}
        \captionof{table}{$ 3\times 3 $ Latin Square Design for the Uber Experiment}\label{uberex1}
        \begin{NiceTabular}{|cc|ccc|}
            \toprule            &   & \multicolumn{3}{c} {\emph{City}}             \\
            &   & Toronto                                          & Vancouver & Montreal \\
            \midrule            & Friday & A                                          & B & C \\
            \emph{Day} & Saturday & C                                          & A & B \\
            & Sunday & B                                          & C & A \\
            \bottomrule
        \end{NiceTabular}
    \end{center}
    We analyze the data with the following logistic regression model:
    \[ \log*{\frac{\pi_i}{1-\pi_i}}=\alpha+\beta_2 x_{i2}+\beta_3 x_{i3}+\gamma_1 z_{i1}+\gamma_2 z_{i2}+\delta_1 w_{i1}+\delta_2 w_{i2} \]
    \begin{itemize}
        \item $ x_{i2} $, $ x_{i3} $, are condition indicators for promotions B, C, where A is the baseline.
        \item $ z_{i1} $, $ z_{i2} $, are day indicators for Saturday, Sunday, where Friday is the baseline.
        \item $ w_{i1} $, $ w_{i2} $, are city indicators for Toronto, Vancouver, where Montreal is the baseline.
    \end{itemize}
    \begin{itemize}
        \item $ \mathbf{H}_0 $: $ \beta_2=\beta_3=0 $.
              \begin{itemize}
                  \item $ p\text{-value}=\Prob*{T\ge t_\text{C}}=\Prob*{T\ge 16.648}=0.00024 $ where $ T \sim \chi^2(2) $.
                  \item Therefore, we reject $ \mathbf{H}_0 $ and conclude that the booking rate is \underline{not} the same
                        for each promotional offer.
              \end{itemize}
        \item $ \mathbf{H}_0 $: $ \gamma_1=\gamma_2=0 $.
              \begin{itemize}
                  \item $ p\text{-value}=\Prob*{T\ge t_{\text{B}_1}}=\Prob*{T\ge 8.9107}=0.01162 $ where $ T \sim \chi^2(2) $.
                  \item Therefore, we reject $ \mathbf{H}_0 $ and conclude that the day-of-week significantly influences booking, and so it is good that we blocked by this factor.
              \end{itemize}
        \item $ \mathbf{H}_0 $: $ \delta_1=\delta_2=0 $.
              \begin{itemize}
                  \item $ p\text{-value}=\Prob*{T\ge t_{\text{B}_2}}=\Prob*{T\ge 2.1193}=0.3466 $ where $ T \sim \chi^2(2) $.
                  \item Therefore, we do not reject $ \mathbf{H}_0 $ and conclude that ``city'' does not significantly influence booking rate, and so blocking by city may have not been necessary.
              \end{itemize}
    \end{itemize}
    \href{https://github.com/Hextical/university-notes/blob/master/year-3/semester-3/STAT 430/code/W6/Latin_square_proportions.R}{[R Code] \texttt{Latin\_square\_proportions}}
\end{Example}
