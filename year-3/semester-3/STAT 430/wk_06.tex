\makeheading{Week 6}
\section{Balanced Incomplete Block Designs}
\begin{itemize}
    \item Randomized Complete Block Designs (RCBD) were a tool for the exploration of \emph{one} design factor ($ m $ levels)
          while controlling for the effect of \emph{one} nuisance factor ($ b $ blocks).
          \begin{itemize}
              \item In a RCBD we carry out \emph{every} experimental condition inside \emph{every} block.
              \item But sometimes, due to practical constraints, this is not possible.
          \end{itemize}
    \item The Gap is experimenting with $ m=3 $ promotional offers:
          \begin{itemize}
              \item Version 1: $ 50\% $ discount on one item.
              \item Version 2: $ 20\% $ discount on your entire order.
              \item Version 3: Spend $ \$ 50 $ and get a $ \$ 10 $ gift card.
          \end{itemize}
    \item Experimenters would like to control for a possible day-of-week effect (block by day).
          \begin{itemize}
              \item Naturally, one might consider a RCBD\@. Suppose we observe data in \emph{every} block-condition combination.
                    \begin{table}[!htbp]
                        \centering
                        \caption{Complete Block Design}
                        \begin{NiceTabular}{|cc|cccccc|}
                            \toprule         &   & \multicolumn{6}{c} {\emph{Day}}                                                                            \\
                            &   & 1                               & 2            & 3            & 4            & 5            & 6            \\
                            \midrule         & 1 & $\checkmark$ & $\checkmark$ & $\checkmark$ & $\checkmark$ & $\checkmark$ & $\checkmark$     \\
                            \emph{Promotion} & 2 & $\checkmark$ & $\checkmark$ & $\checkmark$ & $\checkmark$ & $\checkmark$ & $\checkmark$ \\
                            & 3 & $\checkmark$ & $\checkmark$ & $\checkmark$ & $\checkmark$ & $\checkmark$ & $\checkmark$ \\
                            \bottomrule
                        \end{NiceTabular}
                    \end{table}
          \end{itemize}
    \item But the experiments may only offer two of the three promotions in a single day.
          \begin{itemize}
              \item So we must consider an \textbf{incomplete} block design. Suppose we observe data in only \emph{some} block-condition combinations.
          \end{itemize}
          \begin{table}[!htbp]
              \centering
              \caption{Incomplete Block Design}
              \begin{NiceTabular}{|cc|cccccc|}
                  \toprule         &   & \multicolumn{6}{c} {\emph{Day}}                                                                            \\
                  &   & 1                               & 2            & 3            & 4            & 5            & 6            \\
                  \midrule         & 1 & $\checkmark$                    & $\checkmark$ & $\times$     & $\checkmark$ & $\checkmark$ & $\times$     \\
                  \emph{Promotion} & 2 & $\checkmark$                    & $\times$     & $\checkmark$ & $\checkmark$ & $\times$     & $\checkmark$ \\
                  & 3 & $\times$                        & $\checkmark$ & $\checkmark$ & $\times$     & $\checkmark$ & $\checkmark$ \\
                  \bottomrule
              \end{NiceTabular}
          \end{table}
    \item We refer to the design above as a \textbf{balanced incomplete block design} (BIBD).
          \begin{Remark}{Notation}{}
              \begin{itemize}
                  \item $ m $: number of experimental conditions. In our previous example, $ m=3 $.
                  \item $ b $: number of blocks. In our previous example, $ b=6 $.
                  \item $ m^\star $: number of experimental conditions that can be run in each block. Also known as ``block size.'' In our previous example, $ m^\star=2 $.
                  \item $ r $: number of blocks in which each condition appears. In our previous example, $ r=4 $.
                  \item $ \lambda $: number of blocks that \emph{any} pair of conditions appear in together. In our previous example, $ \lambda=2 $.
              \end{itemize}
          \end{Remark}
    \item The BIBD is ``balanced'' in the sense that:
          \begin{itemize}
              \item The number of conditions in each block is the same for every block ($ m^\star $).
              \item The number of blocks each condition appears in is the same for every condition ($ r $).
              \item The number of blocks each pair of conditions appear in together is the same for every possible
                    condition pairing ($ \lambda $).
          \end{itemize}
    \item This balance allows for the comparison of a metric of interest across m conditions while still accounting
          for a nuisance factor with $b$ levels
          \begin{itemize}
              \item But despite this balance, the ``incompleteness'' requires some sacrifice.
          \end{itemize}
\end{itemize}
\subsection{General Comments on the Design of a BIBD}
\begin{itemize}
    \item Not just any haphazard combination of $ (m,b,m^\star,r,\lambda) $ values will yield a BIBD\@.
    \item Great care must go into planning a BIBD to ensure all forms of balance.
    \item A variety of restrictions must be met:
          \begin{itemize}
              \item Consequences of ``incompleteness:''
                    \[ m^\star<m \]
                    \[ r<b \]
                    \[ \lambda<r \]
          \end{itemize}
          \[ mr=bm^\star \]
          \[ r(m^\star-1)=\lambda(m-1) \]
    \item We use these restrictions as follows:
          \begin{enumerate}
              \item Specify $ m $, $ m^\star $, and $ \lambda $.
              \item Calculate $ r=\lambda(m-1)/(m^\star-1) $, noting that it must be an integer.
              \item Calculate $ b=mr/m^\star $, noting that it must be an integer.
          \end{enumerate}
          \begin{Example}{}{}
              Let $ m=3 $, $ m^\star=2 $, and $ \lambda=1 $. We have
              $ r=(1)(2)/(1) =2 $, and
              $ b=(3)(2)/2 =3 $.
          \end{Example}
          \begin{Example}{Pizza Table}{}
              Let $ m=3 $, $ m^\star=2 $, and $ \lambda=2 $. We have
              $ r=(2)(2)/(1) =4 $, and
              $ b=(3)(4)/2 =6 $.
          \end{Example}
          \begin{Example}{}{}
              Let $ m=3 $, $ m^\star=2 $, and $ \lambda=3 $. We have
              $ r=(3)(2)/(1) =6 $, and
              $ b=(3)(6)/2 =9 $.
          \end{Example}
    \item We select the design based on a trade-off between larger $ \lambda $ values and smaller $ b $ values.
          \begin{itemize}
              \item Larger $ \lambda $ provides more information for pairwise comparisons.
              \item Smaller $ b $ corresponds to fewer blocks and hence a smaller experiment.
          \end{itemize}
\end{itemize}
\begin{table}[!htbp]
    \centering
    \caption{Incomplete Block Design}
    \begin{NiceTabular}{|cc|ccc|}
        \toprule         &   & \multicolumn{3}{c} {\emph{Block}}                               \\
        &   & 1                                 & 2            & 3            \\
        \midrule         & 1 & $\checkmark$                      & $\checkmark$ & $\times$     \\
        \emph{Condition} & 2 & $\checkmark$                      & $\times$     & $\checkmark$ \\
        & 3 & $\times$                          & $\checkmark$ & $\checkmark$ \\
        \bottomrule
    \end{NiceTabular}
\end{table}
\begin{table}[!htbp]
    \centering
    \caption{Incomplete Block Design}
    \begin{NiceTabular}{|cc|ccccccccc|}
        \toprule         &   & \multicolumn{9}{c} {\emph{Block}}                                                                                                                         \\
        &   & 1                                 & 2            & 3            & 4            & 5            & 6            & 7            & 8            & 9            \\
        \midrule         & 1 & $\checkmark$                      & $\checkmark$ & $\checkmark$ & $\checkmark$ & $\checkmark$ & $\checkmark$ & $\times$     & $\times$     & $\times$     \\
        \emph{Condition} & 2 & $\checkmark$                      & $\checkmark$ & $\checkmark$ & $\times$     & $\times$     & $\times$     & $\checkmark$ & $\checkmark$ & $\checkmark$ \\
        & 3 & $\times$                          & $\times$     & $\times$     & $\checkmark$ & $\checkmark$ & $\checkmark$ & $\checkmark$ & $\checkmark$ & $\checkmark$ \\
        \bottomrule
    \end{NiceTabular}
\end{table}
\subsection*{General Comments on the Analysis of a BIBD}
\begin{itemize}
    \item Primary analysis goal:
          \begin{itemize}
              \item Determine whether there exist significant differences among the expected response values from
                    one experimental condition to another.
          \end{itemize}
    \item In a RCBD, we do this by comparing the condition-specific means $ \bar{y}_{\bullet j\bullet} $ to the overall mean $ \bar{y}_{\bullet\bullet\bullet} $.
          This isn't fair in a BIBD because $ \bar{y}_{\bullet\bullet\bullet} $ is calculated from data from blocks that condition $ j $ didn't appear in.
    \item In a BIBD, due to incompleteness, we compare $ \bar{y}_{\bullet j\bullet} $ with the average response from the blocks that
          condition $j$ appeared in:
          \[ \frac{1}{r} \sum_{k=1}^{b} \bar{y}_{\bullet\bullet k}\Ind*{\text{condition $ j $ appears in block $ k $}} \]
    \item In general, the analysis of BIBDs involves an adjustment of this form when evaluating the effect of the
          design factor.
\end{itemize}
\section{Latin Square Designs}
\begin{itemize}
    \item Until now, we have discussed experimental designs that employ blocking to control for one nuisance
          factor:
          \begin{itemize}
              \item If we want to control for \emph{two} nuisance factors, we should use a \textbf{Latin square design}.
              \item If we want to control for \emph{three} nuisance factors, we should use a \textbf{Graeco-Latin square design}.
              \item If we want to control for \emph{four} nuisance factors, we should use a \textbf{Hyper-Graeco-Latin square design}.
          \end{itemize}
    \item A Latin square of order $ p $ is a $ p\times p $ grid containing $ p $ unique symbols.
          \begin{itemize}
              \item Each of these symbols occurs exactly once in each column.
              \item Each of these symbols occurs exactly once in each row.
              \item These ``symbols'' are typically denoted by Latin letters.
          \end{itemize}
          \begin{table}[!htbp]
              \centering
              \caption{$ 3\times 3 $, $ 4\times 4 $, and $ 5\times 5 $ Latin Square Examples}
              \begin{NiceTabular}{|ccc|}
                  \toprule
                  A & C & B \\
                  C & B & A \\
                  B & A & C\\
                  \bottomrule
              \end{NiceTabular}\quad
              \begin{NiceTabular}{|cccc|}
                  \toprule
                  A & B & C & D\\
                  C & D & A & B\\
                  B & C & D & A\\
                  D & A & B & C\\
                  \bottomrule
              \end{NiceTabular}\quad
              \begin{NiceTabular}{|ccccc|}
                  \toprule
                  A & B & C & D & E\\
                  E & A & B & C & D\\
                  D & E & A & B & C\\
                  C & D & E & A & B\\
                  B & C & D & E & A\\
                  \bottomrule
              \end{NiceTabular}
          \end{table}
    \item A Sudoku puzzle is a special example of a $ 9\times 9 $ Latin square.
    \item We exploit this combinatorial structure to help us design experiments that facilitate blocking by two
          nuisance factors.
          \begin{itemize}
              \item We randomly associate the $p$ rows with the levels of the first nuisance factor.
              \item We randomly associate the $p$ columns with the levels of the second nuisance factor.
              \item We randomly associate the $p$ Latin letters with the levels of the design factor.
          \end{itemize}
    \item We present an example with $ p=4 $ in~\Cref{latinsquarex1}.
          \begin{table}[!htbp]
              \centering
              \caption{$ 4\times 4 $ Latin Square Design}\label{latinsquarex1}
              \begin{NiceTabular}{|cc|cccc|}
                  \toprule            &   & \multicolumn{4}{c} {\emph{NF 2}}             \\
                  &   & 1                                          & 2 & 3 & 4 \\
                  \midrule            & 1 & A                                          & B & C & D \\
                  \multirow{2}{*}{\emph{NF 1}} & 2 & D                                          & A & B & C \\
                  & 3 & C                                          & D & A & B \\
                  & 4 & B                                          & C & D & A \\
                  \bottomrule
              \end{NiceTabular}
          \end{table}
          \begin{itemize}
              \item \textbf{Limitation}: we need to experiment with \emph{all} of these factors at $ p $ levels.
              \item $ (3,2) $ element represents the block where NF 1 is at level 3, NF 2 is at level 2, and DF is at level ``D.''
          \end{itemize}
    \item Each cell in this table represents a ``block'' in which we fix the nuisance factors' levels, and the
          Latin letter indicates which the execution of an experimental condition.
    \item Rows, columns, and letters are all orthogonal, allowing us to separately estimate the effects of the
          design factor and each of the two nuisance factors.
    \item We may informally summarize these effects with the overall average and level-specific averages of the
          response variables.
\end{itemize}
