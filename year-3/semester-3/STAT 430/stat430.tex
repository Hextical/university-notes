\documentclass[final]{book}
\usepackage[svgnames]{xcolor}
\usepackage{cfr-lm}
\usepackage{microtype}
\usepackage[math-style=ISO,bold-style=ISO,warnings-off={mathtools-colon,mathtools-overbracket}]{unicode-math}

% Core Packages
\usepackage[margin=1in]{geometry}
\usepackage[unicode,pdfversion=1.7]{hyperref}
\usepackage[shortlabels]{enumitem}
\usepackage[parfill]{parskip}
\usepackage[theorems,breakable]{tcolorbox}
\usepackage{graphicx}
\usepackage{mathtools}
\usepackage{cleveref}
\usepackage{booktabs}
\usepackage{nicematrix}
\usepackage{derivative}
\usepackage{etoolbox}
\usepackage{framed}
\usepackage{float}
\usepackage{tikz}
\usepackage{multirow}
\usepackage[skip=1ex]{caption}
\usepackage{interval}

% Functions
\providecommand\given{}% just to make sure it exists
\DeclarePairedDelimiterXPP{\E}[1]{\operatorname{\mathbb{E}}}[]{}{%
    \renewcommand\given{\nonscript\:\delimsize\vert\nonscript\:\mathopen{}}%
    \ifblank{#1}{\:\cdot\:}%
    #1}%
\DeclarePairedDelimiterXPP{\Var}[1]{\operatorname{\mathbb{V}}}(){}{%
    \renewcommand\given{\nonscript\:\delimsize\vert\nonscript\:\mathopen{}}%
    \ifblank{#1}{\:\cdot\:}%
    #1}%
\DeclarePairedDelimiterXPP\Prob[1]{\operatorname{\mathbb{P}}}(){}{%
    \renewcommand\given{\nonscript\:\delimsize\vert\nonscript\:\mathopen{}}%
    \ifblank{#1}{\:\cdot\:}%
    #1}%
\DeclarePairedDelimiterXPP\Ind[1]{\operatorname{\mathbb{I}}}\{\}{}{%
    \renewcommand\given{\nonscript\:\delimsize\vert\nonscript\:\mathopen{}}%
    \ifblank{#1}{\:\cdot\:}%
    #1}%
\DeclarePairedDelimiterXPP{\Se}[1]{\operatorname{\text{se}}}(){}{%
    \ifblank{#1}{\:\cdot\:}%
    #1}%
\DeclarePairedDelimiterXPP{\Cov}[1]{\operatorname{\text{Cov}}}(){}{%
    \ifblank{#1}{\:\cdot\:}%
    #1}%
\let\exp\relax%
\let\log\relax%
\let\ln\relax%
\DeclarePairedDelimiterXPP{\exp}[1]{\operatorname{\text{exp}}}[]{}{#1}%
\DeclarePairedDelimiterXPP{\log}[1]{\operatorname{\text{log}}}(){}{#1}%
\DeclarePairedDelimiterXPP{\ln}[1]{\operatorname{\text{ln}}}(){}{#1}%

% Distributions
\DeclarePairedDelimiterXPP{\N}[1]{\mathcal{N}}(){}{#1}%
\DeclarePairedDelimiterXPP{\MVN}[1]{\text{MVN}}(){}{#1}%
\DeclarePairedDelimiterXPP{\Binomial}[1]{\text{Binomial}}(){}{#1}
\DeclarePairedDelimiterXPP{\Uniform}[1]{\text{Uniform}}(){}{#1}

\newcommand{\iid}{\overset{\text{iid}}{\sim}}%

\newcommand{\SSC}{\text{SS}_{\text{C}}}%
\newcommand{\SSB}{\text{SS}_{\text{B}}}%
\newcommand{\SSE}{\text{SS}_{\text{E}}}%
\newcommand{\SST}{\text{SS}_{\text{T}}}%
\newcommand{\MSC}{\text{MS}_{\text{C}}}%
\newcommand{\MSB}{\text{MS}_{\text{B}}}%
\newcommand{\MSE}{\text{MS}_{\text{E}}}%
\newcommand{\HN}{\symbfup{H}_0}%
\newcommand{\HA}{\symbfup{H}_{\text{A}}}%

\DeclarePairedDelimiter\abs{\lvert}{\rvert}%
\DeclarePairedDelimiterX\Set[1]\{\}{%
    \renewcommand\given{:}%
    #1%
}%

\AtBeginDocument{%
    \let\mathbb\relax%
    \let\mathcal\relax%
    \DeclareMathAlphabet{\mathbb}{U}{msb}{m}{n}%
    \DeclareMathAlphabet{\mathcal}{OMS}{cmsy}{m}{n}%
}%

\newenvironment{tightcenter}{%
    \setlength\topsep{0pt}%
    \setlength\parskip{0pt}%
    \par\centering%
}{\par\noindent\ignorespacesafterend}%

\providecommand{\RandomVector}[1]{\symbf{#1}}% general vectors in bold italic
\providecommand{\Vector}[1]{\symbfup{#1}}% general vectors in bold italic
\providecommand{\Matrix}[1]{\symbfup{#1}}% matrix in bold roman % also symbfsfup
\providecommand{\Field}[1]{\symbfsfup{#1}}% matrix in bold roman % also symbfsfup

\tcbset{
    regular/.style={
        boxrule=0pt,
        breakable,
        sharp corners
    },
    common/.style={
            coltitle=black,
            boxrule=0pt,
            breakable,
            sharp corners
        },
    theorem/.style={
            common,
            colback=mypurple,
            colframe=mypurple!95!black,
            fontupper=\itshape{}
        },
}

\newcommand{\makeheading}[1]
{
    \begin{figure}[H]
        \centering
        \rule{\columnwidth}{1pt}\\
        {\large \scshape{#1}}\\[-0.6\baselineskip]
        \rule{\columnwidth}{1pt}
        \vspace*{-20pt}
    \end{figure}
}
\DeclareMathOperator{\FWER}{FWER}
\DeclareMathOperator{\FDR}{FDR}
\graphicspath{ {./figures/} }

% Definitions
\definecolor{myyellow}{RGB}{255,255,168}
% Theorems
\definecolor{mypurple}{RGB}{216,216,255}
% Algorithms
\definecolor{mygray}{RGB}{232,232,232}
% Examples
\definecolor{mygreen}{RGB}{216,255,216}
% Exercises
\definecolor{myred}{RGB}{255,216,216}
% Remarks
\definecolor{mycyan}{RGB}{204,229,229}

\newtcbtheorem[number within=section, crefname={definition}{definitions}]
{Definition}{DEFINITION}{
    common,
    colback=myyellow,
    colframe=myyellow!95!black
}{def}

\newtcbtheorem[use counter from=Definition, crefname={example}{examples}]
{Example}{EXAMPLE}{
    common,
    colback=mygreen,
    colframe=mygreen!95!black,
}{ex}

\newtcbtheorem[use counter from=Definition, crefname={exercise}{exercises}]
{Exercise}{EXERCISE}{
    common,
    colback=myred,
    colframe=myred!95!black,
}{exercise}

\newtcbtheorem[use counter from=Definition, crefname={remark}{remarks}]
{Remark}{REMARK}{
    common,
    colback=mycyan,
    colframe=mycyan!95!black,
}{remark}

\newtcbtheorem[use counter from=Definition, crefname={statistical Test}{statistical Tests}]
{Statistical_Test}{STATISTICAL TEST}{
    common,
    colback=Magenta!25!white,
    colframe=Magenta!50!white,
}{stest}

\newtcbtheorem[use counter from=Definition, crefname={theorem}{theorems}]
{Theorem}{THEOREM}{
    theorem
}{thm}

\newtcbtheorem[use counter from=Definition, crefname={proposition}{propositions}]
{Proposition}{PROPOSITION}{
    theorem
}{prop}

\newtcbtheorem[use counter from=Definition, crefname={corollary}{corollaries}]
{Corollary}{COROLLARY}{
    theorem
}{cor}

\newtcbtheorem[use counter from=Definition, crefname={lemma}{lemmas}]
{Lemma}{LEMMA}{
    theorem
}{lem}

\newtcbtheorem[no counter]
{Proof}{Proof of}{
    common,
    colframe=black!10,
    separator sign={}
}{pf}

\hypersetup{colorlinks,linkcolor=[rgb]{0,0.5,1}}%

\hypersetup{pdftitle={Experimental Design (STAT 430/STAT 830)},
pdfauthor={Cameron Roopnarine, Nathaniel Stevens},
pdfsubject={Statistics},
pdfkeywords={University of Waterloo, Spring 2021 (1215)}}

\title{
\LARGE Experimental Design\\
\large STAT 430/STAT 830\\
\normalsize Spring 2021 (1215)}
\author{\LaTeX{}er: \emph{Cameron Roopnarine}\\Instructor: \emph{Nathaniel Stevens}}
\date{\today}

%\usepackage[authordate,bibencoding=auto,strict,backend=biber,natbib]{biblatex-chicago}
%\addbibresource{references.bib}

\begin{document}

\maketitle

\tableofcontents

\chapter{Integration}
\setcounter{section}{1}
\section{Riemann Sums and the Definite Integral}
To begin with, our goal is to develop methods for determining the area under a curve.

We know we can approximate the area using rectangles (or other geometric shapes), but
we want the \emph{exact} area. For this, we will need \emph{Riemann sums}.

\begin{Definition}{Partition}{partition}
    A \textbf{partition}, $P$, for the interval $ \interval{a}{b} $ is a finite
    sequence of increasing numbers of the form
    \[ a=t_0<t_1<t_2\cdots<t_{n-1}<t_n=b \]
    This partition subdivides the interval $ \interval{a}{b} $ into $ n $ subintervals:
    \[ \interval{t_0}{t_1},\ldots,\interval{t_{n-1}}{t_n} \]
\end{Definition}

\begin{Remark}{}{}
    These subintervals may \emph{not} all have the same length.
\end{Remark}

\begin{Definition}{Length}{length}
    Denote the \textbf{length} of the $ i^{\text{th}} $ subinterval,
    $ \interval{t_{i-1}}{t_i} $, by $ \Delta t_i $; that is, $ \Delta t_i=t_i-t_{i-1} $.
\end{Definition}

\begin{Definition}{Norm}{norm}
    The \textbf{norm} of a partition is the length of the widest subinterval:
    \[ \norm{P}=\max(\Delta t_1,\dots,\Delta t_{n}) \]
\end{Definition}

\begin{Definition}{Riemann sum}{riemann_Sum}
    Given a bounded function $ f $ on $ \interval{a}{b} $,
    a partition $ P $ of $ \interval{a}{b} $, and a set
    $ \set{c_1,\dots,c_n} $, where $ c_i\in\interval{t_{i-1}}{t_i} $, then a
    \textbf{Riemann sum} for $ f $ with respect to $ P $ is
    \[ S=\sum\limits_{i=1}^{n} f(c_i)\Delta t_i \]
\end{Definition}

Again, we want the \emph{exact} area, and for that we will need to use infinitely
many points!

But we do need to make sure that the norm of our partitions is getting smaller,
and that the area we get doesn't depend on the choice of Riemann sum.

\begin{Definition}{Integrable, Integral of $ f $}{integrable}
    We say that $ f $ is \textbf{integrable} on $ \interval{a}{b} $ if there exists a unique number
    $ I\in\mathbb{R} $ such that if whenever $ \set{P_n} $ is a sequence of partitions with
    $ \lim\limits_{{n} \to {\infty}}\norm{P_n}=0 $ and $ \set{S_n} $ is any sequence of
    Riemann sums associated to the $ P_n $'s, we have $ \lim\limits_{{n} \to {\infty}} S_n=I $.

    In this case, we call $ I $ the \textbf{integral of $ f $} over $ \interval{a}{b} $
    and denote it by
    \[ \int_{a}^{b} f(x)\, dx \]
    where $ a,b $ are the bounds of integration, $ f(x) $ is the integrand, $ x $ is the
    variable of integration. The complete object is called a definite integral.

    It represents the exact (signed) area under $ f $.
\end{Definition}

\begin{Remark}{}{}
    The variable of integration is a \emph{dummy variable} since we can change it into
    whatever we want and it won't change the value of the integral; that is,
    \[
        \int_{a}^{b} f(x)\,dx =
        \int_{a}^{b} f(t) \,d{t}=
        \int_{a}^{b} f(\cdot)\, d{\cdot}
    \]
\end{Remark}

This looks \emph{horrible} to compute in practice (and it is). The good news is if
$ f $ is continuous, it's not so bad! (still bad though)

\begin{Theorem}{Integrability Theorem for Continuous Functions}{integrability_thm}
    Let $ f $ be continuous on $ \interval{a}{b} $.
    Then $ f $ is integrable on $ \interval{a}{b} $.
\end{Theorem}

\begin{Proof}{\ref{thm:integrability_thm}}{}
    Beyond the scope of this course.
\end{Proof}

This is fantastic! This means that we can \emph{choose} any collection of Riemann sums
we want when computing the integral of a continuous function!

Let's examine a ``nice'' choice: one where the partition is regular and where we just
pick the $ c_i $'s to be the right-hand endpoints!

\begin{Definition}{Regular $n$-partition}{regular_partition}
    For the interval $ \interval{a}{b} $, the \textbf{regular $ n $-partition}
    where all $ n $ subintervals
    have the same length; that is,
    \[ \Delta t=\frac{b-a}{n} \quad\text{and}\quad  t_i=t_0+i\Delta t \]
\end{Definition}

\begin{Definition}{Regular right-hand Riemann sum}{right_hand_reimann}
    Using this, we define the \textbf{regular right-hand Riemann sum} by taking $ c_i=t_i $ for
    all $ i $:
    \[ S_n=\sum\limits_{i=1}^{n} f(t_i)\Delta t=\sum\limits_{i=1}^{n} f(t_i)\left(\frac{b-a}{n}\right) \]
\end{Definition}

\begin{Remark}{}{}
    We can also define the regular left-hand Riemann sum.
\end{Remark}

Now, we can write a nicer formula for integrating continuous functions!

If $ f $ is continuous, then
\[ \boxed{\int_{a}^{b} f(x)\, d{x} =
        \lim\limits_{{n} \to {\infty}} \sum\limits_{i=1}^{n} f(t_i)\left(\frac{b-a}{n}\right)} \]

\begin{Example}{}{}
    Evaluate
    $ \displaystyle\int_{0}^{4} x+x^3\, d{x} $.

    \textbf{Solution.}
    Since $ f(x)=x+x^3 $ is continuous, we can use the above formula.

    In our case: $ \dfrac{b-a}{n} = \dfrac{4}{n} $, and $ t_i = 0+\dfrac{4i}{n} = \dfrac{4i}{n} $.

    So, $ f(t_i) = \dfrac{4i}{n} + \dfrac{64i^3}{n^3} $.
    Then, we get:
    \begin{align}
        \int_{0}^{4} x+x^3\, d{x}
         & = \lim\limits_{{n} \to {\infty}} \sum\limits_{i=1}^{n}
        \left( \frac{4i}{n} +\frac{64i^3}{n^3} \right)\left( \frac{4}{n} \right)                  \\
         & = \lim\limits_{{n} \to {\infty}} \frac{16}{n^2} \sum\limits_{i=1}^{n} i +
        \frac{256}{n^4} \sum\limits_{i=1}^{n} i^3 \label{1.2_reimann}                             \\
         & = \lim\limits_{{n} \to {\infty}} \frac{16}{n^2} \left[ \frac{n(n+1)}{2} \right] +
        \frac{256}{n^4} \left[ \frac{n^2(n+1)^2}{4} \right] \label{1.3_reimann}                   \\
         & = \lim\limits_{{n} \to {\infty}} \frac{8n+8}{n} +64 \left(\frac{n^2+2n+1}{n^2} \right) \\
         & = 8+64                                                                                 \\
         & =72
    \end{align}
    where from~\ref{1.2_reimann} to~\ref{1.3_reimann} we used both of the following:
    \[ \sum\limits_{i=1}^{n} i=\frac{n(n+1)}{2} \text{ and }
        \sum\limits_{i=1}^{n} i^3=\frac{n^2(n+1)^2}{4} \]
\end{Example}

\begin{Remark}{}{}
    The theorem also holds for functions that are bounded and have finitely many
    discontinuities.
\end{Remark}

\section{Properties of the Definite Integral}

Since a definite integral is the limit of a sequence, many limit laws also hold!

\begin{Theorem}{Properties of Integrals}{properties_of_integrals}
    Assume that $ f $ and $ g $ are integrable on the interval $ \interval{a}{b} $. Then:
    \begin{enumerate}[label=(\arabic*)]
        \item\label{property_integral_1} For any $ c\in\mathbb{R} $,
              $ \displaystyle\int_{a}^{b} cf(x)\, d{x} = c \int_{a}^{b} f(x)\, d{x} $.
        \item\label{property_integral_2}
              $ \displaystyle \int_{a}^{b} (f+g)(x)\, d{x} = \int_{a}^{b} f(x)\, d{x} +
                  \int_{a}^{b} g(x)\, d{x} $.
        \item\label{property_integral_3} If $ m\le f(x)\le M $ for all $ x\in\interval{a}{b} $,
              then
              $ \displaystyle m(b-a)\le \int_{a}^{b} f(x)\, d{x} \le M(b-a) $.
        \item\label{property_integral_4} If $ 0\le f(x) $ for all $ x\in\interval{a}{b} $, then
              $ \displaystyle 0\le \int_{a}^{b} f(x)\, d{x} $.
        \item\label{property_integral_5} If $ f(x)\le g(x) $ for all $ x\in\interval{a}{b} $, then
              $ \displaystyle \int_{a}^{b} f(x)\, d{x} \le \int_{a}^{b} g(x)\, d{x} $.
        \item\label{property_integral_6} The function
              $ \abs{f} $ is integrable on $ \interval{a}{b} $ and
              $ \displaystyle \abs[\bigg]{\int_{a}^{b} f(x)\, d{x}}
                  \le \int_{a}^{b} \abs{f(x)}\, d{x} $.
    \end{enumerate}
\end{Theorem}

\begin{Proof}{\ref{thm:properties_of_integrals}}{}
    \begin{itemize}
        \item~\ref{property_integral_1} and~\ref{property_integral_2} follow from limit laws
              for sequences.
        \item~\ref{property_integral_3} implies~\ref{property_integral_4}.
        \item~\ref{property_integral_1},~\ref{property_integral_2},
              and~\ref{property_integral_4} imply~\ref{property_integral_5}.
        \item~\ref{property_integral_6} follows from the triangle inequality.
    \end{itemize}

    We will now prove~\ref{property_integral_3}.

    Suppose $ m\le f(x)\le M $ and partition the interval
    $ a=t_0<\cdots<t_n=b $.

    Note that
    $ \displaystyle\sum\limits_{i=1}^{n} \Delta t=\frac{b-a}{n}(n)=b-a $
    Then, since $ m\le f(x)\le M $, we get
    \[ m(b-a)=\sum\limits_{i=1}^{n} m\Delta t\le \sum\limits_{i=1}^{n} f(t_i)\Delta t
        \le \sum\limits_{i=1}^{n} M\Delta t=M(b-a) \]
    So, taking limits gives
    \[ m(b-a)\le \int_{a}^{b} f(x)\,d{x} \le M(b-a) \]
\end{Proof}

\begin{Definition}{More properties}{more_properties}
    \begin{enumerate}[label=(\Roman*)]
        \item If $ f(a) $ is defined, then
              $ \displaystyle\int_{a}^{a} f(x)\, d{x} =0 $
        \item If $ f $ is integrable on $ \interval{a}{b} $, then
              $ \displaystyle\int_{a}^{b} f(x)\, d{x}=-\int_{b}^{a} f(x)\, d{x} $
    \end{enumerate}
\end{Definition}

\begin{Theorem}{}{extra_integ_property}
    If $ f $ is integrable on an interval $ I $ containing $ a,b $, and $ c $, then
    \[ \int_{a}^{b} f(x)\, d{x}=\int_{a}^{c} f(x)\, d{x}+\int_{c}^{b} f(x)\, d{x} \]
\end{Theorem}

\begin{Proof}{\ref{thm:extra_integ_property}}{}
    Beyond the scope of this course.
\end{Proof}

\begin{Remark}{}{}
    $ c $ does \emph{not} need to be between $ a $ and $ b $!
\end{Remark}

\subsection*{Geometric Interpretation of the Integral}
So far, we have only examined positive functions, but we should note that $ \int_{a}^{b} f(x)\,dx $
returns the \emph{signed} area between $ f $ and the $ x $-axis. That is, if $ f(x)\le 0 $, then
$ \int_{a}^{b} f(x)\,dx\le 0 $ too.

So, in general, $ \int_{a}^{b} f(x)\,dx $ is the area under $ f $ that
lies above the $ x $-axis \emph{minus} the area above the graph of
$ f $ that lies below the $ x $-axis.

\begin{Example}{}{}
    \[ \int_{-1}^{1}x\,dx=R_2-R_1 \]
    but $ R_2=R_2 $, so
    \[ \int_{-1}^{1}x\,dx=0 \]
    \begin{figure}[H]
        \centering
        \tikzset{every picture/.style={line width=0.75pt}} %set default line width to 0.75pt        

        \begin{tikzpicture}[x=0.75pt,y=0.75pt,yscale=-1,xscale=1]
            %Shape: Grid [id:dp8044758903954956] 
            \draw  [draw opacity=0] (252.83,80.33) -- (373.67,80.33) -- (373.67,201.33) -- (252.83,201.33) -- cycle ; \draw  [color={rgb, 255:red, 168; green, 168; blue, 168 }  ,draw opacity=1 ] (252.83,80.33) -- (252.83,201.33)(282.83,80.33) -- (282.83,201.33)(312.83,80.33) -- (312.83,201.33)(342.83,80.33) -- (342.83,201.33)(372.83,80.33) -- (372.83,201.33) ; \draw  [color={rgb, 255:red, 168; green, 168; blue, 168 }  ,draw opacity=1 ] (252.83,80.33) -- (373.67,80.33)(252.83,110.33) -- (373.67,110.33)(252.83,140.33) -- (373.67,140.33)(252.83,170.33) -- (373.67,170.33)(252.83,200.33) -- (373.67,200.33) ; \draw  [color={rgb, 255:red, 168; green, 168; blue, 168 }  ,draw opacity=1 ]  ;
            %Shape: Right Triangle [id:dp6090851976145147] 
            \draw   (313,80.67) -- (373,140.33) -- (313,140.33) -- cycle ;
            %Shape: Right Triangle [id:dp2511359111159521] 
            \draw   (312.83,200.33) -- (252.83,140.5) -- (312.83,140.5) -- cycle ;

            % Text Node
            \draw (375,143.73) node [anchor=north west][inner sep=0.75pt]    {$1$};
            % Text Node
            \draw (231,143.4) node [anchor=north west][inner sep=0.75pt]    {$-1$};
            % Text Node
            \draw (287.83,146.73) node [anchor=north west][inner sep=0.75pt]    {$R_{1}$};
            % Text Node
            \draw (317.83,116.73) node [anchor=north west][inner sep=0.75pt]    {$R_{2}$};
        \end{tikzpicture}
    \end{figure}
\end{Example}

\begin{Remark}{}{}
    If we are lucky, we can use geometric formulas to evaluate integrals
    (see pg 26--28 in the notes). However, we are almost never this lucky\textellipsis{}
\end{Remark}

\section{Average Value of a Function}

\begin{Definition}{Average value}{avg_value}
    If $ f $ is continuous on $ \interval{a}{b} $, the \textbf{average value} of $ f $
    on $ \interval{a}{b} $ is defined as
    $ \displaystyle \frac{1}{b-a} \int_{a}^{b} f(x)\,dx $.
\end{Definition}

\subsection*{Geometric Interpretation}
\begin{Proof}{\ref{thm:avt}}{}
    If $ f $ is continuous on $ \interval{a}{b} $, EVT says there exists $ m,M\in\mathbb{R} $ such that
    $ \displaystyle m\le f(x) \le M $
    for $ x\in\interval{a}{b} $, and $ f(c_1)=m $, $ f(c_2)=M $ for some $ c_1,c_2\in\interval{a}{b} $.

    Also, we know
    \begin{align*}
        m(b-a)\le \int_{a}^{b} f(x)\, d{x} \le M(b-a)
         & \implies m\le \frac{1}{b-a} \int_{a}^{b} f(x)\, d{x} \le M       \\
         & \iff f(c_1)\le \frac{1}{b-a} \int_{a}^{b} f(x)\, d{x} \le f(c_2) \\
    \end{align*}
    IVT says there exists $ c $ between $ c_1 $ and $ c_2 $, so that
    \[ f(c)=\frac{1}{b-a} \int_{a}^{b} f(x)\, d{x} \]
\end{Proof}

\begin{Theorem}{Average Value Theorem (AVT)}{avt}
    Assume $ f $ is continuous on $ \interval{a}{b} $.
    There exists $ c\in\interval{a}{b} $ such that
    $ \displaystyle f(c)=\frac{1}{b-a} \int_{a}^{b} f(x) d{x} $.
\end{Theorem}

\begin{Remark}{}{}
    Note that this theorem holds even if $ b<a $ since
    \begin{align*}
        f(c) & =\frac{1}{a-b} \int_{b}^{a}\, f(x)dx                  \\
             & =\frac{1}{a-b}\biggl(-\int_{a}^{b} f(x)\, d{x}\biggr) \\
             & =\frac{1}{b-a} \int_{a}^{b} f(x)\, d{x}
    \end{align*}
\end{Remark}

The big problem we face now is that evaluating $ \int_{a}^{b} f(x)\, d{x} $ is
monstrously difficult for all but the simplest of functions.

IF ONLY THERE WAS A BETTER WAY\@!

(spoilers: there's a better way! It's the Fundamental Theorem of Calculus!)


\makeheading{Week 2}{\daterange{2021-09-13}{2021-09-17}}%chktex 8
\section{Module 1: Measures of Disease Frequency}
\subsection{Incidence and Prevalence Rates}
\subsubsection*{How do we measure and evaluate patterns of disease within a population?}
\begin{figure}[H]
    \centering
    \includegraphics[width=0.75\textwidth]{1.1/1.pdf}
\end{figure}
\subsubsection*{Primer on HIV/AIDS}
\begin{itemize}
    \item HIV (human immunodeficiency virus) is a virus that attack’s the body’s immune
          system.
    \item HIV is spread through sexual contact, sharing needles, and mother-to-child
          transmission during pregnancy, childbirth, or breastfeeding.
    \item Infection with HIV can lead to AIDS (acquired immunodeficiency syndrome).
    \item Individuals with AIDS are at increased risk of infection and infection-related
          cancers.
    \item Currently, no cure exists, but antiretroviral therapy can slow the progression of the
          disease.
\end{itemize}
\subsubsection*{1.1 Incidence and Prevalence Rates}
\begin{Regular}
    \textcolor{Blue}{Goal}: How do we measure and evaluate patterns of disease within a population?
\end{Regular}
\begin{itemize}
    \item \textcolor{Blue}{Prevalence}: The proportion of the population currently affected by a disease.
    \item \textcolor{Blue}{Incidence}: The rate at which new cases of a disease develop in a population.
\end{itemize}
\subsubsection*{Number of people living with HIV/AIDS 1990--2017}
\begin{figure}[H]
    \centering
    \includegraphics[width=0.75\textwidth]{1.1/2.pdf}
\end{figure}
\subsubsection*{Prevalence}
\textcolor{Blue}{Prevalence}: The proportion of the population currently affected by a disease.
\begin{Regular}
    \[ \begin{tabular}{>{\bfseries}c}
            Point Prevalence \\
            per 1000
        \end{tabular}=\frac{\begin{tabular}{>{\bfseries}c}
                Number of cases (new and pre-existing) in the \\
                population at a fixed point in time
            \end{tabular}}{\begin{tabular}{>{\bfseries}c}
                Number of individuals in the \\
                population at a fixed point in time
            \end{tabular}}\times 1000.  \]
\end{Regular}
\begin{Regular}
    \[ \begin{tabular}{>{\bfseries}c}
            Period Prevalence \\
            per 1000
        \end{tabular}=\frac{\begin{tabular}{>{\bfseries}c}
                Number of cases (new and pre-existing) in the \\
                population over a given time period
            \end{tabular}}{\begin{tabular}{>{\bfseries}c}
                Number of individuals in the \\
                population over a given time period
            \end{tabular}}\times 1000.  \]
\end{Regular}
\subsubsection*{Prevalence of HIV/AIDS 1990--2017}
\begin{figure}[H]
    \centering
    \includegraphics[width=0.75\textwidth]{1.1/3.pdf}
\end{figure}
\subsubsection*{Annual new cases of HIV infection, 1990--2017}
\begin{figure}[H]
    \centering
    \includegraphics[width=0.75\textwidth]{1.1/4.pdf}
\end{figure}
\subsubsection*{Cumulative Incidence}
\textcolor{Blue}{Incidence}: The rate at which \textbf{new cases} of a disease develop in a \textbf{population} over a
specific \textbf{time} period.
\begin{Regular}
    \[ \begin{tabular}{>{\bfseries}c}
            Cumulative Incidence \\
            per 1000
        \end{tabular}=\frac{\begin{tabular}{>{\bfseries}c}
                Number of new cases in the \\
                population over the time period of interest
            \end{tabular}}{\begin{tabular}{>{\bfseries}c}
                Number of individuals at risk in the \\
                population at the start of the time period of interest
            \end{tabular}}\times 1000.  \]
\end{Regular}
\begin{itemize}
    \item Assumes all subjects remain in the population and at risk for the entire time period.
    \item Easily violated: Births, deaths, immigration, emigration, case diagnosis.
    \item Consider two ways to refine the denominator calculation.
          \begin{enumerate}[1.]
              \item Use a mid-interval population estimate.
              \item Calculate the total person-time at risk in the population.
          \end{enumerate}
\end{itemize}
\subsubsection*{Incidence Density or Incidence Rate}
\begin{Regular}
    \[ \begin{tabular}{>{\bfseries}c}
            Incidence Density \\
            per 1000
        \end{tabular}=\frac{\begin{tabular}{>{\bfseries}c}
                Number of new cases in the \\
                population over the time period of interest
            \end{tabular}}{\begin{tabular}{>{\bfseries}c}
                Mid-interval estimate of the population
            \end{tabular}}\times 1000.  \]
\end{Regular}
\begin{Example}
    \begin{itemize}
        \item 3,218 new cases of HIV in Canada, 2016.
        \item 36,264,604 July 1, 2016 Canadian population estimate.
              \[ \begin{tabular}{>{\bfseries}c}
                      Incidence Density \\
                      per 100,000
                  \end{tabular}=\frac{3,218}{36,264,604}\times 100,000=8.873666. \]
        \item \emph{The incidence of HIV infection in Canada in the year 2016 was 8.87 cases per
                  100,000 persons}.
    \end{itemize}
\end{Example}
\subsubsection*{Incidence of HIV per 1,000 uninfected adults, 2000--2017}
\begin{figure}[H]
    \centering
    \includegraphics[width=0.75\textwidth]{1.1/5.pdf}
\end{figure}
\subsubsection*{Person-time at risk}
\begin{itemize}
    \item To account for varying time periods of risk we consider an alternative denominator
          for our incidence calculation.
    \item Person-time at risk is the duration of time an individual is at risk for developing
          a disease.
    \item Assuming they are initially disease free, it is the length of time from baseline until
          the first of:
          \begin{enumerate}[1.]
              \item They develop the disease of interest and become a case.
              \item They cease to be at risk of becoming a case due to either death from unrelated
                    causes or they leave the population.
              \item The end of the time period of interest is reached.
          \end{enumerate}
    \item Total person-time at risk is the sum of the individual contributions over the
          population.
\end{itemize}
\subsubsection*{Incidence Density or Incidence Rate}
\begin{Regular}
    \[ \begin{tabular}{>{\bfseries}c}
            Incidence Density \\
            per 1000
        \end{tabular}=\frac{\begin{tabular}{>{\bfseries}c}
                Number of new cases in the \\
                population over the time period of interest
            \end{tabular}}{\begin{tabular}{>{\bfseries}c}
                Mid-Total person-time at risk in the \\
                population over the time period of interest
            \end{tabular}}\times 1000.  \]
\end{Regular}
\begin{itemize}
    \item Incidence density estimate is more precise than cumulative incidence, but may be
          harder to get information needed, so this measure is often used for small
          populations.
    \item Expressed as per $10^x$ person-years (-month, -day).
\end{itemize}
\subsubsection*{Relationship Between Incidence and Prevalence}
\begin{figure}[H]
    \centering
    \includegraphics[width=0.25\textwidth]{1.1/6.jpg}
\end{figure}
\subsubsection*{Relationship Between Incidence and Prevalence}
\begin{itemize}
    \item \textbf{Incidence}: the rate new cases are diagnosed in a population.
    \item \textbf{Prevalence}: the proportion of the population currently affected by the disease.
\end{itemize}
\begin{Regular}
    \[ \textbf{Prevalence}\approx \textbf{Incidence}\times \textbf{Disease Duration} \]
\end{Regular}
\begin{itemize}
    \item Relationship is approximate but generally holds well if prevalence is low ($<10\,$\%)
          and duration is fairly constant (or an average can be taken).
    \item Note: units must be consistent in order to perform the multiplication operation.
\end{itemize}
\subsubsection*{Prevalence, new cases, and mortality for HIV/AIDS}
\begin{figure}[H]
    \centering
    \includegraphics[width=0.75\textwidth]{1.1/7.pdf}
\end{figure}
\subsubsection*{Exercise: Incidence and Prevalence Calculations}
Total population size of 100. Histories of 12 subjects with disease are below. Subjects
13--100 do not have the disease during the year of study. ($ \triangle $ Diagnosis; $ \times $ Death)
\begin{table}[H]
    \centering
    \begin{tabular}{ccc}
        \toprule
        \textbf{Subject} & \textbf{Diagnosis} $ \triangle $ & \textbf{Death} $ \times $ \\
        \midrule
        1                & $<$ January 1                                                \\
        2                & $<$ January 1                    & April 30                  \\
        3                & $<$ January 1                                                \\
        4                & $<$ January 1                                                \\
        5                & $<$ January 1                    & June 30                   \\
        6                & March 1                          & October 31                \\
        7                & May 1                                                        \\
        8                & May 1                                                        \\
        9                & July 1                                                       \\
        10               & July 1                           & October 31                \\
        11               & NA                               & May 1                     \\
        12               & NA                               & September 1               \\
        13--100          & NA                                                           \\
        \bottomrule
    \end{tabular}
\end{table}
\begin{figure}[H]
    \centering
    \includegraphics[width=0.75\textwidth]{1.1/8.pdf}
\end{figure}
\begin{itemize}
    \item Point Prevalence on July 1.
    \item Period Prevalence (Jan 1 to Dec 31)
    \item Cumulative Incidence (Jan 1 to Dec 31).
    \item Incidence Density (Jan 1 to Dec 31).
\end{itemize}
\begin{align*}
    \begin{tabular}{c}
        Point Prevalence \\
        on July 1
    \end{tabular}
     & =\frac{\text{\# cases in the pop on July 1}}{\text{\# indv in the pop on July 1}}\times 1000=\frac{8}{97}\times 1000=\text{$82.47$ per 1000 persons}.                     \\\\
    \begin{tabular}{c}
        Period Prevalence \\
        (Jan 1-Dec 31)
    \end{tabular}
     & =\frac{\text{\# cases during Jan 1-Dec 31}}{\text{\# indv in the pop Jan 1-Dec 31}}\times 1000=\frac{10}{100}\times 1000=\text{$100$ per 1000 persons}.                   \\\\
    \begin{tabular}{c}
        Cumulative Incidence \\
        (Jan 1-Dec 31)
    \end{tabular}
     & =\frac{\text{\# new cases during Jan 1-Dec 31}}{\text{\# indv at risk on Jan 1}}\times 1000=\frac{5}{95}\times 1000=\text{$52.63$ per 1000 persons}.                      \\\\
    \begin{tabular}{c}
        Incidence Density \\
        (Jan 1-Dec 31)
    \end{tabular}
     & =\frac{\text{\# new cases during Jan 1-Dec 31}}{\text{July 1 population size}}\times 1000=\frac{5}{97}\times 1000=\text{$51.55$ per 1000 persons}.                        \\\\
    \begin{tabular}{c}
        Incidence Density \\
        (Jan 1-Dec 31)
    \end{tabular}
     & =\frac{\text{\# new cases during Jan 1-Dec 31}}{\text{\small Total person-years at risk Jan 1-Dec 31}}\times 1000=\frac{5}{90.83}\times 1000=\text{$55.05$ per 1000 p-y}. \\
\end{align*}
\subsection{Standardization of Rates: Indirect Methods}
% Chapter 2 Part 1
\chapter{Causal Inference and Potential Outcomes}
\makeheading{Week 3}{\daterange{2022-01-17}{2022-01-21}}%chktex 8
\section{Causal Inference}
\subsection{Introduction}
\subsubsection{Reference}
\begin{itemize}
    \item Hernán M.A., \& Robins J.M. (2020). Causal Inference: What
          If. Boca Raton: Chapman Hall/CRC\@.

          \url{https://www.hsph.harvard.edu/miguel-hernan/
              causal-inference-book/}
\end{itemize}
\subsubsection{Causal Inference}
Two notions of causation:
\begin{itemize}
    \item Causes of an effect/outcome.
    \item Effects of a cause.
\end{itemize}
\textbf{Causes of an effect}
\begin{itemize}
    \item What are causes of lung cancer?
    \item What was the cause of outbreak of food poisoning?
\end{itemize}
\textbf{Effects of a cause/intervention}
\begin{itemize}
    \item Does smoking cause lung cancer?
    \item Does mixed feeding cause obesity?
    \item How strong is the effect?
\end{itemize}
\begin{itemize}
    \item We concentrate on effects of a cause/treatment/intervention.
    \item Fundamentally simpler question: search is for useful
          information rather than complete scientific understanding.
    \item Typical approach for estimating causal effects (which may be
          problematic):
          collect sample on treatments/exposures, outcomes, and other
          variables in population; Use standard statistical methods
          (such as multiple regression) to derive inferences about
          associations between observable variables.
\end{itemize}
\subsubsection*{A Note}
\begin{itemize}
    \item In pharmaceutical companies, people used to believe
          conducting randomized clinical trials is the only way to
          evaluate a newly developed drug. However, there is a shifting
          trend going on right now because of:
          \begin{itemize}
              \item Difficult to find control subjects.
              \item Compliance issue.
              \item Exclusion criteria.
              \item Cost issue.
          \end{itemize}
    \item New trend: utilizing existing Electronic Health Records data
          to help find controls.
    \item The study is not randomized any more: observational study.
\end{itemize}
\subsubsection*{Draw Causality}
\textbf{Observational Studies}
\begin{itemize}
    \item No control over which subjects have the exposure and which
          do not.
    \item Exposed and Unexposed groups may be quite different with
          respect to other subject characteristics
    \item It is sometimes useful to use these studies to look at the
          natural history of a disease, but any attempt to identify
          causality b/t a risk factor and outcome must be done w/
          great caution.
\end{itemize}
\subsection{Confounding}
\subsubsection*{Confounding Issue in Observational Studies}
\begin{Example}{}
    Differences in the outcome are not only due to the treatment, but
    also because of the masking effect of covariates (confounders).
    \begin{center}
        \begin{tikzpicture}[thick]
            \node (1) at (0,1) {\textcolor{Blue}{Gender ($ X $)}};
            \node (2) at (-2,0) {\textcolor{Blue}{Smoking ($ A $)}};
            \node (3) at (2,0) {\textcolor{Blue}{Life Expectancy ($ Y $)}};
            \draw[->] (1) to (2);
            \draw[->] (1) to (3);
            \draw[->] (2) to (3);
        \end{tikzpicture}
    \end{center}
    Here, gender is known as a confounder. Very often, in real
    applications, the list of potential confounders could be very large,
    and even high-dimensional.
\end{Example}
\subsubsection*{Another Example of Confounding}
\begin{Example}{}
    Researchers find when the consumption of ice cream increases, the
    death from drowning increases. Does eating ice cream lead to
    drowning?
    \begin{center}
        \begin{tikzpicture}[thick]
            \node (1) at (0,1) {\textcolor{Blue}{Summer ($ X $)}};
            \node (2) at (-2,0) {\textcolor{Blue}{Ice cream ($ A $)}};
            \node (3) at (2,0) {\textcolor{Blue}{Drowning ($ Y $)}};
            \draw[->] (1) to (2);
            \draw[->] (1) to (3);
        \end{tikzpicture}
    \end{center}
    Here, summer (hot weather) is a confounder.
\end{Example}
\subsubsection*{Potential Outcomes Framework}
\begin{itemize}
    \item Useful to have more precise definitions of causal effects.
    \item Demystifies the process of going from association to causation.
    \item Allows explicit statements regarding what assumptions are
          necessary to justify causal inferences.
    \item Allows for more critical, better informed evaluation of causal
          claims.
    \item Helps determine when familiar methods useful or unfamiliar
          methods necessary.
    \item Motivates derivation and use of unfamiliar methods.
\end{itemize}
\section{Potential Outcomes Framework}
\subsection*{Definition of a Causal Effect}
\begin{Regular}{}
    Suppose we have data on subjects $ i=1,\ldots,n $.
    \begin{itemize}
        \item $ \Vector{X}_i=(\Vector{X}_{i1},\Vector{X}_{i2},\ldots,\Vector{X}_{ip})^\top $: baseline covariates/potential confounders.
        \item $ A_i $: treatment assignment/exposure status for subject $ i $
              \[ A_i=\begin{cases*}
                      1, & if exposed/treated,   \\
                      0, & if unexposed/treated.
                  \end{cases*} \]
        \item $ Y_i $: observed outcome for subject $ i $.
    \end{itemize}
    \textbf{Counterfactuals/Potential outcomes}
    \begin{itemize}
        \item $ Y_i^1 $: the potential outcome if subject $ i $ were \textcolor{Blue}{treated/exposed}.
        \item $ Y_i^0 $: the potential outcome if subject $ i $ were \textcolor{Blue}{untreated/unexposed}.
    \end{itemize}
\end{Regular}
\begin{Regular}{}
    The \textbf{individual-level causal effect} for subject $ i $ is:
    \[ Y_i^1-Y_i^0. \]
\end{Regular}
\begin{Regular}{Causal Estimand}
    The \textbf{average causal effect} (ACE) is:
    \[ \ACE=\E{Y_i^1-Y_i^0}=\E{Y_i^1}-\E{Y_i^0}, \]
    where
    \begin{itemize}
        \item $ \E{Y_i^1} $ is the mean potential outcome had all subjects in the population
              were treated/exposed, and
        \item $ \E{Y_i^0} $ is the mean potential outcome had all subjects in the population were untreated/unexposed.
    \end{itemize}
    \tcblower{}
    If $ Y $ is binary,
    \begin{itemize}
        \item ACE is causal excess risk (omit subscript $ i $):
              \[ \E{Y^1-Y^0}=\E{Y^1}-\E{Y^0}=\Prob{Y^1=1}-\Prob{Y^0=1}. \]
        \item \textbf{Causal relative risk}:
              \[ \frac{\Prob{Y^1=1}}{\Prob{Y^0=1}}. \]
        \item \textbf{Causal odds ratio}:
              \[ \frac{\Prob{Y^1=1}/\Prob{Y^1=0}}{\Prob{Y^0=1}/\Prob{Y^0=0}}. \]
        \item \textbf{Crude excess risk}:
              \[ \Prob{Y=1\given A=1}-\Prob{Y=1\given A=0}. \]
        \item \textbf{Crude relative risk}:
              \[ \frac{\Prob{Y=1\given A=1}}{\Prob{Y=1\given A=0}}. \]
        \item \textbf{Crude odds ratio}:
              \[ \frac{\Prob{Y=1\given A=1}/\Prob{Y=0\given A=1}}{\Prob{Y=1\given A=0}/\Prob{Y=0\given A=0}}. \]
    \end{itemize}
\end{Regular}
\subsection*{A Toy Example}
\begin{Example}{}
    Assume we have a population of 8 subjects:
    \[ \begin{array}{cccc}
                  & A & Y^0 & Y^1 \\
            \midrule
            S_{1} & 0 & 0   & 1   \\
            S_{2} & 0 & 1   & 1   \\
            S_{3} & 0 & 0   & 0   \\
            S_{4} & 0 & 0   & 0   \\
            S_{5} & 1 & 0   & 0   \\
            S_{6} & 1 & 1   & 0   \\
            S_{7} & 1 & 1   & 1   \\
            S_{8} & 1 & 0   & 1   \\
            \bottomrule
        \end{array} \]
    We get
    \[ \text{Causal excess risk (ACE)} =\Prob{Y^1=1}-\Prob{Y^0=1}=\frac{4}{8}-\frac{3}{8}=\frac{1}{8}. \]
    For crude excess risk, we have
    \[ \begin{array}{ccccc}
                  & A & Y^0 & Y^1 & Y \\
            \midrule
            S_{1} & 0 & 0   & 1   & 0 \\
            S_{2} & 0 & 1   & 1   & 1 \\
            S_{3} & 0 & 0   & 0   & 0 \\
            S_{4} & 0 & 0   & 0   & 0 \\
            S_{5} & 1 & 0   & 0   & 0 \\
            S_{6} & 1 & 1   & 0   & 0 \\
            S_{7} & 1 & 1   & 1   & 1 \\
            S_{8} & 1 & 0   & 1   & 1 \\
            \bottomrule
        \end{array} \]
    \[ \text{Crude excess risk}=\Prob{Y=1\given A=1}-\Prob{Y=1\given A=0}=\frac{2}{4}-\frac{1}{4}=\frac{1}{4}. \]
\end{Example}
\subsection*{Fundamental Problem of Causal Inference}
\begin{Regular}{}
    For subject $ i $, we only get to observe one of $ Y_i^1 $ and $ Y_i^0 $, that is,
    \[ Y_i=Y_i^1A_i+Y_i^0(1-A_i). \]
    \tcblower{}
    \underline{Remarks}:
    \begin{enumerate}[(1)]
        \item In the literature, the above equality is often referred as the
              consistency assumption for causal inference
        \item For each subject $i$, one of the two potential outcomes is
              always missing.
        \item For this reason, many people believe causal inference is
              essentially a missing data problem.
    \end{enumerate}
\end{Regular}
\section{Estimation}
\begin{Regular}{}
    In \textbf{randomized studies}:
    \begin{itemize}
        \item $ \E{Y\given A=1}=\E{Y^1\given A=1}=\E{Y^1} $, and
        \item $ \E{Y\given A=0}=\E{Y^0\given A=0}=\E{Y^0} $.
    \end{itemize}
    Consequently, an unbiased estimate of ACE is:
    \begin{align*}
        \widehat{\ACE}
         & =\estE{Y^1}-\estE{Y^0}                                                                                   \\
         & =\estE{Y\given A=1}-\estE{Y\given A=0}                                                                   \\
         & =\frac{\sum_{i=1}^{n}Y_i A_i}{\sum_{i=1}^{n}A_i}-\frac{\sum_{i=1}^{n}Y_i(1-A_i)}{\sum_{i=1}^{n}(1-A_i)},
    \end{align*}
    where
    \begin{itemize}
        \item $ \sum_{i=1}^{n}A_i=n_1 $ is the number of treated/exposed subjects in the sample, and
        \item $ \sum_{i=1}^{n}(1-A_i)=n_0 $ is the number of untreated/unexposed subjects in the sample.
    \end{itemize}
\end{Regular}
\begin{Regular}{}
    In \textbf{observational studies}:
    \begin{itemize}
        \item $ \E{Y\given A=1}=\E{Y^1\given A=1}\ne\E{Y^1} $, and
        \item $ \E{Y\given A=0}=\E{Y^0\given A=0}\ne\E{Y^0} $,
    \end{itemize}
    where the inequalities are due to selection bias.
    Therefore, the estimator in randomized studies is biased for ACE in
    observational studies.
\end{Regular}
\section*{Assumptions for Causal Inference}
\section{Assumption 1}
\begin{Regular}{Assumption 1: Strongly Ignorable Treatment Assignment (SITA)}
    \[ (Y^0,Y^1)\indep (A\mid X). \]
    \tcblower{}
    \underline{Remarks}:
    \begin{itemize}
        \item In observational studies, it means $X$ includes all possible
              confounders (no unmeasured confounders).
        \item In randomized studies, we have $ (Y^0,Y^1)\indep A $.
        \item Within a subset of subjects with similar $X$, exposure/treatment
              can be viewed as if it were randomly assigned.
        \item This assumption cannot be verified on the observed data;
              more plausible as the size of $X$ grows.
        \item If violated, instrumental variable approach can be used in
              some cases.
    \end{itemize}
\end{Regular}
\section{Assumptions 2--4}
\begin{Regular}{Assumption 2: Stable Unit Treatment Value Assumption (SUTVA)}
    \[ (Y_i^0,Y_i^1)\indep A_j\text{ for $i\ne j$}. \]
    \tcblower{}
    \underline{Remarks}:
    \begin{itemize}
        \item Each subject's potential outcomes are not influenced by the
              actual treatment status of other subjects.
        \item Counter-example: infectious disease, family studies.
        \item If violated, divide the subjects into clusters.
    \end{itemize}
\end{Regular}
\begin{Regular}{Assumption 3: Common Support Condition (CSC)}
    \[ 0<\Prob{A=1\given X=x}<1\text{ for any $x$}. \]
    \tcblower{}
    \underline{Remarks}:
    \begin{itemize}
        \item It means that $ Y^0 $ and $ Y^1 $ should both exist in principle.
        \item Can be violated if a particular group of subjects in the
              population always receive the treatment or never receive the
              treatment.
        \item If violated, re-define the population (exclude those subjects).
    \end{itemize}
\end{Regular}
\begin{Regular}{Assumption 4: Consistency}
    \[ Y=Y^1 A+Y^0(1-A). \]
    \tcblower{}
    \underline{Remarks}:
    \begin{itemize}
        \item The observed outcome for a subject equals to the potential
              outcome under the actual treatment assignment the subject
              receives.
        \item Can be violated if different versions of treatment have
              different causal effects.
    \end{itemize}
\end{Regular}
\section{Propensity Scores}
\subsection*{Motivation for Propensity Scores}
The SITA assumption $(Y^0,Y^1)\indep (A\mid X)$ gives us some ideas about how to
estimate causal effects for observational studies.
\begin{itemize}
    \item If we condition on $X$, we can estimate the causal effect as in a
          randomized study, which is relatively straightforward.
    \item  However, if $X$ contains a large number of covariates,
          conditioning on $X$ is challenging (curse of dimensionality).
    \item  Solution: propensity score methods
\end{itemize}
\begin{Regular}{}
    \textbf{Propensity score} is the conditional probability of being exposed/treated given baseline covariates:
    \[ \ps{x}=\Prob{A=1\given X=x}. \]
    Also,
    \[ \ps{X}=\Prob{A=1\given X}. \]
    \tcblower{}
    \underline{Remarks}:
    \begin{itemize}
        \item In simple randomized studies, $ \ps{x}=0.5 $.
        \item In observational studies, $ \ps{x} $ is unknown and must be estimated.
    \end{itemize}
\end{Regular}
\subsection*{Properties}
\subsubsection*{Properties of Propensity Score}
\begin{itemize}
    \item Propensity score is a balancing score:
          \[ X\indep (A\mid \ps{X}) \]
    \item If the treatment is strongly ignorable given $ X $, that is,
          \[ (Y^0,Y^1)\indep (A\mid X), \]
          then it is strongly ignorable given $ \ps{x} $
          \[ (Y^0,Y^1)\indep (A\mid \ps{X}). \]
    \item $ \ps{x} $ is a scalar, free of dimension of $ X $.
    \item It is a summary of the contribution of all baseline characteristics to the exposure/treatment assignment.
\end{itemize}
\section{Properties of Propensity Score}
\begin{Result}{}
    The propensity score is a \textbf{balancing score}, that is,
    \[ X\indep(A\mid \ps{X}) \]
    \tcblower{}
    \textbf{Proof}: Rosenbaum and Rubin (1983).
    \begin{align*}
        \Prob[\big]{A=1\given \ps{X},X}
         & =\Prob{A=1\given X} &  & \text{$\ps{X}$ is a function of $X$} \\
         & =\ps{X}.
    \end{align*}
    On the other hand,
    \begin{align*}
        \Prob[\big]{A=1\given \ps{X}}
         & =\E[\big]{A\given \ps{X}}                                                  &  & \text{since $A$ is binary}    \\
         & =\E[\big]{\E{A\given \underbrace{X}_{C_1}}\given\underbrace{\ps{X}}_{C_2}} &  & \text{LIE since $C_2=f(C_1)$} \\
         & =\E[\big]{\ps{X}\given \ps{X}}                                                                                \\
         & =\ps{X}.
    \end{align*}
    Therefore,
    \[ \Prob[\big]{A=1\given \ps{X},X}=\Prob[\big]{A=1\given \ps{X}}. \]
    In other words, $ X\indep (A\mid \ps{X}) $.
\end{Result}
\begin{Result}{}
    If $ (Y^0,Y^1)\indep (A\mid X) $, then
    \[ (Y^0,Y^1)\indep(A\mid \ps{X}). \]
    \tcblower{}
    \textbf{Proof}:
    \begin{align*}
        \Prob{A=1\given Y^0,Y^1,\ps{X}}
         & =\E{A\given Y^0,Y^1,\ps{X}}                                                                 &  & \text{since $A$ is binary}      \\
         & =\E[\big]{\E{A\given \underbrace{Y^0,Y^1,X}_{C_1}}\given \underbrace{Y^0,Y^1,\ps{X}}_{C_2}} &  & \text{LIE since $C_2=f(C_1)$}   \\
         & =\E[\big]{\E{A\given X}\given Y^0,Y^1,\ps{X}}                                               &  & \text{SITA}                     \\
         & =\E[\big]{\ps{X}\given Y^0,Y^1,\ps{X}}                                                                                           \\
         & =\ps{X}                                                                                                                          \\
         & =\Prob[\big]{A=1\given \ps{X}}.                                                             &  & \text{from the previous result}
    \end{align*}
    Therefore,
    \[ (Y^0,Y^1)\indep(A\mid \ps{X}). \]
\end{Result}
\chapter{ARIMA Models Continued}
\section{Stationary Process Forecasting}
Suppose we observe a time series
$ X_1,\ldots,X_T $
that we believe has been generated by an underlying
stationary process. We would like to
produce an $ h $-step ahead
forecast
\[ \hat{X}_{T+h}=\hat{X}_{T+h\mid T}=f(X_t,\ldots,X_1) \]
forecasting $ X_{T+h} $. Ideally, $ \hat{X}_{T+h} $
would minimize the prediction error
\[ L(X_{T+h},\hat{X}_{T+h})=\min_f
    L(X_{T+h},f(X_{T},\ldots,X_1)) \]
where $ L $ is a loss function.

Frequently, the loss function is taken
to be the \emph{mean-squared error} (MSE)
\[ L(X_{T+h},\hat{X}_{T+h})=
    \E[\big]{(X_{T+h}-\hat{X}_{T+h})^2} \]
When using MSE, it is natural to consider
\[ L^2=\set{\text{Random variables } X: \E{X^2}<\infty} \]
$ L^2 $ is a Hilbert space when equipped
with the inner product
\[ \innerp{X}{Y}=\E{XY} \]
Hilbert spaces are generalizations of Euclidean space ($ \mathbf{R}^d $)
in which the geometry and notation of projection
are preserved.
\[ \text{Proj}(X\to Y)=\innerp{X}{Y}Y \]
\begin{Theorem}{Projection Theoren}{}
    We say $ M\subseteq L^2 $
    is a \textbf{closed linear subspace}, if
    \begin{itemize}
        \item Linearity: $ X,Y\in M $, $ \alpha,\beta\in\mathbf{R} $
              then $ \alpha X+\beta Y\in M $
        \item Closed: If $ X_n\to X $ ($ \E{(X_n-X)^2} $),
              and $ X_n\in M $, then $ X\in M $.
    \end{itemize}
    If $ M $ is a closed linear subspace in $ L^2 $
    and $ x\in L^2 $, then there exists a
    unique $ \hat{X}\in M $ such that
    \[ \E{(X-\hat{X})^2}=\inf_{y\in M}\E{(X-Y)^2} \]
    Moreover, $ \hat{X} $ satisfies the prediction equations/normal
    equations:
    \[ (X-\hat{X})\in M^\perp \implies \E{(X-\hat{X})Y}=0\quad (\forall y\in M) \]
\end{Theorem}
In MSE forecasting, we want to choose
$ \hat{X}_{T+h} $ satisfying
\[ \E{(X_{T+h}-\hat{X}_{T+h})^2}=\inf_{y\in M}\E{(X_{T+h}-y)^2} \]
where $ M $ is a closed linear subspace based on the available
data.
\begin{enumerate}[(1)]
    \item $ M=M_1=\set{z:z=f(X_{T},\ldots,X_{1}), f\text{ is any
                  Borel Measurable function}} $
          In this case
          \[ \hat{X}_{T+h}=\E{X_{T+h}\mid X_{T},\ldots,X_1} \]
          Unfortunately $ M_1 $ is enormous and complicated!
    \item $ M=M_2=\Span{1,X_{T},\ldots,X_1}=
              \set{Y:Y=\alpha_0+\sum_{j=1}^{T} \alpha_j X_j,\,\alpha_0,\ldots,\alpha_T\in\mathbf{R}} $
          which is the linear functions of $ X_1,\ldots,X_T $
          $ \hat{X}_{T+h} $ is called the \textbf{best linear predictor} (BLP).
\end{enumerate}

\section{The Fundamental Trigonometric Limit}
\section{Limits at infinity and Asymptotes}
\subsection{Asymptotes and Limits at Infinity}
\subsection{The Fundamental Log Limit}
\subsection{Vertical Asymptotes and Infinite Limits}
\section{Continuity}
\subsection{Types of Discontinuities}
\subsection{Continuity of Certain Functions}
\subsection{Arithmetic Rules for Continuity}
\subsection{Continuity On An Interval}
\chapter{Week 6}
\section{SARIMA Models}
Frequently, time series exhibit ``seasonality.''
\subsection*{Rough Definition of Seasonality}
A time series $ X_t $ is said to be ``seasonal''
if it exhibits regular variation so that for some lag $ s $,
$ X_t $ is ``similar'' to $ X_{t-s} $.
Some sources of seasonality are weather or scheduled events.
These typically lead to yearly, weekly, monthly, or quarterly cycles.
\begin{Remark}{}{}
    ARIMA models are not ideal for modelling seasonality.
    \[ \text{ARIMA Models}\implies\text{Random Walk with Stationary Errors} \]
    Random walks do not seasonality.
\end{Remark}
\begin{Definition}{Seasonal ARIMA}{}
    $ X_t $ is said to follow a \textbf{Seasonal ARIMA} model (SARIMA)
    of orders $ p,d,q $ and $ P,D,Q $ and seasonal period $ s $
    if
    \[ \Phi_P(B^s)\phi_p(B)(1-B^s)^D(1-B)^d Y_t=\Theta_Q(B^s)\theta_q(B)W_t \]
    We abbreviate the SARIMA $ p,d,q,P,D,Q $ model with seasonal
    period $ s $ as $ \text{SARIMA}(p,d,q)\times(P,D,Q)_s $.
    \[ \begin{array}{ccccccccc}
            \Phi_P(B)     & = & 1 & - & \Phi_1 B     & - & \cdots & - & \Phi_P B^P      \\
            \Phi_P(B^s)   & = & 1 & - & \Phi_1 B^s   & - & \cdots & - & \Phi_P B^{Ps}   \\
            \phi_p(B)     & = & 1 & - & \phi_1 B     & - & \cdots & - & \phi_p B^p      \\
            \Theta_Q(B)   & = & 1 & + & \Theta_1 B   & + & \cdots & + & \Theta_Q B^Q    \\
            \Theta_Q(B^s) & = & 1 & + & \Theta_1 B^s & + & \cdots & + & \Theta_Q B^{Qs} \\
            \theta_q(B)   & = & 1 & + & \theta_1 B   & + & \cdots & + & \theta_q B^q    \\
        \end{array} \]
\end{Definition}
\begin{Definition}{}{}
    The \textbf{seasonal} autoregressive and moving average polynomials are
    defined by
    \[ \Phi(z)=1-\Phi_1 z-\cdots \Phi_P z^P \]
    \[ \Theta(z)=1+\Theta_1 z+\cdots+\Theta_Q z^Q \]
\end{Definition}
\begin{Example}{}{}
    Let $ X_t \sim \text{SARIMA}(1,1,1)\times(1,1,1)_{13} $.
    \[ \Phi(z)=1-\Phi_1 z \]
    \[ \phi(z)=1-\phi_1 z \]
    \[ \Theta(z)=1+\Theta_1 z \]
    \[ \theta(z)=1+\theta_1 z \]
    Therefore,
    \[ (1-\Phi_1 B^{13})(1-\phi_1 B)\Uunderbracket{(1-B^{13})(1-B)X_t}_{Y_t}=\Theta(B^{13})\theta(B)W_t \]
    \[ Y_t-\Phi_1 Y_{t-13}-\phi_1 Y_{t-1}-\phi_1 \Phi_1 Y_{t-14}=\text{MA term} \]
    \[ Y_t=f(Y_{t-13},Y_{t-1},\text{MA noise}, Y_{t-14}) \]
    where $ Y_{t-13} $ is the seasonal lag.
\end{Example}
\begin{Remark}{}{}
    \begin{enumerate}[(1)]
        \item $ Y_t=(1-B^s)^D(1-B)^d X_t $, a SARIMA model is just one big ARMA
              model for $ Y_t $.
        \item Advantage over ARMA and ARIMA models is \textbf{parsimony}.
              Since seasonal series have the feature that $ X_t $ is similar to $ X_{t-s} $,
              we introduce just a few additional terms to model $ X_t $ as a function of
              $ X_{t-s} $.
    \end{enumerate}
\end{Remark}
\subsection*{Fitting SARIMA Models}
\begin{enumerate}[(1)]
    \item Usually the seasonal lag $ s $ is known.
    \item Differencing and seasonal differencing can be decided upon by:
          \begin{enumerate}[(a)]
              \item Eye-ball test and/or examining the ACF and PACF\@.
              \item Stationarity tests.
              \item Cross-validation.
          \end{enumerate}
          {\color{blue}We will discuss (b) and (c).}
    \item Choosing the order and estimating the components of $ \Phi,\phi,\Theta,\theta $
          can be done in the same was as with ARMA models.
\end{enumerate}
\section{SARIMA Cardiovascular Mortality Example}
\href{https://github.com/Hextical/university-notes/blob/master/year-3/semester-2/STAT 443/code/6.2 - SARIMA Cmort Example.R}{[R Code] SARIMA Cardiovascular Mortality Example}
\section{Time Series Cross-Validation}
\begin{Definition}{Cross-validation}{}
    \textbf{Cross-validation} is a data driven model evaluation
    and selection tool for predictive models that entails the following.
    \begin{enumerate}[(1)]
        \item Splitting the available data into training and testing sets.
        \item Fitting models on the training sets.
        \item Evaluating predictions of the model on the tests sets as an overall
              evaluation of model quality.
    \end{enumerate}
\end{Definition}
\subsection*{Standard Cross-Validation}
Suppose $ (Y_i,X_i) $ for $ 1\le i\le n $ satisfy $ Y_i=f(X_i)+\varepsilon_i $.
Let $ M $ be a model used to estimate $ f $ using $ \hat{f} $,
with the goal of minimizing $ L(Y_i,\hat{f}(X_i)) $.
\subsection*{$ K $-fold Cross-Validation}
\begin{enumerate}[(1)]
    \item Split $ (Y_i,X_i) $ for $ 1\le i\le n $ randomly into $ K $-groups
          $ G_1,\ldots,G_k $.
    \item For each $ 1\le i\le K $, use $ M $ to estimate $ \hat{f}^{(-j)} $
          when the data $ G_i $ is left out.
    \item Evaluate error on $ G_i $ with
          \[ \text{CV}_j=\sum_{(Y_i,X_i)\in G_j}L(Y_i,\hat{f}^{-j}(X_i))  \]
    \item The total cross-validation error of the model is:
          \[ \text{CV}(M)=\sum_{j=1}^{k} \text{CV}_j \]
\end{enumerate}
\begin{Remark}{}{}
    \begin{itemize}
        \item $ K $ is often called the number of \textbf{folds}.
        \item If $ K=n $, the procedure is often called the ``leave-one-out''
              cross-validation.
        \item $ K=10 $ is called ``10-fold cross validation.''
    \end{itemize}
\end{Remark}
\subsection*{Problems with Time Series Cross-Validation}
\begin{enumerate}[(1)]
    \item Randomly splitting the data scrambles up any serial dependence relationships.
    \item In time series forecasting, it is often most natural to use
          the past (recent past) to predict future values.
\end{enumerate}
\subsection*{Time Series Cross-Validation Algorithm}
\begin{enumerate}[(1)]
    \item Split the data into training and testing ranges $ 1\le t_r\le T $
          where $ t_r\approx 0.75T $ is $ 75\% $ of the training sample.
          The test sample is $ X_{t_r+1},\ldots,X_T $.
    \item For each $ j $ in $ t_r+1,\ldots,T $, use model to forecast
          $ \hat{X}_{j+1\mid j} $ based on $ X_1,\ldots,X_j $. Calculate loss
          \[ L(\hat{X}_{j+1\mid j};X_{j+1})=L_j \]
    \item Cross-validation score of model
          \[ \text{CV}(M)=\sum_{j=t_r+1}^{T} L_j \]
\end{enumerate}
\begin{Remark}{}{}
    \begin{enumerate}[(1)]
        \item If interested in longer horizon forecasting, you can compare
              \[ \hat{X}_{j+1\mid j},\ldots,\hat{X}_{j+h\mid j}\quad\text{to}\quad X_{j+1},\ldots,X_{j+h} \]
              in the loss calculation step.
        \item Stationarity is \emph{crucial} in time series cross validation
              since the model errors in the present must be similar to errors in the future.
        \item One normally cannot cross-validate everything as this is computationally
              infeasible.
    \end{enumerate}
\end{Remark}
\section{Cross-Validation Example}
\href{https://github.com/Hextical/university-notes/blob/master/year-3/semester-2/STAT 443/code/6.4 - Cross-Validation Example.R}{[R Code] Cross-Validation Example} % chktex 15
\section{Simulated and Bootstrapped Prediction Intervals}
Usually forecasts are of the form
\[ \hat{X}_{T+1\mid T}=g(X_T,X_{T-1},\ldots,X_1,W_{T+1}) \]
where $ W_{T+1} $ is a strong white noise innovation.

Often, even models are additive so that
\[ \hat{X}_{T+1\mid T}=g(X_T,\ldots,X_1)+W_{T+1} \]
Simple and powerful models to produce prediction intervals
use simulation!
\subsection*{Simulated Prediction Intervals}
\begin{enumerate}[(1)]
    \item Choose a distribution for $ \set{W_t} $. A common choice is $ W_t \sim \N{0,\hat{\sigma}_W^2} $.
    \item For $ b=1,\ldots,B $ where $ B $ is a large number, simulate $ \set*{W_{T+1}^{(b)}} $.
    \item Compute $ \hat{X}_{T+1\mid T}^{(b)}=g(X_{T},\ldots,X_1)+W_{T+1}^{(b)} $
          for $ b=1,\ldots,B $.
    \item Denote the empirical $ q^{\text{th}} $ quantile of $ \set*{\hat{X}_{T+1}^{(b)}:b=1,\ldots,B} $
          by $ \hat{Q}_{T+1}(q) $. We set the $ (1-\alpha) $ prediction interval as
          \[ \biggl(\hat{Q}_{T+1}\biggl(\frac{\alpha}{2} \biggr),\hat{Q}_{T+1}\biggl(1-\frac{\alpha}{2} \biggr)\biggr) \]
\end{enumerate}
\begin{Remark}{}{}
    For longer horizon forecasts, prediction intervals can be obtained by iteration:
    \[ \hat{X}_{T+h\mid T}^{(b)}=g(\hat{X}_{T+h-1\mid T}^{(b)},\ldots,\hat{X}_{T+1\mid T}^{(b)},X_T,\ldots,X_1)+W_{T+h}^{(b)} \]
    The prediction interval is
    \[ \biggl(\hat{Q}_{T+h}\biggl(\frac{\alpha}{2} \biggr),\hat{Q}_{T+h}\biggl(1-\frac{\alpha}{2} \biggr)\biggr) \]
    where $ \hat{Q}_{T+h}(q) $ the empirical $ q^{\text{th}} $ quantile of $ \hat{X}_{T+h}^{(b)} $.
\end{Remark}
\subsection*{Distributions to Choose for $ W_t $}
\begin{enumerate}[(1)]
    \item $ W_t \sim \N{0,\hat{\sigma}_W^2} $ where $ \hat{\sigma}_W^2 $ is estimated from residuals which
          leads to approximately the same ``well known'' prediction intervals.
    \item A distribution fit to the estimated residuals $ \hat{W}_t $; e.g., a $ t $-distribution,
          Pareto, etc.
    \item The empirical distribution of the residuals $ \hat{W}_t $; that is,
          randomly drawing $ \set*{\hat{W}_1,\ldots,\hat{W}_{T}} $ which is commonly known
          as \textbf{bootstrapping}.

          \underline{Note}: An important consideration of the bootstrap is that the residuals should be white!
          We can check the whiteness of the residuals using the ACF or a white noise test.
\end{enumerate}
\section{Bootstrap Prediction Intervals Example}
\href{https://github.com/Hextical/university-notes/blob/master/year-3/semester-2/STAT 443/code/6.6 - Bootstrap Prediction Intervals Example.R}{[R Code] Bootstrap Prediction Intervals Example}

\chapter{Week 7}
\section{Exponential Smoothing Models Introduction}
\begin{itemize}
    \item \textbf{ARIMA Models}: Model a time series, potentially after differencing towards
          stationarity, in terms of its autocorrelation (linear process).
    \item \textbf{Exponential Smoothing}: Flexibly model the trend and seasonality
          observed in a time series.
\end{itemize}
\subsection*{Simple Exponential Smoothing}
Suppose we wish to forecast a time series $ X_1,\ldots,X_T $. Two extreme
forecasts are
\[ \hat{X}_{T+1\mid T}=X_T\quad\text{[Random Walk]} \]
\[ \hat{X}_{T+1\mid T}=\bar{X}=\frac{1}{T} \sum_{t=1}^{T} X_t \quad\text{[IID Sequence]} \]
\underline{Compromise}: \emph{Exponential Smoothing}.
\[ \hat{X}_{T+1\mid T}=\alpha X_T+\alpha(1-\alpha)X_{T-1}+\cdots+\alpha(1-\alpha)^{T-1}X_1 \]
where $ \alpha\in[0,1] $ is the \textbf{smoothing parameter}.

Weights applied to past observations decrease exponentially quickly.

Simple exponential smoothing can be stated as a recursive system of equations.
\begin{itemize}
    \item Prediction Equation: $ \hat{X}_{T+1}=\ell_T $.
    \item Smoothing/Level Equation: $ \ell_T=\alpha X_T+(1-\alpha)\ell_{T-1}=\ell_T(\alpha,\ell_\sigma) $
          which is a convex combination of last observed value and last prediction or ``level.''
    \item Initial Condition: $ \ell_0 $.
    \item Parameters Defining Model are $ \alpha\in[0,1] $ and $ \ell_0 $.
\end{itemize}
Estimation may be conducted using MLE (later) or OLS\@. For OLS,
\[ (\hat{\alpha},\hat{\ell}_0)=\argmin_{0\le \alpha\le 1,\,\ell_0\in\mathbf{R}}
    \sum_{i=2}^{T} \bigl[X_i-\ell_i(\alpha,\ell_0)\bigr]^2 \]
\[ \hat{X}_{T+1}=\hat{\alpha}X_T+(1-\hat{\alpha})\ell_T(\hat{\alpha}.\hat{\ell}_0) \]
which can be calculated by iterating the level equation back to $ \ell_0 $.

\subsection*{Linear Trend Exponential Smoothing}
\begin{itemize}
    \item Prediction Equation: $ \hat{X}_{T+h}=\ell_T+h b_T $ where $ \ell_T $
          is the \textbf{level} and $ b_T $ is the \textbf{slope}.
    \item Level Equation: $ \ell_T=\alpha X_T+(1-\alpha)(\ell_{T-1}+b_{T-1}) $
          which is the convex combination of last observation and last ``level''
          or prediction.
    \item Trend/Slope Equation: $ b_T=\beta(\ell_T-\ell_{T-1})+(1-\beta)b_{T-1} $
          where $ \ell_T-\ell_{T-1} $ is the last ``observed'' slope or change in level.
    \item Initial Conditions: $ \ell_0 $ and $ b_0 $.
    \item Parameters: $ \alpha,\beta\in[0,1] $, $ \ell_0,\beta_0\in\mathbf{R} $
          which are estimated using MLE/OLS\@.
\end{itemize}

\subsection*{Trend + Seasonal Exponential Smoothing (Holt Winters ES, 1960s)}
Suppose $ h $ is the forecast horizon of interest and time series has seasonal
period $ p $. Set $ k=\lfloor (h-1)/p\rfloor $.
\begin{itemize}
    \item Prediction Equation: $ \hat{X}_{T+1}=\ell_T+h b_T+s_{T+h-p(k+1)} $
          where $ \ell_T $ is the level, $ b_T $ is the slope, and
          $ s_{T+1-p(k+1)} $ is the seasonal effect.
    \item Level Equation: $ \ell_T=\alpha(X_T-s_{T-p})+(1-\alpha)(\ell_{T-1}+b_{T-1}) $.
    \item Slope Equation: $ b_T=\beta(\ell_T-\ell_{T-1})+(1-\beta)b_{T-1} $.
    \item Seasonal Equation: $ s_T=\gamma(X_T-\ell_{T-1}-b_{T-1})+(1-\gamma)s_{T-p} $.
    \item Initial Conditions: $ \ell_0,\beta_0,s_0,\ldots,s_{-p+1} $.
    \item Parameters: $ \alpha,\beta,\gamma\in[0,1] $, $ \ell_0,\beta_0,s_0,\ldots,s_{-p+1}\in\mathbf{R} $.
\end{itemize}

\section{Exponential Smoothing as a State Space Model}
Consider Simple Exponential Smoothing:
\begin{itemize}
    \item Prediction Equation: $ \hat{X}_{t\mid t-1}=\ell_{t-1} $.
    \item Level Equation: $ \ell_t=\alpha X_t+(1-\alpha)\ell_{t-1} $.
\end{itemize}
Re-arranging the level equation gives
\[ \ell_t=\ell_{t-1}+\alpha(\Uunderbracket{X_t-\ell_{t-1}}_{\text{residual $ \varepsilon_t $}})=\ell_{t-1}+\alpha \varepsilon_t \]
Also, $ X_t=\ell_{t-1}+\varepsilon_t $. Therefore, these equations
can be reformulated as:
\begin{itemize}
    \item Prediction Equation: $ X_t=\ell_{t-1}+\varepsilon_t $.
    \item Level Equation: $ \ell_t=\ell_{t-1}+\alpha \varepsilon_t $.
\end{itemize}
\underline{Why is this useful?} If we make a parametric assumption
on $ \varepsilon_t $ (e.g., $ \varepsilon_t \sim \N{0,\sigma_{\varepsilon}^2} $),
then we can use Likelihood techniques (MLE, AIC, simulation based Prediction Intervals).

Such equations are examples of ``State Space'' Models:
\begin{Definition}{State space model}{}
    We say $ X_T $ follows a general \textbf{state space model}
    if:
    \begin{itemize}
        \item Observation Equation: $ X_t= A_t Y_t+\varepsilon_t $
              where $ A_t $ is the \textbf{measurement matrix}, $ Y_t $ is the \textbf{state vector} (unobserved),
              and $ \varepsilon_t $ is an \textbf{observation error}.
        \item State Equation: $ Y_t=\phi Y_{t-1}+W_t $.
    \end{itemize}
    \begin{center}
        \begin{tikzpicture}[
                roundnode/.style={circle, draw=green!60, fill=green!5, very thick, minimum size=15mm},
                roundnodeb/.style={circle, draw=blue!60, fill=blue!5, very thick, minimum size=15mm},
                roundnodes/.style={circle, draw=blue!60, fill=blue!5, very thick, minimum size=5mm},
            ]
            %Nodes
            \node[roundnode]       (maintopic)                              {$Y_{t-1}$};
            \node[roundnode]        (rightcircle)        [right=of maintopic] {$Y_{t}$};
            \node[roundnode]      (rrightcircle)       [right=of rightcircle] {$Y_{t+1}$};
            \node[roundnodeb]        (lowercircle)       [below=of maintopic] {$X_{t-1}$};
            \node[roundnodeb]        (lowerrcircle)       [right=of lowercircle] {$X_{t}$};
            \node[roundnodeb]        (lowerrrcircle)       [below=of rrightcircle] {$X_{t+1}$};
            \node[roundnodes]        (slowercircle)       [left=of lowercircle] {$A_{t-1}$};
            \node[roundnodes]        (slowerrcircle)       [below=of lowerrcircle] {$A_{t}$};
            \node[roundnodes]        (slowerrrcircle)       [right=of lowerrrcircle] {$A_{t+1}$};


            %Lines
            \draw[->] (maintopic.south) -- (lowercircle.north);
            \draw[->] (maintopic.east) -- (rightcircle.west);
            \draw[->] (rightcircle.south) -- (lowerrcircle.north);
            \draw[->] (rightcircle.east) -- (rrightcircle.west);
            \draw[->] (rrightcircle.south) -- (lowerrrcircle.north);
            \draw[->] (slowerrcircle.north) -- (lowerrcircle.south);
            \draw[->] (slowercircle.east) -- (lowercircle.west);
            \draw[->] (slowerrrcircle.west) -- (lowerrrcircle.east);
        \end{tikzpicture}
    \end{center}

    $ \varepsilon_t $ and $ W_t $ are white noise terms that may depend on each other.
\end{Definition}
\begin{Example}{State Space Models}{}
    \begin{itemize}
        \item $ \AR{1} $: $ X_t=Y_t $ where $ Y_t=\phi Y_{t-1}+W_t $
              where $ W_t \sim  $ strong white noise.
        \item Simple Exponential Smoothing:
              \[ X_t=Y_{t-1}+\varepsilon_t \]
              \[ Y_t=Y_{t-1}+\alpha \varepsilon_t \]
              where $ \varepsilon_t \sim  $ strong white noise.
    \end{itemize}
    All ARMA and Exponential Smoothing models can be written in state-space form.
\end{Example}
\subsection*{Parameter Estimation and Model Selection using State-Space Formulation}
\begin{itemize}
    \item $ X_t=\ell_{t-1}+\varepsilon_t $.
    \item $ \ell_t=\ell_{t-1}+\alpha\varepsilon_t $.
    \item $ \varepsilon_t \sim \N{0,\sigma_\varepsilon^2} $.
    \item Initial Condition: $ \ell_0 $.
\end{itemize}
\[ \mathcal{L}(X_1,\ldots,X_T;\alpha,\ell_0,\sigma_\varepsilon^2)=
    \prod_{i=1}^T \Uunderbracket{\mathcal{L}(X_i\mid X_{i-1},\ldots,X_1;\alpha,\ell_0,\sigma_\varepsilon^2)}_{
    \N*{\ell_{i-1}(\alpha,\ell_0),\sigma_\varepsilon^2}
    } \]
Likelihood can be maximized numerically, and we use this to calculate AIC/BIC\@.

\section{Multiplicative Exponential Smoothing Models}
Standard Exponential Smoothing has ``additive'' errors, in the sense that
\[ X_t=\ell_{t-1}+\varepsilon_t \]
\[ \ell_t=\alpha X_t+(1-\alpha)\ell_{t-1} \]
Therefore, $ \varepsilon_t=X_t-\ell_{t-1} $.

We can also formulate exponential smoothing in terms of ``multiplicative''
errors, in the sense that
\[ \varepsilon_t=\frac{X_{t-1}-\ell_{t-1}}{\ell_{t-1}}  \]
where we note that the error is relative to the previous level. Therefore,
\[ X_t=\ell_{t-1}(1+\varepsilon_t) \]
\[ \ell_t=\alpha X_t+(1-\alpha)\ell_{t-1} =\alpha\varepsilon_t\ell_{t-1}+\alpha\ell_{t-1}+(1-\alpha)\ell_{t-1}
    =\ell_{t-1}(1+\alpha\varepsilon_t) \]
\underline{Why consider multiplicative errors?} It is important
to note that since the level follows the same exponential smoothing equation,
the forecasts from multiplicative and additive error models will be the same.
The difference arises from how uncertainty/error propagates in the model.
\begin{itemize}
    \item Additive: $ \hat{X}_{T+1}=\ell_T+\sum_{j=T+1}^{T+h} \varepsilon_j $
          where we note that the MSE scales like $ h $.
    \item Multiplicative: $ \hat{X}_{T+h}=\ell_T \prod_{j=T+1}^{T+h}(1+\varepsilon_j) $
          where we note that the MSE (variance) is scaling like
          \[ \Bigl(\E[\big]{(1+\varepsilon_0)^2}\Bigr)^h \]
          which could grow very quickly as $ h\to\infty $.
\end{itemize}
\subsection*{Multiplicative Linear + Trend and Holt Winters}
Linear + Trend State Space Formulation:
\[ \varepsilon_t=\frac{X_t-(\ell_{t-1}+b_{t-1})}{\ell_{t-1}+b_{t-1}}  \]
\[ X_t=(\ell_{t-1}+b_{t-1})(1+\varepsilon_t) \]
\[ \ell_t=(\ell_{t-1}+b_{t-1})(1+\alpha\varepsilon_t) \]
\[ b_t=b_{t-1}+\beta(\ell_{t-1}+b_{t-1})\varepsilon_t \]
where $ \varepsilon_t \sim \N{0,\sigma_{\varepsilon}^2} $.
\subsection*{Multiplicative Seasonal Exponential Smoothing}
Let $ p $ be the seasonal period.
\[ X_t=(\ell_{t-1}+b_{t-1})s_{t-p}(1+\varepsilon_t) \]
\[ \ell_t=(\ell_{t-1}+b_{t-1})(1+\alpha\varepsilon_t) \]
\[ b_t=b_{t-1}+\beta(\ell_{t-1}+b_{t-1})\varepsilon_t \]
\[ s_t=s_{t-p}(1+\gamma\varepsilon_t) \]
\subsection*{When to use Additive versus Multiplicative}
Seasonal Exponential Smoothing Models:
\begin{enumerate}[(1)]
    \item Multiplicative models imply that as the level increases (decreases)
          the seasonal fluctuations increase (decrease). Additive models suggest
          seasonal fluctuations remain constant as trend fluctuations.
          \[ \text{Seasonal Fluctuations}\uparrow\text{ as }\text{Level}\uparrow\implies
              \text{ Multiplicative}. \]
    \item Use AIC/BIC\@: The AIC can be evaluated for each state-space
          model and compared.
\end{enumerate}
\section{Exponential Smoothing Model Selection}
Given the state-space formulation of exponential smoothing and the use of MLE to estimate
the parameters, it is common to use AIC to choose among competing
Exponential Smoothing (including additive versus multiplicative) models. Other
options include:
\begin{itemize}
    \item Cross-validation.
    \item Residual Analysis (white noise testing).
\end{itemize}
\subsection*{Prediction Intervals}
Using the state-space formulation, valid prediction intervals
may be computed using simulation.
\begin{Example}{Simple Exponential Smoothing}{}
    \[ \hat{X}_{T+1\mid T}=\hat{\ell}_T \]
    State-space formula:
    \[ \hat{X}_{T+1}\cong\hat{\ell}_T+\Uunderbracket{\varepsilon_{T+1}}_{\N{0,\sigma_\varepsilon^2}} \]
    \begin{enumerate}[(1)]
        \item Estimate
              \[ \hat{\sigma}_{\varepsilon}^2=\frac{1}{T-1} \sum_{j=2}^{T} (X_j-\hat{\ell}_{T-1})^2 \]
        \item Simulate
              \[ \hat{X}_{T+1\mid T}^{(b)}=\hat{\ell}_T+\Uunderbracket{\varepsilon_{T+1}^{(b)}}_{\N{0,\hat{\sigma}_\varepsilon^2}} \]
        \item Use $ 5\% $ and $ 95\% $ sample quantiles of $ X_{T+1\mid T}^{(b)} $, $ b=1,\ldots,B $
              as prediction intervals.
    \end{enumerate}
\end{Example}
\begin{Remark}{}{}
    In many cases, the prediction MSE assuming $ \varepsilon_t \sim \N{0,\sigma_{\varepsilon}^2} $
    can be computed explicitly. See $ \S $ 7.7 of HA\@.
\end{Remark}
An important consideration in applying this approach is that $ \varepsilon_t $
should behave like Gaussian white noise. We can check this using a residual analysis.
\begin{itemize}
    \item White noise tests, ACF plots.
    \item Quantile-Quantile plot for Normality.
\end{itemize}
\section{J and J Exponential Smoothing Forecast}
\href{https://github.com/Hextical/university-notes/blob/master/year-3/semester-2/STAT 443/code/7.5 - J and J Exponential Smoothing Forecast.R}{[R Code] J and J Exponential Smoothing Forecast}

\chapter{Week 8}
\section{Neural Network Autoregression}
Simple Neural Network ``Architecture:''
\begin{itemize}
    \item Input layer (covariates/predictors).
    \item Hidden layer (neurons).
    \item Output layer.
\end{itemize}
It's possible to have several hidden layers and multiple
neurons at each layer.

Any particular layer, the inputs are mapped to the $ j^{\text{th}} $
neuron linearly. The value taken on the $ j^{\text{th}} $
neuron is
\[ z_j=b_j+\sum_{i=1}^{4} w_{i,j}x_{i} \]
where $ b_j $ is a function, $ x_i $ is the $ i^{\text{th}} $ input, and $ w_{i,j} $
are the weights.

To calculate the inputs to the next layer, a non-linear transformation
is applied. For example, using the sigmoid function:
\[ S(z)=\frac{1}{1+e^{-z}} \]
The final model is a complex non-linear function of the inputs.

\subsection*{Neural Network AR}
\begin{itemize}
    \item Input layer: $ X_t,\ldots,X_{t-p} $.
    \item Output layer: $ X_{t+1} $.
\end{itemize}
A neural network model with $ k $ hidden states (assuming one hidden layer)
we call a $ \NNAR{p,k} $ model.
\begin{Remark}{}{}
    If $ k=0 $, then $ \NNAR{p}=\AR{p} $. The inputs are mapped
    linearly to the outputs.
\end{Remark}
\subsection*{Seasonal Neural Network AR}
\begin{itemize}
    \item Input layer: $ X_{t},\ldots,X_{t-p},X_{t-m},X_{t-P_m} $
          where $ m $ is the seasonal lag.
    \item Output layer: $ X_{t+1} $.
\end{itemize}
We call this a $ \NNSAR{p,k,P}{m} $ model.

The model selection of choosing $ k $, $ p $, and $ P $
can be carried out using cross-validation where the weights are estimated using ordinary
least squares.

\subsection*{Prediction Intervals}
If $ \symbf{X}_t=(X_t,\ldots,X_{t-p},X_{t-m},\ldots,X_{t-P_m})^\top $
denotes the vector of predictors, then we can posit an additive stochastic model for
$ X_{t+1} $ as
\[ X_{t+1}=f(\symbf{X}_t)+\varepsilon_{t+1} \]
where $ f $ is the neural network.

By calculating the residuals
$ \hat{\varepsilon}_t=X_t-\hat{f}(\symbf{X}_t) $,
prediction intervals can be estimated using the bootstrap
\[ X_{T+1}^{(b)}=\hat{f}(\symbf{X}_T)+\hat{\varepsilon}_{T+1}^{(b)}\quad(b=1,\ldots,B) \]
We can then construct a prediction interval by using
the empirical quantiles from the simulated distribution of the
forecast $ 1 $-step ahead. This process can be iterated multiple
times to produce forecasts as well as prediction intervals for
forecasts at longer time horizons.

\section{Comparing Various Forecasting Methods}

\section{Conditional Heteroscedasticity}

\section{ARCH and GARCH Models}

\section{Stationarity of GARCH Models}

\section{\texorpdfstring{$ \dagger $}{†} Stationarity of General \texorpdfstring{$ \GARCH{p,q} $}{GARCH(𝑝, 𝑞)}}

\section{Identifying GARCH Models}

%\nocite{*}
%\printbibliography[heading=bibintoc,title={References}]

\end{document}
