\chapter{Real Limits, Continuity and Differentiation}
\section{Order Properties in \texorpdfstring{$ \mathbf{R} $}{R}}
\begin{Theorem}{Discreteness Property of $ \mathbf{Z} $}{discrete_Z}
    We state two equivalent definitions.
    \[ \forall k\in\mathbf{Z}\;\forall n\in\mathbf{Z}\;(k\le n \iff k<n+1) \]
    \[ \forall n\in\mathbf{Z}\;\nexists k\in\mathbf{Z}\;(n<k<n+1) \]
\end{Theorem}
\begin{Proof}{\Cref{thm:discrete_Z}}{}
    Accepted axiomatically, without proof.
\end{Proof}
\begin{Definition}{Bounded above, Upper bound}{}
    $ A $ is \textbf{bounded above} (in $ \mathbf{R} $)
    when
    \[ \exists b\in\mathbf{R}\;\forall x\in A\;(x\leq b) \]
    We say that $ b $ is an \textbf{upper bound} for $ A $.
\end{Definition}
\begin{Definition}{Bounded below, Lower bound}{}
    $ A $ is \textbf{bounded below} (in $ \mathbf{R} $) when
    \[ \exists a\in\mathbf{R}\;\forall x\in A\;(a\le x) \]
    We say that $ a $ is a \textbf{lower bound} for $ A $.
\end{Definition}
\begin{Definition}{Bounded}{}
    $ A $ is \textbf{bounded} when $ A $ is both bounded
    above and below.
\end{Definition}
\begin{Definition}{Supremum, Least upper bound, Maximum element}{}
    $ A $ has a \textbf{supremum} (or a \textbf{greatest lower bound})
    when there exists an element $ b\in \mathbf{R} $ such that
    $ b $ is an upper bound for $ A $ with $ b\le c $ for
    every upper bound $ c\in\mathbf{R} $ for $ A $. In this case,
    we say $ b $ is the \textbf{supremum} (or the \textbf{greatest lower bound})
    of $ A $ and write $ b=\Supremum{A} $. When $ b=\Supremum{A}\in A $
    we also say that $ b $ is the \textbf{maximum element} of $ A $,
    and we write $ b=\Maximum{A} $.
\end{Definition}
\begin{Definition}{Infimium, Greatest lower bound, Minimum element}{}
    $ A $ has an \textbf{infimum} (or a \textbf{greatest lower bound})
    when there exists an element $ a\in \mathbf{R} $ such that
    $ a $ is a lower bound for $ A $ with $ c\le a $ for
    every lower bound $ c $ for $ A $. In this case,
    we say $ a $ is the \textbf{infimum} (or the \textbf{greatest lower bound})
    of $ A $ and write $ a=\Infimum{A} $. When $ a=\Infimum{A}\in A $
    we also say that $ a $ is the \textbf{minimum element} of $ A $,
    and we write $ a=\Minimum{A} $.
\end{Definition}
\begin{Example}{}{}
    Let $ A=\mathbf{R}_{>0}=(0,\infty)=\Set{x\in\mathbf{R}\given x>0} $
    and $ B=\interval[open right]{1}{\sqrt{2}}
        =\Set{x\in\mathbf{R}\given 1\le x<\sqrt{2}} $.
    \begin{itemize}
        \item $ A $ is bounded below, but not above.
        \item $ -1 $ and $ 0 $ are both lower bounds for $ A $.
        \item $ \Infimum{A}=0 $
        \item $ A $ has no minimum element, and no maximum element.
        \item $ B $ is bounded both above and below.
        \item $ 0 $ and $ 1 $ are both lower bounds for $ B $
        \item $ \sqrt{2} $ and $ 3 $ are both upper bounds for $ B $.
        \item $ \Infimum{B}=1 $
        \item $ \Supremum{B}=\sqrt{2} $
        \item $ B $ has a minimum element, namely $ \Minimum{B}=1 $,
              but has no maximum element.
    \end{itemize}
\end{Example}
\begin{Theorem}{The Supremum and Infemum Properties of $ \mathbf{R} $}{sup_inf_properties}
    \begin{enumerate}[label=(\arabic*)]
        \item\label{sup_inf_properties_1}Every non-empty subset of $ \mathbf{R} $
              which is bounded above in $ \mathbf{R} $ has a supremum in $ \mathbf{R} $.
        \item\label{sup_inf_properties_2}Every non-empty subset of $ \mathbf{R} $
              which is bounded below in $ \mathbf{R} $ has an infimum in $ \mathbf{R} $.
    \end{enumerate}
\end{Theorem}
\begin{Proof}{\Cref{thm:sup_inf_properties}}{}
    Accepted axiomatically, without proof.
\end{Proof}
\begin{Theorem}{Approximation Property of Supremum and Infimum}{approx_sup_inf}
    Let $ \emptyset\neq A\in\mathbf{R} $.
    \begin{enumerate}[label=(\arabic*)]
        \item\label{approx_sup_inf_1}$ b=\Supremum{A}\implies \forall \varepsilon\in\mathbf{R}_{>0}\;
                  \exists x\in A\;(b-\varepsilon<x\le b) $
        \item\label{approx_sup_inf_2} $ a=\Infimum{A}\implies \forall \varepsilon\in\mathbf{R}_{>0}\;
                  \exists x\in A\;(a\le x<a+\varepsilon) $
    \end{enumerate}
\end{Theorem}
\begin{Proof}{\Cref{thm:approx_sup_inf}}{}
    We prove~\ref{approx_sup_inf_1}. Let $ b=\Supremum{A} $
    and $ \varepsilon>0 $. Suppose for a contradiction that there exists no
    element $ x\in A $ with $ b-\varepsilon<x $, or equivalently
    that for all $ x\in A $ we have $ b-\varepsilon\ge x $.
    Let $ c=b-\varepsilon $. Note that $ c $ is an upper bound
    for $ A $ since $ x\le b-\varepsilon=c $ for all $ x\in A $. Then,
    since $ b=\Supremum{A} $ and $ c $ is an upper bound for
    $ A $, we have $ b\le c $. However, since $ \varepsilon>0 $
    we have $ b>b-\varepsilon=c $, contradiction. Therefore,
    there exists $ x\in A $ with $ b-\varepsilon<x $. Now,
    choose an element $ x\in A $. Then, since $ b=\Supremum{A} $,
    we know that $ b $ is an upper bound for $ A $ and hence $ b\ge x $.
    Therefore, $ b-\varepsilon<x\le b $, as required.
\end{Proof}
\begin{Theorem}{Well-Ordering Properties of $ \mathbf{Z} $ in $ \mathbf{R} $}{well_ord_zinr}
    \begin{enumerate}[label=(\arabic*)]
        \item\label{well_ord_zinr_1}Every non-empty subset of $ \mathbf{Z} $ which is bounded
              above in $ \mathbf{R} $ has a maximum element.
        \item Every non-empty subset of $ \mathbf{Z} $ which is bounded
              below in $ \mathbf{R} $ has a minimum element.
    \end{enumerate}
\end{Theorem}
\begin{Proof}{\Cref{thm:well_ord_zinr}}{}
    We prove~\ref{well_ord_zinr_1}. Let $ A $ be a non-empty subset of
    $ \mathbf{Z} $ which is bounded above.
    By~\Cref{thm:sup_inf_properties}~\ref{sup_inf_properties_1},
    $ A $ has a supremum in $ \mathbf{R} $. Let $ n=\Supremum{A} $.
    We must show that $ n\in A $. Suppose for a contradiction
    that $ n\notin A $. By~\Cref{thm:approx_sup_inf} (using $ \varepsilon=1 $),
    we can choose $ a\in A $ with $ n-1<a\le n $. Note that $ a\neq n $
    since $ a\in A $ and $ n\notin A $, so we have $ a<n $. By~\Cref{thm:approx_sup_inf}
    (using $ \varepsilon=n-a $) we can choose $ b\in A $ with
    $ a<b\le n $. Since $ a<b $ we have $ b-a>0 $. Since $ n-1<a $
    and $ b\le n $, we have $ 1=n-(n-1)>b-a $. However,
    we have $ (b-a)\in\mathbf{Z} $ with $ 0<b-a<1 $,
    which contradicts~\Cref{thm:discrete_Z}. Therefore, $ n\in A $,
    and hence $ A $ has a maximum element.
\end{Proof}
\begin{Theorem}{Floor and Ceiling Properties of $ \mathbf{Z} $ in $ \mathbf{R} $}{floor_ceil_zinr}
    \begin{enumerate}[label=(\arabic*)]
        \item\label{floor_ceil_zinr_1}$ \forall x\in\mathbf{R}\;\exists! n\in\mathbf{Z}\;(x-1<n\le x) $.
        \item\label{floor_ceil_zinr_2}$ \forall x\in\mathbf{R}\;\exists! m\in\mathbf{Z}\;(x\le m<x+1) $.
    \end{enumerate}
\end{Theorem}
\begin{Proof}{\Cref{thm:floor_ceil_zinr}}{}
    We prove~\ref{floor_ceil_zinr_1}.

    \emph{\textbf{Uniqueness}}. Let $ x\in\mathbf{R} $,
    suppose $ n,m\in\mathbf{Z} $ with $ x-1<n\le x $ and
    $ x-1<m\le x $. Since $ x-1<n $ we have $ x<n+1 $.
    Since $ m\le x $ and $ x<n+1 $, we have $ m<n+1 $,
    hence $ m\le n $ by~\Cref{thm:discrete_Z}. Similarly,
    $ n\le m $. Since $ n\le m $ and $ m\le n $, we have $ n=m $
    as required.

    \emph{\textbf{Existence}}. Let $ x\in\mathbf{R} $. First, let us
    consider the case that $ x\ge 0 $. Let $ A=\Set{k\in\mathbf{Z}\given k\le x} $.
    Note that $ A\ne \emptyset $ (because $ 0\in A $), and $ A $ is bounded
    above by $ x $. By~\Cref{thm:well_ord_zinr}, $ A $ has a maximum
    element. Let $ n=\Maximum{A} $. Since $ n\in A $, we have
    $ n\in\mathbf{Z} $ and $ n\le x $. Also, note that $ x-1<n $
    since $ x-1\ge n\implies x\ge n+1\implies n+1\in A\implies
        n\ne \Maximum{A} $. Thus, for $ n=\Maximum{A} $, we have $ n\in\mathbf{Z} $
    with $ x-1<n\le x $ as required.

    Next, consider the case that $ x<0 $. If $ x\in\mathbf{Z} $,
    we can take $ n=x $. Suppose that $ x\notin \mathbf{Z} $.
    We have $ -x>0 $ so, by the previous paragraph, we can choose
    $ m\in\mathbf{Z} $ with $ -x-1<-m<x+1 $. Thus, we can take
    $ n=-m-1 $ to get $ x-1<n<x $.
\end{Proof}
\begin{Definition}{Floor, Floor function}{}
    Let $ x\in\mathbf{R} $. The \textbf{floor} of $ x $,
    denoted by $ \Floor{x} $, is the unique $ n\in\mathbf{Z} $
    with $ x-1<n\le x $. The function $ f:\mathbf{R}\to\mathbf{Z} $
    given by $ f(x)=\Floor{x} $ is called the \textbf{floor function}.
\end{Definition}
\begin{Definition}{Ceiling, Ceiling function}{}
    Let $ x\in\mathbf{R} $. The \textbf{ceiling} of $ x $,
    denoted by $ \Ceil{x} $, is the unique $ n\in\mathbf{Z} $
    with $ x\le n<x+1 $. The function $ f:\mathbf{R}\to\mathbf{Z} $
    given by $ f(x)=\Ceil{x} $ is called the \textbf{ceiling function}.
\end{Definition}
\begin{Theorem}{Archimedean Properties of $ \mathbf{Z} $ in $ \mathbf{R} $}{arch_prop_zinr}
    \begin{enumerate}[label=(\arabic*)]
        \item\label{arch_prop_zinr_1} $ \forall x\in\mathbf{R}\;\exists n\in\mathbf{Z}\;(n>x) $.
        \item\label{arch_prop_zinr_2} $ \forall x\in\mathbf{R}\;\exists m\in\mathbf{Z}\;(m<x) $.
    \end{enumerate}
\end{Theorem}
\begin{Proof}{\Cref{thm:arch_prop_zinr}}{}
    Let $ x\in\mathbf{R} $. Let $ n=\Floor{x}+1 $ and $ m=\Floor{x}-1 $.
    Since $ x-1<\Floor{x} $, we have $ x<\Floor{x}+1=n $ and since
    $ \Floor{x}\le x $, we have $ m=\Floor{x}-1\le x-1<x $.
\end{Proof}
\begin{Theorem}{Density of $ \mathbf{Q} $ in $ \mathbf{R} $}{dens_qinr}
    \[ \forall a\in\mathbf{R}\;\forall b\in\mathbf{R}\;\exists q\in\mathbf{Q}\;
        (a<b\implies a<q<b) \]
\end{Theorem}
\begin{Proof}{}{}
    Let $ a,b\in\mathbf{R} $ with $ a<b $. By~\Cref{thm:arch_prop_zinr},
    we can choose $ n\in\mathbf{Z} $ with $ n>\frac{1}{b-a} >0 $. Then,
    $ n(b-a)>1 $ and so $ nb>na+1 $. Let $ k=\Floor{na+1} $.
    Then we have $ na<k\le na+1<nb $ hence $ a<\frac{k}{n} <b $. Thus,
    we can take $ q=\frac{k}{n} $ to get $ a<q<b $.
\end{Proof}

\section{Limit of Sequences in \texorpdfstring{$ \mathbf{R} $}{R}}
\begin{Definition}{Sequence, Term}{}
    For $ p\in\mathbf{Z} $, let $ Z_{\ge p}=\Set{k\in\mathbf{Z}\given k\ge p} $.
    A \textbf{sequence} in a set $ A $ is a function of the form
    $ x:\mathbf{Z}_{\ge p}\to A $ for some $ p\in\mathbf{Z} $.
    Given a sequence $ x:\mathbf{Z}_{\ge p}\to A $, the $ k^{\text{th}} $
    \textbf{term} of the sequence is the element $ x_k=x(k)\in A $, and
    we denote the sequence $ x $ by
    \[ (x_k)_{k\ge p}=(x_p,x_{p+1},\ldots) \]
    Note that the range of the sequence $ (x_k)_{k\ge p} $
    is the set $ \Set{x_k}_{k\ge p}=\Set{x_k\given k\ge p} $.
\end{Definition}
\begin{Definition}{Limit, Convergence, Divergence}{}
    Let $ (x_k)_{k\ge p} $ be a sequence in $ \mathbf{R} $. For
    $ a\in\mathbf{R} $ we say that $ (x_k)_{k\ge p} $ \textbf{converges}
    to $ a $ (or that the \textbf{limit} of $ (x_k)_{k\ge p} $ is
    equal to $ a $), and we write $ x_k\to a $ (as $ k\to\infty $),
    or we write $ \lim\limits_{{k} \to {\infty}} x_k=a $, when
    \[ \forall\varepsilon\in\mathbf{R}_{>0}\;
        \exists m\in\mathbf{Z}\;\forall k\in\mathbf{Z}_{\ge p}\;
        (k\ge m\implies \abs{x_k-a}<\varepsilon) \]
    We say that the sequence $ (x_k)_{k\ge p} $ converges
    (in $ \mathbf{R} $) when there exists $ a\in\mathbf{R} $
    such that $ (x_k)_{k\ge p} $ converges to $ a $. We say that
    $ (x_k)_{k\ge p} $ \textbf{diverges} (in $ \mathbf{R} $)
    when it does not converge (to any $ a\in\mathbf{R} $).
    We say that $ (x_k)_{k\ge p} $ \textbf{diverges to infinity},
    or that the limit of $ (x_k)_{k\ge p} $ is equal to
    \textbf{infinity}, and we write $ x_k\to\infty $
    (as $ k\to\infty $), or we write $ \lim\limits_{{k} \to {\infty}} x_k=\infty $,
    when
    \[ \forall r\in\mathbf{R}\;\exists m\in\mathbf{Z}\;
        \forall k\in\mathbf{Z}_{\ge p}\;(k\ge m\implies x_k>r) \]
    Similarly, we say that $ (x_k)_{k\ge p} $ \textbf{diverges to}
    $ -\infty $, or that the limit of $ (x_k)_{k\ge p} $ is equal
    to \textbf{negative infinity}, and we write $ x_k\to-\infty $
    (as $ k\to\infty $), or we write $ \lim\limits_{{k} \to {\infty}} x_k=-\infty $
    when
    \[ \forall r\in\mathbf{R}\;\exists m\in\mathbf{Z}\;
        \forall k\in\mathbf{Z}_{\ge p}\;(k\ge m\implies x_k<r) \]
\end{Definition}
\begin{Remark}{}{}
    We shall assume that students are familiar with sequences and limits
    of sequences from first-year calculus. For example, students should know that if
    the limit of a sequence exists, then it is unique. Also, the limit does not depend
    on the first few terms (indeed the first few finitely many terms)
    and so we often omit the starting value $ p $ from our
    notation and write the sequence $ (x_k)_{k\ge p} $ as $ (x_k) $.
    Students should also be able to calculate limits using various
    limit rules, such as Operation on Limits,
    the Comparison Theorem, and the Squeeze Theorem (which can
    all be found in the Appendix).
\end{Remark}
\begin{Definition}{Bounded above, Bounded below, Bounded}{}
    Let $ (x_{k}) $ be a sequence in $ \mathbf{R} $.
    For $ b \in \mathbf{R} $, we say that the sequence
    $ (x_{k}) $ is \textbf{bounded above}
    by $ b $ when the set $\{x_{k}\}$ is bounded above by $ b $;
    that is, when $x_{k} \le b$ for all $ k $, and we say that the sequence
    $ (x_{k}) $ is \textbf{bounded below} by $ b $ when the set $ \{x_{k}\} $
    is bounded below by $ b $; that is, when $ b \le x_{k} $ for all $ k $.
    We say $(x_{k})$ is \textbf{bounded above} when it is bounded above
    by some element $ b \in \mathbf{R} $, we say that $ (x_{k}) $ is
    bounded below when it is bounded below by some $ b \in \mathbf{R} $, and
    we say that $ (x_{k}) $ is \textbf{bounded} when it is bounded above
    and bounded below.
\end{Definition}
\begin{Definition}{Increasing, Non-decreasing, Strictly increasing, Strictly
        decreasing, Monotonic}{}
    Let $ (x_k)_{k\ge p} $ be a sequence in $ \mathbf{R} $.
    \begin{itemize}
        \item $ (x_k) $ is \textbf{increasing} (\textbf{non-decreasing}) when
              \[ \forall k,\ell\in\mathbf{Z}_{\ge p}\;(k\le \ell\implies x_k\le x_\ell) \]
        \item $ (x_k) $ is \textbf{strictly increasing} when
              \[ \forall k,\ell\in\mathbf{Z}_{\ge p}\;(k< \ell\implies x_k< x_\ell) \]
        \item $ (x_k) $ is \textbf{decreasing} (\textbf{non-increasing}) when
              \[ \forall k,\ell\in\mathbf{Z}_{\ge p}\;(k\le \ell\implies x_k\ge x_\ell) \]
        \item $ (x_k) $ is \textbf{strictly decreasing} when
              \[ \forall k,\ell\in\mathbf{Z}_{\ge p}\;(k< \ell\implies x_k> x_\ell) \]
        \item $ (x_k) $ is \textbf{monotonic} when it is either
              increasing or decreasing.
    \end{itemize}
\end{Definition}
\begin{Theorem}{Monotonic Convergence Theorem}{MCT}
    Let $ (x_k) $ be a sequence in $ \mathbf{R} $.
    \begin{enumerate}[(1)]
        \item\label{MCT_1} Suppose $ (x_k) $ is increasing. If $ (x_k) $ is
              bounded above, then $ x_k\to\Supremum{x_k} $, and if
              $ (x_k) $ is not bounded above, then $ x_k\to\infty $.
        \item\label{MCT_2} Suppose $ (x_k) $ is decreasing. If $ (x_k) $ is
              bounded below, then $ x_k\to\Infimum{x_k} $, and if
              $ (x_k) $ is not bounded below, then $ x_k\to-\infty $.
    \end{enumerate}
\end{Theorem}
\begin{Proof}{\Cref{thm:MCT}}{}
    We prove~\ref{MCT_1}. Let $ (x_k) $ be an increasing sequence.
    Assume $ (x_k) $ is bounded above, say by $ b\in\mathbf{R} $.
    Let $ A=\Set{x_k\given k\ge p} $ (so $ A $ is the range of
    the sequence $ (x_k) $). Note that $ A $ is
    non-empty and bounded above (indeed $ b $ is an upper bound for
    $ A $). By~\Cref{thm:sup_inf_properties}~\ref{approx_sup_inf_1},
    $ A $ has a supremum in $ \mathbf{R} $. Let $ a=\Supremum{x_k\given k\ge p} $.
    Note that $ a\ge x_k $ for all $ k\ge p $ and $ a\le b $
    by the definition of supremum. Let $ \varepsilon>0 $. By~\Cref{thm:approx_sup_inf}~\ref{approx_sup_inf_1},
    we can choose an index $ m\ge p $ so that $ x_m\in A $ satisfies
    $ a-\varepsilon<x_m\le a $. Since $ (x_k) $ is increasing, for all
    $ k\ge m $, we have $ x_k\ge x_m $, so we have
    $ a-\varepsilon<x_m\le x_k\le a $, and hence $ \abs{x_k-a}<\varepsilon $.
    Thus, $ \lim\limits_{{k} \to {\infty}} x_k=a\le b $.
\end{Proof}
\begin{Definition}{}{}
    For $ a,b\in\mathbf{R} $ with $ a\le b $, we write
    \begin{itemize}
        \item $ (a,b)=\Set{x\in\mathbf{R}\given a<x<b} $
        \item $ [a,b]=\Set{x\in\mathbf{R}\given a\le x\le b} $
        \item $ \interval[open left]{a}{b}=\Set{x\in\mathbf{R}\given a<x\le b} $
        \item $ \interval[open right]{a}{b}=\Set{x\in\mathbf{R}\given a\le x< b} $
        \item $ (a,\infty)=\Set{x\in\mathbf{R}\given a<x} $
        \item $ \interval[open right]{a}{\infty}=\Set{x\in\mathbf{R}\given a\le x} $
        \item $ (-\infty,b)=\Set{x\in\mathbf{R}\given x<b} $
        \item $ \interval[open left]{-\infty}{b}=\Set{x\in\mathbf{R}\given x\le b} $
        \item $ (-\infty,\infty)=\mathbf{R} $
    \end{itemize}
    An \textbf{interval} in $ \mathbf{R} $ is any set
    of one of the above forms.

    \begin{itemize}
        \item \textbf{Degenerate} intervals: If $ a=b $, then
              $ (a,b)=\interval[open right]{a}{b}=\interval[open left]{a}{b}=\emptyset $,
              and $ [a,b]=\Set{a} $.
        \item \textbf{Non-degenerate} intervals
              contain at least two points.
        \item \textbf{Open} intervals:
              $ \emptyset $, $ (a,b) $, $ (a,\infty) $, $ (-\infty,b) $, and
              $ (-\infty,\infty) $.
        \item \textbf{Closed} intervals:
              $ \emptyset $, $ [a,b] $,
              $ \interval[open right]{a}{\infty} $, $ \interval[open left]{-\infty}{b} $,
              and $ (-\infty,\infty) $.
        \item \textbf{Bounded} intervals:
              $ \emptyset $, $ (a,b) $, $ \interval[open left]{a}{b} $,
              $ \interval[open right]{a}{b} $, and $ [a,b] $.
        \item \textbf{Unbounded} intervals: $ (a,\infty) $, $ \interval[open right]{a}{\infty} $,
              $ (-\infty,b) $, $ \interval[open left]{-\infty}{b} $,
              and $ (-\infty,\infty) $.
    \end{itemize}
\end{Definition}
\begin{Theorem}{Nested Interval Theorem}{NIT}
    Let $ I_1,I_2,I_3,\ldots $ be non-empty, closed, and bounded
    intervals in $ \mathbf{R} $.
    \[ I_1\supseteq I_2\supseteq I_3\supseteq\cdots
        \implies \bigcap_{k\ge 1}I_k\ne \emptyset \]
\end{Theorem}
\begin{Proof}{\Cref{thm:NIT}}{}
    For each $ k\ge 1 $, let $ I_k=[a_k,b_k] $ with $ a_k\le b_k $.
    For each $ k $, since $ I_{k+1}\subseteq I_k $, we have
    $ a_k\le a_{k+1}\le b_{k+1}\le b_k $. Since $ a_k\le a_{k+1} $ for all
    $ k $, the sequence $ (a_k) $ is increasing. Since
    $ a_k\le b_k\le b_{k-1}\le \cdots\le b_1 $ for all $ k $, the sequence
    $ (a_k) $ is bounded above by $ b_1 $. Since
    $ (a_k) $ is increasing, and bounded above, it converges.
    Let $ a=\Supremum{a_k}=\lim\limits_{{k} \to {\infty}} a_k $.
    Similarly, $ (b_k) $ is decreasing, and bounded below by $ a_1 $,
    so it converges. Let $ b=\Infimum{b_k}=\lim\limits_{{k} \to {\infty}} b_k $.
    Since $ a_k\le b_k $ for all $ k $, by the Comparison Theorem, we have
    $ a\le b $, and so the interval $ [a,b] $ is not empty. Since
    $ (a_k) $ is increasing, with $ a_k\to a $, it follows
    (proof as an exercise), that $ a_k\le a $ for all $ k\ge 1 $, Similarly,
    $ b_k\ge b $ for all $ k\ge 1 $, and so
    $ [a,b]\subseteq [a_k,b_k]=I_k $. Therefore,
    \[ [a,b]\subseteq
        \bigcap_{k\ge 1}I_k \implies
        \bigcap_{k\ge 1}I_k\ne \emptyset \]
\end{Proof}
