\makeheading{Lecture 16 | 2020-11-01}
\section{Moment Generating Function Technique}
Idea:
\begin{enumerate}[label=(\arabic*)]
    \item Find the moment generating function of a random variable
    \item Use uniqueness theorem of moment generating function
          to find the distribution of the random variable and then
          the p.d.f.\ of a random variable.
\end{enumerate}
\begin{Theorem}{}{}
    Suppose $ X_1,\ldots,X_n $
    are independent, then $ T=\sum_{i=1}^{n} X_i $
    has moment generating function
    \[ M_T(t)=\E*{e^{t \sum_{i=1}^{n} X_i}}=
        \E*{\prod_{i=1}^n e^{tX_i}}=\prod_{i=1}^n \E*{e^{tX_i}}=
        \prod_{i=1}^n M_{X_i}(t) \]
    In particular, if $ X_1,\ldots,X_n $ are independently
    and identically distributed, then they
    have the exact same moment generating function $ M(t) $;
    that is,
    \[ M_T(t)=\left[ M(t) \right]^n \]
\end{Theorem}
Next, we use the m.g.f.\ technique to find properties
of normal, $ \chi^2 $, $ t $-distribution,
and $ F $-distributions.

\begin{Lemma}{}{lin_comb_normal_lem}
    If $ X \sim \N{\mu,\sigma^2} $, then
    \[ aX+b \sim \N{a\mu+b,a^2\sigma^2} \]
\end{Lemma}
\begin{Proof}{\Cref{lem:lin_comb_normal_lem}}{}
    Recall that the m.g.f.\ of $ X \sim \N{\mu,\sigma^2} $ is
    \[ M_X(t)
        =\exp\left\{ \mu t+\frac{\sigma^2t^2}{2}\right\} \]
    Therefore,
    \begin{align*}
        M_{aX+b}(t)
         & =\E*{e^{t(aX+b)}}                                 \\                                                            \\
         & =e^{b t}\E*{e^{taX}}                              \\
         & =e^{b t}M_X(ta)                                   \\
         & =e^{b t}\expon*{\mu(ta)+\frac{\sigma^2(at)^2}{2}} \\
         & =\expon*{(a\mu+b)t+\frac{a^2\sigma^2t^2}{2}}
    \end{align*}
    which is the m.g.f.\ $ \N{a\mu+b,a^2\sigma^2} $.
\end{Proof}
\begin{Theorem}{}{}
    If $ X \sim \N{\mu,\sigma^2} $, then
    \[ \frac{X-\mu}{\sigma} \sim \N{0,1} \]
\end{Theorem}

\begin{Theorem}{Linear Combination of Independent Normal Random Variables}{lc_ind_normal}
    If $ X_i \sim \N{\mu_i,\sigma_i^2} $, $ i=1,\ldots,n $
    independently, then
    \[ \sum_{i=1}^{n} a_i X_i \sim \N[\bigg]{\sum_{i=1}^{n} a_i\mu_i,
            \sum_{i=1}^{n} a_i^2\sigma_i^2} \]
\end{Theorem}
\begin{Proof}{\Cref{thm:lc_ind_normal}}{}
    By~\Cref{lem:lin_comb_normal_lem}, we have
    $ a_i X_i \sim \N{a_i\mu_i,a_i^2\sigma_i^2} $ for $ i=1,\ldots, n $
    and the m.g.f.\
    \[ M_{a_i X_i}(t)=\expon*{(a_i\mu_i)t+\frac{a_i^2\sigma_i^2}{2}t^2} \]
    Therefore,
    \begin{align*}
        M_{\sum_{i=1}^{n} a_i X_i}(t)
         & =\E*{\expon[\bigg]{t \sum_{i=1}^{n} a_i X_i}}                      \\
         & =\E*{\prod_{i=1}^n e^{(a_i X_i)t}}                                 \\
         & =\prod_{i=1}^n\E*{e^{(a_i X_i)t}}                                  \\
         & =\prod_{i=1}^n M_{a_i X_i}(t)                                      \\
         & =\prod_{i=1}^n \expon*{(a_i\mu_i)t+\frac{\sigma_i^2 a_i^2}{2}t^2 } \\
         & =\expon*{\biggl(\sum_{i=1}^{n} a_i\mu_i\biggr)t
            +\frac{(\sum_{i=1}^{n} a_i^2\sigma_i^2)t^2}{2}}
    \end{align*}
    which is the m.g.f.\ of
    $ \N*{\sum_{i=1}^{n} a_i\mu_i,\sum_{i=1}^{n} a_i^2\sigma_i^2} $.
\end{Proof}
\begin{Corollary}{}{lc_norm_corollary}
    If $ X_1,\ldots,X_n \stackrel{\text{iid}}{\sim} \N{\mu,\sigma^2} $,
    then
    \begin{enumerate}[label=(\arabic*)]
        \item $ \displaystyle  \sum_{i=1}^{n} X_i \sim \N*{n\mu,n\sigma^2} $
        \item $ \displaystyle \bar{X}=\frac{1}{n} \sum_{i=1}^{n} X_i \sim \N*{\mu,\frac{\sigma^2}{n}} $
    \end{enumerate}
\end{Corollary}
\begin{Proof}{\Cref{cor:lc_norm_corollary}}{}
    \begin{enumerate}[label=(\arabic*)]
        \item Let $ a_i=1 $, $ \mu_i=\mu $, $ \sigma_i^2=\sigma $ in~\Cref{thm:lc_ind_normal}.
        \item Let $ a_i=\frac{1}{n} $, $ \mu_i=\mu $, $ \sigma_i^2=\sigma $ in~\Cref{thm:lc_ind_normal}.
    \end{enumerate}
\end{Proof}
\begin{Definition}{Chi-Squared Distribution}{}
    If $ Z_1,\ldots,Z_k \sim \N{0,1} $ are independent
    and $ 0<k\in\mathbb{Z} $, then
    \[ Q=\sum_{i=1}^{k} Z_i^2 \]
    follows a \textbf{chi-squared distribution}
    with $ k $ degrees of freedom and write
    $ Q \sim \chi^2(k) $.
\end{Definition}
If $ X \sim \N{\mu,\sigma^2} $, then
\[ \biggl(\frac{X-\mu}{\sigma}\biggr)^2 \sim \chi^2(1)  \]
If $ Y_i \sim \chi^2(k_i) $ are independent, then
\[ \sum_{i=1}^{n} Y_i \sim \chi^2\biggl(\sum_{i=1}^{n} k_i\biggr) \]
The m.g.f.\ of $ \chi^2(1) $ is $ (1-2t)^{-1/2} $. Derive
the m.g.f.\ $ \chi^2(n) $: $ (1-2t)^{-n/2} $.
\[ \chi^2(n)=\sum_{i=1}^{n} X_i^2\quad X_i\stackrel{\text{iid}}{\sim}\N{0,1}\]
\begin{align*}
    M_{\sum_{i=1}^{n} Y_i}(t)
     & =\prod_{i=1}^n M_{Y_i}(t)         \\
     & =\prod_{i=1}^n(1-2t)^{-k/2}       \\
     & =(1-2t)^{-(\sum_{i=1}^{n} k_i)/2}
\end{align*}
In summary: sum of independent $ \chi^2 $ distributions
follow $ \chi^2 $ with d.f.\ being sum of d.f.\ of
$ \chi^2 $ distributions.

Specifically, if $ X_1,\ldots,X_n\stackrel{\text{iid}}{\sim}\N{\mu,\sigma^2} $
\[ \sum_{i=1}^{n} \biggl(\frac{X_i-\mu}{\sigma}\biggr)^2=
    \frac{\sum_{i=1}^{n} (X_i-\mu)^2}{\sigma^2}\sim \chi(n)  \]
\begin{Definition}{Student's $ t $-distribution}{}
    Let $ Z \sim \N{0,1} $ and $ Q \sim \chi^2(\nu) $
    be independent, then
    \[ T=\frac{Z}{\sqrt{Q/\nu}}  \]
    follows a \textbf{student's t-distribution}
    with $ k $ degrees of freedom and write
    $ T \sim t(\nu) $.

    Support of $ T $: $ (-\infty,\infty) $.
\end{Definition}
\begin{Definition}{$ F $-distribution}{}
    If $ X \sim \chi^2(n) $ and $ Y \sim \chi^2(m) $
    are independent, then
    \[ \frac{X/n}{Y/m} \sim F(n,m) \]
    follows a \textbf{F-distribution}.

    Support of $ F(n,m) $:
    \begin{itemize}
        \item If $ n=1 $: $ \interval[open right]{0}{\infty} $.
        \item If $ n\neq 1 $: $ (0,\infty) $.
    \end{itemize}
\end{Definition}
If $ X \sim \chi^2(n) $ and $ Y \sim \chi^2(m) $ are independent,
then
\[ X+Y \sim \chi^2(n+m) \]
\begin{Exercise}{}{}
    True or false:
    \[ \frac{X/n}{(X+Y)/(n+m)} \sim F(n,n+m) \]
\end{Exercise}
Answer: False.
\begin{Example}{$ \chi^2 $-distribution}{}
    If $ X_1,\ldots,X_n \sim \N{\mu,\sigma^2} $, then
    \[ \sum_{i=1}^{n} \biggl(\frac{X_i-\mu}{\sigma} \biggr) \sim \chi^2(n) \]
    In STAT 231, if we replace $ \mu $ by $ \bar{X} $, then
    \[ \frac{\sum_{i=1}^{n} (X_i-\bar{X})^2}{\sigma^2} \sim \chi^2(n-1)  \]
\end{Example}
Why do we lose one d.f.\ when replacing $ \mu $ by $ \bar{X} $?
\begin{Proof}{}{}
    \begin{align*}
        \frac{\sum_{i=1}^{n} (X_i-\mu)^2}{\sigma^2}
         & =\frac{\sum_{i=1}^{n} (X_i-\bar{X}+\bar{X}-\mu)^2}{\sigma^2} \\
         & =\frac{\sum_{i=1}^{n} (X_i-\bar{X})^2}{\sigma^2}+2
        \frac{\sum_{i=1}^{n} (X_i-\bar{X})(\bar{X}-\mu)}{\sigma^2}+
        \frac{\sum_{i=1}^{n} (\bar{X}-\mu)^2}{\sigma^2}
    \end{align*}
    Note that $ \sum_{i=1}^{n} (X_i-\bar{X})=0 $. So,
    \[ \frac{\sum_{i=1}^{n} (X_i-\mu)^2}{\sigma^2}=
        \frac{\sum_{i=1}^{n} (X_i-\bar{X})^2}{\sigma^2}+
        \frac{n(\bar{X}-\mu)^2}{\sigma^2} \]
    Also,
    \[ \frac{n(\bar{X}-\mu)^2}{\sigma^2}=
        \left[ \frac{\sqrt{n}(\bar{X}-\mu)}{\sigma} \right]^2 \sim \chi^2(1)  \]
    since $ \displaystyle \frac{\sqrt{n}(\bar{X}-\mu)}{\sigma} \sim \N{0,1} $.

    On the left-hand side: $ \displaystyle  \frac{\sum_{i=1}^{n} (X_i-\mu)^2}{\sigma^2}\sim \chi^2(n) $.

    Intuitively,
    \[ \frac{\sum_{i=1}^{n}(X_i-\bar{X})^2}{\sigma^2}\sim \chi^2(n)-\chi^2(1)=\chi^2(n-1)  \]
    A key observation: $ \bar{X} $ and $ \sum_{i=1}^{n} (X_i-\bar{X})^2 $
    are independent.
    \[
        M_{\chi^2(n)}(t)=M_{\frac{\sum_{i=1}^{n} (X_i-\bar{X})^2}{\sigma^2}}(t)
        M_{\frac{n(\bar{X}-\mu)^2}{2}}(t)\\
    \]
    \[ \implies (1-2t)^{-n/2}=M_{\frac{\sum_{i=1}^{n} (X_i-\bar{X})^2}{\sigma^2}}(t)
        (1-2t)^{-1/2} \]
    \[ \implies M_{\frac{\sum_{i=1}^{n} (X_i-\bar{X})^2}{\sigma^2}}(t)=(1-2t)^{-(n-1)/2}  \]
\end{Proof}
Why $ \bar{X} $ is independent of $ \sum_{i=1}^{n} (X_i-\bar{X})^2 $?
\[ (\Uunderbracket{\bar{X}}_{0},X_1-\bar{X},\ldots,X_n-\bar{X})\sim \Mvn{\cdot} \]
Verify that $ \bar{X} $ independent of $ (X_1-\bar{X},\ldots,X_n-\bar{X}) $
by calculating the correlation.
\begin{Example}{$ t $-distribution}{}
    If $ X_1,\ldots,X_n \stackrel{\text{iid}}{\sim}\N{\mu,\sigma^2} $,
    then
    \[ \frac{\bar{X}-\mu}{S/\sqrt{n}}\sim t(n-1)  \]
    where
    \[ S^2=\frac{1}{n-1} \sum_{i=1}^{n} (X_i-\bar{X})^2 \]
    is defined as the sample variance ($ \E{S^2}=\sigma^2 $).

    \textbf{Solution.}
    \[ \frac{\bar{X}-\mu}{\sigma/\sqrt{n}}\sim \N{0,1}  \]
    \[ \frac{(n-1)S^2}{\sigma^2}=\frac{\sum_{i=1}^{n} (X_i-\bar{X})^2}{\sigma^2}
        \sim \chi^2(n-1)   \]
    are independent, then
    \[ \frac{\displaystyle \frac{\bar{X}-\mu}{\sigma/\sqrt{n}}}{
            \displaystyle \sqrt{\frac{(n-1)S^2}{\sigma^2}/(n-1)}
        } =\frac{\bar{X}-\mu}{S/\sqrt{n}} \sim t(n-1) \]
\end{Example}
\begin{Example}{$ F $-distribution}{}
    If $ X_1,\ldots,X_n \stackrel{\text{iid}}{\sim}\N{\mu_1,\sigma_1^2} $
    and $ Y_1,\ldots,Y_m \stackrel{\text{iid}}{\sim}\N{\mu_2,\sigma_2^2} $
    are independent. Define
    \[ S_1^2=\frac{\sum_{i=1}^{n} (X_i-\bar{X})^2}{n-1},\quad
        \bar{X}=\frac{1}{n} \sum_{i=1}^{n} X_i \]
    \[ S_2^2=\frac{\sum_{i=1}^{m} (Y_i-\bar{Y})^2}{m-1},\quad
        \bar{Y}=\frac{1}{m} \sum_{i=1}^{m} Y_i \]
    Then,
    \[ \frac{S_1^2/\sigma_1^2}{S_2^2/\sigma_2^2} \sim F(n-1,m-1) \]
    Reasoning:
    \[ \frac{S_1^2}{\sigma_1^2}=\frac{\displaystyle \frac{\sum_{i=1}^{n} (X_i-\bar{X})^2}{\sigma_1^2}}{
            n-1
        }\sim \frac{\chi^2(n-1)}{n-1}   \]
    \[ \frac{S_2^2}{\sigma_2^2}\sim \frac{\chi^2(m-1)}{m-1}  \]
    are independent, therefore,
    \[ \frac{S_1^2/\sigma_1^2}{S_2^2/\sigma_2^2}\sim
        \frac{\chi^2(n-1)/(n-1)}{\chi^2(m-1)/(m-1)}=F(n-1,m-1)  \]
\end{Example}
