\makeheading{Lecture 20 | 2020-11-15}
\begin{Example}{}{}
    $ X_1,\ldots,X_n \stackrel{\text{iid}}{\sim}\uniform{0,1} $
    and $ U_n=\max_{1\le i\le n}X_i $. In the first two examples
    of this chapter, we've shown that
    \[ U_n=X_{(n)}\stackrel{\mathbb{P}}{\to}1 \]
    \[ U_n\stackrel{\text{d}}{\to}1 \]
    and
    \[ n(1-X_{(n)})=n(1-U_n)\stackrel{\text{d}}{\to} X \sim \exponential{1} \]
    Then,
    \begin{enumerate}[label=(\roman*)]
        \item $ e^{U_n} $
        \item $ \sin(1-U_n) $
        \item $ e^{-n(1-U_n)} $
        \item $ (U_n+1)^2[n(1-U_n)] $
    \end{enumerate}
    \textbf{Solution.}
    \begin{enumerate}[label=(\roman*)]
        \item $ e^{U_n} $
              Take $ g(x)=e^x $. Continuous mapping theorem:
              \[ U_n\stackrel{\mathbb{P}}{\to}1\implies e^{U_n}
                  \stackrel{\mathbb{P}}{\to} e^1 \]
        \item $ \sin(1-U_n) $. Take $ g(x)=\sin(1-x) $.
              \[ \sin(1-U_n)\stackrel{\mathbb{P}}{\to}\sin(1-1)=0 \]
        \item $ e^{-n(1-U_n)} $.
              \[ n(1-U_n)\stackrel{\text{d}}{\to} X \sim \exponential{1} \]
              Continuous mapping theorem. Take $ g(x)=e^{-x} $,
              \[ e^{-n(1-U_n)}\stackrel{\text{d}}{\to}e^{-X}\quad X \sim \exponential{1} \]
              How to find c.d.f.\ of $ e^{-X} $? Let $ Y=e^{-X} $.
              Support of $ Y $ is $ (0,1) $. For any $ 0<y<1 $,
              \begin{align*}
                  F_y(y)
                   & =\Prob{e^{-X}\le y}                    \\
                   & =\Prob{-X\le \ln(y)}                   \\
                   & =\Prob{X\ge -\ln(y)}                   \\
                   & =\int_{-\ln(y)}^{\infty} e^{-x}\, d{x} \\
                   & =y
              \end{align*}
              Therefore,
              \[ e^{-n(1-U_n)}\stackrel{\text{d}}{\to} Y \sim \uniform{0,1} \]
        \item $ (U_n+1)^2[n(1-U_n)] $. Since $ U_n\stackrel{\mathbb{P}}{\to}1 $,
              Take $ g(x)=(1+x)^2 $. Continuous mapping theorem:
              \[ (U_n+1)^2\stackrel{\mathbb{P}}{\to}(1+1)^2=4 \]
              \[ n(1-U_n)\stackrel{\text{d}}{\to}X \sim \exponential{1} \]
              Slutsky's Theorem:
              \[ (U_n+1)^2[n(1-U_n)]\stackrel{\text{d}}{\to}4X \]
              Let $ Y=4X $. Support of $ Y $ is $ (0,\infty) $.
              For $ 0<y<\infty $,
              \begin{align*}
                  \Prob{Y\le y}
                   & =\Prob{4X\le y}               \\
                   & =\Prob*{X\le \frac{y}{4}}     \\
                   & =\int_{0}^{y/4}e^{-x} \, d{x} \\
                   & =1-e^{-y/4}
              \end{align*}
              Hence, the p.d.f.\ of $ Y $ is
              \[ f_Y(y)=\frac{1}{4} e^{-y/4}\quad (y>0) \]
              $ Y \sim \exponential{4} $.
    \end{enumerate}
\end{Example}
\begin{Theorem}{Delta Method}{}
    Let $ X_1,\ldots,X_n $ be a sequence of random variables
    such that
    \[ \sqrt{n}(X_n-\theta)
        \stackrel{\text{d}}{\to}X \sim \N{0,\sigma^2} \]
    and $ g(x) $ is differentiable at $ x=\theta $
    and $ g^{\prime}(\theta)\ne 0 $. Then,
    \[ \sqrt{n}[g(X_n)-g(\theta)]\stackrel{\text{d}}{\to}W \sim \N{0,
            [g^{\prime}(\theta)]^2\sigma^2} \]
\end{Theorem}
Background: $ \sqrt{n}(X_n-\theta)\sqrt{n}(X_n-\theta)
    \stackrel{\text{d}}{\to}X \sim \N{0,\sigma^2} $.
This implies that
\[ \sqrt{n}(X_n-\theta)\stackrel{\text{d}}{\approx}
    \N{0,\sigma^2} \]
equivalently,
\[ X_n\stackrel{\text{d}}{\approx}\N*{\theta,\frac{\sigma^2}{n}} \]
Question: What's the approximate distribution of $ g(X_n) $?
Delta method tells us that
\[ \sqrt{n}[g(X_n)-g(\theta)] \approx\N{0,
        [g^{\prime}(\theta)]^2\sigma^2} \]
\[ \implies g(X_n)\stackrel{\text{d}}{\approx}
    \N*{g(\theta),\frac{[g^\prime(\theta)]\sigma^2}{n} } \]
Not rigorous derivation. By 1st order Taylor
expansion:
\[ f(x)\approx f(x_0)+f^\prime(x_0)(x-x_0)\quad (x\approx x_0) \]
\[ g(X_n)\approx g(\theta)+g^\prime(\theta)(X_n-\theta)\implies
    \sqrt{n}[g(X_n)-g(\theta)]\approx
    \Uunderbracket{\sqrt{n}(X_n-\theta)}_{\N{0,\sigma^2}}
    g^\prime (\theta) \]
By continuous mapping theorem,
\[ \sqrt{n}(X_n-\theta)g^\prime(\theta)
    \stackrel{\text{d}}{\to} g^\prime(\theta)X \sim \N{
        0,[g^\prime(\theta)]^2\sigma^2
    } \]
Not rigorous since we only considered the 1st Taylor expansion,
``why can we drop other terms?''
\begin{Example}{}{}
    $ X_1,\ldots,X_n \stackrel{\text{iid}}{\sim}\poi{\mu} $.
    Find limiting distribution of
    \[ Z_n=\sqrt{n}(\sqrt{\bar{X}_n}-\sqrt{\mu}) \]
    Recall in~\Cref{ex:lec_19_ex}:
    \[ \sqrt{n}(\bar{X}_n-\mu)\stackrel{\text{d}}{\to}
        \N{0,\mu} \]
    since $ \E{X_i}=\mu $, $ \Var{X_i}=\mu $.
    Take $ g(x)=\sqrt{x} $, $ g^\prime(x)=\frac{1}{2} x^{-1/2} $.
    \[ Z_n=\sqrt{n}(\sqrt{\bar{X}_n}-\sqrt{\mu})\stackrel{\text{d}}{\to}
        \N*{0,[g^\prime(\mu)]\sigma^2}=
        \N*{0,\frac{1}{4}\mu^{-1}\mu=\frac{1}{4}}=\N*{0,\frac{1}{4}} \]
\end{Example}
\begin{Example}{}{}
    $ X_1,\ldots,X_n \stackrel{\text{iid}}{\sim}\exponential{\theta} $.
    Find the limiting distribution of
    \begin{enumerate}
        \item $ \bar{X}_n $
        \item $ \displaystyle Z_n=\frac{\sqrt{n}(\bar{X}_n-\theta)}{\bar{X}_n} $
        \item $ U_n=\sqrt{n}(\bar{X}_n-\theta) $
        \item $ V_n=\sqrt{n}(\ln(\bar{X}_n)-\ln(\theta)) $
    \end{enumerate}
    \textbf{Solution.}
    \begin{enumerate}
        \item $ \bar{X}_n $.
              By WLLN, $ \E{X_i}=\theta $, $ \Var{X_i}=\theta^2 $
              (also available on cheat sheet), so $ \bar{X}_n
                  \stackrel{\mathbb{P}}{\to}\theta $.
        \item $ \displaystyle Z_n=\frac{\sqrt{n}(\bar{X}_n-\theta)}{\bar{X}_n} $.
              \[ \frac{\sqrt{n}(\bar{X}_n-\theta)}{\theta} \stackrel{\text{d}}{\to}
                  \N{0,1}\quad \text{CLT} \]
              \[ Z_n=\frac{\sqrt{n}(\bar{X}_n-\theta)}{\bar{X}_n}=
                  \frac{\sqrt{n}(\bar{X}_n-\theta)}{\theta}\frac{\theta}{\bar{X}_n} \]
              by continuous mapping theorem, take $ g(x)=\frac{\theta}{x} $,
              \[ \frac{\theta}{\bar{X}_n}\stackrel{\mathbb{P}}{\to}1 \]
              By Slutsky's Theorem,
              \[ Z_n\stackrel{\text{d}}{\to}Z \sim \N{0,1}(1) \]
        \item $ U_n=\sqrt{n}(\bar{X}_n-\theta) $.
              \[ U_n=\frac{\sqrt{n}(\bar{X}_n-\theta)}{\theta}(\theta)
                  \stackrel{\text{d}}{\to}Z \sim \N{0,1}
              \]
              $ g(x)=\theta x $, continuous mapping theorem
              \[ U_n\stackrel{\text{d}}{\to}\theta Z \sim \N{0,\theta^2} \]
        \item $ V_n=\sqrt{n}(\ln(\bar{X}_n)-\ln(\theta)) $.
              $ g(x)=\ln(x) $. $ g^\prime(x)=1/x $. By Delta Method,
              \[ \sqrt{n}(\bar{X}_n-\theta)\stackrel{\text{d}}{\to}
                  \N{0,\sigma^2} \]
              Delta method,
              \[ \sqrt{n}(\ln(\bar{X}_n)\ln(\theta))
                  \stackrel{\text{d}}{\to}\N*{0,[g^\prime(\sigma)]^2\theta^2}=\N{0,1}
              \]
    \end{enumerate}
\end{Example}
