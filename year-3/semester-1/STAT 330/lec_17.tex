\chapter{Limiting/Asymptotic Distribution}
\makeheading{Lecture 17 | 2020-11-08}
Motivation: We're very interested in the distribution
$ \sqrt{n}(\bar{X}-\mu) $, here $ X_1,\ldots,X_n
    \stackrel{\text{iid}}{\sim} $ with c.d.f.\ $ F $ with
$ \E{X_i}=\mu $ and $ \Var{X_i}=\sigma^2 $,
\[ \bar{X}=\frac{1}{n} \sum_{i=1}^{n} X_i \]
In practice, we don't know the distribution of $ X_i $.
\begin{Remark}{}{}
    \begin{enumerate}[label=(\roman*)]
        \item It is impossible to find the exact distribution of
              $ \sqrt{n}(\bar{X}-\mu) $.
        \item Main idea: are we able to find an approximate
              distribution for $ \sqrt{n}(\bar{X}-\mu) $?
              Concept of limiting/asymptotic distribution
              is introduced for this purpose.
    \end{enumerate}
\end{Remark}
Let $ F_n(x) $ be the c.d.f.\ of $ \sqrt{n}(\bar{X}-\mu) $;
that is, $ F_n(x)=\Prob*{\sqrt{n}(\bar{X}-\mu)\le x} $.
Consider: $ \lim\limits_{{n} \to {\infty}} F_n(x) $ (pointwise
limit) and find that $ \lim\limits_{{n} \to {\infty}} F_n(x)=F(x) $
where $ F(x) $ is a known distribution, e.g.\ normal c.d.f.\
then we can use $ F(x) $ to approximate $ F_n(x) $ for a sufficiently
large $ n $.

To continue, we need some formal definition of this limit
in a mathematical way.

\section{Convergence in Distribution}
\begin{Definition}{Convergence in Distribution}{}
    Let $ X_1,\ldots,X_n $ be a sequence of random variables
    such that $ X_n $ has c.d.f.\ $ F_n(x) $. Let $ X $
    be another random variable with c.d.f.\ $ F(x) $.
    If
    \[ \lim\limits_{{n} \to {\infty}} F_n(x)=F(x) \]
    for all $ x $ at which $ F(x) $ is continuous, then
    we say $ X_n $ \textbf{converges in distribution}
    to $ X $, and write $ X_n\stackrel{\text{d}}{\to} X $.
\end{Definition}
\begin{Remark}{}{}
    \begin{enumerate}[label=(\roman*)]
        \item $ F(x) $ is called the limiting distribution
              (or asymptotic distribution) of $ X_n $.
        \item It's the c.d.f.\ to which $ X_n $ converges to,
              not the random variables. This means,
              $ F_n(x)\approx F(x) $ for $ n $ sufficiently large,
              however $ X_n $ is \underline{\emph{not approximately}},
              $ X $.
        \item $ \lim\limits_{{n} \to {\infty}} F_n(x)=F(x) $
              only for continuous points of $ F(x) $, e.g.\
              \[ F(x)=\begin{cases}
                      1 & x\ge a \\
                      0 & x<a
                  \end{cases} \]
              which is the c.d.f.\
              of constant $ X=a $; that is,
              $ \Prob{X=a}=1 $. It's easy to tell that the
              c.d.f.\ of $ X $ is not continuous.
              $ X_n\to X $ with c.d.f.\ $ F(x) $ if
              $ \lim\limits_{{n} \to {\infty}} F_n(x)=F(x) $
              for $ x\neq a $; that is,
              \[ \lim\limits_{{n} \to {\infty}} F_n(x)=
                  \begin{cases}
                      1 & x>a \\
                      0 & x<a
                  \end{cases} \]
              \underline{we don't care} what's the limit of
              $ F_n(x) $ as $ n\to\infty $.
    \end{enumerate}
\end{Remark}
\begin{Theorem}{$ e $ Limit}{}
    Let $ b,c\in\mathbf{R} $, $ \lim\limits_{{n} \to {\infty}} \psi(n)=0 $.
    \[ \lim\limits_{{n} \to {\infty}}
        \biggr[1+\frac{b}{n} +\frac{\psi(n)}{n} \biggl]^{c n}=e^{b c} \]
\end{Theorem}
\begin{Corollary}{}{}
    Let $ b,c\in\mathbf{R} $.
    \[ \lim\limits_{{n} \to {\infty}}
        \biggl[1+\frac{b}{n}\biggr]^{c n}=e^{b c} \]
\end{Corollary}
\begin{Example}{}{lec_17ex1}
    Suppose that $ X_1,\ldots,X_n\stackrel{\text{iid}}{\sim}
        \uniform{0,1} $. Let $ X_{(1)}=\min(X_1,\ldots,X_n) $
    and $ X_{(n)}=\max(X_1,\ldots,X_n) $. Find the
    limiting distribution of
    \begin{enumerate}[label=(\roman*)]
        \item $ nX_{(1)} $ and $ n(1-X_{(n)}) $
        \item $ X_{(1)} $ and $ X_{(n)} $
    \end{enumerate}
    \textbf{Solution.}
    \begin{enumerate}[label=(\roman*)]
        \item $ nX_{(1)} $. Support is
              $ [0,n] $, so the c.d.f.\ of $ nX_{(1)} $ is:
              \begin{itemize}
                  \item $ x\ge n $, $ F_n(x)=\Prob{n X_{(1)}\le x}=1 $
                  \item $ x\le 0 $, $ F_n(x)=\Prob{n X_{(1)}\le x}=0 $
                  \item $ 0<x<n $,
                        \begin{align*}
                            F_n(x)
                             & =\Prob{n X_{(1)}\le x}                            \\
                             & =\Prob*{X_{(1)}\le \frac{x}{n}}                   \\
                             & =1-\Prob*{X_1>\frac{x}{n},\ldots,X_n>\frac{x}{n}} \\
                             & =1-\biggl[\Prob{X_1>\frac{x}{n}}\biggr]^n         \\
                             & =1-\biggl(1-\frac{x}{n} \biggr)^n
                        \end{align*}
                        Therefore,
                        \[ \Prob{nX_{(1)}\le x}\coloneq
                            F_n(x)=\begin{dcases}
                                0                                & x\le 0 \\
                                1-\biggl(1-\frac{x}{n} \biggr)^n & 0<x<n  \\
                                1                                & x\ge n
                            \end{dcases} \]
                        \[ \implies
                            \lim\limits_{{n} \to {\infty}} F_n(x)=
                            \begin{cases}
                                0        & x\le 0 \\
                                1-e^{-x} & x>0
                            \end{cases} \]
                        Aside: $
                            \displaystyle  \lim\limits_{{n} \to {\infty}}\biggl(1+\frac{x}{n}\biggr)=
                            e^x $
                        which is the c.d.f.\ of $ \exponential{1} $.
              \end{itemize}
              $ n(1-X_{(n)}) $. Support is $ [0,n] $, so the c.d.f.\ of
              $ n(1-X_{(n)}) $ is
              \begin{itemize}
                  \item $ x\ge n $, $ F_n(x)=
                            \Prob{n(1-X_{(n)})\le x}=1 $
                  \item $ x\le 0 $, $ F_n(x)=
                            \Prob{n(1-X_{(n)})\le x}=0 $
                  \item $ 0<x<n $,
                        \begin{align*}
                            F_n(x)
                             & = \Prob{n(1-X_{(n)})\le x}                   \\
                             & =\Prob*{1-X_{(n)}\le \frac{x}{n}}            \\
                             & =1-\Prob*{X_{(n)}<1-\frac{x}{n}}             \\
                             & =1-\Prob*{X_1<1-\frac{x}{n},\ldots,
                            X_n<1-\frac{x}{n}}                              \\
                             & =1-\biggl[\Prob*{X_1<1-\frac{x}{n}}\biggr]^n \\
                             & =1-\biggl(1-\frac{x}{n}\biggr)^n
                        \end{align*}
                        Therefore,
                        \[ F_n(x)=\begin{dcases}
                                0                               & x\le 0 \\
                                1-\biggl(1-\frac{x}{n}\biggr)^n & 0<x<n  \\
                                1                               & x\ge n
                            \end{dcases} \]
                        \[ \implies
                            \lim\limits_{{n} \to {\infty}} F_n(x)=
                            \begin{cases}
                                0        & x\le 0 \\
                                1-e^{-x} & x>0
                            \end{cases} \]
                        which is the c.d.f.\ of $ \exponential{1} $.
              \end{itemize}
        \item $ X_{(1)} $. Support $ (0,1) $.
              \[ F_n(x)=\Prob{X_{(1)}\le x}=
                  \begin{cases}
                      0         & x\le 0 \\
                      1-(1-x)^n & 0<x<1  \\
                      1         & x\ge 1
                  \end{cases} \]
              \[ \implies \lim\limits_{{n} \to {\infty}} F_n(x)=
                  \begin{cases}
                      0 & x\le 0 \\
                      1 & 0<x<1  \\
                      1 & x\ge 1
                  \end{cases}=
                  \begin{cases}
                      0 & x\le 0 \\
                      1 & x>0
                  \end{cases} \]
              Question: What is $ F(x) $?
              \[ F(x)=\begin{cases}
                      0 & x<0    \\
                      1 & x\ge 0
                  \end{cases} \]
              will make $ F(x) $ right-continuous.
              $ F(x) $ is not continuous at $ x=0 $.
              Here, we don't require that $ F_n(x) $ converges
              to $ F(x) $ at $ x=0 $. $ F(x) $
              is actually the c.d.f.\ of $ X $
              which satisfies $ \Prob{X=0}=1 $.

              \[ \lim\limits_{{n} \to {\infty}} F_n(x)=
                  \begin{cases}
                      0 & x\le 0 \\
                      0 & 0<x<1  \\
                      1 & x\ge 1
                  \end{cases}=\begin{cases}
                      0 & x<1    \\
                      1 & x\ge 1
                  \end{cases} \]
              which is right-continuous.

              Therefore,
              $ \lim\limits_{{n} \to {\infty}} F_n(x) =F(x) $
              is the limiting distribution
              \underline{in this case only}.
    \end{enumerate}
\end{Example}
