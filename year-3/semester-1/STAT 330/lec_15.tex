\makeheading{Lecture 15 | 2020-10-01}
Find the p.d.f.\ of $ Y=h(X) $. Two possible ways:
\begin{itemize}
    \item Method I\@: CDF Technique
    \item Method II\@: If $ h(x) $ is a one-to-one function,
          then
          \[ g_Y(y)=f_X(x)\abs*{\frac{dx}{dy} } \]
\end{itemize}
\section{One-to-One Transformations (Bivariate)}
Given $ X $ and $ Y $, the joint p.d.f.\ of $ (X,Y) $
is $ f(x,y) $. We would like to find the joint p.d.f.\
of
\[ U=h_1(X,Y)\quad\text{ and }\quad V=h_2(X,Y) \]
One-to-one bivariate transformation
\[ u=h_1(x,y)\quad\text{ and }\quad v=h_2(x,y) \]
The two functions are one-to-one transformation if there exist
another two unique functions such that
\[ x=\omega_1(u,v)\quad\text{ and }\quad y=\omega_2(u,v) \]
for $ (x,y) $ in support of $ (X,Y) $.
\begin{Theorem}{One-to-One Bivariate Transformations}{}
    The p.d.f.\ of $ U=h_1(X,Y) $ and $ V=h_2(X,Y) $ is given by
    \[ g(u,v)=f(x,y)\abs*{\frac{\partial(x,y)}{\partial(u,v)} } \]
    where
    \[ \frac{\partial(x,y)}{\partial(u,v)} =
        \begin{vmatrix}
            \frac{\partial x}{\partial u} & \frac{\partial x}{\partial v} \\
            \frac{\partial y}{\partial u} & \frac{\partial y}{\partial v}
        \end{vmatrix} \]
    is the Jacobian matrix.
\end{Theorem}
Step 1: Find support of $ (U,V) $ by making use of $ h_1 $, $ h_2 $,
and support of $ (X,Y) $.

Step 2: $ u=h_1(x,y) $ and $ v=h_2(x,y) $ implies
$ x=\omega(u,v) $ and $ y=\omega(u,v) $, compute Jacobian:
\[ \frac{\partial(x,y)}{\partial(u,v)}=\begin{vmatrix}
        \partial x/\partial u & \partial x/\partial v \\
        \partial y/\partial u & \partial y/\partial v
    \end{vmatrix} \]
Step 3:
\[ g(u,v)=f(x,y)\abs*{\frac{\partial(x,y)}{\partial(u,v)} } \]
\begin{Example}{One-to-One Transformation (Bivariate)}{}
    Suppose $ X \sim N(0,1) $ and $ N(0,1) $ independent.
    Find the joint p.d.f.\ of $ U=X+Y $ and $ V=X-Y $.

    \textbf{Solution.} Since $ X $ and $ Y $ are independent,
    the joint p.d.f.\ of $ X $ and $ Y $ is given by
    \[ f(x,y)=f_X(x)f_Y(y)=\frac{1}{\sqrt{2\pi}}\exp\left\{ -\frac{x^2}{2}\right\}
        \frac{1}{\sqrt{2\pi}}\exp\left\{ -\frac{y^2}{2}\right\}=
        \frac{1}{2\pi}\exp\left\{ -\frac{x^2+y^2}{2} \right\}  \]
    Step 1: $ u=x+y $ and $ v=x-y $ implies $ x=(u+v)/2 $
    and $ y=(u-v)/2 $. Support of $ U $ and $ V $ is $ (-\infty,\infty) $.

    Step 2: Jacobian is given by
    \begin{align*}
        \frac{\partial(x,y)}{\partial(u,v)}
         & =\begin{vmatrix}
            \partial x/\partial u & \partial x/\partial v \\
            \partial y/\partial u & \partial y/\partial v
        \end{vmatrix}                             \\
         & =\begin{vmatrix}
            1/2 & 1/2  \\
            1/2 & -1/2
        \end{vmatrix}                             \\
         & =\left( \frac{1}{2} \right)\left( -\frac{1}{2} \right)
        -\left( \frac{1}{2} \right)\left( \frac{1}{2} \right)     \\
         & =-\frac{2}{4}                                          \\
         & =-\frac{1}{2}
    \end{align*}
    Step 3:
    \begin{align*}
        g(u,v)
         & =f(x,y)\abs*{\frac{\partial(x,y)}{\partial(u,v)}}                        \\
         & =\frac{1}{2\pi}\exp\left\{ -\frac{x^2+y^2}{2}\right\}\abs*{-\frac{1}{2}} \\
         & =\frac{1}{4\pi}\exp\left\{ -\frac{[(u+v)/2]^2+[(u-v)/2]^2}{2} \right\}   \\
         & =\frac{1}{4\pi}\exp\left\{ -\frac{u^2+v^2}{4} \right\}
    \end{align*}
\end{Example}
\begin{Example}{One-to-One Transformation (Bivariate)}{}
    Suppose that $ X $ and $ Y $ are continuous random variables with
    joint p.d.f.\ $ f(x,y)=e^{-x-y} $ for $ 0<x<\infty $
    and $ 0<y<\infty $. Find the joint p.d.f.\ of
    $ U=X+Y $ and $ V=X $. Find the marginal p.d.f.\ of $ U $.

    \textbf{Solution.} $ u=x+y $ and $ v=x $ implies
    $ x=v $ and $ y=u-v $. Therefore, $ 0<v<\infty $
    and $ 0<u-v<\infty $. In other words,
    the joint support of $ (U,V) $ is
    $ 0<v<u<\infty $. Jacobian is
    \begin{align*}
        \frac{\partial(x,y)}{\partial(u,v)}
         & =\begin{vmatrix}
            \partial x/\partial u & \partial x/\partial v \\
            \partial y/\partial u & \partial y/\partial v
        \end{vmatrix} \\
         & =\begin{vmatrix}
            0 & 1  \\
            1 & -1
        \end{vmatrix} \\
         & =-1
    \end{align*}
    Therefore, the joint p.d.f.\ of $ (U,V) $ for $ 0<v<u<\infty $ is
    \begin{align*}
        g(u,v)
         & =f(x,y)\abs*{\frac{\partial(x,y)}{\partial(u,v)}} \\
         & =e^{-x-y}\abs{-1}                                 \\
         & =e^{-(x+y)}                                       \\
         & =e^{-u}
    \end{align*}
    Support of $ U $ is $ (0,\infty) $.
    The marginal p.d.f.\ of $ U $ for $ u>0 $ is
    \[ f_1(u)=\int_{-\infty}^{\infty} g(u,v)\, d{v}=
        \int_{0}^{u} e^{-u}\, d{v}=u e^{-u} \]
    Find the p.d.f.\ of $ U=X+Y $.
    \begin{enumerate}
        \item CDF Technique
        \item Define $ V=X $ (or $ U=Y $),
              find $ (U,V) $ with the Theorem.
    \end{enumerate}
\end{Example}
\begin{Example}{Support of One-to-One Transformation (Bivariate)}{}
    Suppose that the support of $ (X,Y) $ is
    $ 0<x<y<1 $. Find the support of $ (U,V) $
    where $ U=X $ and $ V=XY $.

    \textbf{Solution.} $ u=x $ and $ v=xy $
    implies $ x=u $ and $ y=v/u $.
    \[ 0<u<\frac{v}{u} <1\implies 0<u^2<v<u<1 \]
    (multiply by $ u $)
\end{Example}
\begin{Example}{Support of One-to-One Transformation (Bivariate)}{}
    Suppose the support of $ (X,Y) $ is $ 0<x<1 $
    and $ 0<y<1 $. Find the support of $ (U,V) $
    where $ U=X/Y $ and $ V=XY $.

    \textbf{Solution.} $ u=x/y $ and $ v=xy $.
    \[ uv=x^2\implies x=\sqrt{uv} \]
    \[ y=\frac{v}{x} \implies y=\frac{v}{\sqrt{uv}}
        =\frac{v^{1/2}}{u^{1/2}v^{1/2}}=\sqrt{\frac{v}{u}}  \]
    So,
    \[ 0<\sqrt{uv}<1\implies 0<uv<1\implies 0<u<\frac{1}{v}\quad (v>0) \]
    \[ 0<\sqrt{\frac{v}{u}}<1\implies 0<\frac{v}{u}<1\implies 0<v<u\quad (u>0) \]
    Combining, we get $ 0<v<u<1/v $.
\end{Example}
