\makeheading{Lecture 12 | 2020-10-18}
\section{Multinomial Distribution}
\begin{Definition}{Multinomial distribution}{}
    $ (X_1,\ldots,X_k) $ are joint discrete
    random variables with joint p.f.\ given by
    \[ f(x_1,\ldots,x_k)=\Prob{X_1=x_1,\ldots,X_k=x_k}=
        \frac{n!}{x_1!x_2!\cdots x_k!}p_1^{x_1}\cdots p_k^{x_k} \]
    where $ x_i=0,1,\ldots,n $ ($ i=1,2,\ldots,k $). Furthermore,
    $ \sum_{i=1}^{k}x_i=n $, $ \sum_{i=1}^{k} p_i=1 $,
    for $ 0<p_i<1 $ $ i=1,\ldots,k $. Then,
    $ (X_1,\ldots,X_k) $ follows a \textbf{multinomial distribution}.
    \[ (X_1,\ldots,X_k)\sim \mult{n;p_1,\ldots,p_k} \]
\end{Definition}
\begin{Example}{Possible Application}{}
    \begin{itemize}
        \item there are $ k $ boxes and each box
              has same balls
        \item probability of choosing a ball from the
              $ i^{\text{th}} $ box is $ p_i $ for $ i=1,2,\ldots,k $.
        \item randomly choose $ n $ balls from $ k $ boxes.
    \end{itemize}
    Let $ X_i\coloneq $ number of boxes from the
    $ i^{\text{th}} $ box for $ i=1,2,\ldots,k $. Then,
    \[ (X_1,\ldots,X_k) \sim \mult{n;p_1,\ldots,p_k} \]
    Note: if there are only two boxes, then $ X_1\sim \bin{n,p_1} $.
\end{Example}
\begin{Proposition}{Properties --- Multinomial Distribution}{prop_multi}
    If $ (X_1,\ldots,X_k) \sim \mult{n;p_1,\ldots,p_k} $, then
    \begin{enumerate}[label=(\arabic*)]
        \item\label{prop_multi1} $ M(t_1,\ldots,t_k)=\E{e^{t_1 X_1+\cdots+t_k X_k}}=
                  (p_1e^{t_1}+\cdots+p_k e^{t_k})^n $
              where $ \abs{t_i}<\infty $ for $ i=1,\ldots,k $.
        \item\label{prop_multi2} $ X_i \sim \bin{n,p_i} $ for $ i=1,\ldots,k $.
        \item\label{prop_multi3} If $ T=X_i+X_j $ for $ i\neq j $, then
              $ T \sim \bin{n,p_i+p_j} $
        \item\label{prop_multi4} $ \Cov{X_i,X_j}=-n p_i p_j  $
              for $ i\neq j $
        \item\label{prop_multi5} The conditional probability
              function of $ X_i $ given $ X_j=x_j $ for $ i\neq j $ is
              \[ X_i\mid X_j=x_j \sim \bin*{n-x_j,
                      \frac{p_i}{1-p_j}} \]
        \item\label{prop_multi6} The conditional distribution of $ X_i $
              given $ T=X_i+X_j $ for $ i\neq j $ is
              \[ X_i\mid X_i+X_j \sim \bin*{t,\frac{p_i}{p_i+p_j} } \]
    \end{enumerate}
\end{Proposition}
\begin{Proof}{\Cref{prop:prop_multi}}{}
    Proof of~\Cref{prop_multi1}: Too long
    for my poor soul to type. Proof requires the Multinomial Theorem.

    Proof of~\Cref{prop_multi2}: The moment
    generating function of $ X_i $ for $ i=1,\ldots,k $ is
    \[ M(0,\ldots,0,t,0,\ldots,0)=\left[ p_i e^{t_i}+(1-p_i) \right]^n
        \quad t_i\in\mathbf{R} \]
    which is the moment generating function of a
    $ \bin{n,p_i} $ random variable. By~\Cref{thm:uniq_mgf}
    we have $ X_i \sim \bin{n,p_i} $ for $ i=1,\ldots,k $.

    Proof of~\Cref{prop_multi3}: The moment
    generating function of $ T=X_i+X_j $ for $ i\neq j $ is
    \begin{align*}
        M_T(t)
         & =\E*{e^{tT}}                                                                         \\
         & =\E*{e^{t(X_i+X_j)}}                                                                 \\
         & =\E*{e^{t X_i+t X_j}}                                                                \\
         & =M(0,\ldots,0,t,0,\ldots,0,t,0,\ldots,0)                                             \\
         & =(p_1+\cdots+p_i e^{t}+\cdots+p_j e^t+\cdots+p_{k-1}+p_k)^n & \quad & t\in\mathbf{R} \\
         & =\left[ (p_i+p_j)e^t+(1-p_i-p_j) \right]^n                  &       & t\in\mathbf{R}
    \end{align*}
    which is the moment generating function of a
    $ \bin{n,p_i+p_j} $ random variable. By~\Cref{thm:uniq_mgf}
    we have $ T \sim \bin{n,p_i+p_j} $ for $ i\neq j $.

    Proof of~\Cref{prop_multi4}:
    By~\Cref{prop_multi2} we have $ \E{X_i}=n p_i $, $ \Var{X_i}=np_i(1-p_i) $,
    and $ \Var{X_j}=np_j(1-p_j) $. By~\Cref{prop_multi3} we have
    $ X_i+X_j \sim \bin{n,p_i+p_j} $, so
    $ \Var{X_i+X_j}=n(p_i+p_j)(1-p_i-p_j) $.
    Thus,
    \[ \Cov{X_i+X_j,X_i+X_j}=\Var{X_i}+\Var{X_j}+2\Cov{X_i,X_j} \]
    \[ \implies n(p_i+p_j)(1-p_i-p_j)=np_i(1-p_i)+np_j(1-p_j)
        +2\Cov{X_i,X_j} \]
    Therefore, $ \Cov{X_i,X_j}=-n p_i p_j $.

    Proof of~\Cref{prop_multi5}: There are $ x_j $ outcomes
    from the $ j^{\text{th}} $ category. Therefore,
    there are $ (n-x_j) $ balls chosen from the remaining $ (k-1) $
    boxes. We are not allowed to choose from the $ j^{\text{th}} $ box,
    we are only allowed to choose from the remaining $ (k-1) $
    boxes. Therefore, proportionally we get the success probability
    as $ p_i/(1-p_j) $.
\end{Proof}
\begin{Exercise}{}{}
    Prove property~\Cref{prop_multi6} from~\Cref{prop:prop_multi}.
\end{Exercise}
\section{Bivariate Normal Distribution}
\begin{Definition}{Bivariate normal distribution}{}
    Suppose that $ X_1 $ and $ X_2 $
    are continuous random variables with joint probability
    density function
    \[ f(x_1,x_2)=\frac{1}{2\pi\abs*{\Sigma}^{1/2}}
        \exp\left\{ -\frac{1}{2}(\symbf{x}-\symbf{\mu})^\top \Sigma^{-1}(\symbf{x}-\symbf{\mu})
        \right\}\quad(x_1,x_2)\in\mathbf{R}^2     \]
    Also,
    \[ \symbf{x}=\begin{pmatrix}
            x_1 \\
            x_2
        \end{pmatrix}_{2\times 1},\quad
        \symbf{\mu}=\begin{pmatrix}
            \mu_1 \\
            \mu_2
        \end{pmatrix}_{2\times 1},\quad
        \Sigma=
        \begin{pmatrix}
            \sigma_1^2        & p\sigma_1\sigma_2 \\
            p\sigma_1\sigma_2 & \sigma_2^2
        \end{pmatrix}_{k\times k} \]
    and $ \Sigma $ is positive semi-definite. Also,
    $ \abs*{\Sigma} $ is the determinant of $ \Sigma $.
    Then, $ \symbf{X}=(X_1,X_2)^\top $ follows a \textbf{bivariate normal distribution},
    and we write
    \[ \symbf{X}\sim\Bvn{\symbf{\mu},\Sigma}  \]
\end{Definition}
\begin{Remark}{$ \dagger $}{}
    Alternatively, we could write
    \begin{align*}
        f(x_1 & ,x_2)                                          \\
              & =\frac{1}{2\pi\sigma_1\sigma_2\sqrt{1-\rho^2}}
        \exp\left\{ -\frac{1}{2(1-\rho^2)}
        \left[ \left( \frac{x_1-\mu_1}{\sigma_1}  \right)^2+
            \left( \frac{x_2-\mu_2}{\sigma_2}  \right)^2-
            \frac{2\rho(x_1-\mu_1)(x_2-\mu_2)}{\sigma_1\sigma_2} \right] \right\}
    \end{align*}
\end{Remark}
\begin{Proposition}{Properties --- Bivariate Normal Distribution}{prop_bvn}
    \begin{enumerate}[label=(\arabic*)]
        \item\label{prop_bvn_1}$ X_1,X_2 $ has joint moment generating function
              \[ M(t_1,t_2)=\E{
                      e^{t_1X_1+t_2X_2}
                  }=\exp\left\{ \symbf{t}^\top \symbf{\mu}+\frac{1}{2}\symbf{t}^\top \Sigma
                  \symbf{t} \right\}\quad \forall \symbf{t}\in\mathbf{R}^2 \]
        \item\label{prop_bvn_2}Marginally,
              \[ M_{X_1}(t_1)=M(t_1,0)=\exp\left\{ t_1\mu_1+\frac{1}{2} t_1^2
                  \sigma_1^2\right\} \]
              which is the m.g.f.\ of $ \N{\mu_1,\sigma_1^2} $; that is,
              $ X_1 \sim \N{\mu_1,\sigma_1^2} $. Also,
              $ \E{X_1}=\mu_1 $ and $ \Var{X_1}=\sigma_1^2 $.
              \[ M_{X_2}(t_2)=M(0,t_2)=\exp\left\{ t_2\mu_2+\frac{1}{2} t_2^2
                  \sigma_2^2\right\} \]
              which is the m.g.f.\ of $ \N{\mu_2,\sigma_2^2} $; that is,
              $ X_2 \sim \N{\mu_2,\sigma_2^2} $. Also,
              $ \E{X_2}=\mu_2 $ and $ \Var{X_2}=\sigma_2^2 $.
        \item\label{prop_bvn_3}Conditional distribution.
              \[ X_2\mid X_1=x_1 \sim \N*{
                      \mu_2+\frac{\rho\sigma_2(x_1-\mu_1)}{\sigma_1} ,
                      \sigma_2^2(1-\rho^2)} \]
              \[ X_1\mid X_2=x_2 \sim N
                  \left( \mu_1+\frac{\rho\sigma_1(x_2-\mu_2)}{\sigma_2} ,
                  \sigma_1^2(1-\rho^2) \right) \]
              \[ f_2(x_2\mid x_1)=\frac{f(x_1,x_2)}{f_1(x_1)}  \]
              \[ f_1(x_1\mid x_2)=\frac{f(x_1,x_2)}{f_2(x_2)}  \]
        \item\label{prop_bvn_4}$ \Cov{X_1,X_2}=\rho\sigma_1\sigma_2 $
        \item\label{prop_bvn_5}$ \rho=0\iff X_1\text{ and }X_2 $ are independent.
        \item\label{prop_bvn_6}Linear transformations of bivariate
              normal are still normal.
        \item\label{prop_bvn_7}$ (\symbf{X}-\symbf{\mu})^\top \Sigma^{-1}(\symbf{X}-\symbf{\mu})\sim \chi^2(2) $
    \end{enumerate}
\end{Proposition}
\begin{Proof}{\Cref{prop:prop_bvn}}{}
    Proof of~\Cref{prop_bvn_4}: We want to find $ \E{X_1X_2}=\E{\E{X_1X_2\given X_1}} $.

    Step 1:
    \[ \E{X_1X_2\given X_1=x_1}=\E{x_1X_2\given X_1=x_1}=
        x_1\E{X_2\given X_1=x_1}=x_1
        \left(  \mu_2+\frac{\rho\sigma_2(x_1-\mu_1)}{\sigma_1} \right) \]
    Step 2:
    \[ \E{X_1X_2\given X_1}=X_1
        \left(  \mu_2+\frac{\rho\sigma_2(X_1-\mu_1)}{\sigma_1} \right)
    \]
    \begin{align*}
        \E{X_1X_2}
         & = \E*{X_1\mu_2+\frac{X_1\rho\sigma_2(X_1-\mu_1)}{\sigma_1}}            \\
         & =\mu_2\E{X_1}+\frac{\rho\sigma_2}{\sigma_1}
        (\E{X_1^2}-\mu_1\E{X_1})                                                  \\
         & =\mu_2\mu_1+\frac{\rho\sigma_2}{\sigma_1}( \mu_1^2+\sigma_1^2-\mu_1^2) \\
         & =\mu_1\mu_2+\rho\sigma_1\sigma_2
    \end{align*}
    Thus,
    \[ \Cov{X_1,X_2}=\E{X_1X_2}-\frac{\E{X_1}\E{X_2}}{\mu_1\mu_2}=\rho\sigma_1\sigma_2  \]
    \[ \Corr{X_1,X_2}=\frac{\Cov{X_1,X_2}}{\sqrt{\Var{X_1}\Var{X_2}}}=\rho  \]

    Proof of~\Cref{prop_bvn_5}: We know if $ X_1 $ and $ X_2 $ are independent, then $ \rho=0 $.
    If $ \rho=0 $, e.g. $ X_2\given X_1=x_1 \sim \N{\mu_2,\sigma_1^2} $
    and $ X_1\given X_2=x_2 \sim \N{\mu_1,\sigma_1^2} $.
    In summary: If joint bivariate normal then uncorrelated = independence.

    Proof of~\Cref{prop_bvn_6}: Let $ \symbf{c}=(c_1,c_2)^\top $, then
    $ \symbf{c}^\top X=c_1X_2+c_2X_2 \sim \N{c_1\mu_1+c_2\mu_2,\symbf{c}^\top \Sigma\symbf{c}} $.
    Furthermore, if $ A\in\mathbf{R}^{2\times 2} $, and
    $ \symbf{b}=(b_1,b_2)^\top $, then
    \[ AX+\symbf{b} \sim \Bvn{A\symbf{\mu}+\symbf{b},A\Sigma A^\top} \]
    Two linear combinations of BVN is joint BVN.\
\end{Proof}
\begin{Remark}{}{}
    Remark of~\Cref{prop_bvn_7}: Note $ \chi^2(1)\coloneq Z^2 $ where $ Z \sim \N{0,1} $.
    \[ \chi^2(n)=
        \sum_{i=1}^{n} Z_i^2 \]
    where $ Z_1,\ldots,Z_n $ are independent $ \N{0,1} $.
\end{Remark}
