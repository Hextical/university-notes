\documentclass[oneside]{book}\usepackage[]{graphicx}\usepackage[dvipsnames,table,xcdraw]{xcolor}
% maxwidth is the original width if it is less than linewidth
% otherwise use linewidth (to make sure the graphics do not exceed the margin)
\makeatletter
\def\maxwidth{ %
  \ifdim\Gin@nat@width>\linewidth
    \linewidth
  \else
    \Gin@nat@width
  \fi
}
\makeatother

\definecolor{fgcolor}{rgb}{0.345, 0.345, 0.345}
\newcommand{\hlnum}[1]{\textcolor[rgb]{0.686,0.059,0.569}{#1}}%
\newcommand{\hlstr}[1]{\textcolor[rgb]{0.192,0.494,0.8}{#1}}%
\newcommand{\hlcom}[1]{\textcolor[rgb]{0.678,0.584,0.686}{\textit{#1}}}%
\newcommand{\hlopt}[1]{\textcolor[rgb]{0,0,0}{#1}}%
\newcommand{\hlstd}[1]{\textcolor[rgb]{0.345,0.345,0.345}{#1}}%
\newcommand{\hlkwa}[1]{\textcolor[rgb]{0.161,0.373,0.58}{\textbf{#1}}}%
\newcommand{\hlkwb}[1]{\textcolor[rgb]{0.69,0.353,0.396}{#1}}%
\newcommand{\hlkwc}[1]{\textcolor[rgb]{0.333,0.667,0.333}{#1}}%
\newcommand{\hlkwd}[1]{\textcolor[rgb]{0.737,0.353,0.396}{\textbf{#1}}}%
\let\hlipl\hlkwb

\usepackage{framed}
\makeatletter
\newenvironment{kframe}{%
 \def\at@end@of@kframe{}%
 \ifinner\ifhmode%
  \def\at@end@of@kframe{\end{minipage}}%
  \begin{minipage}{\columnwidth}%
 \fi\fi%
 \def\FrameCommand##1{\hskip\@totalleftmargin \hskip-\fboxsep
 \colorbox{shadecolor}{##1}\hskip-\fboxsep
     % There is no \\@totalrightmargin, so:
     \hskip-\linewidth \hskip-\@totalleftmargin \hskip\columnwidth}%
 \MakeFramed {\advance\hsize-\width
   \@totalleftmargin\z@ \linewidth\hsize
   \@setminipage}}%
 {\par\unskip\endMakeFramed%
 \at@end@of@kframe}
\makeatother

\definecolor{shadecolor}{rgb}{.97, .97, .97}
\definecolor{messagecolor}{rgb}{0, 0, 0}
\definecolor{warningcolor}{rgb}{1, 0, 1}
\definecolor{errorcolor}{rgb}{1, 0, 0}
\newenvironment{knitrout}{}{} % an empty environment to be redefined in TeX

\usepackage{alltt}
\usepackage[dvipsnames,table,xcdraw]{xcolor}

\usepackage{fontspec}
\setmainfont{XCharter}
\usepackage{anyfontsize}
\usepackage{microtype}
\usepackage{mathtools}
\usepackage[math-style=ISO,bold-style=ISO,warnings-off={mathtools-colon,mathtools-overbracket}]{unicode-math}

% -----------------------------------------------------------------------------

% CS 246
\usepackage{float}
\usepackage{listings}

\definecolor{light-gray}{gray}{0.95}
\newcommand{\code}[1]{\texttt{#1}}

% -----------------------------------------------------------------------------

% CO 250
\usepackage{tkz-berge}

% -----------------------------------------------------------------------------

% Core Packages
\usepackage{xfrac}
\usepackage[margin=1in]{geometry}
\usepackage[unicode,pdfversion=1.7]{hyperref}
\usepackage[shortlabels]{enumitem}
\usepackage[parfill]{parskip}
\usepackage[theorems,breakable]{tcolorbox}
\usepackage{graphicx}
\usepackage[ruled,linesnumbered,vlined,dotocloa]{algorithm2e}
\usepackage[delims=\lbrack\rbrack]{spalign}
\usepackage{cleveref}
\usepackage{pdfpages}
\usepackage{minted}
\usepackage{tikz}
\usetikzlibrary{patterns}
\usetikzlibrary{positioning}
\usepackage{pgfplots}
\pgfplotsset{compat=1.17}
\usepackage{framed}


% -----------------------------------------------------------------------------

% Better Tables
\usepackage{multicol}
\usepackage{booktabs}
\usepackage{adjustbox}
\usepackage{tabularx}
\newcolumntype{Y}{>{\centering\arraybackslash}X}
\newcolumntype{Z}{>{\centering\arraybackslash\columncolor{light-gray}}X}
\newcolumntype{B}{>{\centering\arraybackslash\bfseries}X}
\usepackage{multirow}
\usepackage[skip=1ex]{caption}

% -----------------------------------------------------------------------------

% Intervals
\usepackage{interval}
\intervalconfig{
    soft open fences,
    separator symbol={,}
}

% -----------------------------------------------------------------------------

\graphicspath{ {./figures/} }

\DeclareMathOperator{\rank}{rank}
\DeclareMathOperator{\slack}{slack}
\DeclareMathOperator{\row}{row}
\DeclareMathOperator{\cone}{cone}
\DeclareMathOperator{\nullspace}{Null}
\DeclareMathOperator{\ch}{char}
\DeclareMathOperator{\ord}{ord}
\DeclareMathOperator{\lcm}{lcm}

\usepackage{etoolbox}

% Functions
\providecommand\given{} % just to make sure it exists
\DeclarePairedDelimiterXPP{\E}[1]{\mathbb{E}}[]{}{
    \renewcommand\given{\nonscript\:\delimsize\vert\nonscript\:\mathopen{}}
    #1}
\DeclarePairedDelimiterXPP{\Var}[1]{\mathbb{V}}(){}{
    \renewcommand\given{\nonscript\:\delimsize\vert\nonscript\:\mathopen{}}
    #1}
\DeclarePairedDelimiterXPP\Prob[1]{\mathbb{P}}(){}{
    \renewcommand\given{\nonscript\:\delimsize\vert\nonscript\:\mathopen{}}
    \ifblank{#1}{\:\cdot\:}
    #1}
\DeclarePairedDelimiterXPP\Ind[1]{\mathbb{I}}\{\}{}{
\renewcommand\given{\nonscript\:\delimsize\vert\nonscript\:\mathopen{}}
\ifblank{#1}{\:\cdot\:}
#1}
\newcommand{\indep}{\perp\!\!\!\perp}
\DeclarePairedDelimiterXPP{\Corr}[1]{\text{Corr}}(){}{#1}
\DeclarePairedDelimiterXPP{\Cov}[1]{\text{Cov}}(){}{#1}
\DeclarePairedDelimiterXPP{\Sd}[1]{\text{Sd}}(){}{#1}
\DeclarePairedDelimiterXPP{\Se}[1]{\text{Se}}(){}{#1}
\let\SS=\relax
\DeclarePairedDelimiterXPP{\SS}[1]{\text{SS}}(){}{\text{#1}}
\DeclarePairedDelimiterXPP{\MS}[1]{\text{MS}}(){}{\text{#1}}
\DeclarePairedDelimiterXPP{\Span}[1]{\text{Span}}(){}{#1}
\DeclarePairedDelimiterXPP{\Spanc}[1]{\overline{\text{Span}}}(){}{#1}
\DeclareMathOperator{\VIF}{VIF}
\DeclarePairedDelimiterXPP{\expon}[1]{\text{exp}}\{\}{}{#1}

% Distributions
\DeclarePairedDelimiterXPP{\N}[1]{\mathcal{N}}(){}{#1}
\DeclarePairedDelimiterXPP{\uniform}[1]{\text{Uniform}}(){}{#1}
\DeclarePairedDelimiterXPP{\hyp}[1]{\text{Hypergeometric}}(){}{#1}
\DeclarePairedDelimiterXPP{\bern}[1]{\text{Bernoulli}}(){}{#1}
\DeclarePairedDelimiterXPP{\bin}[1]{\text{Binomial}}(){}{#1}
\DeclarePairedDelimiterXPP{\nb}[1]{\text{Negative Binomial}}(){}{#1}
\DeclarePairedDelimiterXPP{\geo}[1]{\text{Geometric}}(){}{#1}
\DeclarePairedDelimiterXPP{\poi}[1]{\text{Poisson}}(){}{#1}
\DeclarePairedDelimiterXPP{\mult}[1]{\text{Multinomial}}(){}{#1}
\DeclarePairedDelimiterXPP{\gam}[1]{\text{Gamma}}(){}{#1}
\DeclarePairedDelimiterXPP{\weib}[1]{\text{Weibull}}(){}{#1}
\DeclarePairedDelimiterXPP{\Mvn}[1]{\text{MVN}}(){}{#1}
\DeclarePairedDelimiterXPP{\Bvn}[1]{\text{BVN}}(){}{#1}
\DeclarePairedDelimiterXPP{\exponential}[1]{\text{Exponential}}(){}{#1}

\DeclarePairedDelimiterXPP{\tr}[1]{\text{trace}}(){}{#1}

\DeclarePairedDelimiterX\innerp[2]{\langle}{\rangle}{
    \ifblank{#1}{\:\cdot\:,}#1,
    \ifblank{#2}{\:\cdot\:}#2
}

\DeclarePairedDelimiterXPP{\MA}[1]{\text{MA}}(){}{#1}
\DeclarePairedDelimiterXPP{\AR}[1]{\text{AR}}(){}{#1}
\DeclarePairedDelimiterXPP{\ARMA}[1]{\text{ARMA}}(){}{#1}
\DeclarePairedDelimiterXPP{\AIC}[1]{\text{AIC}}(){}{#1}
\DeclarePairedDelimiterXPP{\BIC}[1]{\text{BIC}}(){}{#1}
\DeclarePairedDelimiterXPP{\ARIMA}[1]{\text{ARIMA}}(){}{#1}
\DeclarePairedDelimiterXPP{\GARCH}[1]{\text{GARCH}}(){}{#1}
\DeclarePairedDelimiterXPP{\NNAR}[1]{\text{NNAR}}(){}{#1}
\DeclarePairedDelimiterXPP{\NNSAR}[2]{\text{NNSAR}}(){_{#2}}{#1}
\DeclarePairedDelimiterXPP{\ARCH}[1]{\text{ARCH}}(){}{#1}

\DeclareMathOperator{\FWER}{FWER}
\let\log\relax
\DeclarePairedDelimiterXPP{\log}[1]{\text{log}}(){}{#1}

% -----------------------------------------------------------------------------

% Table of Contents
\hypersetup{colorlinks, linkcolor=[rgb]{0,0.5,1}}

% -----------------------------------------------------------------------------

% Heading Dates
\newcommand{\makeheading}[1]
{
    \begin{figure}[H]
        \centering
        \rule{\columnwidth}{1pt}\\
        {\large \scshape{#1}}\\[-0.6\baselineskip]
        \rule{\columnwidth}{1pt}
        \vspace*{-20pt}
    \end{figure}
}

% -----------------------------------------------------------------------------

% Definitions
\definecolor{myyellow}{RGB}{255,255,168}
% Theorems
\definecolor{mypurple}{RGB}{216,216,255}
% Algorithms
\definecolor{mygray}{RGB}{232,232,232}
% Examples
\definecolor{mygreen}{RGB}{216,255,216}
% Exercises
\definecolor{myred}{RGB}{255,216,216}
% Remarks
\definecolor{mycyan}{RGB}{204,229,229}

\tcbset{
    common/.style={
            fonttitle=\bfseries,
            coltitle=black,
            boxrule=0pt,
            breakable
        },
    theorem/.style={
            common,
            colback=mypurple,
            colframe=mypurple!95!black,
            fontupper=\itshape{}
        },
}


\newtcbtheorem[number within=section, crefname={definition}{definitions}]
{Definition}{DEFINITION}{
    common,
    colback=myyellow,
    colframe=myyellow!95!black
}{def}

\newtcbtheorem[use counter from=Definition, crefname={example}{examples}]
{Example}{EXAMPLE}{
    common,
    colback=mygreen,
    colframe=mygreen!95!black,
}{ex}

\newtcbtheorem[use counter from=Definition, crefname={exercise}{exercises}]
{Exercise}{EXERCISE}{
    common,
    colback=myred,
    colframe=myred!95!black,
}{exercise}

\newtcbtheorem[use counter from=Definition, crefname={remark}{remarks}]
{Remark}{REMARK}{
    common,
    colback=mycyan,
    colframe=mycyan!95!black,
}{remark}

\newtcbtheorem[use counter from=Definition, crefname={statistical Test}{statistical Tests}]
{Statistical_Test}{STATISTICAL TEST}{
    common,
    colback=Magenta!25!white,
    colframe=Magenta!50!white,
}{stest}

\newtcbtheorem[use counter from=Definition, crefname={theorem}{theorems}]
{Theorem}{THEOREM}{
    theorem
}{thm}

\newtcbtheorem[use counter from=Definition, crefname={proposition}{propositions}]
{Proposition}{PROPOSITION}{
    theorem
}{prop}

\newtcbtheorem[use counter from=Definition, crefname={corollary}{corollaries}]
{Corollary}{COROLLARY}{
    theorem
}{cor}

\newtcbtheorem[use counter from=Definition, crefname={lemma}{lemmas}]
{Lemma}{LEMMA}{
    theorem
}{lem}

\newtcbtheorem[no counter]
{Proof}{Proof of}{
    common,
    colframe=black!10,
    separator sign={}
}{pf}

\DeclarePairedDelimiterX\norm[1]\lVert\rVert{\ifblank{#1}{\:\cdot\:}{#1}}
\DeclarePairedDelimiterX\abs[1]\lvert\rvert{\ifblank{#1}{\:\cdot\:}{#1}}
\DeclarePairedDelimiter\set\{\}
\DeclareMathOperator*{\argmax}{arg\,max}
\DeclareMathOperator*{\argmin}{arg\,min}
\DeclareMathOperator*{\arginf}{arg\,inf}
\DeclareMathOperator*{\argsup}{arg\,sup}

% just to make sure it exists
\providecommand\onto{}
% can be useful to refer to this outside \Set
\newcommand\SetSymbol[1][]{%
    \nonscript\:#1\vert{}
    \allowbreak\nonscript\:
    \mathopen{}}
\DeclarePairedDelimiterX\Set[1]\{\}{%
    \renewcommand\given{:}
    #1
}

\AtBeginDocument{%
    \let\mathbb\relax
    \DeclareMathAlphabet{\mathbb}{U}{msb}{m}{n}%
}

\newenvironment{tightcenter}{%
    \setlength\topsep{0pt}%
    \setlength\parskip{0pt}%
    \par\centering}{\par\noindent\ignorespacesafterend}

\usepackage{nicematrix}
\setcounter{secnumdepth}{3}
\setcounter{tocdepth}{3}

\hypersetup{pdftitle={Applied Linear Models (STAT 331)},
pdfauthor={Cameron Roopnarine, Samuel Wong},
pdfsubject={Statistics},
pdfkeywords={University of Waterloo, Fall 2020 (1209)}}

\title{
\LARGE Applied Linear Models\\
\large STAT 331\\
\normalsize Fall 2020 (1209)\thanks{Online Course}}
\author{Cameron Roopnarine\thanks{\LaTeX{}er}\and Samuel Wong\thanks{Instructor}}%
\date{\today}
\IfFileExists{upquote.sty}{\usepackage{upquote}}{}
\begin{document}

\maketitle

\tableofcontents



\makeheading{Lecture 1 | 2020-09-08}
Regression model infers the relationship between:
\begin{itemize}
    \item Response (dependent) variable:
          variable of primary interest, denoted by
          a capital letter such as $ Y $.
    \item Explanatory (independent) variables:
          (covariates, predictors, features) variables
          that potentially impact response,
          denoted $ (x_1,x_2,\ldots,x_p) $.
\end{itemize}
\underline{Alligator data}:
\begin{itemize}
    \item length (m) $ Y $
    \item male/female (categorical, 0 or 1) $ x_1 $
\end{itemize}
Mass in stomach:
\begin{itemize}
    \item fish $ x_2 $
    \item invertebrates $ x_3 $
    \item reptiles $ x_4 $
    \item birds $ x_5 $
    \item other $ x_6 $
\end{itemize}
We imagine we can explain $ Y $ in terms
of $ (x_1,\ldots,x_p) $ using some function
so that $ Y=f(x_1,\ldots,x_p) $.

In this course, we will be looking at linear models.

Linear regression model assumes that
\[ Y=\beta_0+\beta_1x_1+\cdots+\beta_p x_p+\varepsilon \]
\begin{itemize}
    \item $ Y $ value of response
    \item $ x_1,\ldots,x_p $ values of $ p $ explanatory variables
          (assumed to be fixed constants)
    \item $ \beta_0,\beta_1,\ldots,\beta_p $ model parameters
          \begin{itemize}
              \item $ \beta_0 $ intercept, expected value of $ Y $
                    when all $ x_j=0 $.
              \item $ \beta_1,\ldots,\beta_p $ quantify effect on $ x_j $
                    on $ Y $, $ j=1,\ldots,p $
              \item $ \varepsilon $ random error ``all models are wrong, but some
                    are useful''
          \end{itemize}
\end{itemize}
Assume $ \varepsilon \thicksim N(0,\sigma^2) $.
In general, the model will not perfectly explain the data.

Q\@: What is the distribution of $ Y $ under these assumptions?

$ \E{Y}=\beta_0+\beta_1x_1+\cdots+\beta_p x_p $

$ \Var{Y}=\Var{\varepsilon}=\sigma^2 $.

$ Y \thicksim N(\beta_0+\beta_1x_1+\cdots+\beta_p x_p,\sigma^2) $

\chapter{Review}
\makeheading{Lecture 2 | 2020-09-08}

\underline{Definitions}:
\begin{Definition}{Graph, Vertices, Edges}{}
    A \textbf{graph} $ G=(V,E) $ consists of a set of \textbf{vertices}
    $ V $, and a set of \textbf{edges} $ E $ that are unordered pairs of vertices.
\end{Definition}
\begin{Definition}{Degree}{}
    The degree of a vertex $ v $ is the number of edges incident with $ v $,
    denoted $ d_G(v) $ or $ d(v) $.
\end{Definition}
\begin{Definition}{Walk}{}
    A \textbf{walk} is a sequence of vertices $ v_1,v_2,\ldots,v_k $
    where $ v_i v_{i+1} $ are edges.
\end{Definition}
\begin{Definition}{Path}{}
    A \textbf{path} is a walk where all vertices are distinct.
\end{Definition}
\begin{Definition}{Cycle}{}
    A \textbf{cycle} is a walk $ v_1,v_2,\ldots,v_k,v_1 $
    where $ v_1,\ldots,v_k $ are distinct and $ k\geqslant 3 $.
\end{Definition}

\underline{Connectivity and cuts}:
\begin{Definition}{Connected, Disconnected}{}
    A graph is \textbf{connected} if there is a path between every pair of vertices.
    Otherwise it is \textbf{disconnected.}
\end{Definition}
\begin{Definition}{Cut}{}
    For $ S\subseteq V $, the cut induced by $ S $ is the set of all edges with exactly one end
    in $ S $.
    \[ \delta(S)=\set{uv\in E: u\in S, v\notin S} \]
\end{Definition}
\begin{Definition}{$ s,t $-cut}{}
    If $ s,t\in V $ where $ s\in S $ and $ t\notin S $, then $ \delta(S) $ is an
    \textbf{$\bm{s}$,$\bm{t}$-cut}.
\end{Definition}
\begin{Proposition}{}{}
    There exists an $ s,t $-path if and only of every $ s,t $-cut
    is non-empty.
\end{Proposition}

\underline{Trees}:
\begin{Definition}{Tree}{}
    A \textbf{tree} is a connected graph with no cycles.
\end{Definition}
\begin{Proposition}{}{}
    A tree with $ n $ vertices has $ n-1 $ edges.
\end{Proposition}
\begin{Definition}{Spanning tree}{}
    A \textbf{spanning tree} of a graph $ G $ is a subgraph
    that is a tree which uses all vertices of $ G $.
\end{Definition}
\begin{Proposition}{}{}
    $ G $ has a spanning tree if and only if $ G $ is connected.
\end{Proposition}
\begin{Proposition}{}{}
    If $ T $ is a tree and $ u,v $ are not adjacent, then
    $ T+uv $ has exactly one cycle $ C $. If $ xy $ is an edge of
    $ C $, then $ T+uv-xy $ is another tree.
\end{Proposition}

\makeheading{Lecture 3 | 2020-09-13}
We first introduce a function that will be used.

\begin{Definition}{Gamma function}{}
    The \textbf{gamma function},
    denoted $ \Gamma(\alpha) $ for all $ \alpha>0 $, is given by
    \[ \Gamma(\alpha)=\int_{0}^{\infty} x^{\alpha-1}e^{-x}\, d{x}  \]
\end{Definition}
\begin{Proposition}{Properties of the Gamma Function}{prop_gamma}
    \begin{enumerate}[label=(\arabic*)]
        \item\label{gamma_prop_1}$ \Gamma(\alpha)=(\alpha-1)\Gamma(\alpha-1) $ for $ \alpha>1 $
        \item\label{gamma_prop_2} $ \Gamma(n)=(n-1)! $ when $ n\geqslant 1 $ is a positive integer
        \item\label{gamma_prop_3} $ \displaystyle \Gamma\left( \frac{1}{2} \right)=\sqrt{\pi} $
    \end{enumerate}
\end{Proposition}
We don't need to know the following proof, but I checked it out for fun. Content not found in the
syllabus is usually labelled with a dagger ($ \dagger $).
\begin{Proof}{$ \dagger $~\ref{prop:prop_gamma}}{}
    Proof of~\ref{gamma_prop_1}. Suppose $ \alpha>1 $.
    \[ \Gamma(\alpha)=\int_{0}^{\infty} x^{\alpha-1}e^{-x}\, d{x}  \]
    Let $ u=x^{\alpha-1} \implies du=(\alpha-1)x^{\alpha-2}\,dx $ and $ dv=e^{-x}\,dx\implies v=-e^{-x} $. Now,
    recall from MATH 138:
    \[ \int u\, d{v} =uv-\int v\, d{u} \]
    So,
    \begin{align*}
        \Gamma(\alpha)
         & = \left[ (\alpha-1)x^{\alpha-2}\left( -e^{-x} \right) \right]_0^\infty-\int_{0}^{\infty}\left(-e^{-x}\right)
        (\alpha-1)x^{\alpha-2}\, d{x}                                                                                   \\
         & =0+(\alpha-1)\int_{0}^{\infty}e^{-x} x^{\alpha-2}\, d{x}                                                     \\
         & =(\alpha-1)\Gamma(\alpha)
    \end{align*}
    Proof of~\ref{gamma_prop_2}. Using~\ref{gamma_prop_1}:
    \begin{align*}
        \Gamma(\alpha)
         & =(\alpha-1)\Gamma(\alpha-1)                    \\
         & =(\alpha-1)(\alpha-2)\Gamma(\alpha-3)          \\
         & =(\alpha-1)(\alpha-2)\cdots (3)(2)(1)\Gamma(1) \\
    \end{align*}
    We know that $ \Gamma(1)=1 $ by using the definition (trivial), therefore the result now follows.

    Proof of~\ref{gamma_prop_3}. Sketch:
    \begin{itemize}
        \item Let $ u=x^2 $, so $ du=2x\,dx $. Let $ \alpha=\dfrac{1}{2} $, so the integral looks like:
              \[ \Gamma\left( \frac{1}{2}  \right)=2\int_{0}^{\infty} e^{-u^2}\, d{u}  \]
        \item Compute $ \left[ \Gamma \left( \frac{1}{2}  \right) \right]^2 $. Using polar coordinates,
              compute the following double integral.
              \[ 4 \int_{0}^{\infty} \int_{0}^{\infty} e^{-(u^2+v^2)}\, d{v} \, d{u}  \]
              One will have to compute the Jacobian Matrix.
        \item Solve for $ \Gamma\left( \dfrac{1}{2} \right) $ explicitly now.
    \end{itemize}
    \underline{Author's note}: This was covered in MATH 237 when I took it (F19).
\end{Proof}

\begin{Example}{}{}
    The p.d.f.\ is given by
    \[ f(x)=\begin{dcases}
            \frac{x^{\alpha-1}e^{-x/\beta}}{\Gamma(\alpha)\beta^\alpha} & x>0          \\
            0                                                           & x\leqslant 0
        \end{dcases}
    \]
    when $ \alpha>0 $ and $ \beta>0 $. We say that $ X \sim \gam{\alpha,\beta} $.

    We also say that $ \alpha $ is the scale parameter and $ \beta $ is the shape parameter
    for this distribution.

    Verify that $ f(x) $ is a p.d.f.

    \textbf{Solution.} Showing $ f(x)\geqslant 0 $ is trivial. Now,
    \[ \int_{-\infty}^{\infty} f(x)\, d{x} =
        \int_{0}^{\infty} \frac{x^{\alpha-1}e^{-x/\beta}}{\Gamma(\alpha)
        \beta^{\alpha}} \, d{x}  \]
    Let $ y=x/\beta\implies x=y\beta $ and $ dx=\beta\,dy $.
    Therefore,
    \[\int_{-\infty}^{\infty} f(x)\, d{x} =\int_{0}^{\infty}
        \frac{y^{\alpha-1}\beta^{\alpha-1}e^{-y}}{\Gamma(\alpha)\beta^\alpha}(\beta)  \, d{y}
        =\frac{1}{\Gamma(\alpha)}\int_{0}^{\infty} y^{\alpha-1}e^{-y}\, d{y}=1    \]
\end{Example}

\begin{Example}{}{}
    Suppose the p.d.f.\ is given by
    \[ f(x)=\begin{dcases}
            \frac{\beta}{\theta^\beta}x^{\beta-1}e^{-\left( x/\theta  \right)^\beta} & x>0          \\
            0                                                                        & x\leqslant 0
        \end{dcases} \]
    with $ \alpha>0 $ and $ \beta>0 $.
    Then, $ X \sim \weib{\theta,\beta} $.
    Verify that $ f(x) $ is a p.d.f.

    \textbf{Solution.} $ f(x)\geqslant 0 $ for every $ x\in\mathbb{R} $. Now,
    \[ \int_{-\infty}^{\infty} f(x)\, d{x} =
        \int_{0}^{\infty} \frac{\beta}{\theta^\beta}x^{\beta-1}e^{-\left( x/\theta  \right)^\beta} \, d{x} \]
    Let $ y=(x/\theta)^\beta \implies
        x=\theta y^{1/\beta} $ and $ dx=(\theta/\beta) y^{(1/\beta)-1}\,dy $.
    Therefore,
    \[ \int_{-\infty}^{\infty} f(x)\, d{x}=\int_{0}^{\infty} \frac{\beta}{\theta^\beta} \theta^{\beta-1}
        y^{(\beta-1)/\beta}e^{-y}\frac{\theta}{\beta} y^{(1/\beta)-1}\, d{y}
        =\int_{0}^{\infty} e^{-y}\, d{y}=\Gamma(1)=1  \]
\end{Example}

\begin{Example}{Normal}{}
    The p.d.f.\ is given by
    \[ f(x)=\frac{1}{\sqrt{2\pi}\sigma}\exp\left\{ -\frac{(x-\mu)^2}{2\sigma^2} \right\}  \]
    for $ x\in\mathbb{R} $, $ \mu\in\mathbb{R} $, $ \sigma^2>0 $.
    Verify that $ f(x) $ is a p.d.f.

    \textbf{Solution.}

    $ f(x)\geqslant 0 $ obviously.

    \underline{Case 1}: $ \mu=0 $ and $ \sigma^2=1 $, then
    we say $ X $ follows a \textbf{standard normal} distribution.
    We want to show that
    \[ \int_{-\infty}^{\infty} \frac{1}{\sqrt{2\pi}}\exp\left\{ -\frac{x^2}{2} \right\} \, d{x}=1  \]
    Since the function is symmetrical around 0, we have the following equivalent integral.
    \[ 2\int_{0}^{\infty} \frac{1}{\sqrt{2\pi}}\exp\left\{ -\frac{x^2}{2} \right\} \, d{x}  \]
    Let $ y=x^2/2\implies x=\sqrt{2y} $
    and $ dx=\dfrac{\sqrt{2}}{2} y^{-1/2}\,dy $. Therefore,
    \[ =\frac{2}{\sqrt{2\pi}}\int_{0}^{\infty} e^{-y}\frac{\sqrt{2}}{2} y^{-1/2}\, d{y}
        =\frac{1}{\sqrt{\pi}}\int_{0}^{\infty} y^{1/2-1}e^{-y}\, d{y}=
        \left( \frac{1}{\sqrt{\pi}}  \right)\Gamma\left( \frac{1}{2} \right)=1    \]
    \underline{Case 2}: For general $ \mu $ and $ \sigma^2 $,
    \[ \int_{-\infty}^{\infty}  \frac{1}{\sqrt{2\pi}\sigma}\exp\left\{ -\frac{(x-\mu)^2}{2\sigma^2} \right\} \, d{x} \]
    Let $ z=\dfrac{x-\mu}{\sigma}\implies x=\mu+\sigma z  $
    and $ dx=\sigma\,dz $. Therefore,
    \[ =\int_{-\infty}^{\infty} \frac{1}{\sqrt{2\pi}\sigma}
        \exp\left\{ -\frac{z^2}{2}\right\}\sigma \, d{z}=
        \int_{-\infty}^{\infty} \frac{1}{\sqrt{2\pi}}\exp\left\{ -\frac{z^2}{2} \right\} \, d{z}=1   \]
    using Case 1.
\end{Example}
\section{Expectation}
\begin{Definition}{Expectation of discrete random variable}{}
    Suppose $ X $ is a discrete random variable with support
    $ A $ and p.f. $ f(x) $. Then,
    \[ \E{X}=\sum\limits_{x\in A}x f(x)  \]
    if $ \sum\limits_{x\in A}\abs{x}f(x) < \infty $ (finite).
    If $ \sum\limits_{x\in A}\abs{x}f(x) =\infty $ (infinite), then
    $ \E{X} $ does not exist.
\end{Definition}

\begin{Definition}{Expectation of continuous random variable}{}
    Suppose $ X $ is a continuous random variable with support $ A $
    and p.d.f. $ f(x) $. Then,
    \[ \E{X}=\int_{-\infty}^{\infty} x f(x)\, d{x}  \]
    if $ \displaystyle \int_{-\infty}^{\infty} \abs{x}f(x)\, d{x} <\infty $
    (finite). Similarly,
    if $ \displaystyle \int_{-\infty}^{\infty} \abs{x}f(x)\, d{x} =\infty $
    (infinite),
    then $ \E{X} $ does not exist.
\end{Definition}

\begin{Example}{Discrete}{}
    Suppose
    \[ f(x)=\frac{1}{x(x+1)}=\frac{1}{x} -\frac{1}{x+1}  \]
    for $ x=1,2,\ldots $. The support set is $ A=\set{1,2,\ldots} $.
    We note that $ f(x) $ is a p.f.\ since
    $ f(x)\geqslant 0 $ and
    \[ \sum\limits_{x\in A}f(x)=\sum\limits_{x=1}^{\infty}
        \left( \frac{1}{x} -\frac{1}{x+1} \right)=
        1-\frac{1}{2} +\frac{1}{2} -\frac{1}{3} +\cdots=1  \]
    Find $ \E{X} $.

    \textbf{Solution.}
    \[ \sum\limits_{x\in A}\abs{x}f(x)=\sum\limits_{x=1}^{\infty}
        x\left( \frac{1}{x} -\frac{1}{x+1} \right)
        =\sum\limits_{x=1}^{\infty} \frac{1}{x+1} =\infty  \]
    Therefore, $ \E{X} $ does not exist!
\end{Example}

\begin{Example}{Continuous}{}
    Let the p.d.f.\ be defined as $ f(x)=\dfrac{1}{x^2+1} $ for
    $ x\in\mathbb{R} $. This is known as the Cauchy distribution
    (or Student's T-distribution with 1 degree of freedom). Find $ \E{X} $.

    \textbf{Solution.}
    \[ \int_{-\infty}^{\infty} \abs{x}f(x)\, d{x} =
        \int_{-\infty}^{\infty} \abs{x}\frac{1}{x^2+1} \, d{x}=
        2 \int_{0}^{\infty} \frac{x}{x^2+1} \, d{x} =
        \bigl[\ln\abs{x^2+1}\bigr]_0^\infty=\infty \]
    $ \E{X} $ does not exist! The following is \underline{\textbf{wrong}}:
    \[ \E{X}=\int_{-\infty}^{\infty}x f(x)\, d{x}=
        \int_{-\infty}^{\infty} \frac{x}{1+x^2} \, d{x}=0  \]
    since the integral above with $ \abs{x} $ is infinite. You must
    always remember to check that the $ \E{X} $ is finite
    (using $ \abs{X} $) for both the discrete and continuous case.
\end{Example}

\begin{Example}{Bernoulli and Binomial Random Variable}{}
    Suppose $ X \sim \bern{p} $.
    \[ \Prob{X=1}=p\quad\text{ and }\quad \Prob{X=0}=1-p \]
    We know $ \E{X}=(1)\Prob{X=1}+(0)\Prob{X=0}=p $

    Now suppose
    $ X \sim \bin{n,p} $. Find $ \E{X} $.

    \textbf{Solution.}
    \[ \E{X}=\sum\limits_{x\in A}x f(x)=\sum\limits_{x=0}^{n} x
        \binom{n}{x}p^x(1-p)^{n-x}  \]
    This is hard to do. But, we know we can use the
    relationship between the Binomial and Bernoulli random variable
    so,
    \[ X=\sum\limits_{i=1}^{n} X_i \]
    Therefore,
    \[ \E{X}=\E[\bigg]{\sum\limits_{i=1}^{n} X_i}=\sum\limits_{i=1}^{n}
        \E{X_i}=np \]
\end{Example}
\begin{Example}{}{}
    Suppose for a random variable $ X $ the p.d.f.\ is given by
    $ f(x)=\dfrac{\theta}{x^{\theta+1}} $
    for $ x\geqslant 1 $ and $ 0 $ when $ x<1 $. Assume $ \theta>0 $.
    Find $ \E{X} $ and for what values of $ \theta $,
    does $ \E{X} $ exist.

    \textbf{Solution.}
    \[ \int_{-\infty}^{\infty} \abs{x}f(x)\, d{x}=
        \int_{1}^{\infty} (x) \frac{\theta}{x^{\theta+1}} \, d{x}
        =\theta \int_{1}^{\infty} \frac{1}{x^{\theta}} \, d{x} <\infty
        \iff \theta>1 \]
    from MATH 138. So, if $ \theta>1 $ then $ \E{X} $ exists. Also,
    \[ \E{X}=\int_{-\infty}^{\infty} xf(x)\, d{x}=
        \int_{1}^{\infty} \frac{\theta x}{x^{\theta+1}} \, d{x}
        =\theta \int_{1}^{\infty} \frac{1}{x^{\theta}} \, d{x}
        =\frac{\theta}{\theta-1}   \]

\end{Example}

\begin{Definition}{Expectation (Discrete)}{}
    If $ X $ is a discrete random variable with probability
    function $ f(x) $ and support set $ A $,
    then the \textbf{expectation} of the random variable $ g(X) $
    is defined by
    \[ \E{g(X)}=\sum\limits_{x\in A}g(x)f(x) \]
    provided the sum converges absolutely; that is, provided
    \[ \sum\limits_{x\in A}\abs{g(x)}f(x)<\infty \]
\end{Definition}

\begin{Definition}{Expectation (Continuous)}{}
    If $ X $ is a continuous random variable with p.d.f.
    $ f(x) $ and support set $ A $,
    then the \textbf{expectation} of the random variable $ g(X) $
    is defined by
    \[ \E{g(X)}=\int_{-\infty}^{\infty} g(x)f(x)\, d{x} \]
    provided the integral converges absolutely; that is, provided
    \[ \int_{-\infty}^{\infty} \abs{g(x)}f(x)\, d{x} <\infty \]
\end{Definition}

\begin{Theorem}{Expectation is a Linear Operator}{exp_linear_op}
    Suppose $ X $ is a random variable with probability (density)
    function $ f(x) $, and $ a $ and $ b $ are real constants,
    and $ g(x) $ and $ h(x) $ are real-valued functions. Then,
    \[ \E{aX+b}=a\E{X}+b \]
    \[ \E{a g(X)+b h(X)}=a\E{g(X)}+b\E{h(X)} \]
\end{Theorem}
\begin{Proof}{\ref{thm:exp_linear_op}}{}
    Omitted from the lecture and hence these notes.
    See Course Notes for most of the proof.
\end{Proof}
\begin{Definition}{Variance}{}
    The variance of a random variable is defined as
    \[ \sigma^2=\Var{X}=\E{(X-\mu)^2} \]
    where $ \mu=\E{X} $.
\end{Definition}
\begin{Definition}{Special Expectations}{}
    \begin{enumerate}[label=(\Roman*)]
        \item The $ k $th moment (about the origin) of a random variable
              \[ \E{X^k} \]
        \item The $ k $th moment about the mean of a random variable
              \[ \E{(X-\mu)^k} \]
    \end{enumerate}
\end{Definition}
\begin{Theorem}{Properties of Variance}{prop_var}
    If $ X $ is a random variable, then
    \[ \Var{X}=\E{X^2}-\mu^2 \]
    where $ \mu=\E{X} $. Note that the
    variance of $ X $ exists if $ \E{X^2}<\infty $.
\end{Theorem}
\begin{Proof}{\ref{thm:prop_var}}{}
    Omitted from the lecture and hence these notes.
    See Course Notes for most of the proof.
\end{Proof}

\begin{Example}{}{}
    Suppose $ X \sim \poi{\theta} $, the p.f.\ is defined as
    $ f(x)=\dfrac{\theta^x}{x!} e^{-\theta} $
    for $ x=0,1,2,\ldots $. Find $ \E{X} $ and $ \Var{X} $.

    \textbf{Solution.} The support is non-negative, so $ \abs{x}=x $.
    Therefore,
    \[ \E{X}=\sum\limits_{x=0}^{\infty} x \frac{\theta^x}{x!} e^{-\theta}
        =\sum\limits_{x=1}^{\infty} \frac{x}{x!} \theta^x e^{-\theta}
        =\theta \sum\limits_{x=1}^{\infty} \frac{\theta^{x-1}}{(x-1)!}
        e^{-\theta}  \]
    Let $ y=x-1 $, then
    \[ \E{X}= \theta\sum\limits_{y=0}^{\infty} \frac{\theta^y}{y!} e^{-\theta} \]
    We know $\displaystyle  e^{\theta}=\sum\limits_{y=0}^{\infty} \frac{\theta^y}{y!} $,
    so $ \E{X}= \theta $.

    \[ \Var{X}=\E{X^2}-\mu^2 \]
    Let's find $ \E{X^2} $:
    \begin{align*}
        \E{X^2}
         & =\sum\limits_{x=0}^{\infty} x^2 \frac{\theta^x}{x!} e^{-\theta}         \\
         & =\sum\limits_{x=1}^{\infty} \frac{x}{(x-1)!}\theta^x e^{-\theta}        \\
         & =\sum\limits_{x=1}^{\infty} \frac{(x-1)+1}{(x-1)!} \theta^x e^{-\theta} \\
         & =\sum\limits_{x=1}^{\infty} \frac{x-1}{(x-1)!}\theta^x e^{-\theta}+
        \sum\limits_{x=1}^{\infty} \frac{1}{(x-1)!}\theta^x e^{-\theta}
    \end{align*}
    Looking at the first sum (since the second sum was computed before):
    \[ \sum\limits_{x=2}^{\infty} \frac{\theta^2}{(x-2)!} \theta^{x-2}e^{-\theta}+\theta \]
    Let $ y=x-2 $:
    \[ \E{X^2}=\sum\limits_{y=0}^{\infty}\frac{\theta^2\theta^y}{y!}e^{-\theta}+\theta=
        \theta^2+\theta   \]
    Therefore,
    \[ \Var{X}=\E{X^2}-\mu^2=(\theta^2+\theta)-\theta^2=\theta \]
\end{Example}

\makeheading{Lecture 4 | 2020-09-16}
\section{Prediction}
Suppose we want to predict the response $ y $
for a new value of $ x $, say $ x=x_0 $. Then,
SLR model says
$ Y_0 \sim \N{\beta_0+\beta_1 x_0,\sigma^2} $
where $ Y_0 $ is a random variable for response when $ x=x_0 $;
that is, $ \hat{Y}_0=\hat{\beta}_0+\hat{\beta}_1x_0 $.
The fitted model predicts the \emph{value} of $ y $
to be $ \hat{y}_0=\hat{\beta}_0+\hat{\beta}_1x_0 $.

Also, $\E{\hat{Y}_0}=\E{\hat{\beta}_0}+x_0\E{\hat{\beta}_1}=
  \beta_0+\beta_1x_0=\E{Y_0} $,
since $ \hat{\beta}_i $ for $ i=0,1 $ are unbiased.
Therefore, we can say that $ \hat{Y}_0 $ is an unbiased estimate
of the random variable for the mean of $ Y_0 $. For the variance
of $ \hat{Y}_0 $ we write
\begin{align*}
  \hat{Y}_0
   & =\hat{\beta}_0+\hat{\beta}_1x_0                                                                            \\
   & =\underbrace{\bar{Y}-\hat{\beta}_1\bar{x}}_{\hat{\beta}_0}+\hat{\beta}_1x_0                                \\
   & =\bar{Y}+\hat{\beta}_1(x_0-\bar{x})                                                                        \\
   & =\bar{Y}+\underbrace{\frac{\sum_{i=1}^{n}(x_i-\bar{x})(Y_i-\bar{Y})}{S_{xx}}}_{\hat{\beta}_1}(x_0-\bar{x}) \\
   & =\sum_{i=1}^{n} \biggl[ \frac{Y_i}{n} +(x_0-\bar{x})
  \frac{(x_i-\bar{x})(Y_i-\bar{Y})}{S_{xx}}  \biggr]                                                            \\
   & =\sum_{i=1}^{n} \biggl[ \frac{Y_i}{n} +(x_0-\bar{x})
  \frac{(x_i-\bar{x})Y_i}{S_{xx}} \biggr]                                                                       \\
   & =\sum_{i=1}^{n} \biggl[ \frac{1}{n} +\frac{(x_0-\bar{x})(x_i-\bar{x})}{S_{xx}} \biggr]
  Y_i                                                                                                           \\
   & =\sum_{i=1}^{n} a_i Y_i
\end{align*}
where $ \displaystyle  a_i=\frac{1}{n} +\frac{(x_0-\bar{x})(x_i-\bar{x})}{S_{xx}} $.
Therefore, we have
\begin{align*}
  \Var{\hat{Y}_0}
   & =\Var*{\sum_{i=1}^{n}a_i Y_i}                                    \\
   & =\sum_{i=1}^{n}a_i^2\Var{Y_i}+\sum_{i\ne j}a_i a_j\Cov{Y_i,Y_j}  \\
   & =\sum_{i=1}^{n} \biggl[ \frac{1}{n} +
  \frac{(x_0-\bar{x})(x_i-\bar{x})}{S_{xx}} \biggr]^2\sigma^2+0       \\
   & = \sigma^2\sum_{i=1}^{n} \biggl[ \frac{1}{n^2}+
    \frac{2(x_0-\bar{x})(x_i-\bar{x})}{nS_{xx}}+
  \frac{(x_0-\bar{x})^2(x_i-\bar{x})^2}{(S_{xx})^2}  \biggr]          \\
   & =\sigma^2\Set*{\sum_{i=1}^{n} \frac{1}{n^2}+
    \frac{2(x_0-\bar{x})}{n S_{xx}}\sum_{i=1}^{n} (x_i-\bar{x})
  +\frac{(x_0-\bar{x})^2}{(S_{xx})^2} \sum_{i=1}^{n} (x_i-\bar{x})^2} \\
   & =\sigma^2\Set*{\frac{1}{n} +\frac{2(x_0-\bar{x})}{S_{xx}}(0)+
  \frac{(x_0-\bar{x})^2}{(S_{xx})^2}(S_{xx})}                         \\
   & =\sigma^2\Set*{\frac{1}{n}+\frac{(x_0-\bar{x})^2}{S_{xx}}},
\end{align*}
noting that $ \Cov{Y_i,Y_j}=0 $ for all $ i\ne j $ since $ \varepsilon_i\indep \varepsilon_j $ for all $ i\ne j $. Hence,
we proved the following theorem.
\begin{Theorem}{Distribution of Prediction}{}
  The distribution of the prediction random variable is given by
  \[ \hat{Y}_0 \sim \N*{\beta_0+\beta_1x_0,
      \sigma^2\biggl( \frac{1}{n} +\frac{(x_0-\bar{x})^2}{S_{xx}}  \biggr)} \]
\end{Theorem}
\begin{Definition}{Prediction error}{}
  The random variable for \textbf{prediction error} is defined as
  $ Y_0-\hat{Y}_0 $
  where $ Y_0 $ and $ \hat{Y}_0 $ are independent
  and $ \hat{Y}_0 $ is a function of $ Y_1,\ldots,Y_n $.
\end{Definition}
\[ \E{Y_0-\hat{Y}_0}=\E{Y_0}-\E{\hat{Y}_0}=0 \]
\[ \Var{Y_0-\hat{Y}_0}=\Var{Y_0}+(-1)^2\Var{\hat{Y}_0}
  =\sigma^2+\sigma^2\biggl(\frac{1}{n} +\frac{(x_0-\bar{x})^2}{S_{xx}} \biggr)
\]
We proved the following theorem.
\begin{Theorem}{Distribution of Prediction Error}{}
  The distribution of the prediction error is given by
  \[ Y_0-\hat{Y}_0
    \sim \N*{0,\sigma^2\biggl( 1+\frac{1}{n}+\frac{(x_0-\bar{x})^2}{S_{xx}}  \biggr)} \]
\end{Theorem}
Since $ \sigma $ is unknown, we use $ \hat{\sigma} $ and get the following:
\[ \frac{Y_0-\hat{Y}_0}{
    \hat{\sigma}\sqrt{1+\dfrac{1}{n}+\dfrac{(x_0-\bar{x})^2}{S_{xx}}}
  } \sim t(n-2) \]
Intuition for prediction error composed of 2 terms:
\begin{itemize}
  \item $ \Var{Y_0} $: random error of new observation
  \item $ \Var{\hat{Y}_0} $ (predictor): estimating $ \beta_0 $ and $ \beta_1 $
\end{itemize}
Those are 2 sources of uncertainty.

\begin{Remark}{}{}
  Be careful that the prediction may not make sense if
  $ x_0 $ is outside the range of the $ x_i $'s in the data.
\end{Remark}

A $ (1-\alpha) $ prediction interval for the mean response $ y_0=\beta_0+\beta_1x_0 $
at $ x_0 $
is
\[ \hat{y}_0\pm c\, \hat{\sigma}\sqrt{1+\dfrac{1}{n}+\dfrac{(x_0-\bar{x})^2}{S_{xx}}}
\]
where $ c $ is the $ 1-\dfrac{\alpha}{2} $ quantile of $ t(n-2) $.
\section{Application}
\begin{Example}{Orange production 2018 in FL}{}
  We are given the following information.
  \begin{itemize}
    \item $ x $: acres
    \item $ y $: \# boxes of oranges (thousands)
    \item $ (x_i,y_i) $ recorded for each of 25 FL counties
    \item $ r=0.964 $
    \item $ \bar{x}=16133 $
    \item $ \bar{y}=1798 $
    \item $ S_{xx}=1.245\times 10^{10} $
    \item $ S_{xy}=1.453\times 10^9 $
  \end{itemize}
  Now,
  $ \hat{\beta}_1=S_{xy}/S_{xx}=0.1167 $
  has a positive slope, therefore $ x $ and $ y $ are
  positively correlated.
  The expected number of boxes produced is estimated to be about 117
  higher per an additional acre.

  Computing
  $ \hat{\beta}_0=\bar{y}-\hat{\beta}_1\bar{x}=-85.3 $,
  we see that it is
  not meaningful to interpret, since it
  is the expected production if there were 0 acres
  (outside the range of $ x_i $) as no county has $ x=0 $.

  Now suppose $ \SS{Res}=1.31\times 10^7 $
  the residuals are the differences between $ y_i $ and the fitted regression
  line.
  \begin{itemize}
    \item $ \hat{\sigma}^2=\dfrac{\sum_{i=1}^{n} e_i^2}{n-2}=
            \dfrac{1.31\times 10^7}{25-2}=5.7\times 10^5 $
    \item $ \Se*{\hat{\beta}_1}=\dfrac{\hat{\sigma}}{\sqrt{S_{xx}}}=0.00676 $
    \item To test $ H_0 $: $ \beta_1 =0 $,
          calculate
          $ t=(\hat{\beta}_1-0)/\Se*{\hat{\beta}_1}=0.1167/0.00676
            \approx 17.3 $,
          then elect the $ 0.975 $ quantile (for demonstration purposes) of $ t(23) $
          which is $ 2.07 $.
    \item Note that $ 17.3 $ is very unlikely to see in $ t(23) $.
  \end{itemize}
  Since $ 17.3\gg2.07 $, we reject $ H_0 $ at $ \alpha=0.05 $
  level, and conclude there's a significant linear relationship between
  acres and oranges produced.

  The $ 95\% $ confidence interval for $ \beta_1 $ is given by
  $ 0.1167\pm 2.07(0.00676) $,
  which does not contain $ 0 $.
  \[ p\text{-value}=P(\abs{t_{23}}\geqslant 17.3)=
    2P(t_{23}\geqslant 17.3)\approx 1.2\times 10^{-14} \]
  Predict the \# of boxes in thousands produced if we had
  10000 acres to grow oranges.
  \[ \hat{\beta}_0+\hat{\beta}_1x_0=-85.3+(0.1167)(10000)\approx 1082 \]
  The 95\% prediction interval is given by
  \[ 1082\pm 2.07\sqrt{5.69\times 10^5}\sqrt{1+\frac{1}{25}+
    \frac{(6133)^2}{1.245\times 10^{10}} }=
      [-512.0407, 2675.595] \]
  \begin{Remark}{}{}
    We are \textbf{not} trying to establish causation.
  \end{Remark}
\end{Example}


\subsection{R Demo}
\begin{knitrout}
\definecolor{shadecolor}{rgb}{0.969, 0.969, 0.969}\color{fgcolor}\begin{kframe}
\begin{alltt}
\hlcom{# Read data from florange.csv and input it into the dat}
\hlcom{# vector.}
\hlstd{dat} \hlkwb{<-} \hlkwd{read.csv}\hlstd{(}\hlstr{"csv/florange.csv"}\hlstd{)}
\hlcom{# Done to make the predict function work well.}
\hlstd{x} \hlkwb{<-} \hlstd{dat}\hlopt{$}\hlstd{acres}
\hlstd{y} \hlkwb{<-} \hlstd{dat}\hlopt{$}\hlstd{boxes}
\hlcom{# Output the first 6 rows in dat.}
\hlkwd{head}\hlstd{(dat)}
\end{alltt}
\begin{verbatim}
##      county boxes acres
## 1   Brevard    51   696
## 2 Charlotte   821 13447
## 3   Collier  2088 29351
## 4    DeSoto  7688 66365
## 5    Glades   368  5396
## 6    Hardee  5306 43126
\end{verbatim}
\begin{alltt}
\hlcom{# Draw a scatterplot with x-axis as 'acres' and y-axis as}
\hlcom{# 'boxes'.}
\hlkwd{plot}\hlstd{(x, y)}
\end{alltt}
\end{kframe}

{\centering \includegraphics[width=\maxwidth]{figure/unnamed-chunk-14-1} 

}


\begin{kframe}\begin{alltt}
\hlcom{# Compute some common variables with common functions.}
\hlstd{r} \hlkwb{<-} \hlkwd{cor}\hlstd{(x, y)}
\hlstd{xbar} \hlkwb{<-} \hlkwd{mean}\hlstd{(x)}
\hlstd{ybar} \hlkwb{<-} \hlkwd{mean}\hlstd{(y)}
\hlkwd{cat}\hlstd{(}\hlstr{"r:"}\hlstd{, r,} \hlstr{"xbar:"}\hlstd{, xbar,} \hlstr{"ybar:"}\hlstd{, ybar)}
\end{alltt}
\begin{verbatim}
## r: 0.9635098 xbar: 16132.64 ybar: 1797.56
\end{verbatim}
\end{kframe}
\end{knitrout}

Therefore, $r=0.9635098$,
$\bar{x}=16132.64$, and $\bar{y}= 1797.56$.

\begin{knitrout}
\definecolor{shadecolor}{rgb}{0.969, 0.969, 0.969}\color{fgcolor}\begin{kframe}
\begin{alltt}
\hlcom{# Compute some common variables manually.}
\hlstd{Sxx} \hlkwb{<-} \hlkwd{sum}\hlstd{((x} \hlopt{-} \hlstd{xbar)}\hlopt{^}\hlnum{2}\hlstd{)}
\hlstd{Sxy} \hlkwb{<-} \hlkwd{sum}\hlstd{((x} \hlopt{-} \hlstd{xbar)} \hlopt{*} \hlstd{(y} \hlopt{-} \hlstd{ybar))}
\hlkwd{cat}\hlstd{(}\hlstr{"Sxx: "}\hlstd{, Sxx,} \hlstr{"Sxy: "}\hlstd{, Sxy)}
\end{alltt}
\begin{verbatim}
## Sxx:  12450023404 Sxy:  1453128337
\end{verbatim}
\end{kframe}
\end{knitrout}

Therefore, $S_{xx}=12450023404=1.245\times10^{10}$ and
$S_{xy}=1453128337=1.453\times 10^{9}$.

\begin{knitrout}
\definecolor{shadecolor}{rgb}{0.969, 0.969, 0.969}\color{fgcolor}\begin{kframe}
\begin{alltt}
\hlcom{# R's lm function fits linear models}
\hlstd{lm.1} \hlkwb{<-} \hlkwd{lm}\hlstd{(y} \hlopt{~} \hlstd{x)}
\hlkwd{summary}\hlstd{(lm.1)}
\end{alltt}
\begin{verbatim}
## 
## Call:
## lm(formula = y ~ x)
## 
## Residuals:
##      Min       1Q   Median       3Q      Max 
## -2470.81    -6.17    71.72   106.46  1677.32 
## 
## Coefficients:
##               Estimate Std. Error t value Pr(>|t|)    
## (Intercept) -85.391989 186.178031  -0.459    0.651    
## x             0.116717   0.006761  17.263 1.16e-14 ***
## ---
## Signif. codes:  0 '***' 0.001 '**' 0.01 '*' 0.05 '.' 0.1 ' ' 1
## 
## Residual standard error: 754.4 on 23 degrees of freedom
## Multiple R-squared:  0.9284,	Adjusted R-squared:  0.9252 
## F-statistic:   298 on 1 and 23 DF,  p-value: 1.164e-14
\end{verbatim}
\end{kframe}
\end{knitrout}

From the summary, we can see that
$\hat{\beta}_0=-85.391989$,
$\hat{\beta}_1=0.116717$,
$\Se*{\hat{\beta}_1}=0.006761$,
$t=17.263$, $p\text{-value}=1.64\times 10^{-14}$, and
$\hat{\sigma}=754.4$.


\begin{knitrout}
\definecolor{shadecolor}{rgb}{0.969, 0.969, 0.969}\color{fgcolor}\begin{kframe}
\begin{alltt}
\hlcom{# Sum Squared Fitted Values}
\hlkwd{sum}\hlstd{(lm.1}\hlopt{$}\hlstd{fitted.values}\hlopt{^}\hlnum{2}\hlstd{)}
\end{alltt}
\begin{verbatim}
## [1] 250385207
\end{verbatim}
\begin{alltt}
\hlcom{# Sum Squared Residuals}
\hlkwd{sum}\hlstd{(lm.1}\hlopt{$}\hlstd{residuals}\hlopt{^}\hlnum{2}\hlstd{)}
\end{alltt}
\begin{verbatim}
## [1] 13089860
\end{verbatim}
\end{kframe}
\end{knitrout}

Therefore, $\SS{\text{Res}}=\sum_{i=1}^n e_i^2=13089860=1.31\times10^7$.

\begin{knitrout}
\definecolor{shadecolor}{rgb}{0.969, 0.969, 0.969}\color{fgcolor}\begin{kframe}
\begin{alltt}
\hlcom{# Manual calculation of sigma^2 estimate}
\hlkwd{sum}\hlstd{(lm.1}\hlopt{$}\hlstd{residuals}\hlopt{^}\hlnum{2}\hlstd{)}\hlopt{/}\hlnum{23}
\end{alltt}
\begin{verbatim}
## [1] 569124.3
\end{verbatim}
\end{kframe}
\end{knitrout}

    Therefore, $\hat{\sigma}^2=69124.3=5.7\times 10^{5}$.

\begin{knitrout}
\definecolor{shadecolor}{rgb}{0.969, 0.969, 0.969}\color{fgcolor}\begin{kframe}
\begin{alltt}
\hlcom{# Manual calculation of sigma estimate}
\hlkwd{sqrt}\hlstd{(}\hlkwd{sum}\hlstd{(lm.1}\hlopt{$}\hlstd{residuals}\hlopt{^}\hlnum{2}\hlstd{)}\hlopt{/}\hlnum{23}\hlstd{)}
\end{alltt}
\begin{verbatim}
## [1] 754.4033
\end{verbatim}
\end{kframe}
\end{knitrout}

Therefore, $\hat{\sigma}= 754.4$.

\begin{knitrout}
\definecolor{shadecolor}{rgb}{0.969, 0.969, 0.969}\color{fgcolor}\begin{kframe}
\begin{alltt}
\hlcom{# t distribution values}
\hlkwd{qt}\hlstd{(}\hlnum{0.975}\hlstd{,} \hlnum{23}\hlstd{)}
\end{alltt}
\begin{verbatim}
## [1] 2.068658
\end{verbatim}
\end{kframe}
\end{knitrout}

Therefore, $c=2.07$.

\begin{knitrout}
\definecolor{shadecolor}{rgb}{0.969, 0.969, 0.969}\color{fgcolor}\begin{kframe}
\begin{alltt}
\hlcom{# 95% confidence interval}
\hlkwd{confint}\hlstd{(lm.1)}
\end{alltt}
\begin{verbatim}
##                    2.5 %      97.5 %
## (Intercept) -470.5305905 299.7466119
## x              0.1027305   0.1307034
\end{verbatim}
\begin{alltt}
\hlcom{# 95% prediction interval with predicted boxes if we had}
\hlcom{# 10000 acres}
\hlkwd{predict}\hlstd{(lm.1,} \hlkwd{data.frame}\hlstd{(}\hlkwc{x} \hlstd{=} \hlnum{10000}\hlstd{),} \hlkwc{interval} \hlstd{=} \hlstr{"prediction"}\hlstd{)}
\end{alltt}
\begin{verbatim}
##        fit       lwr      upr
## 1 1081.777 -512.0407 2675.595
\end{verbatim}
\end{kframe}
\end{knitrout}

Q: Is $\sigma$ the same for all values of $y$?

A: It appears to not in the sense that the variance
appears to be higher with respect to higher acres.
Sigma will be smaller when there's less acres.
Later, this will be testing equal variance or homoscedastic
assumption. Later, when we talk about variable
transformations we can consider taking the logarithm.

Q: Are the error terms plausibly independent?
In other words,
does knowing one $e_i$ (residual) help predict $e_j$
(another residual) for a different county?

A: There's diagnostics for checking this. However,
intuitively there could be some common factors
at play when two counties are geographically close.
\makeheading{Lecture 5 | 2020-09-20}
\begin{Example}{Uniqueness Theorem}{}
    Suppose $ M_X(t)=(1-2t)^{-1} $. What is the
    distribution of $ X $?

    \textbf{Solution.} $ X \sim \gam{\alpha=1,\beta=2}$.
\end{Example}

\chapter{Multivariate Random Variables}
\section{Joint and Marginal Cumulative Distribution Functions}
Purpose: to characterize a joint distribution
of two random variables.
\begin{Definition}{Joint cumulative distribution function}{}
    Suppose $ X $ and $ Y $ are two random variables.
    The \textbf{joint cumulative
        distribution function} of $ X $ and $ Y $ is given by
    \[ F(x,y)=\Prob{X\le x,Y\le y} \]
    for $ (x,y)\in\mathbf{R}^2 $.
\end{Definition}
$ \Prob{X\le x,Y\le y} $:
``What is the probability these two events occur simultaneously''
\begin{Remark}{}{}
    Since $ \set{X\le x} $ and $ \set{Y\le y} $
    are both events, $ F(x,y) $ is well-defined as
    we consider $ \set{X\le x}\cap \set{Y\le y} $.
\end{Remark}
\begin{Remark}{}{}
    If we have more than two random variables, say $ X_1,X_2,\ldots,X_n $
    We can similarly define the cumulative distribution function as
    \[ F(x_1,\ldots,x_n)=\Prob{X_1\le x_1,\ldots,X_n\le x_n} \]
    However, in this course we will only focus on two events $ X $ and $ Y $.
\end{Remark}

\begin{Definition}{Joint cumulative distribution function}{}
    \begin{enumerate}[label=(\Roman*)]
        \item $ F $ is non-decreasing in $ x $ for fixed $ y $
        \item $ F $ is non-decreasing in $ y $ for fixed $ x $
        \item $ \displaystyle \lim\limits_{{x} \to {-\infty}} F(x,y)=0 $
              and $ \displaystyle \lim\limits_{{y} \to {-\infty}} F(x,y)=0 $

              By looking at
              \[ \underset{\underset{\text{ as }x\to-\infty}{\to 0}}{\set{X\le x}}\cap
                  \underset{\underset{\text{ as }y\to-\infty}{\to 0}}{\set{Y\le y}} \]
        \item \[ \displaystyle
                  \smashoperator{\lim\limits_{{(x,y)} \to {(-\infty,-\infty)}}}
                  F(x,y)=0\text{ and }
                  \displaystyle
                  \smashoperator{\lim\limits_{{(x,y)} \to {(\infty,\infty)}}}
                  F(x,y)=1 \]
    \end{enumerate}
\end{Definition}

\begin{Definition}{Marginal distribution function}{mdf}
    The \textbf{marginal distribution function} of $ X $ is given by
    \[ F_1(x)=\lim\limits_{{y} \to {\infty}} F(x,y)=\Prob{X\le x} \]
    for $ x\in\mathbf{R} $.

    The \textbf{marginal distribution function} if $ Y $ is given by
    \[ F_2(y)=\lim\limits_{{x} \to {\infty}} F(x,y)=\Prob{Y\le y} \]
    for $ y\in\mathbf{R} $.
\end{Definition}
\begin{Remark}{}{}
    The definition of marginal distribution
    function tells us that we can know all information
    about marginal c.d.f.\ from the joint c.d.f.\ but the
    marginal c.d.f.\ cannot give full information about
    joint c.d.f.\
\end{Remark}
\section{Bivariate Discrete Distributions}
\begin{Definition}{Joint discrete random variables}{}
    Suppose $ X $ and $ Y $ are both discrete random variables,
    then $ (X,Y) $ are \textbf{joint discrete random variables}
    $ X $ and $ Y $.
\end{Definition}

\begin{Definition}{Joint probability function, Support}{}
    Suppose $ X $ and $ Y $ are discrete random variables.
    The \textbf{joint probability function} of $ X $ and $ Y $
    is given by
    \[ f(x,y)=\Prob{X=x,Y=y} \]
    for $ (x,y)\in\mathbf{R}^2 $.

    The set $ A=\set{(x,y):f(x,y)>0} $ is called
    the \textbf{joint support} of $ (X,Y) $.
\end{Definition}

\begin{Definition}{Properties --- Joint Probability Function}{}
    \begin{enumerate}[label=(\Roman*)]
        \item $ f(x,y)\ge 0 $ for $ (x,y)\in\mathbf{R}^2 $
        \item $ \displaystyle \sum\limits_{(x,y)\in A}
                  f(x,y)=1 $
        \item For any set $ R\subseteq \mathbf{R}^2 $
              \[ \Prob{(X,Y)\in R}
                  =\sum\limits_{(x,y)\in R}f(x,y)  \]
    \end{enumerate}
\end{Definition}

\begin{Example}{}{}
    Suppose we want to find $ \Prob{X\le Y} $. What is the
    corresponding set $ R $?

    \textbf{Solution.} $ R=\set{(x,y):x\le y} $

    Suppose we want to find $ \Prob{X+Y\le 1} $. What is the corresponding
    set $ R $?

    \textbf{Solution.} $ R=\set{(x,y):x+y\le 1} $
\end{Example}

\begin{Definition}{Marginal probability function}{}
    Suppose $ X $ and $ Y $ are discrete
    random variables with joint probability
    function $ f(x,y) $.

    The \textbf{marginal probability
        function} of $ X $ is given by
    \[ f_1(x)=\Prob{X=x}
        =\Prob{X=x,Y<\infty}
        =\sum\limits_{y}f(x,y)  \]
    for $ x\in\mathbf{R} $.

    The \textbf{marginal probability function} of $ Y $
    is given by
    \[ f_2(y)=\Prob{Y=y}
        =\Prob{X<\infty,Y=y}
        =\sum\limits_{x}f(x,y)  \]
    for $ y\in\mathbf{R} $.
\end{Definition}

\begin{Example}{}{}
    Suppose that $ X $ and $ Y $ are discrete random variables
    with joint p.f.\ $ f(x,y)=kq^2 p^{x+y} $ where
    \begin{itemize}
        \item $ 0\le x\in\mathbf{Z}$
        \item $ 0\le y\in\mathbf{Z} $
        \item $ 0<p<1 $
        \item $ q=1-p $
    \end{itemize}
    \begin{enumerate}[label=(\roman*)]
        \item Determine $ k $.
        \item Find marginal p.f.\ of $ X $ and
              find marginal p.f.\ of $ Y $.
        \item Find $ \Prob{X\le Y} $.
    \end{enumerate}
    \textbf{Solution.}
    \begin{enumerate}[label=(\roman*)]
        \item $ k>0 $ since if $ k=0 $ then the summation
              of the joint p.f.\ will be 0 (but needs to be 1).
              \[\sum\limits_{x=0}^{\infty}
                  \sum\limits_{y=0}^{\infty} f(x,y)=1\]
              Therefore,
              \[k\biggl(\,\sum\limits_{x=0}^{\infty}
                  \sum\limits_{y=0}^{\infty} p^{x+y}q^2\biggr)=
                  kq^2\biggl(\,\sum\limits_{x=0}^{\infty} p^x\biggr)
                  \biggl(\,\sum\limits_{y=0}^{\infty}p^y\biggr)=kq^2
                  \biggl(\frac{1}{1-p}  \biggr)\biggl( \frac{1}{1-p} \biggr)=k
              \]
              Thus, $ k=1 $.
        \item Marginal p.f.\ of $ X $:
              \[ f_1(x)=\Prob{X=x}=
                  \sum\limits_{y=0}^{\infty} q^2p^{x+y}=
                  q^2 p^x
                  \biggl(\,\sum\limits_{y=0}^{\infty} p^y \biggr)
                  =q^2 p^x \biggl( \frac{1}{1-p}  \biggr)=p^x(1-p) \]
              Support of $ X $: $ \interval[open right]{0}{\infty} $.

              By symmetry,
              \[ f_2(y)=\Prob{Y=y}=qp^y \]
              Support of $ Y $: $ \interval[open right]{0}{\infty} $.
        \item Find $ \Prob{X\le Y} $.

              \begin{align*}
                  \Prob{X\le Y}
                   & =
                  \sum\limits_{x=0}^{\infty}
                  \sum\limits_{y=x}^{\infty}\biggl( q^2 p^{x+y} \biggr) \\
                   & =\sum\limits_{x=0}^{\infty} q^2p^x
                  \sum\limits_{y=x}^{\infty} p^y                        \\
                   & =\sum\limits_{x=0}^{\infty}
                  q^2p^x\biggl( \frac{p^x}{1-p}  \biggr)                \\
                   & =q\sum\limits_{x=0}^{\infty} p^{2x}                \\
                   & =q\biggl( \frac{1}{1-p^2}  \biggr)                 \\
                   & =\frac{1}{1+p}
              \end{align*}

    \end{enumerate}
\end{Example}
\begin{Remark}{Interesting Fact}{}
    If $ X $ and $ Y $ are \emph{continuous} random variables
    and have the same distribution and \emph{\textbf{independent}},
    \[ \Prob{X\le Y}=\frac{1}{2} \]
\end{Remark}
\section{Bivariate Continuous Distributions}
\begin{Definition}{Joint probability density function, Support}{}
    If the joint c.d.f.\ of $ (X,Y) $ can be written as
    \[ F(x,y)=\int_{-\infty}^{x} \int_{-\infty}^{y} f(s,t)\, d{t} \, d{s} \]
    for all $ (x,y)\in\mathbf{R}^2 $, then $ X $ and $ Y $
    are joint continuous random variables with \textbf{joint probability
        density function} $ f(x,y) $ where
    \[ f(x,y)=\begin{dcases*}
            \frac{\partial^2 F(x,y)}{\partial x\partial y} & if exists \\
            0                                              & otherwise
        \end{dcases*} \]
    The set $ A=\set{(x,y):f(x,y)>0} $ is called the \textbf{support} of $ (X,Y) $.
\end{Definition}
\begin{Remark}{}{}
    We will arbitrarily define $ f(x,y) $ to be equal to $ 0 $ when
    $ \displaystyle  \frac{\partial^2}{\partial x\partial y}[F(x,y)]
    $
    does not exist, although we can define it to be any real number.
\end{Remark}
\begin{Definition}{Properties --- Joint Probability Density Function}{}
    \begin{enumerate}[label=(\Roman*)]
        \item $ f(x,y)\ge 0 $ for all $ (x,y)\in\mathbf{R}^2 $
        \item For any set $ R\subseteq \mathbf{R}^2 $:
              \begin{align*}
                  \Prob{(X,Y)\in R}
                   & =\iint\limits_{(x,y)\in R}f(x,y)dx\,dy
              \end{align*}
    \end{enumerate}
\end{Definition}
\begin{Example}{}{}
    To find $ \Prob{X\le Y} $, the region is
    $ R=\set{(x,y):x\le y} $. Therefore,
    \[ \Prob{X\le y}=
        \iint\limits_{x\le y}f(x,y)dx\,dy \]
\end{Example}
\begin{Definition}{Marginal probability density function}{}
    Suppose $ X $ and $ Y $ are continuous random variables with
    p.d.f.\ $ f(x,y) $. The \textbf{marginal probability
        density function} of $ X $ is given by
    \[ f_1(x)=\int_{-\infty}^{\infty} f(x,y)\, d{y} \]
    for $ x\in\mathbf{R} $ and the \textbf{marginal probability
        density function} of $ Y $ is given by
    \[ f_2(y)=\int_{-\infty}^{\infty} f(x,y)\, d{x}  \]
    for $ y\in\mathbf{R} $.
\end{Definition}
\[ \Prob{(X,Y)\in\mathbf{R}}
    =\iint\limits_{R}f(x,y)dx\,dy=
    \int_{x} \int_{y}f(x,y) \, d{x} \, d{y} \]
Helpful theorem from MATH 237 that some of you may have forgotten:

\begin{Theorem}{$ \dagger $}{}
    \underline{$ y $ first, then $ x $}

    Let $ R\subset \mathbf{R}^2 $ be defined by
    \[ y_\ell(x)\le y\le y_u(x)\quad\text{ and }\quad
        x_{\ell}\le x\le x_u \]
    where $ y_{\ell}(x) $ and $ y_u(x) $ are continuous
    for $ x_{\ell}\le x\le x_u $. If $ f(x,y) $
    is continuous on $ R $, then
    \[ \iint\limits_{R}f(x,y)dA=
        \int_{x_\ell}^{x_u} \int_{y_\ell(x)}^{y_u(x)} f(x,y)\, d{y} \, d{x}  \]
    \underline{$ x $ first, then $ y $}

    Let $ R\subset \mathbf{R}^2 $ be defined by
    \[ x_\ell(y)\le x\le x_u(y)\quad\text{ and }\quad
        y_{\ell}\le y\le y_u \]
    where $ x_{\ell}(y) $ and $ x_u(y) $ are continuous
    for $ y_{\ell}\le y\le y_u $. If $ f(x,y) $
    is continuous on $ R $, then
    \[ \iint\limits_{R}f(x,y)dA=
        \int_{y_\ell}^{y_u} \int_{x_\ell(y)}^{x_u(y)} f(x,y)\, d{x} \, d{y}  \]
\end{Theorem}
We use $ \ell $ for ``lower'' and $ u $ for ``upper.''
\begin{Example}{}{}
    Describe the region $ R $ above the $ x $-axis.
    \begin{center}
        \includegraphics[width=0.9\textwidth]{fig1.pdf}
    \end{center}
    \textbf{Solution.} $ R $ can be described by the set of two inequalities
    (you can actually verify this in Desmos if you \emph{really} forgot how this works):
    \[ 0\le y\le 1 \]
    \[ y-1\le x\le 1-y \]
    Using the theorem above,
    \[ \int_{0}^{1} \int_{y-1}^{1-y} f(x,y)\, d{x}\, d{y}  \]
\end{Example}

\section{2019-09-24}
\subsection{Solving LP Problems}
Let $x_1,\dots,x_n$ be all the variables in an optimization problem. Then
assignment of values to all variables such that all constraints are satisfied,
gives a \emph{feasible solution}. An optimization problem is called \emph{feasible}
if it has at least one feasible solution, otherwise it is called \emph{infeasible}.

\subsection{Example (Infeasible LP)}
(LP)
\[\max x_1+2x_2+3x_3+4x_4+5x_5\]
subject to
\begin{align*}
    &
    \begin{matrix}
    1\\
    -2
    \end{matrix}
    \underbrace{
        \begin{bmatrix}
        -3 & 2 & 7 & 1 & -7 \\
        -2 & 1 & 2 & 0 & -4
        \end{bmatrix}}_{A}
    \underbrace{\begin{bmatrix}
        x_1\\
        x_2\\
        x_3\\
        x_4\\
        x_5
    \end{bmatrix}}_{\mathbf{x}}
    =
    \underbrace{\begin{bmatrix}
        6\\
        4
    \end{bmatrix}}_{\mathbf{b}}\\
    &\mathbf{x}\ge \mathbf{0}
\end{align*}
Let $\mathbf{y}:=
\begin{bmatrix}
    1\\
    -2
\end{bmatrix}$
and consider the facts
\begin{align*}
    &A\mathbf{x}=\mathbf{b}\\
    &\implies \mathbf{y}^TA\mathbf{x}=\mathbf{y}^T\mathbf{b}\\
    &\implies \underbrace{\begin{bmatrix}
        1 & 0 & 3 & 1 & 1
    \end{bmatrix}}_{\ge \mathbf{0}^T}
    \underbrace{\mathbf{x}}_{\ge \mathbf{0}}=\underbrace{6-8}_{< 0}=-2
\end{align*}
Therefore, $\nexists$ any solution to $A\mathbf{x}=\mathbf{b}$, $\mathbf{x}\ge 0$.
Thus, the LP is infeasible.

\subsection{Proposition (Infeasibility)}
Let $A\in M_{m\times n}(R), \mathbf{b}\in\mathbb{R}^m$. Suppose
$\exists \mathbf{y}\in\mathbb{R}^m$ such that
\begin{enumerate}
    \item $\mathbf{y}^TA\ge\mathbf{0}^T$
    \item $\mathbf{y}^T\mathbf{b}<0$
\end{enumerate}
For every $\mathbf{c}\in\mathbb{R}^n$, the LP
\[\max \{\mathbf{c}^T\mathbf{x} \mid A\mathbf{x}=\mathbf{b}\text{, }
\mathbf{x}\ge\mathbf{0}\}\]
is infeasible. In particular, we call a vector $\mathbf{y}$ a \emph{certificate of infeasibility}.

\begin{proof}
    Let $A\in M_{m\times n}(R), \mathbf{b}\in\mathbb{R}^m$ and suppose
$\exists \mathbf{y}\in\mathbb{R}^m$ such that
\begin{enumerate}
    \item $\mathbf{y}^TA\ge\mathbf{0}^T$
    \item $\mathbf{y}^T\mathbf{b}<0$
\end{enumerate}
Suppose for a contradiction that $\exists\mathbf{\bar{x}}\in\mathbb{R}^n$ 
such that
\[A\mathbf{\bar{x}}=\mathbf{b} \text{, }\mathbf{\bar{x}}\ge \mathbf{0}\]
\[
    A\mathbf{\bar{x}}=\mathbf{b}
    \implies
    \underbrace{\mathbf{y}^TA}_{\ge\mathbf{0}^T}
    \underbrace{\mathbf{\bar{x}}}_{\ge\mathbf{0}}
    =\underbrace{\mathbf{y}^T\mathbf{b}}_{\nless 0}
\]
a contradiction to 2.
\end{proof}

An optimization problem is called unbounded if $\forall M\in\mathbb{R}$, there
exists a feasible solution of the optimization problem with the objective 
value strictly better than $M$.
\subsection{Example (Unbounded LP)}
\[\max 
\begin{bmatrix}
    -1 & 3 & 0 & 0 & 1
\end{bmatrix}\mathbf{x}\]
subject to
\begin{align*}
    &\begin{bmatrix}
        -1 & 3 & -1 & 1 & 0\\
        -2 & 4 & 1 & 0 & 1
    \end{bmatrix}
    \mathbf{x}
    =
    \begin{bmatrix}
        2\\
        1
    \end{bmatrix}\\
    &\mathbf{x}\ge \mathbf{0}
\end{align*}
Consider
\[\mathbf{\tilde{x}}:=
\underbrace{\begin{bmatrix}
    0\\
    0\\
    0\\
    2\\
    1  
\end{bmatrix}}_{\mathbf{x}}
+
t
\underbrace{\begin{bmatrix}
    1\\
    0\\
    0\\
    1\\
    2
\end{bmatrix}}_{\mathbf{d}} \text{, } t\ge 0
\]

\[
    A\mathbf{x}=
    \begin{bmatrix}
        2\\
        1
    \end{bmatrix}, \bar{\mathbf{x}}\ge \mathbf{0}.\text{Therefore $\bar{\mathbf{x}}$ is a feasible solution.}
\]
\[
    A\mathbf{d}=\begin{bmatrix}
        0\\
        0
    \end{bmatrix}, \mathbf{d}\ge \mathbf{0}.\\
\]

\[A\tilde{\mathbf{x}}=A(\bar{\mathbf{x}}+t\mathbf{d})=A\bar{\mathbf{x}}+t(A\mathbf{d})=
\begin{bmatrix}
    2\\
    1
\end{bmatrix}\]
\[\tilde{\mathbf{x}}=\bar{\mathbf{x}}+t\mathbf{d}\]
Therefore, $\tilde{\mathbf{x}}$ is a feasible solution $\forall t\ge 0$.


\textbf{Objective function value of $\tilde{\mathbf{x}}$:}
\[
\begin{bmatrix}
    -1 & 3 & 0 & 0 & 1
\end{bmatrix}
\left(\begin{bmatrix}
    0\\
    0\\
    0\\
    2\\
    1  
\end{bmatrix}
+
t
\begin{bmatrix}
    1\\
    0\\
    0\\
    1\\
    2
\end{bmatrix}\right)
=
1+t(-1+2)=1+t\rightarrow+\infty \text{ as }t\rightarrow+\infty\]
Therefore the LP is unbounded.

\subsection{Proposition (Unboundedness)}
Suppose $\exists \mathbf{\bar{x}}\in\mathbb{R}^n$ such that
\[A\mathbf{\bar{x}}=\mathbf{b}, \mathbf{x}\ge \mathbf{0}.\]
and $\exists\mathbf{d}\in\mathbb{R}^n$ such that
\begin{enumerate}
    \item $A\mathbf{d}=\mathbf{0}$
    \item $\mathbf{d}\ge \mathbf{0}$
    \item $\mathbf{c}^T\mathbf{d}>0$
\end{enumerate}
For every $\mathbf{c}\in\mathbb{R}^n$, the LP
\[\max \{\mathbf{c}^T\mathbf{x} \mid A\mathbf{x}=\mathbf{b}\text{, }
\mathbf{x}\ge\mathbf{0}\}\]
is unbounded. In particular, we call a pair of vectors $\mathbf{\bar{x}}$, $\mathbf{d}$ a
\emph{certificate of unboundedness}.

\begin{proof}
    Suppose there exists such $\mathbf{d}$. Consider
    \[\tilde{\mathbf{x}}=\bar{\mathbf{x}}+t\mathbf{d}, t\ge 0\]
Then,
\[A\tilde{\mathbf{x}}=
\underbrace{A\bar{\mathbf{x}}}_{\mathbf{b}}+
t\underbrace{(A\mathbf{d})}_{\mathbf{0}}=\mathbf{b}\]
Therefore $\tilde{\mathbf{x}}$ is a feasible solution of the LP, $t\ge 0$.
The objective value of the function is
\[\mathbf{c}^T\tilde{\mathbf{x}}=\mathbf{c}^T\bar{\mathbf{x}}+t
\underbrace{(\mathbf{c}^T\mathbf{d})}_{>\mathbf{0}}\rightarrow +\infty\text{ as }t\rightarrow+\infty\]
Therefore, the LP is unbounded.
\end{proof}
\begin{remark}
    If the LP is $\min$, then flip the equality for 3.
\end{remark}

\subsection{Example (Optimal LP)}
\[\max 10x_1+15x_2\]
subject to
\begin{align*}
    2x_1+x_2+x_3=1600\\
    x_1+3x_2+x_4=1200\\
    \mathbf{x}\ge \mathbf{0}
\end{align*}
Consider
\[\bar{\mathbf{x}}:=
\begin{bmatrix}
    720\\
    160\\
    0\\
    0
\end{bmatrix}\]
\[\mathbf{y}:=
\begin{bmatrix}
    3\\
    4
\end{bmatrix}
\]
$A\bar{\mathbf{x}}=\mathbf{b}$, $\bar{\mathbf{x}}\ge \mathbf{0}$. So $\bar{\mathbf{x}}$ is a feasible solution.


Also, $\mathbf{c}^T\mathbf{x}=7200+2400=9600$.
Every feasible solution satisfies
\begin{align*}
    &A\mathbf{x}=\mathbf{b}\\
    &\implies \mathbf{y}^TA\mathbf{x}=\mathbf{y}^T\mathbf{b}
\end{align*}
\[\mathbf{y}^TA=
\begin{bmatrix}
    10 & 15 & 3 & 4
\end{bmatrix}
\ge
\begin{bmatrix}
    10 & 15 & 0 & 0
\end{bmatrix}=\mathbf{c}^T\]
\[\mathbf{y}^Tb=3\times 1600+4\times 1200=9600=\mathbf{c}^T\mathbf{x}\]
Therefore $\bar{\mathbf{x}}$ is an optimal solution.

\makeheading{ 2020-01-20 }

\begin{defbox}
    \begin{definition}
        Let $ F $ be a field, and $ f\in F[x] $ of degree $ n\geqslant 1 $.
        $ f $ is \textbf{irreducible} over $ F $ if $ f $ cannot be written
        as $ f=gh $, where $ g,h\in F[x] $ and $ \deg(g),\deg(n)\geqslant 1 $.
    \end{definition}
\end{defbox}

\begin{exbox}
    \begin{example}[Irreducible] $ \; $
        \begin{itemize}
            \item $ x^2+1 $ is irreducible over $ \mathbb{R} $
            \item $ x^2+1 $ is reducible over $ \mathbb{C} $ since $ (x+i)(x-i)=x^1+1 $
            \item $ x^2+1 $ is reducible over $ \mathbb{Z}_2 $ since $ (x+1)^2=x^1+1 $
            \item $ x^2+1 $ is irreducible over $ \mathbb{Z}_3 $
        \end{itemize}
    \end{example}
\end{exbox}

\begin{thmbox}
    \begin{theorem}
        Let $ F $ be a field and $ f\in F[x] $ of degree $ n\geqslant 1 $.
        $ F[x]/(f) $ is a field if and only if $ f $ is irreducible over $ F $.
    \end{theorem}
\end{thmbox}

\begin{proof}
    Note that $ F[x]/(f) $ is a commutative ring.

    $ (\impliedby) $ Suppose $ g\in F[x]/(f) $ where $ g\neq 0 $
    and $ \deg(g)<\deg(f) $. Then, $ \gcd(g,f)=1 $ and so by EEA
    for polynomials, there exists $ s,t\in F[x] $ such that
    \[ gs+ft=1 \]
    Reducing both sides modulo $ f $ gives
    \[ gs\equiv 1 \mod f \]
    So, $ g^{-1}=s $. Hence $ F[x]/(f) $ is a field.

    $ (\implies) $ Exercise.
\end{proof}

We need an irreducible polynomial $ f\in\mathbb{Z}_p[x] $ of degree $ n $.
Then, $ \mathbb{Z}[x]/(f) $ is a finite field of order $ p^n $.

\begin{thmbox}
    \begin{theorem}
        For any prime $ p $ and $ n\in\mathbb{Z}_{\geqslant 2} $, there exists
        an irreducible polynomial of degree $ n $ over $ \mathbb{Z}_p $.
    \end{theorem}
\end{thmbox}
The proof is beyond the scope of this course.

\begin{thmbox}
    \begin{theorem}
        There exists a finite field of order $ q $ if and only if
        $ q $ is a prime power.
    \end{theorem}
\end{thmbox}

\begin{exbox}
    \begin{example}
        Construct a finite field of order $ 2^2=4 $.

        \textbf{Solution:} Take $ f(x)=x^2+x+1\in\mathbb{Z}_2[x] $
        which is irreducible over $ \mathbb{Z}_2[x] $. Thus, the field is
        \[ \mathbb{Z}_2[x]/(x^2+x+1)=\left\{ 0,1,x,x+1\right\} \]
        Examples of operations:
        \begin{itemize}
            \item $ x+(x+1)=1 $
            \item $ x(x+1)=x^2+x=1 $
            \item $ x^{-1}=x+1 $
            \item $ 1^{-1}=1 $
            \item $ x^{-1}=x+1 $
            \item $ (x+1)^{-1}=x $
        \end{itemize}
    \end{example}
\end{exbox}

\begin{exbox}
    \begin{example}
        Construct a field of order $ 2^3=8 $.

        \textbf{Solution:} We need an irreducible polynomial of degree $ 3 $
        over $ \mathbb{Z}_2 $. Take $ f_1(x)=x^3+x+1 $ which is
        irreducible over $ \mathbb{Z}_2 $. Then a field of order $ 8 $ is
        \[ F_1=Z_2[x]/(x^3+x+1)=\left\{ 0,1,x,x+1,x^2,x^2+1,x^2+x,x^2+x+1\right\} \]
        Examples of operations:
        \begin{itemize}
            \item $ x^2+(x^2+x+1)=x+1 $
            \item $ x^2(x^2+x+1)=x^4+x^3+x^2=1 $
            \item $ (x^2)^{-1}=x^2+x+1 $
            \item $ x^{-1}=x^2+1 $
        \end{itemize}
    \end{example}
\end{exbox}

\begin{exbox}
    \begin{example}
    Construct a field of order $ 2^3=8 $.

    \textbf{Solution:} Take $ f_2(x)=x^3+x^2+1 $. Then a field of order $ 8 $ is
    \[ F_2=\mathbb{Z}_2[x]/(x^3+x^2+1)=\left\{ 0,1,x,x+1,x^2,x^2+1,x^2+x,x^2+x+1\right\} \]
    Examples of operations:
    \begin{itemize}
        \item $ x^{-1}=x^2+x $
    \end{itemize}
\end{example}
\end{exbox}

\myuline{Note:} $ F_1 $ and $ F_2 $ are two different fields of order $ 2^3=8 $,
but they are isomorphic. That is,
there is a bijection $ \alpha : F_1\rightarrow F_2 $ such that
\[ \alpha(a+b)=\alpha(a)+\alpha(b) \]
\[ \alpha(ab)=\alpha(a)\alpha(b) \]
for all $ a,b\in F_1 $.

\begin{thmbox}
    \begin{theorem}
        Any two finite fields of order $ q $ are isomorphic.
    \end{theorem}
\end{thmbox}

\begin{proof}
    Exericse.
\end{proof}

\begin{defbox}
    \begin{definition}
        We will denote the \textbf{Galois field of order $ q $} by $ GF(q) $.
    \end{definition}
\end{defbox}
We saw two different representations of $ GF(2^3) $.

\makeheading{Lecture 8 | 2020-09-30}
\section{Categorical Predictors}
\subsection{R Demo}
\begin{knitrout}
\definecolor{shadecolor}{rgb}{0.969, 0.969, 0.969}\color{fgcolor}\begin{kframe}
\begin{alltt}
\hlcom{## NASA rocket data example}
\hlcom{## From: R.S. Jankovsky, T.D. Smith, A.J. Pavli (1999).}
\hlcom{## 'High-Area-Ratio Rocket Nozzle at High Combustion}
\hlcom{## Chamber Pressure-Experimental and Analytical}
\hlcom{## Validation'.}
\hlcom{# setwd(...) first if your CSV file is somewhere else}
\hlstd{rocket} \hlkwb{<-} \hlkwd{read.csv}\hlstd{(}\hlstr{"csv/rocket.csv"}\hlstd{)}
\hlcom{# output all data in rocket vector}
\hlstd{rocket}
\end{alltt}
\begin{verbatim}
##    thrust nozzle propratio
## 1   488.0      1      3.97
## 2   481.6      1      5.91
## 3   485.9      1      4.98
## 4   486.0      1      4.91
## 5   484.5      1      3.89
## 6   483.8      1      5.80
## 7   463.2      0      5.99
## 8   471.2      0      4.95
## 9   469.5      0      3.91
## 10  470.5      0      5.85
## 11  469.5      0      4.71
## 12  465.7      0      3.84
\end{verbatim}
\end{kframe}
\end{knitrout}

$Y$ (thrust) is the response variable, and there
are two explanatory variables $x_1,x_2$
(nozzle, propratio) where nozzle is coded
as 1 if it's large.

\begin{knitrout}
\definecolor{shadecolor}{rgb}{0.969, 0.969, 0.969}\color{fgcolor}\begin{kframe}
\begin{alltt}
\hlcom{# Scatter plots where mfrow is used to put multiple plots}
\hlcom{# on one image}
\hlkwd{par}\hlstd{(}\hlkwc{mfrow} \hlstd{=} \hlkwd{c}\hlstd{(}\hlnum{1}\hlstd{,} \hlnum{2}\hlstd{))}
\hlkwd{plot}\hlstd{(rocket}\hlopt{$}\hlstd{nozzle, rocket}\hlopt{$}\hlstd{thrust,} \hlkwc{ylab} \hlstd{=} \hlstr{"Thrust"}\hlstd{,} \hlkwc{xlab} \hlstd{=} \hlstr{"Nozzle size (1 = large)"}\hlstd{)}
\hlkwd{plot}\hlstd{(rocket}\hlopt{$}\hlstd{propratio, rocket}\hlopt{$}\hlstd{thrust,} \hlkwc{ylab} \hlstd{=} \hlstr{"Thrust"}\hlstd{,} \hlkwc{xlab} \hlstd{=} \hlstr{"Propellant to fuel ratio"}\hlstd{)}
\end{alltt}
\end{kframe}

{\centering \includegraphics[width=\maxwidth]{figure/unnamed-chunk-33-1} 

}


\end{knitrout}

Left is nozzle size vs thrust. Right is propellant
relationship vs thrust.

\begin{knitrout}
\definecolor{shadecolor}{rgb}{0.969, 0.969, 0.969}\color{fgcolor}\begin{kframe}
\begin{alltt}
\hlcom{# Fit MLR using lm}
\hlstd{m1} \hlkwb{<-} \hlkwd{lm}\hlstd{(thrust} \hlopt{~} \hlstd{nozzle} \hlopt{+} \hlstd{propratio,} \hlkwc{data} \hlstd{= rocket)}
\hlkwd{summary}\hlstd{(m1)}
\end{alltt}
\begin{verbatim}
## 
## Call:
## lm(formula = thrust ~ nozzle + propratio, data = rocket)
## 
## Residuals:
##     Min      1Q  Median      3Q     Max 
## -3.8459 -1.7555  0.5934  1.2906  3.3008 
## 
## Coefficients:
##             Estimate Std. Error t value Pr(>|t|)    
## (Intercept) 473.6039     4.7158 100.430 4.88e-15 ***
## nozzle       16.7383     1.5329  10.919 1.71e-06 ***
## propratio    -1.0948     0.9414  -1.163    0.275    
## ---
## Signif. codes:  0 '***' 0.001 '**' 0.01 '*' 0.05 '.' 0.1 ' ' 1
## 
## Residual standard error: 2.655 on 9 degrees of freedom
## Multiple R-squared:  0.9303,	Adjusted R-squared:  0.9148 
## F-statistic: 60.05 on 2 and 9 DF,  p-value: 6.238e-06
\end{verbatim}
\begin{alltt}
\hlstd{m2} \hlkwb{<-} \hlkwd{lm}\hlstd{(thrust} \hlopt{~} \hlnum{0} \hlopt{+} \hlstd{nozzle,} \hlkwc{data} \hlstd{= rocket)}
\hlkwd{summary}\hlstd{(m2)}
\end{alltt}
\begin{verbatim}
## 
## Call:
## lm(formula = thrust ~ 0 + nozzle, data = rocket)
## 
## Residuals:
##    Min     1Q Median     3Q    Max 
##  -3.37   0.58 233.12 469.50 471.20 
## 
## Coefficients:
##        Estimate Std. Error t value Pr(>|t|)   
## nozzle    485.0      141.2   3.435  0.00558 **
## ---
## Signif. codes:  0 '***' 0.001 '**' 0.01 '*' 0.05 '.' 0.1 ' ' 1
## 
## Residual standard error: 345.8 on 11 degrees of freedom
## Multiple R-squared:  0.5175,	Adjusted R-squared:  0.4736 
## F-statistic:  11.8 on 1 and 11 DF,  p-value: 0.005575
\end{verbatim}
\begin{alltt}
\hlkwd{anova}\hlstd{(m1)}
\end{alltt}
\begin{verbatim}
## Analysis of Variance Table
## 
## Response: thrust
##           Df Sum Sq Mean Sq  F value    Pr(>F)    
## nozzle     1 836.67  836.67 118.7377 1.743e-06 ***
## propratio  1   9.53    9.53   1.3524    0.2748    
## Residuals  9  63.42    7.05                       
## ---
## Signif. codes:  0 '***' 0.001 '**' 0.01 '*' 0.05 '.' 0.1 ' ' 1
\end{verbatim}
\end{kframe}
\end{knitrout}

On the left it's $Y$ (response variable) and on the
right it's $x_1,x_2$ (explanatory variables).
From summary, we get the estimate vector
$\hat{\symbf{\beta}}=(473.6039, 16.7383,-1.0948)^\top$.

\begin{knitrout}
\definecolor{shadecolor}{rgb}{0.969, 0.969, 0.969}\color{fgcolor}\begin{kframe}
\begin{alltt}
\hlcom{# Manual beta estimates where rep is used to make the}
\hlcom{# columns of 1s}
\hlstd{X} \hlkwb{<-} \hlkwd{cbind}\hlstd{(}\hlkwd{rep}\hlstd{(}\hlnum{1}\hlstd{,} \hlnum{12}\hlstd{), rocket}\hlopt{$}\hlstd{nozzle, rocket}\hlopt{$}\hlstd{propratio)}  \hlcom{# X matrix}
\hlstd{y} \hlkwb{<-} \hlkwd{matrix}\hlstd{(rocket}\hlopt{$}\hlstd{thrust,} \hlkwc{ncol} \hlstd{=} \hlnum{1}\hlstd{)}  \hlcom{# response vector}
\hlstd{beta_hat} \hlkwb{<-} \hlkwd{solve}\hlstd{(}\hlkwd{t}\hlstd{(X)} \hlopt \hlstd{X)} \hlopt \hlkwd{t}\hlstd{(X)} \hlopt \hlstd{y}
\hlstd{beta_hat}
\end{alltt}
\begin{verbatim}
##            [,1]
## [1,] 473.603924
## [2,]  16.738319
## [3,]  -1.094822
\end{verbatim}
\end{kframe}
\end{knitrout}

  \code{solve} is used for the inverse. \code{\%*\%} is used
  for matrix-matrix multiplication, and \code{t(X)}
  is used for transposing $X$.

\begin{knitrout}
\definecolor{shadecolor}{rgb}{0.969, 0.969, 0.969}\color{fgcolor}\begin{kframe}
\begin{alltt}
\hlcom{# Manual sigma estimate}
\hlstd{mu_hat} \hlkwb{<-} \hlstd{X} \hlopt \hlstd{beta_hat}  \hlcom{# fitted values}
\hlstd{e} \hlkwb{<-} \hlstd{y} \hlopt{-} \hlstd{mu_hat}  \hlcom{# residuals}
\hlstd{sigma_hat} \hlkwb{<-} \hlkwd{sqrt}\hlstd{((}\hlkwd{t}\hlstd{(e)} \hlopt \hlstd{e)}\hlopt{/}\hlnum{9}\hlstd{)}  \hlcom{# Note n-p-1 = 12-2-1 = 9}
\hlstd{sigma_hat}
\end{alltt}
\begin{verbatim}
##        [,1]
## [1,] 2.6545
\end{verbatim}
\begin{alltt}
\hlstd{sigma_hat} \hlkwb{<-} \hlkwd{sqrt}\hlstd{(}\hlkwd{sum}\hlstd{(e}\hlopt{^}\hlnum{2}\hlstd{)}\hlopt{/}\hlnum{9}\hlstd{)}  \hlcom{# equivalent}
\hlstd{sigma_hat}
\end{alltt}
\begin{verbatim}
## [1] 2.6545
\end{verbatim}
\end{kframe}
\end{knitrout}
  \begin{itemize}
    \item $\hat{\symbf{\mu}}=X\hat{\symbf{\beta}}$
    \item $\symbf{e}=\symbf{y}-\hat{\symbf{\mu}}$
    \item $\hat{\sigma}=\sqrt{\biggl(\sum_{i=1}^{n}e_i^2\biggr)/9}=2.6545$, or
    \item $\hat{\sigma}=\sqrt{(\symbf{e}^\top\symbf{e})/9}=2.6545$
  \end{itemize}
\begin{knitrout}
\definecolor{shadecolor}{rgb}{0.969, 0.969, 0.969}\color{fgcolor}\begin{kframe}
\begin{alltt}
\hlcom{# Covariance matrix of beta_hat}
\hlkwd{vcov}\hlstd{(m1)}
\end{alltt}
\begin{verbatim}
##             (Intercept)      nozzle   propratio
## (Intercept)   22.238325 -1.02316688 -4.32080608
## nozzle        -1.023167  2.34987593 -0.03102117
## propratio     -4.320806 -0.03102117  0.88631920
\end{verbatim}
\begin{alltt}
\hlkwd{sqrt}\hlstd{(}\hlkwd{diag}\hlstd{(}\hlkwd{vcov}\hlstd{(m1)))}  \hlcom{# SEs of individual betas}
\end{alltt}
\begin{verbatim}
## (Intercept)      nozzle   propratio 
##   4.7157528   1.5329305   0.9414453
\end{verbatim}
\begin{alltt}
\hlcom{# Manual}
\hlstd{se_beta} \hlkwb{<-} \hlstd{sigma_hat} \hlopt{*} \hlkwd{sqrt}\hlstd{(}\hlkwd{diag}\hlstd{(}\hlkwd{solve}\hlstd{(}\hlkwd{t}\hlstd{(X)} \hlopt \hlstd{X)))}
\hlstd{se_beta}
\end{alltt}
\begin{verbatim}
## [1] 4.7157528 1.5329305 0.9414453
\end{verbatim}
\end{kframe}
\end{knitrout}

  \begin{itemize}
    \item $\Se*{\hat{\symbf{\beta}}}=\hat{\sigma}\sqrt{(X^\top X)^{-1}}=(4.71, 1.53, 0.94)^{\top}$
  \end{itemize}

\begin{knitrout}
\definecolor{shadecolor}{rgb}{0.969, 0.969, 0.969}\color{fgcolor}\begin{kframe}
\begin{alltt}
\hlcom{# Estimate the mean response for units with small nozzle}
\hlcom{# and propellant ratio 5.5 include a 95% CI}
\hlkwd{predict}\hlstd{(}\hlkwc{object} \hlstd{= m1,} \hlkwc{newdata} \hlstd{=} \hlkwd{data.frame}\hlstd{(}\hlkwc{nozzle} \hlstd{=} \hlnum{0}\hlstd{,} \hlkwc{propratio} \hlstd{=} \hlnum{5.5}\hlstd{),}
  \hlkwc{interval} \hlstd{=} \hlstr{"confidence"}\hlstd{,} \hlkwc{level} \hlstd{=} \hlnum{0.95}\hlstd{)}
\end{alltt}
\begin{verbatim}
##        fit      lwr      upr
## 1 467.5824 464.7929 470.3719
\end{verbatim}
\end{kframe}
\end{knitrout}

  Therefore, $\hat{y}_0=467.58$. The 95\% confidence interval for the mean
  response given $\symbf{x}_0$ is $[464.7929,470.3719]$.

\begin{knitrout}
\definecolor{shadecolor}{rgb}{0.969, 0.969, 0.969}\color{fgcolor}\begin{kframe}
\begin{alltt}
\hlcom{# Manual calculation}
\hlstd{x0} \hlkwb{<-} \hlkwd{matrix}\hlstd{(}\hlkwd{c}\hlstd{(}\hlnum{1}\hlstd{,} \hlnum{0}\hlstd{,} \hlnum{5.5}\hlstd{),} \hlkwc{nrow} \hlstd{=} \hlnum{1}\hlstd{)}
\hlstd{y0_hat} \hlkwb{<-} \hlstd{x0} \hlopt \hlstd{beta_hat}
\hlstd{y0_hat}
\end{alltt}
\begin{verbatim}
##          [,1]
## [1,] 467.5824
\end{verbatim}
\begin{alltt}
\hlcom{# mu0 is also known as \textbackslash{}hat\{Y\}_0}
\hlstd{se_mu0} \hlkwb{<-} \hlstd{sigma_hat} \hlopt{*} \hlkwd{sqrt}\hlstd{(x0} \hlopt \hlkwd{solve}\hlstd{(}\hlkwd{t}\hlstd{(X)} \hlopt \hlstd{X)} \hlopt \hlkwd{t}\hlstd{(x0))}
\hlstd{se_mu0}
\end{alltt}
\begin{verbatim}
##          [,1]
## [1,] 1.233132
\end{verbatim}
\begin{alltt}
\hlstd{crit_val} \hlkwb{<-} \hlkwd{qt}\hlstd{(}\hlnum{0.975}\hlstd{,} \hlnum{9}\hlstd{)}
\hlstd{ci_lo} \hlkwb{<-} \hlstd{y0_hat} \hlopt{-} \hlstd{crit_val} \hlopt{*} \hlstd{se_mu0}
\hlstd{ci_hi} \hlkwb{<-} \hlstd{y0_hat} \hlopt{+} \hlstd{crit_val} \hlopt{*} \hlstd{se_mu0}
\hlkwd{c}\hlstd{(y0_hat, ci_lo, ci_hi)}
\end{alltt}
\begin{verbatim}
## [1] 467.5824 464.7929 470.3719
\end{verbatim}
\end{kframe}
\end{knitrout}

  \begin{itemize}
    \item $\symbf{x}_0=\begin{bmatrix}1&0&5.5\end{bmatrix}$
    \item $\hat{y}_0=\symbf{x}_0\hat{\symbf{\beta}}=467.5824$
    \item $\Se*{\hat{Y}_0}=\hat{\sigma}\sqrt{\symbf{x}_0(X^\top X)^{-1}\symbf{x}_0^\top}=
            1.233132$
  \end{itemize}
  Therefore, $\hat{y}_0=467.58$. The 95\% confidence interval for the mean
  response given $\symbf{x}_0$ is $[464.7929,470.3719]$.

\begin{knitrout}
\definecolor{shadecolor}{rgb}{0.969, 0.969, 0.969}\color{fgcolor}\begin{kframe}
\begin{alltt}
\hlcom{# Predict the value of the response for a unit with small}
\hlcom{# nozzle and propellant ratio 5.5 include a 95% PI}
\hlkwd{predict}\hlstd{(}\hlkwc{object} \hlstd{= m1,} \hlkwc{newdata} \hlstd{=} \hlkwd{data.frame}\hlstd{(}\hlkwc{nozzle} \hlstd{=} \hlnum{0}\hlstd{,} \hlkwc{propratio} \hlstd{=} \hlnum{5.5}\hlstd{),}
  \hlkwc{interval} \hlstd{=} \hlstr{"prediction"}\hlstd{,} \hlkwc{level} \hlstd{=} \hlnum{0.95}\hlstd{)}
\end{alltt}
\begin{verbatim}
##        fit      lwr      upr
## 1 467.5824 460.9612 474.2036
\end{verbatim}
\end{kframe}
\end{knitrout}

  Therefore, $y_0=467.5824$. The 95\% prediction interval for the
  response $(y_0)$ given $\symbf{x}_0$ is $[460.9612, 474.2036]$.

\begin{knitrout}
\definecolor{shadecolor}{rgb}{0.969, 0.969, 0.969}\color{fgcolor}\begin{kframe}
\begin{alltt}
\hlcom{# Manual calculation for an individual}
\hlstd{x0} \hlkwb{<-} \hlkwd{matrix}\hlstd{(}\hlkwd{c}\hlstd{(}\hlnum{1}\hlstd{,} \hlnum{0}\hlstd{,} \hlnum{5.5}\hlstd{),} \hlkwc{nrow} \hlstd{=} \hlnum{1}\hlstd{)}
\hlstd{y0_hat} \hlkwb{<-} \hlstd{x0} \hlopt \hlstd{beta_hat}
\hlstd{se_y0} \hlkwb{<-} \hlstd{sigma_hat} \hlopt{*} \hlkwd{sqrt}\hlstd{(}\hlnum{1} \hlopt{+} \hlstd{x0} \hlopt \hlkwd{solve}\hlstd{(}\hlkwd{t}\hlstd{(X)} \hlopt \hlstd{X)} \hlopt \hlkwd{t}\hlstd{(x0))}
\hlstd{se_y0}
\end{alltt}
\begin{verbatim}
##          [,1]
## [1,] 2.926941
\end{verbatim}
\begin{alltt}
\hlstd{crit_val} \hlkwb{<-} \hlkwd{qt}\hlstd{(}\hlnum{0.975}\hlstd{,} \hlnum{9}\hlstd{)}
\hlstd{pi_lo} \hlkwb{<-} \hlstd{y0_hat} \hlopt{-} \hlstd{crit_val} \hlopt{*} \hlstd{se_y0}
\hlstd{pi_hi} \hlkwb{<-} \hlstd{y0_hat} \hlopt{+} \hlstd{crit_val} \hlopt{*} \hlstd{se_y0}
\hlkwd{c}\hlstd{(y0_hat, pi_lo, pi_hi)}
\end{alltt}
\begin{verbatim}
## [1] 467.5824 460.9612 474.2036
\end{verbatim}
\end{kframe}
\end{knitrout}

\begin{itemize}
  \item $\Se*{Y_0-\hat{Y}_0}=\hat{\sigma}\sqrt{1+\symbf{x}_0(X^\top X)^{-1}\symbf{x}_0^\top}=2.926941$
\end{itemize}
\makeheading{ 2020-01-22 }

\begin{exbox}
    \begin{example}
        Construct $ GF(2^4=16) $.

        \textbf{Solution.} Take $ f(x)=x^4+x+1\in\mathbb{Z}_2[x] $.
        \begin{itemize}
            \item $ f $ has no roots in $ \mathbb{Z}_2 $ and hence no linear factors
            \item long division shows that $ x^2+x+1\nmid x^4+x+1 $, so $ f $
                  has no irreducible quadratic factors
            \item $ f $ is irreducible over $ \mathbb{Z}_2 $.
        \end{itemize}
        Thus, $ GF(16)=\mathbb{Z}_2[x]/(x^4+x+1) $.
    \end{example}
\end{exbox}

\section{Properties of Finite Fields}
\begin{thmbox}
    \begin{theorem}[Frosh's Dream]
        Let $ \alpha,\beta\in GF(q) $ where $ \ch(GF(q))=p $.
        \[ (\alpha + \beta)^p=\alpha^p+\beta^p \]
    \end{theorem}
\end{thmbox}

\begin{proof}
    \[ (\alpha + \beta)^p=\alpha^p+\sum\limits_{i=1}^{p-1}
        \binom{p}{i}\alpha^i\beta^{p-i}+\beta^p \]
    Now,
    \begin{align*}
        \binom{p}{i}=\frac{p!}{i!(p-i)!}
         & =\frac{p(p-1)\cdots (p-i+1)(p-i)(p-i-1)\cdots (2)(1)}{\left[i(i-1)\cdots(2)(1)\right]\left[(p-i)(p-i-1)\cdots(2)(1)\right]} \\
         & =p\left[\frac{(p-1)\cdots (p-i+2)}{i(i-1)\cdots(2)(1)}\right]
    \end{align*}
    If $ 1\leqslant i\leqslant p-1 $ then $ p\mid $ numerator, but
    $ p \nmid $ denominator. Thus,
    \[ p\mid \binom{p}{i}=p\lambda \]
    where $ \lambda\in\mathbb{N} $ with $ \lambda \neq 0 $ and $ p\nmid \lambda $.
    \begin{align*}
        \sum\limits_{i=1}^{p-1}\binom{p}{i}\alpha^i\beta^{p-i}
         & = \sum\limits_{i=1}^{p-1} (p\lambda_i) \alpha^i\beta^{p-i}                          \\
         & =\sum\limits_{i=1}^{p-1} (\underbrace{1+\cdots+1}_{p})\lambda_i \alpha^i\beta^{p-i} \\
         & =0
    \end{align*}
    Thus, $ (\alpha + \beta)^p=\alpha^p+\beta^p $.
\end{proof}

\begin{thmbox}
    \begin{corollary}
        \[ (\alpha+\beta)^{p^m}=\alpha^{p^m}+\beta^{p^m} \]
        for all $ m\geqslant 1 $.
    \end{corollary}
\end{thmbox}

\begin{proof}
    Exercise. Hint: Induction on $ m $.
\end{proof}

\begin{thmbox}
    \begin{theorem}
        Let $ \alpha\in GF(q) $. Then
        \[ \alpha^q=\alpha \]
    \end{theorem}
\end{thmbox}

\begin{proof}
    If $ \alpha=0 $, then $ \alpha^q=0=\alpha $.

    If $ \alpha\neq 0 $, let $ \{\alpha_1,\ldots ,\alpha_{q-1}\} $ be the
    non-zero elements in $ GF(q) $. Consider
    \[ \{\alpha\alpha_1,\ldots,\alpha\alpha_{q-1}\} \]
    Note that the elements in this list are pairwise distinct because if
    $ \alpha\alpha_i=\alpha\alpha_j $ with $ i\neq j $, then
    \[ \alpha^{-1}\alpha\alpha_i=\alpha^{-1}\alpha\alpha_j \]
    which implies that $ \alpha_i=\alpha_j $ which is a contradiction.
    Also $ \alpha\alpha_i\neq 0 $ for all $ i\in [1,q-1] $.
    Hence, $ \{\alpha_1,\ldots ,\alpha_{q-1}\}=\{\alpha\alpha_1,\ldots ,\alpha\alpha_{q-1}\} $.
    Therefore, $ \alpha_1\cdots\alpha_{q-1}=(\alpha\alpha_1)\cdots(\alpha\alpha_{q-1}) $.
    Hence, $ \alpha^{q-1}=1 $. Thus, $ \alpha^q=\alpha $.
\end{proof}

\begin{defbox}
    \begin{definition}
        Let $ GF(q)^*=GF(q)/\{0\} $.
    \end{definition}
\end{defbox}

\begin{defbox}
    \begin{definition}
        The \textbf{order of $\alpha\in GF(q)^*$}, denoted
        $ \ord(\alpha) $, is the smallest positive integer $ t $ such that
        $ \alpha^t=1 $.
    \end{definition}
\end{defbox}

\begin{exbox}
    \begin{example}
        How many elements of order $ 1 $ are there in $ GF(q) $?

        \textbf{Solution.} $ \alpha=1 $
    \end{example}
\end{exbox}

\begin{exbox}
    \begin{example}
        Find $ \ord(x) $ in $ GF(16)=\mathbb{Z}_2/(x^4+x+1) $.

        \textbf{Solution.}
        \begin{itemize}
            \item $ x^1=x $
            \item $ x^2=x^2 $
            \item $ x^3=x^3 $
            \item $ x^4=x+1 $
            \item $ x^5=x^2+x $
            \item $ x^6=x^3+x^2 $
            \item $ x^7=x^3+x+1 $
            \item $ x^8=x^2+1 $
            \item $ x^9=x^3+x $
            \item $ x^{10}=x^2+x+1 $
            \item $ x^{11}=x^3+x^2+x $
            \item $ x^{12}=x^3+x^2+x+1 $
            \item $ x^{13}=x^3+x^2+1 $
            \item $ x^{14}=x^3+1 $
            \item $ x^15\equiv 1\mod x^4+x+1 $
        \end{itemize}
        Since $ \ord(x)\neq 1,3,5 $ $ \ord(x)\mid 15 $, so we have $ \ord(x)=15 $.
    \end{example}
\end{exbox}

\begin{thmbox}
    \begin{lemma}
        Let $ \alpha\in GF(q)^* $, $ \ord(\alpha)=t $ and $ s\in\mathbb{Z} $.
        \[ \alpha^s=1\iff t\mid s \]
    \end{lemma}
\end{thmbox}

\begin{proof}
    Let $ s\in\mathbb{Z} $. By the division algorithm for integers, 
    \[ s=\ell t+r \]
    where $ 0\leqslant r\leqslant t-1 $. Then
    \[ \alpha^s=\alpha^{\ell t+r}=(\alpha^t)^\ell \alpha^r=\alpha^r \]
    So,
    \begin{align*}
        \alpha^s=1 & \iff a^r=1                                             \\
                   & \iff r=0 \qquad\text{since } 0\leqslant 1\leqslant t-1 \\
                   & \iff t\mid s
    \end{align*}
\end{proof}

\begin{thmbox}
    \begin{corollary}
        If $ \alpha\in GF(q)^* $, then $ \ord(\alpha)\mid (q-1) $.
    \end{corollary}
\end{thmbox}

\begin{proof}
    We know $ \alpha^{q-1}=1 $, so $ \ord(\alpha)\mid (q-1) $ by
    the previous Lemma.
\end{proof}

\begin{defbox}
    \begin{definition}
        An element $ \alpha\in GF(q) $ is a \textbf{generator} of
        $ GF(q)^* $ if 
        \[ \{\alpha^i:i\geqslant 0\}=GF(q)^* \]
        That is, $ \alpha $ generates all the non-zero field elements.
        $ \ord(\alpha)=q-1 $.
    \end{definition}
\end{defbox}

\begin{thmbox}
    \begin{theorem}
        If $ \alpha $ is a generator of $ GF(q)^* $, then
        \[ \{\alpha^1,\ldots ,\alpha^{q-1}\}=GF(q)^* \]
    \end{theorem}
\end{thmbox}

\section{2020-02-04}
\subsection{Preprocessor, Separate Compilation}
Source code $ \rightarrow $ Preprocessor $ \rightarrow $ Compiler
$ \rightarrow $ \code{a.out} executable

\code{\#include} is a direct copy/paste for the preprocessor directive

\code{\#include} with quotes $ \rightarrow $ look in current directory,
e.g. \code{\# include ``file''}

\code{g++ -E FILE} shows what the \code{include} copy/pastes

\code{1201/lectures/c++/4-preprocess}

File: \code{hello.cc}

\begin{itemize}
    \item \code{\#define VAR VALUE} $ \rightarrow $ searches and replaces
          \code{VAR} with \code{VALUE}
          \begin{itemize}
              \item \code{\#define MAX 10} $ \rightarrow $ replaces \code{MAX} with \code{10}
              \item Obsolete because now we have \code{const}
          \end{itemize}
\end{itemize}

\subsection{Conditional Compilation}
File: \code{course.cc}

\code{g++14 -D VAR=VALUE FILE} $ \rightarrow $ changes type in command-line
for \code{course.cc}

Preprocessor comment; nests perfectly:
\begin{lstlisting}
    #if 0
        ...
    #endif
\end{lstlisting}

Block comments (less powerful):
\begin{lstlisting}
    /*
    ...
    */
\end{lstlisting}

\begin{lstlisting}
    #define VAR // VAR gets empty string
    #ifdef VAR // if VAR is defined, true
    #ifndef VAR // if VAR is not defined, true
\end{lstlisting}

File: \code{debug.cc}

\code{g++14 -DDEBUG debug.cc} $ \rightarrow $ defines \code{DEBUG}, so
one can see full verbose due to the \code{\#ifdef}s in \code{debug.cc}

Note: \code{-D} can be used to define multiple variables

\subsection{Separate Compilation}
\begin{itemize}
    \item Header files (\code{.h}): Declarations of functions, Global Variables,
          and Type Definitions
    \item Implementation files (\code{.cc}): Definitions of functions
\end{itemize}

\code{1201/lectures/c++/5-separate}

File: \code{example1}

Compile the files with either:
\begin{itemize}
    \item \code{g++14 main.cc vec.cc}
    \item \code{g++14 *.cc}
\end{itemize}
Notes:
\begin{itemize}
    \item Implementation files (\code{.cc}) are never \code{include}d
    \item Implementation files (\code{.cc}) are compiled
    \item Header (\code{.h}) files are \textbf{never compiled}, they are \code{include}d
\end{itemize}

Want to compile each file separately to produce a position of the executable,
then finally merge these positions.

By default, \code{g++} will compile and link to produce the executable.

\code{g++14 -c vec.cc}, \code{g++14 -c main.cc} $ \rightarrow $ compiles each
separately, without the executable (\code{.o} (Object) files are produced).

\code{g++14 main.o vec.o -o myprog} $ \rightarrow $ merges them into
the \code{myprog} executable

\subsection{Build Tools}
Don't memorize any of this section.

\textbf{make}: Automatically use ``last modified timestamp'' (uses \code{ls -l})

Specify dependencies in a \code{Makefile}.

\code{1201/lectures/tools/1-make}

File: \code{example1}
\begin{itemize}
    \item \code{.PHONY} $ \rightarrow $ specifies that \code{clean} is not a file, but
          a command
          \begin{itemize}
              \item \code{clean} $ \rightarrow $ \code{make clean} will delete any \code{.o} files
                    in the current directory
          \end{itemize}
\end{itemize}

File: \code{example2}
\begin{itemize}
    \item Uses variables for compilation
    \item \code{-Wall} $ \rightarrow $ warn all, compiler will give errors for warnings
\end{itemize}

File: \code{example3}
\begin{itemize}
    \item \code{main.o}, \code{vec.o}, \code{myprog} within
          the Makefile is all one needs to change for A2Q5
\end{itemize}

\section{2020-02-06}
\subsection{Preprocessor, Include Guards, C++ Classes}
File: \code{example3}
\begin{itemize}
      \item won't compile as \code{vec.h} gets included twice
      \item use an include guard to prevent multiple includes
\end{itemize}
File: \code{example4} $ \rightarrow $ fixes the issue above

In \code{vec.h},
\begin{lstlisting}
    #ifndef VEC_H // true
        #define VEC_H
        struct Vec {
            ...
        }
    #endif
\end{lstlisting}
Never put \code{using namespace std;} in a header file since it forces
others to use the \code{namespace}.

\subsection{C++ Classes}
A C++ class is a \code{struct} that may contain functions.

The big innovation of OOP\@: \code{struct}s can have functions.

File: \code{student.h}
\begin{lstlisting}
    struct Student {
        int assign, mt, final;
        // since grade is only relevant for this function, we declare it here
        float grade(); // good style to have declaration only, and not the entire thing
    };
\end{lstlisting}

File: \code{student.cc}
\begin{lstlisting}
    #include "student.h"
    float Student::grade() {
        return 0.4 * assign + 0.2 * mt + 0.4 * final;
    }
\end{lstlisting}
\code{std::ostream} $ \rightarrow $
In the scope of the standard namespace, there is an \code{ostream}.
Above, the same thing is happening.

An \textbf{object} is an instance of a class.
\begin{lstlisting}
    // Bobby is an object.
    Student Bobby{75, 50, 65};
    // Let's compute Bobby's grade.
    cout << Bobby.grade();
\end{lstlisting}

\begin{itemize}
      \item A function within a class is called a \textbf{member function}
            or \textbf{method}.
      \item You can only call methods using objects of the class.
      \item All methods have a hidden parameter named \code{this}
            $ \rightarrow $ a pointer to the object used to call the method.
            \begin{itemize}
                  \item \code{this == \&bobby}
            \end{itemize}
\end{itemize}
\code{ptr -> field} is the same as \code{(*ptr).field}. Equivalently as in
\code{student.cc}.
\begin{lstlisting}
    return 0.4 * this->assign + 0.2 * this->mt + 0.4 * this->final;
\end{lstlisting}

\subsection{Initializing Objects}
C style initialization:

\code{Student Bobby = \{75,50,65\}} $ \rightarrow $ not going to use this syntax,
but it's allowed.

In C++:

Special methods to construct objects are called \textbf{constructors}, they do
not need a return type.

Header file (\code{.h}):
\begin{lstlisting}
    // same name as class
    struct Student {
        ...
        Student (int assign, int mt, int final); // declaration, no return type
    };
\end{lstlisting}
Implementation file (\code{.cc}):
\begin{lstlisting}
    Student::Student (int assign, int mt, int final) {
        // cannot do assign = assign;
        this->assign = assign < 0 ? 0 : assign; // short hand if statement
        this->mt = mt;
        this->final = final;
    }
    Student s1{70, 60, 75};
    Student s2{70, 60}; // final = 0
    Student s3{70}; // mt = final = 0
    Student s4{}; // assign = mt = final = 0
    Student s5; // equivalent to Student s4
\end{lstlisting}
Older initialization:
\begin{lstlisting}
    Student Bobby = Student(75, 50, 65);
\end{lstlisting}
Heap allocated \code{Student}:
\begin{lstlisting}
    // round or curly braces acceptable, but curly braces is good style
    Student *p = new Student{75, 50, 65};
    ...
    delete p;
\end{lstlisting}
\textbf{Default constructor}:
A zero parameter constructor. Alternatively, it is a constructor
where all parameters have default values.

Every class comes with a built-in/free default constructor. It calls default
constructors on any fields that are objects.

\begin{lstlisting}
    struct MyClass {
        int x;
        Student s;
        Vec *p;
    };
    Myclass a;
\end{lstlisting}
For \code{a}, the default constructor:
\begin{itemize}
      \item initializes \code{s} as it is an object \code{Student}.
      \item \textbf{does not} initialize \code{p} as is not
            an object, but \textbf{a pointer to an object} \code{Vec}.
      \item \textbf{does not} initialize \code{x}.
\end{itemize}
As soon as you write any constructor, you lose the built-in default constructor
and C style initialization, for example:
\begin{lstlisting}
    struct Vec {
        int x, y;
        Vec (int x, int y) { // bad style
            this->x = x;
            this->y = y;
        }
    };
    Vec v; // does not compile
\end{lstlisting}
Initializing constant fields:
\begin{lstlisting}
    int m;
    // probably not what you want, all objects here have a constant id of 10
    struct MyClass {
        const int id = 10; // in class initialization
        int &n = m;
    };
\end{lstlisting}
\begin{lstlisting}
    struct Student {
        const int id; // want to figure out how to do this, next class
    };
\end{lstlisting}

\makeheading{2019-10-22}
Given any LP problem, we know how to convert it into an equivalent LP
problem in SEF\@:

(P)
\[\max z:=\bm{c}^\top \bm{x}\]
subject to
\begin{align*}
    A \bm{x}=\bm{b} \\
    \bm{x}\geqslant  0
\end{align*}
where $ A\in\mathbb{R}^{m\times n}$ has $\rank(A)=m $.

Given an LP in SEF, with a given basic feasible solution, we know
how to solve it.

\section{Finding Feasible Solutions}
Given an LP in SEF with $ \rank(A)=m $, how do we find a feasible
solution or prove that none exists.

We will construct an \emph{auxiliary LP problem}.

We can always make sure $ \bm{b}\geqslant  \bm{0} $. (If any $ b_i<0 $, multiply both
sides of that equation by $ (-1) $) Introduce artificial variables
$ x_{n+1},\ldots,x_{n+m} $

\begin{defbox}
    \begin{definition}
        Given (P): $\max \left\{ \bm{c}^\top \bm{x},\,A \bm{x}=\bm{b},\bm{x}\geqslant  \bm{0} \right\}$,
        we define the \emph{auxiliary linear program} of (P) as:

        $ (P_{aux}) $
        \[ \min w:=x_{n+1}+\cdots+x_{n+m} \]
        subject to
        \[
            \left[\begin{array}{c|c}
                    A & I
                \end{array}\right]
            \underbrace{\begin{bmatrix}
                    x_1     \\
                    \vdots  \\
                    x_{n}   \\
                    x_{n+1} \\
                    \vdots  \\
                    x_{n+m}
                \end{bmatrix}}_{\bm{x}}
            =\bm{b}\]
        \[ \bm{x} \geqslant  \bm{0}\]
        where $ \bm{b}\geqslant  \bm{0} $, and $ I $ is the $ m\times m $ identity matrix.

        We call the variables $ x_{n+1},\ldots,x_{n+m} $ \emph{auxiliary variables}.
    \end{definition}
\end{defbox}

For every feasible solution of $ (P_{aux}) $, $ w\geqslant  0 $.

Therefore, $ (P_{aux}) $ is not unbounded.

If the optimal value of $ (P_{aux}) $ is zero, let
$ (\hat{x}_1,\ldots,\hat{x}_{n+m})^\top$
be the basic feasible solution of $ (P_{aux}) $. Then,
$ (\hat{x}_1,\ldots,\hat{x}_{n})^\top$
is a basic feasible solution of (P).

It is basic since $ \{A_j : \hat{x_j}>0\} $ is linearly independent where
$ J $ is the column indices $ j $ of $ A $ for which $ \hat{x}_j\neq 0 $.

If $ |\{j:\hat{x_j}>0\}|=m $, this index set is a basis of $ A $ which
determines $ (\hat{x}_1,\ldots,\hat{x}_{n})^\top$.

If $ |\{j:\hat{x_j}>0\}|\leqslant m-1 $, we can extend this index set
to be a basis of $ A $, since $ \rank(A)=m $.

If the optimal value of $ (P_{aux}) $ is positive, then (P) is
infeasible. We state this as a theorem.

\begin{thmbox}
    \begin{theorem}
        Let $ \bar{\bm{x}}=(\bar{x}_1,\ldots ,\bar{x}_{n+m})^\top $ be an optimal solution
        to $ (P_{aux}) $.
        \begin{enumerate}[label=(\arabic*)]
            \item if $ w=0 $, then $ (\bar{x}_1,\ldots,\bar{x}_n)^\top $ is a solution to (P).
            \item if $ w>0 $, then (P) is infeasible.
        \end{enumerate}
    \end{theorem}
\end{thmbox}

\begin{proof}
    Let $ \bar{\bm{x}}=(\bar{x}_1,\ldots ,\bar{x}_{n+m})^\top $ be an optimal solution
    to $ (P_{aux}) $.

    (1) Assume $ w=0 $, then $ \bar{x}_{n+1}=\cdots=\bar{x}_{n+m}=0 $. Thus
    $ (\bar{x}_1,\ldots \bar{x}_n)^\top $ is a feasible solution of (P).

    (2) Assume $ w>0 $. Suppose for a contradiction that there exists a feasible
    solution $(\bar{x}_1,\ldots \bar{x}_n)^\top$ to (P). Then,
    $ (\bar{x}_1,\ldots \bar{x}_n,\underbrace{0,\ldots ,0}_{m\text{ terms}}) $ is a feasible solution to $ (P_{aux}) $
    with optimal objective value $0$ which is a contradiction to the fact that
    $ \bm{\bar{x}} $ is optimal.
\end{proof}

\begin{algbox}
    \begin{algorithm}[H]
        \caption{Two Phase Method}
        \SetKwInOut{Input}{Input}
        \SetKwInOut{Output}{Output}
        \Input{$A,\bm{b}, \bm{c}$ data for LP in SEF such that
            full row rank and $ \bm{b}\geqslant  \bm{0} $.}
        Construct $ (P_{aux}) $ put into SEF, $ B:=\{n+1,n+2,\ldots,n+m\} $\\
        Put $ (P_{aux}) $ into the canonical form determined by $ B $.\\
        Solve $ (P_{aux}) $ starting with basis $ B $ by Simplex Method.\\
        If the optimal value of $ (P_{aux}) $ is zero, then we have a basic
        feasible solution of (P). Solve (P) using Simplex Method. This is
        known as Phase II.\\
        If the optimal objective value of $ (P_{aux}) $ is not zero, then
        (P) is infeasible (a certificate of infeasibility is given by
        the last $ \bm{\bar{y}} $ computed).\\
    \end{algorithm}
\end{algbox}

As seen above, the original LP can either have an optimal
solution or be infeasible when performing the Two Phase Method.

\subsection{The Two Phase Simplex Algorithm---An Optimal Example}
\begin{exbox}
    \begin{example}[Two Phase---Optimal]
        (P)
        \[ \max z:= \begin{bmatrix} 1 & 2 & -1 \end{bmatrix} \bm{x} \]
        subject to
        \[
            \begin{bmatrix}
                1  & -2 & -3 \\
                -1 & 1  & 1
            \end{bmatrix}
            \bm{x} =
            \begin{bmatrix}
                -3 \\
                1
            \end{bmatrix}
        \]
        \[ \bm{x}\geqslant  \bm{0} \]

        Since $ b_1<0 $
        we write
        \[ \begin{bmatrix}
                -1 & 2 & 3 \\
                -1 & 1 & 1
            \end{bmatrix}
            x=
            \begin{bmatrix}
                3 \\
                1
            \end{bmatrix}
        \]
        we do this because we will not have a feasible solution if $ \bm{b}<\bm{0} $.

        Introduce artificial variables: $ x_4, x_5 $

        \underline{Phase I}

        $ (P_{aux}) $
        \[\max -w:=\begin{bmatrix} 0 & 0 & 0 & -1 & -1 \end{bmatrix} \bm{x} \]
        subject to
        \[
            \begin{bmatrix}
                -1 & 2 & 3 & 1 & 0 \\
                -1 & 1 & 1 & 0 & 1
            \end{bmatrix}
            \bm{x}=
            \begin{bmatrix}
                3 \\
                1
            \end{bmatrix}
        \]
        \[ \bm{x}:=(x_1,x_2,x_3,x_4,x_5)^\top \geqslant  \bm{0} \]

        Turn $ (P_{aux}) $ into canonical form for $ B:=\{4,5\} $
        \[\max -w:=\begin{bmatrix} -2 & 3 & 4 & 0 & 0 \end{bmatrix} \bm{x} \]
        subject to
        \[
            \begin{bmatrix}
                -1 & 2 & 3 & 1 & 0 \\
                -1 & 1 & 1 & 0 & 1
            \end{bmatrix}
            \bm{x}=
            \begin{bmatrix}
                3 \\
                1
            \end{bmatrix}
        \]
        \[ \bm{x}\geqslant  \bm{0} \]
        We solve the LP via Simplex Algorithm and obtain the following
        LP corresponding to the optimal basis of $ B=\{1,2\} $
        \[\max -w:=\begin{bmatrix} 0 & 0 & 0 & -1 & -1 \end{bmatrix} \bm{x} \]
        subject to
        \[
            \begin{bmatrix}
                1 & 0 & 1 & 1 & -2 \\
                0 & 1 & 2 & 1 & -1
            \end{bmatrix}
            \bm{x}=
            \begin{bmatrix}
                1 \\
                2
            \end{bmatrix}
        \]
        \[ \bm{x}\geqslant  \bm{0} \]
        End of Phase I.

        \underline{Phase II}

        We rewrite the original LP in canonical form corresponding
        to basis $ B=\{1,2\} $ to obtain
        \[ \max z:= \begin{bmatrix}0 & 0 & -6 \end{bmatrix}\bm{x}+5 \]
        subject to
        \[ \begin{bmatrix}
                1 & 0 & 1 \\
                0 & 1 & 2
            \end{bmatrix}\bm{x}=
            \begin{bmatrix}
                1 \\
                2
            \end{bmatrix} \]
        \[ \bm{x}:=(x_1,x_2,x_3)^\top\geqslant  \bm{0} \]

        We obtain the optimal basic feasible solution of $ (P_{aux}) $ via the Simplex
        Algorithm
        \[(\hat{x_1}, \hat{x_2}, \hat{x_3}, \hat{x_4}, \hat{x_5}):=
            (1,2,0,0,0)^\top\]
        Thus, the corresponding basic feasible solution of (P) is
        \[(\hat{x_1}, \hat{x_2}, \hat{x_3})=(1,2,0)^\top\]
        with an optimal objective value of $ z:= 5 $.
    \end{example}
\end{exbox}

\subsection{The Two Phase Simplex Algorithm---An Infeasible Example}
\begin{exbox}
    \begin{example}[Two Phase---Infeasible]
        (P)
        \[ \max z:=\begin{bmatrix} 3 & 2 & 4 \end{bmatrix} \bm{x} \]
        subject to
        \[
            \underbrace{
                \begin{bmatrix}
                    5  & 1 & 1 \\
                    -1 & 1 & 2 \\
                \end{bmatrix}}_{A}
            \bm{x}=
            \begin{bmatrix}
                1 \\
                5
            \end{bmatrix}
        \]
        \[\bm{x}\geqslant  \bm{0}\]
        $ (P_{aux}) $
        \[ \max -w:=\underbrace{\left[\begin{array}{ccccc}
                        0 & 0 & 0 & -1 & -1
                    \end{array}\right]}_{\bm{\tilde{c_B}}} \bm{x} \]
        subject to
        \[
            \underbrace{\left[
                    \begin{array}{ccccc}
                        5  & 1 & 1 & 1 & 0 \\
                        -1 & 1 & 2 & 0 & 1
                    \end{array}\right]}_{\tilde{A}}
            \bm{x}=
            \begin{bmatrix}
                1 \\
                5
            \end{bmatrix}
        \]
        \[ \bm{x}:=(x_1,x_2,x_3,x_4,x_5)^\top \geqslant  \bm{0} \]
        Turn $ (P_{aux}) $ into canonical form for $ B:=\{4,5\} $ (by adding
        the constraints up to the original objective function).
        \[ \max -w:=\left[\begin{array}{ccccc}
                    4 & 2 & 3 & 0 & 0
                \end{array}\right] \bm{x} - 4 \]
        subject to
        \[
            \left[
                \begin{array}{ccccc}
                    5  & 1 & 1 & 1 & 0 \\
                    -1 & 1 & 2 & 0 & 1
                \end{array}\right]
            \bm{x}=
            \begin{bmatrix}
                1 \\
                5
            \end{bmatrix}
        \]
        \[ \bm{x}\geqslant  \bm{0} \]
        Starting with the basis $ B=\{4,5\} $, solve $ (P_{aux}) $
        with the Simplex Algorithm to get:
        \[ \max -w=\begin{bmatrix}-11 & -1 & 0 & -3 & 0\end{bmatrix} \bm{x} - 3 \]
        subject to
        \[
            \begin{bmatrix}
                5   & 1  & 1 & 1  & 0 \\
                -11 & -1 & 0 & -2 & 1
            \end{bmatrix}
            \bm{x}=
            \begin{bmatrix}
                1 \\
                3
            \end{bmatrix}
        \]
        \[ \bm{x}\geqslant  \bm{0} \]

        The optimal value of $ (P_{aux}) $ is not zero. Therefore, (P) is
        infeasible. The basis in the last iteration was $ B=\{3,5\} $.

        \[ \bm{y}^\top =\bm{\tilde{c}_B}^\top\tilde{A}_B^{-1} \iff
            \bm{y}^\top
            =
            \underbrace{\begin{bmatrix}0 & -1\end{bmatrix}}_
            {\text{SEF of $ (P_{aux}) $ }}
            \begin{bmatrix}
                1 & 0 \\
                2 & 1
            \end{bmatrix}^{-1}
        \]
        $ \bm{\bar{y}}=(2,-1)^\top $
        is a certificate of infeasibility of (P).

        Compute $ \bm{\bar{y}}^\top  A
            = \begin{bmatrix}11 & 1 & 0 \end{bmatrix}\geqslant  \bm{0}^\top $ and
        $ \bm{\bar{y}}^\top \bm{b}=-3=\bm{c}^\top \bm{x} $
        where $ \bm{\bar{x}}=(0,0,0,1,3)^\top $.

        Thus, $ \bm{\bar{y}} $ is a certificate of optimality for $ (P_{aux}) $.
    \end{example}
\end{exbox}

\begin{thmbox}
    \begin{theorem}[Fundamental Theorem of LP (SEF)]
        Let (P) be an LP problem in SEF, where $ A\in \mathbb{R}^{m\times n} $ has $ \rank(A)=m $.
        \begin{enumerate}[label=(\arabic*)]
            \item if (P) does not have an optimal solution, then (P) is either infeasible or unbounded.
            \item if (P) has a feasible solution, then (P) has a basic feasible solution.
            \item if (P) has an optimal solution, then (P) has an optimal basic feasible solution.
        \end{enumerate}
    \end{theorem}
\end{thmbox}


\subsection{R Demo}
\begin{knitrout}
\definecolor{shadecolor}{rgb}{0.969, 0.969, 0.969}\color{fgcolor}\begin{kframe}
\begin{alltt}
\hlcom{## R demo for Oct 19 Plotting functions and histograms, F}
\hlcom{## distribution, ANOVA tables, F tests, MLR with}
\hlcom{## categorical variables}
\end{alltt}
\end{kframe}
\end{knitrout}
Evaluate the function at many $x$ values, then plot it.
\begin{knitrout}
\definecolor{shadecolor}{rgb}{0.969, 0.969, 0.969}\color{fgcolor}\begin{kframe}
\begin{alltt}
\hlcom{# Plotting functions (e.g., probability density functions)}
\hlcom{# Create sequence from -4 to 4 increasing 0.01 each time.}
\hlstd{x} \hlkwb{<-} \hlkwd{seq}\hlstd{(}\hlopt{-}\hlnum{4}\hlstd{,} \hlnum{4}\hlstd{,} \hlnum{0.01}\hlstd{)}
\hlkwd{head}\hlstd{(x)}
\end{alltt}
\begin{verbatim}
## [1] -4.00 -3.99 -3.98 -3.97 -3.96 -3.95
\end{verbatim}
\begin{alltt}
\hlcom{# Normal probability density function with mean 0, and}
\hlcom{# standard deviation 1.}
\hlstd{y} \hlkwb{<-} \hlkwd{dnorm}\hlstd{(x,} \hlnum{0}\hlstd{,} \hlnum{1}\hlstd{)}
\end{alltt}
\end{kframe}
\end{knitrout}

\code{dnorm} is for density normal.

\begin{knitrout}
\definecolor{shadecolor}{rgb}{0.969, 0.969, 0.969}\color{fgcolor}\begin{kframe}
\begin{alltt}
\hlkwd{plot}\hlstd{(x, y,} \hlkwc{type} \hlstd{=} \hlstr{"l"}\hlstd{)}
\end{alltt}
\end{kframe}

{\centering \includegraphics[width=\maxwidth]{figure/unnamed-chunk-56-1} 

}


\end{knitrout}

\code{type = "l"} is for a smooth line (instead of dots).

We can also plot $y=x^2$ for example.
\begin{knitrout}
\definecolor{shadecolor}{rgb}{0.969, 0.969, 0.969}\color{fgcolor}\begin{kframe}
\begin{alltt}
\hlstd{y} \hlkwb{<-} \hlstd{x}\hlopt{^}\hlnum{2}
\hlkwd{plot}\hlstd{(x, y,} \hlkwc{type} \hlstd{=} \hlstr{"l"}\hlstd{)}
\end{alltt}
\end{kframe}

{\centering \includegraphics[width=\maxwidth]{figure/unnamed-chunk-57-1} 

}


\end{knitrout}

\subsubsection{F-distribution Examples}

\begin{knitrout}
\definecolor{shadecolor}{rgb}{0.969, 0.969, 0.969}\color{fgcolor}\begin{kframe}
\begin{alltt}
\hlstd{x} \hlkwb{<-} \hlkwd{seq}\hlstd{(}\hlnum{0}\hlstd{,} \hlnum{5}\hlstd{,} \hlnum{0.01}\hlstd{)}
\hlkwd{head}\hlstd{(x)}
\end{alltt}
\begin{verbatim}
## [1] 0.00 0.01 0.02 0.03 0.04 0.05
\end{verbatim}
\begin{alltt}
\hlcom{# df is degrees of freedom.  type = 'l' is for a smooth}
\hlcom{# curve}
\hlkwd{plot}\hlstd{(x,} \hlkwc{y} \hlstd{=} \hlkwd{df}\hlstd{(x,} \hlkwc{df1} \hlstd{=} \hlnum{1}\hlstd{,} \hlkwc{df2} \hlstd{=} \hlnum{1}\hlstd{),} \hlkwc{type} \hlstd{=} \hlstr{"l"}\hlstd{,} \hlkwc{xlab} \hlstd{=} \hlstr{"x"}\hlstd{,}
  \hlkwc{ylab} \hlstd{=} \hlstr{"density"}\hlstd{)}
\end{alltt}
\end{kframe}

{\centering \includegraphics[width=\maxwidth]{figure/unnamed-chunk-58-1} 

}


\begin{kframe}\begin{alltt}
\hlcom{# ylim is for the y-axis limits lwd is for line width}
\hlkwd{plot}\hlstd{(x,} \hlkwc{y} \hlstd{=} \hlkwd{df}\hlstd{(x,} \hlkwc{df1} \hlstd{=} \hlnum{1}\hlstd{,} \hlkwc{df2} \hlstd{=} \hlnum{1}\hlstd{),} \hlkwc{type} \hlstd{=} \hlstr{"l"}\hlstd{,} \hlkwc{col} \hlstd{=} \hlstr{"black"}\hlstd{,}
  \hlkwc{xlab} \hlstd{=} \hlstr{"x"}\hlstd{,} \hlkwc{ylab} \hlstd{=} \hlstr{"density"}\hlstd{,} \hlkwc{ylim} \hlstd{=} \hlkwd{c}\hlstd{(}\hlnum{0}\hlstd{,} \hlnum{2.5}\hlstd{),} \hlkwc{lwd} \hlstd{=} \hlnum{2}\hlstd{)}
\hlcom{# Add lines to the existing plot.}
\hlkwd{lines}\hlstd{(x,} \hlkwc{y} \hlstd{=} \hlkwd{df}\hlstd{(x,} \hlkwc{df1} \hlstd{=} \hlnum{1}\hlstd{,} \hlkwc{df2} \hlstd{=} \hlnum{100}\hlstd{),} \hlkwc{type} \hlstd{=} \hlstr{"l"}\hlstd{,} \hlkwc{col} \hlstd{=} \hlstr{"green"}\hlstd{,}
  \hlkwc{lwd} \hlstd{=} \hlnum{2}\hlstd{)}
\hlkwd{lines}\hlstd{(x,} \hlkwc{y} \hlstd{=} \hlkwd{df}\hlstd{(x,} \hlkwc{df1} \hlstd{=} \hlnum{5}\hlstd{,} \hlkwc{df2} \hlstd{=} \hlnum{1}\hlstd{),} \hlkwc{type} \hlstd{=} \hlstr{"l"}\hlstd{,} \hlkwc{col} \hlstd{=} \hlstr{"blue"}\hlstd{,}
  \hlkwc{lwd} \hlstd{=} \hlnum{2}\hlstd{)}
\hlkwd{lines}\hlstd{(x,} \hlkwc{y} \hlstd{=} \hlkwd{df}\hlstd{(x,} \hlkwc{df1} \hlstd{=} \hlnum{5}\hlstd{,} \hlkwc{df2} \hlstd{=} \hlnum{100}\hlstd{),} \hlkwc{type} \hlstd{=} \hlstr{"l"}\hlstd{,} \hlkwc{col} \hlstd{=} \hlstr{"purple"}\hlstd{,}
  \hlkwc{lwd} \hlstd{=} \hlnum{2}\hlstd{)}
\hlkwd{lines}\hlstd{(x,} \hlkwc{y} \hlstd{=} \hlkwd{df}\hlstd{(x,} \hlkwc{df1} \hlstd{=} \hlnum{10}\hlstd{,} \hlkwc{df2} \hlstd{=} \hlnum{1}\hlstd{),} \hlkwc{type} \hlstd{=} \hlstr{"l"}\hlstd{,} \hlkwc{col} \hlstd{=} \hlstr{"red"}\hlstd{,}
  \hlkwc{lwd} \hlstd{=} \hlnum{2}\hlstd{)}
\hlkwd{lines}\hlstd{(x,} \hlkwc{y} \hlstd{=} \hlkwd{df}\hlstd{(x,} \hlkwc{df1} \hlstd{=} \hlnum{10}\hlstd{,} \hlkwc{df2} \hlstd{=} \hlnum{100}\hlstd{),} \hlkwc{type} \hlstd{=} \hlstr{"l"}\hlstd{,} \hlkwc{col} \hlstd{=} \hlstr{"orange"}\hlstd{,}
  \hlkwc{lwd} \hlstd{=} \hlnum{2}\hlstd{)}
\hlcom{# Add a legend to the top-right.  lty = 1 is for a straight}
\hlcom{# solid line.}
\hlkwd{legend}\hlstd{(}\hlstr{"topright"}\hlstd{,} \hlkwc{legend} \hlstd{=} \hlkwd{c}\hlstd{(}\hlstr{"df1=1, df2=1"}\hlstd{,} \hlstr{"df1=1, df2=100"}\hlstd{,}
  \hlstr{"df1=5, df2=1"}\hlstd{,} \hlstr{"df1=5, df2=100"}\hlstd{,} \hlstr{"df1=10, df2=1"}\hlstd{,} \hlstr{"df1=10, df2=100"}\hlstd{),}
  \hlkwc{lty} \hlstd{=} \hlnum{1}\hlstd{,} \hlkwc{col} \hlstd{=} \hlkwd{c}\hlstd{(}\hlstr{"black"}\hlstd{,} \hlstr{"green"}\hlstd{,} \hlstr{"blue"}\hlstd{,} \hlstr{"purple"}\hlstd{,} \hlstr{"red"}\hlstd{,}
    \hlstr{"orange"}\hlstd{))}
\end{alltt}
\end{kframe}

{\centering \includegraphics[width=\maxwidth]{figure/unnamed-chunk-58-2} 

}


\end{knitrout}

\subsubsection{Random numbers for the F-distribution}

\begin{knitrout}
\definecolor{shadecolor}{rgb}{0.969, 0.969, 0.969}\color{fgcolor}\begin{kframe}
\begin{alltt}
\hlcom{# set.seed allows for exact reproduction.}
\hlkwd{set.seed}\hlstd{(}\hlnum{12345678}\hlstd{)}
\hlstd{randF} \hlkwb{<-} \hlkwd{rf}\hlstd{(}\hlnum{1000}\hlstd{,} \hlnum{5}\hlstd{,} \hlnum{100}\hlstd{)}
\hlcom{# Generate histogram for the random numbers with exact.}
\hlkwd{hist}\hlstd{(randF)}
\end{alltt}
\end{kframe}

{\centering \includegraphics[width=\maxwidth]{figure/unnamed-chunk-59-1} 

}


\begin{kframe}\begin{alltt}
\hlcom{# Generate histogram for the random numbers with relative}
\hlcom{# frequency.  This is normalized, so we can superimpose an}
\hlcom{# F-distribution to it.}
\hlkwd{hist}\hlstd{(randF,} \hlkwc{freq} \hlstd{=} \hlnum{FALSE}\hlstd{)}
\hlcom{# Superimpose an F-distribution on the histogram.}
\hlkwd{lines}\hlstd{(x,} \hlkwc{y} \hlstd{=} \hlkwd{df}\hlstd{(x,} \hlkwc{df1} \hlstd{=} \hlnum{5}\hlstd{,} \hlkwc{df2} \hlstd{=} \hlnum{100}\hlstd{),} \hlkwc{type} \hlstd{=} \hlstr{"l"}\hlstd{,} \hlkwc{col} \hlstd{=} \hlstr{"purple"}\hlstd{,}
  \hlkwc{lwd} \hlstd{=} \hlnum{2}\hlstd{)}
\end{alltt}
\end{kframe}

{\centering \includegraphics[width=\maxwidth]{figure/unnamed-chunk-59-2} 

}


\begin{kframe}\begin{alltt}
\hlcom{# Set y-axis limits and more detailed histogram bins using}
\hlcom{# 'breaks = 25'}
\hlkwd{hist}\hlstd{(randF,} \hlkwc{freq} \hlstd{=} \hlnum{FALSE}\hlstd{,} \hlkwc{ylim} \hlstd{=} \hlkwd{c}\hlstd{(}\hlnum{0}\hlstd{,} \hlnum{0.8}\hlstd{),} \hlkwc{breaks} \hlstd{=} \hlnum{25}\hlstd{)}
\hlkwd{lines}\hlstd{(x,} \hlkwc{y} \hlstd{=} \hlkwd{df}\hlstd{(x,} \hlkwc{df1} \hlstd{=} \hlnum{5}\hlstd{,} \hlkwc{df2} \hlstd{=} \hlnum{100}\hlstd{),} \hlkwc{type} \hlstd{=} \hlstr{"l"}\hlstd{,} \hlkwc{col} \hlstd{=} \hlstr{"purple"}\hlstd{,}
  \hlkwc{lwd} \hlstd{=} \hlnum{2}\hlstd{)}
\end{alltt}
\end{kframe}

{\centering \includegraphics[width=\maxwidth]{figure/unnamed-chunk-59-3} 

}


\begin{kframe}\begin{alltt}
\hlcom{# Generate more random F-distributions to get closer to the}
\hlcom{# 'true' density.}
\hlstd{randF} \hlkwb{<-} \hlkwd{rf}\hlstd{(}\hlnum{10000}\hlstd{,} \hlnum{5}\hlstd{,} \hlnum{100}\hlstd{)}
\hlkwd{hist}\hlstd{(randF,} \hlkwc{freq} \hlstd{=} \hlnum{FALSE}\hlstd{,} \hlkwc{ylim} \hlstd{=} \hlkwd{c}\hlstd{(}\hlnum{0}\hlstd{,} \hlnum{0.8}\hlstd{),} \hlkwc{breaks} \hlstd{=} \hlnum{25}\hlstd{)}
\hlkwd{lines}\hlstd{(x,} \hlkwc{y} \hlstd{=} \hlkwd{df}\hlstd{(x,} \hlkwc{df1} \hlstd{=} \hlnum{5}\hlstd{,} \hlkwc{df2} \hlstd{=} \hlnum{100}\hlstd{),} \hlkwc{type} \hlstd{=} \hlstr{"l"}\hlstd{,} \hlkwc{col} \hlstd{=} \hlstr{"purple"}\hlstd{,}
  \hlkwc{lwd} \hlstd{=} \hlnum{2}\hlstd{)}
\end{alltt}
\end{kframe}

{\centering \includegraphics[width=\maxwidth]{figure/unnamed-chunk-59-4} 

}


\end{knitrout}

\subsubsection{Revisit Rocket Example}

\begin{knitrout}
\definecolor{shadecolor}{rgb}{0.969, 0.969, 0.969}\color{fgcolor}\begin{kframe}
\begin{alltt}
\hlstd{rocket} \hlkwb{<-} \hlkwd{read.csv}\hlstd{(}\hlstr{"csv/rocket.csv"}\hlstd{)}
\hlstd{m1} \hlkwb{<-} \hlkwd{lm}\hlstd{(thrust} \hlopt{~} \hlstd{nozzle} \hlopt{+} \hlstd{propratio,} \hlkwc{data} \hlstd{= rocket)}
\hlkwd{summary}\hlstd{(m1)}
\end{alltt}
\begin{verbatim}
## 
## Call:
## lm(formula = thrust ~ nozzle + propratio, data = rocket)
## 
## Residuals:
##     Min      1Q  Median      3Q     Max 
## -3.8459 -1.7555  0.5934  1.2906  3.3008 
## 
## Coefficients:
##             Estimate Std. Error t value Pr(>|t|)    
## (Intercept) 473.6039     4.7158 100.430 4.88e-15 ***
## nozzle       16.7383     1.5329  10.919 1.71e-06 ***
## propratio    -1.0948     0.9414  -1.163    0.275    
## ---
## Signif. codes:  0 '***' 0.001 '**' 0.01 '*' 0.05 '.' 0.1 ' ' 1
## 
## Residual standard error: 2.655 on 9 degrees of freedom
## Multiple R-squared:  0.9303,	Adjusted R-squared:  0.9148 
## F-statistic: 60.05 on 2 and 9 DF,  p-value: 6.238e-06
\end{verbatim}
\begin{alltt}
\hlcom{# Compare summary with ANOVA table on board from Oct. 5.}
\hlkwd{anova}\hlstd{(m1)}
\end{alltt}
\begin{verbatim}
## Analysis of Variance Table
## 
## Response: thrust
##           Df Sum Sq Mean Sq  F value    Pr(>F)    
## nozzle     1 836.67  836.67 118.7377 1.743e-06 ***
## propratio  1   9.53    9.53   1.3524    0.2748    
## Residuals  9  63.42    7.05                       
## ---
## Signif. codes:  0 '***' 0.001 '**' 0.01 '*' 0.05 '.' 0.1 ' ' 1
\end{verbatim}
\begin{alltt}
\hlkwd{anova}\hlstd{(m1)}\hlopt{$}\hlstd{`Sum Sq`}
\end{alltt}
\begin{verbatim}
## [1] 836.670000   9.529332  63.417335
\end{verbatim}
\begin{alltt}
\hlkwd{sum}\hlstd{(}\hlkwd{anova}\hlstd{(m1)}\hlopt{$}\hlstd{`Sum Sq`[}\hlnum{1}\hlopt{:}\hlnum{2}\hlstd{])}
\end{alltt}
\begin{verbatim}
## [1] 846.1993
\end{verbatim}
\begin{alltt}
\hlstd{SSRes} \hlkwb{<-} \hlkwd{anova}\hlstd{(m1)}\hlopt{$}\hlstd{`Sum Sq`[}\hlnum{3}\hlstd{]}
\hlcom{# Test of overall significance.}
\hlstd{m_red} \hlkwb{<-} \hlkwd{lm}\hlstd{(thrust} \hlopt{~} \hlnum{1}\hlstd{,} \hlkwc{data} \hlstd{= rocket)}
\hlkwd{summary}\hlstd{(m_red)}
\end{alltt}
\begin{verbatim}
## 
## Call:
## lm(formula = thrust ~ 1, data = rocket)
## 
## Residuals:
##      Min       1Q   Median       3Q      Max 
## -13.4167  -7.1167  -0.2167   8.2333  11.3833 
## 
## Coefficients:
##             Estimate Std. Error t value Pr(>|t|)    
## (Intercept)  476.617      2.625   181.6   <2e-16 ***
## ---
## Signif. codes:  0 '***' 0.001 '**' 0.01 '*' 0.05 '.' 0.1 ' ' 1
## 
## Residual standard error: 9.094 on 11 degrees of freedom
\end{verbatim}
\begin{alltt}
\hlkwd{anova}\hlstd{(m_red)}
\end{alltt}
\begin{verbatim}
## Analysis of Variance Table
## 
## Response: thrust
##           Df Sum Sq Mean Sq F value Pr(>F)
## Residuals 11 909.62  82.692
\end{verbatim}
\begin{alltt}
\hlstd{SSRes_A} \hlkwb{<-} \hlkwd{anova}\hlstd{(m_red)}\hlopt{$}\hlstd{`Sum Sq`[}\hlnum{1}\hlstd{]}
\hlcom{# Manually calculate F-statistic.}
\hlstd{l} \hlkwb{<-} \hlnum{2}
\hlstd{n} \hlkwb{<-} \hlkwd{nrow}\hlstd{(rocket)}
\hlstd{p} \hlkwb{<-} \hlnum{2}
\hlstd{Fstat} \hlkwb{<-} \hlstd{((SSRes_A} \hlopt{-} \hlstd{SSRes)}\hlopt{/}\hlstd{l)}\hlopt{/}\hlstd{(SSRes}\hlopt{/}\hlstd{(n} \hlopt{-} \hlstd{p} \hlopt{-} \hlnum{1}\hlstd{))}
\hlstd{Fstat}
\end{alltt}
\begin{verbatim}
## [1] 60.04505
\end{verbatim}
\begin{alltt}
\hlstd{pval} \hlkwb{<-} \hlnum{1} \hlopt{-} \hlkwd{pf}\hlstd{(Fstat,} \hlkwc{df1} \hlstd{= l,} \hlkwc{df2} \hlstd{= n} \hlopt{-} \hlstd{p} \hlopt{-} \hlnum{1}\hlstd{)}
\hlstd{pval}
\end{alltt}
\begin{verbatim}
## [1] 6.238398e-06
\end{verbatim}
\begin{alltt}
\hlcom{# Automatically calculate F-statistic.}
\hlkwd{anova}\hlstd{(m1, m_red)}\hlopt{$}\hlstd{F[}\hlnum{2}\hlstd{]}
\end{alltt}
\begin{verbatim}
## [1] 60.04505
\end{verbatim}
\end{kframe}
\end{knitrout}

    \subsubsection{Revist Coffee Example (Coffee Quality Institute, 2018)}

\begin{knitrout}
\definecolor{shadecolor}{rgb}{0.969, 0.969, 0.969}\color{fgcolor}\begin{kframe}
\begin{alltt}
\hlstd{coffee} \hlkwb{<-} \hlkwd{read.csv}\hlstd{(}\hlstr{"csv/coffee_arabica.csv"}\hlstd{)}
\hlstd{mfull} \hlkwb{<-} \hlkwd{lm}\hlstd{(Flavor} \hlopt{~} \hlkwd{factor}\hlstd{(Processing.Method)} \hlopt{+} \hlstd{Aroma} \hlopt{+} \hlstd{Aftertaste} \hlopt{+}
  \hlstd{Body} \hlopt{+} \hlstd{Acidity} \hlopt{+} \hlstd{Balance} \hlopt{+} \hlstd{Sweetness} \hlopt{+} \hlstd{Uniformity} \hlopt{+} \hlstd{Moisture,}
  \hlkwc{dat} \hlstd{= coffee)}
\hlkwd{summary}\hlstd{(mfull)}
\end{alltt}
\begin{verbatim}
## 
## Call:
## lm(formula = Flavor ~ factor(Processing.Method) + Aroma + Aftertaste + 
##     Body + Acidity + Balance + Sweetness + Uniformity + Moisture, 
##     data = coffee)
## 
## Residuals:
##      Min       1Q   Median       3Q      Max 
## -0.68587 -0.08465  0.00079  0.08910  0.63633 
## 
## Coefficients:
##                                                     Estimate Std. Error t value
## (Intercept)                                        -0.728757   0.168516  -4.325
## factor(Processing.Method)Semi-washed / Semi-pulped -0.001396   0.022021  -0.063
## factor(Processing.Method)Washed / Wet              -0.033061   0.011024  -2.999
## Aroma                                               0.220302   0.020447  10.774
## Aftertaste                                          0.468759   0.023912  19.603
## Body                                                0.096140   0.024334   3.951
## Acidity                                             0.216751   0.021194  10.227
## Balance                                             0.046806   0.022558   2.075
## Sweetness                                           0.025507   0.010150   2.513
## Uniformity                                          0.016297   0.009803   1.663
## Moisture                                            0.169012   0.102480   1.649
##                                                    Pr(>|t|)    
## (Intercept)                                        1.67e-05 ***
## factor(Processing.Method)Semi-washed / Semi-pulped  0.94947    
## factor(Processing.Method)Washed / Wet               0.00277 ** 
## Aroma                                               < 2e-16 ***
## Aftertaste                                          < 2e-16 ***
## Body                                               8.28e-05 ***
## Acidity                                             < 2e-16 ***
## Balance                                             0.03823 *  
## Sweetness                                           0.01211 *  
## Uniformity                                          0.09669 .  
## Moisture                                            0.09938 .  
## ---
## Signif. codes:  0 '***' 0.001 '**' 0.01 '*' 0.05 '.' 0.1 ' ' 1
## 
## Residual standard error: 0.148 on 1108 degrees of freedom
## Multiple R-squared:  0.8091,	Adjusted R-squared:  0.8073 
## F-statistic: 469.5 on 10 and 1108 DF,  p-value: < 2.2e-16
\end{verbatim}
\begin{alltt}
\hlkwd{anova}\hlstd{(mfull)}
\end{alltt}
\begin{verbatim}
## Analysis of Variance Table
## 
## Response: Flavor
##                             Df Sum Sq Mean Sq   F value    Pr(>F)    
## factor(Processing.Method)    2  2.313   1.156   52.8096 < 2.2e-16 ***
## Aroma                        1 67.258  67.258 3071.2889 < 2.2e-16 ***
## Aftertaste                   1 29.097  29.097 1328.6722 < 2.2e-16 ***
## Body                         1  1.129   1.129   51.5460  1.28e-12 ***
## Acidity                      1  2.522   2.522  115.1618 < 2.2e-16 ***
## Balance                      1  0.116   0.116    5.2963 0.0215553 *  
## Sweetness                    1  0.251   0.251   11.4392 0.0007442 ***
## Uniformity                   1  0.064   0.064    2.9154 0.0880167 .  
## Moisture                     1  0.060   0.060    2.7200 0.0993839 .  
## Residuals                 1108 24.264   0.022                        
## ---
## Signif. codes:  0 '***' 0.001 '**' 0.01 '*' 0.05 '.' 0.1 ' ' 1
\end{verbatim}
\begin{alltt}
\hlstd{SSRes} \hlkwb{<-} \hlkwd{anova}\hlstd{(mfull)}\hlopt{$}\hlstd{`Sum Sq`[}\hlnum{10}\hlstd{]}
\hlcom{# Reduced model without Uniformity and Moisture}
\hlcom{# (beta9=beta10=0):}
\hlstd{m_red} \hlkwb{<-} \hlkwd{lm}\hlstd{(Flavor} \hlopt{~} \hlkwd{factor}\hlstd{(Processing.Method)} \hlopt{+} \hlstd{Aroma} \hlopt{+} \hlstd{Aftertaste} \hlopt{+}
  \hlstd{Body} \hlopt{+} \hlstd{Acidity} \hlopt{+} \hlstd{Balance} \hlopt{+} \hlstd{Sweetness,} \hlkwc{dat} \hlstd{= coffee)}
\hlkwd{summary}\hlstd{(m_red)}
\end{alltt}
\begin{verbatim}
## 
## Call:
## lm(formula = Flavor ~ factor(Processing.Method) + Aroma + Aftertaste + 
##     Body + Acidity + Balance + Sweetness, data = coffee)
## 
## Residuals:
##      Min       1Q   Median       3Q      Max 
## -0.67907 -0.08487  0.00054  0.08490  0.64763 
## 
## Coefficients:
##                                                     Estimate Std. Error t value
## (Intercept)                                        -0.606791   0.159741  -3.799
## factor(Processing.Method)Semi-washed / Semi-pulped  0.002275   0.021969   0.104
## factor(Processing.Method)Washed / Wet              -0.031115   0.011009  -2.826
## Aroma                                               0.221362   0.020472  10.813
## Aftertaste                                          0.470849   0.023858  19.735
## Body                                                0.087671   0.024102   3.637
## Acidity                                             0.219257   0.021182  10.351
## Balance                                             0.047526   0.022283   2.133
## Sweetness                                           0.032406   0.009597   3.377
##                                                    Pr(>|t|)    
## (Intercept)                                        0.000153 ***
## factor(Processing.Method)Semi-washed / Semi-pulped 0.917539    
## factor(Processing.Method)Washed / Wet              0.004795 ** 
## Aroma                                               < 2e-16 ***
## Aftertaste                                          < 2e-16 ***
## Body                                               0.000288 ***
## Acidity                                             < 2e-16 ***
## Balance                                            0.033160 *  
## Sweetness                                          0.000759 ***
## ---
## Signif. codes:  0 '***' 0.001 '**' 0.01 '*' 0.05 '.' 0.1 ' ' 1
## 
## Residual standard error: 0.1482 on 1110 degrees of freedom
## Multiple R-squared:  0.8081,	Adjusted R-squared:  0.8067 
## F-statistic: 584.2 on 8 and 1110 DF,  p-value: < 2.2e-16
\end{verbatim}
\begin{alltt}
\hlkwd{anova}\hlstd{(m_red)}
\end{alltt}
\begin{verbatim}
## Analysis of Variance Table
## 
## Response: Flavor
##                             Df Sum Sq Mean Sq  F value    Pr(>F)    
## factor(Processing.Method)    2  2.313   1.156   52.637 < 2.2e-16 ***
## Aroma                        1 67.258  67.258 3061.263 < 2.2e-16 ***
## Aftertaste                   1 29.097  29.097 1324.335 < 2.2e-16 ***
## Body                         1  1.129   1.129   51.378 1.387e-12 ***
## Acidity                      1  2.522   2.522  114.786 < 2.2e-16 ***
## Balance                      1  0.116   0.116    5.279 0.0217690 *  
## Sweetness                    1  0.251   0.251   11.402 0.0007591 ***
## Residuals                 1110 24.387   0.022                       
## ---
## Signif. codes:  0 '***' 0.001 '**' 0.01 '*' 0.05 '.' 0.1 ' ' 1
\end{verbatim}
\begin{alltt}
\hlstd{SSRes_A} \hlkwb{<-} \hlkwd{anova}\hlstd{(m_red)}\hlopt{$}\hlstd{`Sum Sq`[}\hlnum{8}\hlstd{]}
\hlcom{# Manually calculate F-statistic.}
\hlstd{l} \hlkwb{<-} \hlnum{2}
\hlstd{n} \hlkwb{<-} \hlkwd{nrow}\hlstd{(coffee)}
\hlstd{p} \hlkwb{<-} \hlnum{10}
\hlstd{Fstat} \hlkwb{<-} \hlstd{((SSRes_A} \hlopt{-} \hlstd{SSRes)}\hlopt{/}\hlstd{l)}\hlopt{/}\hlstd{(SSRes}\hlopt{/}\hlstd{(n} \hlopt{-} \hlstd{p} \hlopt{-} \hlnum{1}\hlstd{))}
\hlstd{Fstat}
\end{alltt}
\begin{verbatim}
## [1] 2.81769
\end{verbatim}
\begin{alltt}
\hlstd{pval} \hlkwb{<-} \hlnum{1} \hlopt{-} \hlkwd{pf}\hlstd{(Fstat,} \hlkwc{df1} \hlstd{= l,} \hlkwc{df2} \hlstd{= n} \hlopt{-} \hlstd{p} \hlopt{-} \hlnum{1}\hlstd{)}
\hlstd{pval}
\end{alltt}
\begin{verbatim}
## [1] 0.06017197
\end{verbatim}
\begin{alltt}
\hlcom{# Automatically calculate F-statistic.}
\hlkwd{anova}\hlstd{(mfull, m_red)}\hlopt{$}\hlstd{F[}\hlnum{2}\hlstd{]}
\end{alltt}
\begin{verbatim}
## [1] 2.81769
\end{verbatim}
\begin{alltt}
\hlcom{# Reduced model without Uniformity and Moisture and setting}
\hlcom{# effect of Dry = Semi (beta1=beta9=beta10=0) 1 = wet, 0}
\hlcom{# otherwise}
\hlstd{coffee}\hlopt{$}\hlstd{method2} \hlkwb{<-} \hlkwd{ifelse}\hlstd{(coffee}\hlopt{$}\hlstd{Processing.Method} \hlopt \hlkwd{c}\hlstd{(}\hlstr{"Natural / Dry"}\hlstd{,}
  \hlstr{"Semi-washed / Semi-pulped"}\hlstd{),} \hlnum{0}\hlstd{,} \hlnum{1}\hlstd{)}
\hlcom{# 1 = semi/dry, 0 o.w}
\hlstd{coffee}\hlopt{$}\hlstd{wet} \hlkwb{<-} \hlkwd{ifelse}\hlstd{(coffee}\hlopt{$}\hlstd{Processing.Method} \hlopt{==} \hlstr{"Washed / Wet"}\hlstd{,}
  \hlnum{0}\hlstd{,} \hlnum{1}\hlstd{)}
\hlstd{m_red2} \hlkwb{<-} \hlkwd{lm}\hlstd{(Flavor} \hlopt{~} \hlstd{method2} \hlopt{+} \hlstd{Aroma} \hlopt{+} \hlstd{Aftertaste} \hlopt{+} \hlstd{Body} \hlopt{+} \hlstd{Acidity} \hlopt{+}
  \hlstd{Balance} \hlopt{+} \hlstd{Sweetness,} \hlkwc{dat} \hlstd{= coffee)}
\hlkwd{summary}\hlstd{(m_red2)}
\end{alltt}
\begin{verbatim}
## 
## Call:
## lm(formula = Flavor ~ method2 + Aroma + Aftertaste + Body + Acidity + 
##     Balance + Sweetness, data = coffee)
## 
## Residuals:
##      Min       1Q   Median       3Q      Max 
## -0.67906 -0.08508  0.00052  0.08490  0.64722 
## 
## Coefficients:
##              Estimate Std. Error t value Pr(>|t|)    
## (Intercept) -0.606597   0.159659  -3.799 0.000153 ***
## method2     -0.031543   0.010200  -3.092 0.002036 ** 
## Aroma        0.221408   0.020458  10.823  < 2e-16 ***
## Aftertaste   0.470861   0.023847  19.745  < 2e-16 ***
## Body         0.087561   0.024068   3.638 0.000287 ***
## Acidity      0.219266   0.021173  10.356  < 2e-16 ***
## Balance      0.047527   0.022273   2.134 0.033077 *  
## Sweetness    0.032462   0.009577   3.389 0.000725 ***
## ---
## Signif. codes:  0 '***' 0.001 '**' 0.01 '*' 0.05 '.' 0.1 ' ' 1
## 
## Residual standard error: 0.1482 on 1111 degrees of freedom
## Multiple R-squared:  0.8081,	Adjusted R-squared:  0.8069 
## F-statistic: 668.3 on 7 and 1111 DF,  p-value: < 2.2e-16
\end{verbatim}
\begin{alltt}
\hlkwd{anova}\hlstd{(m_red2)}
\end{alltt}
\begin{verbatim}
## Analysis of Variance Table
## 
## Response: Flavor
##              Df Sum Sq Mean Sq   F value    Pr(>F)    
## method2       1  2.313   2.313  105.3648 < 2.2e-16 ***
## Aroma         1 67.255  67.255 3063.8526 < 2.2e-16 ***
## Aftertaste    1 29.100  29.100 1325.6571 < 2.2e-16 ***
## Body          1  1.126   1.126   51.3088 1.434e-12 ***
## Acidity       1  2.522   2.522  114.9115 < 2.2e-16 ***
## Balance       1  0.116   0.116    5.2882 0.0216552 *  
## Sweetness     1  0.252   0.252   11.4883 0.0007249 ***
## Residuals  1111 24.388   0.022                        
## ---
## Signif. codes:  0 '***' 0.001 '**' 0.01 '*' 0.05 '.' 0.1 ' ' 1
\end{verbatim}
\begin{alltt}
\hlstd{SSRes_A} \hlkwb{<-} \hlkwd{anova}\hlstd{(m_red2)}\hlopt{$}\hlstd{`Sum Sq`[}\hlnum{8}\hlstd{]}
\hlcom{## Manually calculate F-statistic.}
\hlstd{l} \hlkwb{<-} \hlnum{3}
\hlstd{n} \hlkwb{<-} \hlkwd{nrow}\hlstd{(coffee)}
\hlstd{p} \hlkwb{<-} \hlnum{10}
\hlstd{Fstat} \hlkwb{<-} \hlstd{((SSRes_A} \hlopt{-} \hlstd{SSRes)}\hlopt{/}\hlstd{l)}\hlopt{/}\hlstd{(SSRes}\hlopt{/}\hlstd{(n} \hlopt{-} \hlstd{p} \hlopt{-} \hlnum{1}\hlstd{))}
\hlstd{Fstat}
\end{alltt}
\begin{verbatim}
## [1] 1.882046
\end{verbatim}
\begin{alltt}
\hlstd{pval} \hlkwb{<-} \hlnum{1} \hlopt{-} \hlkwd{pf}\hlstd{(Fstat,} \hlkwc{df1} \hlstd{= l,} \hlkwc{df2} \hlstd{= n} \hlopt{-} \hlstd{p} \hlopt{-} \hlnum{1}\hlstd{)}
\hlstd{pval}
\end{alltt}
\begin{verbatim}
## [1] 0.1308207
\end{verbatim}
\begin{alltt}
\hlcom{# Automatically calculate F-statistic.}
\hlkwd{anova}\hlstd{(mfull, m_red2)}\hlopt{$}\hlstd{F[}\hlnum{2}\hlstd{]}
\end{alltt}
\begin{verbatim}
## [1] 1.882046
\end{verbatim}
\end{kframe}
\end{knitrout}
\section{2019-10-24}
\begin{thmbox}
    \subsection{Theorem (Fundamental Theorem of LP)}
    Let (P) be an LP problem. Then exactly one of the following holds:
    \begin{itemize}
        \item (P) is infeasible
        \item (P) is unbounded
        \item (P) has an optimal solution
    \end{itemize}
\end{thmbox}

\begin{defbox}
    \subsection{Definition (Hyperplane, Half-space)}
    Let $ \bm{a} \in \mathbb{R}^n\setminus{\bm{0}} $, $ \beta \in\mathbb{R} $.

    $ H:=\{\bm{x}\in\mathbb{R}^n:
    \bm{a} ^\top \bm{x}=\beta \} $ is a \emph{hyperplane}.

    $ F:=\{\bm{x}\in\mathbb{R}^n:
    \bm{a} ^\top \bm{x}\le \beta\} $ is a \emph{half-space}.
\end{defbox}

Solution sets of linear equations are intersections of hyperplanes.

\begin{defbox}
    \subsection{Definition (Polyhedron)}
    Let $ A\in \mathbb{R}^{m \times n} $, $ \bm{b}\in \mathbb{R}^m $.
    $ P:=\{\bm{x}\in\mathbb{R}^n:A\bm{x}\le \bm{b}\} $ is a \emph{polyhedron}.
\end{defbox}
\begin{remark}
    The set of solutions to any one of the inequalities of 
    $ A\bm{x}\le \bm{b} $ is a half-space.
\end{remark}

\begin{thmbox}
    \subsection{Proposition}
    The feasible region of an LP is a polyhedron or equivalently the
    intersection of a finite number of half-spaces.
\end{thmbox}
\begin{proof}
    Let $ \bm{a}\in\mathbb{R}^n,\bm{x}\in\mathbb{R}^n,\beta\in \mathbb{R} $.

    Given an inequality of the form $ \bm{a} ^\top \bm{x}\ge \beta $, we can
    rewrite it as $ -\bm{a} ^\top \bm{x}\le -\beta $.

    Given an equation of the form $ \bm{a} ^\top \bm{x}=\beta $ we can rewrite it as
    $ \bm{a} ^\top \bm{x}\ge \beta $ and $ -\bm{a} ^\top \bm{x}\le -\beta $.

    Thus, any set of linear constraints can be rewritten as 
    $ A\bm{x}\le \bm{b} $ for some $ A\in \mathbb{R}^{m \times n} $ and $ \bm{b}\in \mathbb{R}^m $, where $ \bm{a}^\top $ can correspond to each row of $ A $,
    and $ \beta $ can correspond to each row of the column vector $ \bm{b} $.
\end{proof}

Solutions sets of $ A\bm{x}=\bm{b} $ are either $ \emptyset $, a single
point, a line, or in general, an intersection of a hyperplane.

Note that already in $ \mathbb{R}^2 $ there are already equivalent polyhedra.
The mathematical modelling power of LPs are significantly more than that of
linear systems of equations.

\begin{defbox}
    \subsection{Definition (Line segment)}
    The \emph{line segment} joining two points, $ u $ and $ v $ is
    \[ \{\lambda u + (1-\lambda)v:\lambda\in[0,1]\} \]
\end{defbox}
Graphically, the line segment can be seen as:
\begin{center}
    \includegraphics{linesegment}
\end{center}

\begin{defbox}
    \subsection{Definition (Convex)}
    A set $ S\subseteq \mathbb{R}^n $ is \emph{convex} if for 
    every pair of points $ u,v\in S $, the line segment joining $ u $ and $ v $
    is contained in $ S $.
\end{defbox}
\begin{center}
    \includegraphics{convex}
\end{center}


\begin{thmbox}
    \subsection{Proposition}
    Half-spaces are convex.
\end{thmbox}
\begin{proof}
    Let $ H\subseteq\mathbb{R}^n $ be a half-space. Then $ \bm{a} \in \mathbb{R}
    ^n\setminus{\bm{0}} $ and $ \beta\in\mathbb{R}^n $ such that
    \[ \{\bm{x}\in\mathbb{R}^n:\bm{a} ^\top \bm{x}=\beta\} \]
    Let $ u,v\in H $ and let $ \lambda\in[0,1] $ arbitrary.
    \[ \bm{a} ^\top[\lambda u+(1-\lambda)v]=
    \underbrace{\lambda}_{\ge 0}\underbrace{\bm{a} ^\top u}_{\le \beta}+
    \underbrace{(1-\lambda)}_{\ge 0}\underbrace{\bm{a} ^\top v}_{\le \beta}
    \le \lambda \beta + (1-\lambda)\beta=\beta \]
    Thus, $ H $ is convex.
\end{proof}

\begin{thmbox}
    \subsection{Proposition}
    The intersection of any collection of convex sets is convex. 
    That is, a convex set $ C_j $ $ \forall j\in J $, the intersection
    \[ C:=\bigcap_{j\in J} C_j \]
    is convex.
\end{thmbox}

\begin{proof}
    Let $ u,v $ be two points in $ C $. Let $ w $ lie on the line
    segment between $ u $ and $ v $. Then, $ w\in C_j $ since $ C_j $ is convex
    for each $ j\in J $. Thus, $ w\in C $.
\end{proof}

\begin{remark}
    $ J $ can be infinite. That is, the intersection of infinitely many convex sets
    is convex.
\end{remark}

\begin{thmbox}
    \subsection{Proposition}
    Polyhedra are convex.
\end{thmbox}

\begin{defbox}
    \subsection{Definition (Properly Contained)}
    We say that a point $x$ is \emph{properly contained} in a line segment if it is in the line segment and not an endpoint.
\end{defbox}

\begin{defbox}
    \subsection{Definition (Extreme Point)}
    Let $ S\subseteq \mathbb{R}^n $ be a convex set, $ u\in S $. $ u $ is an
    \emph{extreme point} of $ S $ if $ u $ cannot be properly contained in a
    line segment in S. Equivalently, $ u\in S $ and $ \nexists v,w\in S $,
    $ v\neq w $ and $ \lambda\in (0,1) $ such that
    \[ u=\lambda v+(1-\lambda)w \]
\end{defbox}

\begin{center}
    \includegraphics{extremepoint}
\end{center}

\subsection{R Demo}
\begin{knitrout}
\definecolor{shadecolor}{rgb}{0.969, 0.969, 0.969}\color{fgcolor}\begin{kframe}
\begin{alltt}
\hlcom{## Coffee example (Coffee Quality Institute, 2018)}
\hlcom{## continued}
\hlstd{coffee} \hlkwb{<-} \hlkwd{read.csv}\hlstd{(}\hlstr{"csv/coffee_arabica.csv"}\hlstd{)}
\hlcom{# cor(coffee) doesn't work as there's a categorical}
\hlcom{# variable.}
\hlkwd{cor}\hlstd{(coffee[,} \hlopt{-}\hlnum{1}\hlstd{])}  \hlcom{# e.g., remove first column}
\end{alltt}
\begin{verbatim}
##                  Aroma     Flavor Aftertaste        Body     Acidity    Balance
## Aroma       1.00000000  0.7339782  0.6892744  0.56699932  0.60115765  0.6156508
## Flavor      0.73397820  1.0000000  0.8582783  0.67694834  0.73845546  0.7324530
## Aftertaste  0.68927440  0.8582783  1.0000000  0.67407704  0.69408861  0.7657979
## Body        0.56699932  0.6769483  0.6740770  1.00000000  0.60795391  0.6924568
## Acidity     0.60115765  0.7384555  0.6940886  0.60795391  1.00000000  0.6417994
## Balance     0.61565084  0.7324530  0.7657979  0.69245676  0.64179938  1.0000000
## Sweetness   0.06955938  0.1345364  0.1185760  0.03977892  0.06906093  0.1016718
## Uniformity  0.14785498  0.2132347  0.2143116  0.07195778  0.14876428  0.2180726
## Moisture   -0.11567549 -0.1327342 -0.1745366 -0.21009097 -0.10391684 -0.2161964
##             Sweetness Uniformity    Moisture
## Aroma      0.06955938 0.14785498 -0.11567549
## Flavor     0.13453644 0.21323472 -0.13273418
## Aftertaste 0.11857600 0.21431157 -0.17453658
## Body       0.03977892 0.07195778 -0.21009097
## Acidity    0.06906093 0.14876428 -0.10391684
## Balance    0.10167183 0.21807265 -0.21619640
## Sweetness  1.00000000 0.34756414  0.08049300
## Uniformity 0.34756414 1.00000000  0.02105693
## Moisture   0.08049300 0.02105693  1.00000000
\end{verbatim}
\end{kframe}
\end{knitrout}

Plot the pairs (disabled due to loading time on PDF).

\begin{knitrout}
\definecolor{shadecolor}{rgb}{0.969, 0.969, 0.969}\color{fgcolor}\begin{kframe}
\begin{alltt}
\hlkwd{pairs}\hlstd{(}\hlopt{~}\hlstd{Flavor} \hlopt{+} \hlstd{Aroma} \hlopt{+} \hlstd{Aftertaste} \hlopt{+} \hlstd{Body} \hlopt{+} \hlstd{Acidity} \hlopt{+} \hlstd{Balance} \hlopt{+}
  \hlstd{Sweetness} \hlopt{+} \hlstd{Uniformity} \hlopt{+} \hlstd{Moisture,} \hlkwc{data} \hlstd{= coffee)}
\end{alltt}
\end{kframe}
\end{knitrout}

\begin{knitrout}
\definecolor{shadecolor}{rgb}{0.969, 0.969, 0.969}\color{fgcolor}\begin{kframe}
\begin{alltt}
\hlcom{# Code our own indicators, so that we can more easily}
\hlcom{# interpret VIFs.  1 = wet, 0 otherwise}
\hlstd{coffee}\hlopt{$}\hlstd{wet} \hlkwb{<-} \hlkwd{ifelse}\hlstd{(coffee}\hlopt{$}\hlstd{Processing.Method} \hlopt{==} \hlstr{"Washed / Wet"}\hlstd{,}
  \hlnum{1}\hlstd{,} \hlnum{0}\hlstd{)}
\hlcom{# 1 = semi/dry, 0 otherwise}
\hlstd{coffee}\hlopt{$}\hlstd{semi} \hlkwb{<-} \hlkwd{ifelse}\hlstd{(coffee}\hlopt{$}\hlstd{Processing.Method} \hlopt{==} \hlstr{"Semi-washed / Semi-pulped"}\hlstd{,}
  \hlnum{1}\hlstd{,} \hlnum{0}\hlstd{)}
\end{alltt}
\end{kframe}
\end{knitrout}

Model:
$$y_i=\beta_0+\beta_1x_{i1}+\beta_2x_{i2}+\beta_3x_{i3}+
    \beta_4x_{i4}+\beta_5x_{i5}+\beta_6x_{i6}+\beta_7x_{i7}+\beta_8x_{i8}+
    \beta_9x_{i9}+\beta_{10}x_{i(10)}+\varepsilon_i$$
where
\begin{itemize}
    \item $y=$ flavour
    \item $x_1=1$ if wet, 0 otherwise
    \item $x_2=1$ if semi, 0 otherwise
    \item $x_3=$ Aroma
    \item $x_4=$ Aftertaste
    \item $x_5=$ Body
    \item $x_6=$ Acidity
    \item $x_7=$ Balance
    \item $x_8=$ Sweetness
    \item $x_9=$ Uniformity
    \item $x_{10}=$ Moisture
\end{itemize}

\begin{knitrout}
\definecolor{shadecolor}{rgb}{0.969, 0.969, 0.969}\color{fgcolor}\begin{kframe}
\begin{alltt}
\hlcom{# Full MLR with our manually coded indicators.}
\hlstd{mfull} \hlkwb{<-} \hlkwd{lm}\hlstd{(Flavor} \hlopt{~} \hlstd{wet} \hlopt{+} \hlstd{semi} \hlopt{+} \hlstd{Aroma} \hlopt{+} \hlstd{Aftertaste} \hlopt{+} \hlstd{Body} \hlopt{+}
  \hlstd{Acidity} \hlopt{+} \hlstd{Balance} \hlopt{+} \hlstd{Sweetness} \hlopt{+} \hlstd{Uniformity} \hlopt{+} \hlstd{Moisture,} \hlkwc{dat} \hlstd{= coffee)}
\hlkwd{summary}\hlstd{(mfull)}
\end{alltt}
\begin{verbatim}
## 
## Call:
## lm(formula = Flavor ~ wet + semi + Aroma + Aftertaste + Body + 
##     Acidity + Balance + Sweetness + Uniformity + Moisture, data = coffee)
## 
## Residuals:
##      Min       1Q   Median       3Q      Max 
## -0.68587 -0.08465  0.00079  0.08910  0.63633 
## 
## Coefficients:
##              Estimate Std. Error t value Pr(>|t|)    
## (Intercept) -0.728757   0.168516  -4.325 1.67e-05 ***
## wet         -0.033061   0.011024  -2.999  0.00277 ** 
## semi        -0.001396   0.022021  -0.063  0.94947    
## Aroma        0.220302   0.020447  10.774  < 2e-16 ***
## Aftertaste   0.468759   0.023912  19.603  < 2e-16 ***
## Body         0.096140   0.024334   3.951 8.28e-05 ***
## Acidity      0.216751   0.021194  10.227  < 2e-16 ***
## Balance      0.046806   0.022558   2.075  0.03823 *  
## Sweetness    0.025507   0.010150   2.513  0.01211 *  
## Uniformity   0.016297   0.009803   1.663  0.09669 .  
## Moisture     0.169012   0.102480   1.649  0.09938 .  
## ---
## Signif. codes:  0 '***' 0.001 '**' 0.01 '*' 0.05 '.' 0.1 ' ' 1
## 
## Residual standard error: 0.148 on 1108 degrees of freedom
## Multiple R-squared:  0.8091,	Adjusted R-squared:  0.8073 
## F-statistic: 469.5 on 10 and 1108 DF,  p-value: < 2.2e-16
\end{verbatim}
\begin{alltt}
\hlcom{# Full MLR alternative, using the factor command.}
\hlstd{mfull_alternative} \hlkwb{<-} \hlkwd{lm}\hlstd{(Flavor} \hlopt{~} \hlkwd{factor}\hlstd{(Processing.Method)} \hlopt{+}
  \hlstd{Aroma} \hlopt{+} \hlstd{Aftertaste} \hlopt{+} \hlstd{Body} \hlopt{+} \hlstd{Acidity} \hlopt{+} \hlstd{Balance} \hlopt{+} \hlstd{Sweetness} \hlopt{+}
  \hlstd{Uniformity} \hlopt{+} \hlstd{Moisture,} \hlkwc{dat} \hlstd{= coffee)}
\end{alltt}
\end{kframe}
\end{knitrout}

Suppose we want to check the VIF for $j=1$; that is,
$x_1$. Now, we fit:
$$x_{i1}=\alpha_0+\alpha_2x_{i2}+\alpha_3x_{i3}+
    \alpha_4x_{i4}+\alpha_5x_{i5}+\alpha_6x_{i6}+\alpha_7x_{i7}+\alpha_8x_{i8}+
    \alpha_9x_{i9}+\alpha_{10}x_{i(10)}+\varepsilon_i$$

\begin{knitrout}
\definecolor{shadecolor}{rgb}{0.969, 0.969, 0.969}\color{fgcolor}\begin{kframe}
\begin{alltt}
\hlstd{wet_reg} \hlkwb{<-} \hlkwd{lm}\hlstd{(wet} \hlopt{~} \hlstd{semi} \hlopt{+} \hlstd{Aroma} \hlopt{+} \hlstd{Aftertaste} \hlopt{+} \hlstd{Body} \hlopt{+} \hlstd{Acidity} \hlopt{+}
  \hlstd{Balance} \hlopt{+} \hlstd{Sweetness} \hlopt{+} \hlstd{Uniformity} \hlopt{+} \hlstd{Moisture,} \hlkwc{dat} \hlstd{= coffee)}
\hlkwd{summary}\hlstd{(wet_reg)}
\end{alltt}
\begin{verbatim}
## 
## Call:
## lm(formula = wet ~ semi + Aroma + Aftertaste + Body + Acidity + 
##     Balance + Sweetness + Uniformity + Moisture, data = coffee)
## 
## Residuals:
##     Min      1Q  Median      3Q     Max 
## -1.0015 -0.0283  0.1770  0.2522  0.7704 
## 
## Coefficients:
##             Estimate Std. Error t value Pr(>|t|)    
## (Intercept)  0.81748    0.45838   1.783 0.074794 .  
## semi        -0.75675    0.05551 -13.632  < 2e-16 ***
## Aroma        0.09690    0.05562   1.742 0.081774 .  
## Aftertaste  -0.13169    0.06502  -2.026 0.043054 *  
## Body        -0.21885    0.06596  -3.318 0.000936 ***
## Acidity      0.18696    0.05746   3.254 0.001173 ** 
## Balance     -0.10804    0.06136  -1.761 0.078563 .  
## Sweetness    0.08373    0.02753   3.041 0.002413 ** 
## Uniformity   0.03547    0.02668   1.329 0.184053    
## Moisture     0.59486    0.27858   2.135 0.032956 *  
## ---
## Signif. codes:  0 '***' 0.001 '**' 0.01 '*' 0.05 '.' 0.1 ' ' 1
## 
## Residual standard error: 0.4031 on 1109 degrees of freedom
## Multiple R-squared:  0.1911,	Adjusted R-squared:  0.1845 
## F-statistic: 29.11 on 9 and 1109 DF,  p-value: < 2.2e-16
\end{verbatim}
\begin{alltt}
\hlstd{r2_wet} \hlkwb{<-} \hlkwd{summary}\hlstd{(wet_reg)}\hlopt{$}\hlstd{r.squared}
\hlstd{r2_wet}
\end{alltt}
\begin{verbatim}
## [1] 0.191077
\end{verbatim}
\end{kframe}
\end{knitrout}

$R_j$: In our case, $R_1=0.191077$.

\begin{knitrout}
\definecolor{shadecolor}{rgb}{0.969, 0.969, 0.969}\color{fgcolor}\begin{kframe}
\begin{alltt}
\hlstd{VIF_wet} \hlkwb{<-} \hlnum{1}\hlopt{/}\hlstd{(}\hlnum{1} \hlopt{-} \hlstd{r2_wet)}
\hlstd{VIF_wet}
\end{alltt}
\begin{verbatim}
## [1] 1.236212
\end{verbatim}
\end{kframe}
\end{knitrout}

$\text{VIF}_{j}$: $\text{VIF}_1=1.236212$.
    Interpretation: in a regression with all
    the variables compared to a regression with just this one,
    the estimated variance has increased by a factor of 1.24,
    which is not a very large inflation. The variable wet is not
    very linearly correlated or dependent on the other predictors
    that we have in the model.

\begin{knitrout}
\definecolor{shadecolor}{rgb}{0.969, 0.969, 0.969}\color{fgcolor}\begin{kframe}
\begin{alltt}
\hlstd{Aroma_reg} \hlkwb{<-} \hlkwd{lm}\hlstd{(Aroma} \hlopt{~} \hlstd{wet} \hlopt{+} \hlstd{semi} \hlopt{+} \hlstd{Aftertaste} \hlopt{+} \hlstd{Body} \hlopt{+} \hlstd{Acidity} \hlopt{+}
  \hlstd{Balance} \hlopt{+} \hlstd{Sweetness} \hlopt{+} \hlstd{Uniformity} \hlopt{+} \hlstd{Moisture,} \hlkwc{dat} \hlstd{= coffee)}
\hlstd{r2_Aroma} \hlkwb{<-} \hlkwd{summary}\hlstd{(Aroma_reg)}\hlopt{$}\hlstd{r.squared}
\hlstd{r2_Aroma}
\end{alltt}
\begin{verbatim}
## [1] 0.5204716
\end{verbatim}
\begin{alltt}
\hlstd{VIF_Aroma} \hlkwb{<-} \hlnum{1}\hlopt{/}\hlstd{(}\hlnum{1} \hlopt{-} \hlstd{r2_Aroma)}
\hlstd{VIF_Aroma}
\end{alltt}
\begin{verbatim}
## [1] 2.085382
\end{verbatim}
\end{kframe}
\end{knitrout}

$R_3=0.5204716$, $\text{VIF}_3=2.085382$.

\begin{knitrout}
\definecolor{shadecolor}{rgb}{0.969, 0.969, 0.969}\color{fgcolor}\begin{kframe}
\begin{alltt}
\hlstd{Aftertaste_reg} \hlkwb{<-} \hlkwd{lm}\hlstd{(Aftertaste} \hlopt{~} \hlstd{wet} \hlopt{+} \hlstd{semi} \hlopt{+} \hlstd{Aroma} \hlopt{+} \hlstd{Body} \hlopt{+}
  \hlstd{Acidity} \hlopt{+} \hlstd{Balance} \hlopt{+} \hlstd{Sweetness} \hlopt{+} \hlstd{Uniformity} \hlopt{+} \hlstd{Moisture,} \hlkwc{dat} \hlstd{= coffee)}
\hlstd{r2_Aftertaste} \hlkwb{<-} \hlkwd{summary}\hlstd{(Aftertaste_reg)}\hlopt{$}\hlstd{r.squared}
\hlstd{r2_Aftertaste}
\end{alltt}
\begin{verbatim}
## [1] 0.7101012
\end{verbatim}
\begin{alltt}
\hlstd{VIF_Aftertaste} \hlkwb{<-} \hlnum{1}\hlopt{/}\hlstd{(}\hlnum{1} \hlopt{-} \hlstd{r2_Aftertaste)}
\hlstd{VIF_Aftertaste}
\end{alltt}
\begin{verbatim}
## [1] 3.449479
\end{verbatim}
\end{kframe}
\end{knitrout}

\begin{knitrout}
\definecolor{shadecolor}{rgb}{0.969, 0.969, 0.969}\color{fgcolor}\begin{kframe}
\begin{alltt}
\hlcom{# Load car library for automatic VIF calculation using}
\hlcom{# vif()}
\hlkwd{library}\hlstd{(car)}
\hlkwd{vif}\hlstd{(mfull)}
\end{alltt}
\begin{verbatim}
##        wet       semi      Aroma Aftertaste       Body    Acidity    Balance 
##   1.236212   1.178004   2.085382   3.449479   2.317728   2.232210   3.002813 
##  Sweetness Uniformity   Moisture 
##   1.159602   1.209901   1.086101
\end{verbatim}
\end{kframe}
\end{knitrout}

    No serious signs of inflation, all VIFs are less than 10.

\begin{knitrout}
\definecolor{shadecolor}{rgb}{0.969, 0.969, 0.969}\color{fgcolor}\begin{kframe}
\begin{alltt}
\hlcom{## Python in FL everglades example (2017) Sex, length,}
\hlcom{## total mass, fat mass, and specimen condition data for}
\hlcom{## 248 Burmese pythons (Python bivittatus) collected in the}
\hlcom{## Florida Everglades}
\hlstd{python} \hlkwb{<-} \hlkwd{read.csv}\hlstd{(}\hlstr{"csv/FLpython.csv"}\hlstd{)}
\hlkwd{head}\hlstd{(python)}
\end{alltt}
\begin{verbatim}
##   sex  svl mass length    fat
## 1   F 70.0  186   77.5  6.000
## 2   M 76.0  310   83.8 11.000
## 3   M 77.0  260   86.1  6.000
## 4   M 78.0  262   87.1  8.000
## 5   M 81.0  306   91.1  4.000
## 6   M 93.5  605  104.6 18.959
\end{verbatim}
\begin{alltt}
\hlstd{python}\hlopt{$}\hlstd{male} \hlkwb{<-} \hlkwd{ifelse}\hlstd{(python}\hlopt{$}\hlstd{sex} \hlopt{==} \hlstr{"M"}\hlstd{,} \hlnum{1}\hlstd{,} \hlnum{0}\hlstd{)}  \hlcom{# 1 = M, 0 =F}
\hlkwd{cor}\hlstd{(python[,} \hlopt{-}\hlnum{1}\hlstd{])}
\end{alltt}
\begin{verbatim}
##               svl       mass     length        fat       male
## svl     1.0000000  0.8843022  0.9994935  0.8098652 -0.1602418
## mass    0.8843022  1.0000000  0.8858256  0.9419114 -0.2190993
## length  0.9994935  0.8858256  1.0000000  0.8114658 -0.1593512
## fat     0.8098652  0.9419114  0.8114658  1.0000000 -0.2933111
## male   -0.1602418 -0.2190993 -0.1593512 -0.2933111  1.0000000
\end{verbatim}
\end{kframe}
\end{knitrout}

\begin{knitrout}
\definecolor{shadecolor}{rgb}{0.969, 0.969, 0.969}\color{fgcolor}\begin{kframe}
\begin{alltt}
\hlkwd{pairs}\hlstd{(python[,} \hlopt{-}\hlnum{1}\hlstd{])}
\end{alltt}
\end{kframe}
\end{knitrout}

\begin{knitrout}
\definecolor{shadecolor}{rgb}{0.969, 0.969, 0.969}\color{fgcolor}\begin{kframe}
\begin{alltt}
\hlstd{mpf} \hlkwb{<-} \hlkwd{lm}\hlstd{(fat} \hlopt{~} \hlstd{male} \hlopt{+} \hlstd{svl} \hlopt{+} \hlstd{mass} \hlopt{+} \hlstd{length,} \hlkwc{data} \hlstd{= python)}
\hlkwd{summary}\hlstd{(mpf)}
\end{alltt}
\begin{verbatim}
## 
## Call:
## lm(formula = fat ~ male + svl + mass + length, data = python)
## 
## Residuals:
##      Min       1Q   Median       3Q      Max 
## -2445.77  -137.41    -5.29   110.00  1527.27 
## 
## Coefficients:
##               Estimate Std. Error t value Pr(>|t|)    
## (Intercept)  2.021e+02  1.331e+02   1.518    0.130    
## male        -1.971e+02  4.732e+01  -4.165 4.32e-05 ***
## svl         -3.370e+00  1.125e+01  -0.300    0.765    
## mass         1.178e-01  5.302e-03  22.210  < 2e-16 ***
## length       1.594e+00  1.010e+01   0.158    0.875    
## ---
## Signif. codes:  0 '***' 0.001 '**' 0.01 '*' 0.05 '.' 0.1 ' ' 1
## 
## Residual standard error: 360.9 on 243 degrees of freedom
## Multiple R-squared:  0.897,	Adjusted R-squared:  0.8953 
## F-statistic:   529 on 4 and 243 DF,  p-value: < 2.2e-16
\end{verbatim}
\begin{alltt}
\hlkwd{vif}\hlstd{(mpf)}
\end{alltt}
\begin{verbatim}
##        male         svl        mass      length 
##    1.058699  994.546545    4.813078 1007.484200
\end{verbatim}
\begin{alltt}
\hlstd{mpf_l} \hlkwb{<-} \hlkwd{lm}\hlstd{(length} \hlopt{~} \hlstd{male} \hlopt{+} \hlstd{svl} \hlopt{+} \hlstd{mass,} \hlkwc{data} \hlstd{= python)}
\hlnum{1}\hlopt{/}\hlstd{(}\hlnum{1} \hlopt{-} \hlkwd{summary}\hlstd{(mpf_l)}\hlopt{$}\hlstd{r.squared)}
\end{alltt}
\begin{verbatim}
## [1] 1007.484
\end{verbatim}
\end{kframe}
\end{knitrout}

Misleading conclusion: \code{svl} and \code{length} are both irrelevant
(this is not the case). Also, the standard errors are very large.

\begin{knitrout}
\definecolor{shadecolor}{rgb}{0.969, 0.969, 0.969}\color{fgcolor}\begin{kframe}
\begin{alltt}
\hlcom{# remove 'length' based on VIF}
\hlstd{mpf2} \hlkwb{<-} \hlkwd{lm}\hlstd{(fat} \hlopt{~} \hlstd{male} \hlopt{+} \hlstd{mass} \hlopt{+} \hlstd{svl,} \hlkwc{data} \hlstd{= python)}
\hlkwd{summary}\hlstd{(mpf2)}\hlopt{$}\hlstd{adj}
\end{alltt}
\begin{verbatim}
## [1] 0.8957164
\end{verbatim}
\begin{alltt}
\hlkwd{vif}\hlstd{(mpf2)}
\end{alltt}
\begin{verbatim}
##     male     mass      svl 
## 1.056139 4.720065 4.611903
\end{verbatim}
\begin{alltt}
\hlkwd{anova}\hlstd{(mpf2)}
\end{alltt}
\begin{verbatim}
## Analysis of Variance Table
## 
## Response: fat
##            Df    Sum Sq   Mean Sq   F value  Pr(>F)    
## male        1  26435988  26435988  203.7689 < 2e-16 ***
## mass        1 248624259 248624259 1916.3988 < 2e-16 ***
## svl         1    567377    567377    4.3734 0.03754 *  
## Residuals 244  31655372    129735                      
## ---
## Signif. codes:  0 '***' 0.001 '**' 0.01 '*' 0.05 '.' 0.1 ' ' 1
\end{verbatim}
\begin{alltt}
\hlkwd{AIC}\hlstd{(mpf2)}
\end{alltt}
\begin{verbatim}
## [1] 3629.527
\end{verbatim}
\end{kframe}
\end{knitrout}
    \code{svl} now has a significant $t$-statistic.
\section{2020-02-25}
\subsection{Copy/Move Elision}
C++ allows compilers (not requires) to avoid copy or move constructors
even if this changes program behaviour.

\subsection{Rule of 5}
If you implement one of the Big 5, typically you need to implement
all 5.

Implementing operators as methods or functions?

\code{operator=} must be implemented as a method; recall \code{n1=n2}
is actually \code{n1.operator=(n2)}.

\code{.h}
\begin{lstlisting}
    struct Vec {
        int x;
        int y;
        Vec operator+(const Vec &v2) {
            return {x + v2.x, y + v2.y};
        }
        Vec operator*(int k); // v1 * 5
        // can't implement 5 * v1 since this will refer to 5 which is not an object
    };
    Vec operator*(int k, const Vec &v1) {
        return v1 * k;
    }
\end{lstlisting}
\code{cout <{}< v1;} $ \rightarrow $ cannot be implemented within the method as with
\code{cin >{}> v2;}
\code{.h}
\begin{lstlisting}
    ostream &operator<<(const Vec &v1);
\end{lstlisting}
\code{.c}
\begin{lstlisting}
    ostream &Vec::operator<<(ostream &out) {
        out << x << " " << y;
        return out;
    }
    v1 << cout; // you can do this, but don't since:
    v2 << (v1 << cout); // needs brackets, still don't do this.
\end{lstlisting}

Following operators \textbf{must} be implemented as methods.
\begin{itemize}
      \item \code{operator=}
      \item \code{operator->}
      \item \code{operator[]}; like q3
      \item \code{operator()} $ \rightarrow $ allows you treat an object as a function
      \item \code{operator T()} $ \rightarrow $ allows you to implicitly convert
            an object as another type given as \code{T}; \code{istream} to \code{bool}s
            is an example of this
\end{itemize}

\subsection{Arrays of Objects}
\code{.h}
\begin{lstlisting}
    struct Vec {
        int x, y;
        Vec(int x, int y);
    };
    Vec arr[3]; // stack array of 3 Vec objects; does not compile
    Vec *parr = new Vec[3]; // heap array of 3 Vec objects; does not compile
\end{lstlisting}
Won't compile, no default constructor. Options:
\begin{itemize}
      \item implement default constructor
      \item for stack arrays, use array initialization syntax; e.g.
            \code{Vec arr[3] = \{Vec\{0, 0\}, Vec\{1, 2\}, Vec\{3, 4\}\};}
      \item create an array of pointers to objects
\end{itemize}
\begin{lstlisting}
    // Keep all on the stack or all on the heap, or else deallocation will be confusing
    Vec *arr[3]; // stack array of 3 pointers, each element is a pointer to the Vec objects
    Vec **p = new Vec*[3]; // each element is a Vec*
    p[0] = new Vec{0, 0};
    p[1] = new Vec{1, 2};
    p[2] = new Vec{3, 4};
    // Deallocate the array p:
    for (...) {
        delete p[i];
    }
    delete[] p;
\end{lstlisting}

\subsection{Constant Methods}
\code{.h, .cc}
\begin{lstlisting}
    struct Student {
        int assign, mt, final;
        float grade() {
            return assign * 0.4 + mt * 0.2 + final * 0.4;
        }
    };
    const Student billy{70, 50, 75};
    cout << billy.grade(); // does not compile
    // we add `const' after the signature of the method, so we modify as follows:
    float grade() const {...}
\end{lstlisting}

\textbf{Bad Style the Language Supports}
\begin{lstlisting}
    struct Student {
        int assigns, mt, final;
        int count = 0;
        // does not compile since count field is being modified in a constant method
        float grade() const {
            ++count;
            return ...;
        }
    };
    // adding `mutable' before int count = 0; will allow this code to compile now
\end{lstlisting}

\subsection{Invariants and Encapsulation}
\code{.h}
\begin{lstlisting}
    struct Node {
        int data;
        Node *next;
        Node(int, Node *);
        ~Node() {
            delete next;
        }
    };
\end{lstlisting}
\code{.cc}
\begin{lstlisting}
    Node n1{1, New Node{2, nullptr}};
    // deallocating stack memory (segmentation fault if deleting n2)
    Node n2{10, &n1};
\end{lstlisting}

An invariant is an assumption that needs to stay true for the class to function
correctly. For example, the invariant for the Node class is that \code{next}
is either \code{nullptr} or points to the heap.

Use encapsulation:
\begin{itemize}
      \item informally treating an object as a black box
      \item using an exposed interface to interact with the object
\end{itemize}

\textbf{Encapsulating the Vector Class}

\code{.h}
\begin{lstlisting}
    struct Vec {
        // default visibility is `public'
        private: 
            int x, y; // hidden from outside the class
        public:
            Vec(int, int);
            Vec operator+(const Vec &other) {
                return {x + other.x, y + other.y};
            }
    };
\end{lstlisting}
\code{.cc}
\begin{lstlisting}
    int main() {
        Vec v{1, 2};
        Vec v1 = v + v;
        cout << v1.x << v1.y; // does not compile, accessing private fields: x, y
    }
\end{lstlisting}

Advice: at a minimum keep all fields \code{private}, make certain methods \code{public}
(makes the interface).

\begin{lstlisting}
    class Vec {
        // default visibility is `private'
        int x, y;
        public:
            Vec(int, int);
            Vec operator+(...);
    };
\end{lstlisting}

\makeheading{Lecture 14}
\textbf{Example}

Suppose you send a bit string over a noisy connection with
each bit independently having a probability $ 0.01 $ of being
flipped. What is the probability that
\begin{enumerate}[label=(\alph*)]
    \item it takes 50 bits to get 5 errors?
    \item a 50 bit message has 5 errors?
\end{enumerate}

\textbf{Solution.}

(b) Let $ Y= $ \# of errors in 50 bits.
\[ Y\sim\bin(50,0.01) \]

\[ P(Y=5)=\binom{50}{5}(0.01)^5(0.99)^{45} \]

(a) Let $ X= $ \# of correct bits until 5 errors.
\[ X\sim\nb(5,0.01) \]

\[ P(X=45)=\binom{49}{4}(0.01)^5(0.99)^{45} \]

\section{Geometric Distribution}

\begin{defbox}
    \subsection{Definition (Geometric Distribution)}
    Consider the exact same process as in the Negative Binomial distribution
    case, but with $ k=1 $. That is, we repeat the Bernoulli trials until
    the first success. Let $ X $ be the number of failures obtained before
    the first success. Then $ X $ has a \emph{Geometric distribution}
    and we write
    \[ X \sim \geo(p) \]
\end{defbox}

\textbf{Find the probability function, $ f(x)$}

Substitute $ k=1 $ into the Negative Binomial distribution to obtain
\[ f(x)=p(1-p)^x \]
for $ x\in\rinterval{0}{\infty} $.

\textbf{Prove the following:}

$ \sum\limits_{x=0}^{\infty} f(x) = 1$

\begin{proof}
    \begin{align*}
        \sum\limits_{x=0}^{\infty} (1-p)^x p
         & =\underbrace{p+p(1-p)+\cdots}_
        \text{ (geo.\ series: $a=p$, $r=1-p$)} \\
         & =\frac{p}{1-(1-p)}                  \\
         & =1
    \end{align*}
\end{proof}

\textbf{Find the cumulative distribution function, $ F(x)$}
\begin{align*}
    F(x) & =P(X\le x)                                                     \\
         & =1-P(X>x)                                                      \\
         & =1-\left[f(x+1)+f(x+2)+\cdots\right]                           \\
         & =1-\underbrace{\left[p(1-p)^{x+1}+p(1-p)^{x+2}+\cdots\right]}_
    \text{ (geo.\ series: $a=p(1-p)^{x+1}$, $r=1-p$)}                     \\
         & =1-\frac{p(1-p)^{x+1}}{1-(1-p)}                                \\
         & =1-(1-p)^{x+1}
\end{align*}
for $ x\in\rinterval{0}{\infty} $.

If $ x\in\mathbb{R} $, then
\[ F(x)=
    \left\{\begin{array}{cc}
        1-(1-p)^{\lfloor x \rfloor +1} & x\ge 0 \\
        0                              & x < 0
    \end{array}\right.
\]


\subsection{R Demo}
\begin{knitrout}
\definecolor{shadecolor}{rgb}{0.969, 0.969, 0.969}\color{fgcolor}\begin{kframe}
\begin{alltt}
\hlcom{## Coffee example (Coffee Quality Institute, 2018)}
\hlcom{## continued.}
\hlstd{coffee} \hlkwb{<-} \hlkwd{read.csv}\hlstd{(}\hlstr{"csv/coffee_arabica.csv"}\hlstd{)}
\hlstd{mfull} \hlkwb{<-} \hlkwd{lm}\hlstd{(Flavor} \hlopt{~} \hlkwd{factor}\hlstd{(Processing.Method)} \hlopt{+} \hlstd{Aroma} \hlopt{+} \hlstd{Aftertaste} \hlopt{+}
  \hlstd{Body} \hlopt{+} \hlstd{Acidity} \hlopt{+} \hlstd{Balance} \hlopt{+} \hlstd{Sweetness} \hlopt{+} \hlstd{Uniformity} \hlopt{+} \hlstd{Moisture,}
  \hlkwc{dat} \hlstd{= coffee)}
\hlkwd{summary}\hlstd{(mfull)}\hlopt{$}\hlstd{adj.r.squared}
\end{alltt}
\begin{verbatim}
## [1] 0.8073297
\end{verbatim}
\begin{alltt}
\hlkwd{AIC}\hlstd{(mfull)}
\end{alltt}
\begin{verbatim}
## [1] -1087.524
\end{verbatim}
\begin{alltt}
\hlkwd{BIC}\hlstd{(mfull)}
\end{alltt}
\begin{verbatim}
## [1] -1027.282
\end{verbatim}
\begin{alltt}
\hlkwd{library}\hlstd{(leaps)}
\hlcom{# Exhaustive, brute-force search.}
\hlstd{all_regs} \hlkwb{<-} \hlkwd{regsubsets}\hlstd{(Flavor} \hlopt{~} \hlstd{.,} \hlkwc{data} \hlstd{= coffee,} \hlkwc{nvmax} \hlstd{=} \hlnum{10}\hlstd{,}
  \hlkwc{nbest} \hlstd{=} \hlnum{2}\hlopt{^}\hlnum{10}\hlstd{,} \hlkwc{really.big} \hlstd{=} \hlnum{TRUE}\hlstd{)}
\hlstd{all_regs_summ} \hlkwb{<-} \hlkwd{summary}\hlstd{(all_regs)}
\end{alltt}
\end{kframe}
\end{knitrout}
\begin{knitrout}
\definecolor{shadecolor}{rgb}{0.969, 0.969, 0.969}\color{fgcolor}\begin{kframe}
\begin{alltt}
\hlstd{all_regs_summ}\hlopt{$}\hlstd{which}
\hlstd{all_regs_summ}\hlopt{$}\hlstd{adjr2}
\hlstd{all_regs_summ}\hlopt{$}\hlstd{bic}
\end{alltt}
\end{kframe}
\end{knitrout}
\begin{knitrout}
\definecolor{shadecolor}{rgb}{0.969, 0.969, 0.969}\color{fgcolor}\begin{kframe}
\begin{alltt}
\hlcom{# Organize results according to number of variables in}
\hlcom{# model.}
\hlstd{p} \hlkwb{<-} \hlnum{10}
\hlstd{k} \hlkwb{<-} \hlkwd{c}\hlstd{(}\hlkwd{rep}\hlstd{(}\hlnum{1}\hlstd{,} \hlkwd{choose}\hlstd{(p,} \hlnum{1}\hlstd{)),} \hlkwd{rep}\hlstd{(}\hlnum{2}\hlstd{,} \hlkwd{choose}\hlstd{(p,} \hlnum{2}\hlstd{)),} \hlkwd{rep}\hlstd{(}\hlnum{3}\hlstd{,} \hlkwd{choose}\hlstd{(p,}
  \hlnum{3}\hlstd{)),} \hlkwd{rep}\hlstd{(}\hlnum{4}\hlstd{,} \hlkwd{choose}\hlstd{(p,} \hlnum{4}\hlstd{)),} \hlkwd{rep}\hlstd{(}\hlnum{5}\hlstd{,} \hlkwd{choose}\hlstd{(p,} \hlnum{5}\hlstd{)),} \hlkwd{rep}\hlstd{(}\hlnum{6}\hlstd{,} \hlkwd{choose}\hlstd{(p,}
  \hlnum{6}\hlstd{)),} \hlkwd{rep}\hlstd{(}\hlnum{7}\hlstd{,} \hlkwd{choose}\hlstd{(p,} \hlnum{7}\hlstd{)),} \hlkwd{rep}\hlstd{(}\hlnum{8}\hlstd{,} \hlkwd{choose}\hlstd{(p,} \hlnum{8}\hlstd{)),} \hlkwd{rep}\hlstd{(}\hlnum{9}\hlstd{,} \hlkwd{choose}\hlstd{(p,}
  \hlnum{9}\hlstd{)),} \hlkwd{rep}\hlstd{(}\hlnum{10}\hlstd{,} \hlkwd{choose}\hlstd{(p,} \hlnum{10}\hlstd{)))}
\hlkwd{boxplot}\hlstd{(all_regs_summ}\hlopt{$}\hlstd{adjr2} \hlopt{~} \hlstd{k,} \hlkwc{xlab} \hlstd{=} \hlstr{"Number of predictors"}\hlstd{,}
  \hlkwc{ylab} \hlstd{=} \hlkwd{expression}\hlstd{(R[adj]}\hlopt{^}\hlnum{2}\hlstd{),} \hlkwc{ylim} \hlstd{=} \hlkwd{c}\hlstd{(}\hlnum{0}\hlstd{,} \hlnum{1}\hlstd{))}
\hlkwd{abline}\hlstd{(}\hlkwc{h} \hlstd{=} \hlkwd{c}\hlstd{(}\hlnum{0}\hlstd{,} \hlnum{1}\hlstd{),} \hlkwc{lty} \hlstd{=} \hlnum{2}\hlstd{,} \hlkwc{col} \hlstd{=} \hlstr{"red"}\hlstd{)}
\end{alltt}
\end{kframe}

{\centering \includegraphics[width=\maxwidth]{figure/unnamed-chunk-101-1} 

}


\begin{kframe}\begin{alltt}
\hlkwd{boxplot}\hlstd{(all_regs_summ}\hlopt{$}\hlstd{bic} \hlopt{~} \hlstd{k,} \hlkwc{xlab} \hlstd{=} \hlstr{"Number of predictors"}\hlstd{,}
  \hlkwc{ylab} \hlstd{=} \hlstr{"BIC"}\hlstd{)}
\end{alltt}
\end{kframe}

{\centering \includegraphics[width=\maxwidth]{figure/unnamed-chunk-101-2} 

}


\begin{kframe}\begin{alltt}
\hlkwd{max}\hlstd{(all_regs_summ}\hlopt{$}\hlstd{adjr2)}
\end{alltt}
\begin{verbatim}
## [1] 0.8075027
\end{verbatim}
\begin{alltt}
\hlstd{bestR2adj} \hlkwb{<-} \hlkwd{which.max}\hlstd{(all_regs_summ}\hlopt{$}\hlstd{adjr2)}
\hlkwd{min}\hlstd{(all_regs_summ}\hlopt{$}\hlstd{bic)}
\end{alltt}
\begin{verbatim}
## [1] -1793.389
\end{verbatim}
\begin{alltt}
\hlstd{bestBIC} \hlkwb{<-} \hlkwd{which.min}\hlstd{(all_regs_summ}\hlopt{$}\hlstd{bic)}
\hlcom{# Find out which predictors in those models.}
\hlstd{all_regs_summ}\hlopt{$}\hlstd{which[bestR2adj, ]}
\end{alltt}
\begin{verbatim}
##                                (Intercept) 
##                                       TRUE 
## Processing.MethodSemi-washed / Semi-pulped 
##                                      FALSE 
##              Processing.MethodWashed / Wet 
##                                       TRUE 
##                                      Aroma 
##                                       TRUE 
##                                 Aftertaste 
##                                       TRUE 
##                                       Body 
##                                       TRUE 
##                                    Acidity 
##                                       TRUE 
##                                    Balance 
##                                       TRUE 
##                                  Sweetness 
##                                       TRUE 
##                                 Uniformity 
##                                       TRUE 
##                                   Moisture 
##                                       TRUE
\end{verbatim}
\begin{alltt}
\hlstd{all_regs_summ}\hlopt{$}\hlstd{which[bestBIC, ]}
\end{alltt}
\begin{verbatim}
##                                (Intercept) 
##                                       TRUE 
## Processing.MethodSemi-washed / Semi-pulped 
##                                      FALSE 
##              Processing.MethodWashed / Wet 
##                                       TRUE 
##                                      Aroma 
##                                       TRUE 
##                                 Aftertaste 
##                                       TRUE 
##                                       Body 
##                                       TRUE 
##                                    Acidity 
##                                       TRUE 
##                                    Balance 
##                                      FALSE 
##                                  Sweetness 
##                                       TRUE 
##                                 Uniformity 
##                                      FALSE 
##                                   Moisture 
##                                      FALSE
\end{verbatim}
\begin{alltt}
\hlstd{coffee}\hlopt{$}\hlstd{wet} \hlkwb{<-} \hlkwd{ifelse}\hlstd{(coffee}\hlopt{$}\hlstd{Processing.Method} \hlopt{==} \hlstr{"Washed / Wet"}\hlstd{,}
  \hlnum{1}\hlstd{,} \hlnum{0}\hlstd{)}  \hlcom{# 1 = wet, 0 otherwise}
\hlstd{coffee}\hlopt{$}\hlstd{semi} \hlkwb{<-} \hlkwd{ifelse}\hlstd{(coffee}\hlopt{$}\hlstd{Processing.Method} \hlopt{==} \hlstr{"Semi-washed / Semi-pulped"}\hlstd{,}
  \hlnum{1}\hlstd{,} \hlnum{0}\hlstd{)}  \hlcom{# 1 = semi/dry, 0 otherwise}
\hlstd{coffee}\hlopt{$}\hlstd{Processing.Method} \hlkwb{<-} \hlkwa{NULL}
\hlstd{m_bestr2adj} \hlkwb{<-} \hlkwd{lm}\hlstd{(Flavor} \hlopt{~} \hlstd{wet} \hlopt{+} \hlstd{Aroma} \hlopt{+} \hlstd{Aftertaste} \hlopt{+} \hlstd{Body} \hlopt{+}
  \hlstd{Acidity} \hlopt{+} \hlstd{Balance} \hlopt{+} \hlstd{Sweetness} \hlopt{+} \hlstd{Uniformity} \hlopt{+} \hlstd{Moisture,} \hlkwc{dat} \hlstd{= coffee)}
\hlkwd{summary}\hlstd{(m_bestr2adj)}
\end{alltt}
\begin{verbatim}
## 
## Call:
## lm(formula = Flavor ~ wet + Aroma + Aftertaste + Body + Acidity + 
##     Balance + Sweetness + Uniformity + Moisture, data = coffee)
## 
## Residuals:
##      Min       1Q   Median       3Q      Max 
## -0.68587 -0.08469  0.00080  0.08923  0.63660 
## 
## Coefficients:
##              Estimate Std. Error t value Pr(>|t|)    
## (Intercept) -0.728709   0.168439  -4.326 1.65e-05 ***
## wet         -0.032797   0.010197  -3.216  0.00134 ** 
## Aroma        0.220278   0.020434  10.780  < 2e-16 ***
## Aftertaste   0.468749   0.023901  19.612  < 2e-16 ***
## Body         0.096194   0.024308   3.957 8.06e-05 ***
## Acidity      0.216754   0.021185  10.232  < 2e-16 ***
## Balance      0.046793   0.022547   2.075  0.03819 *  
## Sweetness    0.025480   0.010136   2.514  0.01209 *  
## Uniformity   0.016291   0.009798   1.663  0.09665 .  
## Moisture     0.168439   0.102033   1.651  0.09906 .  
## ---
## Signif. codes:  0 '***' 0.001 '**' 0.01 '*' 0.05 '.' 0.1 ' ' 1
## 
## Residual standard error: 0.1479 on 1109 degrees of freedom
## Multiple R-squared:  0.8091,	Adjusted R-squared:  0.8075 
## F-statistic: 522.1 on 9 and 1109 DF,  p-value: < 2.2e-16
\end{verbatim}
\begin{alltt}
\hlkwd{AIC}\hlstd{(m_bestr2adj)}
\end{alltt}
\begin{verbatim}
## [1] -1089.52
\end{verbatim}
\begin{alltt}
\hlkwd{BIC}\hlstd{(m_bestr2adj)}
\end{alltt}
\begin{verbatim}
## [1] -1034.298
\end{verbatim}
\begin{alltt}
\hlstd{m_bestBIC} \hlkwb{<-} \hlkwd{lm}\hlstd{(Flavor} \hlopt{~} \hlstd{wet} \hlopt{+} \hlstd{Aroma} \hlopt{+} \hlstd{Aftertaste} \hlopt{+} \hlstd{Body} \hlopt{+} \hlstd{Acidity} \hlopt{+}
  \hlstd{Sweetness,} \hlkwc{dat} \hlstd{= coffee)}
\hlkwd{summary}\hlstd{(m_bestBIC)}
\end{alltt}
\begin{verbatim}
## 
## Call:
## lm(formula = Flavor ~ wet + Aroma + Aftertaste + Body + Acidity + 
##     Sweetness, data = coffee)
## 
## Residuals:
##      Min       1Q   Median       3Q      Max 
## -0.65627 -0.08781  0.00032  0.08529  0.63010 
## 
## Coefficients:
##              Estimate Std. Error t value Pr(>|t|)    
## (Intercept) -0.609003   0.159910  -3.808 0.000148 ***
## wet         -0.032852   0.010198  -3.221 0.001313 ** 
## Aroma        0.225969   0.020378  11.089  < 2e-16 ***
## Aftertaste   0.490988   0.021938  22.381  < 2e-16 ***
## Body         0.103438   0.022926   4.512 7.11e-06 ***
## Acidity      0.225638   0.020994  10.748  < 2e-16 ***
## Sweetness    0.033445   0.009582   3.491 0.000501 ***
## ---
## Signif. codes:  0 '***' 0.001 '**' 0.01 '*' 0.05 '.' 0.1 ' ' 1
## 
## Residual standard error: 0.1484 on 1112 degrees of freedom
## Multiple R-squared:  0.8073,	Adjusted R-squared:  0.8063 
## F-statistic: 776.4 on 6 and 1112 DF,  p-value: < 2.2e-16
\end{verbatim}
\begin{alltt}
\hlkwd{AIC}\hlstd{(m_bestBIC)}
\end{alltt}
\begin{verbatim}
## [1] -1085.26
\end{verbatim}
\begin{alltt}
\hlkwd{BIC}\hlstd{(m_bestBIC)}
\end{alltt}
\begin{verbatim}
## [1] -1045.098
\end{verbatim}
\begin{alltt}
\hlcom{# Let's also try stepwise methods.}
\hlkwd{library}\hlstd{(MASS)}
\hlcom{# Full model and empty model with just intercept.}
\hlstd{full} \hlkwb{<-} \hlkwd{lm}\hlstd{(Flavor} \hlopt{~} \hlstd{.,} \hlkwc{data} \hlstd{= coffee)}
\hlstd{empty} \hlkwb{<-} \hlkwd{lm}\hlstd{(Flavor} \hlopt{~} \hlnum{1}\hlstd{,} \hlkwc{data} \hlstd{= coffee)}
\hlcom{# Default stepAIC uses AIC criterion.}
\hlstd{m_f_AIC} \hlkwb{<-} \hlkwd{stepAIC}\hlstd{(}\hlkwc{object} \hlstd{= empty,} \hlkwc{scope} \hlstd{=} \hlkwd{list}\hlstd{(}\hlkwc{upper} \hlstd{= full,}
  \hlkwc{lower} \hlstd{= empty),} \hlkwc{direction} \hlstd{=} \hlstr{"forward"}\hlstd{,} \hlkwc{trace} \hlstd{=} \hlnum{0}\hlstd{)}
\hlcom{# Let's get stepAIC to use BIC by specifying the penalty k}
\hlcom{# = log(n).  Add 'trace = 0' to hide the output.  Forward.}
\hlstd{m_f} \hlkwb{<-} \hlkwd{stepAIC}\hlstd{(}\hlkwc{object} \hlstd{= empty,} \hlkwc{scope} \hlstd{=} \hlkwd{list}\hlstd{(}\hlkwc{upper} \hlstd{= full,} \hlkwc{lower} \hlstd{= empty),}
  \hlkwc{direction} \hlstd{=} \hlstr{"forward"}\hlstd{,} \hlkwc{trace} \hlstd{=} \hlnum{0}\hlstd{,} \hlkwc{k} \hlstd{=} \hlkwd{log}\hlstd{(}\hlkwd{nrow}\hlstd{(coffee)))}
\hlkwd{summary}\hlstd{(m_f)}
\end{alltt}
\begin{verbatim}
## 
## Call:
## lm(formula = Flavor ~ Aftertaste + Acidity + Aroma + Body + Sweetness + 
##     wet, data = coffee)
## 
## Residuals:
##      Min       1Q   Median       3Q      Max 
## -0.65627 -0.08781  0.00032  0.08529  0.63010 
## 
## Coefficients:
##              Estimate Std. Error t value Pr(>|t|)    
## (Intercept) -0.609003   0.159910  -3.808 0.000148 ***
## Aftertaste   0.490988   0.021938  22.381  < 2e-16 ***
## Acidity      0.225638   0.020994  10.748  < 2e-16 ***
## Aroma        0.225969   0.020378  11.089  < 2e-16 ***
## Body         0.103438   0.022926   4.512 7.11e-06 ***
## Sweetness    0.033445   0.009582   3.491 0.000501 ***
## wet         -0.032852   0.010198  -3.221 0.001313 ** 
## ---
## Signif. codes:  0 '***' 0.001 '**' 0.01 '*' 0.05 '.' 0.1 ' ' 1
## 
## Residual standard error: 0.1484 on 1112 degrees of freedom
## Multiple R-squared:  0.8073,	Adjusted R-squared:  0.8063 
## F-statistic: 776.4 on 6 and 1112 DF,  p-value: < 2.2e-16
\end{verbatim}
\begin{alltt}
\hlcom{# Backward.}
\hlstd{m_b} \hlkwb{<-} \hlkwd{stepAIC}\hlstd{(}\hlkwc{object} \hlstd{= full,} \hlkwc{scope} \hlstd{=} \hlkwd{list}\hlstd{(}\hlkwc{upper} \hlstd{= full,} \hlkwc{lower} \hlstd{= empty),}
  \hlkwc{direction} \hlstd{=} \hlstr{"backward"}\hlstd{,} \hlkwc{trace} \hlstd{=} \hlnum{0}\hlstd{,} \hlkwc{k} \hlstd{=} \hlkwd{log}\hlstd{(}\hlkwd{nrow}\hlstd{(coffee)))}
\hlkwd{summary}\hlstd{(m_b)}
\end{alltt}
\begin{verbatim}
## 
## Call:
## lm(formula = Flavor ~ Aroma + Aftertaste + Body + Acidity + Sweetness + 
##     wet, data = coffee)
## 
## Residuals:
##      Min       1Q   Median       3Q      Max 
## -0.65627 -0.08781  0.00032  0.08529  0.63010 
## 
## Coefficients:
##              Estimate Std. Error t value Pr(>|t|)    
## (Intercept) -0.609003   0.159910  -3.808 0.000148 ***
## Aroma        0.225969   0.020378  11.089  < 2e-16 ***
## Aftertaste   0.490988   0.021938  22.381  < 2e-16 ***
## Body         0.103438   0.022926   4.512 7.11e-06 ***
## Acidity      0.225638   0.020994  10.748  < 2e-16 ***
## Sweetness    0.033445   0.009582   3.491 0.000501 ***
## wet         -0.032852   0.010198  -3.221 0.001313 ** 
## ---
## Signif. codes:  0 '***' 0.001 '**' 0.01 '*' 0.05 '.' 0.1 ' ' 1
## 
## Residual standard error: 0.1484 on 1112 degrees of freedom
## Multiple R-squared:  0.8073,	Adjusted R-squared:  0.8063 
## F-statistic: 776.4 on 6 and 1112 DF,  p-value: < 2.2e-16
\end{verbatim}
\begin{alltt}
\hlcom{# Forward-backward.}
\hlstd{m_h} \hlkwb{<-} \hlkwd{stepAIC}\hlstd{(}\hlkwc{object} \hlstd{= empty,} \hlkwc{scope} \hlstd{=} \hlkwd{list}\hlstd{(}\hlkwc{upper} \hlstd{= full,} \hlkwc{lower} \hlstd{= empty),}
  \hlkwc{direction} \hlstd{=} \hlstr{"both"}\hlstd{,} \hlkwc{trace} \hlstd{=} \hlnum{0}\hlstd{,} \hlkwc{k} \hlstd{=} \hlkwd{log}\hlstd{(}\hlkwd{nrow}\hlstd{(coffee)))}
\hlkwd{summary}\hlstd{(m_h)}
\end{alltt}
\begin{verbatim}
## 
## Call:
## lm(formula = Flavor ~ Aftertaste + Acidity + Aroma + Body + Sweetness + 
##     wet, data = coffee)
## 
## Residuals:
##      Min       1Q   Median       3Q      Max 
## -0.65627 -0.08781  0.00032  0.08529  0.63010 
## 
## Coefficients:
##              Estimate Std. Error t value Pr(>|t|)    
## (Intercept) -0.609003   0.159910  -3.808 0.000148 ***
## Aftertaste   0.490988   0.021938  22.381  < 2e-16 ***
## Acidity      0.225638   0.020994  10.748  < 2e-16 ***
## Aroma        0.225969   0.020378  11.089  < 2e-16 ***
## Body         0.103438   0.022926   4.512 7.11e-06 ***
## Sweetness    0.033445   0.009582   3.491 0.000501 ***
## wet         -0.032852   0.010198  -3.221 0.001313 ** 
## ---
## Signif. codes:  0 '***' 0.001 '**' 0.01 '*' 0.05 '.' 0.1 ' ' 1
## 
## Residual standard error: 0.1484 on 1112 degrees of freedom
## Multiple R-squared:  0.8073,	Adjusted R-squared:  0.8063 
## F-statistic: 776.4 on 6 and 1112 DF,  p-value: < 2.2e-16
\end{verbatim}
\end{kframe}
\end{knitrout}

10 variables is still a fairly small problem:
in this example all 3 approaches identify the same BIC-based model as
the exhaustive search.
\section{2020-02-07}
\underline{Roadmap}:
\begin{itemize}
    \item PPDAC example
    \item Interval estimation
          \begin{itemize}
              \item Intervals using the likelihood function
              \item Confidence intervals
          \end{itemize}
\end{itemize}
\underline{PPDAC}
\begin{itemize}
    \item Problem
    \item Plan
    \item Data
    \item Analysis
    \item Conclusion
\end{itemize}
\underline{Problem}
\begin{itemize}
    \item What kind of study is this?
          \begin{itemize}
              \item Observational
              \item Experimental
          \end{itemize}
    \item What kind of problem is this?
          \begin{itemize}
              \item Descriptive
              \item Causative
              \item Predictive
          \end{itemize}
    \item What is the target population?
          \begin{itemize}
              \item Target population: Population of interest
          \end{itemize}
    \item What are the variates and attributes of interest?
          \begin{itemize}
              \item Attribute $ = $ function of the variate of interest
              \item $ \theta= $ proportion of Canadians who believe climate change is the number one issue
          \end{itemize}
    \item What is the study population?
          \begin{itemize}
              \item Study population: The act of observing from which the sample is drawn
          \end{itemize}
    \item What is the sampling protocol?
          \begin{itemize}
              \item How is the sample collected?
          \end{itemize}
    \item What could be a source of study error?
    \item What could be a source of sampling error?
\end{itemize}
\underline{Analysis}

\underline{Data}: Try to avoid \textbf{bias} where bias is systematic error.

Blind study: \underline{Medical tests}
\begin{itemize}
    \item Control group $ \rightarrow $ Placebo (sugar pill)
    \item Experimental group $ \rightarrow $ Actual drug
    \item The patient does not know.
\end{itemize}
Double blind study: the doctors do not know

\underline{Types of errors}
\begin{itemize}
    \item Study errors: the difference in the value of the attribute between
          the target population and the study population
          \begin{itemize}
              \item $ \phi= $ proportion of people in Kitchener-Waterloo area who
                    believe climate change is the number one issue: $ \theta-\phi $
          \end{itemize}
    \item Sampling errors: the difference in value of the attribute between the study
          population and the sample: $ \phi-\hat{\pi} $ where $ \hat{\pi}= $
          sample proportion
    \item Measurement errors: the value of the variate vs what is actually recorded
          in the data
\end{itemize}
\underline{Conclusion}: Non-mathematical discussion of the final result

\underline{Interval estimation}

\underline{Objective}:
\begin{itemize}
    \item To find the ``reasonable'' values of $ \theta $, given by data set
    \item To quantify the ``reasonableness'' of your constructed interval
\end{itemize}
\underline{Method 1}: Through the likelihood function (likelihood interval)

\begin{Definition}{}{}
    The $ 100p\% $ likelihood interval where $ p\in[0,1] $, is given by
    \[ \left\{ \theta:R(\theta)\geqslant p\right\} \]
    where $ R(\theta)= $ relative likelihood function.
\end{Definition}


\begin{Example}{}{}
    Find the $ 10\% $ likelihood interval given the figure below.
    \begin{center}
        \includegraphics{likelihood interval.png}
    \end{center}
\end{Example}

\begin{center}
    \underline{Guidelines for Interpreting Likelihood Intervals}
\end{center}
\begin{center}
    \begin{tabular}{|c|}
        \hline
        Values of $ \theta $ inside a $ 50\% $ likelihood interval are very plausible in light of
        the observed data. \\
        \hline
        Values of $ \theta $ inside a $ 10\% $ likelihood interval are plausible in light of
        the observed data. \\
        \hline
        Values of $ \theta $ outside a $ 10\% $ likelihood interval are implausible in light of
        the observed data. \\
        \hline
        Values of $ \theta $ outside a $ 1\% $ likelihood interval are very implausible in light of
        the observed data. \\
        \hline
    \end{tabular}
\end{center}
\underline{Clicker Question 1}: THE MLE $ \hat{\theta} $ is in every likelihood
interval for all $ p\in[0,1] $.
\begin{enumerate}[label=(\alph*)]
    \item \textbf{True}
    \item False
\end{enumerate}
\underline{Clicker Question 2}: If $ \theta $ is in the $ p\% $ likelihood
interval, it has to be in the $ q\% $ likelihood interval if $ q>p $.
\begin{enumerate}[label=(\alph*)]
    \item True
    \item \textbf{False}
\end{enumerate}


\subsection{R Demo}
\begin{knitrout}
\definecolor{shadecolor}{rgb}{0.969, 0.969, 0.969}\color{fgcolor}\begin{kframe}
\begin{alltt}
\hlcom{### Residual plots/diagnostics demo.}
\hlcom{## Florida oranges revisited.}
\hlstd{dat} \hlkwb{<-} \hlkwd{read.csv}\hlstd{(}\hlstr{"csv/florange.csv"}\hlstd{)}
\hlkwd{plot}\hlstd{(dat}\hlopt{$}\hlstd{acres, dat}\hlopt{$}\hlstd{boxes)}
\end{alltt}
\end{kframe}

{\centering \includegraphics[width=\maxwidth]{figure/unnamed-chunk-105-1} 

}


\begin{kframe}\begin{alltt}
\hlstd{lm.1} \hlkwb{<-} \hlkwd{lm}\hlstd{(dat}\hlopt{$}\hlstd{boxes} \hlopt{~} \hlstd{dat}\hlopt{$}\hlstd{acres)}
\hlkwd{summary}\hlstd{(lm.1)}
\end{alltt}
\begin{verbatim}
## 
## Call:
## lm(formula = dat$boxes ~ dat$acres)
## 
## Residuals:
##      Min       1Q   Median       3Q      Max 
## -2470.81    -6.17    71.72   106.46  1677.32 
## 
## Coefficients:
##               Estimate Std. Error t value Pr(>|t|)    
## (Intercept) -85.391989 186.178031  -0.459    0.651    
## dat$acres     0.116717   0.006761  17.263 1.16e-14 ***
## ---
## Signif. codes:  0 '***' 0.001 '**' 0.01 '*' 0.05 '.' 0.1 ' ' 1
## 
## Residual standard error: 754.4 on 23 degrees of freedom
## Multiple R-squared:  0.9284,	Adjusted R-squared:  0.9252 
## F-statistic:   298 on 1 and 23 DF,  p-value: 1.164e-14
\end{verbatim}
\begin{alltt}
\hlcom{# Residual plot: vs fitted values.}
\hlkwd{plot}\hlstd{(lm.1}\hlopt{$}\hlstd{fitted.values, lm.1}\hlopt{$}\hlstd{residuals,} \hlkwc{xlab} \hlstd{=} \hlstr{"Fitted Values"}\hlstd{,}
  \hlkwc{ylab} \hlstd{=} \hlstr{"Residuals"}\hlstd{)}
\end{alltt}
\end{kframe}

{\centering \includegraphics[width=\maxwidth]{figure/unnamed-chunk-105-2} 

}


\begin{kframe}\begin{alltt}
\hlcom{# Residual plot: vs predictor (just one in this case).}
\hlkwd{plot}\hlstd{(dat}\hlopt{$}\hlstd{acres, lm.1}\hlopt{$}\hlstd{residuals,} \hlkwc{xlab} \hlstd{=} \hlstr{"Acres"}\hlstd{,} \hlkwc{ylab} \hlstd{=} \hlstr{"Residuals"}\hlstd{)}
\end{alltt}
\end{kframe}

{\centering \includegraphics[width=\maxwidth]{figure/unnamed-chunk-105-3} 

}


\begin{kframe}\begin{alltt}
\hlcom{# Residual plot: vs i (just to demo plot; no time/space}
\hlcom{# ordering here).}
\hlkwd{plot}\hlstd{(}\hlnum{1}\hlopt{:}\hlkwd{nrow}\hlstd{(dat), lm.1}\hlopt{$}\hlstd{residuals,} \hlkwc{xlab} \hlstd{=} \hlstr{"Index"}\hlstd{,} \hlkwc{ylab} \hlstd{=} \hlstr{"Residuals"}\hlstd{)}
\end{alltt}
\end{kframe}

{\centering \includegraphics[width=\maxwidth]{figure/unnamed-chunk-105-4} 

}


\begin{kframe}\begin{alltt}
\hlcom{# Histogram of residuals.}
\hlkwd{hist}\hlstd{(lm.1}\hlopt{$}\hlstd{residuals)}
\end{alltt}
\end{kframe}

{\centering \includegraphics[width=\maxwidth]{figure/unnamed-chunk-105-5} 

}


\begin{kframe}\begin{alltt}
\hlcom{# QQ plot of residuals.}
\hlkwd{qqnorm}\hlstd{(lm.1}\hlopt{$}\hlstd{residuals)}
\hlkwd{qqline}\hlstd{(lm.1}\hlopt{$}\hlstd{residuals,} \hlkwc{col} \hlstd{=} \hlstr{"blue"}\hlstd{,} \hlkwc{lwd} \hlstd{=} \hlnum{2}\hlstd{)}
\end{alltt}
\end{kframe}

{\centering \includegraphics[width=\maxwidth]{figure/unnamed-chunk-105-6} 

}


\begin{kframe}\begin{alltt}
\hlcom{## Rocket data revisited.}
\hlstd{rocket} \hlkwb{<-} \hlkwd{read.csv}\hlstd{(}\hlstr{"csv/rocket.csv"}\hlstd{)}
\hlstd{mr} \hlkwb{<-} \hlkwd{lm}\hlstd{(thrust} \hlopt{~} \hlstd{nozzle} \hlopt{+} \hlstd{propratio,} \hlkwc{data} \hlstd{= rocket)}
\hlkwd{summary}\hlstd{(mr)}
\end{alltt}
\begin{verbatim}
## 
## Call:
## lm(formula = thrust ~ nozzle + propratio, data = rocket)
## 
## Residuals:
##     Min      1Q  Median      3Q     Max 
## -3.8459 -1.7555  0.5934  1.2906  3.3008 
## 
## Coefficients:
##             Estimate Std. Error t value Pr(>|t|)    
## (Intercept) 473.6039     4.7158 100.430 4.88e-15 ***
## nozzle       16.7383     1.5329  10.919 1.71e-06 ***
## propratio    -1.0948     0.9414  -1.163    0.275    
## ---
## Signif. codes:  0 '***' 0.001 '**' 0.01 '*' 0.05 '.' 0.1 ' ' 1
## 
## Residual standard error: 2.655 on 9 degrees of freedom
## Multiple R-squared:  0.9303,	Adjusted R-squared:  0.9148 
## F-statistic: 60.05 on 2 and 9 DF,  p-value: 6.238e-06
\end{verbatim}
\begin{alltt}
\hlcom{# Residual plot: vs fitted values.}
\hlkwd{plot}\hlstd{(mr}\hlopt{$}\hlstd{fitted.values, mr}\hlopt{$}\hlstd{residuals,} \hlkwc{xlab} \hlstd{=} \hlstr{"Fitted Values"}\hlstd{,}
  \hlkwc{ylab} \hlstd{=} \hlstr{"Residuals"}\hlstd{)}
\end{alltt}
\end{kframe}

{\centering \includegraphics[width=\maxwidth]{figure/unnamed-chunk-105-7} 

}


\begin{kframe}\begin{alltt}
\hlcom{# Residual plot: vs predictors.}
\hlkwd{plot}\hlstd{(rocket}\hlopt{$}\hlstd{nozzle, mr}\hlopt{$}\hlstd{residuals,} \hlkwc{xlab} \hlstd{=} \hlstr{"Nozzle (1 = large)"}\hlstd{,}
  \hlkwc{ylab} \hlstd{=} \hlstr{"Residuals"}\hlstd{)}
\end{alltt}
\end{kframe}

{\centering \includegraphics[width=\maxwidth]{figure/unnamed-chunk-105-8} 

}


\begin{kframe}\begin{alltt}
\hlkwd{plot}\hlstd{(rocket}\hlopt{$}\hlstd{propratio, mr}\hlopt{$}\hlstd{residuals,} \hlkwc{xlab} \hlstd{=} \hlstr{"Propellant to fuel ratio"}\hlstd{,}
  \hlkwc{ylab} \hlstd{=} \hlstr{"Residuals"}\hlstd{)}
\end{alltt}
\end{kframe}

{\centering \includegraphics[width=\maxwidth]{figure/unnamed-chunk-105-9} 

}


\begin{kframe}\begin{alltt}
\hlcom{# Histogram of residuals,}
\hlkwd{hist}\hlstd{(mr}\hlopt{$}\hlstd{residuals)}
\end{alltt}
\end{kframe}

{\centering \includegraphics[width=\maxwidth]{figure/unnamed-chunk-105-10} 

}


\begin{kframe}\begin{alltt}
\hlcom{# QQ plot of residuals,}
\hlkwd{qqnorm}\hlstd{(mr}\hlopt{$}\hlstd{residuals)}
\hlkwd{qqline}\hlstd{(mr}\hlopt{$}\hlstd{residuals,} \hlkwc{col} \hlstd{=} \hlstr{"blue"}\hlstd{,} \hlkwc{lwd} \hlstd{=} \hlnum{2}\hlstd{)}
\end{alltt}
\end{kframe}

{\centering \includegraphics[width=\maxwidth]{figure/unnamed-chunk-105-11} 

}


\end{knitrout}
\makeheading{Lecture 16 | 2020-10-01}
\section{Moment Generating Function Technique}
Idea:
\begin{enumerate}[label=(\arabic*)]
    \item Find the moment generating function of a random variable
    \item Use uniqueness theorem of moment generating function
          to find the distribution of the random variable and then
          the p.d.f.\ of a random variable.
\end{enumerate}
\begin{Theorem}{}{}
    Suppose $ X_1,\ldots,X_n $
    are independent, then $ T=\sum_{i=1}^{n} X_i $
    has moment generating function
    \[ M_T(t)=\E*{e^{t \sum_{i=1}^{n} X_i}}=
        \E*{\prod_{i=1}^n e^{tX_i}}=\prod_{i=1}^n \E*{e^{tX_i}}=
        \prod_{i=1}^n M_{X_i}(t) \]
    In particular, if $ X_1,\ldots,X_n $ are independently
    and identically distributed, then they
    have the exact same moment generating function $ M(t) $;
    that is,
    \[ M_T(t)=\left[ M(t) \right]^n \]
\end{Theorem}
Next, we use the m.g.f.\ technique to find properties
of normal, $ \chi^2 $, $ t $-distribution,
and $ F $-distributions.

If $ X \sim N(\mu,\sigma^2) $, then
\[ aX+b \sim N(a\mu+b,a^2\sigma^2) \]
Recall that
\[ M_X(t)
    =\exp\left\{ \mu t+\frac{\sigma^2t^2}{2}\right\} \]
Therefore,
\begin{align*}
    M_{aX+b}(t)
     & =e^{b t}M_X(at)                                                              \\
     & =\exp\left\{ b t\right\}\exp\left\{ a\mu t+\frac{a^2\sigma^2t^2}{2} \right\} \\
     & =\expon*{b t}\expon*{a\mu t+\frac{a^2\sigma^2t^2}{2}}                        \\
     & =\expon*{(a\mu+b)t+\frac{a^2\sigma^2}{2} t^2}
\end{align*}
which is the m.g.f.\ of $ \N{a\mu+b,a^2\sigma^2} $.

If $ X \sim \N{\mu,\sigma^2} $, then
\[ \frac{X-\mu}{\sigma} \sim \N{0,1} \]

If $ X_i \sim \N{\mu_i,\sigma_i^2} $ for $ i=1,\ldots,n $
are independent, then
\[ \sum_{i=1}^{n} a_i X_i \sim \N[\bigg]{\sum_{i=1}^{n} a_i\mu_i,
        \sum_{i=1}^{n} a_i^2\sigma_i^2} \]
We know $ a_i X_i \sim \N{a_i\mu_i,a_i^2\sigma_i^2} $
for $ i=1,\ldots, n $.
\[ M_{a_i X_i}(t)=\expon*{(a_i\mu_i)t+\frac{a_i^2\sigma_i^2}{2}t^2} \]
Therefore,
\begin{align*}
    M_{\sum_{i=1}^{n} a_i X_i}(t)
     & =\E*{e^{t \sum_{i=1}^{n} a_i X_i}}                                 \\
     & =\E*{\prod_{i=1}^n e^{(a_i X_i)t}}                                 \\
     & =\prod_{i=1}^n\E*{e^{(a_i X_i)t}}                                  \\
     & =\prod_{i=1}^n M_{a_i X_i}(t)                                      \\
     & =\prod_{i=1}^n \expon*{(a_i\mu_i)t+\frac{\sigma_i^2 a_i^2}{2}t^2 } \\
     & =\expon*{\biggl(\sum_{i=1}^{n} a_i\mu_i\biggr)t
        +\frac{\sum_{i=1}^{n} a_i^2\sigma_i^2}{2}t^2}
\end{align*}
which is the m.g.f.\ of
$ \N*{\sum_{i=1}^{n} a_i\mu_i,\sum_{i=1}^{n} a_i^2\sigma_i^2} $.

If $ X_1,\ldots,X_n \stackrel{\text{iid}}{\sim} \N{\mu,\sigma^2} $,
then
\[ \sum_{i=1}^{n} X_i \sim \N*{n\mu,n\sigma^2} \]
\[ \frac{1}{n} \sum_{i=1}^{n} X_i \sim \N*{\mu,\frac{\sigma^2}{n}} \]
Let $ \displaystyle \bar{X}=\frac{1}{n} \sum_{i=1}^{n}X_i $, then
\[ \bar{X}\sim \N*{\mu,\frac{\sigma^2}{n} } \]
\[ \frac{\bar{X}-\mu}{\sigma}\sim \N{0,1}  \]
\begin{Definition}{Chi-Squared Distribution}{}
    If $ Z_1,\ldots,Z_k \sim \N{0,1} $ are independent
    and $ 0<k\in\mathbb{Z} $, then
    \[ Q=\sum_{i=1}^{k} Z_i^2 \]
    follows a \textbf{chi-squared distribution}
    with $ k $ degrees of freedom and write
    $ Q \sim \chi^2(k) $.
\end{Definition}
If $ X \sim \N{\mu,\sigma^2} $, then
\[ \biggl(\frac{X-\mu}{\sigma}\biggr)^2 \sim \chi^2(1)  \]
If $ Y_i \sim \chi^2(k_i) $ are independent, then
\[ \sum_{i=1}^{n} Y_i \sim \chi^2\biggl(\sum_{i=1}^{n} k_i\biggr) \]
The m.g.f.\ of $ \chi^2(1) $ is $ (1-2t)^{-1/2} $. Derive
the m.g.f.\ $ \chi^2(n) $: $ (1-2t)^{-n/2} $.
\[ \chi^2(n)=\sum_{i=1}^{n} X_i^2\quad X_i\stackrel{\text{iid}}{\sim}\N{0,1}\]
\begin{align*}
    M_{\sum_{i=1}^{n} Y_i}(t)
     & =\prod_{i=1}^n M_{Y_i}(t)         \\
     & =\prod_{i=1}^n(1-2t)^{-k/2}       \\
     & =(1-2t)^{-(\sum_{i=1}^{n} k_i)/2}
\end{align*}
In summary: sum of independent $ \chi^2 $ distributions
follow $ \chi^2 $ with d.f.\ being sum of d.f.\ of
$ \chi^2 $ distributions.

Specifically, if $ X_1,\ldots,X_n\stackrel{\text{iid}}{\sim}\N{\mu,\sigma^2} $
\[ \sum_{i=1}^{n} \biggl(\frac{X_i-\mu}{\sigma}\biggr)^2=
    \frac{\sum_{i=1}^{n} (X_i-\mu)^2}{\sigma^2}\sim \chi(n)  \]
\begin{Definition}{Student's $ t $-distribution}{}
    Let $ Z \sim \N{0,1} $ and $ Q \sim \chi^2(\nu) $
    be independent, then
    \[ T=\frac{Z}{\sqrt{Q/\nu}}  \]
    follows a \textbf{student's t-distribution}
    with $ k $ degrees of freedom and write
    $ T \sim t(\nu) $.

    Support of $ T $: $ (-\infty,\infty) $.
\end{Definition}
\begin{Definition}{$ F $-distribution}{}
    If $ X \sim \chi^2(n) $ and $ Y \sim \chi^2(m) $
    are independent, then
    \[ \frac{X/n}{Y/m} \sim F(n,m) \]
    follows a \textbf{F-distribution}.

    Support of $ F(n,m) $:
    \begin{itemize}
        \item If $ n=1 $: $ \interval[open right]{0}{\infty} $.
        \item If $ n\neq 1 $: $ (0,\infty) $.
    \end{itemize}
\end{Definition}
If $ X \sim \chi^2(n) $ and $ Y \sim \chi^2(m) $ are independent,
then
\[ X+Y \sim \chi^2(n+m) \]
\begin{Exercise}{}{}
    True or false:
    \[ \frac{X/n}{(X+Y)/(n+m)} \sim F(n,n+m) \]
\end{Exercise}
Answer: False.
\begin{Example}{$ \chi^2 $-distribution}{}
    If $ X_1,\ldots,X_n \sim \N{\mu,\sigma^2} $, then
    \[ \sum_{i=1}^{n} \biggl(\frac{X_i-\mu}{\sigma} \biggr) \sim \chi^2(n) \]
    In STAT 231, if we replace $ \mu $ by $ \bar{X} $, then
    \[ \frac{\sum_{i=1}^{n} (X_i-\bar{X})^2}{\sigma^2} \sim \chi^2(n-1)  \]
\end{Example}
Why do we lose one d.f.\ when replacing $ \mu $ by $ \bar{X} $?
\begin{Proof}{}{}
    \begin{align*}
        \frac{\sum_{i=1}^{n} (X_i-\mu)^2}{\sigma^2}
         & =\frac{\sum_{i=1}^{n} (X_i-\bar{X}+\bar{X}-\mu)^2}{\sigma^2} \\
         & =\frac{\sum_{i=1}^{n} (X_i-\bar{X})^2}{\sigma^2}+2
        \frac{\sum_{i=1}^{n} (X_i-\bar{X})(\bar{X}-\mu)}{\sigma^2}+
        \frac{\sum_{i=1}^{n} (\bar{X}-\mu)^2}{\sigma^2}
    \end{align*}
    Note that $ \sum_{i=1}^{n} (X_i-\bar{X})=0 $. So,
    \[ \frac{\sum_{i=1}^{n} (X_i-\mu)^2}{\sigma^2}=
        \frac{\sum_{i=1}^{n} (X_i-\bar{X})^2}{\sigma^2}+
        \frac{n(\bar{X}-\mu)^2}{\sigma^2} \]
    Also,
    \[ \frac{n(\bar{X}-\mu)^2}{\sigma^2}=
        \left[ \frac{\sqrt{n}(\bar{X}-\mu)}{\sigma} \right]^2 \sim \chi^2(1)  \]
    since $ \displaystyle \frac{\sqrt{n}(\bar{X}-\mu)}{\sigma} \sim \N{0,1} $.

    On the left-hand side: $ \displaystyle  \frac{\sum_{i=1}^{n} (X_i-\mu)^2}{\sigma^2}\sim \chi^2(n) $.

    Intuitively,
    \[ \frac{\sum_{i=1}^{n}(X_i-\bar{X})^2}{\sigma^2}\sim \chi^2(n)-\chi^2(1)=\chi^2(n-1)  \]
    A key observation: $ \bar{X} $ and $ \sum_{i=1}^{n} (X_i-\bar{X})^2 $
    are independent.
    \[
        M_{\chi^2(n)}(t)=M_{\frac{\sum_{i=1}^{n} (X_i-\bar{X})^2}{\sigma^2}}(t)
        M_{\frac{n(\bar{X}-\mu)^2}{2}}(t)\\
    \]
    \[ \implies (1-2t)^{-n/2}=M_{\frac{\sum_{i=1}^{n} (X_i-\bar{X})^2}{\sigma^2}}(t)
        (1-2t)^{-1/2} \]
    \[ \implies M_{\frac{\sum_{i=1}^{n} (X_i-\bar{X})^2}{\sigma^2}}(t)=(1-2t)^{-(n-1)/2}  \]
\end{Proof}
Why $ \bar{X} $ is independent of $ \sum_{i=1}^{n} (X_i-\bar{X})^2 $?
\[ (\Uunderbracket{\bar{X}}_{0},X_1-\bar{X},\ldots,X_n-\bar{X})\sim \Mvn{\cdot} \]
Verify that $ \bar{X} $ independent of $ (X_1-\bar{X},\ldots,X_n-\bar{X}) $
by calculating the correlation.
\begin{Example}{$ t $-distribution}{}
    If $ X_1,\ldots,X_n \stackrel{\text{iid}}{\sim}\N{\mu,\sigma^2} $,
    then
    \[ \frac{\bar{X}-\mu}{S/\sqrt{n}}\sim t(n-1)  \]
    where
    \[ S^2=\frac{1}{n-1} \sum_{i=1}^{n} (X_i-\bar{X})^2 \]
    is defined as the sample variance ($ \E{S^2}=\sigma^2 $).

    \textbf{Solution.}
    \[ \frac{\bar{X}-\mu}{\sigma/n}\sim \N{0,1}  \]
    \[ \frac{(n-1)S^2}{\sigma^2}=\frac{\sum_{i=1}^{n} (X_i-\bar{X})^2}{\sigma^2}
        \sim \chi^2(n-1)   \]
    are independent, then
    \[ \frac{\displaystyle \frac{\bar{X}-\mu}{\sigma/\sqrt{n}}}{
            \displaystyle \sqrt{\frac{(n-1)S^2}{\sigma^2}/(n-1)}
        } =\frac{\bar{X}-\mu}{S/\sqrt{n}} \sim t(n-1) \]
\end{Example}
\begin{Example}{$ F $-distribution}{}
    If $ X_1,\ldots,X_n \stackrel{\text{iid}}{\sim}\N{\mu_1,\sigma_1^2} $
    and $ Y_1,\ldots,Y_m \stackrel{\text{iid}}{\sim}\N{\mu_2,\sigma_2^2} $
    are independent. Define
    \[ S_1^2=\frac{\sum_{i=1}^{n} (X_i-\bar{X})^2}{n-1},\quad
        \bar{X}=\frac{1}{n} \sum_{i=1}^{n} X_i \]
    \[ S_2^2=\frac{\sum_{i=1}^{m} (Y_i-\bar{Y})^2}{m-1},\quad
        \bar{Y}=\frac{1}{m} \sum_{i=1}^{m} Y_i \]
    Then,
    \[ \frac{S_1^2/\sigma_1^2}{S_2^2/\sigma_2^2} \sim F(n-1,m-1) \]
    Reasoning:
    \[ \frac{S_1^2}{\sigma_1^2}=\frac{\displaystyle \frac{\sum_{i=1}^{n} (X_i-\bar{X})^2}{\sigma_1^2}}{
            n-1
        }\sim \frac{\chi^2(n-1)}{n-1}   \]
    \[ \frac{S_2^2}{\sigma_2^2}\sim \frac{\chi^2(m-1)}{m-1}  \]
    are independent, therefore,
    \[ \frac{S_1^2/\sigma_1^2}{S_2^2/\sigma_2^2}\sim
        \frac{\chi^2(n-1)/(n-1)}{\chi^2(m-1)/(m-1)}=F(n-1,m-1)  \]
\end{Example}


\subsection{R Demo}
\begin{knitrout}
\definecolor{shadecolor}{rgb}{0.969, 0.969, 0.969}\color{fgcolor}\begin{kframe}
\begin{alltt}
\hlkwd{library}\hlstd{(MASS)}
\hlcom{## Demo for transformations and interactions}
\hlcom{## Florida oranges revisited}
\hlstd{dat} \hlkwb{<-} \hlkwd{read.csv}\hlstd{(}\hlstr{"csv/florange.csv"}\hlstd{)}
\hlstd{lm.1} \hlkwb{<-} \hlkwd{lm}\hlstd{(dat}\hlopt{$}\hlstd{boxes} \hlopt{~} \hlstd{dat}\hlopt{$}\hlstd{acres)}
\hlkwd{summary}\hlstd{(lm.1)}
\end{alltt}
\begin{verbatim}
## 
## Call:
## lm(formula = dat$boxes ~ dat$acres)
## 
## Residuals:
##      Min       1Q   Median       3Q      Max 
## -2470.81    -6.17    71.72   106.46  1677.32 
## 
## Coefficients:
##               Estimate Std. Error t value Pr(>|t|)    
## (Intercept) -85.391989 186.178031  -0.459    0.651    
## dat$acres     0.116717   0.006761  17.263 1.16e-14 ***
## ---
## Signif. codes:  0 '***' 0.001 '**' 0.01 '*' 0.05 '.' 0.1 ' ' 1
## 
## Residual standard error: 754.4 on 23 degrees of freedom
## Multiple R-squared:  0.9284,	Adjusted R-squared:  0.9252 
## F-statistic:   298 on 1 and 23 DF,  p-value: 1.164e-14
\end{verbatim}
\begin{alltt}
\hlcom{# Recall: residuals had non-constant variance (variance}
\hlcom{# increases with fitted values)}
\hlkwd{plot}\hlstd{(lm.1}\hlopt{$}\hlstd{fitted.values, lm.1}\hlopt{$}\hlstd{residuals,} \hlkwc{xlab} \hlstd{=} \hlstr{"Fitted Values"}\hlstd{,}
  \hlkwc{ylab} \hlstd{=} \hlstr{"Residuals"}\hlstd{)}
\end{alltt}
\end{kframe}

{\centering \includegraphics[width=\maxwidth]{figure/unnamed-chunk-108-1} 

}


\begin{kframe}\begin{alltt}
\hlkwd{qqnorm}\hlstd{(lm.1}\hlopt{$}\hlstd{residuals)}
\hlkwd{qqline}\hlstd{(lm.1}\hlopt{$}\hlstd{residuals,} \hlkwc{col} \hlstd{=} \hlstr{"blue"}\hlstd{,} \hlkwc{lwd} \hlstd{=} \hlnum{2}\hlstd{)}
\end{alltt}
\end{kframe}

{\centering \includegraphics[width=\maxwidth]{figure/unnamed-chunk-108-2} 

}


\begin{kframe}\begin{alltt}
\hlcom{# Try log-transforming y}
\hlstd{lm.log} \hlkwb{<-} \hlkwd{lm}\hlstd{(}\hlkwd{log}\hlstd{(dat}\hlopt{$}\hlstd{boxes)} \hlopt{~} \hlstd{dat}\hlopt{$}\hlstd{acres)}
\hlkwd{summary}\hlstd{(lm.log)}
\end{alltt}
\begin{verbatim}
## 
## Call:
## lm(formula = log(dat$boxes) ~ dat$acres)
## 
## Residuals:
##     Min      1Q  Median      3Q     Max 
## -2.0175 -0.7767  0.1142  0.7106  1.6102 
## 
## Coefficients:
##              Estimate Std. Error t value Pr(>|t|)    
## (Intercept) 5.093e+00  2.425e-01  20.997  < 2e-16 ***
## dat$acres   6.748e-05  8.808e-06   7.661 8.95e-08 ***
## ---
## Signif. codes:  0 '***' 0.001 '**' 0.01 '*' 0.05 '.' 0.1 ' ' 1
## 
## Residual standard error: 0.9828 on 23 degrees of freedom
## Multiple R-squared:  0.7184,	Adjusted R-squared:  0.7062 
## F-statistic: 58.69 on 1 and 23 DF,  p-value: 8.948e-08
\end{verbatim}
\begin{alltt}
\hlkwd{plot}\hlstd{(lm.log}\hlopt{$}\hlstd{fitted.values, lm.log}\hlopt{$}\hlstd{residuals,} \hlkwc{xlab} \hlstd{=} \hlstr{"Fitted Values"}\hlstd{,}
  \hlkwc{ylab} \hlstd{=} \hlstr{"Residuals"}\hlstd{)}
\end{alltt}
\end{kframe}

{\centering \includegraphics[width=\maxwidth]{figure/unnamed-chunk-108-3} 

}


\begin{kframe}\begin{alltt}
\hlkwd{plot}\hlstd{(dat}\hlopt{$}\hlstd{acres, lm.log}\hlopt{$}\hlstd{residuals,} \hlkwc{xlab} \hlstd{=} \hlstr{"Fitted Values"}\hlstd{,} \hlkwc{ylab} \hlstd{=} \hlstr{"Residuals"}\hlstd{)}
\end{alltt}
\end{kframe}

{\centering \includegraphics[width=\maxwidth]{figure/unnamed-chunk-108-4} 

}


\begin{kframe}\begin{alltt}
\hlkwd{qqnorm}\hlstd{(lm.log}\hlopt{$}\hlstd{residuals)}
\hlkwd{qqline}\hlstd{(lm.log}\hlopt{$}\hlstd{residuals,} \hlkwc{col} \hlstd{=} \hlstr{"blue"}\hlstd{,} \hlkwc{lwd} \hlstd{=} \hlnum{2}\hlstd{)}
\end{alltt}
\end{kframe}

{\centering \includegraphics[width=\maxwidth]{figure/unnamed-chunk-108-5} 

}


\begin{kframe}\begin{alltt}
\hlcom{# Does the plot of residuals vs x suggest a problem Let's}
\hlcom{# take a closer look}
\hlkwd{plot}\hlstd{(dat}\hlopt{$}\hlstd{acres,} \hlkwd{log}\hlstd{(dat}\hlopt{$}\hlstd{boxes))}  \hlcom{# evidently not linear!}
\end{alltt}
\end{kframe}

{\centering \includegraphics[width=\maxwidth]{figure/unnamed-chunk-108-6} 

}


\begin{kframe}\begin{alltt}
\hlcom{# Log-transform x as well}
\hlkwd{plot}\hlstd{(}\hlkwd{log}\hlstd{(dat}\hlopt{$}\hlstd{acres),} \hlkwd{log}\hlstd{(dat}\hlopt{$}\hlstd{boxes))}  \hlcom{# looks much more linear!}
\end{alltt}
\end{kframe}

{\centering \includegraphics[width=\maxwidth]{figure/unnamed-chunk-108-7} 

}


\begin{kframe}\begin{alltt}
\hlstd{lm.loglog} \hlkwb{<-} \hlkwd{lm}\hlstd{(}\hlkwd{log}\hlstd{(dat}\hlopt{$}\hlstd{boxes)} \hlopt{~} \hlkwd{log}\hlstd{(dat}\hlopt{$}\hlstd{acres))}
\hlkwd{qqnorm}\hlstd{(lm.loglog}\hlopt{$}\hlstd{residuals)}
\hlkwd{qqline}\hlstd{(lm.loglog}\hlopt{$}\hlstd{residuals,} \hlkwc{col} \hlstd{=} \hlstr{"blue"}\hlstd{,} \hlkwc{lwd} \hlstd{=} \hlnum{2}\hlstd{)}
\end{alltt}
\end{kframe}

{\centering \includegraphics[width=\maxwidth]{figure/unnamed-chunk-108-8} 

}


\begin{kframe}\begin{alltt}
\hlkwd{plot}\hlstd{(lm.loglog}\hlopt{$}\hlstd{fitted.values, lm.loglog}\hlopt{$}\hlstd{residuals,} \hlkwc{xlab} \hlstd{=} \hlstr{"Fitted Values"}\hlstd{,}
  \hlkwc{ylab} \hlstd{=} \hlstr{"Residuals"}\hlstd{)}
\end{alltt}
\end{kframe}

{\centering \includegraphics[width=\maxwidth]{figure/unnamed-chunk-108-9} 

}


\begin{kframe}\begin{alltt}
\hlkwd{plot}\hlstd{(}\hlkwd{log}\hlstd{(dat}\hlopt{$}\hlstd{acres), lm.loglog}\hlopt{$}\hlstd{residuals,} \hlkwc{xlab} \hlstd{=} \hlstr{"Fitted Values"}\hlstd{,}
  \hlkwc{ylab} \hlstd{=} \hlstr{"Residuals"}\hlstd{)}
\end{alltt}
\end{kframe}

{\centering \includegraphics[width=\maxwidth]{figure/unnamed-chunk-108-10} 

}


\begin{kframe}\begin{alltt}
\hlcom{## Python data revisited}
\hlstd{python} \hlkwb{<-} \hlkwd{read.csv}\hlstd{(}\hlstr{"csv/FLpython.csv"}\hlstd{)}
\hlstd{python}\hlopt{$}\hlstd{male} \hlkwb{<-} \hlkwd{ifelse}\hlstd{(python}\hlopt{$}\hlstd{sex} \hlopt{==} \hlstr{"M"}\hlstd{,} \hlnum{1}\hlstd{,} \hlnum{0}\hlstd{)}  \hlcom{# 1 = M, 0 =F}
\hlstd{mpf2} \hlkwb{<-} \hlkwd{lm}\hlstd{(fat} \hlopt{~} \hlstd{male} \hlopt{+} \hlstd{mass} \hlopt{+} \hlstd{svl,} \hlkwc{data} \hlstd{= python)}
\hlkwd{summary}\hlstd{(mpf2)}
\end{alltt}
\begin{verbatim}
## 
## Call:
## lm(formula = fat ~ male + mass + svl, data = python)
## 
## Residuals:
##      Min       1Q   Median       3Q      Max 
## -2444.44  -137.38    -6.66   109.22  1530.81 
## 
## Coefficients:
##               Estimate Std. Error t value Pr(>|t|)    
## (Intercept)  204.09840  132.30121   1.543   0.1242    
## male        -196.71705   47.16396  -4.171 4.22e-05 ***
## mass           0.11788    0.00524  22.495  < 2e-16 ***
## svl           -1.59841    0.76433  -2.091   0.0375 *  
## ---
## Signif. codes:  0 '***' 0.001 '**' 0.01 '*' 0.05 '.' 0.1 ' ' 1
## 
## Residual standard error: 360.2 on 244 degrees of freedom
## Multiple R-squared:  0.897,	Adjusted R-squared:  0.8957 
## F-statistic: 708.2 on 3 and 244 DF,  p-value: < 2.2e-16
\end{verbatim}
\begin{alltt}
\hlcom{# Residual plot: vs fitted values}
\hlkwd{plot}\hlstd{(mpf2}\hlopt{$}\hlstd{fitted.values, mpf2}\hlopt{$}\hlstd{residuals,} \hlkwc{xlab} \hlstd{=} \hlstr{"Fitted Values"}\hlstd{,}
  \hlkwc{ylab} \hlstd{=} \hlstr{"Residuals"}\hlstd{)}
\end{alltt}
\end{kframe}

{\centering \includegraphics[width=\maxwidth]{figure/unnamed-chunk-108-11} 

}


\begin{kframe}\begin{alltt}
\hlcom{## QQ plot of residuals}
\hlkwd{qqnorm}\hlstd{(mpf2}\hlopt{$}\hlstd{residuals)}
\hlkwd{qqline}\hlstd{(mpf2}\hlopt{$}\hlstd{residuals,} \hlkwc{col} \hlstd{=} \hlstr{"blue"}\hlstd{,} \hlkwc{lwd} \hlstd{=} \hlnum{2}\hlstd{)}
\end{alltt}
\end{kframe}

{\centering \includegraphics[width=\maxwidth]{figure/unnamed-chunk-108-12} 

}


\begin{kframe}\begin{alltt}
\hlcom{# Try a Box-Cox transformation}
\hlstd{bc} \hlkwb{<-} \hlkwd{boxcox}\hlstd{(mpf2)}
\end{alltt}
\end{kframe}

{\centering \includegraphics[width=\maxwidth]{figure/unnamed-chunk-108-13} 

}


\begin{kframe}\begin{alltt}
\hlstd{lambda} \hlkwb{<-} \hlstd{bc}\hlopt{$}\hlstd{x[}\hlkwd{which.max}\hlstd{(bc}\hlopt{$}\hlstd{y)]}
\hlstd{mpf3} \hlkwb{<-} \hlkwd{lm}\hlstd{((fat}\hlopt{^}\hlstd{lambda} \hlopt{-} \hlnum{1}\hlstd{)}\hlopt{/}\hlstd{lambda} \hlopt{~} \hlstd{male} \hlopt{+} \hlstd{mass} \hlopt{+} \hlstd{svl,} \hlkwc{data} \hlstd{= python)}
\hlkwd{summary}\hlstd{(mpf3)}
\end{alltt}
\begin{verbatim}
## 
## Call:
## lm(formula = (fat^lambda - 1)/lambda ~ male + mass + svl, data = python)
## 
## Residuals:
##     Min      1Q  Median      3Q     Max 
## -19.146  -2.910   0.297   3.688  15.568 
## 
## Coefficients:
##               Estimate Std. Error t value Pr(>|t|)    
## (Intercept) -8.0558134  2.1813183  -3.693 0.000273 ***
## male        -1.7849310  0.7776166  -2.295 0.022560 *  
## mass         0.0004461  0.0000864   5.164 5.03e-07 ***
## svl          0.1431492  0.0126019  11.359  < 2e-16 ***
## ---
## Signif. codes:  0 '***' 0.001 '**' 0.01 '*' 0.05 '.' 0.1 ' ' 1
## 
## Residual standard error: 5.939 on 244 degrees of freedom
## Multiple R-squared:  0.8356,	Adjusted R-squared:  0.8336 
## F-statistic: 413.5 on 3 and 244 DF,  p-value: < 2.2e-16
\end{verbatim}
\begin{alltt}
\hlkwd{plot}\hlstd{(mpf3}\hlopt{$}\hlstd{fitted.values, mpf3}\hlopt{$}\hlstd{residuals)}
\end{alltt}
\end{kframe}

{\centering \includegraphics[width=\maxwidth]{figure/unnamed-chunk-108-14} 

}


\begin{kframe}\begin{alltt}
\hlkwd{plot}\hlstd{(python}\hlopt{$}\hlstd{mass, mpf3}\hlopt{$}\hlstd{residuals)}
\end{alltt}
\end{kframe}

{\centering \includegraphics[width=\maxwidth]{figure/unnamed-chunk-108-15} 

}


\begin{kframe}\begin{alltt}
\hlkwd{plot}\hlstd{(python}\hlopt{$}\hlstd{svl, mpf3}\hlopt{$}\hlstd{residuals)}
\end{alltt}
\end{kframe}

{\centering \includegraphics[width=\maxwidth]{figure/unnamed-chunk-108-16} 

}


\begin{kframe}\begin{alltt}
\hlkwd{qqnorm}\hlstd{(mpf3}\hlopt{$}\hlstd{residuals)}
\hlkwd{qqline}\hlstd{(mpf3}\hlopt{$}\hlstd{residuals,} \hlkwc{col} \hlstd{=} \hlstr{"blue"}\hlstd{,} \hlkwc{lwd} \hlstd{=} \hlnum{2}\hlstd{)}
\end{alltt}
\end{kframe}

{\centering \includegraphics[width=\maxwidth]{figure/unnamed-chunk-108-17} 

}


\begin{kframe}\begin{alltt}
\hlcom{# still some skew, but better!}
\end{alltt}
\end{kframe}
\end{knitrout}
\section{Lecture 17*}
\subsection{Summary}
Today we looked at what happens when we replace the relative frequency in the sample mean with a theoretical probability. We get the expected value of $X$ (or theoretical mean) given by: $E[X] = \sum x f(x)$ (where the sum is over all $x$ in the range of $X$.) 

For our MLIW, we noticed that the sample mean in the class was quite different from the theoretical mean, when $X$ was the number of kids in a family. There could be many reasons for this, but likely the most important is that the sample in the class was not representative of the population of Canada, since that includes many young families with one child that may have more, whereas most of the people in the class will not be gaining any new siblings. Any time you're building a machine learning algorithm, it's only as good as the data you build it on. So if the data is biased and does not reflect reality, the predictions from the model will be biased as well. An important concept in machine learning is data stewardship - making sure the data going in is accurate, representative, and appropriate for the purpose of the model.

In addition to the formal definition of the expected value of $X$, we may be interested in a function of X, so we also defined the expected value of a function $g(X)$ to be $E[g(X)]$ = $\sum g(x) f(x)$. Expectation is a linear operator so we can split up sums and pull out constants (i.e. $E[aX+b] = aE[X] + b$) but for a general non-linear function, unfortunately $E[g(X)]\neq g(E[X])$.

Then we looked at some applications of expectation, including caching and Roulette. I encourage you to read the other applications in section 7.3 for some more examples.
\begin{itemize}
    \item If you're interested, see if you can determine how small the probability of a cache hit would have to be (in our example) in order for it not to be worth it to use a cache. Post the answer in the follow-up if you get it.
    \item In Roulette, a game where there are 38 sections that can be chosen with equal probability, and you can bet on lots of different outcomes. It turns out that no matter what betting strategy you use or how you split up your money, the expected payoff from any \$1 bet is always 0.94737, so you essentially lose about 5.3 cents every time you play! (Over Reading Week, I encourage you to imagine a betting strategy and verify this fact - but I do not encourage actually gambling!)
    \item Of course, different betting strategies will have different amounts of risk, even if the expected value is the same. This is the idea of Variance, which we'll start talking about on Monday after Reading Week. :)
\end{itemize}

Have a fantastic Reading Week! I recommend setting realistic goals for yourself (including both some dedicated time to relax and dedicated time to catch up / get ahead on school work) and have both a productive and fun week!

\subsection{Expectation of a Random Variable (7.2)}
Imagine we know the theoretical probability of each
\# of kids in a family.

\begin{tabular}{| *{6}{>{\centering\arraybackslash}p{1cm} |}}
    \hline
    $x$ & 1 & 2 & 3 & 4 & 5\\
    \hline
    $f(x)$ & 0.43 & 0.4 & 0.12 & 0.04 & 0.01\\
    \hline
\end{tabular}

Now we replace the observed proportion in the sample mean
with $ f(x) $
\[ \sum\limits_{\text{all } x} x f(x)=(1)(0.43)+(2)(0.4)+
(3)(0.12)+(4)(0.04)+(5)(0.01)=1.8 \]
which is the theoretical mean.

Why do we have sample mean > theoretical mean?
\begin{itemize}
    \item urban vs rural population
    \item income level
    \item sampled max family size but theoretical includes growing families
    \item selection bias (if you randomly select people rather than families, people with lots of siblings will be over-represented
\end{itemize}

\begin{defbox}
    \subsubsection{Definition (Expected Value)}
    Let $X$ be a discrete random variable and probability function $f(x)$. The
    \emph{expected value} (also called the mean or the expectation) of $X$ is
    given by
    \[ \mu=E[X]=\sum\limits_{\text{all } x}xf(x) \]
\end{defbox}
\begin{remark}
    $ \mu $ will be within the range but not necessarily
    equal to a possible value of $ x $.

    We might be interested in the expected value of
    some function of $ X $, $ g(X) $.
\end{remark}

\subsection{Example}
Tax credit of $ \$ 1000 $ plus $ \$250 $ per kid. Find the
average cost.

\begin{tabular}{| *{6}{>{\centering\arraybackslash}p{1cm} |}}
    \hline
    $x$ & 1 & 2 & 3 & 4 & 5\\
    \hline
    $g(x)$ & 1250 & 1500 & 1750 & 2000 & 2250\\
    \hline
\end{tabular}

Average cost = weighted average of $ g(x) $ values
$ =(1250)(0.43)+\cdots+(2250)(0.01)=1450 $


\begin{thmbox}
    \subsubsection{Theorem}
    Let $X$ be a discrete random variable and probability function $f(x)$. The
    expected value of a some function $ g(X) $ of $ X $ is given by
    \[ E[g(X)]=\sum\limits_{\text{all } x} g(x)f(x) \]
\end{thmbox}

Note that $ g(x)=1000+250x$ from last example.
\[ E[g(X)]=1000+250E[X]=1450 \]

What if tax credit = $ \frac{2000}{x} $
\[ E[g(X)]=(2000)(0.43)+(1000)(0.40)+\cdots+(400)(0.01)=1364 \]
But $ \frac{2000}{E[X]}=\frac{2000}{1.8}=1111.11 $. Therefore
\[ E[g(X)]\neq g(E[X]) \]
unless $ g $ is a linear function. That is, if $ g(X)=aX+b $, then
$ E[g(X)]=aE[X]+b $

\subsection{Example}
A web server has a cache. Takes 10ms to check, $ 20 $\% of the requests are
found (cache hit) and immediately shown. If it's not found (cache miss),
it takes $ \underbrace{50}_{\text{to server}}+\underbrace{70}_{\text{lookup}}
+\underbrace{50}_{\text{to client}} $ additional milliseconds to get info and display.
Is it worth it? Let $ X= $ \# of milliseconds to display the information.

\begin{tabular}{| *{3}{>{\centering\arraybackslash}p{4cm} |}}
    \hline
    $x$ & 10 & 10+50+70+50=180\\
    \hline
    $f(x)$ & 0.2 & 0.8\\
    \hline
\end{tabular}
\[ E[X]=(10)(0.2)+(180)(0.8)=146\text{ms} \]
Time no cache = $ 50+70+50=170\text{ms} $.

Since $ 146\text{ms}<170\text{ms, it's worth it!} $

\subsection{Example}
Roulette: each of $38$ numbers is equally likely

(1) If you bet $1$ dollar on number $7$ $ \rightarrow $ pays $ 35:1 $

OR

(2) If you bet $50$ cents on red $ \rightarrow $ pays $ 1:1 $
and $ 50 $ cents on first $ 12 $ numbers $ \rightarrow $ pays $ 2:1 $

(1)
\begin{tabular}{| *{3}{>{\centering\arraybackslash}p{1cm} |}}
    \hline
    $x$ & $0$ & $36$\\
    \hline
    $f(x)$ & $\nicefrac{37}{38}$ & $\nicefrac{1}{38}$ \\
    \hline
\end{tabular}

(2)
\begin{tabular}{| *{5}{>{\centering\arraybackslash}p{2.5cm} |}}
    \hline
    $y$ & 0 & 1 & 1.50 & 2.50\\
    \hline
    $f(y)$ & $ \underbrace{\nicefrac{14}{38}}_{\text{neither}} $ & $\underbrace{\nicefrac{12}{38}}_{\text{red}}$ & $\underbrace{\nicefrac{6}{38}}_{\text{black}}$ & $\underbrace{\nicefrac{6}{38}}_{\text{both red}}$ \\
    \hline
\end{tabular}

\[ E[X]=0(\frac{37}{38})+36(\frac{1}{38})=0.94737 \]
\[ E[Y]=0(\frac{14}{38})+1(\frac{12}{38})+1.5(\frac{6}{38})+2.5(\frac{6}{38})=0.94737\]

\makeheading{Lecture 18}
\section{Means and Variances of Distributions}

The mean $ E[X] $ tells us where the distribution is on average. We
also need a way to describe how spread out a distribution is. Variance could
be $ E[X-\mu]=0 $.

What about $ E[|X-\mu|] $
\begin{itemize}
    \item need cases to evaluate
    \item non-differentiable at point $ X-\mu $
    \item linear penalty for being away from the mean
\end{itemize}
Instead we use $ E[(X-\mu)^2] $

\begin{defbox}
    \subsection{Definition (Variance)}
    The \emph{variance} of a random variable $X$, denoted by $Var(X)$ or by
    $ \sigma^2 $, is
    \[ \sigma^2=Var(X)=E\left[(X-\mu)^2\right] \]
\end{defbox}

\textbf{Example}

$ X= $ \# on fair 6-sided die

$ E[X]=3.5 $

$ E[(x-3.5)^2] $

$ E[X]^2-3.5^2 $

\begin{tabular}{| *{7}{>{\centering\arraybackslash}p{1cm} |}}
    \hline
    $x$   & 1 & 2 & 3 & 4  & 5  & 6  \\
    \hline
    $x^2$ & 1 & 4 & 9 & 16 & 25 & 36 \\
    \hline
\end{tabular}

Alternate form (calculation form)
\begin{align*}
    Var(X) & =E[(X-E[X])^2]                                                                       \\
           & =E[X^2-2XE[X]+E[X]^2]                                                                \\
           & =E[X^2]-2E[X]E[X]+E[X]^2                                                             \\
           & =E[X^2]-2(E[X])^2+E[X]^2                                                             \\
           & =E[X^2]-E[X]^2                                                                       \\
           & =\sum\limits_{\text{all }x}x^2 f(x)-\left(\sum\limits_{\text{all } x}x f(x)\right)^2
\end{align*}

\subsection{Example (Roulette)}

$ X=0 $ or $ 36 $ (dollars)

\begin{tabular}{| *{3}{>{\centering\arraybackslash}p{1cm} |}}
    \hline
    $x$    & $0$              & $36$            \\
    \hline
    $f(x)$ & $\sfrac{37}{38}$ & $\sfrac{1}{38}$ \\
    \hline
\end{tabular}

\[ E[X]=0.94737 \]
\[ Var(X)=E[X^2]-0.94737^2=36^2(\frac{1}{36})-0.94737^2=33.207\text{ dollars}^2 \]
To interpret the variance better, we often take the square root to get the same
units of the original variable.

\begin{defbox}
    \subsection{Definition (Standard Deviation)}
    The \emph{standard deviation} of a random variable $X$ is
    \[ \sigma=SD(X)=\sqrt{Var(X)} \]
\end{defbox}
\[ SD(X)=\sqrt{33.207}=5.76 \]

What if we bet $ \$1 $ on red. Y=winnings

\begin{tabular}{| *{3}{>{\centering\arraybackslash}p{1cm} |}}
    \hline
    $y$    & $0$              & $2$              \\
    \hline
    $f(y)$ & $\sfrac{20}{38}$ & $\sfrac{18}{38}$ \\
    \hline
\end{tabular}
\[ E[Y]=0.94737 \]
\[ Var(Y)=E[Y^2]-0.94737^2=0.97723 \]
\[ SD(Y)=0.9986 \]

\subsection{Linear Transformations}
If $ Y=aX+b $, and we know $ E[X] $ and $ Var(X) $, what
can we say about $ E[Y] $ and $ Var(Y) $.
\[ E[Y]=aE[X]+b \]
\begin{align*}
    Var(Y) & =E[(Y-E[Y])^2]           \\
           & =E[(aX+b)-(aE[X]+b)^2]   \\
           & =E[a^2X^2-2XE[X]+E[X]^2] \\
           & =a^2E[(X-E[X])^2]
\end{align*}
\[ Var(Y)=a^2Var(X) \]
\[ SD(Y)=|a|SD(X) \]


\subsection{R Demo}
\begin{knitrout}
\definecolor{shadecolor}{rgb}{0.969, 0.969, 0.969}\color{fgcolor}\begin{kframe}
\begin{alltt}
\hlcom{## Effect of individual observations}
\hlcom{## Python data revisited}
\hlstd{python} \hlkwb{<-} \hlkwd{read.csv}\hlstd{(}\hlstr{"csv/FLpython.csv"}\hlstd{)}
\hlstd{python}\hlopt{$}\hlstd{male} \hlkwb{<-} \hlkwd{ifelse}\hlstd{(python}\hlopt{$}\hlstd{sex} \hlopt{==} \hlstr{"M"}\hlstd{,} \hlnum{1}\hlstd{,} \hlnum{0}\hlstd{)}  \hlcom{# 1 = M, 0 =F}
\hlstd{mpf2} \hlkwb{<-} \hlkwd{lm}\hlstd{(fat} \hlopt{~} \hlstd{male} \hlopt{+} \hlstd{mass} \hlopt{+} \hlstd{svl,} \hlkwc{data} \hlstd{= python)}
\hlcom{# Last time we used a Box-Cox transformation}
\hlkwd{library}\hlstd{(MASS)}
\hlstd{bc} \hlkwb{<-} \hlkwd{boxcox}\hlstd{(mpf2)}
\end{alltt}
\end{kframe}

{\centering \includegraphics[width=\maxwidth]{figure/unnamed-chunk-111-1} 

}


\begin{kframe}\begin{alltt}
\hlstd{lambda} \hlkwb{<-} \hlstd{bc}\hlopt{$}\hlstd{x[}\hlkwd{which.max}\hlstd{(bc}\hlopt{$}\hlstd{y)]}
\hlstd{mpf3} \hlkwb{<-} \hlkwd{lm}\hlstd{((fat}\hlopt{^}\hlstd{lambda} \hlopt{-} \hlnum{1}\hlstd{)}\hlopt{/}\hlstd{lambda} \hlopt{~} \hlstd{male} \hlopt{+} \hlstd{mass} \hlopt{+} \hlstd{svl,} \hlkwc{data} \hlstd{= python)}
\hlkwd{summary}\hlstd{(mpf3)}
\end{alltt}
\begin{verbatim}
## 
## Call:
## lm(formula = (fat^lambda - 1)/lambda ~ male + mass + svl, data = python)
## 
## Residuals:
##     Min      1Q  Median      3Q     Max 
## -19.146  -2.910   0.297   3.688  15.568 
## 
## Coefficients:
##               Estimate Std. Error t value Pr(>|t|)    
## (Intercept) -8.0558134  2.1813183  -3.693 0.000273 ***
## male        -1.7849310  0.7776166  -2.295 0.022560 *  
## mass         0.0004461  0.0000864   5.164 5.03e-07 ***
## svl          0.1431492  0.0126019  11.359  < 2e-16 ***
## ---
## Signif. codes:  0 '***' 0.001 '**' 0.01 '*' 0.05 '.' 0.1 ' ' 1
## 
## Residual standard error: 5.939 on 244 degrees of freedom
## Multiple R-squared:  0.8356,	Adjusted R-squared:  0.8336 
## F-statistic: 413.5 on 3 and 244 DF,  p-value: < 2.2e-16
\end{verbatim}
\begin{alltt}
\hlkwd{plot}\hlstd{(mpf3}\hlopt{$}\hlstd{fitted.values, mpf3}\hlopt{$}\hlstd{residuals)}
\end{alltt}
\end{kframe}

{\centering \includegraphics[width=\maxwidth]{figure/unnamed-chunk-111-2} 

}


\begin{kframe}\begin{alltt}
\hlkwd{qqnorm}\hlstd{(mpf3}\hlopt{$}\hlstd{residuals)}
\hlkwd{qqline}\hlstd{(mpf3}\hlopt{$}\hlstd{residuals,} \hlkwc{col} \hlstd{=} \hlstr{"blue"}\hlstd{,} \hlkwc{lwd} \hlstd{=} \hlnum{2}\hlstd{)}
\end{alltt}
\end{kframe}

{\centering \includegraphics[width=\maxwidth]{figure/unnamed-chunk-111-3} 

}


\end{knitrout}
\begin{knitrout}
\definecolor{shadecolor}{rgb}{0.969, 0.969, 0.969}\color{fgcolor}\begin{kframe}
\begin{alltt}
\hlcom{# Quantities for individual observations}
\hlkwd{studres}\hlstd{(mpf3)}  \hlcom{# studentized residuals}
\hlkwd{hatvalues}\hlstd{(mpf3)}  \hlcom{# leverage}
\hlkwd{cooks.distance}\hlstd{(mpf3)}  \hlcom{# Cook's distance}
\end{alltt}
\end{kframe}
\end{knitrout}
\begin{knitrout}
\definecolor{shadecolor}{rgb}{0.969, 0.969, 0.969}\color{fgcolor}\begin{kframe}
\begin{alltt}
\hlcom{# Residual plots with studentized residuals}
\hlkwd{plot}\hlstd{(mpf3}\hlopt{$}\hlstd{fitted.values,} \hlkwd{studres}\hlstd{(mpf3),} \hlkwc{xlab} \hlstd{=} \hlstr{"Fitted values"}\hlstd{,}
  \hlkwc{ylab} \hlstd{=} \hlstr{"Studentized residuals"}\hlstd{)}
\hlkwd{abline}\hlstd{(}\hlkwc{h} \hlstd{=} \hlkwd{c}\hlstd{(}\hlnum{3}\hlstd{,} \hlopt{-}\hlnum{3}\hlstd{),} \hlkwc{col} \hlstd{=} \hlstr{"red"}\hlstd{,} \hlkwc{lty} \hlstd{=} \hlnum{2}\hlstd{)}
\end{alltt}
\end{kframe}

{\centering \includegraphics[width=\maxwidth]{figure/unnamed-chunk-113-1} 

}


\begin{kframe}\begin{alltt}
\hlkwd{which}\hlstd{(}\hlkwd{abs}\hlstd{(}\hlkwd{studres}\hlstd{(mpf3))} \hlopt{>} \hlnum{3}\hlstd{)}
\end{alltt}
\begin{verbatim}
## 122 181 245 
## 122 181 245
\end{verbatim}
\begin{alltt}
\hlkwd{qqnorm}\hlstd{(}\hlkwd{studres}\hlstd{(mpf3))}
\hlkwd{qqline}\hlstd{(}\hlkwd{studres}\hlstd{(mpf3),} \hlkwc{col} \hlstd{=} \hlstr{"blue"}\hlstd{,} \hlkwc{lwd} \hlstd{=} \hlnum{2}\hlstd{)}
\end{alltt}
\end{kframe}

{\centering \includegraphics[width=\maxwidth]{figure/unnamed-chunk-113-2} 

}


\begin{kframe}\begin{alltt}
\hlcom{# Leverage}
\hlkwd{plot}\hlstd{(}\hlkwd{hatvalues}\hlstd{(mpf3),} \hlkwc{ylab} \hlstd{=} \hlstr{"Leverage"}\hlstd{)}
\hlkwd{abline}\hlstd{(}\hlkwc{h} \hlstd{=} \hlnum{2} \hlopt{*} \hlkwd{mean}\hlstd{(}\hlkwd{hatvalues}\hlstd{(mpf3)),} \hlkwc{col} \hlstd{=} \hlstr{"red"}\hlstd{,} \hlkwc{lty} \hlstd{=} \hlnum{2}\hlstd{)}
\end{alltt}
\end{kframe}

{\centering \includegraphics[width=\maxwidth]{figure/unnamed-chunk-113-3} 

}


\begin{kframe}\begin{alltt}
\hlkwd{which}\hlstd{(}\hlkwd{hatvalues}\hlstd{(mpf3)} \hlopt{>} \hlnum{2} \hlopt{*} \hlkwd{mean}\hlstd{(}\hlkwd{hatvalues}\hlstd{(mpf3)))}
\end{alltt}
\begin{verbatim}
##   1   2   3   4   5   6   8 219 236 241 242 243 244 245 246 247 248 
##   1   2   3   4   5   6   8 219 236 241 242 243 244 245 246 247 248
\end{verbatim}
\begin{alltt}
\hlstd{python[}\hlkwd{which}\hlstd{(}\hlkwd{hatvalues}\hlstd{(mpf3)} \hlopt{>} \hlnum{2} \hlopt{*} \hlkwd{mean}\hlstd{(}\hlkwd{hatvalues}\hlstd{(mpf3))), ]}
\end{alltt}
\begin{verbatim}
##     sex   svl  mass length      fat male
## 1     F  70.0   186   77.5    6.000    0
## 2     M  76.0   310   83.8   11.000    1
## 3     M  77.0   260   86.1    6.000    1
## 4     M  78.0   262   87.1    8.000    1
## 5     M  81.0   306   91.1    4.000    1
## 6     M  93.5   605  104.6   18.959    1
## 8     F 105.0   800  117.5   17.000    0
## 219   M 285.0 27000  316.2 3230.000    1
## 236   M 330.0 32600  370.9 4374.000    1
## 241   F 376.0 38280  424.2 3156.000    0
## 242   F 381.0 43910  424.9 4002.000    0
## 243   F 384.5 34540  432.4 3500.000    0
## 244   F 405.5 41660  455.3 5688.000    0
## 245   F 409.0 49900  460.2 2988.000    0
## 246   F 416.0 55260  469.1 4618.000    0
## 247   F 422.0 49350  473.4 6818.000    0
## 248   F 482.0 75500  545.0 8406.000    0
\end{verbatim}
\begin{alltt}
\hlcom{# Cook's distance}
\hlkwd{plot}\hlstd{(}\hlkwd{cooks.distance}\hlstd{(mpf3),} \hlkwc{ylab} \hlstd{=} \hlstr{"Cook's distance"}\hlstd{)}
\hlkwd{abline}\hlstd{(}\hlkwc{h} \hlstd{=} \hlnum{0.5}\hlstd{,} \hlkwc{col} \hlstd{=} \hlstr{"red"}\hlstd{,} \hlkwc{lty} \hlstd{=} \hlnum{2}\hlstd{)}
\end{alltt}
\end{kframe}

{\centering \includegraphics[width=\maxwidth]{figure/unnamed-chunk-113-4} 

}


\begin{kframe}\begin{alltt}
\hlkwd{which}\hlstd{(}\hlkwd{cooks.distance}\hlstd{(mpf3)} \hlopt{>} \hlnum{0.5}\hlstd{)}
\end{alltt}
\begin{verbatim}
## 248 
## 248
\end{verbatim}
\begin{alltt}
\hlcom{# Let's look at actual changes in beta estimates}
\hlstd{mpf3}\hlopt{$}\hlstd{coefficients}  \hlcom{# with all the data}
\end{alltt}
\begin{verbatim}
##   (Intercept)          male          mass           svl 
## -8.0558134354 -1.7849310101  0.0004461197  0.1431491887
\end{verbatim}
\begin{alltt}
\hlcom{# e.g., fit without obs 248}
\hlstd{mpf4} \hlkwb{<-} \hlkwd{lm}\hlstd{((fat}\hlopt{^}\hlstd{lambda} \hlopt{-} \hlnum{1}\hlstd{)}\hlopt{/}\hlstd{lambda} \hlopt{~} \hlstd{male} \hlopt{+} \hlstd{mass} \hlopt{+} \hlstd{svl,} \hlkwc{data} \hlstd{= python[}\hlopt{-}\hlnum{248}\hlstd{,}
  \hlstd{])}
\hlstd{mpf4}\hlopt{$}\hlstd{coefficients}
\end{alltt}
\begin{verbatim}
##   (Intercept)          male          mass           svl 
## -6.6475616056 -1.6605858218  0.0005743312  0.1313793189
\end{verbatim}
\begin{alltt}
\hlcom{# e.g., fit without obs 50}
\hlstd{mpf5} \hlkwb{<-} \hlkwd{lm}\hlstd{((fat}\hlopt{^}\hlstd{lambda} \hlopt{-} \hlnum{1}\hlstd{)}\hlopt{/}\hlstd{lambda} \hlopt{~} \hlstd{male} \hlopt{+} \hlstd{mass} \hlopt{+} \hlstd{svl,} \hlkwc{data} \hlstd{= python[}\hlopt{-}\hlnum{50}\hlstd{,}
  \hlstd{])}
\hlstd{mpf5}\hlopt{$}\hlstd{coefficients}
\end{alltt}
\begin{verbatim}
##   (Intercept)          male          mass           svl 
## -8.0628754675 -1.7805093651  0.0004462354  0.1431573753
\end{verbatim}
\end{kframe}
\end{knitrout}
\makeheading{2020-02-24}
\section{Ideals and Cyclic Subspaces}
\begin{defbox}
    \begin{definition}
        A \textbf{monic polynomial} $ g(x) $ is a single-variable
        polynomial in which the non-zero coefficient of the highest degree of $ x $
        is $ 1 $. That is,
        \[ g(x)=c_0+\cdots+c_{\ell -1}x^{\ell -1}+ x^\ell \]
        for some constants $ c_i $ where $ i\in[\ell-1,1] $.
    \end{definition}
\end{defbox}
If $ I\neq \{0\} $, then we took $ g(x)=a $ non-zero
polynomial of smallest degree in $ I $. Note, we can
take $ g(x) $ to be monic. If $ g(x) $ is not monic, say
\[ g(x)=c_0+\cdots+c_\ell x^{\ell} \]
where $ c_\ell \neq 0, 1 $, then
\[ c_{\ell}^{-1}g(x)=c_\ell^{-1} g_0+\cdots x^\ell \]
is monic and is also in $ I $. We'll call this process
\textbf{making $\bm{g(x)$} monic}.

\begin{defbox}
    \begin{definition}
        Let $ I $ be an ideal in $ R=F[x]/(x^n-1) $.

        The \textbf{generator polynomial of $ \bm{I} $} is:
        \begin{enumerate}[label=(\arabic*)]
            \item $ x^n-1 $ since $ x^n-1\equiv 0 \mod x^n-1 $ when $ I=\{0\} $.
            \item \textbf{the}
                  monic polynomial of least degree in $ I $ when $ I \neq \{0\} $.
        \end{enumerate}
    \end{definition}
\end{defbox}

\begin{thmbox}
    \begin{theorem}
        Let $ I $ be a non-zero ideal in $ R=F[x]/(x^n-1) $.
        \begin{enumerate}[label=(\arabic*)]
            \item There is a \textbf{unique} monic polynomial
                  g(x) of smallest degree in $ I $.
            \item $ g(x)\mid (x^n-1) $
        \end{enumerate}
    \end{theorem}
\end{thmbox}

\begin{proof}
    (1) Suppose
    there exists two monic polynomials $ g(x) $ and $ h(x) $
    of the same smallest degree in $ I $.
    Then, $ g(x)-h(x)\in I $ and $ \deg(g-h)<\deg (g) $. Hence, we must
    have $ g-h=0 $, so $ g=h $.

    (2) We can write
    \[ x^n-1=\ell(x)g(x)+r(x) \]
    where $ \ell,r\in F[x] $ and $ \deg(r)<\deg(g) $. Then,
    \[ 0\equiv \ell (x)g(x)+r(x)\mod x^n-1\iff r(x)\equiv -\ell(x)g(x)\mod x^n-1 \]
    Since $ \langle g(x)\rangle = I $, we must have $ r(x)\in I $.
    Hence, $ \deg(r)<\deg(g) $ so we must have $ r(x)=0 $. Thus,
    \[ g(x)\mid (x^n-1) \]
\end{proof}

\begin{thmbox}
    \begin{theorem}
        Let $ h(x) $ be a monic divisor of $ x^n-1 $ in $ F[x] $.
        Then, \textbf{the} generator polynomial of $ \langle h(x)\rangle $
        is $ h(x) $.
    \end{theorem}
\end{thmbox}

\begin{proof}
    If $ h(x)=x^n-1 $, then $ I=\{0\} $ and by definition, its
    generator polynomial is $ x^n-1 $.

    If $ \deg(h)<n $, then $ I\neq \{0\} $. Let $ g(x) $
    be \textbf{the} monic polynomial of smallest degree in $ I $.
    Since $ g $ is a generator of $ I $, we can write
    \[ g(x)\equiv a(x)h(x)\mod x^n-1\implies g(x)=a(x)h(x)+\ell(x)(x^n-1) \]
    for some $ \ell\in F[x] $. Since $ h\mid (x^n-1) $, and $ h\mid ah $,
    we have $ h\mid g $. So, $ \deg(h)\leqslant \deg(g) $ since
    $ g $ is a monic polynomial of smallest degree in $ I $,
    we must have $ \deg(g)\leqslant \deg(h) $, so $ \deg(g)=\deg(h) $.
    Since $ g $ and $ h $ are both monic, we have
    $ g=h $.
\end{proof}

\begin{thmbox}
    \begin{corollary}
        There is a 1-1 correspondence between monic
        divisors of $ x^n-1 $ in $ F[x] $ and ideals in $ R $.
        There is a 1-1 correspondence between monic
        divisors of $ x^n-1 $ in $ F[x] $ and cyclic
        subspaces of $ V_n(F) $.
    \end{corollary}
\end{thmbox}

\begin{exbox}
    \begin{example}
        Find all cyclic subspaces of $ V_3(\mathbb{Z}_2) $.

        \textbf{Solution.} The complete factorization
        of $ x^3-1 $ over $ \mathbb{Z}_2 $ is
        \[ x^3-1=(1+x)(1+x+x^2) \]

        \begin{table}[H]
            \centering
            \begin{tabularx}{0.8\linewidth}{@{}YYY@{}}
                Monic divisor of $ x^3-1 $ & $ \langle g_i(x) \rangle $  & $ \dim \langle g_i(x) \rangle $ \\
                \midrule
                \midrule
                $ g_1(x)=1 $               & $ \{000,001,\ldots ,111\} $ & $ 3 $                           \\
                \midrule
                $ g_2(x)=1+x $             & $ \{000,110,001,101\} $     & $ 2 $                           \\
                \midrule
                $ g_3(x)=1+x+x^2 $         & $ \{000,111\} $             & $ 1 $                           \\
                \midrule
                $ g_4(x)=1+x^3 $           & $ \{0\} $                   & $ 0 $                           \\
            \end{tabularx}
        \end{table}
    \end{example}
\end{exbox}


\subsection{R Demo}
\begin{knitrout}
\definecolor{shadecolor}{rgb}{0.969, 0.969, 0.969}\color{fgcolor}\begin{kframe}
\begin{alltt}
\hlcom{## Cross-validation}
\hlcom{## Coffee example (Coffee Quality Institute, 2018)}
\hlcom{## continued}
\hlstd{coffee} \hlkwb{<-} \hlkwd{read.csv}\hlstd{(}\hlstr{"csv/coffee_arabica.csv"}\hlstd{)}
\hlcom{# 1 = wet, 0 otherwise}
\hlstd{coffee}\hlopt{$}\hlstd{wet} \hlkwb{<-} \hlkwd{ifelse}\hlstd{(coffee}\hlopt{$}\hlstd{Processing.Method} \hlopt{==} \hlstr{"Washed / Wet"}\hlstd{,}
  \hlnum{1}\hlstd{,} \hlnum{0}\hlstd{)}
\hlcom{# 1 = semi/dry, 0 otherwise}
\hlstd{coffee}\hlopt{$}\hlstd{semi} \hlkwb{<-} \hlkwd{ifelse}\hlstd{(coffee}\hlopt{$}\hlstd{Processing.Method} \hlopt{==} \hlstr{"Semi-washed / Semi-pulped"}\hlstd{,}
  \hlnum{1}\hlstd{,} \hlnum{0}\hlstd{)}
\hlstd{coffee}\hlopt{$}\hlstd{Processing.Method} \hlkwb{<-} \hlkwa{NULL}
\hlstd{N} \hlkwb{<-} \hlkwd{nrow}\hlstd{(coffee)}
\hlcom{## Train and validation set split}
\hlkwd{set.seed}\hlstd{(}\hlnum{12345678}\hlstd{)}
\hlstd{trainInd} \hlkwb{<-} \hlkwd{sample}\hlstd{(}\hlnum{1}\hlopt{:}\hlstd{N,} \hlkwd{round}\hlstd{(N} \hlopt{*} \hlnum{0.8}\hlstd{),} \hlkwc{replace} \hlstd{= F)}
\hlstd{trainSet} \hlkwb{<-} \hlstd{coffee[trainInd, ]}
\hlstd{validSet} \hlkwb{<-} \hlstd{coffee[}\hlopt{-}\hlstd{trainInd, ]}
\hlcom{# Calculate RMSE on three models each with different}
\hlcom{# variables included}
\hlstd{m1} \hlkwb{<-} \hlkwd{lm}\hlstd{(Flavor} \hlopt{~} \hlstd{wet} \hlopt{+} \hlstd{semi} \hlopt{+} \hlstd{Aroma} \hlopt{+} \hlstd{Aftertaste} \hlopt{+} \hlstd{Body,} \hlkwc{dat} \hlstd{= trainSet)}
\hlstd{pred1} \hlkwb{<-} \hlkwd{predict}\hlstd{(m1,} \hlkwc{newdata} \hlstd{= validSet)}
\hlkwd{sqrt}\hlstd{(}\hlkwd{mean}\hlstd{((validSet}\hlopt{$}\hlstd{Flavor} \hlopt{-} \hlstd{pred1)}\hlopt{^}\hlnum{2}\hlstd{))}  \hlcom{# RMSE}
\end{alltt}
\begin{verbatim}
## [1] 0.1577479
\end{verbatim}
\begin{alltt}
\hlkwd{mean}\hlstd{(}\hlkwd{abs}\hlstd{(validSet}\hlopt{$}\hlstd{Flavor} \hlopt{-} \hlstd{pred1))}  \hlcom{# MAE}
\end{alltt}
\begin{verbatim}
## [1] 0.113643
\end{verbatim}
\begin{alltt}
\hlstd{m2} \hlkwb{<-} \hlkwd{lm}\hlstd{(Flavor} \hlopt{~} \hlstd{wet} \hlopt{+} \hlstd{Aroma} \hlopt{+} \hlstd{Aftertaste} \hlopt{+} \hlstd{Body} \hlopt{+} \hlstd{Acidity} \hlopt{+}
  \hlstd{Balance} \hlopt{+} \hlstd{Sweetness} \hlopt{+} \hlstd{Uniformity} \hlopt{+} \hlstd{Moisture,} \hlkwc{dat} \hlstd{= trainSet)}
\hlstd{pred2} \hlkwb{<-} \hlkwd{predict}\hlstd{(m2,} \hlkwc{newdata} \hlstd{= validSet)}
\hlkwd{sqrt}\hlstd{(}\hlkwd{mean}\hlstd{((validSet}\hlopt{$}\hlstd{Flavor} \hlopt{-} \hlstd{pred2)}\hlopt{^}\hlnum{2}\hlstd{))}
\end{alltt}
\begin{verbatim}
## [1] 0.1426565
\end{verbatim}
\begin{alltt}
\hlstd{m3} \hlkwb{<-} \hlkwd{lm}\hlstd{(Flavor} \hlopt{~} \hlstd{Aroma} \hlopt{+} \hlstd{Aftertaste,} \hlkwc{dat} \hlstd{= trainSet)}
\hlstd{pred3} \hlkwb{<-} \hlkwd{predict}\hlstd{(m3,} \hlkwc{newdata} \hlstd{= validSet)}
\hlkwd{sqrt}\hlstd{(}\hlkwd{mean}\hlstd{((validSet}\hlopt{$}\hlstd{Flavor} \hlopt{-} \hlstd{pred3)}\hlopt{^}\hlnum{2}\hlstd{))}
\end{alltt}
\begin{verbatim}
## [1] 0.1615385
\end{verbatim}
\begin{alltt}
\hlcom{# K fold cross validation}
\hlstd{K} \hlkwb{<-} \hlnum{5}
\hlstd{validSetSplits} \hlkwb{<-} \hlkwd{sample}\hlstd{((}\hlnum{1}\hlopt{:}\hlstd{N)}\hlopt\hlstd{K} \hlopt{+} \hlnum{1}\hlstd{)}
\hlstd{RMSE1} \hlkwb{<-} \hlkwd{c}\hlstd{()}
\hlstd{RMSE2} \hlkwb{<-} \hlkwd{c}\hlstd{()}
\hlstd{RMSE3} \hlkwb{<-} \hlkwd{c}\hlstd{()}
\hlkwa{for} \hlstd{(k} \hlkwa{in} \hlnum{1}\hlopt{:}\hlstd{K) \{}
  \hlstd{validSet} \hlkwb{<-} \hlstd{coffee[validSetSplits} \hlopt{==} \hlstd{k, ]}
  \hlstd{trainSet} \hlkwb{<-} \hlstd{coffee[validSetSplits} \hlopt{!=} \hlstd{k, ]}
  \hlstd{m1} \hlkwb{<-} \hlkwd{lm}\hlstd{(Flavor} \hlopt{~} \hlstd{wet} \hlopt{+} \hlstd{semi} \hlopt{+} \hlstd{Aroma} \hlopt{+} \hlstd{Aftertaste} \hlopt{+} \hlstd{Body,}
    \hlkwc{dat} \hlstd{= trainSet)}
  \hlstd{pred1} \hlkwb{<-} \hlkwd{predict}\hlstd{(m1,} \hlkwc{newdata} \hlstd{= validSet)}
  \hlstd{RMSE1[k]} \hlkwb{<-} \hlkwd{sqrt}\hlstd{(}\hlkwd{mean}\hlstd{((validSet}\hlopt{$}\hlstd{Flavor} \hlopt{-} \hlstd{pred1)}\hlopt{^}\hlnum{2}\hlstd{))}
  \hlstd{m2} \hlkwb{<-} \hlkwd{lm}\hlstd{(Flavor} \hlopt{~} \hlstd{wet} \hlopt{+} \hlstd{Aroma} \hlopt{+} \hlstd{Aftertaste} \hlopt{+} \hlstd{Body} \hlopt{+} \hlstd{Acidity} \hlopt{+}
    \hlstd{Balance} \hlopt{+} \hlstd{Sweetness} \hlopt{+} \hlstd{Uniformity} \hlopt{+} \hlstd{Moisture,} \hlkwc{dat} \hlstd{= trainSet)}
  \hlstd{pred2} \hlkwb{<-} \hlkwd{predict}\hlstd{(m2,} \hlkwc{newdata} \hlstd{= validSet)}
  \hlstd{RMSE2[k]} \hlkwb{<-} \hlkwd{sqrt}\hlstd{(}\hlkwd{mean}\hlstd{((validSet}\hlopt{$}\hlstd{Flavor} \hlopt{-} \hlstd{pred2)}\hlopt{^}\hlnum{2}\hlstd{))}
  \hlstd{m3} \hlkwb{<-} \hlkwd{lm}\hlstd{(Flavor} \hlopt{~} \hlstd{Aroma} \hlopt{+} \hlstd{Aftertaste,} \hlkwc{dat} \hlstd{= trainSet)}
  \hlstd{pred3} \hlkwb{<-} \hlkwd{predict}\hlstd{(m3,} \hlkwc{newdata} \hlstd{= validSet)}
  \hlstd{RMSE3[k]} \hlkwb{<-} \hlkwd{sqrt}\hlstd{(}\hlkwd{mean}\hlstd{((validSet}\hlopt{$}\hlstd{Flavor} \hlopt{-} \hlstd{pred3)}\hlopt{^}\hlnum{2}\hlstd{))}
\hlstd{\}}
\hlstd{RMSE1}
\end{alltt}
\begin{verbatim}
## [1] 0.1479415 0.1653329 0.1556385 0.1656876 0.1482716
\end{verbatim}
\begin{alltt}
\hlstd{RMSE2}
\end{alltt}
\begin{verbatim}
## [1] 0.1427025 0.1525461 0.1478815 0.1620440 0.1384244
\end{verbatim}
\begin{alltt}
\hlstd{RMSE3}
\end{alltt}
\begin{verbatim}
## [1] 0.1513836 0.1667202 0.1616626 0.1675113 0.1532496
\end{verbatim}
\begin{alltt}
\hlkwd{mean}\hlstd{(RMSE1)}
\end{alltt}
\begin{verbatim}
## [1] 0.1565744
\end{verbatim}
\begin{alltt}
\hlkwd{mean}\hlstd{(RMSE2)}
\end{alltt}
\begin{verbatim}
## [1] 0.1487197
\end{verbatim}
\begin{alltt}
\hlkwd{mean}\hlstd{(RMSE3)}
\end{alltt}
\begin{verbatim}
## [1] 0.1601055
\end{verbatim}
\end{kframe}
\end{knitrout}
\makeheading{2020-02-26}
Midterm review session.


\subsection{R Demo}
\begin{knitrout}
\definecolor{shadecolor}{rgb}{0.969, 0.969, 0.969}\color{fgcolor}\begin{kframe}
\begin{alltt}
\hlcom{## Cross-validation with model selection}
\hlcom{# Dataset from paper 'Where does Haydn end and Mozart}
\hlcom{# begin?  Composer classification of string quartets'}
\hlcom{# (Journal of New Music Research, vol 49, 457-476)}
\hlstd{HM} \hlkwb{<-} \hlkwd{read.csv}\hlstd{(}\hlstr{"haydn-mozart.csv"}\hlstd{)}
\hlstd{HM[,} \hlnum{1}\hlstd{]} \hlkwb{<-} \hlkwa{NULL}  \hlcom{# first col is just name of quartet, remove it}
\hlkwd{dim}\hlstd{(HM)}  \hlcom{# 285 observations, 1116 columns}
\hlcom{# Let's treat 'number of notes in violin part' as the}
\hlcom{# response That's variable name 'count_pitch_1' and column}
\hlcom{# 683 of data matrix So we have 1115 possible predictors}
\hlcom{# More model selection, for clarity start with one}
\hlcom{# train/validation split}
\hlstd{N} \hlkwb{<-} \hlkwd{nrow}\hlstd{(HM)}
\hlkwd{set.seed}\hlstd{(}\hlnum{12345678}\hlstd{)}
\hlstd{trainInd} \hlkwb{<-} \hlkwd{sample}\hlstd{(}\hlnum{1}\hlopt{:}\hlstd{N,} \hlkwd{round}\hlstd{(N} \hlopt{*} \hlnum{0.8}\hlstd{),} \hlkwc{replace} \hlstd{= F)}
\hlstd{trainSet} \hlkwb{<-} \hlstd{HM[trainInd, ]}
\hlstd{validSet} \hlkwb{<-} \hlstd{HM[}\hlopt{-}\hlstd{trainInd, ]}
\hlkwd{library}\hlstd{(MASS)}
\hlcom{# Full model and empty model with just intercept}
\hlstd{full} \hlkwb{<-} \hlkwd{lm}\hlstd{(count_pitch_1} \hlopt{~} \hlstd{.,} \hlkwc{data} \hlstd{= trainSet)}
\hlstd{empty} \hlkwb{<-} \hlkwd{lm}\hlstd{(count_pitch_1} \hlopt{~} \hlnum{1}\hlstd{,} \hlkwc{data} \hlstd{= trainSet)}
\hlcom{# Stepwise forward with BIC}
\hlkwd{stepAIC}\hlstd{(}\hlkwc{object} \hlstd{= empty,} \hlkwc{scope} \hlstd{=} \hlkwd{list}\hlstd{(}\hlkwc{upper} \hlstd{= full,} \hlkwc{lower} \hlstd{= empty),}
  \hlkwc{direction} \hlstd{=} \hlstr{"forward"}\hlstd{,} \hlkwc{k} \hlstd{=} \hlkwd{log}\hlstd{(}\hlkwd{nrow}\hlstd{(trainSet)))}
\hlstd{m1} \hlkwb{<-} \hlkwd{lm}\hlstd{(}\hlkwc{formula} \hlstd{= count_pitch_1} \hlopt{~} \hlstd{Prop_m3_num_0_8.1} \hlopt{+} \hlstd{count_pitch_3} \hlopt{+}
  \hlstd{Prop_m3_num_0_8.3} \hlopt{+} \hlstd{Prop_m3_mean_8.1} \hlopt{+} \hlstd{count_pitch_4} \hlopt{+} \hlstd{Dev_count_8_thresh4.393.1} \hlopt{+}
  \hlstd{Prop_m3_mean_8.3} \hlopt{+} \hlstd{mean_time_3} \hlopt{+} \hlstd{Dev_count_t_14_thresh0.216.3} \hlopt{+}
  \hlstd{voicepair_int_dist_6_1.2} \hlopt{+} \hlstd{Prop_m3_sd_8.3} \hlopt{+} \hlstd{Prop_m3_sd_8.1} \hlopt{+}
  \hlstd{Prop_m3_num_.6_8.1} \hlopt{+} \hlstd{simult_rest_perc} \hlopt{+} \hlstd{Dev_perc_t_14.1} \hlopt{+}
  \hlstd{count_pitch_2} \hlopt{+} \hlstd{Prop_m3_num_0_8.2} \hlopt{+} \hlstd{Prop_m3_mean_8.2} \hlopt{+} \hlstd{Dev_count_8_thresh4.024.2} \hlopt{+}
  \hlstd{Dev_count_18_thresh3.899.2} \hlopt{+} \hlstd{Expo_t_count_14.thresh0.7.1} \hlopt{+}
  \hlstd{Prop_m3_num_0_12.1} \hlopt{+} \hlstd{Expo_acc_8.1} \hlopt{+} \hlstd{Expo_perc_8.2} \hlopt{+} \hlstd{Prop_m3_num_0_12.2} \hlopt{+}
  \hlstd{Dev_count_t_8_thresh0.247.1} \hlopt{+} \hlstd{Expo_t_count_8.thresh0.7.1} \hlopt{+}
  \hlstd{voicepair_int_dist_1_1.3} \hlopt{+} \hlstd{Expo_perc_12.3} \hlopt{+} \hlstd{Pairwise_voice_int_mean.1.3} \hlopt{+}
  \hlstd{Expo_t_count_18.thresh0.7.3} \hlopt{+} \hlstd{Prop_m3_q3_16.2} \hlopt{+} \hlstd{Dev_perc_t_14.4} \hlopt{+}
  \hlstd{Prop_m3_q3_14.4} \hlopt{+} \hlstd{Dev_count_t_14_thresh0.187.3,} \hlkwc{data} \hlstd{= trainSet)}
\hlcom{# we can use the AIC function with our own k for the L0}
\hlcom{# penalty}
\hlkwd{AIC}\hlstd{(m1,} \hlkwc{k} \hlstd{=} \hlkwd{log}\hlstd{(}\hlkwd{nrow}\hlstd{(trainSet)))}
\hlkwd{BIC}\hlstd{(m1)}  \hlcom{# in this case matches BIC as we expect (1977.6)}
\hlstd{pred1} \hlkwb{<-} \hlkwd{predict}\hlstd{(m1,} \hlkwc{newdata} \hlstd{= validSet)}
\hlkwd{sqrt}\hlstd{(}\hlkwd{mean}\hlstd{((validSet}\hlopt{$}\hlstd{count_pitch_1} \hlopt{-} \hlstd{pred1)}\hlopt{^}\hlnum{2}\hlstd{))}  \hlcom{# RMSE on validation}
\hlkwd{sqrt}\hlstd{(}\hlkwd{mean}\hlstd{(m1}\hlopt{$}\hlstd{residuals}\hlopt{^}\hlnum{2}\hlstd{))}  \hlcom{# RMSE on train}
\hlcom{# Try ICM to search for a model with a potentially better}
\hlcom{# BIC than the one found with stepwise}
\hlstd{pen} \hlkwb{<-} \hlkwd{log}\hlstd{(}\hlkwd{nrow}\hlstd{(trainSet))}  \hlcom{#}
\hlstd{varlist} \hlkwb{=} \hlkwd{c}\hlstd{()}
\hlstd{varnames} \hlkwb{=} \hlkwd{names}\hlstd{(trainSet)}
\hlstd{n} \hlkwb{=} \hlkwd{nrow}\hlstd{(trainSet)}
\hlstd{varorder} \hlkwb{<-} \hlkwd{sample}\hlstd{(}\hlnum{1}\hlopt{:}\hlkwd{ncol}\hlstd{(trainSet))}  \hlcom{# random order of variables}
\hlstd{minCrit} \hlkwb{=} \hlnum{Inf}
\hlstd{noChange} \hlkwb{=} \hlstd{F}
\hlkwa{while} \hlstd{(}\hlopt{!}\hlstd{noChange) \{}
  \hlstd{noChange} \hlkwb{=} \hlstd{T}
  \hlkwa{for} \hlstd{(i} \hlkwa{in} \hlstd{varorder) \{}
    \hlkwa{if} \hlstd{(i} \hlopt{==} \hlnum{683}\hlstd{)}
      \hlkwa{next}
    \hlkwa{if} \hlstd{(i} \hlopt \hlstd{varlist} \hlopt{&} \hlkwd{length}\hlstd{(varlist)} \hlopt{>} \hlnum{1}\hlstd{) \{}
      \hlstd{index} \hlkwb{=} \hlkwd{c}\hlstd{(}\hlnum{683}\hlstd{, varlist[varlist} \hlopt{!=} \hlstd{i])}
      \hlstd{trainVars} \hlkwb{=} \hlstd{trainSet[, index]}
      \hlstd{fit} \hlkwb{=} \hlkwd{lm}\hlstd{(count_pitch_1} \hlopt{~} \hlstd{.,} \hlkwc{data} \hlstd{= trainVars)}
      \hlkwa{if} \hlstd{(}\hlkwd{AIC}\hlstd{(fit,} \hlkwc{k} \hlstd{= pen)} \hlopt{<} \hlstd{minCrit) \{}
        \hlstd{minCrit} \hlkwb{=} \hlkwd{AIC}\hlstd{(fit,} \hlkwc{k} \hlstd{= pen)}
        \hlstd{varlist} \hlkwb{=} \hlstd{varlist[varlist} \hlopt{!=} \hlstd{i]}
        \hlkwd{print}\hlstd{(}\hlkwd{paste0}\hlstd{(}\hlstr{"Criterion: "}\hlstd{,} \hlkwd{round}\hlstd{(minCrit,} \hlnum{1}\hlstd{),}
          \hlstr{", variables: "}\hlstd{,} \hlkwd{paste0}\hlstd{(varnames[varlist],}
          \hlkwc{collapse} \hlstd{=} \hlstr{" "}\hlstd{)))}
        \hlstd{best.model} \hlkwb{=} \hlstd{fit}
        \hlstd{noChange} \hlkwb{=} \hlstd{F}
      \hlstd{\}}
    \hlstd{\}} \hlkwa{else if} \hlstd{(}\hlopt{!}\hlstd{i} \hlopt \hlstd{varlist) \{}
      \hlstd{index} \hlkwb{=} \hlkwd{c}\hlstd{(}\hlnum{683}\hlstd{, varlist, i)}
      \hlstd{trainVars} \hlkwb{=} \hlstd{trainSet[, index]}
      \hlstd{fit} \hlkwb{=} \hlkwd{lm}\hlstd{(count_pitch_1} \hlopt{~} \hlstd{.,} \hlkwc{data} \hlstd{= trainVars)}
      \hlkwa{if} \hlstd{(}\hlkwd{AIC}\hlstd{(fit,} \hlkwc{k} \hlstd{= pen)} \hlopt{<} \hlstd{minCrit) \{}
        \hlstd{minCrit} \hlkwb{=} \hlkwd{AIC}\hlstd{(fit,} \hlkwc{k} \hlstd{= pen)}
        \hlstd{varlist} \hlkwb{=} \hlkwd{c}\hlstd{(varlist, i)}
        \hlkwd{print}\hlstd{(}\hlkwd{paste0}\hlstd{(}\hlstr{"Criterion: "}\hlstd{,} \hlkwd{round}\hlstd{(minCrit,} \hlnum{1}\hlstd{),}
          \hlstr{", variables: "}\hlstd{,} \hlkwd{paste0}\hlstd{(varnames[varlist],}
          \hlkwc{collapse} \hlstd{=} \hlstr{" "}\hlstd{)))}
        \hlstd{best.model} \hlkwb{=} \hlstd{fit}
        \hlstd{noChange} \hlkwb{=} \hlstd{F}
      \hlstd{\}}
    \hlstd{\}}
  \hlstd{\}}
\hlstd{\}}
\hlkwd{summary}\hlstd{(best.model)}
\hlstd{predICM} \hlkwb{<-} \hlkwd{predict}\hlstd{(best.model,} \hlkwc{newdata} \hlstd{= validSet)}
\hlkwd{sqrt}\hlstd{(}\hlkwd{mean}\hlstd{((validSet}\hlopt{$}\hlstd{count_pitch_1} \hlopt{-} \hlstd{predICM)}\hlopt{^}\hlnum{2}\hlstd{))}  \hlcom{# RMSE on validation}
\hlkwd{sqrt}\hlstd{(}\hlkwd{mean}\hlstd{(best.model}\hlopt{$}\hlstd{residuals}\hlopt{^}\hlnum{2}\hlstd{))}  \hlcom{# RMSE on train}
\hlcom{# Try stepwise again, with a larger L0 penalty (e.g., twice}
\hlcom{# the usual BIC penalty)}
\hlkwd{stepAIC}\hlstd{(}\hlkwc{object} \hlstd{= empty,} \hlkwc{scope} \hlstd{=} \hlkwd{list}\hlstd{(}\hlkwc{upper} \hlstd{= full,} \hlkwc{lower} \hlstd{= empty),}
  \hlkwc{direction} \hlstd{=} \hlstr{"forward"}\hlstd{,} \hlkwc{k} \hlstd{=} \hlnum{2} \hlopt{*} \hlkwd{log}\hlstd{(}\hlkwd{nrow}\hlstd{(trainSet)))}
\hlstd{m2} \hlkwb{<-} \hlkwd{lm}\hlstd{(}\hlkwc{formula} \hlstd{= count_pitch_1} \hlopt{~} \hlstd{Prop_m3_num_0_8.1} \hlopt{+} \hlstd{count_pitch_3} \hlopt{+}
  \hlstd{Prop_m3_num_0_8.3} \hlopt{+} \hlstd{Prop_m3_mean_8.1} \hlopt{+} \hlstd{count_pitch_4} \hlopt{+} \hlstd{Dev_count_8_thresh4.393.1} \hlopt{+}
  \hlstd{Prop_m3_mean_8.3} \hlopt{+} \hlstd{mean_time_3} \hlopt{+} \hlstd{Dev_count_t_14_thresh0.216.3,}
  \hlkwc{data} \hlstd{= trainSet)}
\hlkwd{AIC}\hlstd{(m2,} \hlkwc{k} \hlstd{=} \hlnum{2} \hlopt{*} \hlkwd{log}\hlstd{(}\hlkwd{nrow}\hlstd{(trainSet)))}
\hlcom{# calculate the value of criterion based on this larger L0}
\hlcom{# penalty}
\hlstd{pred2} \hlkwb{<-} \hlkwd{predict}\hlstd{(m2,} \hlkwc{newdata} \hlstd{= validSet)}
\hlkwd{sqrt}\hlstd{(}\hlkwd{mean}\hlstd{((validSet}\hlopt{$}\hlstd{count_pitch_1} \hlopt{-} \hlstd{pred2)}\hlopt{^}\hlnum{2}\hlstd{))}  \hlcom{# RMSE on validation}
\hlkwd{sqrt}\hlstd{(}\hlkwd{mean}\hlstd{(m2}\hlopt{$}\hlstd{residuals}\hlopt{^}\hlnum{2}\hlstd{))}  \hlcom{# RMSE on train}
\hlcom{# Try ICM as well with this penalty}
\hlstd{pen} \hlkwb{<-} \hlnum{2} \hlopt{*} \hlkwd{log}\hlstd{(}\hlkwd{nrow}\hlstd{(trainSet))}
\hlstd{varlist} \hlkwb{=} \hlkwd{c}\hlstd{()}
\hlstd{varnames} \hlkwb{=} \hlkwd{names}\hlstd{(trainSet)}
\hlstd{n} \hlkwb{=} \hlkwd{nrow}\hlstd{(trainSet)}
\hlstd{varorder} \hlkwb{<-} \hlkwd{sample}\hlstd{(}\hlnum{1}\hlopt{:}\hlkwd{ncol}\hlstd{(trainSet))}
\hlstd{minCrit} \hlkwb{=} \hlnum{Inf}
\hlstd{noChange} \hlkwb{=} \hlstd{F}
\hlkwa{while} \hlstd{(}\hlopt{!}\hlstd{noChange) \{}
  \hlstd{noChange} \hlkwb{=} \hlstd{T}
  \hlkwa{for} \hlstd{(i} \hlkwa{in} \hlstd{varorder) \{}
    \hlkwa{if} \hlstd{(i} \hlopt{==} \hlnum{683}\hlstd{)}
      \hlkwa{next}
    \hlkwa{if} \hlstd{(i} \hlopt \hlstd{varlist} \hlopt{&} \hlkwd{length}\hlstd{(varlist)} \hlopt{>} \hlnum{1}\hlstd{) \{}
      \hlstd{index} \hlkwb{=} \hlkwd{c}\hlstd{(}\hlnum{683}\hlstd{, varlist[varlist} \hlopt{!=} \hlstd{i])}
      \hlstd{trainVars} \hlkwb{=} \hlstd{trainSet[, index]}
      \hlstd{fit} \hlkwb{=} \hlkwd{lm}\hlstd{(count_pitch_1} \hlopt{~} \hlstd{.,} \hlkwc{data} \hlstd{= trainVars)}
      \hlkwa{if} \hlstd{(}\hlkwd{AIC}\hlstd{(fit,} \hlkwc{k} \hlstd{= pen)} \hlopt{<} \hlstd{minCrit) \{}
        \hlstd{minCrit} \hlkwb{=} \hlkwd{AIC}\hlstd{(fit,} \hlkwc{k} \hlstd{= pen)}
        \hlstd{varlist} \hlkwb{=} \hlstd{varlist[varlist} \hlopt{!=} \hlstd{i]}
        \hlkwd{print}\hlstd{(}\hlkwd{paste0}\hlstd{(}\hlstr{"Criterion: "}\hlstd{,} \hlkwd{round}\hlstd{(minCrit,} \hlnum{1}\hlstd{),}
          \hlstr{", variables: "}\hlstd{,} \hlkwd{paste0}\hlstd{(varnames[varlist],}
          \hlkwc{collapse} \hlstd{=} \hlstr{" "}\hlstd{)))}
        \hlstd{best.model} \hlkwb{=} \hlstd{fit}
        \hlstd{noChange} \hlkwb{=} \hlstd{F}
      \hlstd{\}}
    \hlstd{\}} \hlkwa{else if} \hlstd{(}\hlopt{!}\hlstd{i} \hlopt \hlstd{varlist) \{}
      \hlstd{index} \hlkwb{=} \hlkwd{c}\hlstd{(}\hlnum{683}\hlstd{, varlist, i)}
      \hlstd{trainVars} \hlkwb{=} \hlstd{trainSet[, index]}
      \hlstd{fit} \hlkwb{=} \hlkwd{lm}\hlstd{(count_pitch_1} \hlopt{~} \hlstd{.,} \hlkwc{data} \hlstd{= trainVars)}
      \hlkwa{if} \hlstd{(}\hlkwd{AIC}\hlstd{(fit,} \hlkwc{k} \hlstd{= pen)} \hlopt{<} \hlstd{minCrit) \{}
        \hlstd{minCrit} \hlkwb{=} \hlkwd{AIC}\hlstd{(fit,} \hlkwc{k} \hlstd{= pen)}
        \hlstd{varlist} \hlkwb{=} \hlkwd{c}\hlstd{(varlist, i)}
        \hlkwd{print}\hlstd{(}\hlkwd{paste0}\hlstd{(}\hlstr{"Criterion: "}\hlstd{,} \hlkwd{round}\hlstd{(minCrit,} \hlnum{1}\hlstd{),}
          \hlstr{", variables: "}\hlstd{,} \hlkwd{paste0}\hlstd{(varnames[varlist],}
          \hlkwc{collapse} \hlstd{=} \hlstr{" "}\hlstd{)))}
        \hlstd{best.model} \hlkwb{=} \hlstd{fit}
        \hlstd{noChange} \hlkwb{=} \hlstd{F}
      \hlstd{\}}
    \hlstd{\}}
  \hlstd{\}}
\hlstd{\}}
\hlstd{predICM} \hlkwb{<-} \hlkwd{predict}\hlstd{(best.model,} \hlkwc{newdata} \hlstd{= validSet)}
\hlkwd{sqrt}\hlstd{(}\hlkwd{mean}\hlstd{((validSet}\hlopt{$}\hlstd{count_pitch_1} \hlopt{-} \hlstd{predICM)}\hlopt{^}\hlnum{2}\hlstd{))}  \hlcom{# RMSE on validation}
\hlkwd{sqrt}\hlstd{(}\hlkwd{mean}\hlstd{(best.model}\hlopt{$}\hlstd{residuals}\hlopt{^}\hlnum{2}\hlstd{))}  \hlcom{# RMSE on train}
\hlcom{# K fold cross validation to choose model selection method}
\hlstd{K} \hlkwb{<-} \hlnum{5}
\hlstd{validSetSplits} \hlkwb{<-} \hlkwd{sample}\hlstd{((}\hlnum{1}\hlopt{:}\hlstd{N)}\hlopt\hlstd{K} \hlopt{+} \hlnum{1}\hlstd{)}
\hlstd{RMSE1} \hlkwb{<-} \hlkwd{c}\hlstd{()}
\hlstd{RMSE2} \hlkwb{<-} \hlkwd{c}\hlstd{()}
\hlkwa{for} \hlstd{(k} \hlkwa{in} \hlnum{1}\hlopt{:}\hlstd{K) \{}
  \hlstd{validSet} \hlkwb{<-} \hlstd{HM[validSetSplits} \hlopt{==} \hlstd{k, ]}
  \hlstd{trainSet} \hlkwb{<-} \hlstd{HM[validSetSplits} \hlopt{!=} \hlstd{k, ]}
  \hlstd{full} \hlkwb{<-} \hlkwd{lm}\hlstd{(count_pitch_1} \hlopt{~} \hlstd{.,} \hlkwc{data} \hlstd{= trainSet)}
  \hlstd{empty} \hlkwb{<-} \hlkwd{lm}\hlstd{(count_pitch_1} \hlopt{~} \hlnum{1}\hlstd{,} \hlkwc{data} \hlstd{= trainSet)}
  \hlstd{m1} \hlkwb{<-} \hlkwd{stepAIC}\hlstd{(}\hlkwc{object} \hlstd{= empty,} \hlkwc{scope} \hlstd{=} \hlkwd{list}\hlstd{(}\hlkwc{upper} \hlstd{= full,}
    \hlkwc{lower} \hlstd{= empty),} \hlkwc{direction} \hlstd{=} \hlstr{"forward"}\hlstd{,} \hlkwc{k} \hlstd{=} \hlkwd{log}\hlstd{(}\hlkwd{nrow}\hlstd{(trainSet)))}
  \hlstd{pred1} \hlkwb{<-} \hlkwd{predict}\hlstd{(m1,} \hlkwc{newdata} \hlstd{= validSet)}
  \hlstd{RMSE1[k]} \hlkwb{<-} \hlkwd{sqrt}\hlstd{(}\hlkwd{mean}\hlstd{((validSet}\hlopt{$}\hlstd{count_pitch_1} \hlopt{-} \hlstd{pred1)}\hlopt{^}\hlnum{2}\hlstd{))}
  \hlstd{m2} \hlkwb{<-} \hlkwd{stepAIC}\hlstd{(}\hlkwc{object} \hlstd{= empty,} \hlkwc{scope} \hlstd{=} \hlkwd{list}\hlstd{(}\hlkwc{upper} \hlstd{= full,}
    \hlkwc{lower} \hlstd{= empty),} \hlkwc{direction} \hlstd{=} \hlstr{"forward"}\hlstd{,} \hlkwc{k} \hlstd{=} \hlnum{2} \hlopt{*} \hlkwd{log}\hlstd{(}\hlkwd{nrow}\hlstd{(trainSet)))}
  \hlstd{pred2} \hlkwb{<-} \hlkwd{predict}\hlstd{(m2,} \hlkwc{newdata} \hlstd{= validSet)}
  \hlstd{RMSE2[k]} \hlkwb{<-} \hlkwd{sqrt}\hlstd{(}\hlkwd{mean}\hlstd{((validSet}\hlopt{$}\hlstd{count_pitch_1} \hlopt{-} \hlstd{pred2)}\hlopt{^}\hlnum{2}\hlstd{))}
\hlstd{\}}
\hlstd{RMSE1}
\hlstd{RMSE2}
\hlkwd{mean}\hlstd{(RMSE1)}
\hlkwd{mean}\hlstd{(RMSE2)}
\hlcom{# turns out m2 is indeed the better procedure among these}
\hlcom{# two based on CV prediction error if we decide on}
\hlcom{# procedure m2, we can apply procedure m2 to the entire 285}
\hlcom{# observations to get a final model for future prediction}
\hlcom{# e.g.,}
\hlstd{full} \hlkwb{<-} \hlkwd{lm}\hlstd{(count_pitch_1} \hlopt{~} \hlstd{.,} \hlkwc{data} \hlstd{= HM)}
\hlstd{empty} \hlkwb{<-} \hlkwd{lm}\hlstd{(count_pitch_1} \hlopt{~} \hlnum{1}\hlstd{,} \hlkwc{data} \hlstd{= HM)}
\hlstd{mfinal} \hlkwb{<-} \hlkwd{stepAIC}\hlstd{(}\hlkwc{object} \hlstd{= empty,} \hlkwc{scope} \hlstd{=} \hlkwd{list}\hlstd{(}\hlkwc{upper} \hlstd{= full,}
  \hlkwc{lower} \hlstd{= empty),} \hlkwc{direction} \hlstd{=} \hlstr{"forward"}\hlstd{,} \hlkwc{k} \hlstd{=} \hlnum{2} \hlopt{*} \hlkwd{log}\hlstd{(}\hlkwd{nrow}\hlstd{(trainSet)))}
\end{alltt}
\end{kframe}
\end{knitrout}
\section{2020-02-28}
\begin{itemize}
    \item 5 min recap
    \item The Chi-squared and the T-distribution
    \item Normal problem with unknown variance
    \item Clicker questions
\end{itemize}
\underline{Confidence intervals}

Case I\@: Confidence interval for the mean for normal when $ \sigma $ is known
\[ \bar{y}\pm z^* \frac{\sigma}{\sqrt{n}}  \]
\begin{itemize}
    \item $ \bar{y}= $ sample mean
    \item $ \sigma= $ population standard deviation
    \item $ n= $ sample size
    \item $ z^* $ depends on the level of confidence
          \begin{itemize}
              \item $ z^*=1.96 $ if confidence level is $ 95\% $ for every $ n $
          \end{itemize}
\end{itemize}

Case II\@: Binomial Confidence
\[ Y \sim \bin{n,\theta} \]
\begin{itemize}
    \item $ \theta= $ probability of success (unknown)
\end{itemize}
Confidence interval is given by
\[ \hat{\theta}\pm \underbrace{z^* \sqrt{\frac{\hat{\theta}\left( 1-\hat{\theta} \right)}{n}}}_{\text{
            margin of error
        }} \]
\begin{itemize}
    \item $ \hat{\theta}= $ sample proportion
    \item $ n= $ sample size
\end{itemize}
If we want the margin of error to be $ \leqslant \ell $, then
\[ n\geqslant \left( \frac{z^*}{\ell} \right)^2\left( \frac{1}{4} \right) \]

\underline{The Chi-Squared Distriubtion}

\begin{Definition}{}{}
    $ W $ is a continuous random variable taking all non-negative values.
    $ W $ is said to follow a \textbf{\emph{Chi-Squared}} distribution
    with $ n $ degrees of freedom (d.f), denoted $ W \sim \chi^2(n) $,
    if
    \[ W=Z_1^2+\cdots+Z_n^2 \]
    where $ Z_i \sim N(0,1) $ with $ Z_i $'s independent.
\end{Definition}

\underline{Properties of the Chi-Squared}
\begin{enumerate}[label=(\roman*)]
    \item $ n= $ d.f. = parameter of the Chi-squared. Once $ n $ is specified, the d.f.\ is known
    \item Density function looks like a gaussian distribution as $ df\rightarrow\infty $
    \item If $ W \sim \chi^2_n $, then $ E(W)=n $ and $ Var(W)=2n $
\end{enumerate}
Cases:
\begin{itemize}
    \item Case I\@: $ n=1 $, then $ W=Z^2 $
    \item Case II\@: $ n=2 $, then $ W \sim \exponential{2} $
    \item Case III\@: $ n $ is ``large'', then $ W \sim N(n,2n) $ approximately
    \item Case IV\@: $ n $ is intermediate, then we use the table
\end{itemize}

Let $ (X,Y) $ be a random point on a Cartesian plane. Assuming $ X $ and $ Y $
have independent $ G(0,1) $ distributions, the probability that a point is greater
than $ 1.96 $ away from the origin is

\textbf{Hint}: The distance formula is $ x^2+y^2=d^2 $.
\begin{enumerate}[label=(\Alph*)]
    \item \textbf{less than $ \symbf{40\%} $}
    \item at least $ 40\% $ but less than $ 60\% $
    \item at least $ 60\% $ but less than $ 80\% $
    \item at least $ 80\% $
\end{enumerate}
Why? We know that $ D^2 \sim \exponential{2} $, then we compute the following.
\[ P(D\geqslant 1.96)=1-F(1.96)=1-\left( 1-\frac{1}{2} e^{-1.96/2} \right)\approx 0.19=19\% \]

\underline{The Student's T-distribution}

\begin{Definition}{}{}
    $ T $ is said to follow a \textbf{\emph{Student's T-distribution}} with
    $ n $ degrees of freedom, denoted $ T \sim t(n) $, if
    \[ T=\frac{Z}{\sqrt{W/n}}  \]
    where $ Z \sim N(0,1) $ and $ W \sim \chi^2(n) $.
\end{Definition}

\underline{Properties}
\begin{enumerate}[label=(\roman*)]
    \item $ T $ can take all possible values
    \item $ T $ is symmetric around zero
    \item Similar to $ Z $, but with flatter tails
    \item As $ n\rightarrow+\infty $, then $ T\rightarrow Z $
\end{enumerate}
\underline{Clicker Question}:
\begin{itemize}
    \item $ Z \sim N(0,4) $
    \item $ T \sim t(15) $
    \item $ W \sim \chi^2(3) $
    \item $ Z,\;T,\;W $ are all independent
\end{itemize}
$ \E{W+T+\left( \frac{Z}{2} \right)^2} = $
\begin{enumerate}[label=(\Alph*)]
    \item $ 3 $
    \item $ \symbf{4} $
    \item $ 5 $
    \item None of the above.
\end{enumerate}
Why?
\begin{itemize}
    \item $ \E{W}=3 $
    \item $ \E{T}=0 $ since $ T $ is symmetric around zero for $ n>1 $
    \item Let $ Y=\frac{Z}{2} $. Then,
          \[ \E{Y^2}=\Var{Y}+\E{Y}^2
              =\left( \frac{1}{2} \right)^2 \Var{Z}+\frac{1}{2} \E{Z}
              =\frac{1}{4}(4)+0
              =1 \]
\end{itemize}
Thus, $ \E{W+T+Y}=3+0+1=4 $.

\makeheading{Lecture 22 | 2020-11-22}
\begin{Example}{}{}
    $ X_1,\ldots,X_n $ are i.i.d.\
    \begin{enumerate}
        \item $ \exponential{\theta} $
        \item $ \uniform{0,\theta} $
        \item $ f(x;\theta)=\theta x^{\theta-1} $ with $ 0<x<1 $ and $ \theta>0 $
    \end{enumerate}
    \textbf{Solution.}
    \begin{enumerate}
        \item $ \exponential{\theta} $. $ \mu_1=\E{X_1}=\theta $. $ \mu_1(\theta)=\theta $
              \[ \hat{\mu}_1=\frac{1}{n} \sum_{i=1}^{n} X_i \]
              $ \mu_1(\hat{\theta})=\hat{\mu}_1 $. Since $ \mu_1 $ is the identity
              map,
              \[ \hat{\theta}=\hat{\mu}_1=\frac{1}{n} \sum_{i=1}^{n} X_i \]
        \item $ \uniform{0,\theta} $.
              \[ \mu_1=\E{X_1}=\int_{0}^{\theta} x\biggl(\frac{1}{\theta} \biggr)\, d{x}=
                  \theta/2 \]
              $ \mu_1(\theta)=\frac{\theta}{2} $
              \[ \hat{\mu}_1=\frac{1}{n} \sum_{i=1}^{n} X_i \]
              \[ \mu_1(\hat{\theta})=\frac{\hat{\theta}}{2} =\hat{\mu}_1 \]
              Therefore,
              \[ \hat{\theta}_{\text{MM}}=2\hat{\mu}_1=\frac{2}{n} \sum_{i=1}^{n} X_i \]
        \item $ f(x;\theta)=\theta x^{\theta-1} $ with $ 0<x<1 $ and $ \theta>0 $
              \[ \mu_1=\E{X_1}=\int_{0}^{1} x \theta x^{\theta-1}\, d{x}=
                  \frac{\theta}{1+\theta} \]
              $ \displaystyle \mu_1(\theta)=\frac{\theta}{1+\theta} $
              \[ \hat{\mu}_1=\frac{1}{n}\sum_{i=1}^{n}  X_i \]
              \[ \mu_1(\hat{\theta})=\frac{\hat{\theta}}{1+\hat{\theta}}=\hat{\mu}_1 \]
              Therefore,
              \[ \hat{\theta}_{\text{MM}}=\frac{\hat{\mu}_1}{1-\hat{\mu}_1}=\frac{\bar{X}}{1-\bar{X}}   \]
        \item $ X_1,\ldots,X_n\stackrel{\text{iid}}{\sim}\N{\mu,\sigma^2} $.
              \[ \theta=\begin{pmatrix}
                      \mu \\
                      \sigma^2
                  \end{pmatrix} \]
              \begin{align*}
                  \mu_1 & =\E{X_1}=\mu                                  \\
                  \mu_2 & =\E{X_1^2}=\Var{X}+[\E{X_1}]^2=\mu^2+\sigma^2
              \end{align*}
              \begin{align*}
                  \mu_1(\mu,\sigma^2) & =\mu            \\
                  \mu_2(\mu,\sigma^2) & =\mu^2+\sigma^2
              \end{align*}
              \[ \hat{\mu}_1=\frac{1}{n} \sum_{i=1}^n X_i \]
              \[ \hat{\mu}_2=\frac{1}{n} \sum_{i=1}^{n} X_i^2 \]
              \[ \mu_1(\hat{\mu},\hat{\sigma}^2)=\hat{\mu}=\hat{\mu}_1=\bar{X} \]
              \[ \mu_2(\hat{\mu},\hat{\sigma}^2)=(\hat{\mu})^2+\hat{\sigma}^2=\hat{\mu}_2 \]
              Therefore,
              \begin{align*}
                  \hat{\mu}_{\text{MM}} & =\bar{X}_n                                    \\
                  \hat{\sigma}^2_{\text{MM}}
                                        & =\hat{\mu}_2-(\bar{X}_n)^2                    \\
                                        & =\frac{1}{n} \sum_{i=1}^{n} X_i^2-\bar{X}_n   \\
                                        & =\frac{1}{n} \sum_{i=1}^{n} (X_i-\bar{X}_n)^2
              \end{align*}
    \end{enumerate}
\end{Example}
\section{Maximum Likelihood Method}
This section: introduce the most commonly used method for estimating unknown
parameter $ \theta $ referred to as maximum likelihood method.
\begin{itemize}
    \item Likelihood function
          \begin{enumerate}
              \item Suppose $ X_1,\ldots,X_n $ are i.i.d.\ from $ f(x;\theta) $
              \item Given $ (x_1,\ldots,x_n) $, the observed value of $ (X_1,\ldots,X_n) $.
                    We calculate the joint p.f.\ of $ (X_1,\ldots,X_n) $ at observed
                    data $ (x_1,\ldots,x_n) $ or joint p.d.f.\ of $ (X_1,\ldots,X_n) $
                    at observed data $ (x_1,\ldots,x_n) $.

                    Discrete random variables joint p.d.f.\ of $ (X_1,\ldots,X_n) $
                    at $ (x_1,\ldots,x_n) $:
                    \[ \Prob{X_1=x_1,\ldots,X_n=x_n}=
                        \prod_{i=1}^n \Prob{X_i=x_i}=
                        \prod_{i=1}^n f(x_i;\theta) \]
                    Continuous random variables joint p.d.f.\ of $ (X_1,\ldots,X_n) $
                    at $ (x_1,\ldots,x_n) $:
                    \[ f_{X_1,\ldots,X_n}(x_1,\ldots,x_n)=
                        \prod_{i=1}^n f(x_i;\theta) \]
              \item We use $ L(\theta;x_1,\ldots,x_n) $ or simply
                    $ L(\theta) $ to denote it. That is to say,
                    \[ L(\theta;x_1,\ldots,x_n)=
                        \begin{cases*}
                            \Prob{X_1=x_1,\ldots,X_n=x_n}      & discrete   \\
                            f_{X_1,\ldots,X_n}(x_1,\ldots,x_n) & continuous
                        \end{cases*}=\prod_{i=1}^n f(x_i;\theta) \]
                    Here, $ L(\theta;x_1,\ldots,x_n) $ is called the likelihood function
                    of $ \theta $.
          \end{enumerate}
\end{itemize}
Comments:
\begin{enumerate}
    \item Likelihood function measures how likely we get
          the observed data for a given $ \theta $.
    \item Smaller $ L(\theta) $ means $ \theta $ is less likely
          to generate the observed data.
    \item Larger $ L(\theta) $ means $ \theta $ is more likely
          to generate the observed data.
\end{enumerate}
\subsection*{Idea of Maximum Likelihood Method}
Choose $ \theta $ to maximize $ L(\theta) $ or choose
$ \theta $ such that it most likely generates the observed data.

Maximum likelihood estimator/estimate (MLE)
\begin{enumerate}
    \item ML estimate maximizes $ L(\theta) $, and we use
          $ \hat{\theta}=\hat{\theta}(x_1,\ldots,x_n) $ to denote it.
          \[ \hat{\theta}=\hat{\theta}(x_1,\ldots,x_n)=
              \arg\max_{\theta\in \Theta}L(\theta) \]
    \item ML estimator: $ \hat{\theta}=\hat{\theta}(X_1,\ldots,X_n) $
    \item Log-likelihood function: log of likelihood function:
          \[ \ell(\theta)=\ln[L(\theta)] \]
          Then: an immediate result is:
          \[ \hat{\theta}=\hat{\theta}(x_1,\ldots,x_n)=
              \arg\max_{\theta\in \Theta}\ell(\theta)
              \arg\max_{\theta\in \Theta}L(\theta) \]
    \item Invariance principal of ML estimator
          $ \tau(\theta) $ is a function of $ \theta $.
          $ \tau(\hat{\theta}) $ is the ML estimator of
          $ \tau(\theta) $ if $ \hat{\theta} $ is the ML
          estimator of $ \theta $.
\end{enumerate}
\begin{Example}{}{}
    $ X_1,\ldots,X_n \stackrel{\text{iid}}{\sim} \poi{\theta} $.
    Find ML estimator of $ \theta $.

    \textbf{Solution.}
    \[ f(x;\theta)=\frac{\theta^x}{x!} e^{-\theta} \]
    \[ L(\theta)=\prod_{i=1}^n f(x_i;\theta)=
        \prod_{i=1}^n \frac{\theta^{x_i}}{x_i!}e^{-\theta}=
        \frac{\theta^{\sum_{i=1}^{n} x_i}}{\prod_{i=1}^n(x_i!)}e^{-n\theta}   \]
    \[ \ell(\theta)=\biggl(\,\sum_{i=1}^{n} x_i\biggr)\ln(\theta)-n \theta-
        \sum_{i=1}^{n} \ln(x_i!) \]
    \[ \frac{d\ell(\theta)}{d\theta}=\frac{\sum_{i=1}^{n} x_i}{\theta}-n   \]
    ML estimator of $ \theta $ satisfies
    \[ \biggl[\frac{d\ell}{d\theta}\biggr]_{\theta=\hat{\theta}}=0\implies
        \frac{\sum_{i=1}^{n} x_i}{\hat{\theta}}-n=0\implies
        \hat{\theta}=\frac{\sum_{i=1}^{n} x_i}{n}   \]
    ML estimator of $ \theta $ is
    \[ \hat{\theta}=\frac{\sum_{i=1}^{n} X_i}{n}\quad\text{(same as the MM estimator)} \]
\end{Example}
\begin{Remark}{}{}
    \begin{itemize}
        \item ML estimator of $ \theta^2 $ is $ (\hat{\theta})^2 $
        \item ML estimator of $ e^{-\theta} $ is $ e^{-\hat{\theta}} $
    \end{itemize}
\end{Remark}
\begin{Example}{}{}
    $ X_1,\ldots,X_n $ are i.i.d.\ from $ f(x;\theta)=\theta x^{\theta-1} $
    with $ 0<x<1 $, $ \theta>0 $.
    Find ML estimator of $ \theta $.

    \textbf{Solution.}
    \[ L(\theta)=\prod_{i=1}^n f(x_i;\theta)=
        \prod_{i=1}^n \theta x_i^{\theta-1}=\theta^n\biggl(\prod_{i=1}^n x_i \biggr)^{\!\theta-1} \]
    \[ \ell(\theta)=n\ln(\theta)+(\theta-1)\sum_{i=1}^{n} \ln(x_i) \]
    \[ \frac{d\ell(\theta)}{d\theta}=\frac{n}{\theta} +\sum_{i=1}^{n} \ln(x_i)  \]
    ML estimate $ \hat{\theta} $ satisfies
    \[ \biggl[\frac{d\ell}{d\theta}\biggr]_{\theta=\hat{\theta}}=0\implies
        \frac{n}{\hat{\theta}} +\sum_{i=1}^{n} \ln(x_i)=0
        \implies \hat{\theta}=-\frac{n}{\sum_{i=1}^{n} \ln(x_i)}  \]
    ML estimator:
    \[ \hat{\theta}=-\frac{n}{\sum_{i=1}^{n} \ln(X_i)}\quad\text{(is different from MM estimator)} \]
\end{Example}


\end{document}
