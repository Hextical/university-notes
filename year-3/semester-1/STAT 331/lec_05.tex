\makeheading{Lecture 5 | 2020-09-21}
\section{Multiple Linear Regression (MLR)}
$ p $ explanatory variables which can be categorical,
continuous, etc.

\underline{Rocket}
\begin{itemize}
    \item $ x_1 $: nozzle area (large or small)
    \item $ x_2 $: mixture in propellent, ratio oxidized fuel
    \item $ Y $: thrust
\end{itemize}
Want to develop linear relationship between $ y $
and $ x_1,x_2,\ldots,x_p $.

\underline{Data}
$ n $ observations each consists of
response and explanatory variables ($ y_i , x_{i1},x_{i2},\ldots,x_{ip} $).
Then,
\[ Y_i \sim N(
    \underbrace{\beta_0+\beta_1x_{i1}+
        \beta_2x_{i2}+\cdots +\beta_p x_{ip}}_{\E{Y_i}=
        \mu_i},\sigma^2) \]
or $ Y_i=\mu_i+\varepsilon_i $ where $ \varepsilon_i \sim N(0,\sigma^2) $.

We can write in vector/matrix form
\[ \begin{bmatrix}
        Y_1    \\
        Y_2    \\
        \vdots \\
        Y_n
    \end{bmatrix}_{n\times 1}=
    \begin{bmatrix}
        \beta_0+\beta_1x_{11}+\cdots+\beta_p x_{1p} \\
        \beta_0+\beta_1x_{21}+\cdots+\beta_p x_{2p} \\
        \vdots                                      \\
        \beta_0+\beta_1x_{n1}+\cdots+\beta_p x_{np}
    \end{bmatrix}+
    \begin{bmatrix}
        \varepsilon_1 \\
        \varepsilon_2 \\
        \vdots        \\
        \varepsilon_n
    \end{bmatrix} \]
Which we can write as
\[ \vec{Y}=X\vec{\beta}+\vec{\varepsilon} \]
where
\[ \vec{Y}=\begin{bmatrix}
        Y_1    \\
        \vdots \\
        Y_{n}
    \end{bmatrix}_{n\times 1},\,
    X=
    \begin{bmatrix}
        1 & x_{11} & x_{12} & \cdots & x_{1p} \\
        1 & x_{21} & x_{22} & \cdots & x_{2p} \\
          &        & \vdots                   \\
        1 & x_{n1} & x_{n2} & \cdots & x_{np}
    \end{bmatrix}_{n\times(p+1)},\,
    \vec{\beta}=
    \begin{bmatrix}
        \beta_0 \\
        \vdots  \\
        \beta_p
    \end{bmatrix}_{(p+1)\times 1},\,
    \vec{\varepsilon}=
    \begin{bmatrix}
        \varepsilon_1 \\
        \vdots        \\
        \varepsilon_n
    \end{bmatrix}_{n\times 1} \]
We call $ \vec{Y}=(Y_1,Y_2,\ldots,Y_{n})^\top $
a \textbf{random vector} (vector of r.v.'s), analogue of
expectation and variance properties.
\begin{itemize}
    \item Mean vector:
          \[ \E{\vec{Y}}=\begin{bmatrix}
                  \E{Y_1} \\
                  \E{Y_2} \\
                  \vdots  \\
                  \E{Y_n}
              \end{bmatrix} \]
    \item Covariance matrix (variance-covariance matrix):
          \[ \Var{\vec{Y}}=\begin{bmatrix}
                  \Var{Y_1}     & \Cov{Y_1,Y_2} & \cdots & \Cov{Y_1,Y_n} \\
                  \Cov{Y_2,Y_1} & \Var{Y_2}     & \cdots & \Cov{Y_2,Y_n} \\
                  \vdots                                                 \\
                  \Cov{Y_n,Y_1} & \Cov{Y_n,Y_2} & \cdots & \Var{Y_n}
              \end{bmatrix} \]
          \begin{itemize}
              \item symmetric since $ \Cov{Y_i,Y_j}=\Cov{Y_j,Y_i} $
              \item positive semi-definite since
                    $ \vec{a}^\top\Var{\vec{Y}}\vec{a}\geqslant 0 $
                    for all $ \vec{a}\in\mathbb{R}^n $.
              \item $ \Var{\vec{Y}}=
                        \E{\left( \vec{Y}-\E{\vec{Y}} \right)\left( \vec{Y}-\E{\vec{Y}} \right)^\top} $
          \end{itemize}
\end{itemize}
Properties of random vector: let $ \vec{a} $ be a $ 1\times n $
matrix (row vector) of constants and $ A $ be an $ n\times n $
matrix of constants.
\[ \E{\vec{a}\vec{Y}}=\vec{a}\vec{Y} \]
\[ \E{A\vec{Y}}=A\E{\vec{Y}} \]
\[ \Var{\vec{a}\vec{Y}}=\vec{a}\Var{\vec{Y}}\vec{a}^\top \]
\[ \Var{A\vec{Y}}=A\Var{\vec{Y}}A^\top \]
Derivation of (4):
\begin{align*}
    \Var{A\vec{Y}}
     & =\E{\left(A\vec{Y}-\E{A\vec{Y}}\right)
        \left(A\vec{Y}-\E{A\vec{Y}}\right)^\top}                                             \\
     & =\E{\left( A\vec{Y}-A\E{\vec{Y}} \right)
        \left( A\vec{Y}-A\E{\vec{Y}} \right)^\top}                                           \\
     & =\E{A\left( \vec{Y}-\E{\vec{Y}} \right)
        \left( A\left( \vec{Y}-\E{\vec{Y}} \right) \right)^\top}                             \\
     & =\E{A\left( \vec{Y}-\E{\vec{Y}} \right)
        \left( \vec{Y}-\E{\vec{Y}} \right)^\top A^\top}                                      \\
     & =A\E{\left( \vec{Y}-\E{\vec{Y}} \right)\left( \vec{Y}-\E{\vec{Y}} \right)^\top}A^\top \\
     & =A\Var{\vec{Y}}A^\top
\end{align*}
Numerical example: $ \vec{Y}=(Y_1,Y_2,Y_3)^\top $. Suppose
\[ \E{\vec{Y}}=\begin{bmatrix}
        3 \\
        1 \\
        2
    \end{bmatrix} \]
and
\[ \Var{Y}=\begin{bmatrix}
        4   & 1/2 & -2 \\
        1/2 & 1   & 0  \\
        -2  & 0   & 3
    \end{bmatrix} \]
and
\[ \vec{a}=\begin{bmatrix}
        1 & -1 & 2
    \end{bmatrix} \]
and
\[ A=\begin{bmatrix}
        1 & 2 & 3 \\
        4 & 5 & 6 \\
        7 & 8 & 9
    \end{bmatrix} \]
\underline{Exercise}:
\begin{itemize}
    \item $ \E{\vec{a}\vec{Y}} $
    \item $ \Var{\vec{a}\vec{Y}} $
    \item $ \E{A\vec{Y}} $
    \item $ \Var{A\vec{Y}} $
\end{itemize}
Let's do the first two,
\[ \E{\vec{a}\vec{Y}}=\vec{a}\E{\vec{Y}}=
    \begin{bmatrix}
        1 & -1 & 2
    \end{bmatrix}\begin{bmatrix}
        3 \\
        1 \\
        2
    \end{bmatrix}=1(3)-1(1)+2(2)=6 \]
\begin{align*}
    \Var{\vec{a}\vec{Y}}
     & =
    \begin{bmatrix}
        1 & -1 & 2
    \end{bmatrix}
    \begin{bmatrix}
        4   & 1/2 & -2 \\
        1/2 & 1   & 0  \\
        -2  & 0   & 3
    \end{bmatrix}
    \begin{bmatrix}
        1  \\
        -1 \\
        2
    \end{bmatrix}     \\
     & =
    \begin{bmatrix}
        1 & -1 & 2
    \end{bmatrix}
    \begin{bmatrix}
        4(1)+(1/2)(-1)-2(2) \\
        (1/2)(1)+1(-1)+0(2) \\
        -2(1)+0(-1)+3(2)
    \end{bmatrix}     \\
     & =\begin{bmatrix}
        1 & -1 & 2
    \end{bmatrix}
    \begin{bmatrix}
        -1/2 \\
        -1/2 \\
        4
    \end{bmatrix}     \\
     & =1(-1/2)-1(-1/2)+2(4)       \\
     & =8
\end{align*}
Multivariate normal distribution (MVN):
We say that $ \vec{Y} \sim \text{MVN}(\vec{\mu},\Sigma) $
where $ \vec{\mu}= $ mean vector and $ \Sigma= $
covariance matrix. Suppose $ Y=(Y_1,\ldots,Y_n)^\top $.
\[ f(\vec{y};\vec{\mu},\Sigma)=
    \frac{1}{(2\pi)^{n/2}\abs{\Sigma}^{1/2}}
    \exp\left\{ -\frac{1}{2}(\vec{y}-\vec{\mu})
    \Sigma^{-1}(\vec{y}-\vec{\mu})^\top \right\}  \]
where $ \Sigma^{-1} $ is the inverse of the covariance matrix
and $ \abs{\Sigma} $ is the determinant of $ \Sigma $.

\underline{Properties of MVN}:
Suppose $ \vec{Y}=(Y_1,\ldots,Y_n)^\top \sim \text{MVN}(\vec{\mu},\Sigma) $
and $ \vec{a} $ is a $ 1\times n $ constant
and $ A $ is an $ n\times n  $ matrix of constants.
\begin{enumerate}
    \item Linear transformations of MVN is MVN, so
          \[ \vec{a}\vec{Y}\sim \text{MVN}(\vec{a}\vec{\mu},\vec{a}
              \Sigma\vec{a}^\top) \]
          \[ A\vec{Y} \sim \text{MVN}(A\vec{\mu},A\Sigma A^\top) \]
    \item Marginal distribution of $ Y_i $ is
          Normal,
          \[ Y_i \sim N(\mu_i,\Sigma_{ii}) \]
          In fact, any subset of $ Y_i $'s is MVN
    \item Conditional MVN is MVN, e.g. $ Y_1\mid Y_2,\ldots,Y_n $
    \item Another property:
          \[ \Cov{Y_i,Y_j}=0\iff Y_i,Y_j\text{ independent} \]
          that is, $ Y_i $ and $ Y_j $ are uncorrelated.
          \[ \Sigma_{ij}=0 \]

\end{enumerate}
