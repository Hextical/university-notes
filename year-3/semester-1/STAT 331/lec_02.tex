\makeheading{Lecture 2 | 2020-09-09}

A linear model with response variable ($ Y $)
and \emph{one} explanatory variable ($ x $)
is called a \textbf{simple linear regression}; that is,
\[ \bar{Y}=\beta_0+\beta_1x+\varepsilon \]
Data consists of pairs $ (x_i,y_i) $
where $ i=1,\ldots,n $.

Before fitting any model, we might
\begin{itemize}
    \item make a scatterplot to visualize if there
          is a linear relationship between $ x $ and $ y $
    \item calculate \emph{correlation}
\end{itemize}
If $ X $ and $ Y $ are random variables,
then
\[ \rho=\Corr{X,Y}=\frac{\Cov{X,Y}}{\Sd{X}\Sd{Y}}  \]
Based on $ (x_i,y_i) $ we can estimate the sample correlation:
\begin{align*}
    r
     & =\frac{\frac{1}{n-1} \sum\limits_{i=1}^{n}(x_i-\bar{x})(y_i-\bar{y})}
    {\sqrt{\frac{1}{n-1}\sum\limits_{i=1}^{n} (x_i-\bar{x})^2}
        \sqrt{\frac{1}{n-1}\sum\limits_{i=1}^{n}(y_i-\bar{y})}}              \\
     & =\frac{\sum\limits_{i=1}^{n} (x_i-\bar{x})(y_i-\bar{y})}{
        \sqrt{\sum\limits_{i=1}^{n} (x_i-\bar{x})^2}
        \sqrt{\sum\limits_{i=1}^{n} (y_i-\bar{y})^2}
    }                                                                        \\
     & =\frac{S_{xy}}{\sqrt{S_{xx}S_{yy}}}
\end{align*}
The sample correlation measures the strength and direction of
the \emph{linear} relationship
between $ X $ and $ Y $.
\begin{itemize}
    \item $ \abs{r}\approx 1 $ strong linear relationship
    \item $ \abs{r}\approx 0 $ lack of linear relationship
    \item $ r>0 $ positive relationship
    \item $ r<0 $ negative relationship
    \item $ -1\leqslant r\leqslant 1 $
\end{itemize}
But does not tell us how to predict $ Y $ from $ X $. To do so,
we need to estimate $ \beta_0 $ and $ \beta_1 $.

For data $ (x_i,y_i) $ for $ i=1,\ldots,n $, the
simple linear regression model is
\[ Y_i=\beta_0+\beta_1x_i+\varepsilon_i \]
Assume
\[ \varepsilon\stackrel{\text{iid}}{\sim}N(0,\sigma^2) \]
Therefore,
\[ Y_i\sim N(\beta_0+\beta_1x_i,\sigma^2) \]
In other words, \[
    \E{Y_i}=\mu_i=\beta_0+\beta_1x_i \text{ and }
    \Var{Y_i}=\sigma^2 \]
Note that the $ Y_i $'s are independent, but they are \emph{not}
independently distributed.

Use the \emph{Least Squares} (LS) to estimate $ \beta_0 $ and $ \beta_1 $.
\[ \min_{\beta_0,\beta_1}\sum\limits_{i=1}^{n}
    \left[ y_i-(\beta_0+\beta_1 x_i) \right]^2=S(\beta_0,\beta_1) \]
LS is equivalent to MLE when $ \varepsilon_i $'s are iid and Normal.

Taking partial derivatives:
\[ \frac{dS}{d\beta_0}=2
    \sum\limits_{i=1}^{n} \left[ y_i-(\beta_0+\beta_1x_i) \right](-1)  \]
\[ \frac{dS}{d\beta_1}=2
    \sum\limits_{i=1}^{n} \left[ y_i-(\beta_0+\beta_1x_i) \right](-x_i)  \]
Now,
\[ \frac{dS}{d\beta_0}=0
    \iff \sum\limits_{i=1}^{n}y_i-n\beta_0-
    \beta_1 \sum\limits_{i=1}^{n} x_i=0
    \iff \beta_0=\bar{y}-\beta_1\bar{x}  \]
\begin{align*}
    \frac{dS}{d\beta_1}=0
     & \stackrel{\text{plug }\beta_0}{\iff}
    \sum\limits_{i=1}^{n} \left[ y_i-\bar{y}+\beta_1\bar{x}-\beta_1 x_i \right]x_i=0    \\
     & \iff \sum\limits_{i=1}^{n} x_i(y_i-\bar{y})-\beta_1
    \sum\limits_{i=1}^{n} x_i(x_i-\bar{x})=0                                            \\
     & \iff \beta_1=\frac{\sum\limits_{i=1}^{n} x_i(y_i-\bar{y})}{\sum\limits_{i=1}^{n}
        x_i
        (x_i-\bar{x})}
\end{align*}
We can also show that
\[ \beta_1=\frac{\sum\limits_{i=1}^{n} (x_i-\bar{x})(y_i-\bar{x})}{
        \sum\limits_{i=1}^{n} (x_i-\bar{x})^2
    }  \]
We use a hat on the $ \beta $'s to show that they are estimates; that is,
\[ \hat{\beta}_0=\bar{y}-\hat{\beta}_1\bar{x} \]
\[ \hat{\beta}_1=
    \frac{\sum\limits_{i=1}^{n} (x_i-\bar{x})(y_i-\bar{y})}{
        \sum\limits_{i=1}^{n} (x_i-\bar{x})^2
    }  \]

Call $ \hat{\mu}_i=\hat{\beta}_0+\hat{\beta}_1x_i $ the \textbf{fitted values}
and $ e_i=y_i-\hat{\mu}_i $ the \textbf{residual}.
