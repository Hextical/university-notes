\makeheading{Lecture 10 | 2020-10-07}
\section{General Linear Hypothesis Tests Based on F Distribution}
So far we've tested $ H_0 $: $ \beta_j=0 $
vs $ H_A $: $ \beta_j\neq 0 $
involving individual parameters,
using $ t $ distribution.

Now consider hypothesis test of the form $ H_0 $: $ A\symbf{\beta}=\symbf{0} $
where $ A $ is a matrix of constraints specifying
linear combinations of parameters.

\begin{Example}{Coffee Continued}{}
    The full model is:
    \[ Y_i=\beta_0+\beta_1x_{i1}+\beta_2x_{i2}+\beta_3x_{i3}+\varepsilon_i \]
    \begin{itemize}
        \item $ Y_i $ is the flavour
        \item $ x_{i1} $ is acidity
        \item $ x_{i2} $ is 1 if semi, and 0 otherwise.
        \item $ x_{i3} $ is 1 if wet, and 0 otherwise.
    \end{itemize}
    Example 1.
    \begin{itemize}
        \item $ H_0 $: $ \beta_1=\beta_2=\beta_3=0 $ versus
        \item $ H_A $: at least one of $ \beta_1,\beta_2,\beta_3 $
              not 0.
        \item If $ H_0 $ is true, the model reduces to
              $ Y_i=\beta_0+\varepsilon_i $.
        \item This tests overall significance of regression
              (whether any of predictors impact response)
        \item $ A=\begin{bmatrix}
                      0 & 1 & 0 & 0 \\
                      0 & 0 & 1 & 0 \\
                      0 & 0 & 0 & 1
                  \end{bmatrix} $. Note that row $ i $ considers
              the constraint of $ \beta_i=0 $ for $ i=1,2,3 $
              in this example.
    \end{itemize}
    Example 2.
    \begin{itemize}
        \item $ H_0 $: $ \beta_2=\beta_3=0 $
        \item If $ H_0 $ is true, $ Y_i=\beta_0+\beta_1x_{i1}+\varepsilon_i $
        \item Q\@: Is reduced model with only acidity plausible?
        \item $ A=\begin{bmatrix}
                      0 & 0 & 1 & 0 \\
                      0 & 0 & 0 & 1
                  \end{bmatrix} $. Note that $ A\symbf{\beta}=\symbf{0}_{1\times 2} $
    \end{itemize}
    Example 3.
    \begin{itemize}
        \item $ H_0 $: $ \beta_2-\beta_3=0 $
        \item $ H_A $: $ \beta_2\neq \beta_3 $
        \item $ Y_i=\beta_0+\beta_1x_{i1}+\beta_2(x_{i2}+x_{i3})+\varepsilon_i $
              where $ (x_{i2}+x_{i3}) $ is
              $ 1 $ if semi/wet and $ 0 $ if dry.
        \item Do the wet and semi methods have the same impact
              on the response (holding acidity constant)?
        \item $ A=\begin{bmatrix}
                      0 & 0 & 1 & -1
                  \end{bmatrix} $
    \end{itemize}
\end{Example}

In general, with $ \ell $ constraints. $ A $
is an $ \ell \times (p+1) $ matrix
with rank $ \ell $. Recall
that
\[ \Span{X}=\set{\beta_0\symbf{1}+\beta_1\symbf{x}_1+
        \cdots+\beta_p\symbf{x}_p} \]
Let
\[ \Span{X}_A=\set{\beta_0\symbf{1}+\beta_1\symbf{x}_1+
        \cdots+\beta_p\symbf{x}_p:A\symbf{\beta}=\symbf{0}} \]
which is a subspace of $ \Span{X} $
since any vector in $ \Span{X}_A $
is also in $ \Span{X} $.
We call $ \Span{X}_A $ the
$ \Span{X} $ with constraint $ A $ on $ \symbf{\beta} $.

Let $ \hat{\symbf{\mu}}_A $
denote the fitted values from fitting the reduced model.
The residual if we fit the model with $ A\symbf{\beta}=\symbf{0} $
is $ \symbf{e}_A=\symbf{y}-\hat{\symbf{\mu}}_A $.

If $ H_0 $: $ A\symbf{\beta}=\symbf{0} $ is true, then
$ \hat{\symbf{\mu}} $ and $ \hat{\symbf{\mu}}_A $ should be close;
that is, the model makes similar predictions whether
we set $ A\symbf{\beta}=\symbf{0} $ or not when fitting the model.

So to assess whether $ H_0 $ is plausible,
look at $ \norm{\hat{\symbf{\mu}}-\hat{\symbf{\mu}}_A} $
where $ \norm{\cdot} $ is Euclidean or $ L_2 $ norm. That is,
\[ \norm{\hat{\symbf{\mu}}-\hat{\symbf{\mu}}_A}=
    \sqrt{(\hat{\symbf{\mu}}-\hat{\symbf{\mu}}_A)^\top(\hat{\symbf{\mu}}-\hat{\symbf{\mu}}_A)} \]
If it's ``large'' or ``small'' (close to 0)
where large gives evidence against $ H_0 $
and small gives evidence for $ H_0 $.

By Pythagoras,
\[ \norm{\symbf{y}-\hat{\symbf{\mu}}_A}^2=
    \norm{\symbf{y}-\hat{\symbf{\mu}}}^2+
    \norm{\hat{\symbf{\mu}}-\hat{\symbf{\mu}}_A}^2
    \quad\text{ or }\quad
    \norm{\symbf{e}_A}^2=\norm{\symbf{e}}^2+\norm{\hat{\symbf{\mu}}-
        \hat{\symbf{\mu}}_A
    }^2 \]
or equivalently $ \symbf{e}_A^\top \symbf{e}_A=\symbf{e}^\top \symbf{e}
    +\norm{\hat{\symbf{\mu}}-
        \hat{\symbf{\mu}}_A
    }^2 $
where $ \symbf{e}_A^\top \symbf{e}_A $ is the sum of squares
residual in the reduced model and $ \symbf{e}^\top \symbf{e} $
is the sum of squares residual in the full model.

We define $ \symbf{e}_A^\top \symbf{e}_A=\SS{Res}_A $
and $ \symbf{e}^\top \symbf{e} =\SS{Res} $.

Thus, $ \norm{\hat{\symbf{\mu}}-\hat{\symbf{\mu}}_A}^2=
    \SS{Res}_A-\SS{Res} \geqslant 0 $
additional sum of squares explained by full model vs reduced
one with constraints $ A $.

Practical implications:
\begin{itemize}
    \item $ \SS{Res} $ cannot decrease when constraints applied.
    \item Equivalently, full model always has small (or equal)
          $ \SS{Res} $ for a fixed $ \SS{Tot} $
          and thus higher $ R^2 $ compared to a reduced model.
\end{itemize}

Define test statistic:
\[ F=\frac{(\SS{Res}_A-\SS{Res})/\ell}{\SS{Res}/(n-p-1)}=
    \frac{(\SS{Res}_A-\SS{Res})/\ell}{\hat{\sigma}^2}  \]
\begin{Definition}{$F$ distribtion}{}
    If $ U \sim \chi^2(a) $ and $ V \sim \chi^2(b) $
    are independent. We say $ F $ follows an
    $ F $ \textbf{distribution} if
    \[ F=\frac{U/a}{V/b} \]
    and write $ F \sim F(a,b) $.
\end{Definition}
Here, we have these facts when $ H_0 $ is true
\[ V=\frac{\hat{\sigma}^2(n-p-1)}{\sigma^2}\sim \chi^2(n-p-1)  \]
\[ U=\frac{\norm{\hat{\symbf{\mu}}-\hat{\symbf{\mu}}_A}^2}{\sigma^2}\sim \chi^2(\ell)  \]
where $ U $ and $ V $ are independent. Therefore,
\[ F=\dfrac{\dfrac{\norm{\hat{\symbf{\mu}}-\hat{\symbf{\mu}}_A}^2}{\sigma^2}
        \biggl( \dfrac{1}{\ell} \biggr)}{\dfrac{\hat{\sigma}^2(n-p-1)}{\sigma^2}
        \biggl( \dfrac{1}{n-p-1} \biggr)} \sim F(\ell,n-p-1)   \]
when $ H_0 $ is true. Reject
$ H_0 $: $ A\symbf{\beta}=\symbf{0} $ at level $ \alpha $
if $ F $ is greater than $ (1-\alpha) $
quantile of $ F(\ell,n-p-1) $
and $ p $-value is $ P(Y\geqslant F) $ where $ Y \sim F(\ell,n-p-1) $.

\underline{Relation to $ T $ distribution}:
Say $ Y \sim t(a) $
\[ Y=\frac{Z}{\sqrt{U/a}}  \]
where $ Z \sim \N{0,1} $
and $ U \sim \chi^2(a) $ are independent.
Squaring everything,
\[ Y^2=\frac{Z^2}{U/a} \]
and we know $ Z^2 \sim \chi^2(1) $. Therefore,
$ Y^2 \sim F(1,a) $ (we divide by 1 in the numerator).

Thus, if our hypothesis test has one constraint,
then $ F $ test is equal to $ t $ test of same hypothesis;
for example, $ H_0 $: $ \beta_1=0 $ versus $ H_A $: $ \beta_1\neq 0 $.
