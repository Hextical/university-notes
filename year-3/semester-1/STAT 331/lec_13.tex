\makeheading{Lecture 13 | 2020-10-26}
\section{Model Selection Criteria}
Given $ p $ explanatory variables, find the subset
$ k\leqslant p $ of explanatory variables
(``reduced model'') that gives us
the ``best'' model:
\begin{itemize}
    \item goodness of fit
    \item interpretability
    \item predictive performance
\end{itemize}

Some related concepts:
\begin{enumerate}
    \item $ F $ tests compare between $ 2 $
          specific models where test adequacy of
          a ``reduced'' model (subset, ``nested'')
          relative to full model.

          \underline{Quiz 4}: $ \beta_1=\beta_2 $
          in part d-f
    \item Multicollinearity: can affect interpretability
          of $ \hat{\beta}_j $ usual interpretation
          ``holding other variables constant''
          doesn't really work when $ x_j $ is
          strongly correlated with other predictors.
    \item $ R^2 $ is the proportion
          of variability in the response explained by
          the regression model. It always increases
          when adding variables.
    \item $ \hat{\sigma}^2 $ is estimated residual variable,
          used for prediction, want $ \hat{\sigma}^2 $ small
          to give good predictive performance
\end{enumerate}
Two key ingredients:
\begin{itemize}
    \item Metric (or criterion) for comparing
          different models with potentially different number
          of predictors
    \item Selection/search strategy (which models
          should we fit and test?)
\end{itemize}
Examples of metrics for model selection:

\begin{Definition}{Adjusted $ R^2 $}{}
    \[ R^2_{\text{adj}}=1-\frac{\SS{Res}/(n-k-1)}{
            \SS{Total}/(n-1)
        }=1-\frac{\hat{\sigma}^2}{s^2}  \]
    for model with $ k $ predictors.
\end{Definition}
Compared to
\[ R^2=1-\frac{\SS{Res}}{\SS{Total}}=1-\frac{\hat{\sigma}^2(n-k-1)}{s^2(n-1)}   \]
\begin{itemize}
    \item $ \SS{Res}/(n-k-1) $
          estimated $ \hat{\sigma}^2 $
          for model with $ k $ predictors
    \item $ \SS{Total}/(n-1) $ is the sample variance
          of responses $ y_i $.
\end{itemize}
\begin{align*}
    R^2_{\text{adj}}=1-\frac{n-1}{n-k-1}(1-R^2)
     & = 1-\left( 1+\frac{k}{n-k-1} \right)
    (1-R^2)                                                             \\
     & =1-\left[ 1(1-R^2)+\left( \frac{k}{n-k-1} \right)(1-R^2) \right] \\
     & =1-\left[ 1-R^2+\left( \frac{k}{n-k-1}  \right)(1-R^2) \right]   \\
     & =1-1+R^2-\frac{k}{n-k-1} (1-R^2)                                 \\
     & =R^2-(1-R^2)\frac{k}{n-k-1}
\end{align*}
Intuition: $ R^2_{\text{adj}} $
accounts for number variables in model,
\emph{penalizes} inclusion of unimportant
predictors; that is, $ \SS{Res} $
has little decrease when adding
such variables. Meanwhile, $ R^2 $
always increases with more predictors,
but $ R^2_{\text{adj}} $ can decrease
if $ \SS{Res} $ change is small.

While $ R^2_{\text{adj}} $ loses its usual
interpretation of $ R^2 $, but can be used
as a measure of ``goodness of fit''
and model selection criterion (e.g.
pick subset of predictors that gives the highest
$ R^2_{\text{adj}} $).


\begin{Example}{}{}
    Given
    \begin{itemize}
        \item $ n=25 $
        \item $ \SS{Total}=20 $
        \item $ p=6 $
    \end{itemize}
    Suppose
    we're considering on a subset of $ k=4 $
    predictors, and find:

    \begin{center}
        \begin{tabularx}{0.8\linewidth}{@{}YYY@{}}
                                 & Reduced                                 & Full                                              \\
            \midrule
                                 & $ k=4 $                                 & $ p=6 $                                           \\
            $ \SS{Total} $       & $ 20 $                                  & $ 20 $                                            \\
            $ \SS{Res} $         & $ 10 $                                  & $ 9.8 $                                           \\
            \midrule
            $ R^2 $              & $ 10/20=0.5 $                           & $ 9.8/20=0.49 $                                   \\
            $ R^2_{\text{adj}} $ & $ 1-\frac{10/(25-4-1)}{20/(25-1)}=0.4 $ & $ 1-\frac{9.8/(25-6-1)}{20/(25-1)}\approx 0.347 $ \\
            $ \hat{\sigma}^2 $   & $ 10/(25-4-1)=0.5 $                     & $ 9.8/(25-6-1)\approx 0.544 $
        \end{tabularx}
    \end{center}
    \begin{itemize}
        \item $ n-k-1 $ d.f. Res in reduced
        \item $ n-p-1 $ d.f. Res in full
    \end{itemize}
    Remarks:
    \begin{itemize}
        \item $ R^2_{\text{adj}}<R^2 $, but as $ n\to\infty $,
              $ R^2_{\text{adj}}\to R^2 $.
        \item Model with higher $ R^2_{\text{adj}} $
              has lower $ \hat{\sigma}^2 $,
              thus is a reasonable metric for model selection.
    \end{itemize}
\end{Example}
\begin{Definition}{Akaike Information Criterion (AIC)}{}
    Let $ n $ be sample size and $ q $ be the number of estimated parameters.
    \begin{itemize}
        \item MLR\@: $ q=p+2 $ since we have $ p $ predictors + 1 intercept $ (\beta_0) $ + 1
              $ (\sigma^2) $.
    \end{itemize}
    The \textbf{Akaike information criterion} (AIC) is defined as
    \[ \text{AIC}=2q-2\ln[L(\hat{\theta})] \]
    where $ L(\hat{\theta}) $ is the likelihood function evaluated
    at $ \hat{\theta} $ (parameter estimates).
\end{Definition}
\begin{Remark}{}{}
    \begin{itemize}
        \item The least square estimates of
              $ \symbf{\beta} $ are equivalent to maximum likelihood estimates
              under the usual normal assumptions on $ \symbf{\varepsilon} $.
        \item $ 2q $ is the penalty for including more predictors.
              With more parameters, $ L(\hat{\theta}) $
              increases, offset by penalty $ 2q $.
        \item The model with lower AIC
              is preferred; that is, differences in AIC matter not the value itself.
    \end{itemize}
\end{Remark}
\begin{Remark}{$ \dagger $}{}
    If we want to measure
    just the \underline{difference} of AIC, we can do
    \[ \text{AIC}=n\ln \biggl[\frac{\SS{Res}}{n}\biggr]+2q \]
\end{Remark}
\begin{Definition}{Bayesian Information Criterion (BIC)}{}
    Let $ n $ be sample size and $ q $ be the number of estimated parameters.
    \begin{itemize}
        \item MLR\@: $ q=p+2 $ since we have $ p $ predictors + 1 intercept $ (\beta_0) $ + 1
              $ (\sigma^2) $.
    \end{itemize}
    The \textbf{Bayesian information criterion} (BIC) is defined as
    \[ \text{BIC}=q\ln(n)-2\ln[L(\hat{\theta})] \]
    where $ L(\hat{\theta}) $ is the likelihood function evaluated
    at $ \hat{\theta} $ (parameter estimates).
\end{Definition}
\begin{Remark}{}{}
    AIC is similar to BIC, but BIC strongly penalizes
    inclusion of more variables. Note that in BIC,
    $ q\ln(n) $ depends on sample size.
\end{Remark}
\begin{Remark}{$ \dagger $}{}
    If we want to measure
    just the \underline{difference} of BIC, we can do
    \[ \text{BIC}=n\ln\biggl[ \frac{\SS{Res}}{n} \biggr]+q\ln(n) \]
\end{Remark}
\underline{Recap}:
\begin{itemize}
    \item $ R^2 $, AIC, BIC
          are all based on comparing the fitted models.
          In other words, they look at the explanatory power of
          the model.
    \item They all have penalties to try to prevent
          ``overfitting.'' That is, having too many
          variables might end up modelling
          spurious relationships that are actually
          noise.
\end{itemize}

\underline{Mean Square Prediction Error} (MSPE)

Consider predictive performance of model on \emph{new}
data; that is, data \emph{not} used in fitting
of models. ``Is model generalizable to new data?''
Overfitted models tend to have high prediction error.

For example, via cross-validation schemes.
We've given 4 examples of metrics/criteria for comparing
models. Imagine we have $ p $ predictors:
\[ \begin{matrix}
        \binom{p}{1} & 1 \text{ predictor}  \\
        \binom{p}{2} & 2 \text{ predictors} \\
        \vdots                              \\
        \binom{p}{p} & p \text{ predictors} \\
    \end{matrix}
    \implies
    \sum_{j=0}^{p}\binom{p}{j}=2^p
\]
Occam's Razor: ``The simplest explanation is
usually the best one.''---William Ockham
