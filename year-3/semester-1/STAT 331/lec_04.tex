\makeheading{Lecture 4 | 2020-09-16}
\underline{Prediction for SLR}: Suppose
we want to predict the response $ y $
for a new value of $ x $. Say $ x=x_0 $. Then,
SLR model says
\[ Y_0 \sim N(\beta_0+\beta_1 x_0,\sigma^2) \]
where $ Y_0 $ is a r.v.\ for response when $ x=x_0 $.

The fitted model predicts the \emph{value} of $ y $
to be
\[ \hat{y}_0=\hat{\beta}_0+\hat{\beta}_1x_0 \]
As a random variable,
\[ \hat{Y}_0=\hat{\beta}_0+\hat{\beta}_1x_0 \]
then,
\[ \E{\hat{Y}_0}=\E{\hat{\beta}_0}+x_0\E{\hat{\beta}_1}=
    \beta_0+\beta_1x_0=\E{Y_0} \]
since $ \hat{\beta}_i $ for $ i=0,1 $ are unbiased.
We can say that $ \hat{Y}_0 $ is an unbiased estimate
of the random variable for the prediction: $ Y_0 $.

We claim that:
\[  \Var{\hat{Y}_0}=\sigma^2\left( \frac{1}{n} +\frac{(x_0-\bar{x})^2}{S_{xx}}  \right) \]
by expressing $ \hat{Y}_0=\sum\limits_{i=1}^{n} a_i Y_i $. This implies
that,
\[ \hat{Y}_0 \sim N\left( \beta_0+\beta_1x_0,
    \sigma^2\left( \frac{1}{n} +\frac{\left( x_0-\bar{x} \right)^2}{S_{xx}}  \right) \right) \]
The random variable for prediction error is
\[ Y_0-\hat{Y}_0 \]
where $ Y_0 $ and $ \hat{Y}_0 $ are independent.
\[ \E{Y_0-\bar{Y}_0}=\E{Y_0}-\E{\hat{Y}_0}=0 \]
\[ \Var{Y_0-\bar{Y}_0}=\Var{Y_0}+(-1)^2\Var{\hat{Y}_0}
    =\sigma^2+\sigma^2\left( \frac{1}{n} +\frac{\left( x_0-\bar{x} \right)^2}{S_{xx}} \right)
\]
Again, we have a linear combination of independent Normals, so
\[ Y_0-\hat{Y}_0
    \sim N\left( 0,\sigma^2\left( 1+\frac{1}{n}+\frac{\left( x_0-\bar{x} \right)^2}{S_{xx}}  \right) \right) \]
Since $ \sigma $ is unknown, we use $ \hat{\sigma} $ and get the following:
\[ \frac{Y_0-\hat{Y}_0}{
        \hat{\sigma}\sqrt{1+\dfrac{1}{n}+\dfrac{(x_0-\bar{x})^2}{S_{xx}}}
    } \sim t(n-2) \]
Intuition for prediction error composed of 2 terms:
\begin{itemize}
    \item $ \Var{Y_0} $: random error of new observation
    \item $ \Var{\hat{Y}_0} $ (predictor): estimating $ \beta_0 $ and $ \beta_1 $
\end{itemize}
Those are 2 sources of uncertainty.

\underline{Note}: Be careful that the prediction may not make sense if
$ x_0 $ is outside the range of the $ x_i $'s in the data.

$ (1-\alpha) $ prediction interval for $ y_0 $:
\[ \hat{y}_0\pm c \hat{\sigma}\sqrt{1+\dfrac{1}{n}+\dfrac{(x_0-\bar{x})^2}{S_{xx}}}
\]
where $ c $ is the $ 1-\dfrac{\alpha}{2} $ quantile of $ t(n-2) $.

\underline{Orange production 2018 in FL}
\begin{itemize}
    \item $ x $: acres
    \item $ y $: \# boxes of oranges (thousands)
    \item $ (x_i,y_i) $ recorded for each of 25 FL counties
    \item $ r=0.964 $
    \item $ \bar{x}=16133 $
    \item $ \bar{y}=1798 $
    \item $ S_{xx}=1.245\times 10^{10} $
    \item $ S_{xy}=1.453\times 10^9 $
\end{itemize}
\[ \hat{\beta}_1=\frac{S_{xy}}{S_{xx}}=0.1167 \]
which is a positive slope (positive correlation between $ x $ and $ y $).
The expected number of boxes produced is estimated to be about 117
higher per an additional acre.
\[ \hat{\beta}_0=\bar{y}-\hat{\beta}_1\bar{x}=-85.3 \]
Not meaningful to interpret, since it
is the expected production if there were 0 acres
(outside the range of $ x_i $) as no county has $ x=0 $.

Now suppose
\[ \Ss{\text{Res}}=1.31\times 10^7 \]
the residuals are the differences between $ y_i $ and the fitted regression
line.
\begin{itemize}
    \item $ \hat{\sigma}^2=\dfrac{\sum\limits_{i=1}^{n} e_i^2}{n-2}=
              \dfrac{1.31\times 10^7}{25-2}=5.7\times 10^5 $
    \item $ \Se{\hat{\beta}_1}=\dfrac{\hat{\sigma}}{\sqrt{S_{xx}}}=0.00676 $
    \item To test $ H_0 $: $ \beta_1 =0 $,
          calculate
          \[ t=\frac{\hat{\beta}_1-0}{\Se{\hat{\beta}_0}}=\frac{0.1167}{0.00676}
              \approx 17.3 \]
          Select the $ 0.975 $ quantile (for demonstration purposes) of $ t(23) $
          is $ 2.07 $.
    \item Note that $ 17.3 $ is very unlikely to see in $ t(23) $.
\end{itemize}
Since $ 17.3>2.07 $, we reject $ H_0 $ at $ \alpha=0.05 $
level, conclude there's a significant linear relationship between
acres and oranges produced.

The $ 95\% $ confidence interval for $ \beta_1 $ is
\[ 0.1167\pm 2.07(0.00676) \]
which does not contain $ 0 $.

\[ p\text{-value}=P(\abs{t_{23}}\geqslant 17.3)=
    2P(t_{23}\geqslant 17.3)\approx 1.2\times 10^{-14} \]
Predict the \# of boxes in thousands produced if we had
10000 acres to grow oranges.
\[ \hat{\beta}_0+\hat{\beta}_1x_0=-85.3+(0.1167)(10000)\approx 1082 \]
The 95\% prediction interval is:
\[ 1082\pm 2.07\sqrt{5.69\times 10^5}\sqrt{1+\frac{1}{25}+
        \frac{(6133)^2}{1.245\times 10^{10}} } \]
\underline{Note}: \textbf{not} trying to establish causation.

\underline{Check LEARN for \code{florange.csv}}.

Is $ \sigma $ the same for all values of $ y $?

It appears to be violated, can consider taking the $ \log $.

Are the error terms plausibly independent? (e.g.\
does knowing one $ e_i $ help predict $ e_j $
for a different county?)

