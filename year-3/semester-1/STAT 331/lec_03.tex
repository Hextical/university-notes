\makeheading{Lecture 3 | 2020-09-14}
Model: $ Y_i \sim N(\beta_0+\beta_1x_i,\sigma^2) $

Equation of fitted line: $ y=\hat{\beta}_0+\hat{\beta}_1x $

Interpretation:
\begin{itemize}
    \item $ \hat{\beta}_0 $ is the estimate
          of the expected response when $ x=0 $ (but not always
          meaningful if outside range of $ x_i $'s in data)
    \item $ \hat{\beta}_1 $ is the estimate of
          expected change in response for unit increase in $ x $
          \[ \hat{\beta}_1=\frac{\sum\limits_{i=1}^{n}(x_i-\bar{x})
                  \sum\limits_{i=1}^{n} (y_i-\bar{y})}{
                  \sum\limits_{i=1}^{n} (x_i-\bar{x})^2
              }=\frac{S_{xy}}{S_{xx}}   \]
    \item $ \sigma^2 $ is the
          ``variability around the line.''

          Recall that $ \sigma^2=\Var{\varepsilon_i}=\Var{Y_i} $
\end{itemize}
Q\@: How to estimate $ \sigma^2 $?
\[ \varepsilon_i=Y_i-(\beta_0+\beta_1x_i) \]
\[ e_i=y_i-(\hat{\beta}_0+\hat{\beta}_1x_i) \]
Intuition: use variability in residuals to estimate
$ \sigma^2 $.

We use \[ \hat{\sigma}^2
    =\frac{\sum\limits_{i=1}^{n} (e_i-\bar{e})^2}{n-2} \]
which looks looks like sample variance of $ e_i $'s. Therefore,
\[ \hat{\sigma}^2=\frac{\sum\limits_{i=1}^{n} e_i^2}{n-2}=
    \frac{\Ss{\text{Res}}}{n-2} \]
Note that ``Square Sum'' is abbreviated as ``Ss''. Now,
\[ \bar{e}=\bar{y}-(\hat{\beta}_0+\hat{\beta}_1\bar{x})=0 \]
The $ n-2 $ will be looked are more carefully later, but for now
it suffices to say that $ n-2= $ d.f. $ = $ number of parameters
estimated. It allows $ \hat{\sigma}^2 $ to be an unbiased estimator
for the true value of $ \sigma^2 $; that is,
\[ \E{\hat{\sigma}^2}=\sigma^2 \]
whenever $ \hat{\sigma}^2 $ is viewed as a random variable.

Q\@: Is there a statistically significant relationship?

Fact (proved using mgf in STAT 330): Suppose $ Y_i \sim N(\mu_i,\sigma_i^2) $
are all independent. Then,
\[ \sum\limits_{i=1}^{n} a_i Y_i \sim N
    \biggl( \sum\limits_{i=1}^{n} a_i\mu_i,\sum\limits_{i=1}^{n} a_i^2\sigma_i^2 \biggr) \]
for any constant $ a_i $.

In words,
\begin{quote}
    ``Linear combination of Normal is Normal.''
\end{quote}
Viewing $ \hat{\beta}_1 $ as a random variable:
\[ \hat{\beta}_1=\frac{\sum\limits_{i=1}^{n} (x_i-\bar{x})
        (Y_i-\bar{Y})}{
        \sum\limits_{i=1}^{n} (x_i-\bar{x})^2
    }
    =
    \frac{\sum\limits_{i=1}^{n} (x_i-\bar{x})Y_i-
        \bar{Y}\sum\limits_{i=1}^{n} (x_i-\bar{x})}{
        \sum\limits_{i=1}^{n} (x_i-\bar{x})x_i-\bar{x}
        \sum\limits_{i=1}^{n} (x_i-\bar{x})
    }
    =
    \frac{\sum\limits_{i=1}^{n} (x_i-\bar{x})Y_i}{
        \sum\limits_{i=1}^{n} (x_i-\bar{x})x_i
    }  \]
So,
\[ \hat{\beta}_1=\sum\limits_{i=1}^{n} a_i Y_i \]
where $ a_i=\dfrac{x_i-\bar{x}}{\sum\limits_{i=1}^{n} x_i(x_i-\bar{x})}  $.
\begin{align*}
    \E{\hat{\beta}_1}
     & = \sum\limits_{i=1}^{n} a_i\E{Y_i}                                  \\
     & =\sum\limits_{i=1}^{n}a_i(\beta_0+\beta_1x_i)                       \\
     & =\frac{\sum\limits_{i=1}^{n}(x_i-\bar{x})(\beta_0+\beta_1x_i)}{
        \sum\limits_{i=1}^{n} x_i(x_i-\bar{x})
    }                                                                      \\
     & =\frac{\beta_0 \overbrace{\sum\limits_{i=1}^{n}(x_i-\bar{x})}^{=0}+
        \beta_1\sum\limits_{i=1}^{n} x_i(x_i-\bar{x})}{
        \sum\limits_{i=1}^{n} x_i(x_i-\bar{x})
    }                                                                      \\
     & =
    \beta_1
\end{align*}
On average, $ \hat{\beta}_1 $ is an unbiased estimator for $ \beta_1 $.

Now, we calculate the variance of $ \hat{\beta}_1 $:
\begin{align*}
    \Var{\hat{\beta}_1}
     & =\sum\limits_{i=1}^{n} a_i^2\Var{Y_i}                   \\
     & =\sigma^2
    \frac{\sum\limits_{i=1}^{n} (x_i-\bar{x})^2}{
        \left[ \sum\limits_{i=1}^{n} x_i(x_i-\bar{x}) \right]^2
    }                                                          \\
     & =\sigma^2 \frac{\sum\limits_{i=1}^{n} (x_i-\bar{x})^2}{
        \left[ \sum\limits_{i=1}^{n} (x_i-\bar{x})^2 \right]^2
    }                                                          \\
     & =\frac{\sigma^2}{S_{xx}}
\end{align*}
So, since $ \hat{\beta}_1 $ is a linear combination of Normals,
\[ \hat{\beta}_1\sim
    N\left( \beta_1,\frac{\sigma^2}{S_{xx}}  \right) \]
In a similar manner,
\[ \hat{\beta}_0
    \sim N\left( \beta_0,\sigma^2
    \left(\frac{1}{n}+\frac{\bar{x}^2}{S_{xx}}\right)  \right) \]
That is, $ \hat{\beta}_0 $ and $ \hat{\beta}_1 $ are
unbiased estimates.

Then,
\[ \frac{\hat{\beta}_1-\beta_1}{\sigma/\sqrt{S_{xx}}} \sim N(0,1)  \]
However, $ \sigma $ is unknown, so need to estimate
with $ \hat{\sigma} $:
\[ \frac{\hat{\beta}_1}{\hat{\sigma}/\sqrt{S_{xx}}}
    \sim t(n-2)  \]
Since $ \Sd{\hat{\beta}_1}=\sigma^2/S_{xx} $,
we say the standard error of $ \hat{\beta}_1 $ is
$ \Se{\hat{\beta}_1}=\hat{\sigma}/\sqrt{S_{xx}} $
\begin{Definition}{Student's T-distribution}{}
    $ T $ is said to follow a \textbf{Student's T-distribution} with
    $ k $ degrees of freedom, denoted $ T \sim t(k) $, if
    \[ T=\frac{Z}{\sqrt{U/k}}  \]
    where $ Z \sim N(0,1) $ and $ U \sim \chi^2(k) $.
\end{Definition}

\underline{Fact}\@: For the simple linear regression model,
\[ \frac{\hat{\sigma}^2(n-2)}{\sigma^2}=\frac{\Ss{\text{Res}}}{\sigma^2}
    \sim \chi^2(n-2)  \]

\[ \frac{\hat{\beta}_1-\beta_1}{\hat{\sigma}/\sqrt{S_{xx}}}
    =\dfrac{\dfrac{\hat{\beta}_1-\beta_1}{\sigma/\sqrt{S_{xx}}}}{
        \sqrt{\dfrac{\hat{\sigma}^2(n-2)}{\sigma^2}\left( \dfrac{1}{n-2} \right)}
    } \sim t(n-2)  \]
A $ (1-\alpha) $ confidence interval for $ \beta_1 $ is
\[ \hat{\beta}_1\pm (c)\Se{\hat{\beta}_1} \]
where $ c $ is the $ 1-\dfrac{\alpha}{2} $ quantile
of $ t(n-2) $; that is,
\begin{itemize}
    \item $ P(\abs{T}\leqslant c)=1-\alpha $, or
    \item $ P(T\leqslant c)=1-\dfrac{\alpha}{2} $
\end{itemize}
where $ T \sim t(n-2) $.

\underline{Hypothesis test}:
$ H_0 $: $ \beta=0 $ versus
$ H_A $: $ \beta_1\neq 0 $.

If $ H_0 $ is true, then
\[ \frac{\hat{\beta}_1-\beta_1}{\Se{\hat{\beta}_1}}
    =\frac{\hat{\beta}_1}{\Se{\hat{\beta}_1}} \sim t(n-2)  \]
so calculate
\[ t=\frac{\hat{\beta}_1}{\Se{\hat{\beta}_1}}  \]
and reject $ H_0 $ at level $ \alpha $ if $ \abs{t}>c $
where $ c $ is $ 1-\dfrac{\alpha}{2} $ quantile of $ t(n-2) $.

\[ p\text{-value}=P(\abs{T}\geqslant \abs{t})
    =2P(T\geqslant \abs{t}) \]
