\makeheading{Lecture 3 | 2020-09-14}
\section{Inference}
For $ Y_i \sim \N{\beta_0+\beta_1x_i,\sigma^2} $,
the equation of fitted line
is given by $ y=\hat{\beta}_0+\hat{\beta}_1x $. Our
interpretation of the parameters is as follows.
\begin{itemize}
    \item $ \hat{\beta}_0 $ is the estimate
          of the expected response when $ x=0 $ (but not always
          meaningful if outside range of $ x_i $'s in data)
    \item $ \hat{\beta}_1 $ is the estimate of
          expected change in response for unit increase in $ x $
    \item $ \sigma^2 $ is the
          ``variability around the line''
          where $ \sigma^2=\Var{\varepsilon_i}=\Var{Y_i} $
\end{itemize}
Q\@: How should we estimate $ \sigma^2 $?
\[ \varepsilon_i=Y_i-(\beta_0+\beta_1x_i)
    \quad\text{ and }\quad e_i=y_i-(\hat{\beta}_0+\hat{\beta}_1x_i) \]
Our intuition tells us to use variability in the residuals to estimate
$ \sigma^2 $, so we use
\[ \hat{\sigma}^2
    =\frac{\sum_{i=1}^{n} (e_i-\bar{e})^2}{n-2}=
    \frac{\sum_{i=1}^{n} e_i^2}{n-2}  \]
where the first term looks like sample variance of $ e_i $'s.
The second equality follows since
$ \bar{e}=\bar{y}-(\hat{\beta}_0+\hat{\beta}_1\bar{x})=0 $
by definition of our $ \beta_0 $ estimate.
\begin{Definition}{Residual sum of squares}{}
    $ \SS{Res}=
        \sum_{i=1}^{n} (y_i-\hat{\mu}_i)^2=
        \sum_{i=1}^{n} e_i^2 $,
    is known as the \textbf{residual} (\textbf{error})
    \textbf{sum of squares}.
\end{Definition}
\begin{Remark}{}{}
    The $ n-2 $ will be looked at in more detail later, but for now
    it suffices to say that the degrees of freedom
    is $ n-2 $ or equivalently, $ n- \text{number of parameters
            estimated}$. It allows $ \hat{\sigma}^2 $ to be an unbiased estimator
    for the true value of $ \sigma^2 $; that is,
    $ \E{\hat{\sigma}^2}=\sigma^2 $
    whenever $ \hat{\sigma}^2 $ is viewed as a random variable.
\end{Remark}
\begin{Theorem}{Linear Combination of Independent Normal Random Variables}{lc_norm}
    If $ Y_i \sim \N{\mu_i,\sigma^2} $, $ i=1,\ldots,n $
    independently, then
    \[ \sum_{i=1}^{n} a_i Y_i \sim N
        \biggl( \sum_{i=1}^{n} a_i\mu_i,\sum_{i=1}^{n} a_i^2\sigma_i^2 \biggr) \]
\end{Theorem}
\begin{Proof}{\ref{thm:lc_norm}}{}
    The proof is completed in STAT 330 with moment generating functions.
\end{Proof}
Viewing $ \hat{\beta}_1 $ as a random variable:
\[ \hat{\beta}_1=\frac{\sum_{i=1}^{n} (x_i-\bar{x})
        (Y_i-\bar{Y})}{
        \sum_{i=1}^{n} (x_i-\bar{x})^2
    }
    =
    \frac{\sum_{i=1}^{n} (x_i-\bar{x})Y_i-
        \bar{Y}\Uoverbracket{\sum_{i=1}^{n} (x_i-\bar{x})}^{0}}{
        \sum_{i=1}^{n} (x_i-\bar{x})x_i-\bar{x}
        \Uunderbracket{\sum_{i=1}^{n} (x_i-\bar{x})}_{0}
    }
    =
    \frac{\sum_{i=1}^{n} (x_i-\bar{x})Y_i}{
        \sum_{i=1}^{n} (x_i-\bar{x})x_i
    }=\sum_{i=1}^{n} a_i Y_i  \]
where $ a_i=\dfrac{x_i-\bar{x}}{\sum_{i=1}^{n} x_i(x_i-\bar{x})}  $.
Therefore,
\[
    \E{\hat{\beta}_1}
    = \sum_{i=1}^{n} a_i\E{Y_i}
    =\frac{\sum_{i=1}^{n}(x_i-\bar{x})(\beta_0+\beta_1x_i)}{
        \sum_{i=1}^{n} x_i(x_i-\bar{x})
    }
    =\frac{\beta_0 \Uoverbracket{\sum_{i=1}^{n}(x_i-\bar{x})}^{0}+
        \beta_1\sum_{i=1}^{n} x_i(x_i-\bar{x})}{
        \sum_{i=1}^{n} x_i(x_i-\bar{x})
    }
    =
    \beta_1
\]
Now, we calculate the variance of $ \hat{\beta}_1 $:
\[
    \Var{\hat{\beta}_1}
    =\sum_{i=1}^{n} a_i^2\Var{Y_i}
    =\sigma^2
    \frac{\sum_{i=1}^{n} (x_i-\bar{x})^2}{
        \left[ \sum_{i=1}^{n} x_i(x_i-\bar{x}) \right]^2
    }
    =\sigma^2 \frac{\sum_{i=1}^{n} (x_i-\bar{x})^2}{
        \left[ \sum_{i=1}^{n} (x_i-\bar{x})^2 \right]^2
    }
    =\frac{\sigma^2}{S_{xx}}
\]
Using our calculations from $ \hat{\beta}_1 $,
and viewing $ \hat{\beta}_0 $ as a random variable:
\[ \E{\hat{\beta}_0}=
    \E{\bar{Y}}-\bar{x}\E{\hat{\beta}_1}=
    \E*{\frac{\sum_{i=1}^{n}Y_i}{n}}-\bar{x}\beta_1=
    \frac{\sum_{i=1}^{n}(\beta_0+\beta_1x_i)}{n}-
    \beta_1\bar{x}=\beta_0+\beta_1\bar{x}-\beta_1\bar{x}=\beta_0
\]
Now, we calculate the variance of $ \hat{\beta}_0 $:
\[
    \Var{\hat{\beta}_1}
    =\Var{\bar{Y}-\beta_1\bar{x}}
    =\Var{\bar{Y}}+(-\bar{x}^2)\Var{\beta_1}
    =\Var*{\frac{\sum_{i=1}^{n} Y_i}{n}}+\bar{x}^2\left( \frac{\sigma^2}{S_{xx}} \right)
    =\frac{n\sigma^2}{n^2}+\frac{\sigma^2 x^2}{S_{xx}}
\]
Also, since $ \hat{\beta}_1 $ and $ \hat{\beta}_0 $
are linear combination of Normal random variables,
they follow a Normal distribution. Therefore, we get the following theorem.
\begin{Theorem}{Distribution of LSEs}{}
    The distribution of the least square estimates are given by
    \[ \hat{\beta}_1\sim
        N\left( \beta_1,\frac{\sigma^2}{S_{xx}}  \right)\quad\text{ and }\quad
        \hat{\beta}_0
        \sim N\left( \beta_0,\sigma^2
        \left(\frac{1}{n}+\frac{\bar{x}^2}{S_{xx}}\right)  \right) \]
\end{Theorem}
Since $ \E{\hat{\beta}_1}=\beta_1 $,
we say $ \hat{\beta}_1 $ is an unbiased
estimator of $ \beta_1 $. This implies
that as $ n\to\infty $, the average
of the estimates $ \hat{\beta}_1 $; that is,
$ \E{\hat{\beta}_1} $ coincides with the true
value of $ \beta_1 $. A similar argument can be made for $ \beta_0 $.

Then, $ \dfrac{\hat{\beta}_1-\beta_1}{\sigma/\sqrt{S_{xx}}} \sim \N{0,1} $,
but $ \sigma $ is unknown, so need to use $ \hat{\sigma} $
to get
$ \dfrac{\hat{\beta}_1-\beta_1}{\hat{\sigma}/\sqrt{S_{xx}}}
    \sim t(n-2) $.
\begin{Definition}{Standard deviation and standard error of $ \hat{\beta}_1 $}{}
    The \textbf{standard deviation} of $ \hat{\beta}_1 $
    is defined as $ \Sd{\hat{\beta}_1}=\sigma/\sqrt{S_{xx}} $.
    The \textbf{estimated} standard deviation of $ \hat{\beta}_1 $
    is also referred to as the \textbf{standard error}
    of the estimate $ \hat{\beta}_1 $, and
    we write $ \Se{\hat{\beta}_1}=\hat{\sigma}/\sqrt{S_{xx}} $.
\end{Definition}
\begin{Definition}{Student $ t $ distribution}{}
    Suppose $ Z \sim \N{0,1} $
    and $ U \sim \chi^2(\nu) $,
    with $ Z $ and $ U $ independent. Then,
    $ T=Z/\sqrt{U/\nu} $ has a \textbf{Student $t$ distribution}
    with $ \nu $ degrees of freedom.
\end{Definition}
\begin{Theorem}{}{ss_dist}
    For a simple linear regression model,
    \[ \frac{\hat{\sigma}^2(n-2)}{\sigma^2}=\frac{\SS{Res}}{\sigma^2}
        \sim \chi^2(n-2) \]
\end{Theorem}
\begin{Proof}{\ref{thm:ss_dist}}{}
    Too hard for sure.
\end{Proof}
Using the theorem stated, we justify
the fact that replacing $ \sigma $ with $ \hat{\sigma} $
gives us a $ t(n-2) $ distribution.
\[ \frac{\hat{\beta}_1-\beta_1}{\hat{\sigma}/\sqrt{S_{xx}}}
    =\dfrac{\dfrac{\hat{\beta}_1-\beta_1}{\sigma/\sqrt{S_{xx}}}}{
        \sqrt{\dfrac{\hat{\sigma}^2(n-2)}{\sigma^2}\left( \dfrac{1}{n-2} \right)}
    }=
    \frac{Z}{\sqrt{U/\nu}}=T \sim t(n-2)  \]
where $ \dfrac{\hat{\sigma}^2(n-2)}{\sigma^2}=U $,
$ \nu=n-2 $, and $ Z=\dfrac{\hat{\beta}_1-\beta_1}{\hat{\sigma}/\sqrt{S_{xx}}} $.
A $ (1-\alpha) $ confidence interval for $ \beta_1 $ is
\[ \hat{\beta}_1\pm c\,\Se{\hat{\beta}_1} \]
where $ c $ is the $ 1-\dfrac{\alpha}{2} $ quantile
of $ t(n-2) $; that is,
$ P(\abs{T}\leqslant c)=1-\alpha $ or
$ P(T\leqslant c)=1-\dfrac{\alpha}{2} $
where $ T \sim t(n-2) $.

\underline{Hypothesis test}:
$ H_0 $: $ \beta=0 $ versus
$ H_A $: $ \beta_1\neq 0 $.
If $ H_0 $ is true, then $ \hat{\beta}_1/\Se{\hat{\beta}_1} \sim t(n-2) $,
so calculate the \textbf{t statistic}
$ t=\hat{\beta}_1/\Se{\hat{\beta}_1} $,
and reject $ H_0 $ at level $ \alpha $ if $ \abs{t}>c $
where $ c $ is $ 1-\dfrac{\alpha}{2} $ quantile of $ t(n-2) $.
Therefore,
$ p\text{-value}=P(\abs{T}\geqslant \abs{t})
    =2P(T\geqslant \abs{t}) $.
