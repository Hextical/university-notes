\makeheading{Lecture 7 | 2020-09-28}
Recall that $ \symbf{Y}=X\symbf{\beta}+\symbf{\varepsilon}
    \sim \text{MVN}(X\symbf{\beta},\sigma^2 I) $, and
\begin{itemize}
    \item Estimates: $ \hat{\symbf{\beta}}=(X^\top X)^{-1}X^\top \symbf{Y} $
    \item Fitted values: $ \hat{\symbf{\mu}}=X\hat{\symbf{\beta}} $
    \item Residuals: $ \symbf{e}=\symbf{y}-\hat{\symbf{\mu}} $
    \item Constants: $ X=\begin{bmatrix}
                  \symbf{1} & \symbf{x}_1 & \cdots & \symbf{x}_p
              \end{bmatrix}_{n\times(p+1)} $
    \item Values of responses: $
              \symbf{y}=(y_1,y_2,\ldots,y_n)^\top \in\mathbb{R}^n $
\end{itemize}
\underline{Author's Note}: Geometric interpretation of data is omitted in these notes because
I'm simply too lazy.

The span of $ X $ is $ \text{Span}(X)=
    \set{b_0\symbf{1}+b_1\symbf{x}_1+\cdots+b_p\symbf{x}_p:b_0,\ldots,b_p\in\mathbb{R}}
    \subset \mathbb{R}^n $
which is all linear combinations of columns of $ X $ which is a subspace
of $ \mathbb{R}^n $, and by assumption we know $ \rank(X)=p+1 $.

We can say $ \text{Span}(X) $ represents all possible vector
values $ X\symbf{b} $ where $ \symbf{b}=(b_0,b_1,\ldots,b_p)^\top $.

Generally, $ \symbf{y}\notin\text{Span}(X) $, so since
the linear model is an approximation, $ \symbf{\varepsilon} $
variability not explained by model.

Intuitively, it makes sense to choose an estimate
$ \hat{\symbf{\beta}} $ so that $ X\hat{\symbf{\beta}} $
is as close to $ \symbf{y} $ as possible. Therefore,
$ \symbf{e} $ must be orthogonal to $ \text{Span}(X)
    \iff \symbf{e} $ is orthogonal to all columns of $ X $.
\begin{align*}
    \symbf{1}^\top\cdot(\symbf{y}-\hat{\symbf{\mu}})   & =      0 \\
    \symbf{x}_1^\top\cdot(\symbf{y}-\hat{\symbf{\mu}}) & =      0 \\
    \vdots                                                        \\
    \symbf{x}_p^\top\cdot(\symbf{y}-\hat{\symbf{\mu}}) & =      0
\end{align*}
which is the same as LS estimates. We also know
$ \hat{\symbf{\mu}}=X\hat{\symbf{\beta}} $ and
$ \symbf{e}=\symbf{y}-\hat{\symbf{\mu}} $.
\begin{Definition}{Hat matrix}{}
    The \textbf{hat matrix} is defined as
    $ H=X(X^\top X)^{-1}X^{\top} $.
\end{Definition}

\begin{Proposition}{Properties of Hat Matrix}{}
    Let $ H $ be a hat matrix, then $ H $ has the following properties.
    \begin{enumerate}[label=(\arabic*)]
        \item $ H $ is symmetric; that is, $ H=H^\top $.
        \item $ H $ is idempotent; that is, $ H^2=HH=H $.
        \item $ I-H $ is symmetric idempotent; that is,
              $ (I-H)^2=(I-H)(I-H)=I-H $.
    \end{enumerate}
\end{Proposition}
\begin{proof} We prove all three because it's easy.
    \begin{enumerate}[label=(\arabic*)]
        \item $ H^\top=[X(X^\top X)^{-1}X^\top]^\top=X(X^\top X)^{-1}X^\top=H $.
        \item  $ HH=X(X^\top X)^{-1}(X^\top X)(X^\top X)^{-1}X^\top=H $.
        \item  $ (I-H)(I-H)=I(I-H)-H(I-H)=II-IH-HI+HH=I-2H+HH=I-2H+H=I-H $.
    \end{enumerate}
\end{proof}
Let's view $ \hat{\symbf{\mu}} $ and $ \symbf{e} $
as random vectors
\[ \hat{\symbf{\mu}}=X\hat{\symbf{\beta}}=
    X(X^\top X)^{-1}X^\top \symbf{Y}=H\symbf{Y} \]
\[ \symbf{e}=\symbf{Y}-\hat{\symbf{\mu}}=I\symbf{Y}-H\symbf{Y}=
    (I-H)\symbf{Y} \]
\[ \E{\hat{\symbf{\mu}}}=\E{H\symbf{Y}}=H\E{\symbf{Y}}=
    X(X^\top X)^{-1}X^\top \Uunderbracket{X\symbf{\beta}}_{\E{\symbf{Y}}}
    =X\symbf{\beta}
\]
\[ \Var{\hat{\symbf{\mu}}}=\Var{H\symbf{Y}}=
    H\Var{\symbf{Y}}H^\top=H\sigma^2I H^\top=\sigma^2(HH^\top)=\sigma^2H \]
\[ \E{ \symbf{e}}=
    \E{(I-H)\symbf{Y}}=\E{\symbf{Y}}-\E{H\symbf{Y}}=X\symbf{\beta}-X\symbf{\beta}=0 \]
\[ \Var{\symbf{e}}=
    (I-H)\Var{\symbf{Y}}(I-H)^\top=\sigma^2(I-H)(I-H)^\top=\sigma^2(I-H) \]
So since $ \hat{\symbf{\mu}} $ and $ \symbf{e} $
are linear transformations of $ \symbf{Y} $ we have proved the following theorem.
\begin{Theorem}{Distribution of $ \hat{\symbf{\mu}} $ and $ \symbf{e} $}{}
    $ \hat{\symbf{\mu}} $ and $ \hat{\symbf{e}} $ have the following distribution.
    \[ \hat{\symbf{\mu}}\sim\text{MVN}(X\symbf{\beta},\sigma^2 H) \]
    \[ \hat{\symbf{e}}\sim\text{MVN}(0,\sigma^2(I-H)) \]
\end{Theorem}
Suppose we want to predict response for $ \symbf{x}_0 $
where the first 1 represents the intercept in the row vector.
\[ \symbf{x}_0=\begin{bmatrix}
        1 & x_{01} & x_{02} & \cdots & x_{0p}
    \end{bmatrix}_{1\times (p+1)} \]
Let $ Y_0 $ random variable representing the response
associated with $ \symbf{x}_0 $. The MLR says
\[ Y_0 \sim N(\beta_0+\beta_1x_{01}+\cdots+\beta_p x_{0p},\sigma^2) \]
So we predict the value
\[ \hat{y}_0=\hat{\beta}_0+\hat{\beta}_1x_{01}+\cdots+
    \hat{\beta}_p x_{0p}=\symbf{x}_0\hat{\symbf{\beta}} \]
which represents the estimated mean response given
$ x_{01},x_{02},\ldots,x_{0p} $. Corresponding distribution
has
\[ \E{\hat{Y}_0}=\symbf{x}_0\E{\hat{\symbf{\beta}}}=\symbf{x}_0\symbf{\beta}
    =\E{Y_0} \]
\[ \Var{\hat{Y}_0}=\symbf{x}_0\Var{\hat{\symbf{\beta}}}\symbf{x}_0^\top=
    \symbf{x}_0\sigma^2(X^\top X)^{-1}\symbf{x}_0^\top \]
We have proved the following theorem.
\begin{Theorem}{Distribution of Predictor}{}
    The distribution of $ \hat{Y}_0 $ which is a function
    of $ Y_1,\ldots,Y_n $ is
    \[ \hat{Y}_0 \sim N(\symbf{x}_0\symbf{\beta},\sigma^2\symbf{x}_0(X^\top X)^{-1}
        \symbf{x}_0^\top) \]
\end{Theorem}
\[ \frac{\hat{Y}_0-\symbf{x}_0\symbf{\beta}}{\sigma\sqrt{
            \symbf{x}_0(X^\top X)^{-1}\symbf{x}_0^\top
        }}\sim N(0,1)  \]

\[ \frac{\hat{Y}_0-\symbf{x}_0\symbf{\beta}}{\hat{\sigma}\sqrt{
            \symbf{x}_0(X^\top X)^{-1}\symbf{x}_0^\top
        }}\sim t(n-(p+1))=t(n-p-1)  \]
A $ (1-\alpha) $ confidence interval for the mean
response $ y_0=\symbf{x}_0\hat{\symbf{\beta}} $ given $ \symbf{x}_0 $ is
\[ \hat{y}_0\pm c \hat{\sigma}\sqrt{
        \symbf{x}_0(X^\top X)^{-1}\symbf{x}_0^\top} \]
where $ c $ is the $ 1-\alpha/2 $ quantile of $ t(n-p-1) $.

Prediction error: $ Y_0-\hat{Y}_0 $ which are independent
since $ Y_0 $ is a random variable with variance $ \sigma^2 $
and $ \hat{Y}_0 $ is a function of $ Y_1,\ldots,Y_n $. Therefore,
\[ \E{Y_0-\hat{Y}_0}=\symbf{x}_0\symbf{\beta}-\symbf{x}_0\symbf{\beta}=0 \]
\[ \Var{Y_0-\hat{Y}_0}=\Var{Y_0}+(-1)^2\Var{\hat{Y}_0}=
    \sigma^2+\sigma^2(\symbf{x}_0(X^\top X)^{-1}\symbf{x}_0^\top) \]
We have proved the following theorem.
\begin{Theorem}{Distribution of Prediction Error}{}
    The distribution of the prediction error is
    \[ Y_0-\hat{Y}_0 \sim N(
        0,\sigma^2(1+\symbf{x}_0(X^\top X)^{-1}\symbf{x}_0^\top)
        ) \]
\end{Theorem}

A $ (1-\alpha) $ prediction interval for the mean
response $ y_0=\symbf{x}_0\hat{\symbf{\beta}} $ given $ \symbf{x}_0 $ is
\[ \hat{y}_0\pm c\hat{\sigma}\sqrt{1+\symbf{x}_0(X^\top X)^{-1}\symbf{x}_0^\top} \]
where $ c $ is the $ 1-\alpha/2 $ quantile of $ t(n-p-1) $.

\begin{Remark}{}{}
    Our intuition tells us that the prediction interval is wider than
    the confidence interval for mean. In other words, estimating
    an average is ``easier'' than an individual response.
\end{Remark}
