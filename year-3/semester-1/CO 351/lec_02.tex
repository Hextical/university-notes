\chapter{Review}
\makeheading{Lecture 2 | 2020-09-08}

\underline{Definitions}:
\begin{Definition}{Graph, Vertices, Edges}{}
    A \textbf{graph} $ G=(V,E) $ consists of a set of \textbf{vertices}
    $ V $, and a set of \textbf{edges} $ E $ that are unordered pairs of vertices.
\end{Definition}
\begin{Definition}{Degree}{}
    The degree of a vertex $ v $ is the number of edges incident with $ v $,
    denoted $ d_G(v) $ or $ d(v) $.
\end{Definition}
\begin{Definition}{Walk}{}
    A \textbf{walk} is a sequence of vertices $ v_1,v_2,\ldots,v_k $
    where $ v_i v_{i+1} $ are edges.
\end{Definition}
\begin{Definition}{Path}{}
    A \textbf{path} is a walk where all vertices are distinct.
\end{Definition}
\begin{Definition}{Cycle}{}
    A \textbf{cycle} is a walk $ v_1,v_2,\ldots,v_k,v_1 $
    where $ v_1,\ldots,v_k $ are distinct and $ k\geqslant 3 $.
\end{Definition}

\underline{Connectivity and cuts}:
\begin{Definition}{Connected, Disconnected}{}
    A graph is \textbf{connected} if there is a path between every pair of vertices.
    Otherwise it is \textbf{disconnected.}
\end{Definition}
\begin{Definition}{Cut}{}
    For $ S\subseteq V $, the cut induced by $ S $ is the set of all edges with exactly one end
    in $ S $.
    \[ \delta(S)=\set{uv\in E: u\in S, v\notin S} \]
\end{Definition}
\begin{Definition}{$ s,t $-cut}{}
    If $ s,t\in V $ where $ s\in S $ and $ t\notin S $, then $ \delta(S) $ is an
    \textbf{$\bm{s}$,$\bm{t}$-cut}.
\end{Definition}
\begin{Proposition}{}{}
    There exists an $ s,t $-path if and only of every $ s,t $-cut
    is non-empty.
\end{Proposition}

\underline{Trees}:
\begin{Definition}{Tree}{}
    A \textbf{tree} is a connected graph with no cycles.
\end{Definition}
\begin{Proposition}{}{}
    A tree with $ n $ vertices has $ n-1 $ edges.
\end{Proposition}
\begin{Definition}{Spanning tree}{}
    A \textbf{spanning tree} of a graph $ G $ is a subgraph
    that is a tree which uses all vertices of $ G $.
\end{Definition}
\begin{Proposition}{}{}
    $ G $ has a spanning tree if and only if $ G $ is connected.
\end{Proposition}
\begin{Proposition}{}{}
    If $ T $ is a tree and $ u,v $ are not adjacent, then
    $ T+uv $ has exactly one cycle $ C $. If $ xy $ is an edge of
    $ C $, then $ T+uv-xy $ is another tree.
\end{Proposition}
