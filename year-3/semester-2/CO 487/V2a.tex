\chapter*{V2: Symmetric-Key Cryptography}
\addcontentsline{toc}{chapter}{V2: Symmetric-Key Cryptography}
\setcounter{chapter}{1}

\section*{V2a: Basic Concepts}
\addcontentsline{toc}{section}{V2a: Basic Concepts}
\setcounter{section}{1}

\begin{Definition}{Symmetric-key Encryption Scheme (SKES)}{}
    A \textbf{symmetric-key encryption scheme} (SKES) consists of:
    \begin{itemize}
        \item $ M $: the plaintext space,
        \item $ C $: the ciphertext space,
        \item $ K $: the key space,
        \item a family of encryption functions, $ E_k $: $ M\to C $,
              $ \forall k\in K $,
        \item a family of decryption functions, $ D_k $: $ C\to M $,
              $ \forall k \in K $, such that $ D_k(E_k(m))=m $ for
              all $ m\in M $, $ k\in K $.
    \end{itemize}
\end{Definition}
Using a SKES to Achieve Confidentiality
\begin{enumerate}
    \item Alice and Bob agree on a \emph{secret key} $ k\in K $
          by communicating over the \emph{secure channel}.
    \item Alice computes $ c=E_k(m) $ and sends the ciphertext
          $ c $ to Bob over the \emph{unsecured channel}.
    \item Bob retrieves the plaintext by computing $ m=D_k(c) $.
\end{enumerate}
