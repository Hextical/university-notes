\section*{V1b: Course Outline}
\addcontentsline{toc}{section}{V1b: Course Outline}
\setcounter{section}{1}

\subsection*{About the Course}
\begin{itemize}
    \item Coverage will \underline{favour breadth at the expense of depth}
          \begin{itemize}
              \item For depth, try the \underline{optional} readings
              \item See also: \emph{CO 485} (Mathematics of Public-Key Crypto)
              \item See also: \emph{CS 458} (Computer Security and Privacy)
          \end{itemize}
    \item This course is not a traditional textbook course!
          \begin{itemize}
              \item \emph{Watching the video lectures} is strongly recommended
              \item There are no good sources of ``practice questions''
              \item \emph{Your job}: Identify and understand the important (technical
                    and non-technical) \emph{concepts} presented in the lectures
          \end{itemize}
    \item \underline{Optional text}: Understanding Cryptography
          \begin{itemize}
              \item Available for free download from the UW library
              \item \underline{Optional} readings are posted on the course website
          \end{itemize}
\end{itemize}
\subsection*{Learning Outcomes}
\begin{enumerate}
    \item Understand the fundamental cryptographic tools of
          symmetric-key encryption, message authentication,
          authenticated encryption, hash functions, public-key
          encryption, and signatures;
    \item Appreciate the challenges with assessing the security
          of these tools;
    \item Gain exposure to how these cryptographic tools are
          used to secure large-scale applications;
    \item Understand why key management is an essential
          process that underpins the security of many
          applications.
\end{enumerate}
\subsection*{Course Administration}
\begin{itemize}
    \item \underline{Website}: http://learn.uwaterloo.ca
          \begin{itemize}
              \item Course outline and \emph{weekly schedule}.
              \item Videos, slides, assignments \& solutions, handouts.
              \item Optional readings.
          \end{itemize}
    \item \underline{Assistance}:
          \begin{itemize}
              \item Instructor and TA online office hours (see LEARN).
              \item Piazza (see the Course Outline for the password).
              \item Participation is \emph{strongly encouraged}, but optional.
          \end{itemize}
    \item \underline{Communications}: All important course announcements will
          be sent to you by \emph{email}. (Apologies in advance for the spam.)
    \item \underline{Assignments} (50\%): submitted via Crowdmark.
          \begin{itemize}
              \item Due dates: \emph{Jan 29}, \emph{Feb 12}, \emph{Mar 5}, \emph{Mar 26}, \emph{Apr 9}.
          \end{itemize}
    \item \underline{Quizzes} (15 / 7.5 / 0\%): \emph{Feb 26} and \emph{Mar 19} (on LEARN).
    \item \underline{Final Assessment}\@: (35 /42.5 / 50\%).
\end{itemize}
\subsection*{Academic Integrity}
\begin{itemize}
    \item \underline{Assignments}:
          \begin{itemize}
              \item No collaboration permitted on one or two questions.
              \item Collaboration permitted (and \emph{encouraged}) on the other
                    questions (but solutions must be written up independently).
          \end{itemize}
    \item \underline{Quizzes}:
          \begin{itemize}
              \item \emph{Open book}: slides, your notes, assignments, solutions
                    (ONLY).
          \end{itemize}
    \item \underline{Final Assessment}:
          \begin{itemize}
              \item \emph{Open book}: slides, your notes, assignments, solutions
                    (ONLY).
          \end{itemize}
    \item \underline{General}: no cheating, Course Hero, Chegg, online discussion
          groups, solutions from previous course offerings, etc.
    \item \underline{Policy 71} \emph{will be enforced}.
\end{itemize}
\subsection*{Should You Take This Course?}
\emph{Course philosophy}: CO 487 is an \emph{elective course} for everyone.

As such, it is intended to be:
\begin{itemize}
    \item \emph{interesting},
    \item \emph{fun},
    \item \emph{relevant}, and
    \item \emph{not-too-difficult}.
\end{itemize}
\subsection*{Amount of Material}
\emph{Course content}:
\begin{itemize}
    \item 65\%: Material you need to know for quizzes and the final exam.
    \item 25\%: Show-and-tell, including ``Crypto Cafe.''
    \item 10\%: Cryptography gossip.
\end{itemize}
\emph{Guidance for the quizzes and the final exam}:

There will not be any ``unreasonable'' expectations of what you need
to memorize for the quizzes and the final assessment. For example,
you do \emph{not} need to memorize the descriptions of RC4, the WEP
security protocol, ChaCha20, AES, AES-GCM, SHA-256,
RSA-OAEP, the elliptic curve addition formulae, ECDSA, the
Bluetooth security protocol, the Signal protocol, etc. If asked any
questions on such topics, then all the relevant background
information will be provided. You also don't need to memorize any
\emph{trivia}, e.g.\ historical notes, dates, spelling out of acronyms, etc.
