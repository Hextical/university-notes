\section{Lecture 44.00: Stratified, Allocation}
Today we're going to talk about something called \emph{Allocation}.
For example, allocation is when you have a sample of $100$ units,
and you have four strata. How should you spend those $100$ units?
Should three quarters of them go to one stratum, and the remaining quarter be split
among the last three strata? We define how you divide them up to be stratification.
To make that decision there are two types that we're going to talk about today.
\begin{enumerate}[(1)]
    \item Proportional.
    \item Neyman or optimal.
\end{enumerate}
\subsection*{Proportional}
We sample based on the size of the strata. In other words, the bigger
the strata size, the bigger the sample size.
\[ n_h=w_h n \]
\begin{Example}{}{}
    \[ \begin{matrix}
            \text{Provinces} & \text{Population (millions)} \\
            \midrule
            \text{ON}        & 10                           \\
            \text{QUE}       & 5                            \\
            \text{BC}        & 3                            \\
            \text{ALB}       & 2                            \\
            \midrule
            \text{Total}     & 20
        \end{matrix} \]
    If we have $ n=100 $ units to sample, ON should get $ n_{\text{ON}}=w_{\text{ON}}(n)=1/2(100)=50 $
    units.
\end{Example}
\subsection*{Neyman}
In Neyman allocation, we select our sample size and values that minimize the stratified variance.
\[ \Var{\tilde{\mu}}=\sum_{i=1}^{H} w_i^2 \frac{\sigma_i^2}{n_i} \biggl(1-\frac{n_i}{N_i} \biggr) \]
subject to $ n=n_1+n_2+\cdots+n_H $. This ends up being a Lagrange multiplication problem.
So minimize
\[ W(\tilde{\mu})=\sum_{i=1}^{H} w_i^2 \frac{\sigma_i^2}{n_i} \biggl(1-\frac{n_i}{N_i} \biggr)+
    \lambda(n-n_1-\cdots-n_H) \]
Find $ \pdv{W}{\lambda},\pdv{W}{n_i} $ and set to zero to get:
\[ n_i=\frac{n\sigma_i w_i}{\sum_{j=1}^{H} \sigma_j w_j}  \]
\begin{Remark}{}{}
    \begin{itemize}
        \item $ n_i \propto \sigma_i $. So, if you have more variability in your stratum,
              then you're going to want a larger sample size. That should make a lot of
              sense because if you have a lot of variability, you can reduce the variability
              by having a larger sample size, remember, the variability
              is the average of the variance divided by $ n $.
              So, the larger your sample size the smaller the variance ends up being.
        \item $ n_i\propto w_i $. The larger the strata, the more
              units you will want allocated to it.
        \item If $ \sigma_1=\sigma_2=\cdots=\sigma_H $, then
              \[ n_i=\frac{n w_i}{\sum_{i=1}^{H} w_i}=n w_i  \]
              which is proportional allocation since $ \sum_{i=1}^{H} w_i=1 $.
    \end{itemize}
\end{Remark}
Just like when we did with the small sample size, we take a small sample, and you use the small sample
to estimate these unknown $ \sigma $'s. Once you've estimated these unknown $ \sigma $'s,
you'd use them to determine how you should allocate your larger sample size to the actual strata of interest.
