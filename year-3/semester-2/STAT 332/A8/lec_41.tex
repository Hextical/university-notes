\section{Lecture 41.00: Ratio Estimation}
Suppose we had six students in our class. We selected them out at random
using SRS\@. We record their grade in Calculus 1 and their gender.
\[ \begin{matrix}
        \text{Gender} & \text{Grade} \\
        \midrule
        M             & 70           \\
        F             & 70           \\
        M             & 85           \\
        F             & 85           \\
        M             & 90           \\
        F             & 90
    \end{matrix} \]
The males on average: $ \bar{x}_M=(70+85+90)/3 $.
Let
\begin{itemize}
    \item $ y_i $ be the grade, and
    \item $ z_i=\begin{cases}
                  1 & \text{ if male}  \\
                  0 & \text{otherwise}
              \end{cases} $
\end{itemize}
Our estimate is $ \hat{\theta} $,
\[ \hat{\theta}=\frac{\sum_{i\in\mathcal{S}}y_i z_i}{\sum_{i\in \mathcal{S}}z_i }  \]
\begin{itemize}
    \item $ \sum_{i\in\mathcal{S}}y_i z_i $ counts the number of grades of those males.
    \item $ \sum_{i\in\mathcal{S}} z_i $ counts the number of males you happen to have.
\end{itemize}
Our parameter is $ \theta $,
\[ \theta=\frac{\sum_{i=1}^{N} \frac{y_i z_i}{N} }{\sum_{i=1}^{N} \frac{z_i}{N} }=\frac{\mu}{\pi}   \]
\begin{itemize}
    \item $ \mu $ is the average of the male grades.
    \item $ \pi $ is the proportion of the people that are male.
\end{itemize}
Therefore, we can write our estimate as $ \hat{\theta}=\hat{\mu}/\hat{\pi} $.
\subsection*{Estimator}
Our estimator asks ``what's random?''
What's random is whether you're in the sample.
We define an indicator variable $ I_i $ which will be 1 if it's in our population.
\[ \tilde{\theta}=\frac{\frac{\sum_{i=1}^{N}I_i y_i z_i}{n} }{\frac{\sum_{i=1}^{N} I_i z_i}{n}}=\frac{\tilde{\mu}}{\tilde{\pi}}   \]
Now there's a bit of math involved in this because unfortunately we have never
looked at the ratio of two random variables, and it's very difficult to do.
Instead, we're going to use \emph{Taylor's Approximation} (Lecture 41.50 goes more in detail).

Taylor's Approximation gives:
\[ \frac{\tilde{\mu}}{\tilde{\pi}}\approx \frac{\mu}{\pi}+\frac{1}{\pi}(\tilde{\mu}-\mu)-\frac{\mu}{\pi^2}(\tilde{\pi}-\pi)  \]
where $ \tilde{\mu} $ and $ \tilde{\pi} $ are both obtained by SRS\@. So we obtain the proportion and average from simple random sampling.
\begin{itemize}
    \item The approximation is approximately normal (there's a Gaussian extension that allows
          this to be true).
\end{itemize}
\subsection*{Expectation and Variance}
\begin{align*}
    \E*{\frac{\tilde{\mu}}{\tilde{\pi}}}
     & \approx \E*{\frac{\mu}{\pi}+\frac{1}{\pi}(\tilde{\mu}-\mu)-\frac{\mu}{\pi^2}(\tilde{\pi}-\pi)}                         \\
     & =\frac{\mu}{\pi}+\frac{1}{\pi} \biggl(\E*{\tilde{\mu}}-\mu\biggr)-\frac{\mu}{\pi^2} \biggl(\E*{\tilde{\pi}}-\pi\biggr) \\
     & =\frac{\mu}{\pi}
\end{align*}
since $ \E{\tilde{\mu}}=\mu $ and $ \E*{\tilde{\pi}}=\pi $ (by SRS, unbiased).
\begin{align*}
    \Var*{\frac{\tilde{\mu}}{\tilde{\pi}}}
     & \approx \Var*{\frac{\mu}{\pi}+\frac{1}{\pi}(\tilde{\mu}-\mu)-\frac{\mu}{\pi^2}(\tilde{\pi}-\pi)} \\
     & =\frac{1}{\pi^2} \Var{\tilde{\mu}-\mu-\frac{\mu\tilde{\pi}}{\pi}+\mu}                            \\
     & =\frac{1}{\pi^2} \Var*{\tilde{\mu}-\frac{\mu \tilde{\pi}}{\pi}}
\end{align*}
This is an average. Therefore, we end up with an estimated variance.
\[  \Var*{\frac{\tilde{\mu}}{\tilde{\pi}}}\approx \frac{1}{\pi^2} \frac{\sigma^2_{\text{ratio}}}{n}\biggl(1-\frac{n}{N}\biggr)  \]
A confidence interval is:
\[ \text{EST}\pm c\,\text{SE}=\hat{\theta}\pm c\, \frac{1}{\hat{\pi}} \sqrt{1-\frac{n}{N}}\frac{\hat{\sigma}_{\text{ratio}}}{\sqrt{n}}  \]
where
\[ \hat{\sigma}_{\text{ratio}}^2=\frac{\sum_{i\in\mathcal{S}}\bigl(y_i-\hat{\theta}z_i\bigr)^2 }{n-1}  \]
\section{Lecture 41.50: Taylor's Approximation}
\emph{Calculus 1}: $ f(x)\approx f(x_0)+f^\prime(x_0)(x-x_0) $.
\begin{Example}{Calculus 1 Taylor's Approximation}{}
    Approximate $ f(1.1)=\ln(1.1) $ about $ x_0=1 $.

    \textbf{Solution.}
    \[ f(1.1)\approx f(1)+f^\prime(1)(1.1-1)=\ln(1)+\eval{\frac{1}{x}}_{x=1}(1.1-1)=0+1(0.1)=0.1 \]
\end{Example}
\emph{Calculus 3}:
\[ f(x,y)\approx f(x_0,y_0)+\pd{f(x_0,y_0)}{x}(x-x_0)+\pd{f(x_0,y_0)}{y}(y-y_0) \]
\begin{Example}{Calculus 3 Taylor's Approximation}{}
    Approximate $ f(1.1,1.1)=\ln(1.1\times 1.1) $ about the point $ (1,1) $.

    \textbf{Solution.}
    \begin{align*}
        f(1.1,1.1)
         & \approx f(1,1)+\dpd{f(1,1)}{x}(x-x_0)+\dpd{f(1,1)}{y}(y-y_0)                    \\
         & =\ln(1)+\eval{\frac{1}{x}}_{x=1,y=1}(1.1-1)+\eval{\frac{1}{y}}_{x=1,y=1}(1.1-1) \\
         & =0+0.1+0.1                                                                      \\
         & =0.2
    \end{align*}
\end{Example}
Approximate: $ f(x,y)=x/y $ about the point $ (x_0,y_0) $.
\begin{align*}
    f(x,y)
     & \approx f(x_0,y_0)+\frac{1}{y_0}(x-x_0)+\biggl(-\frac{x_0}{y_0^2} \biggr)(y-y_0) \\
     & =\frac{x_0}{y_0}+\frac{1}{y_0} (x-x_0)-\frac{x_0}{y_0^2}(y-y_0)
\end{align*}
Therefore, approximating $ \tilde{\mu}/\tilde{\pi} $ about $ (\mu,\pi) $:
\[ \frac{\tilde{\mu}}{\tilde{\pi}}\approx \frac{\mu}{\pi}+\frac{1}{\pi} (\tilde{\mu}-\mu)-\frac{\mu}{\pi^2}(\tilde{\pi}-\pi)   \]
