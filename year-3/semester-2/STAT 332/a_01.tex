\chapter{Assignment 1}
\section{Lecture 1.00 - PPDAC + Example}
PPDAC\@: \underline{P}roblem, \underline{P}lan, \underline{D}ata,
\underline{A}nalysis, \underline{C}onclusion.
\begin{itemize}
    \item \underline{P}roblem: Define the problem.
          \begin{itemize}
              \item \textbf{Target population} (TP): The group of units referred to in the problem step.
              \item \textbf{Response}: The answer provided by the TP to the problem.
              \item \textbf{Attribute}: Statistic of the response.
                    \begin{Example}{}{}
                        What is the average grade of the students in STAT 101?
                        \begin{itemize}
                            \item Target population: All STAT 101 students
                            \item Response: Grade of a STAT 101 student.
                            \item Attribute: Average grade.
                        \end{itemize}
                    \end{Example}
          \end{itemize}
    \item \underline{P}lan:
          \begin{itemize}
              \item \textbf{Study population} (SP): The set of units you \emph{can} study
                    \begin{Example}{}{}
                        Does a drug reduce hair loss?
                        \begin{itemize}
                            \item Target population: People.
                            \item Study population: Mice.
                        \end{itemize}
                    \end{Example}
              \item \textbf{Sample}: A subset of the study population.
          \end{itemize}
    \item \underline{A}nalysis: We analyze the data.
    \item \underline{C}onclusion: Refers back to the problem. We also
          note some common \emph{errors}.
          \begin{itemize}
              \item \textbf{Study error}: The attribute of the population the
                    target population differs from the parameter of the study population.
                    \begin{Example}{}{}
                        Mathematically we can write it down as $ a(\text{TP})-\mu $,
                        however this error is qualitative. Therefore, we cannot
                        actually calculate it.
                    \end{Example}
              \item \textbf{Sample error}: The parameter differs from the
                    sample statistic (estimate).
                    \begin{Example}{}{}
                        Mathematically we can write it down as $ \mu-\bar{x} $,
                        however this error is qualitative. Therefore, we cannot
                        actually calculate it.
                    \end{Example}
              \item \textbf{Measurement error}: The difference
                    between what \emph{we want} to calculate and what \emph{we do}
                    calculate.
          \end{itemize}
\end{itemize}

\section{Lecture 2.00 - Models, Model 1}
\begin{Definition}{Model}{}
    A \textbf{model} relates a parameter to a response.
\end{Definition}
\begin{Definition}{Model 1}{}
    \textbf{Model 1} is defined as
    \[ Y_j=\mu+R_j\quad(R_j\sim\N{0,\sigma^2}) \]
    where
    \begin{itemize}
        \item $ Y_j $: random parameter that is the response of unit $ j $.
        \item $ \mu $: non-random unknown parameter that is the study population mean.
        \item $ R_j $: the distribution of responses about $ \mu $.
    \end{itemize}
\end{Definition}
\begin{Remark}{}{}
    \begin{itemize}
        \item $ R_j $'s are always independent.
        \item \textbf{Gauss}' Theorem: Any linear combination of
              normal random variables is normal.
        \item $ Y_j\sim\N{\mu,\sigma^2} $ since
              \[ \E{Y_j}=\E{\mu+R_j}=\E{\mu}+\E{R_j}=\mu+0=\mu \]
              \[ \Var{Y_j}=\Var{\mu+R_j}=\Var{R_j}=\sigma^2 \]
    \end{itemize}
\end{Remark}
\begin{Example}{}{}
    Average grade of STAT 101 students.
    \[ Y_j=\mu+R_j\quad(R_j\sim\N{0,\sigma^2}) \]
\end{Example}

\section{Lecture 3.00 - Independent Groups}
\begin{itemize}
    \item Dependent: we randomly select one group and
          we find a match, having the same explanatory variates, for
          each unit of the first group. For example, twins, reusing
          members of a group, or matching.
    \item Independent: are formed when we select units at random
          from mutually exclusive groups. For example, broken parts
          and non-broken parts.
\end{itemize}

\section{Lecture 4.00 - Models 2A and 2B}
\begin{Definition}{Model 2A}{}
    \textbf{Model 2A} is used when we assume
    the groups have the same standard deviation and is defined as
    \[ Y_{ij}=\mu_i+R_{ij}\quad(R_{ij}\sim\N{0,\sigma^2}) \]
    where
    \begin{itemize}
        \item $ Y_{ij} $: response of unit $ j $ in group $ i $.
        \item $ \mu_i $: mean for group $ i $.
        \item $ R_{ij} $: the distribution of responses about $ \mu_i $.
    \end{itemize}
\end{Definition}

\begin{Definition}{Model 2B}{}
    \textbf{Model 2B} is used when $ \sigma_1\ne \sigma_2 $
    and is defined as
    \[ Y_{ij}=\mu_i+R_{ij}\quad(R_{ij}\sim\N{0,\sigma_i^2}) \]
\end{Definition}

\section{Lecture 5.00 - Model 3}
We subtract Model 2A from Model 2B to model a difference between two groups,
and we get \emph{Model 3}.
\[ \begin{array}{cccccc}
          & Y_{1j}        & = & \mu_1       & + & R_{1j}        \\
        - & Y_{2j}        & = & \mu_2       & + & R_{2j}        \\
        \midrule
          & Y_{1j}-Y_{2j} & = & \mu_1-\mu_2 & + & R_{1j}-R_{2j}
    \end{array} \]
Let
\begin{itemize}
    \item $ Y_{1j}-Y_{2j}=Y_{dj} $
    \item $ \mu_1-\mu_2=\mu_d $
    \item $ R_{1j}-R_{2j}=R_{dj} $
\end{itemize}
\begin{Definition}{Model 3}{}
    \textbf{Model 3} is defined as
    \[ Y_{dj}=\mu_d+R_{dj}\quad(R_{dj}\sim\N{0,\sigma_d^2}) \]
\end{Definition}
\begin{Example}{Model 3}{}
    \begin{tabularx}{\linewidth}{@{}YYY@{}}
        Heart Rate Before Exercise & Heart Rate After Exercise & $ d $ \\
        \midrule
        70                         & 80                        & 10    \\
        80                         & 100                       & 20    \\
        90                         & 90                        & 0
    \end{tabularx}

    We could use Model 3.
\end{Example}
\section{Lecture 6.00 - Model 4}
Suppose $ Y\sim\bin{n,p} $; that is, we have $ n $
outcomes where each outcome is binary.
\[ \E{Y}=np \]
\[ \Var{Y}=np(1-p) \]
By the Central Limit Theorem, $ Y\stackrel{\cdot}{\sim}\N{np,np(1-p)} $.
The proportion is
\[ \frac{Y}{n} \stackrel{\cdot}{\sim}\N*{p,\frac{p(1-p)}{n} } \]
Let's find the expected value and variance of $ Y/n $.
\[ \E*{\frac{Y}{n}}=\frac{\E{Y}}{n}=\frac{np}{n}=p \]
\[ \Var*{\frac{Y}{n}}=\frac{\Var{Y}}{n^2}=\frac{np(1-p)}{n^2}=\frac{p(1-p)}{n}  \]
\begin{Definition}{Model 4}{}
    \textbf{Model 4} is defined as
    \[ \frac{Y}{n} \sim \N*{p,\frac{p(1-p)}{n} } \]
\end{Definition}

\section{Lecture 7.00 - MLE}
\begin{itemize}
    \item What is MLE? It connects the population parameter $ \theta $
          to your sample statistic $ \hat{\theta} $.
    \item How? It chooses the most probable value of $ \theta $
          given our data $ y_1,\ldots,y_n $.
\end{itemize}
\underline{Process}:
\begin{enumerate}[(1)]
    \item Define the \textbf{likelihood function}.
          \[ L=f(Y_1=y_1,Y_2=y_2,\ldots,Y_n=y_n) \]
          We assume $ Y_i\perp Y_j $ for all $ i\ne j $. Therefore,
          \[ L=f(Y_1=y_1)f(Y_2=y_2)\cdots f(Y_n=y_n) \]
    \item Define the \textbf{log-likelihood function} and use log rules to
          clean it up!
    \item Find $ \pd{\ell}{\theta} $.
    \item Set $ \pd{\ell}{\theta}=0 $, put hat on all $ \theta $'s.
    \item Solve for $ \hat{\theta} $.
\end{enumerate}
\begin{Example}{}{}
    Let $ Y_{ij}=\mu_i+R_{ij} $ where $ R_{ij}\sim\N{0,\sigma^2} $.
    \begin{align*}
        L & =f(Y_{11}=y_{11},\ldots,Y_{2n_2}=y_{2n_2})                                                \\
          & =\prod_{j=1}^{n_1}f(y_{1j})\prod_{j=1}^{n_2}f(y_{2j})                                     \\
          & =\prod_{j=1}^{n_1}\frac{1}{\sqrt{2\pi}\sigma}\expon*{-\frac{(y_{1j}-\mu_1)^2}{2\sigma^2}}
        \prod_{j=1}^{n_2}\frac{1}{\sqrt{2\pi}\sigma}\expon*{-\frac{(y_{2j}-\mu_2)^2}{2\sigma^2}}
    \end{align*}
    Let $ n_1+n_2=n $, then
    \[ L=(2\pi)^{-n/2}\sigma^{-n}\expon*{-\frac{\sum_{j=1}^{n_1}(y_{1j}-\mu_1)^2}{2\sigma^2}}
        \expon*{-\frac{\sum_{j=1}^{n_2}(y_{2j}-\mu_2)^2}{2\sigma^2}} \]
    The log-likelihood is given by
    \[ \ell=-\frac{n}{2}\ln(2\pi)-n\ln(\sigma)-\frac{\sum_{j=1}^{n_1} (y_{1j}-\mu_1)^2}{2\sigma^2}-
        -\frac{\sum_{j=1}^{n_2}(y_{2j}-\mu_2)^2}{2\sigma^2} \]
    Now,
    \[ \pd{\ell}{\hat{\mu}_1}=0+0-\frac{\sum_{j=1}^{n_1}2(y_{1j}-\hat{\mu})(-1) }{2\hat{\sigma}^2}+0=0 \]
    Hence,
    \[ 0=\sum_{j=1}^{n_1}(y_{1j}-\hat{\mu})\implies \sum_{j=1}^{n_1}y_{1j}=\sum_{j=1}^{n_1}\hat{\mu} \]
    Note that
    \[ \sum_{j=1}^{n_1}y_{1j}=\frac{n_1}{n_1} \sum_{j=1}^{n_1} y_{1j}=n_1\bar{y}_{1+}  \]
    Therefore,
    \[ n_1\bar{y}_{1+}=n_1\hat{\mu}\implies \bar{y}_{1+}=\hat{\mu}_1 \]
    By symmetry,
    \[ \bar{y}_{2+}=\hat{\mu}_2 \]
    The second partial is given by
    \[ \pd{\ell}{\sigma}=0+\frac{(-n)}{\hat{\sigma}}-\frac{\sum_{j=1}^{n_1} (y_{1j}-\hat{\mu}_1)^2}{2}(-2\hat{\sigma}^{-3})-
        -\frac{\sum_{j=1}^{n_2}(y_{2j}-\hat{\mu}_2)^2}{2}(-2\hat{\sigma}^{-3})
    \]
    Multiply both sizes by $ \hat{\sigma}^3 $, yields
    \[ 0=-n\hat{\sigma}^2+\sum_{j=1}^{n_1} (y_{1j}-\hat{\mu}_1)^2
        +\sum_{j=1}^{n_2}(y_{2j}-\hat{\mu}_2)^2 \]
    Divide both sizes by $ n $ and rearrange to get
    \[ \hat{\sigma}^2=\frac{\sum_{j=1}^{n_1} (y_{1j}-\hat{\mu}_1)^2+\sum_{j=1}^{n_{2}}(y_{2j}-\hat{\mu}_2) }{n} \]
    Recall that
    \[ s^2=\sum_{i=1}^{n} \frac{(y_i-\bar{y})^2}{n-1}  \]
    \[ s_1^2=\sum_{j=1}^{n_1}\frac{(y_{1j}-\bar{y}_{1+})^2}{n_1-1} \]
    \[ s_2^2=\sum_{j=1}^{n_2}\frac{(y_{2j}-\bar{y}_{2+})^2}{n_2-1} \]
    Therefore,
    \[ \hat{\sigma}^2=s_p^2=\frac{(n_1-1)s_1^2+(n_2-1)s_2^2}{n_1+n_2-2}  \]
\end{Example}

\section{Lecture 8.00 - LS}
\begin{itemize}
    \item What is LS\@? Another technique to find $ \hat{\theta} $.
    \item How? It minimizes the ``residuals.''
    \item Models:
          \[ \text{Response}=\text{Deterministic Part}+\text{Random Part} \]
          \[ Y=f(\theta)+R \]
          Let $ y_1,y_2,\ldots,y_n $ be realizations of $ Y $. Let
          $ \hat{y}_i=f(\hat{\theta}) $, where $ f(\hat{\theta}) $
          is simply $ f(\theta) $ with $ \theta $ replaced
          by $ \hat{\theta} $. We call $ \hat{y}_i $ our ``prediction.''
          \begin{Definition}{Residual}{}
              A \textbf{residual} is
              \[ r_i=y_i-f(\hat{\theta})=y_i-\hat{y}_i \]
          \end{Definition}
\end{itemize}
\underline{Process}:
\begin{enumerate}[(1)]
    \item Define the $ W $ function, $ W=\sum r^2 $.
    \item Calculate $ \pd{W}{\theta} $ for all non-$ \sigma $
          parameters
    \item Set $ \pd{W}{\theta}=0 $ and replace $ \theta $ by $ \hat{\theta} $.
    \item Solve for $ \hat{\theta} $.
\end{enumerate}

\section{Lecture 9.00 - LS Example}
Let's determine the LS of Model 2A.
\[ Y_{ij}=\mu_i+R_{ij} \]
Also, let $ n=n_1+n_2 $.
\begin{align*}
    W=\sum_{ij}r_{ij}^2
     & =\sum_{ij}(y_{ij}-\hat{\mu}_i)^2                                                  \\
     & =\sum_{j=1}^{n} \sum_{i=1}^{2} (y_{ij}-\hat{\mu}_i)^2                             \\
     & =\sum_{j=1}^{n_1}(y_{1j}-\hat{\mu}_1)^2 + \sum_{j=1}^{n_2} (y_{2j}-\hat{\mu}_2)^2
\end{align*}
\begin{align*}
    0 & =\pd{W}{\hat{\mu}_1}                                                 \\
      & =\sum_{j=1}^{n_1} (y_{1j}-\hat{\mu}_1)(-2)                           \\
      & =\frac{n_1}{n_1} \sum_{j=1}^{n_1} y_{ij}-\sum_{j=1}^{n_1}\hat{\mu}_1 \\
      & =n_1\bar{y}_{1+}-n\hat{\mu}_1                                        \\
\end{align*}
Therefore, $ \hat{\mu}_1=\bar{y}_{1+} $ and by symmetry $ \hat{\mu}_2=\bar{y}_{2+} $.
\begin{Remark}{}{}
    For LS, $ \hat{\sigma}^2 $ is always of the form
    \[ \hat{\sigma}^2=\frac{W}{n-q+c} \]
    where
    \begin{itemize}
        \item $ n= $ number of units
        \item $ q= $ number of non-$ \sigma $ parameters
        \item $ c= $ number of constraints
    \end{itemize}
    Note that $ \hat{\sigma}^2=s_p^2 $.
\end{Remark}
\begin{Remark}{MLE versus LS}{}
    \begin{itemize}
        \item LS is from 1860's. Unbiased provided $ R_j $ is normal.
        \item MLE is a recent technique and it is much more flexible
              since it does not require $ R_j $ to be normal.
        \item Minimum? You need to calculate the second derivative,
              but we're too lazy and unrigorous in this course. No thanks.
    \end{itemize}
\end{Remark}

\section{Lecture 10.00 - Estimators}
Our sample data is $ y_1,\ldots,y_n $.
It is non-random and is a realization of a random
variable $ Y_1,\ldots,Y_n $. A statistic is a function
of the sample data; $ \hat{\theta} $. It is non-random,
but if $ y_1,\ldots,y_n $ changes, then so does $ \hat{\theta} $.
For that reason, you can think of $ \hat{\theta} $
as the realization of a random variable $ \tilde{\theta} $,
called an estimator. To move from $ \hat{\theta} $
to $ \tilde{\theta} $ we capitalize our $ Y $'s.

\begin{Example}{}{}
    Model 2A\@:
    $ \displaystyle \Uunderbracket{\hat{\mu}_1=\bar{y}_{1+}}_{\text{STATISTIC}}\to
        \Uunderbracket{\tilde{\mu}_1=\bar{Y}_{1+}}_{\text{ESTIMATOR}} $
\end{Example}
\begin{Theorem}{Gauss' Theorem}{}
    Any linear combination of normal random variables is still normal.
\end{Theorem}
\begin{Example}{}{}
    Let $ X\sim\N{\mu_X,\sigma_X^2} $, $ Y \sim \N{\mu_Y,\sigma^2_Y} $
    be independent random variables and $ a,b,c\in\mathbf{R} $, then
    \[ L=aX+bY+c\sim\N{\E{L},\Var{L}} \]
\end{Example}
\begin{Theorem}{Central Limit Theorem (CLT)}{}
    Let $ Y_1,\ldots,Y_n $ be a i.i.d.\ random variables with
    $ \E{Y_i}=\mu $, $ \Var{Y_i}=\sigma^2<\infty $, then
    \[ \bar{Y} \sim \N*{\mu,\frac{\sigma^2}{n}} \]
\end{Theorem}

\section{Lecture 11.00 - Estimators Example}
\begin{Example}{}{}
    Model 2A\@: $ Y_{ij}=\mu_i+R_{ij} $ where $ R_{ij} \sim \N{0,\sigma^2} $.
    What is the distribution of $ \tilde{\mu} $?

    \textbf{Solution.} Using LS or MLE we obtain
    \[ \hat{\mu}=\bar{y}_{1+} \]
    Or corresponding estimator is
    \[ \tilde{\mu}_1=\bar{Y}_{1+}=\frac{\sum_{j=1}^{n_1}Y_{1j}}{n_1}  \]
    and by Gauss it is normal!
    \[
        \E{\tilde{\mu}_1}
        =\E*{\frac{\sum_{j=1}^{n_1} Y_{1j}}{n_1}}
        =\frac{\sum_{j=1}^{n_1}\E{Y_{1j}}}{n_1}
        =\frac{\sum_{j=1}^{n_1} \E{\mu+R_{1j}}}{n_1}
        =\frac{\sum_{j=1}^{n_1} \mu+\E{R_{1j}}}{n_1}
        =\mu_1
    \]
    \begin{Definition}{Unbiased estimator}{}
        If $ \E{\tilde{\theta}}=\theta $, we say $ \tilde{\theta} $
        is an \textbf{unbiased estimator} of $ \theta $.
    \end{Definition}
    \begin{align*}
        \Var{\tilde{\mu}_1}
         & =\Var{\bar{Y}_{1+}}                                                                       \\
         & =\Var*{\frac{\sum_{j=1}^{n_1} Y_{1j}}{n_1} }                                              \\
         & =\frac{1}{n_1^2}\Var*{\sum_{j=1}^{n_1} Y_{1j}}                                            \\
         & =\frac{1}{n_1^2} \sum_{j=1}^{n_1} \Var{Y_{ij}}       &  & \text{since }Y_{1j}\perp Y_{1i} \\
         & =\frac{1}{n_1^2} \sum_{j=1}^{n_1} \Var{\mu_1+R_{1j}}                                      \\
         & =\frac{1}{n_1^2} \sum_{j=1}^{n_1} \Var{Y_{1j}}                                            \\
         & =\frac{1}{n_1^2} (n_1\sigma^2)                                                            \\
         & =\frac{\sigma^2}{n_1}
    \end{align*}
    Therefore,
    \[ \tilde{\mu}_1\sim \N*{\mu_1,\frac{\sigma^2}{n_1}} \]
    and by symmetry
    \[ \tilde{\mu}_2 \sim \N*{\mu_2,\frac{\sigma^2}{n_2}} \]
\end{Example}
\section{Lecture 12.00 - Sigma}
\begin{Theorem}{}{thm_1}
    Let $ Z \sim \N{0,1} $, then $ Z^2 \sim \chi^2(1) $
\end{Theorem}
\begin{Theorem}{}{thm_2}
    Let $ X \sim \chi^2(m) $, $ Y \sim \chi^2(n) $
    be independent, then
    \[ X+Y \sim \chi^2(n+m) \]
\end{Theorem}
\begin{Theorem}{}{thm_3}
    Let $ Z \sim \N{0,1} $ and $ X \sim \chi^2(m) $, then
    \[ \frac{Z}{\sqrt{X/m}} \sim t(m) \]
\end{Theorem}
\begin{Theorem}{}{thm_4}
    Let $ \displaystyle Y=\frac{(n-q+c)\tilde{\sigma}^2}{\sigma^2} $,
    then $ Y \sim \chi^2(n-q+c) $.
\end{Theorem}
\section{Lecture 13.00 - Sigma Example}
\begin{Example}{}{}
    Model 1: $ Y_j=\mu+R_j $ where $ R_j \sim \N{0,\sigma^2} $.
    What is the distribution of $ \displaystyle \frac{\tilde{\mu}-\mu}{\tilde{\sigma}/\sqrt{n}} $?

    \textbf{Solution.} We know by LS or MLE that $ \hat{\mu}=\bar{y}_+ $, therefore
    $ \tilde{\mu}=\bar{Y}_+ $. We know $ \tilde{\mu} \sim \N*{\mu,\frac{\sigma^2}{n}} $.
    We standardize
    \[ Z=\frac{\tilde{\mu}-\mu}{\sigma/\sqrt{n}}\sim \N{0,1}  \]
    By~\Cref{thm:thm_4}, we know
    \[ X=\frac{(n-1)\tilde{\sigma}^2}{\sigma^2}\sim \chi^2(n-1)  \]
    By~\Cref{thm:thm_3},
    \[ \frac{Z}{\sqrt{X/(n-1)}}
        =\frac{\frac{\tilde{\mu}-\mu}{\sigma/\sqrt{n}}}{\frac{(n-1)\tilde{\sigma}^2}{\sigma^2}}=
        \frac{\tilde{\mu}-\mu}{\tilde{\sigma}/\sqrt{n}}\sim t(n-1)   \]
\end{Example}
\begin{Remark}{}{}
    Recall that
    \[ \frac{\tilde{\mu}-\mu}{\sigma/\sqrt{n}}\sim \N{0,1}  \]
    By replacing $ \sigma $ by $ \tilde{\sigma} $, we end up using a $ t $-distribution
    instead of a normal distribution.
\end{Remark}
\section{Lecture 14.00 - CI}
We assume our estimator is
\[ \tilde{\theta}\sim \N{0,\Var{\tilde{\theta}}} \]
The CI\@:
\[ \theta:\text{EST}\pm c\,\text{SE}=\hat{\theta}\pm c\sqrt{\Var{\tilde{\theta}}} \]
If we don't know $ \sigma $, we replace it by $ \hat{\sigma} $
and obtain
\[ \theta:\hat{\theta}\pm c\sqrt{\widehat{\Var{\tilde{\theta}}}} \]
\begin{Example}{}{}
    Model 1: $ Y_{j}=\mu+R_j $ where $ R_j \sim \N{0,\sigma^2} $.
    By LS we know $ \hat{\mu}=\bar{y}_+ $. The estimator is $ \tilde{\mu}=\bar{Y}_+ $
    with distribution
    \[ \tilde{\mu} \sim \N*{\mu,\frac{\sigma^2}{n}} \]
    Our CI\@:
    \[ \mu:\text{EST}\pm c\,\text{SE}=\hat{\mu}\pm c\frac{\sigma}{\sqrt{n}}
        =\bar{y}_+\pm c \frac{\sigma}{\sqrt{n}}\quad (c \sim \N{0,1})  \]
    \[ \mu:\bar{y}_+ \pm c \frac{s}{\sqrt{n}}\sim t(n-1)  \]
    Recall: $ \displaystyle s=\frac{\sum_{i=1}^{n} (y_i-\bar{y})^2}{n-1} $.
\end{Example}
\begin{Example}{}{}
    Model 2A\@: $ Y_{ij} =\mu_i+R_{ij} $ where $ R_{ij}\sim \N{0,\sigma^2} $.
    By LS, $ \hat{\mu}_1=\bar{y}_{1+} $ and $ \hat{\mu}_2=\bar{y}_{2+} $.
    The estimators $ \tilde{\mu}_1=\bar{Y}_{1+} $ and $ \tilde{\mu}_2=\bar{Y}_{2+} $.
    The distributions are
    \[ \tilde{\mu}_1 \sim \N*{\mu_1,\frac{\sigma^2}{n_1} } \]
    \[ \tilde{\mu}_2 \sim \N*{\mu_2,\frac{\sigma^2}{n_2} } \]
    \[ \tilde{\mu}_1-\tilde{\mu}_2
        \sim \N*{\mu_1-\mu_2,\sigma^2\biggl(\frac{1}{n_1}+\frac{1}{n_2}\biggr)} \]
    Our CI\@:
    \[ \mu_1-\mu_2:\text{EST}\pm c\,\text{SE}
        =\hat{\mu}_1-\hat{\mu}_2\pm c\,\sigma\sqrt{\frac{1}{n_1}+\frac{1}{n_2} }
        \quad(c \sim \N{0,1}) \]
    \[ \mu_1-\mu_2:\text{EST}\pm c\,\text{SE}
        =\hat{\mu}_1-\hat{\mu}_2\pm c\, s_p\sqrt{\frac{1}{n_1}+\frac{1}{n_2} }
        \quad(c \sim t(n_1+n_2-2)) \]
\end{Example}
\begin{Example}{}{}
    Model 2B\@: $ Y_{ij}=\mu_i=R_{ij} $ where $ R_{ij}\sim \N{0,\sigma_i^2} $.
    \[ \tilde{\mu}_1-\tilde{\mu}_2
        \sim \N*{\mu_1-\mu_2,\frac{\sigma_1^2}{n_1} +\frac{\sigma_2^2}{n_2} } \]
    Our CI\@:
    \[ \hat{\mu}_1-\hat{\mu}_2\pm c\sqrt{\frac{\sigma_1^2}{n_1} +\frac{\sigma_2^2}{n_2}}
        \quad(c \sim \N{0,1}) \]
    \[ \hat{\mu}_1-\hat{\mu}_2\pm c\sqrt{\frac{s_1^2}{n_1} +\frac{s_2^2}{n_2}}
        \quad(c \sim t(n_1+n_2-2)) \]
\end{Example}
\begin{Example}{}{}
    Model 3: $ Y_{dj}=\mu_d+R_{dj} $ where $ R_{dj} \sim \N{0,\sigma^2_d} $,
    which is the same as Model 1.
    \[ \mu_d:\bar{y}_{d+}\pm c \frac{\sigma_d}{\sqrt{n_d}}\quad (c \sim \N{0,1})  \]
    \[ \mu_d:\bar{y}_{d+} \pm c \frac{s_d}{\sqrt{n_d}}\sim t(n_d-1)  \]
\end{Example}
\begin{Example}{}{}
    Model 4:
    \[ \tilde{p} \sim \N*{p,\frac{p(1-p)}{n}} \]
    Our CI\@:
    \[ \hat{p}\pm c\sqrt{\frac{\hat{p}(1-\hat{p})}{n}}\quad(c \sim \N{0,1}) \]
\end{Example}
\begin{table}[!htbp]
    \centering
    \caption{Confidence Intervals}
    \begin{tabularx}{0.95\linewidth}{@{}>{\hsize=0.1\hsize}YYYY@{}}
        \toprule
        \# & Model                                                            & CI                                                                               & df            \\
        \midrule
        1  & $ \underset{R_i \sim \N{0,\sigma^2}}{Y_i=\mu+R_i} $              & $ \bar{y}\pm t^* \dfrac{s}{\sqrt{n}} $                                           & $ n-1 $       \\
        \midrule
        2A & $ \underset{R_{ij} \sim \N{0,\sigma^2}}{Y_{ij}=\mu_i+R_{ij}} $   & $ \bar{y}_{1+} \pm t^*\dfrac{s_1}{\sqrt{n_1}} $                                  & $ n_1-1 $     \\
           &                                                                  & $ \bar{y}_{1+}-\bar{y}_{2+}\pm t^* s_p\sqrt{\dfrac{1}{n_1}+\dfrac{1}{n_2}} $     & $ n_1+n_2-2 $ \\
        \midrule
        2B & $ \underset{R_{ij} \sim \N{0,\sigma_i^2}}{Y_{ij}=\mu_i+R_{ij}} $ & $ \bar{y}_{1+} \pm t^*\dfrac{s_1}{\sqrt{n_1}} $                                  & $ n_1-1 $     \\
           &                                                                  & $ \bar{y}_{1+}-\bar{y}_{2+}\pm t^*\sqrt{\dfrac{s_1^2}{n_1}+\dfrac{s_2^2}{n_2}} $ & $ n_1+n_2-2 $ \\
        \midrule
        3  & $ \underset{R_{dj} \sim \N{0,\sigma_d^2}}{Y_{dj}=\mu_d+R_{dj}} $ & $ \bar{y}_d\pm t^*\, \dfrac{s_d}{\sqrt{n_d}}  $                                  & $ n_d-1 $     \\
        \midrule
        4  & $ \dfrac{Y}{n}\sim \N*{p,\dfrac{p(1-p)}{n}} $                    & $ \hat{p}\pm z^*\sqrt{\dfrac{\hat{p}(1-\hat{p})}{n}} $                           & $ \N{0,1} $
    \end{tabularx}
\end{table}

\section{Lecture 15.00 - CI Examples}
\begin{Example}{Model 1}{}
    \begin{itemize}
        \item \textbf{Problem}: What is the mean calculus grade of students in STAT 332?
        \item \textbf{Plan}: We randomly select 5 students from the class.
        \item \textbf{Data}: 65, 70, 80, 85, 75
        \item \textbf{Analysis}: Build a 95\% confidence interval for the mean grade.
    \end{itemize}
    \[ \mu:\bar{y}\pm t^* \frac{s}{\sqrt{n}} \]
    \begin{minted}{R}
    dat <- c(65, 70, 80, 85, 75)
    y.bar <- mean(dat)
    s <- sd(dat)
    n <- length(dat)
    df <- length(dat) - 1
    t <- qt(0.975, df)
    left <- y.bar - t * s / sqrt(n)
    right <- y.bar + t * s / sqrt(n)
    \end{minted}
    The 95\% confidence interval is: $(65.18, 84.82)$. We are 95\% confident
    that the mean grade is in the interval. What we mean is that
    if we drew 100 samples, and built 100 confidence intervals for these
    samples, then we would expect to find $\mu$ in 95 of these intervals
    that we created. This is not a probability because at the end of the day,
    you estimated your data for $ y $.
\end{Example}
\begin{Example}{Model 2A}{}
    \begin{itemize}
        \item \textbf{Problem}: In grade 9, there is a standardized test in Ontario.
              We wish to compare the mean performance of girls to boys.
        \item \textbf{Plan}: They collect data from a class of 30 students;
              15 boys and girls. Their response is their grade on the standardized test.
              If necessary, assume the variances of the two groups are the same.
        \item \textbf{Data}:
              \begin{itemize}
                  \item Boys: 39, 42, 47, 50, 52, 52, 54, 55, 55, 56, 56, 56, 58, 60, 62
                  \item Girls: 44, 45, 48, 50, 51, 52, 53, 53, 57, 58, 59, 60, 62, 63, 64
              \end{itemize}
        \item \textbf{Analysis}: Build a 95\% confidence interval for the mean difference
              in grades.
    \end{itemize}
    \begin{minted}{R}
    boys <- c(39, 42, 47, 50, 52, 52, 54, 55, 55, 56, 56, 56, 58, 60, 62)
    girls <- c(44, 45, 48, 50, 51, 52, 53, 53, 57, 58, 59, 60, 62, 63, 64)
    y_b.bar <- mean(boys)
    y_g.bar <- mean(girls)
    s_b.sq <- var(boys)
    s_g.sq <- var(girls)
    n_b <- length(boys)
    n_g <- length(girls)
    s_p.sq <-
    ((n_g - 1) * s_g.sq + (n_b - 1) * s_b.sq) / (n_g + n_b - 2)
    df <- n_g + n_b - 2
    t <- qt(0.975, df)
    left <- (y_b.bar - y_g.bar) - t * sqrt(s_p.sq * (1 / n_g + 1 / n_b))
    right <- (y_b.bar - y_g.bar) + t * sqrt(s_p.sq * (1 / n_g + 1 / n_b))
    \end{minted}
    \begin{itemize}
        \item $ \bar{y}_{b+}=52.9 $
        \item $ \bar{y}_{g+}=54.6 $
        \item $ s_b^2=39.6 $
        \item $ s_g^2=41 $
        \item $ s_p^2=40.3 $
        \item $ t^*=2.048 $
    \end{itemize}
    The 95\% confidence interval for the mean difference grade is
    $ (-6.4,3.1) $. Is there a difference between male and female grades?
    0 is in this interval, so we conclude there is no difference between male
    and female grades.
\end{Example}
\begin{Example}{Model 3}{}
    \begin{itemize}
        \item \textbf{Problem}: In grade 9 there is a standardized test in Ontario.
              We wish to compare the mean performance of girls to boys.
        \item \textbf{Plan}: They collect data from a class of 30 students; 15 boys and 15 girls. Each girl is selected so that she was born in the
              same month as a boy in the class. The response is their grade on the standardized test. If necessary assume the
              variances of the two groups are different.
        \item \textbf{Data}:
              \begin{itemize}
                  \item Boys: 39, 42, 47, 50, 52, 52, 54, 55, 55, 56, 56, 56, 58, 60, 62
                  \item Girls: 44, 45, 48, 50, 51, 52, 53, 53, 57, 58, 59, 60, 62, 63, 64
              \end{itemize}
        \item \textbf{Analysis}: Build a 95\% confidence interval for the mean difference
              in grades.
    \end{itemize}
    By matching, they have created a dependent group. Paired data implies
    we use Model 3.
    \begin{minted}{R}
    boys <- c(39, 42, 47, 50, 52, 52, 54, 55, 55, 56, 56, 56, 58, 60, 62)
    girls <- c(44, 45, 48, 50, 51, 52, 53, 53, 57, 58, 59, 60, 62, 63, 64)
    diff <- boys - girls
    y_d.bar <- mean(diff)
    s_d <- sd(diff)
    n_d <- length(diff)
    df <- length(diff) - 1 
    t <- qt(0.975, df)
    left <- y_d.bar - t * s_d / sqrt(n_d)
    right <- y_d.bar + t * s_d / sqrt(n_d)        
    \end{minted}
    \begin{itemize}
        \item $ \bar{y}_{d+}=1.7 $
        \item $ s_d=2.1 $
        \item $ n_d=15 $
        \item $ t^*=2.145 $
    \end{itemize}
    The 95\% confidence interval for the mean difference grade is
    $ (-2.9,-0.5) $. Is there a difference between male and female grades?
    0 is not in this interval, so we conclude there is a difference between male
    and female grades. In fact, we may argue that the boys are doing worse than the girls.
\end{Example}
\begin{Example}{Model 2B}{}
    \begin{itemize}
        \item \textbf{Problem}: In grade 9 there is a standardized test in Ontario. We wish to compare the mean performance of girls to boys.
        \item \textbf{Plan}: They collect data from a class of 30 students; 15 boys and 15 girls. The response is their grade on the standardized
              test. If necessary assume the variances of the two groups are different.
        \item \textbf{Data}:
              \begin{itemize}
                  \item Boys: 39, 42, 47, 50, 52, 52, 54, 55, 55, 56, 56, 56, 58, 60, 62
                  \item Girls: 44, 45, 48, 50, 51, 52, 53, 53, 57, 58, 59, 60, 62, 63, 64
              \end{itemize}
        \item \textbf{Analysis}: Build a 95\% confidence interval for the mean difference
              in grades.
    \end{itemize}
    \begin{minted}{R}
    boys <- c(39, 42, 47, 50, 52, 52, 54, 55, 55, 56, 56, 56, 58, 60, 62)
    girls <- c(44, 45, 48, 50, 51, 52, 53, 53, 57, 58, 59, 60, 62, 63, 64)
    y_b.bar <- mean(boys)
    y_g.bar <- mean(girls)
    s_b.sq <- var(boys)
    s_g.sq <- var(girls)
    n_b <- length(boys)
    n_g <- length(girls)
    df <- n_g + n_b - 2
    t <- qt(0.975, df)
    left <- (y_b.bar - y_g.bar) - t * sqrt(s_b.sq / n_g + s_g.sq / n_b)
    right <-
    (y_b.bar - y_g.bar) + t * sqrt(s_b.sq / n_g + s_g.sq / n_b)
    \end{minted}
    The 95\% confidence interval for the mean difference grade is
    $ (-6.41,3.08) $.
\end{Example}
\begin{Example}{Model 4}{}
    \begin{itemize}
        \item \textbf{Problem}: In October there will be a federal election. Prior to the election pollsters will gauge the popularity of the
              candidates. One party of interest will be the Liberals.
        \item \textbf{Plan}: They ask 430 randomly selected people whether they would vote liberal.
        \item \textbf{Data}: 267 people would be willing to vote Liberal.
        \item \textbf{Analysis}: Build a 95\% confidence interval for the proportion of people willing to vote Liberal.
    \end{itemize}
    \begin{minted}{R}
    n <- 430
    voters <- 267
    p.hat <- 267 / 430
    z <- qnorm(0.975)
    left <- p.hat - z * sqrt((p.hat * (1 - p.hat)) / n)
    right <-  p.hat + z * sqrt((p.hat * (1 - p.hat)) / n)
    \end{minted}
    \begin{itemize}
        \item $ n=430 $
        \item $ \hat{p}=267/430 $
        \item $ z^*=1.96 $
    \end{itemize}
    The 95\% confidence interval for the proportion of people willing to vote Liberal
    is $ (0.575, 0.667) $.
\end{Example}

\section{Lecture 16.00 - HT}
\begin{enumerate}[(1)]
    \item Define the hypothesis
          \begin{table}[!htbp]
              \centering
              \caption{Hypotheses}
              \begin{tabularx}{0.5\linewidth}{@{}YY@{}}
                  $ H_0 $                & $ H_a $                \\
                  \midrule
                  $ \theta=\theta_0 $    & $ \theta\ne \theta_0 $ \\
                  $ \theta\ge \theta_0 $ & $ \theta<\theta_0 $    \\
                  $ \theta\le \theta_0 $ & $ \theta>\theta_0 $
              \end{tabularx}
          \end{table}
    \item Discrepancy
          \[ d=\frac{\text{EST}-\text{$H_0$ value}}{\text{SE}}=\frac{\hat{\theta}-\theta_0}{\sqrt{\Var{\tilde{\theta}}}}   \]
    \item Given $ \tilde{\theta} \sim \N{\theta,\Var{\tilde{\theta}}} $
          where $ D \sim \N{0,1} $ when $ \sigma $ is known or
          $ D \sim t(n-q+c) $ when $ \sigma $ is known.
    \item $ p $-value
          \begin{table}[!htbp]
              \centering
              \caption{$ p $-value}
              \begin{tabularx}{0.5\linewidth}{@{}YY@{}}
                  $ H_a $                & $ p $-value           \\
                  \midrule
                  $ \theta\ne \theta_0 $ & $ 2\Prob{D>\abs{d}} $ \\
                  $ \theta<\theta_0 $    & $ \Prob{D<d} $        \\
                  $ \theta>\theta_0 $    & $ \Prob{D>d} $
              \end{tabularx}
          \end{table}
    \item Conclusion
          \begin{table}[!htbp]
              \centering
              \caption{Guidelines for interpreting $ p $-values}
              \begin{tabularx}{0.7\linewidth}{@{}YY@{}}
                  $ p $-value        & Interpretation                    \\
                  \midrule
                  $ p>0.1 $          & No evidence to reject $ H_0 $.    \\
                  $ 0.05<p\le 0.10 $ & Weak evidence against $ H_0 $.    \\
                  $ 0.01<p<0.05 $    & Evidence against $ H_0 $.         \\
                  $ p<0.01 $         & Tons of evidence against $ H_0 $.
              \end{tabularx}
          \end{table}
\end{enumerate}
\section{Lecture 17.00 - HT Examples}
\begin{Example}{}{}
    Willow trees are grown from cuttings. These cuttings from 6 willow trees were subjected to
    two soils: high and low acidity. 2 cuttings from each tree are assigned to the two levels of
    acidity. After 1 year the height, in cm, of the cuttings was measured.

    \begin{center}
        \begin{tabularx}{0.7\linewidth}{@{}YYYYYYY@{}}
            \textbf{Cutting} & 1  & 2  & 3  & 4  & 5  & 6  \\
            \midrule
            \textbf{High}    & 11 & 19 & 32 & 12 & 7  & 14 \\
            \textbf{Low}     & 17 & 21 & 14 & 11 & 18 & 9
        \end{tabularx}
    \end{center}

    Is the growth in high and low acidity equal? Use an appropriate hypothesis test. If necessary
    assume group variances are the same.

    \textbf{Solution.}
    \begin{itemize}
        \item $H_0$: $ \mu_d=0 $
        \item $H_a$: $ \mu_d\ne 0 $
    \end{itemize}
    \begin{minted}{R}
    high <- c(11, 19, 32, 12, 7, 14)
    low <- c(17, 21 , 14 , 11 , 18 , 9)
    diff <- high - low
    y_d.bar <- mean(diff)
    s_d <- sd(diff)
    n_d <- length(diff)
    df <- length(diff) - 1
    d <- (y_d.bar - 0) / (s_d / sqrt(n_d))
    pval <- 2 * (1 - pt(d, df))
    \end{minted}
    \begin{itemize}
        \item $ \bar{y}_{d+}=0.83 $
        \item $ s_d=10.07 $
        \item $ d=0.2 $
        \item $ p=2\Prob{D>\abs{d}}=2[1-\Prob{D\le 0.2}]=$ \code{2 * (1 - pt(d, df))} $\approx
                  0.84$.
    \end{itemize}
    We obtain a $ p $-value of $ 0.84 $. There is no evidence to reject $ H_0 $.
    In other words, we can argue in favour of saying that they have the same
    growth in different acidic soils.
\end{Example}
\begin{Example}{}{}
    A random assortment of pumpkin seeds were planted and fertilized using two types of plant
    feed, coke and water. After 4 weeks the plant heights, in cm, were measured.
    \begin{itemize}
        \item Coke: 8, 7, 18, 42, 21
        \item Water: 5, 11, 21, 9, 14
    \end{itemize}
    Is coke a better fertilizer for pumpkin’s than water? Use an appropriate hypothesis test. If
    necessary assume group variances are the same.
    \begin{itemize}
        \item $ H_0 $: $ \mu_c=\mu_w $
        \item $ H_a $: $ \mu_c-\mu_w>0 $
    \end{itemize}
    \begin{minted}{R}
    coke <- c(8, 7, 18, 42, 21)
    water <- c(5, 11, 21, 9, 14)
    y_c.bar <- mean(coke)
    y_w.bar <- mean(water)
    s_c.sq <- var(coke)
    s_w.sq <- var(water)
    n_c <- length(coke)
    n_w <- length(water)
    s_p.sq <-
    ((n_w - 1) * s_w.sq + (n_c - 1) * s_c.sq) / (n_w + n_c - 2)
    df <- n_c + n_w - 2
    t <- qt(0.975, df)
    d <- (y_c.bar - y_w.bar) / sqrt(s_p.sq * (1 / n_w + 1 / n_c))
    pval <- 1 - pt(d, df)
    \end{minted}
    \begin{itemize}
        \item $ \bar{y}_{c+}=19.2 $
        \item $ \bar{y}_{w+}=12 $
        \item $ n_w=n_c=5 $
        \item $ s_c^2=199.7 $
        \item $ s_w^2=36 $
        \item $ s_p^2=117.85 $
        \item $ \displaystyle d=\frac{\bar{y}_{c+}-\bar{y}_{d+}-0}{s_p\sqrt{\frac{1}{n_w} +\frac{1}{n_c} }}=1.049 $
        \item $ p=\Prob{D>d}=1-\Prob{D\le 1.049}=$ \code{1 - pt(d, df)} $\approx 0.162$.
    \end{itemize}
    There is no evidence to reject $ H_0 $. Therefore, coke is not a
    better fertilizer than water.
\end{Example}
