\chapter{Assignment 4}
\section{Lecture 27.00 - Model 6}
\begin{Definition}{Unbalanced CRD, Model 6}{}
    The \textbf{unbalanced completely randomized design} is defined as
    \[ Y_{ij}=\mu+\tau_i+R_{ij}\quad(R_{ij}\sim \N{0,\sigma^2}) \]
    for $ i=1,2,\ldots,t $ (\# of treatments),
    $ j=1,2,\ldots,r_i $ (\# of replicates/treatment).
    In this course, this is \textbf{Model 6}.

    \underline{Constraint}: $ \sum_{i=1}^{t} r_i \tau_i=0 $.
\end{Definition}
\begin{Example}{LS for Model 6}{}
    The LS for Model 6 is
    \[ W=\sum r_{ij}^2 +\lambda\biggl(\sum_{i=1}^{t} r_i\tau_i\biggr)  \]
    and results in
    \[ \hat{\mu}=\bar{y}_{++} \]
    \[ \hat{\tau}_i=\bar{y}_{i+}-\bar{y}_{++} \]
    \[ \hat{\sigma}^2=\frac{W}{(r_1+r_2+\cdots+r_t)-(t+1)+1}  \]
\end{Example}
\begin{Example}{}{m_6_ex_1}
    Refer to~\Cref{ex:m_5_ex_1}, we remove the last element of group 2.
    \begin{minted}{R}
grp1 = c(50, 53, 52, 58)
grp2 = c(62, 55, 58)
Y = c(grp1, grp2)
x = as.factor(c(rep(1, 4), rep(2, 3)))
# Group Averages
grp_av = tapply(Y, x, mean, na.rm = T)
mu = mean(Y)
# Treatment Effects
tau1 = (grp_av - mean(Y))[1]
tau2 = (grp_av - mean(Y))[2]
# Estimated Sigma
sigma = summary(lm(Y ~ x))$sigma
\end{minted}
    We obtain
    \begin{itemize}
        \item $ \hat{\sigma}=3.447221 $
        \item $ \hat{\tau}_1=-2.178571 $
        \item $ \hat{\tau}_2=2.904762 $
        \item $ \hat{\mu}=55.42857 $
        \item Obviously, $ 4(\hat{\tau}_1)+3(\hat{\tau}_2)=0 $
    \end{itemize}
    We will answer the same questions defined in~\Cref{ex:m_5_ex_1}.

    \textbf{Solution 1.} $ \hat{\tau}_1=-2.18 $

    \textbf{Solution 2.} $ \theta=\tau_1-\tau_2\implies \tilde{\tau}=\tilde{\tau}_1-\tilde{\tau}_2 $.
    \[ \E{\tilde{\theta}}=\tau_1-\tau_2 \]
    \[ \Var{\tilde{\theta}}=\Var{\bar{Y}_{1+}-\bar{Y}_{2+}}=\frac{\sigma^2}{4} +\frac{\sigma^2}{3}=\frac{7\sigma^2}{12}  \]
    Confidence interval:
    \[ \hat{\tau}_1-\hat{\tau}_2\pm c\sqrt{\frac{7\hat{\sigma}^2}{12}}=(-11.85,1.68) \]
    In R,
    \begin{minted}{R}
tau1 - tau2 - qt(0.975, 5) * sqrt((7 * sigma ^ 2) / 12)
tau1 - tau2 + qt(0.975, 5) * sqrt((7 * sigma ^ 2) / 12)
\end{minted}
    \begin{minted}{R}
anova(lm(Y ~ x))
# ANOVA Table
Analysis of Variance Table

Response: Y
          Df Sum Sq Mean Sq F value Pr(>F)
x          1 44.298  44.298  3.7277 0.1114
Residuals  5 59.417  11.883   
\end{minted}
    No evidence against $ H_0 $: $ \tau_1=\cdots=\tau_t=0 $, so this model is not great.
\end{Example}

\section{Lecture 28.00 - Model 7}
\begin{Definition}{Randomized block design, Model 7}{}
    The \textbf{randomized block design} (RBD) is defined as
    \[ Y_{ij}=\mu+\tau_i+B_j+R_{ij}\quad(R_{ij} \sim \N{0,\sigma^2}) \]
    where $ B_j $ is the $ j^{\text{th}} $ block (BIK) effect. Note that
    \begin{itemize}
        \item $ i=1,2,\ldots,t $
        \item $ j=1,2,\ldots,r $
        \item $ \sum_{i=1}^t \tau_i=0  $
        \item $\sum_{j=1}^{r} B_j=0$
    \end{itemize}
\end{Definition}
\begin{Example}{LS for Model 7}{}
    The LS for Model 7 is
    \[ W=\sum_{ij}r_{ij}+\lambda_1\biggl(\sum_{i=1}^{t} \tau_i\biggr)+
        \lambda_2 \biggl(\sum_{j=1}^{r} B_j\biggr)  \]
    Solving
    \[ \hat{\mu}=\bar{y}_{++} \]
    \[ \hat{\tau}_i=\bar{y}_{i+}-\bar{y}_{++} \]
    \[ \hat{B}_j=\bar{y}_{+j}-\bar{y}_{++} \]
    \[ \hat{\sigma}^2=\frac{W}{(rt)-(t+r+1)+2}  \]
\end{Example}

\section{Lecture 29.00 - Model 7, Example}
