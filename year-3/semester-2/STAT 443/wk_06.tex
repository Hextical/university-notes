\chapter{Week 6}
\section{SARIMA Models}
Frequently, time series exhibit ``seasonality.''
\subsection*{Rough Definition of Seasonality}
A time series $ X_t $ is said to be ``seasonal''
if it exhibits regular variation so that for some lag $ s $,
$ X_t $ is ``similar'' to $ X_{t-s} $.
Some sources of seasonality are weather or scheduled events.
These typically lead to yearly, weekly, monthly, or quarterly cycles.
\begin{Remark}{}{}
      ARIMA models are not ideal for modelling seasonality.
      \[ \text{ARIMA Models}\implies\text{Random Walk with Stationary Errors} \]
      Random walks do not seasonality.
\end{Remark}
\begin{Definition}{Seasonal ARIMA}{}
      $ X_t $ is said to follow a \textbf{Seasonal ARIMA} model (SARIMA)
      of orders $ p,d,q $ and $ P,D,Q $ and seasonal period $ s $
      if
      \[ \Phi_P(B^s)\phi_p(B)(1-B^s)^D(1-B)^d=\Theta_Q(B^s)\theta_q(B)W_t \]
      We abbreviate the SARIMA $ p,d,q,P,D,Q $ model with seasonal
      period $ s $ as $ \text{SARIMA}(p,d,q)\times(P,D,Q)_s $.
      \[ \begin{array}{ccccccccc}
                  \Phi_P(B)     & = & 1 & - & \Phi_1 B     & - & \cdots & - & \Phi_P B^P      \\
                  \Phi_P(B^s)   & = & 1 & - & \Phi_1 B^s   & - & \cdots & - & \Phi_P B^{Ps}   \\
                  \phi_p(B)     & = & 1 & - & \phi_1 B     & - & \cdots & - & \phi_p B^p      \\
                  \Theta_Q(B)   & = & 1 & + & \Theta_1 B   & + & \cdots & + & \Theta_Q B^Q    \\
                  \Theta_Q(B^s) & = & 1 & + & \Theta_1 B^s & + & \cdots & + & \Theta_Q B^{Qs} \\
                  \theta_q(B)   & = & 1 & + & \theta_1 B   & + & \cdots & + & \theta_q B^q    \\
            \end{array} \]
\end{Definition}
\begin{Definition}{}{}
      The \textbf{seasonal} autoregressive and moving average polynomials are
      defined by
      \[ \Phi(z)=1-\Phi_1 z-\cdots \Phi_P z^P \]
      \[ \Theta(z)=1+\Theta_1 z+\cdots+\Theta_Q z^Q \]
\end{Definition}
\begin{Example}{}{}
      Let $ X_t \sim \text{SARIMA}(1,1,1)\times(1,1,1)_{13} $.
      \[ \Phi(z)=1-\Phi_1 z \]
      \[ \phi(z)=1-\phi_1 z \]
      \[ \Theta(z)=1+\Theta_1 z \]
      \[ \theta(z)=1+\theta_1 z \]
      Therefore,
      \[ (1-\Phi_1 B^{13})(1-\phi_1 B)\Uunderbracket{(1-B^{13})(1-B)X_t}_{Y_t}=\Theta(B^{13})\theta(B)W_t \]
      \[ Y_t-\Phi_1 Y_{t-13}-\phi_1 Y_{t-1}-\phi_1 \Phi_1 Y_{t-14}=\text{MA term} \]
      \[ Y_t=f(Y_{t-13},Y_{t-1},\text{MA noise}, Y_{t-14}) \]
      where $ Y_{t-13} $ is the seasonal lag.
\end{Example}
\begin{Remark}{}{}
      \begin{enumerate}[(1)]
            \item $ Y_t=(1-B^s)^D(1-B)^d X_t $, a SARIMA model is just one big ARMA
                  model for $ Y_t $.
            \item Advantage over ARMA and ARIMA models is \textbf{parsimony}.
                  Since seasonal series have the feature that $ X_t $ is similar to $ X_{t-s} $,
                  we introduce just a few additional terms to model $ X_t $ as a function of
                  $ X_{t-s} $.
      \end{enumerate}
\end{Remark}
\subsection*{Fitting SARIMA Models}
\begin{enumerate}[(1)]
      \item Usually the seasonal lag $ s $ is known.
      \item Differencing and seasonal differencing can be decided upon by:
            \begin{enumerate}[(a)]
                  \item Eye-ball test and/or examining the ACF and PACF\@.
                  \item Stationarity tests.
                  \item Cross-validation.
            \end{enumerate}
            {\color{blue}We will discuss (b) and (c).}
      \item Choosing the order and estimating the components of $ \Phi,\phi,\Theta,\theta $
            can be done in the same was as with ARMA models.
\end{enumerate}
\section{SARIMA Cardiovascular Mortality Example}
\href{https://github.com/Hextical/university-notes/blob/master/year-3/semester-2/STAT%20443/code/6.2%20-%20SARIMA%20Cmort%20Example.R}{[R Code] SARIMA Cardiovascular Mortality Example}
\section{Time Series Cross-Validation}
\begin{Definition}{Cross-validation}{}
      \textbf{Cross-validation} is a data driven model evaluation
      and selection tool for predictive models that entails the following.
      \begin{enumerate}[(1)]
            \item Splitting the available data into training and testing sets.
            \item Fitting models on the training sets.
            \item Evaluating predictions of the model on the tests sets as an overall
                  evaluation of model quality.
      \end{enumerate}
\end{Definition}
\subsection*{Standard Cross-Validation}
Suppose $ (Y_i,X_i) $ for $ 1\le i\le n $ satisfy $ Y_i=f(X_i)+\varepsilon_i $.
Let $ M $ be a model used to estimate $ f $ using $ \hat{f} $,
with the goal of minimizing $ L(Y_i,\hat{f}(X_i)) $.
\subsection*{$ K $-fold Cross-Validation}
\begin{enumerate}[(1)]
      \item Split $ (Y_i,X_i) $ for $ 1\le i\le n $ randomly into $ K $-groups
            $ G_1,\ldots,G_k $.
      \item For each $ 1\le i\le K $, use $ M $ to estimate $ \hat{f}^{(-j)} $
            when the data $ G_i $ is left out.
      \item Evaluate error on $ G_i $ with
            \[ \text{CV}_j=\sum_{(Y_i,X_i)\in G_j}L(Y_i,\hat{f}^{-j}(X_i))  \]
      \item The total cross-validation error of the model is:
            \[ \text{CV}(M)=\sum_{j=1}^{k} \text{CV}_j \]
\end{enumerate}
\begin{Remark}{}{}
      \begin{itemize}
            \item $ K $ is often called the number of \textbf{folds}.
            \item If $ K=n $, the procedure is often called the ``leave-one-out''
                  cross-validation.
            \item $ K=10 $ is called ``10-fold cross validation.''
      \end{itemize}
\end{Remark}
\subsection*{Problems with Time Series Cross-Validation}
\begin{enumerate}[(1)]
      \item Randomly splitting the data scrambles up any serial dependence relationships.
      \item In time series forecasting, it is often most natural to use
            the past (recent past) to predict future values.
\end{enumerate}
\subsection*{Time Series Cross-Validation Algorithm}
\begin{enumerate}[(1)]
      \item Split the data into training and testing ranges $ 1\le t_r\le T $
            where $ t_r\approx 0.75T $ is $ 75\% $ of the training sample.
            The test sample is $ X_{t_r+1},\ldots,X_T $.
      \item For each $ j $ in $ t_r+1,\ldots,T $, use model to forecast
            $ \hat{X}_{j+1\mid j} $ based on $ X_1,\ldots,X_j $. Calculate loss
            \[ L(\hat{X}_{j+1\mid j};X_{j+1})=L_j \]
      \item Cross-validation score of model
            \[ \text{CV}(M)=\sum_{j=t_r+1}^{T} L_j \]
\end{enumerate}
\begin{Remark}{}{}
      \begin{enumerate}[(1)]
            \item If interested in longer horizon forecasting, you can compare
                  \[ \hat{X}_{j+1\mid j},\ldots,\hat{X}_{j+h\mid j}\quad\text{to}\quad X_{j+1},\ldots,X_{j+h} \]
                  in the loss calculation step.
            \item Stationarity is \emph{crucial} in time series cross validation
                  since the model errors in the present must be similar to errors in the future.
            \item One normally cannot cross-validate everything as this is computationally
                  infeasible.
      \end{enumerate}
\end{Remark}
\section{Cross-Validation Example}
\href{https://github.com/Hextical/university-notes/blob/master/year-3/semester-2/STAT%20443/code/6.4%20-%20Cross-Validation%20Example.R}{[R Code] Cross-Validation Example} % chktex 15
\section{Simulated and Bootstrapped Prediction Intervals}
Usually forecasts are of the form
\[ \hat{X}_{T+1\mid T}=g(X_T,X_{T-1},\ldots,X_1,W_{T+1}) \]
where $ W_{T+1} $ is a strong white noise innovation.

Often, even models are additive so that
\[ \hat{X}_{T+1\mid T}=g(X_T,\ldots,X_1)+W_{T+1} \]
Simple and powerful models to produce prediction intervals
use simulation!
\subsection*{Simulated Prediction Intervals}
\begin{enumerate}[(1)]
      \item Choose a distribution for $ \set{W_t} $. A common choice is $ W_t \sim \N{0,\hat{\sigma}_W^2} $.
      \item For $ b=1,\ldots,B $ where $ B $ is a large number, simulate $ \set*{W_{T+1}^{(b)}} $.
      \item Compute $ \hat{X}_{T+1\mid T}^{(b)}=g(X_{T},\ldots,X_1)+W_{T+1}^{(b)} $
            for $ b=1,\ldots,B $.
      \item Denote the empirical $ q^{\text{th}} $ quantile of $ \set*{\hat{X}_{T+1}^{(b)}:b=1,\ldots,B} $
            by $ \hat{Q}_{T+1}(q) $. We set the $ (1-\alpha) $ prediction interval as
            \[ \biggl(\hat{Q}_{T+1}\biggl(\frac{\alpha}{2} \biggr),\hat{Q}_{T+1}\biggl(1-\frac{\alpha}{2} \biggr)\biggr) \]
\end{enumerate}
\begin{Remark}{}{}
      For longer horizon forecasts, prediction intervals can be obtained by iteration:
      \[ \hat{X}_{T+h\mid T}^{(b)}=g(\hat{X}_{T+h-1\mid T}^{(b)},\ldots,\hat{X}_{T+1\mid T}^{(b)},X_T,\ldots,X_1)+W_{T+h}^{(b)} \]
      The prediction interval is
      \[ \biggl(\hat{Q}_{T+h}\biggl(\frac{\alpha}{2} \biggr),\hat{Q}_{T+h}\biggl(1-\frac{\alpha}{2} \biggr)\biggr) \]
      where $ \hat{Q}_{T+h}(q) $ the empirical $ q^{\text{th}} $ quantile of $ \hat{X}_{T+h}^{(b)} $.
\end{Remark}
\subsection*{Distributions to Choose for $ W_t $}
\begin{enumerate}[(1)]
      \item $ W_t \sim \N{0,\hat{\sigma}_W^2} $ where $ \hat{\sigma}_W^2 $ is estimated from residuals which
            leads to approximately the same ``well known'' prediction intervals.
      \item A distribution fit to the estimated residuals $ \hat{W}_t $; e.g., a $ t $-distribution,
            Pareto, etc.
      \item The empirical distribution of the residuals $ \hat{W}_t $; that is,
            randomly drawing $ \set*{\hat{W}_1,\ldots,\hat{W}_{T}} $ which is commonly known
            as \textbf{bootstrapping}.

            \underline{Note}: An important consideration of the bootstrap is that the residuals should be white!
            We can check the whiteness of the residuals using the ACF or a white noise test.
\end{enumerate}
\section{Bootstrap Prediction Intervals Example}
\href{https://github.com/Hextical/university-notes/blob/master/year-3/semester-2/STAT%20443/code/6.6%20-%20Bootstrap%20Prediction%20Intervals%20Example.R}{[R Code] Bootstrap Prediction Intervals Example}
