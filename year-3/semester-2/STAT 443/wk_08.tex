\chapter{Week 8}
\section{Neural Network Autoregression}
Simple Neural Network ``Architecture:''
\begin{itemize}
    \item Input layer (covariates/predictors).
    \item Hidden layer (neurons).
    \item Output layer.
\end{itemize}
It's possible to have several hidden layers and multiple
neurons at each layer.

Any particular layer, the inputs are mapped to the $ j^{\text{th}} $
neuron linearly. The value taken on the $ j^{\text{th}} $
neuron is
\[ z_j=b_j+\sum_{i=1}^{4} w_{i,j}x_{i} \]
where $ b_j $ is a function, $ x_i $ is the $ i^{\text{th}} $ input, and $ w_{i,j} $
are the weights.

To calculate the inputs to the next layer, a non-linear transformation
is applied. For example, using the sigmoid function:
\[ S(z)=\frac{1}{1+e^{-z}} \]
The final model is a complex non-linear function of the inputs.

\subsection*{Neural Network AR}
\begin{itemize}
    \item Input layer: $ X_t,\ldots,X_{t-p} $.
    \item Output layer: $ X_{t+1} $.
\end{itemize}
A neural network model with $ k $ hidden states (assuming one hidden layer)
we call a $ \NNAR{p,k} $ model.
\begin{Remark}{}{}
    If $ k=0 $, then $ \NNAR{p}=\AR{p} $. The inputs are mapped
    linearly to the outputs.
\end{Remark}
\subsection*{Seasonal Neural Network AR}
\begin{itemize}
    \item Input layer: $ X_{t},\ldots,X_{t-p},X_{t-m},X_{t-P_m} $
          where $ m $ is the seasonal lag.
    \item Output layer: $ X_{t+1} $.
\end{itemize}
We call this a $ \NNSAR{p,k,P}{m} $ model.

The model selection of choosing $ k $, $ p $, and $ P $
can be carried out using cross-validation where the weights are estimated using ordinary
least squares.

\subsection*{Prediction Intervals}
If $ \symbf{X}_t=(X_t,\ldots,X_{t-p},X_{t-m},\ldots,X_{t-P_m})^\top $
denotes the vector of predictors, then we can posit an additive stochastic model for
$ X_{t+1} $ as
\[ X_{t+1}=f(\symbf{X}_t)+\varepsilon_{t+1} \]
where $ f $ is the neural network.

By calculating the residuals
$ \hat{\varepsilon}_t=X_t-\hat{f}(\symbf{X}_t) $,
prediction intervals can be estimated using the bootstrap
\[ X_{T+1}^{(b)}=\hat{f}(\symbf{X}_T)+\hat{\varepsilon}_{T+1}^{(b)}\quad(b=1,\ldots,B) \]
We can then construct a prediction interval by using
the empirical quantiles from the simulated distribution of the
forecast $ 1 $-step ahead. This process can be iterated multiple
times to produce forecasts as well as prediction intervals for
forecasts at longer time horizons.

\section{Comparing Various Forecasting Methods}

\section{Conditional Heteroscedasticity}
\[ \Uunderbracket{\text{Hetero}}_{\text{different}}\text{-}\Uunderbracket{\text{scedasticity}}_{\text{variance}} \]
\[ \Uunderbracket{\text{Hetero}}_{\text{same}}\text{-}\Uunderbracket{\text{scedasticity}}_{\text{variance}} \]
\begin{Definition}{Heteroscedastic}{}
    We say a time series $ X_t $ is \textbf{heteroscedastic}
    if $ \Var{X_t}=\sigma_{X,t}^2 $; that is, the variance
    depends on $ t $ and changes at some points.
\end{Definition}
\begin{Remark}{}{}
    Heteroscedastic time series are \underline{not} stationary.
\end{Remark}
Asset price data terminology: In the context of conditionally
heteroscedastic time series, we often consider asset price or ``financial''
time series. Suppose $ X_t= $ price of an asset at time $ t $.
\begin{Definition}{Returns, Log-returns}{}
    If $ X_t $ is the value of an asset at time $ t $, then
    the \textbf{return} (relative gain) $ Y_t $ of the
    asset at time $ t $ is
    \[ Y_t=X_t-X_{t-1}=\nabla X_t \]
    Furthermore, the \textbf{log-returns}
    of a positive asset price series $ X_t $
    are
    \[ Y_t=\log\biggl(\frac{X_t}{X_{t-1}} \biggr)=\log(X_t)-\log(X_{t-1}) \]
\end{Definition}
\begin{Remark}{}{}
    ``Volatility'' $ \iff $ ``Variance''.
\end{Remark}
TODO Code

A common observation, especially prominent with financial and asset
price data, is that periods of volatility or heteroscedastic
tend to cluster.

Why? Big ``shocks'' cause volatile periods, that further propagate
volatility until things ``calm down.''

ARMA and linear time series models are not useful for capturing this phenomenon
as we will see in the next example.
\begin{Example}{}{}
    Let $ X_t \sim \AR{1} $; that is, $ X_t=\phi X_{t-1}+W_t $ where $ \abs{\phi}<1 $.
    \[ \E{X_t\given X_{t-1},X_{t-2},\ldots}=\phi X_{t-1} \]
    ARMA models ``model'' the conditional mean $ X_{t-1},X_{t-2},\ldots $.
    \[ \Var{X_t\given X_{t-1},X_{t-2},\ldots}=\sigma_W^2 \]
    $ X_{t-1},X_{t-2},\ldots $ leave the variance untouched.
\end{Example}
\begin{Definition}{Conditionally heteroscedastic}{}
    We say a time series $ X_t $ is \textbf{conditionally heteroscedastic}
    if
    \[ \Var{X_t\given X_{t-1},X_{t-2},\ldots}=\sigma_{X,t}^2 \]
    that is, the variance changes with $ t $.
\end{Definition}

\section{ARCH and GARCH Models}
\begin{Definition}{Autoregressive conditionally heteroscedastic (ARCH)}{}
    Let $ W_t $ be a unit variance strong white noise; that is,
    $ \E{W_t}=0 $ and $ \Var{W_t}=1 $. We say $ X_t $
    follows an \textbf{autoregressive conditionally heteroscedastic}
    (ARCH) model if there exists parameters $ \omega> 0 $, $ \alpha_1\ge 0 $
    such that $ X_t=\sigma_t W_t $ where
    \[ \sigma_t^2=\omega+\alpha_1 X_{t-1}^2 \]
    where $ \sigma_t $ is the conditional variance and $ W_t $
    is a white noise.
\end{Definition}
\begin{Remark}{}{}
    ARCH is from Robert Engle, 1982.
\end{Remark}
\begin{Definition}{Autoregressive conditionally heteroscedastic $ [\ARCH{p}] $}{}
    We say $ X_t $ follows an \textbf{autoregressive conditionally heteroscedastic} model
    of order $ p $, if $ W_t $ is a strong white noise with $ \E{W_t^2}=1 $
    and
    \[ X_t=\sigma_t W_t \]
    \[ \sigma_t^2=\omega+\sum_{j=1}^{p} \alpha_j X_{t-j}^2 \]
    where $ p>0 $, $ \omega>0$, and $\alpha_j\ge 0$ for $j=1,\ldots,p$.

    We write $ X_t \sim \ARCH{p} $.
\end{Definition}
\begin{Remark}{}{}
    \begin{enumerate}[(1)]
        \item $ \sigma_t^2 $ is called the ``conditional variance'' or ``volatility.''
              Imagine that there exist a representation $ X_t=g(W_t,\ldots,W_{t-1}) $
              (stationary process satisfying the ARCH model). Then, for example, in the
              $ \ARCH{1} $ case,
              \[ \sigma_t=w+\alpha_1 X_{t-1}^2=g_\sigma(W_{t-1}W_{t-2},\ldots) \]
              Therefore,
              \[ \Var{X_t\given W_{t-1},W_{t-2},\ldots}=\Var{\sigma_tW_t\given W_{t-1},\ldots}=\sigma_t^2\Var{W_t}=\sigma_t^2 \]
              $ \Var{W_t}=1 $ identifies $ \sigma_t^2 $ as conditional variance.
        \item Engle won the nobel prize in economics in part for ``methods of analyzing economic
              time series with time varying volatility (ARCH)'' in 2003.
        \item One problem noted early on was that $ \ARCH{p} $
              models required large orders of $ p $ to model asset returns.
    \end{enumerate}
\end{Remark}
\begin{Definition}{Generalized autoregressive conditional heteroskedasticity (GARCH)}{}
    We say $ X_t $ follows a \textbf{generalized autoregressive conditional heteroskedasticity}
    (GARCH) model
    if $ W_t $ is unit variance strong white noise and
    \[ X_t=\sigma_t W_t \]
    \[ \sigma_t^2=\omega+\sum_{j=1}^{p} \alpha_j X_{t-j}^2+\sum_{k=1}^{q} \beta_k \sigma_{t-k}^2 \]
    where $ q\ge 0 $, $ p>0 $, $ \omega>0 $, $ \alpha_j\ge 0 $ for $ j=1,\ldots,p $, and
    $ \beta_k\ge 0 $ for $ k=1,\ldots,q $.

    We write $ X_t \sim \GARCH{p,q} $.
\end{Definition}
\begin{Remark}{}{}
    The $ \GARCH{p,q} $ model was proposed by Bollerslev (1986).
\end{Remark}
\begin{Remark}{}{}
    \begin{itemize}
        \item $ \GARCH{p,0}\equiv \ARCH{p} $.
        \item $ \GARCH{0,0} $ is a white noise.
    \end{itemize}
\end{Remark}
\begin{Proposition}{Properties of GARCH}{}
    Suppose for the moment that there exists a stationary and causal time
    series $ X_t $ satisfying the $ \GARCH{p,q} $ model,
    $ X_t=g(W_t,W_{t-1},\ldots)\implies \sigma_t^2=g_\sigma(W_{t-1},W_{t-2},\ldots) $,
    then
    \begin{enumerate}
        \item $ \E{X_t}=\E{\sigma_t}\E{W_t}=0 $ since $ \sigma_t $ and $ W_t $ are
              independent.
              \[ \gamma_X(h)=\E{X_{t+h}X_t}=\E{\sigma_{t+h}W_{t+h}\sigma_tW_t}=0 \]
              since $ W_{t+h} $ is independent of the rest.
              ``GARCH series have mean zero and are serially uncorrelated.''
        \item Suppose $ X_t \sim \ARCH{1} $.
              \begin{align*}
                  X_t^2
                   & =\sigma_t^2 W_{t}^2                        \\
                   & =\sigma_t^2(W_t^2+1-1)                     \\
                   & =\sigma_t^2+(W_t^2-1)                      \\
                   & =w+\alpha_1 X_{t-1}^2 +\sigma_t^2(W_t^2-1)
              \end{align*}
              Therefore, $ X_t^2 \sim \AR{1} $ process (weak white noise innovations).
              In general, if $ X_t \sim \GARCH{p,q} $, then $ X_t^2 $
              follows an ARMA model with weak white noise innovations.
              \[ X_r \sim \text{GARCH},\quad X_t^2\text{ is serially correlated.} \]
    \end{enumerate}
\end{Proposition}
TODO Code

\section{Stationarity of GARCH Models}
Suppose $ X_t \sim \GARCH{p,q} $ model, then we have
$ X_t=\sigma_t W_t $, $ \E{W_t}=0 $, $ \Var{W_t}=\E{W_t^2}=1 $, and
\[ \sigma_t^2=\omega+\sum_{j=1}^{p} \alpha_j X_{t-j}^2+\sum_{k=1}^{q}\beta_k \sigma^2_{t-k}  \]
where $ \omega>0 $ and $ \alpha_1,\ldots,\alpha_p,\beta_1,\ldots,\beta_q \ge 0 $.

\underline{Question}: Under what conditions on $ \omega,\alpha_1,\ldots,\alpha_p,
    \beta_1,\ldots,\beta_p $, does a stationary process $ \set{X_t}_{t\in\mathbf{Z}} $
satisfy these questions?

\begin{Remark}{}{}
    Suppose a stationary solution exists that is a causal Bernoulli shift; that is,
    \[ X_t=g(W_t,W_{t-1},\ldots) \]
    \[ \sigma_t^2=g_\sigma(W_{t-1},W_{t-2},\ldots) \]
    If $ \Var{X_t}<\infty $, note
    \[ \Var{X_\sigma}=\Var{\sigma_t W_t}=\E{\sigma_t^2 W_T^2}=\E{\sigma_t^2}=\sigma_X^2 \]
    and using the GARCH recursion:
    \[ \E{\sigma_t^2}=\omega+\sum_{j=1}^{p} \alpha_j \E{X_{t-j}^2}+\sum_{k=1}^{q} \beta_k \E{\sigma^2_{t-k}} \]
    \[ \sigma_X^2=\omega+\sum_{j=1}^{p} \alpha_j \sigma_X^2 +\sum_{k=1}^{q}\beta_k \sigma^2_X  \]
    Solving gives
    \[ \sigma_X^2=\frac{\omega}{1-\sum_{j=1}^{p} \alpha_j-\sum_{k=1}^{q} \beta_k}  \]
    Suggests that in order for a solution to exist in $ L^2 $, we need
    \[ \sum_{j=1}^{p} \alpha_j + \sum_{k=1}^{q} \beta_k <1 \]
    (Bollerslev, 1986)
\end{Remark}
Consider $ \GARCH{1,1} $ case; that is,
\[ X_t=\sigma_t W_t \]
\[ \sigma_t^2=\omega+\alpha X_{t-1}^2+\beta \sigma_{t-1}^2 \]
$ X_t $ in order to get stationary solution and $ \sigma_t^2 $ need
a stationary conditional variance process. Let $ f(z)=\alpha z^2+\beta $.
Iterate GARCH recursion:
\begin{align*}
    \sigma_t^2
     & =\omega+\alpha X_{t-1}^2+\beta \sigma_{t-1}^2                                                              \\
     & =\omega+\alpha(\sigma_{t-1}^2 W_{t-1}^2)+\beta \sigma_{t-1}^2                                              \\
     & =\omega+(\alpha W_{t-1}^2+\beta)\sigma_{t-1}^2                                                             \\
     & =\omega+f(W_{t-1})(\omega+\alpha X_{t-2}^2+\beta \sigma_{t-2}^2)                                           \\
     & =\omega+\omega f(W_{t-1})+f(W_{t-1})(\alpha X_{t-2}^2+\beta \sigma_{t-2}^2)                                \\
     & =\omega+\omega f(W_{t-1})+\omega f(W_{t-1})f(W_{t-2})+f(W_{t-1})f(W_{t-2})(\alpha X_{t-3}^2+\beta_{t-3}^2) \\
     & \vdotswithin{=}                                                                                            \\
     & =\omega\biggl(1+\sum_{i=1}^{\infty} \prod_{j=1}^i f(W_{t-j})\biggr)                                        \\
     & =g_\sigma (W_{t-1},W_{t-2},\ldots)
\end{align*}
Posit solution
\[ \sigma_t^2=\omega\biggl(1+\sum_{j=1}^{\infty} \prod_{i=1}^j f(W_{t-i})\biggr) \]
\underline{Question}: When is this series well-defined?
\[ \prod_{i=1}^j f(W_{t-i})=\expon[\bigg]{\sum_{i=1}^{j} \log\bigl[f(W_{t-j})\bigr]} \]
Now,
\[ \sum_{i=1}^{j} \log\bigl[f(W_{t-j})\bigr]\to
    \begin{cases*}
        +\infty                                               & with probability $ 1 $ if $ \E*{\log\bigl[f(W_0)\bigr]}>0 $ \\
        -\infty                                               & if $ \E*{\log\bigl[f(W_0)\bigr]}<0 $                        \\
        \text{oscillates between $ -\infty $ and $ +\infty $} & if $\E*{\log\bigl[f(W_0)\bigr]}=0$
    \end{cases*} \]
\begin{Theorem}{}{}
    A stationary solution $ X_t $ exists to the $ \GARCH{1,1} $ equations if and only if
    \[ \gamma=\E{\log(\alpha W_0^2+\beta)}<0\quad\text{[Top Lyapunov Exponent]} \]
    The solution is of the form
    \[ X_t=\sigma_t W_t \]
    \[ \sigma_t^2=\omega\biggl(1+\sum_{j=1}^{\infty} \prod_{i=1}^j (\alpha W_{t-j}^2+\beta)\biggr)=
        g(W_{t-1},W_{t-2},\ldots) \]
    where $ g $ is a function that is not linear; that is, we have a non-linear time series.
\end{Theorem}
\begin{Remark}{}{rem_garch11}
    \begin{enumerate}[(1)]
        \item If $ \gamma<0 $, $ \omega=0 $ forces $ X_t\equiv 0 $. Therefore, we will
              normally assume $ \omega>0 $.
        \item The condition $ \gamma=\E{\log(\alpha W_0^2+\beta)}<0 $ depends
              on the distribution of $ W_t $.
        \item A sufficient condition is $ \alpha_1+\beta_1<1 $.
    \end{enumerate}
\end{Remark}
\begin{Proof}{\Cref{remark:rem_garch11} (3)}{}
    Jensen's Inequality: If $ f:\mathbf{R}\to\mathbf{R} $ is convex, then
    \[ f\bigl(\E{X}\bigr)\le \E[\big]{f(X)} \]
    and the opposite holds if $ f $ is concave. We note that $ \log(x) $
    is concave, hence
    \[ \E[\big]{\log(\alpha W_0^2+\beta)}\le \log\bigl(\E{\alpha W_0^2+\beta}\bigr)=\log(\alpha+\beta)<0 \]
    If $ \alpha+\beta<1 $.
\end{Proof}
\begin{Remark}{Second-order Stationarity}{}
    If $ \alpha_1+\beta_1>1 $, we have seen that $ \Var{X_t} $ is not well-defined.
    If $ \alpha_1+\beta_1<1 $, then
    \[ \E{\sigma_t^2}=\E*{\frac{\omega}{1-\alpha-\beta}}<\infty \]
    Assuming $ \alpha_1+\beta_1<1 $, $ X_t $ is a weak white noise.
    \[ \gamma_X(h)=\E{X_{t+h}X_t}=\E{\sigma_{t+h}W_{t+h}\sigma_t W_t}=0 \]
\end{Remark}

\section{\texorpdfstring{$ \dagger $}{†} Stationarity of General \texorpdfstring{$ \GARCH{p,q} $}{GARCH(𝑝, 𝑞)}}
General conditions exist for when a $ \GARCH{p,q} $ process has a strictly stationary solution: Let
\begin{align*}
    \tau_{t} & =(\beta_{1}+\alpha_{1} W_{t}^{2}, \beta_{2}, \ldots, \beta_{q-1})\in \mathbf{R}^{q-1}                  \\
    \xi_{t}  & =(X_{t}^{2}, 0, \ldots, 0) \in \mathbf{R}^{q-1}                                                        \\
    \alpha   & =(\alpha_{2}, \ldots, \alpha_{p-1}) \in \mathbf{R}^{p-2}                                               \\
    I_{c}    & =c\times c \text{ identity matrix.}                                                                    \\
    N        & =(\omega, 0, \ldots, 0) \in \mathbf{R}^{p+q-1}                                                         \\
    Y_{t}    & =(\sigma_{t}^{2}, \ldots, \sigma_{t-q+1}^{2}, X_{t}^{2}, \ldots, X_{t-p+1}^{2}) \in \mathbf{R}^{p+q-1} \\
    M_{t}    & =\begin{bmatrix}
        \tau_{t} & \beta_{q} & \alpha  & \alpha_{p} \\
        I_{q-1}  & 0         & 0       & 0          \\
        \xi_{t}  & 0         & 0       & 0          \\
        0        & 0         & I_{p-2} & 0
    \end{bmatrix}\in \mathbf{R}^{p+q-1 \times p+q-1}
\end{align*}

\begin{Theorem}{}{}
    $X_{t}$ solves the $ \GARCH{p,q} $ equations if and only if
    \[
        Y_{t}=M_{t} Y_{t-1}+N
    \]
    This representation is known as the Markov representation of the GARCH equations.
    This defines a first order vector autoregression for $Y_{t}$ with
    (random) matrix coefficients $M_{t}$.
\end{Theorem}
Let $A_{t}$ be a stationary sequence of random $p+q-1 \times p+q-1$ matrices, and define,
for an arbitrary norm on matrices $\norm{}$ the scalar random variables.
\[ r_{t}=\norm{A_{t} A_{t-1} \ldots A_{1}} \]
under some relatively mild conditions (ergodicity)
\[ \gamma=\lim\limits_{{t} \to {\infty}}\biggl[\frac{1}{t} \E[\big]{\log(r_{t})}\biggr] \]
is well-defined and is called the top Lyapounov exponent of the sequence
$A_{t}$ for $t \in \mathbf{Z}$.
This result is coming from Ergodic theory in the 1970s.
\begin{Theorem}{}{}
    A stationary solution to the $ \GARCH{p,q} $ equations exists if and only if
    \[ \gamma<0 \]
    where $\gamma$ is the top Lyapounov exponent of sequence
    $M_{t}$ for $t \in \mathbb{Z}$ appearing in the Markov representation.
    When a stationary solution exists, it is causal and unique.
\end{Theorem}
\begin{Theorem}{Theorem 1 of Bollerslev (1986)}{thm_1_bollerslev_1986}
    A necessary and sufficient condition for there to exist a second order stationary
    solution to the $ \GARCH{p,q} $ equations is that
    \[ \sum_{j=1}^{p} \alpha_{j}+\sum_{\ell=1}^{q} \beta_{\ell}<1 \]
\end{Theorem}

\section{Identifying GARCH Models}
The decision to fit a volatility (GARCH) model to a time series often
arises from
\begin{enumerate}[(1)]
    \item Observing volatility (conditional heteroskedasticity)
          in a series.
    \item Conditional variance forecasting is of specific interest
          (e.g., risk analysis, financial TS analysis).
\end{enumerate}
If strong serial correlation is observed in the series, one
often fits initially an ARMA model, and then
fits a GARCH model to the residuals.
\subsection*{Identifying Serial Correlation}
Recall that the normal ACF bounds (blue lines)
are constructed based on the assumption that
the series is a \emph{strong} white noise. A GARCH
model is a \emph{weak} white noise.

\subsection*{ACF Bounds for Weak White Noise}
Suppose for example that $ X_t \sim \GARCH{1,1} $, then
\[ \gamma_X(h)=0\quad(h\ge 1) \]
\[ \hat{\gamma}_X(h)\approx \frac{1}{T} \sum_{j=1}^{T-h} X_t X_{t+h}\implies \E{\hat{\gamma}_X(h)}=0 \]
\begin{align*}
    \Var*{\sqrt{T}\hat{\gamma}_X(h)}
     & =\frac{1}{T} \sum_{j=1}^{T-h} \sum_{k=1}^{T-h} \E{X_{j}X_{j+h}X_{k}X_{k+h}}                                            \\
     & =\frac{1}{T} \sum_{j=1}^{T-h} \sum_{k=1}^{T-h} \E{\sigma_j W_j \sigma_{j+h}W_{j+h}\sigma_{k+h}W_{k+h}\sigma_k W_{k+h}} \\
     & =\frac{1}{T} \sum_{j=1}^{T-h} \E{X_j+h^2 X_j^2}                                                                        \\
     & \approx \E{X_0^2 X_{-h}^2}
\end{align*}
Note that $ \E{X_j+h^2 X_j^2} $ does not simplify to a product $ \sigma_X^4 $ since
$ X_{j+h}^2 $ is correlated with $ X_j^2 $.
\begin{Theorem}{}{}
    If $ X_t $ is a weak white noise (suitably weakly dependent), then
    \[ \sqrt{T}\hat{\gamma}_X(h)\xrightarrow{D}\N{0,\E{X_0 X_{-h}^2}} \]
\end{Theorem}
\begin{Remark}{}{}
    \begin{enumerate}[(1)]
        \item $ \E{X_0^2 X_{-h}^2} $ can be consistently estimated by
              \[ \hat{\sigma}_h^2=\frac{1}{T} \sum_{j=1}^{T-h} X_{j+h}^2 X_j^2 \]
              An approximate $ (1-\alpha) $ prediction interval for $ \hat{\rho}(h) $
              under the assumption of a weak white noise is
              \[ \pm \frac{1}{\sqrt{T}}z_{1-\alpha/2}\frac{\hat{\sigma}_h}{\hat{\gamma}(0)}   \]
        \item Note that
              \[ \E{X_0 X_{-h}^2}=\bigl(\E{X_0^2}\bigr)^2+\Uunderbracket{\Cov{X_0^2,X_{-h}^2}}_{>0\iff\text{GARCH}} \]
              In GARCH setting, weak white noise intervals for ACF are larger.
    \end{enumerate}
\end{Remark}
TODO Code
