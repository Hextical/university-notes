\chapter{Week 8}
\section{Neural Network Autoregression}
Simple Neural Network ``Architecture:''
\begin{itemize}
    \item Input layer (covariates/predictors).
    \item Hidden layer (neurons).
    \item Output layer.
\end{itemize}
It's possible to have several hidden layers and multiple
neurons at each layer.

Any particular layer, the inputs are mapped to the $ j^{\text{th}} $
neuron linearly. The value taken on the $ j^{\text{th}} $
neuron is
\[ z_j=b_j+\sum_{i=1}^{4} w_{i,j}x_{i} \]
where $ b_j $ is a function, $ x_i $ is the $ i^{\text{th}} $ input, and $ w_{i,j} $
are the weights.

To calculate the inputs to the next layer, a non-linear transformation
is applied. For example, using the sigmoid function:
\[ S(z)=\frac{1}{1+e^{-z}} \]
The final model is a complex non-linear function of the inputs.

\subsection*{Neural Network AR}
\begin{itemize}
    \item Input layer: $ X_t,\ldots,X_{t-p} $.
    \item Output layer: $ X_{t+1} $.
\end{itemize}
A neural network model with $ k $ hidden states (assuming one hidden layer)
we call a $ \NNAR{p,k} $ model.
\begin{Remark}{}{}
    If $ k=0 $, then $ \NNAR{p}=\AR{p} $. The inputs are mapped
    linearly to the outputs.
\end{Remark}
\subsection*{Seasonal Neural Network AR}
\begin{itemize}
    \item Input layer: $ X_{t},\ldots,X_{t-p},X_{t-m},X_{t-P_m} $
          where $ m $ is the seasonal lag.
    \item Output layer: $ X_{t+1} $.
\end{itemize}
We call this a $ \NNSAR{p,k,P}{m} $ model.

The model selection of choosing $ k $, $ p $, and $ P $
can be carried out using cross-validation where the weights are estimated using ordinary
least squares.

\subsection*{Prediction Intervals}
If $ \symbf{X}_t=(X_t,\ldots,X_{t-p},X_{t-m},\ldots,X_{t-P_m})^\top $
denotes the vector of predictors, then we can posit an additive stochastic model for
$ X_{t+1} $ as
\[ X_{t+1}=f(\symbf{X}_t)+\varepsilon_{t+1} \]
where $ f $ is the neural network.

By calculating the residuals
$ \hat{\varepsilon}_t=X_t-\hat{f}(\symbf{X}_t) $,
prediction intervals can be estimated using the bootstrap
\[ X_{T+1}^{(b)}=\hat{f}(\symbf{X}_T)+\hat{\varepsilon}_{T+1}^{(b)}\quad(b=1,\ldots,B) \]
We can then construct a prediction interval by using
the empirical quantiles from the simulated distribution of the
forecast $ 1 $-step ahead. This process can be iterated multiple
times to produce forecasts as well as prediction intervals for
forecasts at longer time horizons.

\section{Comparing Various Forecasting Methods}

\section{Conditional Heteroscedasticity}

\section{ARCH and GARCH Models}

\section{Stationarity of GARCH Models}

\section{\texorpdfstring{$ \dagger $}{†} Stationarity of General \texorpdfstring{$ \GARCH{p,q} $}{GARCH(𝑝, 𝑞)}}

\section{Identifying GARCH Models}
